\secstar{സ്വതന്ത്ര സോഫ്റ്റ്‌വെയര്‍ - ചില നിരീക്ഷണങ്ങള്‍}
\vskip 2pt

സിബുവിന്റെ ഒരു പോസ്റ്റും\footnote{\url{http://cibu.blogspot.com/2007/11/blog-post.html}} അവിടുത്തെ കമന്റുകളും, സ്വതന്ത്ര സോഫ്റ്റ്‌വെയറിനെക്കുറിച്ചു് എനിക്കറിയാവുന്ന കാര്യങ്ങള്‍ എഴുതാന്‍ പ്രേരിപ്പിക്കുന്നു.

സോഴ്സ് ഉപയോഗിച്ചു് പുതിയ സങ്കേതങ്ങള്‍ രൂപപ്പെടുത്തുന്നതു് സോഫ്റ്റ്‌വെയറിനെ സാങ്കേതികമായി സമീപിക്കുന്നവരുടെ ആവശ്യമാണു്. എനിക്കറിയാവുന്നിടത്തോളം സോഫ്റ്റ്‌വെയര്‍ ഉപയോക്താക്കളുടെ ഒരു ചെറിയ ശതമാനമേ ഇത്തരത്തിലുള്ളവരുള്ളു. ഇന്നത്തെ വലിയൊരു ശതമാനം സ്വതന്ത്ര സോഫ്റ്റ്‌വെയര്‍ ഉപയോക്താക്കളും, അതിന്റെ സ്വാതന്ത്ര്യം തിരിച്ചറിഞ്ഞു് ഉപയോഗിച്ചു തുടങ്ങിയവരാണു്.

സാധാരണ സോഫ്റ്റ്‌വെയര്‍ ഉപയോക്താക്കളില്‍ ഭൂരിഭാഗവും പ്രോഡക്ടിനെ മാത്രം ആശ്രയിക്കുന്നവരാണു്. അവര്‍ അതിന്റെ സര്‍വ്വീസ് മാത്രമാണു് ഉപയോഗിക്കുന്നതു്. അതിനാവശ്യമായ സപ്പോര്‍ട്ടാണു് അപ്പോളവിടെ വലിയ പ്രോഡക്ടു്. അതു നല്‍കാന്‍ സ്വതന്ത്ര സോഫ്റ്റ്‌വെയര്‍ അനുവദിക്കുന്നുമുണ്ടു്. പണം വാങ്ങരുതെന്നു് എനിക്കറിയാവുന്നിടത്തോളം എവിടെയും പറയുന്നുമില്ല. മാത്രവുമല്ല, പരിപൂര്‍ണ്ണസ്വാതന്ത്ര്യം ഉപയോക്താവിനു് നല്‍കുന്നുമുണ്ടു്. സ്വാതന്ത്ര്യം ദുരുപയോഗപ്പെടുത്താന്‍ അനുവദിക്കുന്നുമില്ല.

ഞാന്‍ കണ്ടുപിടിച്ച സങ്കേതം ഞാന്‍ മാത്രമേ സര്‍വ്വീസ് ചെയ്യൂ എന്നും, അവിടെ എനിക്ക് മോണോപ്പോളി വേണം എന്നുമുള്ള വാദങ്ങളെ സ്വതന്ത്ര സോഫ്റ്റ്‌വെയര്‍ തള്ളിക്കളയുന്നു. "തുറന്ന" വിപണിയുടെ യഥാര്‍ത്ഥ ഗുണഭോക്താക്കളായി ഉപയോക്താക്കളെ മാറ്റുന്നു. ഗവേഷണങ്ങളെ സ്വതന്ത്രമായ രീതിയില്‍ collaborative development ആക്കി മാറ്റാന്‍ ശ്രമിക്കുന്നു.

കൊടുംലാഭം ഇവിടെ ആര്‍ക്കും പ്രതീക്ഷിക്കാനാവില്ല, ശരിയാണു്. മാത്രമല്ല, മത്സരം കടുക്കുന്നതിനാല്‍ സപ്പോര്‍ട്ടെന്നാല്‍ input കുറച്ചു് output കൂട്ടുന്ന രീതിയാക്കാനും കഴിയില്ല. ഇന്നത്തെ വിപണിയിലെ കൊടുംലാഭത്തിന്റെ കുതിപ്പു് ഉണ്ടാവില്ല, പക്ഷെ മാന്യമായ ലാഭം കഴിവുള്ളവനു് ലഭിക്കും. വിപണിയില്‍ പിടിച്ചുനില്‍ക്കാന്‍ പുതിയ കണ്ടുപിടുത്തങ്ങള്‍ വേണ്ടിവരും, സാങ്കേതികവിദ്യ വെളിവാക്കേണ്ടതിനാല്‍ സ്വതന്ത്ര ഗവേഷണം നടത്തേണ്ടി വരും. ഉപയോക്താവു് സ്വാതന്ത്ര്യം ആവശ്യപ്പെടുന്നതിനാല്‍ ഒളിച്ചുകളി നടപ്പില്ല. അതുകൊണ്ടുതന്നെ, ഗവേഷണങ്ങള്‍ തുല്യശക്തികളുടെ collaborative attempt ആയിമാറുന്നു. പ്രോഡക്ടു് നന്നാവുന്നു.

സാമ്പത്തികമായി ലാഭം ഉണ്ടാക്കാം, കൊള്ളനടത്താന്‍ കഴിയില്ല - ഇങ്ങനെയാണു് എനിക്കു് സ്വതന്ത്ര സോഫ്റ്റ്‌വെയറിനെക്കുറിച്ചു് തോന്നിയിട്ടുള്ളതു്. എല്ലാരീതിയിലും സ്വതന്ത്രമായി സോഫ്റ്റ്‌‌വെയര്‍ ഉപയോഗിക്കുന്ന കാലം വരില്ല എന്നു് പറയാനാവില്ല. പക്ഷെ ഉപയോക്താവു് അവകാശങ്ങളെക്കുറിച്ചു് ബോധവാനാവാത്തിടത്തോളം അതുണ്ടാവില്ല എന്നു പറയാം.

ഇതൊക്കെ ഇക്കാലത്തു നടക്കുമോ, ഉപയോക്താവു് സ്വാതന്ത്ര്യത്തെ തിരിച്ചറിയുമോ എന്നൊക്കെ ചോദിച്ചാല്‍, ഈ ഉപയോക്താവു് എന്നു പറയുന്ന ആള്‍ നമ്മളോരോരുത്തരുമാണെന്നും ആവശ്യം നമ്മുടേതാണെന്നും മനസ്സിലാക്കുക. മനസ്സിലാക്കിയതു് രണ്ടുപേരോടുകൂടിപ്പറയുക, പ്രവര്‍ത്തിക്കുക, ഇത്തരം പ്രവര്‍ത്തനങ്ങള്‍ക്ക് ആകാവുന്ന സഹായം ചെയ്യുക. ലോകത്തില്‍ ഒരാശയവും പടര്‍ന്നുപന്തലിച്ചതു ഒരു ദിവസം കൊണ്ടല്ല എന്നോര്‍ക്കുക.

ഇവിടെ ഞാന്‍ പങ്കുവയ്ക്കാന്‍ ശ്രമിച്ചതു് സ്വതന്ത്ര സോഫ്റ്റ്‌വെയറിന്റെ സ്വാതന്ത്ര്യം ഉപയോക്താവു് തിരിച്ചറിഞ്ഞു പ്രവര്‍ത്തിച്ചാല്‍ എന്തു സംഭവിക്കാം എന്നാണു്. അതു് തിരിച്ചറിഞ്ഞുകഴിഞ്ഞ കമ്പനിയുടമയുടെ വാക്കുകള്‍ പ്രവീണിന്റെ പരിഭാഷയില്‍ ഇവിടെ\footnote{\url{http://pravi.livejournal.com/15198.html}} വായിക്കൂ.

സ്വതന്ത്ര സോഫ്റ്റ്‌വെയറിന്റെ നന്മയും ഗുണവും തിരിച്ചറിയുന്നവര്‍ അതു് മറ്റുള്ളവരെക്കൂടി മനസ്സിലാക്കാനും തിരിച്ചറിയിക്കാനും ശ്രമിക്കണം എന്നാണു്, എന്റെ അഭിപ്രായം. 
എല്ലാ സഹയാത്രികരുടെയും സഹകരണങ്ങള്‍ക്കഭ്യര്‍ത്ഥിച്ചുകൊണ്ടു് നിര്‍ത്തുന്നു. കൂടുതല്‍ അഭിപ്രായങ്ങളും നിര്‍ദ്ദേശങ്ങളും പ്രതീക്ഷിക്കുന്നു.

(November 03, 2007)
\newpage
