\secstar{കുഞ്ഞന്‍ ടീമുകളുടെ ബിസിനസ് മോഡല്‍}
\vskip 2pt

ഇ­ന്ത്യന്‍ പ്രീ­മി­യര്‍ ലീ­ഗി­ലെ കു­ഞ്ഞന്‍ ടീ­മു­ക­ളാ­ണ് ജയ്‌­പൂര്‍ ആസ്ഥാ­ന­മായ രാ­ജ­സ്ഥാന്‍ റോ­യല്‍­സും, മൊ­ഹാ­ലി 
ആസ്ഥാ­ന­മായ കി­ങ്സ് ഇല­വന്‍ പഞ്ചാ­ബും, കൊല്‍­ക്ക­ത്ത ആസ്ഥാ­ന­മായ കൊല്‍­ക്ക­ത്ത നൈ­റ്റ് റൈ­ഡേ­ഴ്സും. 
മൂ­ന്നും മു­ന്നൂ­റു കോ­ടി­യില്‍ താ­ഴെ മു­തല്‍ മു­ട­ക്കു­ള്ള­വ. മറ്റു രണ്ടു ടീ­മു­ക­ളായ ചെ­ന്നൈ സൂ­പ്പര്‍ കി­ങ്സും ഡല്‍­ഹി ഡെ­യര്‍ 
ഡെ­വിള്‍­സും അക്ഷ­രാര്‍­ത്ഥ­ത്തില്‍ മദ്ധ്യ­നി­ര­ക്കാ­രാ­ണ്: പണ­ത്തി­ന്റെ കാ­ര്യ­ത്തി­ലും പ്ര­ക­ട­ന­ത്തി­ന്റെ കാ­ര്യ­ത്തി­ലും­.

എ­ല്ലാ സീ­സ­ണി­ലും സെ­മി­ക­ളി­ക്കു­ക­യും, രണ്ടു സീ­സ­ണില്‍ ഫൈ­ന­ലി­ലെ­ത്തു­ക­യും ഇപ്രാ­വ­ശ്യം ചാ­മ്പ്യന്‍­മാ­രാ­യി തങ്ങ­ളു­ടെ 
കഴി­വു­തെ­ളി­യി­ക്കു­ക­യും ചെ­യ്തു, ചെ­ന്നൈ. പക്ഷെ ലീ­ഗ് പട്ടി­ക­യി­ലെ പ്ര­ക­ട­ന­ത്തി­ന്റെ അടി­സ്ഥാ­ന­ത്തില്‍ അവര്‍ 
മദ്ധ്യ­നി­ര­ക്കാ­രാ­ണ്. രണ്ടു­സീ­സ­ണില്‍ സെ­മി­ക­ളി­ച്ച ഡല്‍­ഹി, ചെ­ന്നൈ­യെ അപേ­ക്ഷി­ച്ച് മോ­ശ­മാ­ണെ­ങ്കി­ലും വ്യ­ക്ത­മായ 
മദ്ധ്യ­നിര പ്ര­ക­ട­ന­മാ­ണ് പു­റ­ത്തെ­ടു­ത്ത­ത്. ഇപ്രാ­വ­ശ്യം അഞ്ചാ­മ­താ­യാ­ണ് അവര്‍ ലീ­ഗില്‍ ഫി­നി­ഷ് ചെ­യ്ത­ത്.

ഇ­ന്ത്യ സി­മ­ന്റ്സ് ഉട­മ­യും, ­ബി­സി­സി­ഐ­ സെ­ക്ര­ട്ട­റി­യും ഐപി­എല്‍ ഭര­ണ­സ­മി­തി അം­ഗ­വു­മായ എന്‍ ശ്രീ­നി­വാ­സ­നാ­ണ് 
ചെ­ന്നൈ ടീ­മു­ട­മ. കോണ്‍­ഫ്ലി­ക്റ്റ് ഓഫ് ഇന്റ­റ­സ്റ്റി­ന് ഇതി­ലും വ്യ­ക്ത­മായ ഉദാ­ഹ­ര­ണ­മൊ­ന്നും തരാന്‍ കഴി­യു­മെ­ന്നു തോ­ന്നു­ന്നി­ല്ല.
എന്‍ ശ്രീ­നി­വാ­സ­ന് ചെ­ന്നൈ ടീ­മിന്റെ ഉട­മ­സ്ഥാ­വ­കാ­ശം നി­ല­നിര്‍­ത്താന്‍ വേ­ണ്ടി ബി­സി­സിഐ അതി­ന്റെ ഭര­ണ­ഘ­ട­ന­യില്‍ 
പോ­ലും ഭേ­ദ­ഗ­തി വരു­ത്തു­ക­യു­ണ്ടാ­യി. മുന്‍ ബി­സി­സിഐ സെ­ക്ര­ട്ട­റി എസി മു­ത്ത­യ്യ ഇതി­നെ­തി­രെ ഇപ്പോ­ഴും ശ്രീ­നി­വാ­സ­നു­മാ­യി 
നി­യ­മ­പ്പോ­രാ­ട്ട­ത്തി­ലാ­ണ്.

­ഡല്‍­ഹി ടീം മറ്റൊ­രു പ്യു­വര്‍ കോര്‍­പ്പ­റേ­റ്റ് ടീ­മാ­ണ്. ഇന്‍­ഫ്രാ­സ്ട്ര­ക്ചര്‍ രം­ഗ­ത്തെ ഭീ­മന്‍­മാ­രായ ജി­എം­ആര്‍ ഗ്രൂ­പ്പാ­ണ് ഉട­മ­സ്ഥര്‍. 
ഡല്‍­ഹി ഇന്ദി­രാ­ഗാ­ന്ധി അന്താ­രാ­ഷ്ട്ര വി­മാ­ന­ത്താ­വള വി­ക­സ­ന­ത്തി­ന­പ്പു­റം തല­സ്ഥാ­ന­ത്ത് അവര്‍­ക്കു­ള്ള താല്‍­പ്പ­ര്യ­ങ്ങള്‍ 
സം­ര­ക്ഷി­ക്കാന്‍ വേ­ണ്ട ബ്രാന്‍­ഡ് ഇമേ­ജ് സൃ­ഷ്ടി­ക്കുക എന്ന ലക്ഷ്യ­വും ഡല്‍­ഹി ടീം സ്വ­ന്ത­മാ­ക്കി­യ­തി­നു പി­ന്നി­ലു­ണ്ടാ­വ­ണം. 
പ്ര­ധാ­ന­മാ­യും ദക്ഷി­ണേ­ന്ത്യ കേ­ന്ദ്ര­മാ­ക്കി, ബാം­ഗ്ലൂര്‍ ആസ്ഥാ­ന­മാ­ക്കി പ്ര­വര്‍­ത്തി­ക്കു­ന്ന ആന്ധ്രാ­പ്ര­ദേ­ശു­കാര്‍ ഡല്‍­ഹി ടീം വില 
കൊ­ടു­ത്തു വാ­ങ്ങി­യെ­ങ്കില്‍, പ്രാ­ദേ­ശിക ­ക്രി­ക്ക­റ്റ് ടീം സ്വ­ന്ത­മാ­ക്കു­ന്ന­തി­ലൂ­ടെ ലഭി­ക്കു­ന്ന ലോ­യല്‍­റ്റി­യും ബ്രാന്‍­ഡ് ഇമേ­ജും ഒരു 
ലക്ഷ്യ­മാ­യി­രി­ക്ക­ണം­.

­വീ­രേ­ന്ദര്‍ സേ­വാ­ഗും ഗൌ­തം ഗം­ബീ­റും നയി­ക്കു­ന്ന ടീം കളി­ക്ക­ള­ത്തി­ലെ പ്ര­ക­ട­ന­ത്തി­ലൂ­ടെ ഒരി­ക്ക­ലും ഉട­മ­സ്ഥ­രെ 
നി­രാ­ശ­രാ­ക്കി­യ­തു­മി­ല്ല. ഫേ­വ­റൈ­റ്റു­ക­ളാ­യി­ത്ത­ന്നെ കളി തു­ട­ങ്ങു­ക­യും, വി­ശാ­ല­മായ ഒരു ഫാന്‍­ബേ­സ് വളര്‍­ത്തി­യെ­ടു­ക്ക­യും 
ചെ­യ്ത് ടീം വളര്‍­ച്ച­യു­ടെ പാ­ത­യി­ലാ­ണ്. ടീ­മി­ന്റെ സാ­മ്പ­ത്തിക വി­വ­ര­ങ്ങള്‍ ഇതു­വ­രെ പു­റ­ത്തു­വ­ന്നി­ല്ലെ­ങ്കി­ലും, 
ലാ­ഭ­മു­ണ്ടാ­ക്കി­ത്തു­ട­ങ്ങി­യി­രി­ക്ക­ണ­മെ­ന്നാ­ണ് വി­ദ­ഗ്ദ്ധ­മ­തം. മാ­ത്ര­മ­ല്ല, സാ­മ്പ­ത്തിക ക്ര­മ­ക്കേ­ടു­ക­ളെ­ച്ചു­റ്റി­പ്പ­റ്റി­യു­ള്ള 
അന്വേ­ഷ­ണ­ങ്ങ­ളില്‍ പെ­ടാ­ത്ത മൂ­ന്നു കോര്‍­പ്പ­റേ­റ്റ് ടീ­മു­ക­ളില്‍ ഒന്നാ­ണ് ഡല്‍­ഹി­.

ഇ­തു­വ­രെ നമ്മള്‍ കണ്ട അഞ്ചു ടീ­മു­ക­ളില്‍ നി­ന്നും വ്യ­ത്യ­സ്ത­മാ­ണ് മറ്റു­മൂ­ന്നു ടീ­മു­ക­ളു­ടെ അവ­സ്ഥ. മൂ­ന്നും സാ­മ്പ­ത്തിക 
ക്ര­മ­ക്കേ­ടു­കള്‍­ക്ക് അന്വേ­ഷ­ണം നേ­രി­ടു­ന്ന ടീ­മു­ക­ളാ­ണെ­ന്നു തന്നെ പ്ര­ധാ­നം. ഷാ­രൂ­ഖ് ഗൌ­രി ഖാന്‍ ദമ്പ­തി­ക­ളു­ടെ റെ­ഡ് 
ചി­ല്ലി എന്റര്‍­ടൈന്‍­മെ­ന്റും, ജൂ­ഹി ചൌ­ള­യു­ടെ ഭര്‍­ത്താ­വ് ജയ് മേ­ത്ത­യും (ഇ­പ്പോള്‍ കേള്‍­ക്കു­ന്ന­ത്, ആദ്യം പ്ര­ച­രി­പ്പി­ക്ക­പ്പെ­ട്ട­തില്‍ 
നി­ന്ന് വി­രു­ദ്ധ­മാ­യി, ജയ് മേ­ത്ത ഓഹ­രി സ്വ­ന്ത­മാ­ക്കി­യ­ത് ആദ്യ സീ­സ­ണി­നു ശേ­ഷ­മാ­ണെ­ന്നാ­ണ്) പ്ര­മോ­ട്ടു ചെ­യ്യു­ന്ന ടീ­മാ­ണ് 
കൊല്‍­ക്ക­ത്ത നൈ­റ്റ് റൈ­ഡേ­ഴ്സ്. ടീ­മി­ന്റെ ഏറ്റ­വും വലിയ ആകര്‍­ഷ­ണം ഷാ­രൂ­ഖ് തന്നെ­യാ­ണ്.

­ലീ­ഗില്‍ ഗം­ഭീര പ്ര­ക­ട­ന­മൊ­ന്നും ഇതു വരെ കാ­ഴ്ച­വ­ച്ചി­ല്ലെ­ങ്കി­ലും, ഏറ്റ­വും കൂ­ടു­തല്‍ സ്പോണ്‍­സര്‍­ഷി­പ്പ് സ്വ­ന്ത­മാ­യു­ള്ള ടീ­മാ­ണ് 
നൈ­റ്റ് റൈ­ഡേ­ഴ്സ്. കളി­ക്കാന്‍ ഈഡന്‍ ഗാര്‍­ഡന്‍­സ് പോ­ലൊ­രു ഹോം­ഗ്രൌ­ണ്ടും, ദാ­ദ­യെ­ക്കാ­ണാന്‍ വേ­ണ്ടി ജീ­വന്‍ 
നല്‍­കാ­നും തയ്യാ­റാ­കു­ന്ന കാ­ണി­ക­ളും ഉള്ള ടീം. കു­റ­ച്ച് സാ­മ്പ­ത്തിക അച്ച­ട­ക്കം കൂ­ടി കാ­ട്ടി­യി­രു­ന്നെ­ങ്കില്‍ മി­ക­ച്ച­താ­കാ­മാ­യി­രു­ന്നു­.

­ടീ­മി­ന്റെ പ്ര­ധാന സാ­മ്പ­ത്തിക സ്രോ­ത­സ്സ് മു­ക­ളില്‍ പറ­ഞ്ഞ­പോ­ലെ ബ്രാന്‍­ഡ് ഷാ­രൂ­ഖാ­ണ്. ക്രി­ക്ക­റ്റ് കള­ത്തി­ലെ 
പ്ര­ക­ട­ന­ത്തേ­ക്കാ­ളും, കള­ത്തി­നു പു­റ­ത്തെ ഗ്ലാ­മര്‍ ഉപ­യോ­ഗി­ച്ച് ഒരു ടീം നട­ത്തി­ക്കൊ­ണ്ടു­പോ­കാം എന്ന­തി­ന്റെ ഉത്തമ 
ഉദാ­ഹ­ര­ണ­മാ­ണ് കൊല്‍­ക്ക­ത്ത. ഗ്ലാ­മര്‍ ടീം വരു­മാ­ന­ത്തി­ലെ വലി­യൊ­രു പങ്കു വഹി­ക്കു­ന്ന­തി­നാല്‍, പര­മ്പ­രാ­ഗത ക്രി­ക്ക­റ്റ് 
പ്രേ­മി­കള്‍­ക്ക് ദഹി­ക്കാ­ത്ത ആഫ്റ്റര്‍ മാ­ച്ച് പാര്‍­ട്ടി­ക­ളും, ഫാ­ഷന്‍ ഷോ­ക­ളും മറ്റും നട­ത്തി ടീ­മി­ന്റെ ഗ്ലാ­മര്‍ ഉയര്‍­ത്തു­ന്ന­തി­ലും 
ബദ്ധ­ശ്ര­ദ്ധ­നാ­ണ് ഷാ­രൂ­ഖ്. എന്തി­നേ­റെ, ഷാ­രൂ­ഖാ­യി­രു­ന്നു ഇക്കൊ­ല്ല­ത്തെ ഐപി­എല്‍ അവാര്‍­ഡി­ന്റെ (അ­വാര്‍­ഡ് 
നൈ­റ്റ് പാര്‍­ട്ടി­യു­ടെ) കോ ഹോ­സ്റ്റ്.

­മുന്‍ ഐപി­എല്‍ കമ്മീ­ഷ­ണര്‍ ലളി­ത് മോ­ഡി­യു­ടെ ബന്ധു­ക്കള്‍­ക്കു­ള്ള ഓഹ­രി­യു­ടെ പേ­രില്‍ വി­മര്‍­ശ­ന­വി­ധേ­യ­രാ­യ­താ­ണ് 
മൊ­ഹാ­ലി ടീ­മും രാ­ജ­സ്ഥാന്‍ ടീ­മും. രണ്ടു ടീ­മു­ക­ളു­ടെ­യും വി­വ­ര­ങ്ങള്‍ ധാ­രാ­ളം പത്ര­ത്തി­ലും മറ്റും ഇടം പി­ടി­ച്ചി­ട്ടൂ­ള്ള­തി­നാല്‍ വീ­ണ്ടും 
വി­സ്ത­രി­ക്കാന്‍ ശ്ര­മി­ക്കു­ന്നി­ല്ല. വള­രെ വേ­ഗ­ത്തില്‍­ത്ത­ന്നെ നി­ക്ഷേ­പ­കര്‍ ലാ­ഭ­മു­ണ്ടാ­ക്കി­യേ­ക്കാ­വു­ന്ന ടീം എന്നാ­യി­രു­ന്നു 
രാ­ജ­സ്ഥാ­നെ­പ്പ­റ്റി­യു­ള്ള അഭി­പ്രാ­യം­.

ആ­ദ്യ സീ­സണ്‍ ജേ­താ­ക്ക­ളാ­യ­തോ­ടെ ചോ­ദ്യം എന്നു ടീം ലാ­ഭം ഇര­ട്ടി­പ്പി­ക്കു­മെ­ന്നാ­യി. 2009 സീ­സണ്‍ തീര്‍­ന്ന­പ്പോള്‍­ത്ത­ന്നെ, 
ടീം 7.5 മി­ല്യണ്‍ ഡോ­ളര്‍ ലാ­ഭ­മു­ണ്ടാ­ക്കി­യ­താ­യാ­ണ് പ്ര­മോ­ട്ടര്‍­മാര്‍ പറ­ഞ്ഞ­ത്. മാ­ത്ര­മ­ല്ല, ടീ­മി­ന്റെ വാ­ല്യു­വേ­ഷ­നും ഇര­ട്ടി­യോ­ള­മാ­യി 
വര്‍­ദ്ധി­ച്ചു. മോ­ഡി­യു­ടെ ബന്ധു­വായ സു­രേ­ഷ് ചെ­ല്ലാ­റാ­മും ന്യൂ­സ് കോര്‍­പ്പ് ഉടമ റൂ­പര്‍­ട്ട് മര്‍­ഡോ­ക്കി­ന്റെ മകന്‍ ലക്കാന്‍ 
മര്‍­ഡോ­ക്കും പ്ര­ധാന നി­ക്ഷേ­പ­ക­രായ എമര്‍­ജി­ങ് മീ­ഡിയ ഗ്രൂ­പ്പും, ശില്‍­പ്പാ ഷെ­ട്ടി­യും രാ­ജ് കു­ന്ദേ­ര­യു­മാ­ണ് ഇപ്പോള്‍ ടീം 
ഉട­മ­സ്ഥര്‍. ശില്‍­പ്പാ ഷെ­ട്ടി­യു­ടെ വര­വോ­ടെ കളി­യി­ലെ പ്ര­ക­ട­ന­ത്തോ­ടൊ­പ്പം ഗ്ലാ­മ­റും പ്ര­ധാന വരു­മാ­ന­മാര്‍­ഗ്ഗ­മാ­ക്കി­യാ­ണ് 
ടീം മു­ന്നേ­റി­യ­ത്.

­മും­ബൈ­യി­ലെ ലേ­റ്റ് നൈ­റ്റ് പാര്‍­ട്ടി­ക­ളി­ലെ നി­ത്യ സാ­ന്നി­ധ്യ­മായ ഷെ­ട്ടി സി­സ്റ്റേ­ഴ്സി­ന്റെ സാ­ന്നി­ധ്യം ഒര­നു­ഗ്ര­ഹ­വു­മാ­യെ­ന്നു 
പറ­യ­ണം. ഐപി­എ­ല്ലി­ലെ ഗ്ലാ­മര്‍ ഘട­ക­ത്തെ ടൈം­സ് ഓഫ് ഇന്ത്യ­യ്ക്കു വേ­ണ്ടി വി­ല­യി­രു­ത്തി­യി­രു­ന്ന മന്ദി­രാ ബേ­ഡി, 
വി­ജ­യ് മല്യ­യ്ക്കും, ഷാ­രൂ­ഖി­നും ശേ­ഷം, മൂ­ന്നാം സ്ഥാ­ന­മാ­ണ് ഐപി­എല്‍ പാര്‍­ട്ടി­ക­ളു­ടെ കാ­ര്യ­ത്തില്‍ ഷെ­ട്ടി സി­സ്റ്റേ­ഴ്സി­നു 
നല്‍­കി­യ­ത്.

­ഗ്ലാ­മ­റി­ന്റെ കാ­ര്യ­ത്തി­ലും, കളി­യു­ടെ കാ­ര്യ­ത്തി­ലും, അച്ച­ട­ക്ക­ത്തി­ന്റെ കാ­ര്യ­ത്തി­ലും എല്ലാം ശരാ­ശ­രി നി­ലാ­വാ­രം പു­ലര്‍­ത്തിയ 
ടീ­മാ­ണ് മൊ­ഹാ­ലി. കളി­യു­ടെ കാ­ര്യ­ത്തില്‍, ഒന്നാം സീ­സ­ണില്‍ സെ­മി ഫൈ­ന­ലി­സ്റ്റു­ക­ളും, രണ്ടാം സീ­സ­ണില്‍ അഞ്ചാം 
സ്ഥാ­ന­ക്കാ­രു­മാ­യി­രു­ന്ന ടീം അവ­സാന സ്ഥാ­ന­ക്കാ­രാ­യാ­ണ് ഇക്ക­ഴി­ഞ്ഞ സീ­സണ്‍ പൂര്‍­ത്തി­യാ­ക്കി­യ­ത്.

ഈ സീ­സ­ണില്‍ കളി­ക്ക­ള­ത്തി­ലെ കളി­യേ­ക്കാള്‍, പു­റ­ത്തെ കളി­കള്‍­കൊ­ണ്ടാ­ണ് ടീം വാര്‍­ത്ത­ക­ളില്‍ നി­റ­ഞ്ഞ­ത്. 
മോ­ഡി­യു­ടെ ബന്ധു­വായ ഡാ­ബര്‍ ഉടമ മോ­ഹി­ത് ബര്‍­മ്മ­നാ­യി­രു­ന്നു പ്ര­ധാന കാ­ര­ണം. ഇന്നേ­വ­രെ ടീം ടാ­ക്സ് റി­ട്ടേ­ണു­കള്‍ 
സമര്‍­പ്പി­യ്ക്കു­ക­യോ, ഓഡി­റ്റ് റി­പ്പോര്‍­ട്ട് നല്‍­കു­ക­യോ ചെ­യ്തി­ട്ടി­ല്ലെ­ന്ന­തും പത്ര­ത്താ­ളു­ക­ളില്‍ നി­റ­ഞ്ഞു­.

2009ല്‍ ആസ്ത്രേ­ല്യന്‍ കളി­ക്കാ­രു­ടെ അഭാ­വ­മാ­യി­രു­ന്നു പ്ര­ധാന പ്ര­ശ്ന­മാ­യ­തെ­ങ്കില്‍, 2010ല്‍ പ്ര­ധാന താ­രം യു­വ­രാ­ജ് സി­ങ് 
ഫോ­മി­ലേ­ക്കു­യ­രാ­ഞ്ഞ­തും സ്ഥി­ര­ത­യും മൂര്‍­ച്ച­യു­മി­ല്ലാ­ത്ത ബൌ­ളി­ങ്ങു­മാ­ണ് ടീ­മി­നെ കു­ഴ­ക്കി­യ­ത്. സ്പോണ്‍­സര്‍­ഷി­പ്പു­കള്‍ വഴി­യും 
ഷാ­രൂ­ഖി­നെ പിന്‍­പ­റ്റി ടീ­മി­ന്റെ ഗ്ലാ­മര്‍ വര്‍­ദ്ധി­പ്പി­ക്കു­ന്ന പാര്‍­ട്ടി­കള്‍ വഴി­യും സാ­മ്പ­ത്തി­ക­ലാ­ഭ­മാ­ണ് പ്ര­മോ­ട്ടര്‍­മാര്‍ ലക്ഷ്യ­മി­ട്ടി­രു­ന്ന­ത്. 
ധാ­രാ­ളം സ്പോണ്‍­സര്‍­മാര്‍ ടീ­മി­നു­ണ്ടു­താ­നും. പക്ഷെ സ്പോണ്‍­സര്‍­ഷി­പ്പു­കള്‍ നി­ല­നിര്‍­ത്താ­നാ­വ­ശ്യ­മായ ശ്ര­മം 
കളി­ക്ക­ള­ത്തി­ലു­ണ്ടാ­വാ­ത്ത­തും, അനാ­വ­ശ്യ­വി­വാ­ദ­ങ്ങ­ളും, ഗു­രു­ത­ര­മായ സാ­മ്പ­ത്തിക അല­സ­ത­യും ടീ­മി­നെ കു­ഴ­ക്കു­ക­യാ­ണി­പ്പോള്‍. 
എല്ലാ­ത­ര­ത്തി­ലും താ­ഴോ­ട്ടാ­യി­രു­ന്നു കഴി­ഞ്ഞ മൂ­ന്നു സീ­സ­ണില്‍ ടീ­മി­ന്റെ പോ­ക്കെ­ന്ന് നി­സ്സം­ശ­യം പറ­യാം­.

(12 May 2010)\footnote{http://malayal.am/പലവക/പരമ്പര/ബിസിനസ്-ലീഗ്/5401/കുഞ്ഞന്‍-ടീമുകളുടെ-ബിസിനസ്-മോഡല്‍}

\newpage
