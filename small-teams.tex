\secstar{കുഞ്ഞന്‍ ടീമുകളുടെ ബിസിനസ് മോഡല്‍}
\vskip 2pt

ഇന്ത്യന്‍ പ്രീമിയര്‍ ലീഗിലെ കുഞ്ഞന്‍ ടീമുകളാണു് ജയ്‌പൂര്‍ ആസ്ഥാനമായ രാജസ്ഥാന്‍ റോയല്‍സും, മൊഹാലി 
ആസ്ഥാനമായ കിങ്സ് ഇലവന്‍ പഞ്ചാബും, കൊല്‍ക്കത്ത ആസ്ഥാനമായ കൊല്‍ക്കത്ത നൈറ്റ് റൈഡേഴ്സും. 
മൂന്നും മുന്നൂറുകോടിയില്‍ താഴെ മുതല്‍മുടക്കുള്ളവ. മറ്റു രണ്ടു ടീമുകളായ ചെന്നൈ സൂപ്പര്‍ കിങ്സും ഡല്‍ഹി ഡെയര്‍ 
ഡെവിള്‍സും അക്ഷരാര്‍ത്ഥത്തില്‍ മദ്ധ്യനിരക്കാരാണു്: പണത്തിന്റെ കാര്യത്തിലും പ്രകടനത്തിന്റെ കാര്യത്തിലും.

എല്ലാ സീസണിലും സെമികളിക്കുകയും, രണ്ടു സീസണില്‍ ഫൈനലിലെത്തുകയും ഇപ്രാവശ്യം ചാമ്പ്യന്‍മാരായി തങ്ങളുടെ 
കഴിവുതെളിയിക്കുകയും ചെയ്തു, ചെന്നൈ. പക്ഷെ ലീഗ് പട്ടികയിലെ പ്രകടനത്തിന്റെ അടിസ്ഥാനത്തില്‍ അവര്‍ 
മദ്ധ്യനിരക്കാരാണു്. രണ്ടുസീസണില്‍ സെമികളിച്ച ഡല്‍ഹി, ചെന്നൈയെ അപേക്ഷിച്ചു് മോശമാണെങ്കിലും വ്യക്തമായ 
മദ്ധ്യനിര പ്രകടനമാണു് പുറത്തെടുത്തതു്. ഇപ്രാവശ്യം അഞ്ചാമതായാണു് അവര്‍ ലീഗില്‍ ഫിനിഷ് ചെയ്തതു്.

ഇന്ത്യ സിമന്റ്സ് ഉടമയും ബിസിസിഐ സെക്രട്ടറിയും ഐപിഎല്‍ ഭരണസമിതി അംഗവുമായ എന്‍ ശ്രീനിവാസനാണു് 
ചെന്നൈ ടീമുടമ. കോണ്‍ഫ്ലിക്റ്റ് ഓഫ് ഇന്ററസ്റ്റിനു് ഇതിലും വ്യക്തമായ ഉദാഹരണമൊന്നും തരാന്‍ കഴിയുമെന്നു തോന്നുന്നില്ല.
എന്‍ ശ്രീനിവാസനു് ചെന്നൈ ടീമിന്റെ ഉടമസ്ഥാവകാശം നിലനിര്‍ത്താന്‍വേണ്ടി ബിസിസിഐ അതിന്റെ ഭരണഘടനയില്‍പോലും 
ഭേദഗതി വരുത്തുകയുണ്ടായി. മുന്‍ ബിസിസിഐ സെക്രട്ടറി എ.സി. മുത്തയ്യ ഇതിനെതിരെ ഇപ്പോഴും ശ്രീനിവാസനുമായി 
നിയമപ്പോരാട്ടത്തിലാണു്.

ഡല്‍ഹി ടീം പ്യുവര്‍ കോര്‍പ്പറേറ്റ് ടീമാണു്. ഇന്‍ഫ്രാസ്ട്രക്ചര്‍ രംഗത്തെ ഭീമന്‍മാരായ ജിഎംആര്‍ ഗ്രൂപ്പാണു് ഉടമസ്ഥര്‍. 
ഡല്‍ഹി ഇന്ദിരാഗാന്ധി അന്താരാഷ്ട്രവിമാനത്താവളവികസനത്തിനപ്പുറം, തലസ്ഥാനത്തു് അവര്‍ക്കുള്ള താല്‍പ്പര്യങ്ങള്‍ 
സംരക്ഷിക്കാന്‍വേണ്ട ബ്രാന്‍ഡ് ഇമേജ് സൃഷ്ടിക്കുക എന്ന ലക്ഷ്യവും ഡല്‍ഹി ടീം സ്വന്തമാക്കിയതിനു പിന്നിലുണ്ടാവണം. 
പ്രധാനമായും ദക്ഷിണേന്ത്യ കേന്ദ്രമാക്കി (ബാംഗ്ലൂര്‍ ആസ്ഥാനമാക്കി) പ്രവര്‍ത്തിക്കുന്ന ആന്ധ്രാപ്രദേശുകാര്‍ ഡല്‍ഹി ടീം വിലകൊടുത്തു
 വാങ്ങിയെങ്കില്‍, പ്രാദേശിക ക്രിക്കറ്റ് ടീം സ്വന്തമാക്കുന്നതിലൂടെ ലഭിക്കുന്ന ലോയല്‍റ്റിയും ബ്രാന്‍ഡ് ഇമേജും ഒരു 
ലക്ഷ്യമായിരിക്കണം.

വീരേന്ദര്‍ സേവാഗും ഗൌതം ഗംബീറും നയിക്കുന്ന ടീം കളിക്കളത്തിലെ പ്രകടനത്തിലൂടെ ഒരിക്കലും ഉടമസ്ഥരെ 
നിരാശരാക്കിയതുമില്ല. ഫേവറൈറ്റുകളായിത്തന്നെ കളി തുടങ്ങുകയും വിശാലമായ ഒരു ഫാന്‍ബേസ് വളര്‍ത്തിയെടുക്കുകയും 
ചെയ്തു് ടീം വളര്‍ച്ചയുടെ പാതയിലാണു്. ടീമിന്റെ സാമ്പത്തികവിവരങ്ങള്‍ ഇതുവരെ പുറത്തുവന്നില്ലെങ്കിലും 
ലാഭമുണ്ടാക്കിത്തുടങ്ങിയിരിക്കണമെന്നാണു് വിദഗ്ദ്ധമതം. മാത്രമല്ല, സാമ്പത്തിക ക്രമക്കേടുകളെച്ചുറ്റിപ്പറ്റിയുള്ള 
അന്വേഷണങ്ങളില്‍പെടാത്ത മൂന്നു കോര്‍പ്പറേറ്റ് ടീമുകളില്‍ ഒന്നാണു് ഡല്‍ഹി.

ഇതുവരെ നമ്മള്‍ കണ്ട അഞ്ചു ടീമുകളില്‍നിന്നും വ്യത്യസ്തമാണു് മറ്റുമൂന്നു ടീമുകളുടെ അവസ്ഥ. മൂന്നും സാമ്പത്തിക 
ക്രമക്കേടുകള്‍ക്കു് അന്വേഷണം നേരിടുന്ന ടീമുകളാണെന്നുള്ളതുതന്നെ പ്രധാനം. ഷാരൂഖ് ഗൌരി ഖാന്‍ ദമ്പതികളുടെ റെഡ് 
ചില്ലി എന്റര്‍ടൈന്‍മെന്റും, ജൂഹി ചൌളയുടെ ഭര്‍ത്താവു് ജയ് മേത്തയും (ഇപ്പോള്‍ കേള്‍ക്കുന്നതു്, ആദ്യം പ്രചരിപ്പിക്കപ്പെട്ടതില്‍നിന്നു് 
വിരുദ്ധമായി, ജയ് മേത്ത ഓഹരി സ്വന്തമാക്കിയതു് ആദ്യസീസണിനു ശേഷമാണെന്നാണു്) പ്രമോട്ടു ചെയ്യുന്ന ടീമാണു് 
കൊല്‍ക്കത്ത നൈറ്റ് റൈഡേഴ്സ്. ടീമിന്റെ ഏറ്റവും വലിയ ആകര്‍ഷണം ഷാരൂഖ് തന്നെയാണു്.

ലീഗില്‍ ഗംഭീര പ്രകടനമൊന്നും ഇതുവരെ കാഴ്ചവച്ചില്ലെങ്കിലും, ഏറ്റവും കൂടുതല്‍ സ്പോണ്‍സര്‍ഷിപ്പു് സ്വന്തമായുള്ള ടീമാണു് 
നൈറ്റ് റൈഡേഴ്സ്. കളിക്കാന്‍ ഈഡന്‍ ഗാര്‍ഡന്‍സ് പോലൊരു ഹോംഗ്രൌണ്ടും, ദാദയെ കാണാന്‍വേണ്ടി ജീവന്‍ 
നല്‍കാനും തയ്യാറാകുന്ന കാണികളുമുള്ള ടീം. കുറച്ചു് സാമ്പത്തിക അച്ചടക്കം കൂടി കാട്ടിയിരുന്നെങ്കില്‍ മികച്ചതാകാമായിരുന്നു.

ടീമിന്റെ പ്രധാന സാമ്പത്തിക സ്രോതസ്സ് മുകളില്‍ പറഞ്ഞപോലെ ബ്രാന്‍ഡ് ഷാരൂഖാണു്. ക്രിക്കറ്റ് കളത്തിലെ 
പ്രകടനത്തേക്കാളും, കളത്തിനു പുറത്തെ ഗ്ലാമര്‍ ഉപയോഗിച്ചു് ഒരു ടീം നടത്തിക്കൊണ്ടുപോകാം എന്നതിന്റെ ഉത്തമ 
ഉദാഹരണമാണു് കൊല്‍ക്കത്ത. ഗ്ലാമര്‍ ടീം വരുമാനത്തിലെ വലിയൊരു പങ്കു വഹിക്കുന്നതിനാല്‍, പരമ്പരാഗത ക്രിക്കറ്റ് 
പ്രേമികള്‍ക്കു് ദഹിക്കാത്ത ആഫ്റ്റര്‍ മാച്ച് പാര്‍ട്ടികളും ഫാഷന്‍ ഷോകളും മറ്റും നടത്തി ടീമിന്റെ ഗ്ലാമര്‍ ഉയര്‍ത്തുന്നതിലും 
ബദ്ധശ്രദ്ധനാണു് ഷാരൂഖ്. എന്തിനേറെ, ഷാരൂഖായിരുന്നു ഇക്കൊല്ലത്തെ ഐപിഎല്‍ അവാര്‍ഡിന്റെ (അവാര്‍ഡ് 
നൈറ്റ് പാര്‍ട്ടിയുടെ) കോ ഹോസ്റ്റ്.

മുന്‍ ഐപിഎല്‍ കമ്മീഷണര്‍ ലളിത് മോഡിയുടെ ബന്ധുക്കള്‍ക്കുള്ള ഓഹരിയുടെ പേരില്‍ വിമര്‍ശനവിധേയരായതാണു് 
മൊഹാലി ടീമും രാജസ്ഥാന്‍ ടീമും. രണ്ടു ടീമുകളുടെയും വിവരങ്ങള്‍ ധാരാളം പത്രത്തിലും മറ്റും ഇടംപിടിച്ചിട്ടുള്ളതിനാല്‍ വീണ്ടും 
വിസ്തരിക്കാന്‍ ശ്രമിക്കുന്നില്ല. വളരെ വേഗത്തില്‍ത്തന്നെ നിക്ഷേപകര്‍ ലാഭമുണ്ടാക്കിയേക്കാവുന്ന ടീം എന്നായിരുന്നു 
രാജസ്ഥാനെപ്പറ്റിയുള്ള അഭിപ്രായം.

ആദ്യ സീസണ്‍ ജേതാക്കളായതോടെ എന്നാണു് ടീം ലാഭം ഇരട്ടിപ്പിക്കുക എന്നായി ചോദ്യം.  2009 സീസണ്‍ തീര്‍ന്നപ്പോള്‍ത്തന്നെ, 
ടീം 7.5 മില്യണ്‍ ഡോളര്‍ ലാഭമുണ്ടാക്കിയതായാണു് പ്രമോട്ടര്‍മാര്‍ പറഞ്ഞതു്. മാത്രമല്ല, ടീമിന്റെ വാല്യുവേഷനും ഇരട്ടിയോളമായി 
വര്‍ദ്ധിച്ചു. മോഡിയുടെ ബന്ധുവായ സുരേഷ് ചെല്ലാറാമും ന്യൂസ് കോര്‍പ്പ് ഉടമ റൂപര്‍ട്ട് മര്‍ഡോക്കിന്റെ മകന്‍ ലക്കാന്‍ 
മര്‍ഡോക്കും പ്രധാന നിക്ഷേപകരായ എമര്‍ജിങ് മീഡിയ ഗ്രൂപ്പും, ശില്‍പ്പാ ഷെട്ടിയും രാജ് കുന്ദേരയുമാണു് ഇപ്പോള്‍ ടീം 
ഉടമസ്ഥര്‍. ശില്‍പ്പാ ഷെട്ടിയുടെ വരവോടെ കളിയിലെ പ്രകടനത്തോടൊപ്പം ഗ്ലാമറും പ്രധാന വരുമാനമാര്‍ഗ്ഗമാക്കിയാണു് 
ടീം മുന്നേറിയതു്.

മുംബൈയിലെ ലേറ്റ് നൈറ്റ് പാര്‍ട്ടികളിലെ നിത്യസാന്നിധ്യമായ ഷെട്ടി സിസ്റ്റേഴ്സിന്റെ സാന്നിധ്യം ഒരനുഗ്രഹവുമായെന്നു 
പറയണം. ഐപിഎല്ലിലെ ഗ്ലാമര്‍ ഘടകത്തെ ടൈംസ് ഓഫ് ഇന്ത്യയ്ക്കുവേണ്ടി വിലയിരുത്തിയിരുന്ന മന്ദിരാ ബേഡി, 
വിജയ് മല്യയ്ക്കും ഷാരൂഖിനും ശേഷം മൂന്നാംസ്ഥാനമാണു് ഐപിഎല്‍ പാര്‍ട്ടികളുടെ കാര്യത്തില്‍ ഷെട്ടി സിസ്റ്റേഴ്സിനു 
നല്‍കിയതു്.

ഗ്ലാമറിന്റെ കാര്യത്തിലും കളിയുടെ കാര്യത്തിലും അച്ചടക്കത്തിന്റെ കാര്യത്തിലുമെല്ലാം ശരാശരി നിലവാരം പുലര്‍ത്തിയ 
ടീമാണു് മൊഹാലി. കളിയുടെ കാര്യത്തി ഒന്നാം സീസണില്‍ സെമിഫൈനലിസ്റ്റുകളും, രണ്ടാം സീസണില്‍ 
അഞ്ചാംസ്ഥാനക്കാരുമായിരുന്ന ടീം അവസാന സ്ഥാനക്കാരായാണു് ഇക്കഴിഞ്ഞ സീസണ്‍ പൂര്‍ത്തിയാക്കിയതു്.

ഈ സീസണില്‍ കളിക്കളത്തിലെ കളിയേക്കാള്‍, പുറത്തെ കളികള്‍കൊണ്ടാണു് ടീം വാര്‍ത്തകളില്‍ നിറഞ്ഞതു്. 
മോഡിയുടെ ബന്ധുവായ ഡാബര്‍ ഉടമ മോഹിത് ബര്‍മ്മനായിരുന്നു പ്രധാന കാരണം. ഇന്നേവരെ ടീം ടാക്സ് റിട്ടേണുകള്‍ 
സമര്‍പ്പിയ്ക്കുകയോ, ഓഡിറ്റ് റിപ്പോര്‍ട്ടു് നല്‍കുകയോ ചെയ്തിട്ടില്ലെന്നതും പത്രത്താളുകളില്‍ നിറഞ്ഞു.

2009ല്‍ ആസ്ത്രേല്യന്‍ കളിക്കാരുടെ അഭാവമായിരുന്നു പ്രധാന പ്രശ്നമായതെങ്കില്‍, 2010ല്‍ പ്രധാന താരം യുവരാജ് സിങ് 
ഫോമിലേക്കുയരാഞ്ഞതും, സ്ഥിരതയും മൂര്‍ച്ചയുമില്ലാത്ത ബൌളിങ്ങുമാണു് ടീമിനെ കുഴക്കിയതു്. സ്പോണ്‍സര്‍ഷിപ്പുകള്‍ വഴിയും, 
ഷാരൂഖിനെ പിന്‍പറ്റി ടീമിന്റെ ഗ്ലാമര്‍ വര്‍ദ്ധിപ്പിക്കുന്ന പാര്‍ട്ടികള്‍ വഴിയും സാമ്പത്തികലാഭമാണു് പ്രമോട്ടര്‍മാര്‍ ലക്ഷ്യമിട്ടിരുന്നതു്. 
ധാരാളം സ്പോണ്‍സര്‍മാര്‍ ടീമിനുണ്ടുതാനും. പക്ഷെ സ്പോണ്‍സര്‍ഷിപ്പുകള്‍ നിലനിര്‍ത്താനാവശ്യമായ ശ്രമം 
കളിക്കളത്തിലുണ്ടാവാത്തതും അനാവശ്യവിവാദങ്ങളും ഗുരുതരമായ സാമ്പത്തിക അലസതയും ടീമിനെ കുഴക്കുകയാണിപ്പോള്‍. 
എല്ലാതരത്തിലും താഴോട്ടായിരുന്നു കഴിഞ്ഞ മൂന്നു സീസണില്‍ ടീമിന്റെ പോക്കെന്നു് നിസ്സംശയം പറയാം.

\begin{flushright}(12 May, 2010)\footnote{http://malayal.am/പലവക/പരമ്പര/ബിസിനസ്-ലീഗ്/5401/കുഞ്ഞന്‍-ടീമുകളുടെ-ബിസിനസ്-മോഡല്‍}\end{flushright}

\newpage
