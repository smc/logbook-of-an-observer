\section*{ശാസ്ത്രവും മതവും - ചില അഭിപ്രായങ്ങള്‍}
\vskip 2pt

ഇവിടെ എഴുതുന്ന കാര്യങ്ങള്‍ എന്റെ ചിന്തകളും നിരീക്ഷണങ്ങളും തോന്നലും മാത്രമാണ്, അത് ഖണ്ഡിക്കാന്‍ ആര്‍ക്കും 
അവകാശമുണ്ട്, ഞാന്‍ തെളിവുനിരത്തി വാദിക്കുകയല്ല, എന്റെ ചില ചിന്തകള്‍ പങ്കുവയ്ക്കുകയാണ്, വിയോജിപ്പുകള്‍ 
രേഖപ്പെടുത്തുന്നതും, തെളിവുകള്‍ നിരത്തുന്നതും, ഭാവിയില്‍ ഇതു കാണുന്നവര്‍ക്കുപകരിച്ചേക്കും.

ശാസ്ത്രവും മതവും ഒന്നാണോ രണ്ടാണോ, അവയുടെ രണ്ടിന്റേയും വ്യവസ്ഥാപിതരീതികള്‍ എന്താണ് എന്നൊന്നും 
എനിക്കറിയില്ല. ചെറുതായിമാത്രം വായിച്ചിട്ടേയുള്ളു. ശാസ്ത്രത്തിന്റെ കാര്യത്തില്‍ കോണ്‍ഫറന്‍സുകളിലും, 
പ്രസിദ്ധീകരണങ്ങളിലും വിദഗ്ദ്ധരുടെയും അംഗീകാരം (പല സമയത്തും ഇതു വെറും പാഴ്‌വേലയാവാറുമുണ്ട്) നേടി വരുന്ന 
വിവരങ്ങളാണ് ആധികാരികം എന്നറിയപ്പെടുന്നത്. മതത്തിന്റെ കാര്യത്തില്‍ (ഇവിടെയിങ്ങനെ മതം എന്നെഴുതാമോ 
എന്നെനിക്കറിയില്ല, ഞാന്‍ എഴുതുന്നതും പറയുന്നതും ഇന്ത്യന്‍ ഫിലോസോഫിയേയും സാഹിത്യത്തേയും കുറിച്ചാണ്). 
പലപ്പോഴും മാറ്റാനാവാത്തതാണ് (പാടില്ലാത്തതാണ്) മതകാര്യങ്ങള്‍ എന്നാണ് പറയുന്നത്.

അതുകൊണ്ട് ശാസ്ത്രീയമാണോ അല്ലയോ എന്ന പരീക്ഷണങ്ങളും ശാസ്ത്രം മാത്രമാണോ എല്ലാം എന്ന ചിന്തകളൂം ഉപേക്ഷിച്ച്, 
കാലമിത്രയും മാറിയിട്ടും ഈ രീതികള്‍ മാറേണ്ടതില്ല എന്നു പറയുന്നതിന്റെ സങ്കേതത്തെക്കുറിച്ച് ആലോചിക്കണമിപ്പോള്‍ 
എന്നാണെനിക്കുതോന്നുന്നത്. അറിവ് പൂര്‍ണ്ണമല്ലെന്നും അവ അപ്‌ഡേറ്റ് ചെയ്യണമെന്നും അങ്ങനെ ചെയ്താല്‍ എന്തു 
സംഭവിക്കുമെന്നും എന്നാരും എഴുതിക്കാണാത്തതില്‍ എനിക്ക് ശരിക്കും വിഷമമുണ്ട്. ആധുനിക ശാസ്ത്രത്തിന്റെ പോലെ 
എല്ലാം ചര്‍ച്ചചെയ്ത് അപ്പപ്പോഴേക്ക് ബോദ്ധ്യം വന്നതാവണം കൂടുതല്‍ ശരി എന്ന് തീരുമാനിക്കുന്ന സംവിധാനത്തേക്കാളും, 
ഗുരു ശിഷ്യര്‍ക്ക് കാര്യങ്ങള്‍ പകര്‍ന്നുകൊടുക്കുന്ന രീതിയില്‍ പരസ്പരം എതിര്‍ക്കുന്ന രീതികള്‍ക്കും നിലനില്‍ക്കാന്‍ 
ഇടമുണ്ടായിരുന്നു എന്നത് ഒരേ സമയം നല്ലതും ചീത്തയുമായി ഭവിച്ചു (പഴയ ഇന്ത്യന്‍ ഫിലോസഫിരീതിയിലെങ്കിലും) 
എന്നാണെനിക്കുതോന്നുന്നത്. പക്ഷേ പിന്നീട് ഇതൊക്കെ അറിവിനു പകരം ചോദ്യം ചെയ്യാന്‍ പാടില്ലാത്ത അറിവായിമാറുമ്പോഴാണ് 
പ്രശ്നങ്ങള്‍ നിറയുന്നത്. വേറിട്ടു നിലനിന്നിരുന്ന സാഹിത്യവും, വിശ്വാസവും ശാസ്ത്രവും ഒന്നാണെന്നു വരുത്തുമ്പോഴാണ് 
പല കഥകളും തെളിയിക്കേണ്ടത് ആവശ്യമായിമാറുന്നത്. ഞാന്‍ കരുതുന്നിടത്തോളം, തങ്ങള്‍ പൂര്‍ണ്ണരല്ല എന്ന് ബോധമുള്ള 
ആരും ഒരിക്കലും ഇതാണ് എല്ലാം എന്നോ, അല്ലെങ്കില്‍ കൂടുതല്‍ മികച്ച വിശദീകരണങ്ങള്‍ ആവശ്യമില്ല എന്നോ പറയാനിടയില്ല. 
അങ്ങനെയായിരുന്നെങ്കില്‍ തര്‍ക്ക ശാസ്ത്രത്തിന്റെ ആവശ്യമില്ലായിരുന്നല്ലോ! തര്‍ക്കിച്ചും ചര്‍ച്ച ചെയ്തും അംഗീകരിക്കുക 
എന്നത് അന്ന് സ്വീകാര്യമായ രീതിയായിരുന്നിരിക്കണം. പക്ഷേ, സര്‍വ്വകലാശാലാ സംവിധാനമൊന്നുമില്ലാത്ത അന്ന് അറിവിന്റെ
അപ്ഡേഷന്‍ വളരെ പതുക്കെയായിരുന്നിരിക്കണം.

ഇന്ന് ശാസ്ത്ര സങ്കേതമുപയോഗിച്ച് പഴയകാര്യങ്ങള്‍ അരക്കിട്ടുറപ്പിക്കുന്നവര്‍ മറക്കുന്ന കാര്യം, ശാസ്ത്രം, ഇതേതെങ്കിലും 
തെറ്റാണെന്നു കണ്ടാല്‍ ഉടനെ മാറ്റിയെഴുതും, മാറ്റിയെഴുതാനാവാത്ത അറിവിന്റെ കാര്യത്തിലോ? നിലനില്‍പ്പിനു വേണ്ടി 
ശാസ്ത്രത്തെ കൂട്ടുപിടിക്കുന്ന വിഡ്ഢിത്തത്തേക്കാളും എനിക്ക് ഉചിതമായിത്തോന്നുന്നത്, ഇത് അറിവാണെന്ന് അംഗീകരിക്കുകയും, 
ഇതിലെ ശാസ്ത്രവും, സാഹിത്യവുമെല്ലാം തിരഞ്ഞുമാറ്റാന്‍ ശ്രമിക്കുകയുമാണ്. വിഗ്രഹാരാധനയുടെ ശാസ്ത്രീയതയെക്കുറിച്ച് 
പഠിക്കാതെ, അതിന്റെ കാരണങ്ങളെക്കുറിച്ചും, ക്ഷേത്രങ്ങളുടെ സാമൂഹ്യപ്രാധാന്യത്തെക്കുറിച്ചും വിഗ്രഹങ്ങള്‍ 
(കല്ലായാലും ലോഹമായാലും മനുഷ്യനായാലും) ഉണ്ടാക്കാന്‍ നിര്‍ബന്ധിച്ച സാമൂഹ്യ സാഹചര്യങ്ങളെക്കുറിച്ചും 
ചിന്തിക്കുന്നതായിരിക്കും കുറച്ചുകൂടി യുക്തം എന്നെനിക്കു തോന്നുന്നു. പിന്നെ മതമെന്നത് മനുഷ്യന്റെ ആത്മീയാവശ്യം 
നിറവേറ്റാന്‍ വേണ്ടിമാത്രമുള്ളതാണെങ്കില്‍ അതിന് ആധുനിക ശാസ്ത്രത്തിന്റെ സാക്ഷ്യപത്രം എന്തിനാണെന്നാണെനിക്കു 
മനസ്സിലാവാത്തത്. ആത്മാവോ ആത്മീയതയോ അംഗീകരിക്കാത്ത ഒരു സംവിധാനത്തിന്റെ?

കാലോചിതമായി മാറ്റങ്ങള്‍ വരുത്താന്‍ തന്നെ പലയിടങ്ങളിലും വഴികളോ നിര്‍ദ്ദേശങ്ങളോ ഇല്ല. അപ്പോള്‍ 
കാലോചിതമാറ്റങ്ങള്‍ വേണ്ടതല്ലെ എന്നു ചിന്തിക്കുന്ന ജനങ്ങളെ ഞങ്ങള്‍ കാലത്തിനു മുമ്പേ നടക്കുന്നവരാണ് 
എന്ന് മനസ്സിലാക്കിക്കാന്‍ വേണ്ടിമാത്രമാണ് ശാസ്ത്രമുപയോഗിക്കുന്നത്. അവിടെ റിസല്‍ട്ട് ആദ്യമേ റെഡിയായിട്ടുള്ള 
പരീക്ഷണമായതുകൊണ്ട് കാര്യങ്ങള്‍ വളരെ എളുപ്പവുമാണ്. അതായത്, ഉത്തരം നേരത്തേ അറിയാം, എങ്ങനേയെങ്കിലും 
തെളിവുണ്ടാക്കിയാല്‍ മതി എന്ന് രീതിയില്‍ നമ്മള്‍ പരീക്ഷയെഴുതുന്ന പോലെ. ഇത്തരം റിസല്‍ട്ടുകളൊക്കെ വച്ച് പ്രൂവ് 
ചെയ്യുന്ന സംഗതിയുടെ വാലിഡിറ്റി, അതു ബേസ് ചെയ്ത കാര്യം തെറ്റാണെന്നാരെങ്കിലും തെളിയിച്ചാല്‍ തീരും എന്ന് പല 
മുറി ശാസ്ത്രജ്ഞരും മനസ്സിലാക്കുന്നില്ലെന്നു മാത്രം. ഭൂമി ഉരുണ്ടതാണെന്ന് പണ്ട് പറഞ്ഞിട്ടുണ്ട് എന്ന് പറയും പോലെ 
എളുപ്പമല്ല, ദൃഢവുമല്ല, കോസ്മിക് എനര്‍ജിയും, ഊര്‍ജ്ജപ്രസരണവും, റേഡിയേഷനും വച്ചു കളിക്കുന്നത്, ആധുനിക 
ശാസ്ത്രം പലകാര്യങ്ങളിലും നിലപാടുമാറ്റിയേക്കാം, തെളിവുകളുടെ അടിസ്ഥാനത്തില്‍. ആദ്യം ശാസ്ത്രം ഉപയോഗിക്കണമെങ്കില്‍ 
എല്ലാം മാറ്റത്തിനു വിധേയമാണെന്ന് അംഗീകരിക്കേണ്ടിവരും, പലപ്പോഴും ആത്മഹത്യാപരമായ കാര്യം. എന്റെ 
അഭിപ്രായത്തില്‍, പല വിശ്വാസങ്ങളെയും ശാസ്ത്രസത്യങ്ങളാക്കുന്നതിലും നല്ലത്, അതിലേക്ക് നയിച്ച സാമൂഹിക 
സാഹചര്യങ്ങളെക്കുറിച്ച് പഠിക്കുകയെന്നതാണ്. മനുഷ്യനെ മനുഷ്യനാക്കാന്‍ അതായിരിക്കും കുറച്ചുകൂടി ഉപകാരപ്പെടുക.

\subsection*{പ്രതികരണങ്ങള്‍}
\begin{enumerate}
 \item{കിരണ്‍ തോമസ് തോമ്പില്‍}

നല്ല നിരീക്ഷണങ്ങള്‍. ഇതാണ്‌ ഉണ്ടാകേണ്ടത്‌. മതത്തെ ശാസ്ത്രീയമാക്കാന്‍ ശ്രമിക്കുന്നവര്‍ ഇങ്ങനെ ഒക്കെ ചിന്തിച്ചെങ്കില്‍ 
എന്ന് തോന്നിപ്പോകുന്നു. ഉദാഹരണമായി വിഗ്രഹാരാധനയെ സ്വാമി വിവെകാനന്ദന്‍ നിര്‍വചിച്ചത്‌ ഓര്‍ക്കുന്നുണ്ടല്ലോ. 
വിഗ്രഹാരാധനയെ പുച്ഛിച്ച ഒരു രാജാവിനോട്‌ അദ്ദേഹം പറഞ്ഞു അങ്ങയുടെ പിതാവിന്റ ചിത്രം വികൃതമാക്കിയാല്‍ 
അങ്ങേക്ക്‌ വേദനിക്കില്ലെ അത്‌ വെറും കടലാസും പെയിന്റുമാണ്‌ എന്ന് താങ്കള്‍ക്കറിയാം എന്നാല്‍ അതിലൂടെ താങ്കളുടെ 
പിതാവിനെ അനുസ്മരിക്കുന്നു. അതു പോലെ ഒരു ഭക്തന്‍ ഒരു വിഗ്രഹത്തില്‍ പ്രാര്‍ത്ഥിക്കുമ്പോള്‍ അയാള്‍ അതിലെ 
കല്ലിനേയും മണ്ണിനേയുമല്ല മറിച്ച്‌ ദൈവത്തോടാണ്‌ പ്രാര്‍ത്ഥിക്കുന്നത്‌. എന്നാല്‍ ഇന്നത്തെ മത ശാസ്ത്ര ചിന്തകര്‍ പറയുന്നതു് 
മന്ത്രം ചൊല്ലുമ്പോള്‍ കമ്പി വൈബ്രേറ്റ്‌ ചെയ്യുന്നതിനാല്‍ ഊര്‍ജ്ജം വരും അത്‌ ഭക്തരിലെക്ക്‌ പടരും എന്നൊക്കെയാണ്‌.

 \item{പാര്‍ത്ഥന്‍}

    ഇത്തരം ചിന്തകളാണ്‌ ഓരോരുത്തരുടെ മനസ്സിലും വളരേണ്ടത്‌. താങ്കളെഴുതിയതില്‍ ഇഷ്ടമായ ഏറ്റവും പ്രധാനമായ വരികള്‍ : .. (1) പിന്നെ മതമെന്നത് മനുഷ്യന്റെ ആത്മീയാവശ്യം നിറവേറ്റാന്‍ വേണ്ടിമാത്രമുള്ളതാണെങ്കില്‍ അതിന് ആധുനിക ശാസ്ത്രത്തിന്റെ സാക്ഷ്യപത്രം എന്തിനാണെന്നാണെനിക്കു മനസ്സിലാവാത്തത്. ആത്മാവോ ആത്മീയതയോ അംഗീകരിക്കാത്ത ഒരു സംവിധാനത്തിന്റെ? (2) പല വിശ്വാസങ്ങളെയും ശാസ്ത്രസത്യങ്ങളാക്കുന്നതിലും നല്ലത്, അതിലേക്ക് നയിച്ച സാമൂഹിക സാഹചര്യങ്ങളെക്കുറിച്ച് പഠിക്കുകയെന്നതാണ്. ഇതുപോലെ ഗുരുവായൂര്‍ ചുരിദാര്‍ പ്രശ്നത്തില്‍ ഭഗവാന്റെ ഇഷ്ടം എന്ന് പറഞ്ഞതിനു പകരം, ഡ്രസ്സ്‌ കോഡിന്റെ പ്രത്യേകതകള്‍ ശാസ്ത്രീയവും സാമൂഹികവുമായ വീക്ഷണത്തില്‍ വിശദീകരിക്കാമായിരുന്നു. അങ്ങിനെ ഭക്തന്മാരുടെ മാനം രക്ഷിക്കാമായിരുന്നു എന്നെനിയ്ക്ക്‌ തോന്നുന്നു.
 
  \item{Simy Chacko :: സിമി ചാക്കൊ}

    വളരെ നല്ല നിരീക്ഷനങ്ങള്‍... അടുത്ത ഇടെ ഇതേ വിഷയവുമയി വന്ന മിക്ക പൊസ്റ്റികലും കമന്റുകളും വായിച്ചു.. 
ഈ പോസ്റ്റില്‍ വെല്ലുവിളിയുടെ ഭാഷ ഇല്ലാത്തത് എനിക്കിഷ്ടായി
    
\end{enumerate}

\newpage
