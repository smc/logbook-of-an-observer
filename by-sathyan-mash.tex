\newpage
\secstar{ജീവനുള്ള ഓര്‍മ്മ}
ഓരോ മനുഷ്യനും പൂര്‍ണ്ണത പ്രാപിക്കുന്നതു് അവനില്‍ സ്പീഷീസ് ഏല്‍പ്പിച്ചിട്ടുള്ള ധര്‍മ്മം നിറവേറ്റുമ്പോഴാണു് എന്നു് ഞാന്‍ വിശ്വസിക്കുന്നു. "ഏല്‍പ്പിച്ച ധര്‍മ്മം" എന്നതു് ഒരു അമൂര്‍ത്തമായ ആശയമല്ല. മനുഷ്യശരീരത്തില്‍ ഉള്ള അറുപതിനായിരം കോടി കോശങ്ങള്‍ക്കും ഓരോ ധര്‍മ്മം ഉണ്ടു് എന്നു് നമുക്കു് അറിയാം. ഒരേ സൈഗോട്ടില്‍നിന്നു് ജന്മമെടുത്ത ഈ കോശങ്ങള്‍ എങ്ങിനെയാണു്
വിഭിന്ന ധര്‍മ്മങ്ങള്‍ക്കായി നിയോഗിക്കപ്പെടുന്നതു് എന്നതു് നമുക്കറിയില്ല. ഇതുപോലെത്തന്നെ ഓരോ മനുഷ്യനും ജന്മമെടുക്കുമ്പോള്‍ ഏതെങ്കിലും കോശധര്‍മ്മത്തിന്റെ അവതാരമായി തീരുകയും ആ ധര്‍മ്മം ശരീരത്തില്‍ കോശമെന്ന പോലെ സ്പീഷിസില്‍ മനുഷ്യനായി
വര്‍ത്തിക്കുന്നു. റെറ്റീന കോശങ്ങള്‍ അധീശത്ത്വം പ്രാപിച്ച മനുഷ്യന്‍ ഐസ്പെഷലിസ്റ്റാകണം എന്നു് സാരം.

\subsection*{സ്വാതന്ത്ര്യവും ക്രമവും}
ക്രമം സ്വാതന്ത്ര്യത്തിനെതിരാകുമ്പോളൊക്കെ ജൈവീകത അക്രമത്തിന്റെ പാത സ്വീകരിച്ചേക്കാം. മരണമില്ലാത്ത ബാക്റ്റീരിയല്‍ കോശങ്ങള്‍ തനിക്കു് തോന്നുന്ന രീതിയില്‍ വിഭജിച്ചു് അതിനു തോന്നിയ രീതിയില്‍ ലോകത്തില്‍ ഒഴുകി നടന്ന ഒരു നല്ല കാലമുണ്ടായിരുന്നു. പരിണാമത്തിന്റെ കുത്തൊഴുക്കില്‍ ലഘുഭംഗം വഴി വിഭജിച്ചു് പാറിനടന്ന ഈ കോശങ്ങള്‍ ക്രമഭംഗം സീകരിച്ചു് മനുഷ്യ
ശരീരം സ്വീകരിച്ചു് അടിമത്തത്തിലേക്കു് ഓടിക്കയറി. അപ്പോഴും ലഘുഭംഗത്തെ, അതിന്റെ മരണമില്ലായ്മയെ, സ്വാതന്ത്ര്യത്തെ കൈവിടാതെ സൂക്ഷിച്ച ചില കോശങ്ങള്‍ നിലനിന്നുവന്നിട്ടുണ്ടു്. അവ ചിലപ്പോഴെങ്കിലും മനുഷ്യ ശരീരത്തില്‍ അവതാരം എടുത്തിട്ടുണ്ടു്.


\subsection*{പോരാട്ടം}
ലഘുഭംഗത്തിന്റെ ഓര്‍മ്മ കൈമുതലായ ഇക്കൂട്ടര്‍ സ്വാതന്ത്ര്യപ്പോരാളികളാവുകയും മനുഷ്യ വര്‍ഗ്ഗത്തില്‍ അനസ്വീതം നടന്നുകൊണ്ടിരിക്കുന്ന റെജിമെന്റേഷനെതിരെ നിലകൊള്ളുകയും ചെയ്യും.
ഇവര്‍ കാഴ്ചയില്ലാത്തവര്‍ക്കു് കാഴ്ചയും നടക്കാന്‍ കഴിയാത്തവര്‍ക്കു് പാദവും, മണ്ണിനു് ഉപ്പുമായി പ്രവര്‍ത്തിക്കും. ഇവരുടെ ഭാഷ അന്യം നിന്നു പോയെങ്കിലും
അവര്‍ അടിമവര്‍ഗ്ഗത്തോടു് നിരന്തരം അലറിവിളിക്കും. വെള്ളിക്കാശിനു് ഗുരുവിനെ വില്ക്കുമ്പോള്‍ അവന്‍ കരയും.

ജിനേഷെന്ന സ്വതന്ത്ര മനുഷ്യന്‍ അവന്റെ സ്പീഷിസിനോടുള്ള ധര്‍മ്മം നിറവേറ്റിയിരിക്കുന്നു. അവന്‍ എന്റെ കാഴ്ചയില്ലാത്ത കൂട്ടരെ കണ്ണിന്റെ സൗഖ്യത്തിലേക്കു് കൈപിടിച്ചുയര്‍ത്തിയിരിക്കുന്നു. സ്വതന്ത്ര സോഫ്റ്റ്‌വെയറിനു് വേണ്ടി പ്രവര്‍ത്തിക്കുകയും മലയാളം വായിക്കാനുള്ള സോഫറ്റ്‌വെയറിന്റെ (o.c.r) ആധാരം സൃഷ്ടിക്കുകയും വഴി അവന്‍ എന്റെ കൂട്ടരുടെ വഴി കാട്ടി ആയിരിക്കുന്നു.
ജിനേഷ് കാണിച്ച സ്വാതന്ത്ര്യം സ്വന്തം ശരീരത്തിലെ വെളുത്ത രക്താണുക്കളുടേതു് ആണെന്ന തിരിച്ചറിവു് നമ്മെ ഞെട്ടിക്കേണ്ടതില്ല, പിന്നെയോ അവന്റെ കൂട്ടരെ കണ്ടെത്തി കൂട്ടായ്മ സൃഷ്ടിക്കാന്‍ ഉപകരിക്കുംവിധം ഊര്‍ജ്ജം ഉത്പാദിപ്പിക്കാന്‍ ഉപകാരപ്പെടേണ്ടതാണു്.

\hspace*{2em}സത്യശീലന്‍ മാസ്റ്റര്‍, കേരള ഫെഡറേഷന്‍ ഓഫ് ദി ബ്ലൈന്‍ഡ്
\newpage
