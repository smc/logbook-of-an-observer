\section*{Hospital Log 4}
\vskip 2pt

When you are a leukemia patient, the world around you changes a lot. You will find yourself befriending lot of patients and will hear a lot of amazing stories. In a world where names are irrelevant, one of those stories I heard was of her. She is of my age and till recently, used to live right next door. I heard her story in bits and pieces from narratives of her mother, sister and at times, from herself. From what I could gather, she is practically the complete opposite of who I am. Hardworking, God fearing, socially conscious and above all, highly scholarly. In my short life so far, of 25 years, most of which I spent in Kerala, I am yet to see a woman who does not fear the opinion of others. May be the world is supposed to be so and manipulators and tricksters like me are an exemption.

Anyway, when I met her, she was in treatment for a relapsed ALL which took its toll on her during the final days of first year MBBS. It seems she used to complain about constant headaches and other issues now and then. However, her parents dismissed it for her compulsive addiction to studies (If I wake up everyday at 4 and study like she used to, I guess I would have been in bed sooner :)). Interestingly, she had to do blood smear (a pathology test) as a part of her practicals, with her own blood. She found blasts, but thought it was due to some mistake in the procedure. Later, after the exams, when she went to her ancestral home in north Kerala and fell sick only did they figure out the real culprit. Without wasting time she availed treatment and for most of her second year at college, she was in Vellore than in Thrissur. But unlike me, she spent the time reading her textbooks and wrote exams the same year. To everyone's surprise, she came second in her class. It is quite difficult to get your head straight and think normally while in chemotherapy let alone studying for exams, that too for MBBS second year. 

That itself shows her commitment and hard work. In the very few conversations I had with her, I quite felt the immense confidence and courage in that fragile body frame (may be effect of chemotherapy). She was someone capable of bringing a difference to many in the society and I hope she ends up doing it. I don't know how much her MD in Anesthetics (if I guess right) will help her in that, but a good doctor could change a lot of lives. 

That said, we were undergoing the same regime of high dose chemotherapy and both of us were scheduled for a Bone Marrow Transplant soon after. Our conversations often hovered just around treatment regime, hospital doctors, medicines etc. of which she is kind of an expert, and for me, it was all new. May be that deceived me a little. She rarely talked about her earlier stint in Vellore, but her mother and sister gave me lot of details regarding it. I still remember how exactly she was diagnosed for the first time (if she knew I would write about it, she might have never told me any of these :)). But I don�t really know anything about exactly how the relapse happened, except that, I had called her just before the blood test result came to enquire about the details of chemotherapy. Nobody needs to teach me the art of survival, but seldom we find someone who survives catastrophes so elegantly that makes us look back at our whole life and feel ashamed.

All my life as an adult, I led a quite secured one. I made my choices on instincts. Even though doubtful about the paths I liked to explore, my family always stood beside me. But for her, practically her adult life started with ALL, and it was completely under the shadow of one of the most dangerous diseases in existence. I can�t really imagine living so, let alone actually suffering from it. At least I had some 7 years of no-worry and I still have great support and help for being the people�s man that I am.

She once said that seeing lot of successful BMT patients during OP was comforting. Likewise for me, it was comforting to see, befriend and have some wonderful time as well, with someone of my own age and superior scholarly attitude next door. To crown it, seeing her successfully go through the BMT and the difficulties soon after was really helping my confidence too.

Interestingly, meeting her first time made me write. I wrote a small monograph on the social relevance of hair for women and the untold laws governing how one should grow their own hair. I must say the spark in her eyes when she showed her picture with hair made me write it.

\newpage     
