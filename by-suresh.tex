\newpage
\secstar{ജിനേഷ് അനുസ്മരണം}

ജിനേഷിനെ ആദ്യമായി കാണുന്നതു് smc കൂട്ടായ്മയില്‍ ഒരു localization camp-ല്‍ വച്ചാണെന്നാണു് ഓര്‍മ്മ. പിന്നീടു് Google SoC പ്രൊജക്റ്റില്‍ അദ്ദേഹം പങ്കെടുത്തപ്പോള്‍ അതിനൊപ്പം ഒരു മെന്റര്‍ എന്ന നിലയില്‍ പ്രവര്‍ത്തിക്കുകയും ചെയ്തു. പിന്നെയറിഞ്ഞതു് കുറ്റിപ്പുറം എം.ഇ.എസ്. ലെ എന്‍ജിനിയറിംഗ്  പഠനത്തിനു ശേഷം ജിനേഷ് ഹൈദരാബാദ് IIIT യില്‍ ചേര്‍ന്നതാണു്. അതിനു ശേഷം വ്യക്തിപരമായി ഇടപാടുകള്‍ കാര്യമായി ഉണ്ടായിരുന്നില്ലെങ്കിലും ജിനേഷ് എഴുതിയിരുന്ന Log Book of an Observer എന്ന സ്വന്തം ബ്ലോഗ്, മലയാളം വാര്‍ത്താ പോര്‍ട്ടല്‍ എന്നിവ മുഖേന വ്യക്തി എന്ന നിലയില്‍ അദ്ദേഹത്തിന്റെ കാഴ്ചപ്പാടുകളും നിലപാടുകളും സവിശേഷതകളും അടുത്തറിയാന്‍ സാധിച്ചിരുന്നു.

രോഗബാധിതനായ ജിനേഷിനെപറ്റി വളരെ വൈകിയാണറിഞ്ഞതു്. കഠിനമായ രോഗാവസ്ഥയിലും തികഞ്ഞ ആത്മവിശ്വാസവും യാഥാര്‍ത്ഥ്യ ബോധവും വച്ചുപുലര്‍ത്താന്‍ അദ്ദേഹത്തിനായി എന്നകാര്യം  അദ്ദേഹത്തെപ്പറ്റി വന്ന പല ലേഖനങ്ങളും അനുസ്മരണങ്ങളും വായിച്ചപ്പോള്‍ മാത്രമാണു്  വ്യക്തമായതു്. വരുംകാലങ്ങളിലും ജിനേഷിനെ അനുസ്മരിക്കാന്‍ എന്തുകൊണ്ടും ഉചിതമായതാണു് അദ്ദേഹത്തിന്റെ ലേഖനങ്ങളുടെ പുസ്തകപ്പതിപ്പു്. അതിനായി അദ്ദേഹത്തിന്റെ കര്‍മ്മമണ്ഡലം കൂടിയായിരുന്ന സ്വതന്ത്ര/തുറസ്സായ സോഫ്റ്റ് വെയര്‍ സങ്കേതങ്ങള്‍ തന്നെ ഉപയോഗിച്ചതു് മറ്റൊരന്വയം കൂടിയായി. എല്ലാ ആശംസകളും.

\begin{flushright}സുരേഷ് പി (സുറുമ), സ്വതന്ത്രമലയാളം  കമ്പ്യൂട്ടിങ്ങ്\end{flushright}
\newpage
