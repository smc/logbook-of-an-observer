\secstar{ഫോര്‍മുല വണ്‍ ഇന്ത്യയിലെത്തുമ്പോള്‍}
\vskip 2pt
%\enlargethispage*{4\baselineskip}

ഫോര്‍മുല വണ്‍ 2010 സീസണ്‍ അവസാനിച്ചിട്ടേതാണ്ടു് രണ്ടുമാസം തികയുന്നു. ഫോര്‍മുല വണ്‍ സര്‍ക്യൂട്ടു് 
ക്രിസ്മസ് അവധിക്കുശേഷം വീണ്ടും സജീവമായിത്തുടങ്ങി. മുന്‍നിരടീമുകളൊക്കെ അവരുടെ ഡ്രൈവര്‍മാരെ 
നിലനിര്‍ത്തിയപ്പോള്‍ മധ്യനിര-വാലറ്റടീമുകളില്‍ ധാരാളം അഴിച്ചുപണികള്‍ നടക്കുന്നു. പല ടീമുകളും ഡ്രൈവര്‍മാരെ 
പ്രഖ്യാപിച്ചു കഴിഞ്ഞു. ഈ മാസം കൂടുതല്‍ പ്രഖ്യാപനങ്ങളുണ്ടാകുമെന്നു പ്രതീക്ഷിക്കപ്പെടുന്നു.

ജനുവരി രണ്ടാംവാരത്തോടെ ടീമുകള്‍ പുതിയ കാറുകള്‍ ഇറക്കിത്തുടങ്ങും. ഫോഴ്സ് ഇന്ത്യ സ്പെയിനില്‍ നടക്കുന്ന 
ടെസ്റ്റിങ്ങില്‍ പഴയ കാറായിരിക്കും ഉപയോഗിക്കുക എന്നു് പ്രഖ്യാപിച്ചുകഴിഞ്ഞു. ഫെറാരി ജനുവരി അവസാനം 2011 
സീസണിലേക്കുള്ള കാര്‍ പുറത്തിറക്കുമെന്നു് പ്രഖ്യാപിച്ചിട്ടുണ്ടു്. വലന്‍സിയയില്‍ ഫെബ്രുവരി ഒന്നുമുതല്‍ മൂന്നുവരെ 
നടക്കുന്ന ടെസ്റ്റിങ്ങിലായിരിക്കും മിക്ക കാറുകളുടേയും അരങ്ങേറ്റം. എന്നാല്‍ ഫോഴ്സ് ഇന്ത്യയും, കഴിഞ്ഞ വര്‍ഷം 
റെഡ്ബുള്ളിന്റെ നയം പിന്തുടരാന്‍ താല്‍പ്പര്യമുള്ള മറ്റുള്ളവരും ഒരാഴ്ചയ്ക്കുശേഷം ജെറെസ്സില്‍ നടക്കുന്ന 
ടെസ്റ്റിങ്ങിലായിരിക്കും കാര്‍ പുറത്തിറക്കാന്‍ സാധ്യത. ഇതു് സാധാരണ കാറിന്റെ എയ്റോഡൈനാമിക് കഴിവുകള്‍ 
മികച്ചതാക്കാനാണു് ഉപയോഗിക്കാറു്. കഴിഞ്ഞവര്‍ഷം ഹിസ്പാനിക് റേസിങ് ചെയ്തതുപോലെ റേസ് ഡെബ്യൂ ആരും 
ചെയ്യില്ലെന്നു കരുതാം.

കഴിഞ്ഞവര്‍ഷത്തെ അപേക്ഷിച്ചു് ടീമുകളുടെ എണ്ണത്തിലൊന്നും വ്യത്യാസമുണ്ടായിട്ടില്ല. റേസുകളുടെ എണ്ണം 20 
ആയി വര്‍ദ്ധിച്ചു. ഒക്റ്റോബര്‍ അവസാനം നോയിഡയിലെ ട്രാക്കില്‍ നടക്കുന്ന ഇന്ത്യന്‍ ഗ്രാന്‍പ്രിയാണു് പുതുതായി 
കലണ്ടറില്‍ ഇടംപിടിച്ചതു്. ആദ്യറേസ് ബഹ്റൈനിലെ സാക്കിറില്‍ മാര്‍ച്ചു് 11, 12, 13 തിയ്യതികളിലാണെങ്കില്‍ 
സീസണ്‍ ഫിനാലെ സാവോപോളോയില്‍ നവംബര്‍ 25, 26, 27 തിയ്യതികളിലാണു്. എട്ടു് റേസുകള്‍ ഏഷ്യയിലും 
ഒന്‍പതെണ്ണം യൂറോപ്പിലുമാണു്. ചൈനീസ് റേസിനുശേഷം തുര്‍ക്കിയില്‍ തുടങ്ങുന്ന യൂറോപ്യന്‍പാദം ഇടയ്ക്കൊരു 
വേനലവധിയോടുകൂടി ആവസാനിക്കുന്നതു് സെപ്റ്റമ്പര്‍ 9, 10, 11 തിയ്യതികളില്‍ നടക്കുന്ന ഇറ്റാലിയന്‍ 
റേസോടുകൂടിയാണു്. ഇതിനിടയ്ക്കു് കനേഡിന്‍ ഗ്രാന്‍പ്രീ മാത്രമാണു് യൂറോപ്പിനു പുറത്തുള്ളതു്. പിന്നീടു് ഏഷ്യയില്‍ 
തിരിച്ചെത്തുന്ന സീസണ്‍, ഫിനാലെയ്ക്കായി ബ്രസീലിലേക്കു പോകും.

റെഡ്ബുള്ളും മക്‌ലാരനും ഫെറാരിയും മെഴ്സിഡസും തങ്ങളുടെ ഡ്രൈവര്‍മാരെ നിലനിര്‍ത്തിയപ്പോള്‍ വില്യംസ് 
ബാരിക്കെല്ലോവിനെ നിലനിര്‍ത്തുകയും ഹള്‍ക്കന്‍ബര്‍ഗ്ഗിനെ കൈവിടുകയും ചെയ്തു. റെനോയും ലോട്ടസും അവരുടെ 
ഡ്രൈവര്‍മാരെ നിലനിര്‍ത്തിയിട്ടുണ്ടു്. ലൂകാസ് ഡി ഗ്രാസ്സിക്കുപകരം വിര്‍ജിന്‍ ജെറോം ഡി അമ്പ്രോസ്സിയോയെ ടിമോ 
ഗ്ലോക്കിന്റെ കൂട്ടാളിയാക്കി. വില്യംസില്‍ ബാരിക്കെല്ലോവിനു കൂട്ടാവുന്നതു് വെനുസ്വേലക്കാരന്‍ പാസ്റ്റര്‍ 
മാല്‍ഡൊണാഡോ ആണു്.

ഹിസ്പാനിക് റേസിങ്ങിനുവേണ്ടി ഇന്ത്യന്‍ ഡ്രൈവര്‍ നരേന്‍ കാര്‍ത്തികേയന്‍ ഒരിക്കല്‍ക്കൂടി ട്രാക്കിലിറങ്ങും. ആരാണു് 
കാര്‍ത്തികേയനു കൂട്ടാവുകയെന്നതു് ഇതുവരെ ഉറപ്പായിട്ടില്ലെങ്കിലും സെന്നയായിരിക്കില്ലെന്നു ടീം വ്യക്തമാക്കിക്കഴിഞ്ഞു. 
കൊബിയാഷിക്കു കൂട്ടായി ഹെഡ്ഫീല്‍ഡിനു പകരം സെര്‍ജിയോ പെരസ് സൗബറില്‍ സ്ഥാനമുറപ്പാക്കി. ടോറോ
 റോസോയും ഫോഴ്സ് ഇന്ത്യയും ലൈനപ്പ് ഇനിയും പ്രഖ്യാപിച്ചിട്ടില്ല. ബ്യുയമിയും അല്‍ഗ്യുസുരിയും ടോറോ റോസോയില്‍ 
തുടരാന്‍ സാധ്യതയുണ്ടെന്നാണറിയുന്നതു്.

ബ്രിഡ്ജ്സ്റ്റോണ്‍ പിന്‍വാങ്ങിയതിനെത്തുടര്‍ന്നു് റേസിലെ ഏക ടയര്‍ സപ്ലയറായി പിറേലി വീണ്ടും രംഗത്തുവരും. 20 
കൊല്ലത്തിനുശേഷമാണു് പിറേലി ഫോര്‍മുല വണ്ണില്‍ തിരിച്ചെത്തുന്നതു്. വരുന്ന മൂന്നുവര്‍ഷത്തേക്കാണു് കരാര്‍. ഓരോ
റേസ് വാരാന്ത്യത്തിനും അനുവദിച്ചിരുന്ന ടയര്‍ സെറ്റുകളുടെ എണ്ണം പതിനാലില്‍നിന്നും പതിനൊന്നായിക്കുറച്ചിട്ടുണ്ടു്. 
ടയര്‍ റിട്ടേണ്‍ പോളിസിയിലും മാറ്റങ്ങളുണ്ടു്. അതുപോലെ ഡ്രൈറേസില്‍ ഓപ്ഷനും പ്രൈമും 
ഉപയോഗിച്ചില്ലെങ്കില്‍ മുപ്പതു സെക്കന്റ് പെനാല്‍ട്ടിയുണ്ടു്. ഗിയര്‍ബോക്സുകള്‍ നാലിനുപകരം അഞ്ചു് റേസുകളില്‍ 
ഉപയോഗിക്കണമെന്നും നിയമമുണ്ടു്.

മറികടക്കലും ആവേശവും വര്‍ദ്ധിപ്പിക്കാനായി, KERS തിരിച്ചുവരുമ്പോള്‍, ഡ്രൈവര്‍ക്കു് അഡ്ജസ്റ്റു ചെയ്യാവുന്ന 
പിന്‍ചിറകുകള്‍ പുതുതായി വരുന്നുണ്ടു്. എന്നാല്‍ ഡിഫ്യൂസറുകളും ഏറെ വിവാദമായ എഫ്-ഡക്റ്റും നിരോധിച്ചിട്ടുമുണ്ടു്. 
ടയറുകളുടെ സുരക്ഷ വര്‍ദ്ധിപ്പിക്കാനായി ഒരു പിടുത്തം (tether) അധികംവയ്ക്കാനും നിയമമുണ്ടാക്കിയിട്ടുണ്ടു്.

എന്നാല്‍ 107% നിയമമായിരിക്കും 2011ല്‍ ഏറ്റവും വിഷയമാവുക. യോഗ്യതാ റൗണ്ടിന്റെ ആദ്യപാദത്തിലെ 
ഏറ്റവും വേഗമേറിയ സമയത്തെക്കാള്‍ 107% പിറകിലുള്ളവരെ റേസില്‍ പങ്കെടുക്കാന്‍ അനുവദിക്കില്ലെന്നതാണിതു്. 
എന്നാല്‍ ചില പ്രത്യേക സാഹചര്യങ്ങളില്‍ (പരിശീലന റൗണ്ടുകളില്‍ മികച്ച സമയം രേഖപ്പെടുത്തിയിട്ടുണ്ടെങ്കിലും മറ്റും) 
സ്റ്റ്യുവാര്‍ഡുകള്‍ക്കു് അനുവാദം നല്‍കാം. സ്റ്റ്യൂവാര്‍ഡുകള്‍ക്കു് കൂടുതല്‍ അധികാരം നല്‍കിയപ്പോള്‍ ടീം ഓര്‍ഡറുകളുടെ 
മേലുണ്ടായിരുന്ന നിരോധനം പിന്‍വലിച്ചിട്ടുമുണ്ടു്. ടീം കര്‍ഫ്യൂ ആണു് മറ്റൊരു പുതിയ നിയമം. രാത്രി പന്ത്രണ്ടുമുതല്‍ 
രാവിലെ ആറുവരെയോ (സെഷന്‍ പത്തുമണിക്കാണെങ്കില്‍), രാത്രി ഒന്നുമുതല്‍ ഏഴുവരെയോ (സെഷന്‍ പതിനൊന്നു 
മണിക്കാണെങ്കില്‍) കാറിന്റെ വര്‍ക്കുമായി ബന്ധപ്പെട്ട ആരും റേസ് സര്‍ക്യൂട്ടില്‍ പ്രവേശിക്കരുതെന്നാണു് പുതിയ നിയമം.

ഇന്ത്യയിലേക്കു് ആദ്യമായി ഫോര്‍മുല വണ്‍ എത്തുമ്പോള്‍ നരേന്‍ കാര്‍ത്തികേയനും ഫോഴ്സ് ഇന്ത്യയും ഇന്ത്യന്‍ 
സാന്നിദ്ധ്യങ്ങളായി ട്രാക്കിലുണ്ടാകുമെന്നു് ഏതാണ്ടുറപ്പായിക്കഴിഞ്ഞു. ഫെബ്രുവരി ഒന്നിനു് വലന്‍സിയയില്‍ ടെസ്റ്റിങ്ങിനു 
തുടക്കമാവുന്നതോടുകൂടി, ഔദ്യോഗികമായി 2011 സീസണിനു തുടക്കമാവും. ഫോര്‍മുലവണ്‍ ലോകവും 
സജീവമായിക്കൊണ്ടിരിക്കുകയാണു്. കഴിഞ്ഞ നാലുവര്‍ഷങ്ങളില്‍ നാലു് വ്യത്യസ്ത ചാമ്പ്യന്‍മാരെയാണു് നമുക്കു് 
സീസണ്‍ സമ്മാനിച്ചതു്. മാത്രമല്ല ട്രാക്കില്‍ തീപാറുന്ന പോരാട്ടങ്ങളും അവരൊരുക്കി. ആരായിരിക്കും 2011ന്റെ ചാമ്പ്യന്‍? 
മുപ്പതിനുമേലെയുള്ള ധാരാളംപേര്‍ അണിനിരക്കുന്ന 2011ല്‍ അവരാരെങ്കിലും കിരീടം ചൂടുമോ? സാക്കിര്‍ ട്രാക്കില്‍ 
എന്‍ജിന്‍ മുഴങ്ങുംവരെ നമുക്കു കാത്തിരിക്കാം.

\begin{flushright}(13 January, 2011)\footnote{http://malayal.am/വിനോദം/കായികം/9663/ഫോര്‍മുല-വണ്‍-ഇന്ത്യയിലെത്തുമ്പോള്‍}\end{flushright}

\newpage
