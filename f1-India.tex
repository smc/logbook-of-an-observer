\secstar{ഫോര്‍മുല വണ്‍ ഇന്ത്യയിലെത്തുമ്പോള്‍}
\vskip 2pt

ഫോര്‍­മുല വണ്‍ 2010 സീ­സണ്‍ അവ­സാ­നി­ച്ചി­ട്ടേ­താ­ണ്ടു് രണ്ടു­മാ­സം തി­ക­യു­ന്നു. ­ഫോര്‍­മുല വണ്‍ സര്‍­ക്യൂ­ട്ടു് 
ക്രി­സ്മ­സ് അവ­ധി­ക്കു ശേ­ഷം വീ­ണ്ടും സജീ­വ­മാ­യി­ത്തു­ട­ങ്ങി. മുന്‍­നിര ടീ­മു­ക­ളൊ­ക്കെ അവ­രു­ടെ ഡ്രൈ­വര്‍­മാ­രെ 
നി­ല­നിര്‍­ത്തി­യ­പ്പോള്‍ മധ്യ­നി­ര, വാ­ല­റ്റ­ടീ­മു­ക­ളില്‍ ധാ­രാ­ളം അഴി­ച്ചു­പ­ണി­കള്‍ നട­ക്കു­ന്നു. പല ടീ­മു­ക­ളും ഡ്രൈ­വര്‍­മാ­രെ 
പ്ര­ഖ്യാ­പി­ച്ചു കഴി­ഞ്ഞു. ഈ മാ­സം കൂ­ടു­തല്‍ പ്ര­ഖ്യാ­പ­ന­ങ്ങ­ളു­ണ്ടാ­കു­മെ­ന്നു പ്ര­തീ­ക്ഷി­ക്ക­പ്പെ­ടു­ന്നു.

­ജ­നു­വ­രി രണ്ടാം വാ­ര­ത്തോ­ടെ ടീ­മു­കള്‍ പു­തിയ കാ­റു­കള്‍ ഇറ­ക്കി­ത്തു­ട­ങ്ങും. ഫോ­ഴ്സ് ഇന്ത്യ സ്പെ­യി­നില്‍ നട­ക്കു­ന്ന 
ടെ­സ്റ്റി­ങ്ങില്‍ പഴയ കാ­റാ­യി­രി­ക്കും ഉപ­യോ­ഗി­ക്കുക എന്നു പ്ര­ഖ്യാ­പി­ച്ചു കഴി­ഞ്ഞു. ഫെ­റാ­രി ജനു­വ­രി അവ­സാ­നം 2011 
സീ­സ­ണി­ലേ­ക്കു­ള്ള കാര്‍ പു­റ­ത്തി­റ­ക്കു­മെ­ന്നു പ്ര­ഖ്യാ­പി­ച്ചി­ട്ടു­ണ്ടു്. വലന്‍­സി­യ­യില്‍ ഫെ­ബ്രു­വ­രി ഒന്നു മു­തല്‍ മൂ­ന്നു വരെ 
നട­ക്കു­ന്ന ടെ­സ്റ്റി­ങ്ങി­ലാ­യി­രി­ക്കും മി­ക്ക കാ­റു­ക­ളു­ടേ­യും അര­ങ്ങേ­റ്റം. എന്നാല്‍ ഫോ­ഴ്സ് ഇന്ത്യ­യും, കഴി­ഞ്ഞ വര്‍­ഷം 
റെ­ഡ്ബു­ള്ളി­ന്റെ നയം പി­ന്തു­ട­രാന്‍ താല്‍­പ്പ­ര്യ­മു­ള്ള മറ്റു­ള്ള­വ­രും ഒരാ­ഴ്ച­യ്ക്കു ശേ­ഷം ജെ­റെ­സ്സില്‍ നട­ക്കു­ന്ന 
ടെ­സ്റ്റി­ങ്ങി­ലാ­യി­രി­ക്കും കാര്‍ പു­റ­ത്തി­റ­ക്കാന്‍ സാ­ധ്യ­ത. ഇതു സാ­ധാ­രാണ കാ­റി­ന്റെ എയ്റോ­ഡൈ­നാ­മി­ക് കഴി­വു­കള്‍ 
മി­ക­ച്ച­താ­ക്കാ­നാ­ണു് ഉപ­യോ­ഗി­ക്കാ­റു്. കഴി­ഞ്ഞ വര്‍­ഷം ഹി­സ്പാ­നി­ക് റേ­സി­ങ് ചെ­യ്ത­തു പോ­ലെ റേ­സ് ഡെ­ബ്യൂ ആരും 
ചെ­യ്യി­ല്ലെ­ന്നു കരു­താം­.

­ക­ഴി­ഞ്ഞ വര്‍­ഷ­ത്തെ അപേ­ക്ഷി­ച്ച് ടീ­മു­ക­ളു­ടെ എണ്ണ­ത്തി­ലൊ­ന്നും വ്യ­ത്യാ­സ­മു­ണ്ടാ­യി­ട്ടി­ല്ല. റേ­സു­ക­ളു­ടെ എണ്ണം 20 
ആയി വര്‍­ദ്ധി­ച്ചു. ഒക്റ്റോ­ബര്‍ അവ­സാ­നം നോ­യി­ഡ­യി­ലെ ട്രാ­ക്കില്‍ നട­ക്കു­ന്ന ഇന്ത്യന്‍ ഗ്രാന്‍­പ്രി­യാ­ണു് പു­തു­താ­യി 
കല­ണ്ട­റില്‍ ഇടം പി­ടി­ച്ച­തു്. ആദ്യ­റേ­സ് ബഹ്റൈ­നി­ലെ സാ­ക്കി­റില്‍ മാര്‍­ച്ച് 11, 12, 13 തി­യ്യ­തി­ക­ളി­ലാ­ണെ­ങ്കില്‍ 
സീ­സണ്‍ ഫി­നാ­ലെ സാ­വോ­പോ­ളോ­യില്‍ നവം­ബര്‍ 25, 26, 27 തി­യ്യ­തി­ക­ളി­ലാ­ണു്. എട്ടു­റേ­സു­കള്‍ ഏഷ്യ­യി­ലും 
ഒന്‍­പ­തെ­ണ്ണം യൂ­റോ­പ്പി­ലു­മാ­ണു്. ചൈ­നീ­സു് റേ­സി­നു ശേ­ഷം തുര്‍­ക്കി­യില്‍ തു­ട­ങ്ങു­ന്ന യൂ­റോ­പ്യന്‍­പാ­ദം, ഇട­യ്ക്കൊ­രു 
വേ­ന­ല­വ­ധി­യോ­ടു­കൂ­ടി ആവ­സാ­നി­ക്കു­ന്ന­തു് സെ­പ്റ്റ­മ്പര്‍ 9, 10, 11 തി­യ്യ­തി­ക­ളില്‍ നട­ക്കു­ന്ന ഇറ്റാ­ലി­യന്‍ 
റേ­സോ­ടു­കൂ­ടി­യാ­ണു്. ഇതി­നി­ട­യ്ക്കു് കനേ­ഡിന്‍ ഗ്രാന്‍­പ്രീ മാ­ത്ര­മാ­ണു് യൂ­റോ­പ്പി­നു പു­റ­ത്തു­ള്ള­തു്. പി­ന്നീ­ടു് ഏഷ്യ­യില്‍ 
തി­രി­ച്ചെ­ത്തു­ന്ന സീ­സണ്‍, ഫി­നാ­ലെ­യ്ക്കാ­യി ബ്ര­സീ­ലി­ലേ­ക്കു പോ­കും­.

­റെ­ഡ്ബു­ള്ളും മക്‌­ലാ­ര­നും ഫെ­റാ­രി­യും മെ­ഴ്സി­ഡ­സും തങ്ങ­ളു­ടെ ഡ്രൈ­വര്‍­മാ­രെ നി­ല­നിര്‍­ത്തി­യ­പ്പോള്‍ വി­ല്യം­സ് 
ബാ­രി­ക്കെ­ല്ലോ­വി­നെ നി­ല­നിര്‍­ത്തു­ക­യും ഹള്‍­ക്കന്‍­ബര്‍­ഗ്ഗി­നെ കൈ­വി­ടു­ക­യും ചെ­യ്തു. റെ­നോ­യും ലോ­ട്ട­സും അവ­രു­ടെ 
ഡ്രൈ­വര്‍­മാ­രെ നി­ല­നിര്‍­ത്തി­യി­ട്ടു­ണ്ടു്. ലൂ­കാ­സ് ഡി ഗ്രാ­സ്സി­ക്കു­പ­ക­രം വിര്‍­ജിന്‍ ജെ­റോം ഡി അമ്പ്രോ­സ്സി­യോ­യെ ടി­മോ 
ഗ്ലോ­ക്കി­ന്റെ കൂ­ട്ടാ­ളി­യാ­ക്കി. വി­ല്യം­സില്‍ ബാ­രി­ക്കെ­ല്ലോ­വി­നു കൂ­ട്ടാ­വു­ന്ന­തു് വെ­നു­സ്വേ­ല­ക്കാ­രന്‍ പാ­സ്റ്റര്‍ 
മാല്‍­ഡൊ­ണാ­ഡോ ആണു്.

­ഹി­സ്പാ­നി­ക് റേ­സി­ങ്ങി­നു വേ­ണ്ടി ഇന്ത്യന്‍ ഡ്രൈ­വര്‍ നരേന്‍ കാര്‍­ത്തി­കേ­യന്‍ ഒരി­ക്കല്‍­ക്കൂ­ടി ട്രാ­ക്കി­ലി­റ­ങ്ങും. ആരാ­ണു 
കാര്‍­ത്തി­കേ­യ­നു കൂ­ട്ടാ­വു­ക­യെ­ന്ന­തു ഇതു­വ­രെ ഉറ­പ്പാ­യി­ട്ടി­ല്ലെ­ങ്കി­ലും സെ­ന്ന­യാ­യി­രി­ക്കി­ല്ലെ­ന്നു ടീം വ്യ­ക്ത­മാ­ക്കി­ക്ക­ഴി­ഞ്ഞു. 
കൊ­ബി­യാ­ഷി­ക്കു കൂ­ട്ടാ­യി ഹെ­ഡ്ഫീല്‍­ഡി­നു പക­രം സെര്‍­ജി­യോ പെ­ര­സ് സൌ­ബ­റില്‍ സ്ഥാ­ന­മു­റ­പ്പാ­ക്കി. ടോ­റോ
 റോ­സോ­യും ഫോ­ഴ്സ്ഇ­ന്ത്യ­യും ലൈ­ന­പ്പ് ഇനി­യും പ്ര­ഖ്യാ­പി­ച്ചി­ട്ടി­ല്ല. ബ്യു­യ­മി­യും അല്‍­ഗ്യു­സു­രി­യും ടോ­റോ റോ­സോ­യില്‍ 
തു­ട­രാന്‍ സാ­ധ്യ­ത­യു­ണ്ടെ­ന്നാ­ണ­റി­യു­ന്ന­തു്.

­ബ്രി­ഡ്ജ്സ്റ്റോണ്‍ പിന്‍­വാ­ങ്ങി­യ­തി­നെ­ത്തു­ടര്‍­ന്നു് റേ­സി­ലെ ഏക ടയര്‍ സപ്ല­യ­റാ­യി പി­റേ­ലി വീ­ണ്ടും രം­ഗ­ത്തു­വ­രും. 20 
കൊ­ല്ല­ത്തി­നു ശേ­ഷ­മാ­ണു് പി­റേ­ലി ഫോര്‍­മുല വണ്ണില്‍ തി­രി­ച്ചെ­ത്തു­ന്ന­തു്. വരു­ന്ന മൂ­ന്നു വര്‍­ഷ­ത്തേ­ക്കാ­ണു കരാര്‍. ഓരോ
റേ­സ് വാ­രാ­ന്ത്യ­ത്തി­നും അനു­വ­ദി­ച്ചി­രു­ന്ന ടയര്‍ സെ­റ്റു­ക­ളു­ടെ എണ്ണം പതി­നാ­ലില്‍ നി­ന്നും പതി­നൊ­ന്നാ­യി­ക്കു­റ­ച്ചി­ട്ടു­ണ്ടു്. 
അതു­പോ­ലെ ടയര്‍ റി­ട്ടേണ്‍ പോ­ളി­സി­യി­ലും മാ­റ്റ­ങ്ങ­ളു­ണ്ടു്. അതു­പോ­ലെ ഡ്രൈ­റേ­സില്‍ ഓപ്ഷ­നും പ്രൈ­മും 
ഉപ­യോ­ഗി­ച്ചി­ല്ലെ­ങ്കില്‍ മു­പ്പ­തു സെ­ക്ക­ന്റ് പെ­നാല്‍­ട്ടി­യു­ണ്ടു്. ഗി­യര്‍­ബോ­ക്സു­കള്‍ നാ­ലി­നു പക­രം അഞ്ചു റേ­സു­ക­ളില്‍ 
ഉപ­യോ­ഗി­ക്ക­ണ­മെ­ന്നും നി­യ­മ­മു­ണ്ടു്.

­മ­റി­ക­ട­ക്ക­ലും ആവേ­ശ­വും വര്‍­ദ്ധി­പ്പി­ക്കാ­നാ­യി, KERS തി­രി­ച്ചു­വ­രു­മ്പോള്‍, ഡ്രൈ­വര്‍­ക്കു അഡ്ജ­സ്റ്റു ചെ­യ്യാ­വു­ന്ന പിന്‍ 
ചി­റ­കു­കള്‍ പു­തു­താ­യി വരു­ന്നു­ണ്ടു്. എന്നാല്‍ ഡി­ഫ്യൂ­സ­റു­ക­ളും ഏറെ വി­വാ­ദ­മായ എഫ്-ഡക്റ്റും നി­രോ­ധി­ച്ചി­ട്ടു­മു­ണ്ടു്. 
ടയ­റു­ക­ളു­ടെ സു­ര­ക്ഷ വര്‍­ദ്ധി­പ്പി­ക്കാ­നാ­യി ഒരു പി­ടു­ത്തം (tether) അധി­കം വയ്ക്കാ­നും നി­യ­മ­മു­ണ്ടാ­ക്കി­യി­ട്ടു­ണ്ടു്.

എ­ന്നാല്‍ 107% നി­യ­മ­മാ­യി­രി­ക്കും 2011ല്‍ ഏറ്റ­വും വി­ഷ­യ­മാ­വു­ക. യോ­ഗ്യ­താ റൌ­ണ്ടി­ന്റെ ആദ്യ­പാ­ദ­ത്തി­ലെ 
ഏറ്റ­വും വേ­ഗ­മേ­റിയ സമ­യ­ത്തെ­ക്കാള്‍ 107% പി­റ­കി­ലു­ള്ള­വ­രെ റേ­സില്‍ പങ്കെ­ടു­ക്കാന്‍ അനു­വ­ദി­ക്കി­ല്ലെ­ന്ന­താ­ണി­തു്. എന്നാല്‍ ചില പ്ര­ത്യേക സാ­ഹ­ച­ര്യ­ങ്ങ­ളില്‍ (പ­രി­ശീ­ലന റൌ­ണ്ടു­ക­ളില്‍ മി­ക­ച്ച സമ­യം രേ­ഖ­പ്പെ­ടു­ത്തി­യി­ട്ടു­ണ്ടെ­ങ്കി­ലും മറ്റും) 
സ്റ്റ്യു­വാര്‍­ഡു­കള്‍­ക്കു് അനു­വാ­ദം നല്‍­കാം. സ്റ്റ്യൂ­വാര്‍­ഡു­കള്‍­ക്കു കൂ­ടു­തല്‍ അധി­കാ­രം നല്‍­കി­യ­പ്പോള്‍ ടീം ഓര്‍­ഡ­റു­ക­ളു­ടെ 
മേ­ലു­ണ്ടാ­യി­രു­ന്ന നി­രോ­ധ­നം പിന്‍­വ­ലി­ച്ചി­ട്ടു­മു­ണ്ടു്. ടീം കര്‍­ഫ്യൂ ആണു മറ്റൊ­രു പു­തിയ നി­യ­മം. രാ­ത്രി പന്ത്ര­ണ്ടു മു­തല്‍ 
രാ­വി­ലെ ആറു­വ­രെ­യോ (സെ­ഷന്‍ പത്തു­മ­ണി­ക്കാ­ണെ­ങ്കില്‍), രാ­ത്രി ഒന്നു മു­തല്‍ ഏഴു­വ­രേ­യോ (സെ­ഷന്‍ പതി­നൊ­ന്നു 
മണി­ക്കാ­ണെ­ങ്കില്‍) കാ­റി­ന്റെ വര്‍­ക്കു­മാ­യി ബന്ധ­പ്പെ­ട്ട ആരും റേ­സ് സര്‍­ക്യൂ­ട്ടില്‍ പ്ര­വേ­ശി­ക്ക­രു­തെ­ന്നാ­ണു് പു­തിയ നി­യ­മം­.

ഇ­ന്ത്യ­യി­ലേ­ക്കു് ആദ്യ­മാ­യി ഫോര്‍­മുല വണ്‍ എത്തു­മ്പോള്‍ നരേന്‍ കാര്‍­ത്തി­കേ­യ­നും ഫോ­ഴ്സ് ഇന്ത്യ­യും ഇന്ത്യന്‍ 
സാ­ന്നി­ധ്യ­ങ്ങ­ളാ­യി ട്രാ­ക്കി­ലു­ണ്ടാ­കു­മെ­ന്ന് ഏതാ­ണ്ടു­റ­പ്പാ­യി­ക്ക­ഴി­ഞ്ഞു. ഫെ­ബ്രു­വ­രി ഒന്നി­നു വലന്‍­സി­യ­യില്‍ ടെ­സ്റ്റി­ങ്ങി­നു 
തു­ട­ക്ക­മാ­വു­ന്ന­തോ­ടു­കൂ­ടി, ഔദ്യോ­ഗി­ക­മാ­യി 2011 സീ­സ­ണി­നു തു­ട­ക്ക­മാ­വും. ഫോര്‍­മു­ല­വണ്‍ ലോ­ക­വും 
സജീ­വ­മാ­യി­ക്കൊ­ണ്ടി­രി­ക്കു­ക­യാ­ണു്. കഴി­ഞ്ഞ നാ­ലു വര്‍­ഷ­ങ്ങ­ളില്‍ നാ­ലു വ്യ­ത്യ­സ്ത ചാ­മ്പ്യന്‍­മാ­രെ­യാ­ണു് നമു­ക്ക് 
സീ­സണ്‍ സമ്മാ­നി­ച്ച­തു്. മാ­ത്ര­മ­ല്ല ട്രാ­ക്കില്‍ തീ­പാ­റു­ന്ന പോ­രാ­ട്ട­ങ്ങ­ളും അവ­രൊ­രു­ക്കി. ആരാ­യി­രി­ക്കും 2011­ന്റെ ചാ­മ്പ്യന്‍? 
മു­പ്പ­തി­നു മേ­ലെ­യു­ള്ള ധാ­രാ­ളം പേര്‍ അണി­നി­ര­ക്കു­ന്ന 2011ല്‍ അവ­രാ­രെ­ങ്കി­ലും കി­രീ­ടം ചൂ­ടു­മോ? സാ­ക്കിര്‍ ട്രാ­ക്കില്‍ 
എന്‍­ജിന്‍ മു­ഴ­ങ്ങും വരെ നമു­ക്കു കാ­ത്തി­രി­ക്കാം­.

(13 January 2011)\footnote{http://malayal.am/വിനോദം/കായികം/9663/ഫോര്‍മുല-വണ്‍-ഇന്ത്യയിലെത്തുമ്പോള്‍}

\newpage
