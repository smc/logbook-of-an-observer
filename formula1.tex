\secstar{കാറോട്ടത്തിന്റെ മാസ്മരികത: ഫോര്‍മുല വണ്‍}
\vskip 2pt
ഈ ഞായറാഴ്ച കഴിഞ്ഞ ചൈനീസ് ഗ്രാന്‍പ്രീയോടെ ഫോര്‍മുല വണ്‍ ആദ്യ എഷ്യന്‍പാദം പൂര്‍ത്തിയാക്കിയിരിക്കുകയാണു്. 
സെപ്റ്റമ്പര്‍ 24-26നു് സിംഗപ്പൂരില്‍ നടക്കുന്ന ഗ്രാന്‍പ്രീയിലാണു് ഇനി എഫ് വണ്‍ എഷ്യയിലേക്കു് തിരിച്ചെത്തുന്നതു്.
കാനഡയില്‍ ജൂണ്‍ ആദ്യം നടക്കുന്ന ഗ്രാന്‍പ്രീ ഒഴിവാക്കിയാല്‍ ഫോര്‍മുല വണ്ണിന്റെ യൂറോപ്യന്‍ പാദമാണു് അടുത്ത 
നാലരമാസക്കാലമെന്നു് നിസ്സംശയം പറയാം.

നിലവിലെ ചാമ്പ്യനായ മക്‌ലാരന്റെ ജെന്‍സണ്‍ ബട്ടണ്‍ അറുപതു പോയിന്റുകളുമായി ഇക്കൊല്ലവും മുന്നിലാണു്. 
അന്‍പതു പോയിന്റുമായി മെഴ്സിഡസിന്റെ നികോ റൊസ്‌ബര്‍ഗ് ആണു് രണ്ടാമതു്. നാല്‍പ്പത്തിയൊന്‍പതു പോയിന്റുകളുമായി 
ഫെറാരിയുടെ ഫെര്‍ണാണ്ടോ അലോണ്‍സോയും മക്‌ലാരന്റെ ലൂയിസ് ഹാമില്‍ട്ടണും റൊസ്ബര്‍ഗിനൊപ്പത്തിനൊപ്പം 
നില്‍ക്കുന്നു. ഇക്കൊല്ലം മത്സരരംഗത്തേക്കു് തിരിച്ചെത്തിയ എഫ് വണ്‍ ഇതിഹാസം മൈക്കല്‍ ഷുമാക്കര്‍ പത്തു 
പോയിന്റുമായി പത്താമതാണു്. ഫോഴ്സ് ഇന്ത്യയുടെ അഡ്രിയാന്‍ സുടില്‍ ഒന്‍പതാമതും, വിറ്റാന്‍ടോണിയോ ലിയുസ്സി 
പതിനൊന്നാമതുമാണു്. ഇന്ത്യക്കാരന്‍ കരണ്‍ ചന്ദോക്ക് പോയിന്റൊന്നുമില്ലാതെ പത്തൊന്‍പതാമതാണു്. ടീമുകളുടെ 
കാര്യത്തില്‍ മക്‌ലാരന്‍ 109 പോയിന്റുകളുമായി മുന്നിട്ടുനില്‍ക്കുന്നു. തൊണ്ണൂറു പോയിന്റുമായി ഫെറാരി രണ്ടാമതും, 
എഴുപത്തിമൂന്നു പോയിന്റുമായി റെഡ്ബുള്‍ മൂന്നാമതുമാണു്. അറുപതു പോയിന്റുമായി മെഴ്സിഡസ് നാലാമതും. 
നാല്‍പ്പതുപോയിന്റ് നേടിയ റോബര്‍ട്ടു് കുബിത്സയുടെ ബലത്തില്‍ നാല്‍പ്പത്തിയാറു പോയിന്റുമായി റെനോ 
അഞ്ചാമതാണു്. പതിനെട്ടുപോയിന്റുമായി ഫോഴ്സ് ഇന്ത്യ ആറാമതും.

ഒന്നാംസ്ഥാനക്കാര്‍ക്കു് (പത്തിന്റെ സ്ഥാനത്തു് ഇരുപത്തഞ്ചു്) രണ്ടാംസ്ഥാനക്കാരേക്കാള്‍ (എട്ടിന്റെ 
സ്ഥാനത്തു് പതിനെട്ടു്) ഏഴു് പോയിന്റ് കൂടുതല്‍ നല്‍കുന്ന പുതിയ സംവിധാനം പോയിന്റ് നിലയില്‍ ആദ്യപാദത്തില്‍ത്തന്നെ 
കാണുന്ന വന്‍വ്യത്യാസത്തിനു കാരണമാണു്. കഴിഞ്ഞവര്‍ഷം യൂറോപ്യന്‍ പാദത്തില്‍ നല്ല പ്രകടനം കാഴ്ചവച്ച 
ഫോഴ്സ് ഇന്ത്യക്കു് വരുംആഴ്ചകളില്‍ ഇതു മുന്‍തൂക്കം കൊടുക്കുന്നു. മത്സരത്തിനിടയില്‍ ഇന്ധനം നിറക്കാനനുവദിക്കാത്ത 
പുതിയ നിയമവും യൂറോപ്പില്‍ രണ്ടു പോളുകള്‍ നേടിയ ഫോഴ്സ് ഇന്ത്യക്കു് അനുകൂലമാകും.

മാറിയ നിയമങ്ങള്‍ മത്സരത്തുടക്കങ്ങള്‍ ഒട്ടൊന്നു വിരസമാക്കിയെങ്കിലും അവസാനലാപ്പുകളില്‍ മധ്യനിരയില്‍ നല്ല 
പോരാട്ടങ്ങള്‍ക്കു് അതു് അവസരമൊരുക്കി. ഇന്ധന പരിപാലനത്തില്‍നിന്നും ടീമുകളുടെ ശ്രദ്ധ ടയര്‍ 
പരിപാലനത്തിലേക്കുമാറിയതോടെ, വണ്‍ സ്റ്റോപ്പ് സ്ട്രാറ്റജി സര്‍‌വ്വസാധാരണമായി. അതുകൊണ്ടുതന്നെ, കൃത്യമായി 
കാലാവസ്ഥ പ്രവചിക്കാന്‍ കഴിയുന്നവര്‍ക്കു് മഴയില്‍ കുതിര്‍ന്ന റേസുകളില്‍ വലിയമുന്‍തൂക്കം ലഭിക്കും. തന്റെ നല്ലകാലത്തു് 
ഇതില്‍ മിടുക്കനായിരുന്നു ഷുമി.

ഇനിയും പതിനഞ്ചു് റേസുകള്‍ ശേഷിക്കുകയും, 1515 പോയിന്റുകള്‍ പങ്കുവയ്ക്കപ്പെടാന്‍ കാത്തിരിക്കുകയും ചെയ്യുന്നതുകൊണ്ടു് 
ഇതിപ്പോഴും ആരുടെയും കയ്യിലൊതുങ്ങിയിട്ടില്ല. മാത്രവുമല്ല, മുന്‍നിര ടീമുകളെല്ലാം നല്ല പ്രകടനം കാഴ്ചവച്ചിട്ടുള്ളതിനാല്‍ 
യൂറോപ്പിലെ റേസുകളില്‍ വരുംആഴ്ചകളില്‍ തീ പാറുമെന്നുറുപ്പു്. 

\begin{flushright}(20 April, 2010)\footnote{http://malayal.am/വിനോദം/കായികം/4845/കാറോട്ടത്തിന്റെ-മാസ്മരികത-ഫോര്‍മുല-വണ്‍}\end{flushright}

\newpage
