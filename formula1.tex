\secstar{കാറോട്ടത്തിന്റെ മാസ്മരികത: ഫോര്‍മുല വണ്‍}
\vskip 2pt

ഈ ഞാ­യ­റാ­ഴ്ച കഴി­ഞ്ഞ ചൈ­നീ­സ് ഗ്രാന്‍­പ്രീ­യോ­ടെ ­ഫോര്‍­മുല വണ്‍ ആദ്യ എഷ്യന്‍ പാ­ദം പൂര്‍­ത്തി­യാ­ക്കി­യി­രി­ക്കു­ക­യാ­ണ്. 
സെ­പ്റ്റ­മ്പര്‍ 24-26­ന് സിം­ഗ­പ്പൂ­രില്‍ നട­ക്കു­ന്ന ഗ്രാന്‍­പ്രീ­യി­ലാ­ണ് ഇനി എഫ് വണ്‍ എഷ്യ­യി­ലേ­ക്ക് തി­രി­ച്ചെ­ത്തു­ന്ന­ത്. 
കാ­ന­ഡ­യില്‍ ജൂണ്‍ ആദ്യം നട­ക്കു­ന്ന ­ഗ്രാന്‍­പ്രീ­ ഒഴി­വാ­ക്കി­യാല്‍ ഫോര്‍­മുല വണ്ണി­ന്റെ യൂ­റോ­പ്യന്‍ പാ­ദ­മാ­ണ് അടു­ത്ത 
നാ­ലര മാ­സ­ക്കാ­ല­മെ­ന്ന് നി­സ്സം­ശ­യം പറ­യാം­.

­നി­ല­വി­ലെ ചാ­മ്പ്യ­നായ മക്‌­ലാ­ര­ന്റെ ജെന്‍­സണ്‍ ബട്ടണ്‍ അറു­പ­തു പോ­യി­ന്റു­ക­ളു­മാ­യി ഇക്കൊ­ല്ല­വും മു­ന്നി­ലാ­ണ്. 
അന്‍­പ­തു പോ­യി­ന്റു­മാ­യി മെ­ഴ്സി­ഡ­സി­ന്റെ നി­കോ റൊ­സ്‌­ബര്‍­ഗ് ആണ് രണ്ടാ­മ­ത്. നാല്‍­പ്പ­ത്തി­യൊന്‍­പ­തു പോ­യി­ന്റു­ക­ളു­മാ­യി 
ഫെ­റാ­രി­യു­ടെ ഫെര്‍­ണാ­ണ്ടോ അലോണ്‍­സോ­യും മക്‌­ലാ­ര­ന്റെ ലൂ­യി­സ് ഹാ­മില്‍­ട്ട­ണും റൊ­സ്ബര്‍­ഗി­നൊ­പ്പ­ത്തി­നൊ­പ്പം 
നില്‍­ക്കു­ന്നു. ഇക്കൊ­ല്ലം മത്സ­ര­രം­ഗ­ത്തേ­ക്ക് തി­രി­ച്ചെ­ത്തിയ എഫ് വണ്‍ ഇതി­ഹാ­സം ­മൈ­ക്കല്‍ ഷു­മാ­ക്കര്‍ പത്തു 
പോ­യി­ന്റു­മാ­യി പത്താ­മ­താ­ണ്. ഫോ­ഴ്സ് ഇന്ത്യ­യു­ടെ അഡ്രി­യാന്‍ സു­ടില്‍ ഒന്‍­പ­താ­മ­തും, ­വി­റ്റാന്‍­ടോ­ണി­യോ ലി­യു­സ്സി­ 
പതി­നൊ­ന്നാ­മ­തു­മാ­ണ്. ഇന്ത്യ­ക്കാ­രന്‍ കരണ്‍ ചന്ദോ­ക്ക് പോ­യി­ന്റൊ­ന്നു­മി­ല്ലാ­തെ പത്തൊന്‍­പ­താ­മ­താ­ണ്. ടീ­മു­ക­ളു­ടെ 
കാ­ര്യ­ത്തില്‍ മക്‌­ലാ­രന്‍ 109 പോ­യി­ന്റു­ക­ളു­മാ­യി മു­ന്നി­ട്ടു നില്‍­ക്കു­ന്നു. തൊ­ണ്ണൂ­റു പോ­യി­ന്റു­മാ­യി ­ഫെ­റാ­രി­ രണ്ടാ­മ­തും, 
എഴു­പ­ത്തി­മൂ­ന്നു പോ­യി­ന്റു­മാ­യി ­റെ­ഡ്ബുള്‍ മൂ­ന്നാ­മ­തു­മാ­ണ്. അറു­പ­തു പോ­യി­ന്റു­മാ­യി മെ­ഴ്സി­ഡ­സ് നാ­ലാ­മ­തും, 
നാല്‍­പ്പ­തു­പോ­യി­ന്റ് നേ­ടിയ റോ­ബര്‍­ട്ട് കു­ബി­ത്സ­യു­ടെ ബല­ത്തില്‍ നാല്‍­പ്പ­ത്തി­യാ­റു പോ­യി­ന്റു­മാ­യി റെ­നോ 
അഞ്ചാ­മ­തു­മാ­ണ്. പതി­നെ­ട്ടു­പോ­യി­ന്റു­മാ­യി ഫോ­ഴ്സ് ഇന്ത്യ ആറാ­മ­താ­ണ്.

ഒ­ന്നാം സ്ഥാ­ന­ക്കാര്‍­ക്ക് (­പ­ത്തി­ന്റെ സ്ഥാ­ന­ത്ത് ഇരു­പ­ത്ത­ഞ്ച്) ഏഴു പോ­യി­ന്റ് രണ്ടാം സ്ഥാ­ന­ക്കാ­രേ­ക്കാള്‍ (എ­ട്ടി­ന്റെ 
സ്ഥാ­ന­ത്ത് പതി­നെ­ട്ട്) ­കൂ­ടു­തല്‍ നല്‍­കു­ന്ന പു­തിയ പോ­യി­ന്റ് സം­വി­ധാ­ന­വും പോ­യി­ന്റ് നി­ല­യില്‍ ആദ്യ­പാ­ദ­ത്തില്‍­ത്ത­ന്നെ 
കാ­ണു­ന്ന വന്‍ വ്യ­ത്യാ­സ­ത്തി­നു കാ­ര­ണ­മാ­ണ്. കഴി­ഞ്ഞ വര്‍­ഷം യൂ­റോ­പ്യന്‍ പാ­ദ­ത്തില്‍ നല്ല പ്ര­ക­ട­നം കാ­ഴ്ച വെ­ച്ച 
ഫോ­ഴ്സ് ഇന്ത്യ­ക്ക് വരും ആഴ്ച­ക­ളില്‍ ഇതു മുന്‍­തൂ­ക്കം കൊ­ടു­ക്കു­ന്നു. മത്സ­ര­ത്തി­നി­ട­യില്‍ ഇന്ധ­നം നി­റ­ക്കാ­ന­നു­വ­ദി­ക്കാ­ത്ത 
പു­തിയ നി­യ­മ­വും യൂ­റോ­പ്പില്‍ രണ്ടു പോ­ളു­കള്‍ നേ­ടിയ ഫോ­ഴ്സ് ഇന്ത്യ­ക്ക് അനു­കൂ­ല­മാ­കും.

­മാ­റിയ നി­യ­മ­ങ്ങള്‍ മത്സ­ര­തു­ട­ക്ക­ങ്ങള്‍ ഒട്ടൊ­ന്നു വി­ര­സ­മാ­ക്കി­യെ­ങ്കി­ലും അവ­സാ­ന­ലാ­പ്പു­ക­ളില്‍ മധ്യ­നി­ര­യില്‍ നല്ല 
പോ­രാ­ട്ട­ങ്ങള്‍­ക്കും അത് അവ­സ­ര­മൊ­രു­ക്കി. ഇന്ധന പരി­പാ­ല­ന­ത്തില്‍­നി­ന്നും ടീ­മു­ക­ളു­ടെ ശ്ര­ദ്ധ ടയര്‍ 
പരി­പാ­ല­ന­ത്തി­ലേ­ക്കു­മാ­റി­യ­തോ­ടെ, വണ്‍ സ്റ്റോ­പ്പ് സ്ട്രാ­റ്റ­ജി സര്‍‌­വ്വ സാ­ധാ­ര­ണ­മാ­യി. അതു­കൊ­ണ്ടു­ത­ന്നെ, കൃ­ത്യ­മാ­യി 
കാ­ലാ­വ­സ്ഥ പ്ര­വ­ചി­ക്കാന്‍ കഴി­യു­ന്ന­വര്‍­ക്ക് മഴ­യില്‍ കു­തിര്‍­ന്ന റേ­സു­ക­ളില്‍ വലി­യ­മുന്‍­തൂ­ക്കം ലഭി­ക്കും. തന്റെ നല്ല­കാ­ല­ത്ത് 
ഇതില്‍ മി­ടു­ക്ക­നാ­യി­രു­ന്നു ഷു­മി.

ഇ­നി­യും പതി­ന­ഞ്ച് റേ­സു­കള്‍ ശേ­ഷി­ക്കു­ക­യും, 1515 പോ­യി­ന്റു­കള്‍ പങ്കു­വ­യ്ക്ക­പ്പെ­ടാന്‍ കാ­ത്തി­രി­ക്കു­ക­യും ചെ­യ്യു­ന്ന­തു­കൊ­ണ്ട് 
ഇതി­പ്പോ­ഴും ആരു­ടെ­യും കയ്യി­ലൊ­തു­ങ്ങി­യി­ട്ടി­ല്ല. മാ­ത്ര­വു­മ­ല്ല, മുന്‍­നിര ടീ­മു­ക­ളെ­ല്ലാം നല്ല പ്ര­ക­ട­നം കാ­ഴ്ച വച്ചി­ട്ടു­ള്ള­തി­നാല്‍ 
യൂ­റോ­പ്പി­ലെ റേ­സു­ക­ളില്‍ വരും ആഴ്ച­ക­ളില്‍ തീ പാ­റു­മെ­ന്നു­റു­പ്പ്. 

(20 April 2010)\footnote{\url{http://malayal.am/വിനോദം/കായികം/4845/കാറോട്ടത്തിന്റെ-മാസ്മരികത-ഫോര്‍മുല-വണ്‍}}

\newpage
