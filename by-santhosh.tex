\newpage
\secstar{ജിനേഷ്}
 
ജിനേഷിനെ ഞാന്‍ ആദ്യമായി കാണുന്നതും പരിചയപ്പെടുന്നതും 2006ല്‍ ആണു്. പാലക്കാടു് കോട്ടമൈതനാത്തു് പാലക്കാട്ടെ സ്വതന്ത്ര സോഫ്റ്റ്‌‌വെയര്‍ പ്രവര്‍ത്തകരുടെ ഒരു യോഗത്തില്‍ വെച്ചായിരുന്നു. അന്നു് ജിനേഷ് വിദ്യാര്‍ത്ഥിയാണു്. പിന്നീടു് ഞങ്ങള്‍ പലതവണ നേരില്‍ കണ്ടിട്ടുണ്ടു്- സ്വതന്ത്ര സോഫ്റ്റ്‌വെയര്‍ സംബന്ധിച്ച പരിപാടികളായിരുന്നു മിക്കതും. പക്ഷേ നേരില്‍ സംസാരിച്ചിട്ടുള്ളതിനേക്കാള്‍ ഞങ്ങള്‍ ഇന്റര്‍നെറ്റിലൂടെയായിരുന്നു സംസാരിച്ചിരുന്നതു്. പലപ്പോഴും മണിക്കൂറുകളോളം നീളുമായിരുന്ന സംഭാഷണങ്ങളില്‍ രാഷ്ട്രീയം, സാങ്കേതികവിദ്യ, മലയാള ഭാഷ, അക്കാദമിക് ഗവേഷണങ്ങള്‍, പുസ്തകങ്ങള്‍ എന്നിവയാണു് നിറഞ്ഞു നിന്നതു്. തുടക്കം മുതലേ ജിനേഷിന്റെ ഓരോ വിഷയങ്ങളിലുമുള്ള വിശകലന ശൈലി എന്നെ വല്ലാതെ ആകര്‍ഷിച്ചിരുന്നു. സംസാരിക്കുന്ന വിഷയങ്ങളില്‍ ആഴത്തിലുള്ള അറിവു നേടാന്‍ വല്ലാത്ത ഉത്സാഹമായിരുന്നു ജിനേഷിനു്.
 
സ്വതന്ത്ര മലയാളം കമ്പ്യൂട്ടിങ്ങ് ഗൂഗിള്‍ സമ്മര്‍ ഓഫ് കോഡ് പരിപാടിയില്‍ പങ്കെടുക്കാന്‍ തിരഞ്ഞെടുക്കപ്പെടുന്നതു് 2007ല്‍ ആണു്. അതില്‍ ജിനേഷും പങ്കെടുത്തു, ലളിത എന്ന പേരിലുള്ള ഒരു കീബോഡ് ലേയൌട്ട്  മലയാളത്തിനു വികസിപ്പിച്ചെടുക്കുകയും  ഗ്നു/ലിനക്സ് സിസ്റ്റങ്ങളില്‍ അതു് ലഭ്യമാക്കുകയും ചെയ്തിരുന്നു. പക്ഷേ അതിനെക്കാള്‍ ഞാന്‍ പ്രാധാന്യത്തോടെ കണ്ടതു് അന്നു് ഉണ്ടായിരുന്ന ഇന്‍സ്ക്രിപ്റ്റ് കീബോഡിലെ പിഴവുകള്‍ തിരുത്തി സ്റ്റാന്‍ഡേര്‍ഡനുസൃതമാക്കിയതായിരുന്നു. ആ കീബോഡ് ലേയൌട്ടാണു്  ഇപ്പോള്‍ ഗ്നു/ലിനക്സ് സിസ്റ്റങ്ങളിലെല്ലാം ഉള്ളതു്. ഇതു്  ഉപയോഗിക്കാനായി പിന്നീടു് കേരളസര്‍ക്കാറിന്റെ മലയാളം കമ്പ്യൂട്ടിങ്ങ് പദ്ധതിയുടെ ഭാഗമായി പലര്‍ക്കും ടൈപ്പിങ്ങ് പരിശീലനം നല്‍കിയിരുന്നു. ഈ ലേയൌട്ട് തന്നെയാണു് ഐടി @ സ്കൂള്‍ പദ്ധതിയുടെ ഭാഗമായി സ്കൂളുകളില്‍ പഠിപ്പിയ്ക്കുന്നതും. 2010 ഓടെ സീഡാക്‍ ഈ ലേയൌട്ടില്‍ വീണ്ടും പരിഷ്കാരങ്ങള്‍ കൊണ്ടുവരാന്‍ ആരംഭിച്ചു. സീഡാകിന്റെ ഇന്‍സ്ക്രിപ്റ്റ് ലേയൌട്ട് ഒരു പാടു പിശകുള്ളതായിരുന്നു. ജിനേഷ് ഈ ലേയൌട്ട് പഠിച്ചു് വിശദമായ ഒരു വിശകലനക്കുറിപ്പു് എഴുതിയിരുന്നു. വര്‍ഷങ്ങള്‍ കഴിഞ്ഞെങ്കിലും, പല തവണ ചൂണ്ടിക്കാണിച്ചിട്ടും സീഡാക്‍ ഇതു് അവഗണിക്കുകയാണുണ്ടായതു്. 2012 ലും ഈ ലേയൌട്ടു് സീഡാക്‍ പുറത്തിറക്കിയതായി അറിയാന്‍ കഴിഞ്ഞിട്ടില്ല. ഏകദേശം ഈ കാലത്തു തന്നെയാണു് വെബ് വിലാസങ്ങള്‍ മലയാളത്തില്‍ ലഭ്യമാക്കാനുള്ള രൂപരേഖ സീഡാക്‍ തയ്യാറാക്കിയതു്. ഒരു പാടു അബദ്ധങ്ങള്‍ നിറഞ്ഞ ഈ രൂപരേഖയ്ക്കു് ഞങ്ങള്‍ തയ്യാറാക്കിയ പഠനത്തില്‍ ജിനേഷ് ഒരു പാടു സംഭാവനകള്‍ നല്‍കിയിരുന്നു. ഈ സ്റ്റാന്‍ഡേഡിലും സീഡാക്‍ 2012 ലും ഒരു തീരുമാനവുമെടുത്തിട്ടില്ല.
 
ഹൈദരാബാദില്‍ IIIT യില്‍ ഒപ്റ്റിക്കല്‍ കാരക്ടര്‍ റെക്കഗ്നീഷന്‍ പ്രൊജക്ടില്‍ ജിനേഷ് ഗവേഷണം ചെയ്തിരുന്നു. ഇന്ത്യന്‍ അക്കാദമിക് ഗവേഷണ രംഗത്തു നടക്കുന്ന സംരംഭങ്ങളുടെ മെല്ലെപ്പോക്കും ഉപയോഗ ശൂന്യതയും സുതാര്യതയില്ലായ്മയും ഞങ്ങള്‍ പല തവണ ചര്‍ച്ച ചെയ്തിരുന്നു. ഈ സംരംഭങ്ങളുടെ പുറത്തു് ഒപ്റ്റിക്കല്‍ റെക്കഗ്നീഷനു വേണ്ടി ദേബയാന്‍ ബാനര്‍ജി തുടങ്ങിയ ഇന്‍ഡിക് ടെസ്സറാക്ട് പ്രൊജക്ടില്‍ ജിനേഷ് സഹകരിച്ചിരുന്നു. ദേവനാഗരി, ബംഗാളി ലിപികളില്‍ 90\% ത്തോളം കൃത്യത വന്നെങ്കിലും മലയാളത്തെ സംബന്ധിച്ചിടത്തോളം ആ പ്രൊജക്ട് വിജയകരമായിരുന്നില്ല.
 
അസുഖബാധിതനായി ആശുപത്രിയില്‍ ആയ സമയത്താണു് ജിനേഷ് ഞാന്‍ തുടങ്ങിവെച്ച ശില്പ (സ്വതന്ത്ര സോഫ്റ്റ്‌വെയര്‍ അധിഷ്ഠിതമായ ഭാരതീയ ഭാഷകള്‍ക്കാവശ്യമായ സാങ്കേതിക വിദ്യ വികസിപ്പിച്ചെടുക്കാനുള്ള ഒരു സംരംഭം) പ്രൊജക്ടിലേക്കു് ഒരു പാടു കോഡെഴുതുന്നതു്. അതിനു മുന്നേ ജിനേഷ് പല ക്രിയാത്മക നിര്‍ദ്ദേശങ്ങളും തന്നിരുന്നെങ്കിലും പ്രൊജക്ടില്‍ പങ്കെടുത്തിരുന്നില്ല. ഈ സംരംഭം ഇപ്പോഴും മെല്ലെയാണെങ്കിലും മുന്നോട്ടുപോകുന്നു. പ്രധാനമായും സങ്കീര്‍ണ്ണലിപികള്‍ പിഡിഎഫില്‍ ചിത്രീകരിക്കാനുള്ള സംവിധാനത്തിലാണു് ജിനേഷ് കുറേ സഹായിച്ചിരുന്നതു്.
 
ഒരു വ്യക്തിയെ സമൂഹം ഓര്‍ക്കുന്നതു് ജീവിച്ചിരുന്ന കാലത്തെ നിസ്വാര്‍ത്ഥ പ്രവര്‍ത്തനങ്ങളുടെ പേരിലാണു്. ജിനേഷിനെ സംബന്ധിച്ചിടത്തോളം വളരെ ചെറിയ കാലയളവില്‍ ചെയ്ത പ്രവര്‍ത്തനങ്ങളുടെ പേരില്‍ എന്നെന്നും ഓര്‍ക്കപ്പെടും.

സന്തോഷ് തോട്ടിങ്ങല്‍, സ്വതന്ത്ര മലയാളം കമ്പ്യൂട്ടിങ്ങ്
\newpage
