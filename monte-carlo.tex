\secstar{അപകടങ്ങളുടെ മോണ്ടേകാര്‍ലോ}
\vskip 2pt

പ്രതീക്ഷിച്ചതുപോലെ ആവേശകരമായ ഗ്രാന്‍പ്രീയായിരുന്നു മോണ്ടേ കാര്‍ലോയിലേതു് (മേയ് പതിനാറു്). മാര്‍ക്ക് 
വെബ്ബര്‍ തുടര്‍ച്ചയായി രണ്ടാമത്തെ വിജയം നേടി. റെഡ്ബുള്ളിനുവേണ്ടി സെബാസ്റ്റ്യന്‍ വെറ്റല്‍ രണ്ടാമതെത്തിയപ്പോള്‍ 
റെനോയുടെ റോബര്‍ട്ട് കുബിത്സ മൂന്നാമതെത്തി. ഇരുപത്തിനാലുപേരുമായിത്തുടങ്ങിയ മൊണാകൊ ഗ്രാന്‍പ്രീ 
അവസാനിച്ചപ്പോള്‍ റേസ് പൂര്‍ത്തിയാക്കിയതു് പന്ത്രണ്ടുപേരാണു്.

ആദ്യലാപ്പില്‍ വില്യംസിന്റെ നികോ ഹല്‍ക്കന്‍ബര്‍ഗ് തുടങ്ങിവച്ച ആക്സിഡന്റുകളുടെ പരമ്പര, അവസാനലാപ്പുകളില്‍ 
ലോട്ടസിന്റെ യാനോ ട്രൂലിയാണു് പൂര്‍ത്തിയാക്കിയതു്. സേഫ്റ്റികാറിന്റെ പിന്നില്‍ റേസ് പൂര്‍ത്തിയാക്കിയെങ്കിലും, 
സേഫ്റ്റികാര്‍ പിന്‍വലിഞ്ഞു് റേസ് പൂര്‍ത്തിയാകുന്നിടംവരെയുള്ള ചെറിയദൂരത്തില്‍ ഫെറാരിയുടെ ഫെര്‍ണാണ്ടോ 
അലോണ്‍സോ വരുത്തിയ പിഴവു് മുതലെടുത്തു് ആറാംസ്ഥാനം പിടിച്ചെടുത്ത മെഴ്സിഡസിന്റെ മൈക്കല്‍ ഷൂമാക്കര്‍,
തനിക്കു കിട്ടുന്ന സൂചിപ്പഴുതുപോലും ധാരാളമാണെന്നു് ഒരിക്കല്‍കൂടി യുവഎതിരാളികളെ ഓര്‍മ്മിപ്പിച്ചു. പിന്നീടു് ഷൂമാക്കറുടെ 
മറികടക്കല്‍ നിയമവിരുദ്ധമാണെന്നു വിധിയെഴുതിയ സ്റ്റ്യുവര്‍ട്ടുമാര്‍ 20 സെക്കന്റ് ഡ്രൈവ് ത്രൂ പെനാല്‍ട്ടി നല്‍കി.

മൊത്തം സീസണിലെ ഏറ്റവും വിഷമംപിടിച്ച റേസെന്ന വിശേഷണമുള്ള മോണ്ടേ കാര്‍ലോ ഗ്രാന്‍പ്രീ സ്വന്തംപേരു് 
നിലനിര്‍ത്തിയെന്നു് വേണമെങ്കില്‍ പറയാം. ആകെ 78 ലാപ്പുകളുള്ള റേസില്‍ നാലുതവണയാണു് സേഫ്റ്റികാര്‍ 
വിന്യസിക്കപ്പെട്ടതു്. രണ്ടുതവണയും വില്യംസിന്റെ റൂബന്‍ ബാരിക്കെല്ലോയും ഹള്‍ക്കന്‍ബര്‍ഗുമായിരുന്നു കാരണക്കാര്‍. 
മുന്‍ചിറകുകള്‍ (front wing) പ്രവര്‍ത്തനരഹിതമായതാണു് ഹള്‍ക്കന്‍ബര്‍ഗിനെ കുഴക്കിയതെങ്കില്‍, മുപ്പത്തിരണ്ടാം 
ലാപ്പില്‍ കാറിന്റെ പിന്‍ഭാഗമാണു് ബാരിക്കെല്ലോയെ ചതിച്ചതു്. മൂന്നാമതു് സേഫ്റ്റികാര്‍ വിന്യസിച്ചതു് മഴവെള്ളം 
ഒഴുകിപ്പോകാനുള്ള ചാലുകളുടെ മൂടി ഒരെണ്ണം തുറന്നുകിടന്നതുകൊണ്ടായിരുന്നു. നാലാംതവണ മെഴ്സിഡസ് 
SLS AMG ട്രാക്കു് നിയന്ത്രിക്കാനായി എത്തിയതു്, ലോട്ടസിന്റെ യാനോ ട്രൂലി ഹിസ്പാനിക് റേസിങ് ടീമിന്റെ ഇന്ത്യന്‍ 
ഡ്രൈവര്‍ കരണ്‍ ചന്ദോക്കിന്റെ മുകളില്‍കൂടി കയറിമറിഞ്ഞതിനാണു്.

ട്രാക്കിനുള്ളില്‍ക്കൂടി ഓവര്‍ടേക്ക് ചെയ്യാനുള്ള ശ്രമത്തിനിടയില്‍ ബാലന്‍സ് നഷ്ടപ്പെട്ട ട്രൂലിയുടെ കാര്‍ ചന്ദോക്കിന്റെ 
കാറിന്റെ മുകളില്‍കൂടി മറിയുകയായിരുന്നു. തക്കസമയത്തു് തലതാഴ്ത്താന്‍ ചന്ദോക്കിനു തോന്നിയിരുന്നില്ലെങ്കില്‍ 
അപകടവാര്‍ത്തയ്ക്കുപകരം മരണവാര്‍ത്ത നമുക്കു വായിക്കേണ്ടിവന്നേനെ. റേസിന്റെ എഴുപത്തിനാലാം 
ലാപ്പില്‍ നടന്ന ഈ അപകടം ഇനി അത്ഭുതങ്ങളൊന്നും ട്രാക്കില്‍ കാണില്ലെന്നു് ഏകദേശം ഉറപ്പാക്കി. പക്ഷേ, അപ്പോഴും 
വിട്ടുകൊടുക്കാന്‍ തയ്യാറാകാതിരുന്ന മൈക്കേല്‍ ഷൂമാക്കര്‍ അവസാനനിമിഷം എല്ലാവരെയും അമ്പരപ്പിച്ചു് 
അലോണ്‍സൊയെ മറികടന്നു.

ഈ വിജയത്തോടെ ടീം മേറ്റ് വെറ്റലിനൊപ്പം ചാമ്പ്യന്‍ഷിപ്പ് പോരാട്ടത്തില്‍ മുന്നിട്ടുനില്‍ക്കിന്ന വെബ്ബറിനു് 
വെല്ലുവിളിയുയര്‍ത്താന്‍ വെറ്റലിനു് ഒരിക്കലും സാധിച്ചില്ല. മാത്രവുമല്ല, കുബിത്സയയില്‍നിന്നു് ആവശ്യത്തിനു 
സമ്മര്‍ദ്ദത്തിലുമായിരുന്നു വെറ്റല്‍. ഒരു ചെറിയ പിഴവുപോലും വരുത്താതെ മൂന്നു സേഫ്റ്റികാറുകളില്‍നിന്നും രക്ഷപ്പെട്ട 
വെബ്ബര്‍ സ്പെയിനിലെ തന്റെ ഫോം തുടരുകയായിരുന്നുവെന്നു പറയണം. എന്നാല്‍ റേസിലെ താരം, പിറ്റ് ലേനില്‍നിന്നു് 
റേസ് ആരംഭിച്ചു് തുടക്കത്തില്‍ത്തന്നെ ആവേശകരമായ മറികടക്കലുകളിലൂടെയും, പിറ്റ് സ്റ്റോപ്പെടുക്കാതെ മുഴുവന്‍ റേസും 
ഒരു ടയറില്‍ തീര്‍ത്ത തന്ത്രത്തിലൂടെയും ആറാമത് ഫിനിഷ് ചെയ്ത ഫെറാരിയുടെ ഫെര്‍ണാണ്ടോ അലോണ്‍സോയാണു്. 
(ഏതാണ്ടു് മുഴുവന്‍ റേസും, ആദ്യ യെല്ലോഫ്ലാഗ് വന്നപ്പോള്‍ പിറ്റ് ചെയ്തു് സോഫ്റ്റ് ടയറുകള്‍ മാറ്റി ഓപ്ഷന്‍ ടയറുകള്‍ 
എടുക്കുകയാണു് അലോണ്‍സോ ചെയ്തത്). രണ്ടായിരത്തിയാറിലെ ഷൂമാക്കറുടെ മൊണാകൊ പ്രകടനത്തെ ഓര്‍മ്മിപ്പിച്ചു 
ഇത്. അന്നു്, അലോണ്‍സോയുടെ യോഗ്യതാലാപ്പു് അലങ്കോലമാക്കാന്‍ കാര്‍ ട്രാക്കിനു വിലങ്ങനെയിട്ട ഷൂമാക്കറിനെ, 
പിഴചുമത്തി പിറ്റ് ലേനിലെത്തിക്കുകയായിരുന്നു. ആവേശകരമായ ഒരു റേസിലൂടെ ഷുമാക്കര്‍ അന്നു് അഞ്ചാമത് ഫിനിഷ് 
ചെയ്തു.

നിലവിലെ ചാമ്പ്യന്‍ മക്‌ലാരന്റെ ജെന്‍സണ്‍ ബട്ടണു് മൊണാകൊയില്‍ കാര്യങ്ങള്‍ അത്ര ശുഭകരമല്ലായിരുന്നു. 
ചാമ്പ്യന്‍ഷിപ്പ് പട്ടികയില്‍ എഴുപതു പോയിന്റോടെ ഒന്നാമനായെത്തിയ ബട്ടണ്‍ മടങ്ങുമ്പോള്‍ നാലാമതാണു്. മൂന്നാം 
ലാപ്പില്‍ എന്‍ജിന്‍ തകരാറുമൂലം പുറത്തുപോകേണ്ടിവന്ന ബട്ടനെ, റെഡ്ബുള്ളിന്റെ ഡ്രൈവര്‍മാരും (78 പോയിന്റു വീതം),
ഫെര്‍ണാണ്ടോ അലോണ്‍സോയുമാണു് (75 പോയിന്റ്) മറികടന്നതു്. ഫെറാരിയുടെ ഫെലിപെ മസ്സ 61 പോയിന്റുമായും, 
മൊണാകൊയിലെ മൂന്നാംസ്ഥാനക്കാരന്‍ റെനോയുടെ റോബര്‍ട്ട് കുബിത്സയും, മക്‌ലാരന്റെ 2008ലെ ലോകചാമ്പ്യന്‍ 
ലൂയിസ് ഹാമില്‍ട്ടണ്‍ 59 പോയിന്റുമായും ബട്ടണു തൊട്ടുപിറകിലുണ്ടു്. ആദ്യ ഏട്ടുസ്ഥാനക്കാരെ വെറും 25 പോയിന്റ് 
വേര്‍തിരിക്കുന്ന പട്ടിക ഇപ്പോഴും ഇതൊരു തുറന്ന പോരാട്ടമാണെന്നു വ്യക്തമാക്കുന്നു. മെഴ്സിഡസിന്റെ നികൊ 
റോസ്ബര്‍ഗാണു് എട്ടാമതു്. ഒന്‍പതാമത്, മൈക്കല്‍ ഷൂമാക്കറും, പത്താമതു്, ഫോഴ്സ് ഇന്ത്യയുടെ അഡ്രിയാന്‍ സുടിലാണു്.

ടീമുകളുടെ കാര്യത്തില്‍ റെഡ്ബുള്‍ 156 പോയിന്റോടെ രണ്ടാംസ്ഥാനത്തുള്ള ഫെറാരിയേക്കാള്‍ 22 പോയിന്റിനു 
മുന്നിലാണു്. മൂന്നാംസ്ഥാനത്തു നില്‍ക്കുന്ന മക്‌ലാരനും ഫെറാരിയും തമ്മില്‍ വെറും 5 പോയിന്റ് വ്യത്യാസമേയുള്ളൂ. 
എന്നാല്‍ മെഴ്സിഡസ് 78 പോയിന്റുമായി ബഹുദൂരം പിന്നിലാണു്. അഞ്ചാംസ്ഥാനത്തുള്ള റെനോ കുബിത്സയുടെ 
ബലത്തില്‍ 65 പോയിന്റുമായി മെഴ്സിഡസിനു് വെല്ലുവിളിയുയര്‍ത്തുന്നു. വ്യക്തമായ മധ്യനിരപ്രകടനവുമായി ഫോഴ്സ് 
ഇന്ത്യ 30 പോയിന്റോടെ ആറാമതാണു്.

അടുത്തറേസിനു വലിയ കാലതാമസമില്ലാത്തതിനാല്‍ (മേയ് അവസാനവാരം തുര്‍ക്കിയില്‍) വലിയ മാറ്റങ്ങളൊന്നും 
പ്രതീക്ഷിക്കേണ്ടതില്ല. എങ്കിലും തന്റെ വെല്ലുവിളി മധ്യനിരയില്‍നിന്നു് മുന്‍നിരയിലേക്കെത്തിക്കാന്‍ മൈക്കേല്‍ ഷൂമാക്കര്‍ 
ആവേശപൂര്‍വ്വം ശ്രമിക്കുന്നതും, ആദ്യ ഏട്ടുസ്ഥാനത്തില്‍ ഓരോ റേസിലും മാറ്റങ്ങള്‍ പ്രതീക്ഷിക്കാമെന്നതും തുര്‍ക്കിയില്‍ 
ആവേശമുണര്‍ത്തും.

പിന്‍കുറിപ്പു്: ഏഴു റേസുകളുമായി 1950ലാണു് ആദ്യഫോര്‍മുലവണ്‍ ചാമ്പ്യന്‍ഷിപ്പു് തുടങ്ങുന്നത്. കൃത്യം പറഞ്ഞാല്‍ 
1950 മേയ് 13ന് ഇംഗ്ലണ്ടിലെ സില്‍വര്‍സ്റ്റോണില്‍. 2010 റേസ് കലണ്ടറില്‍ അന്നുണ്ടായിരുന്നതില്‍ നാലു 
ട്രാക്കുകളില്‍ ഇന്നും ചാമ്പ്യന്‍ഷിപ്പു് മത്സരങ്ങള്‍ നടക്കുന്നുണ്ടു്. മൊണ്ടേ കാര്‍ലോ, സില്‍വര്‍സ്റ്റോണ്‍, മോണ്‍സ 
(ഇറ്റാലിയന്‍), സ്പാ (ബെല്‍ജിയന്‍) എന്നിവയാണതു്. അറുപതു വര്‍ഷങ്ങള്‍ക്കുശേഷം ഏറെ മാറ്റങ്ങളുമായി എഫ് 
വണ്‍ പ്രയാണം തുടരുമ്പോള്‍, ഇന്നും യൂറോപ്യന്‍ ടീമുകളാണു് ചാമ്പ്യന്‍ഷിപ്പിനെ നിയന്ത്രിക്കുന്നതെന്നതു് മറ്റൊരു 
സത്യം.

\hspace*{2em}(18 May, 2010)\footnote{http://malayal.am/വിനോദം/കായികം/5513/അപകടങ്ങളുടെ-മോണ്ടേകാര്‍ലോ}

\newpage
