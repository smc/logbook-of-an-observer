\secstar{അപകടങ്ങളുടെ മോണ്ടേകാര്‍ലോ}
\vskip 2pt

­പ്ര­തീ­ക്ഷി­ച്ച­തു പോ­ലെ ആവേ­ശ­ക­ര­മായ ഗ്രാന്‍­പ്രീ­യാ­യി­രു­ന്നു മോണ്ടേ കാര്‍­ലോ­യി­ലേ­ത് (മേ­യ് പതി­നാ­റ്). മാര്‍­ക്ക് 
വെ­ബ്ബര്‍ തു­ടര്‍­ച്ച­യാ­യി രണ്ടാ­മ­ത്തെ വി­ജ­യം നേ­ടി. റെ­ഡ്ബു­ള്ളി­നു വേ­ണ്ടി സെ­ബാ­സ്റ്റ്യന്‍ വെ­റ്റല്‍ രണ്ടാ­മ­തെ­ത്തി­യ­പ്പോള്‍ 
റെ­നോ­യു­ടെ റോ­ബര്‍­ട്ട് കു­ബി­ത്സ മൂ­ന്നാ­മ­തെ­ത്തി. ഇരു­പ­ത്തി­നാ­ലു­പേ­രു­മാ­യി­ത്തു­ട­ങ്ങിയ മൊ­ണാ­കൊ ­ഗ്രാന്‍­പ്രീ­ 
അവ­സാ­നി­ച്ച­പ്പോള്‍ ­റേ­സ് പൂര്‍­ത്തി­യാ­ക്കി­യ­ത് പന്ത്ര­ണ്ടു പേ­രാ­ണ്.

ആ­ദ്യ­ലാ­പ്പില്‍ വി­ല്യം­സി­ന്റെ നി­കോ ഹല്‍­ക്കന്‍­ബര്‍­ഗ് തു­ട­ങ്ങി­വ­ച്ച ആക്സി­ഡ­ന്റു­ക­ളു­ടെ പര­മ്പ­ര, അവ­സാ­ന­ലാ­പ്പു­ക­ളില്‍ 
ലോ­ട്ട­സി­ന്റെ യാ­നോ ട്രൂ­ലി­യാ­ണ് പൂര്‍­ത്തി­യാ­ക്കി­യ­ത്. സേ­ഫ്റ്റി­കാ­റി­ന്റെ പി­ന്നില്‍ റേ­സ് പൂര്‍­ത്തി­യാ­ക്കി­യെ­ങ്കി­ലും, 
സേ­ഫ്റ്റി­കാര്‍ പിന്‍­വ­ലി­ഞ്ഞ് റേ­സ് പൂര്‍­ത്തി­യാ­കു­ന്നി­ടം വരെ­യു­ള്ള ചെ­റി­യ­ദൂ­ര­ത്തില്‍ ഫെ­റാ­രി­യു­ടെ ഫെര്‍­ണാ­ണ്ടോ 
അലോണ്‍­സോ വരു­ത്തിയ പി­ഴ­വു മു­ത­ലെ­ടു­ത്ത് ആറാം സ്ഥാ­നം പി­ടി­ച്ചെ­ടു­ത്ത മെ­ഴ്സി­ഡ­സി­ന്റെ മൈ­ക്കല്‍ ഷൂ­മാ­ക്കര്‍ 
സൂ­ചി­പ്പ­ഴു­തു ധാ­രാ­ള­മാ­ണു തനി­ക്കെ­ന്ന് ഒരി­ക്കല്‍­കൂ­ടി തന്റെ യുവ എതി­രാ­ളി­ക­ളെ ഓര്‍­മ്മി­പ്പി­ച്ചു (പി­ന്നീ­ട് ഷൂ­മാ­ക്ക­റു­ടെ 
മറി­ക­ട­ക്കല്‍ നി­യ­മ­വി­രു­ദ്ധ­മാ­ണെ­ന്നു വി­ധി­യെ­ഴു­തിയ സ്റ്റ്യു­വര്‍­ട്ടു­മാര്‍ 20 സെ­ക്ക­ന്റ് ഡ്രൈ­വ് ത്രൂ പെ­നാല്‍­ട്ടി നല്‍­കി­).

­മൊ­ത്തം സീ­സ­ണി­ലെ ഏറ്റ­വും വി­ഷ­മം പി­ടി­ച്ച റേ­സെ­ന്ന വി­ശേ­ഷ­ണ­മു­ള്ള ­മോ­ണ്ടേ കാര്‍­ലോ­ ഗ്രാന്‍­പ്രീ സ്വ­ന്തം പേ­രു 
നി­ല­നിര്‍­ത്തി­യെ­ന്നു വേ­ണ­മെ­ങ്കില്‍ പറ­യാം. ആകെ 78 ലാ­പ്പു­ക­ളു­ള്ള റേ­സില്‍ നാ­ലു­ത­വ­ണ­യാ­ണ് സേ­ഫ്റ്റി­കാര്‍ 
വി­ന്യ­സി­ക്ക­പ്പെ­ട്ട­ത്. രണ്ടു­ത­വ­ണ­യും വി­ല്യം­സി­ന്റെ റൂ­ബന്‍ ബാ­രി­ക്കെ­ല്ലോ­യും, ഹള്‍­ക്കന്‍­ബര്‍­ഗു­മാ­യി­രു­ന്നു കാ­ര­ണ­ക്കാര്‍. 
മുന്‍­ചി­റ­കു­കള്‍ (front wing) പ്ര­വര്‍­ത്ത­ന­ര­ഹി­ത­മാ­യ­താ­ണ് ഹള്‍­ക്കന്‍­ബര്‍­ഗി­നെ കു­ഴ­ക്കി­യ­തെ­ങ്കില്‍, മു­പ്പ­ത്തി­ര­ണ്ടാം 
­ലാ­പ്പില്‍ കാ­റി­ന്റെ പിന്‍­ഭാ­ഗ­മാ­ണ് ബാ­രി­ക്കെ­ല്ലോ­യെ ചതി­ച്ച­ത്. മൂ­ന്നാ­മ­ത് സേ­ഫ്റ്റി­കാര്‍ വി­ന്യ­സി­ച്ച­ത് മഴ­വെ­ള്ളം 
ഒഴു­കി­പ്പോ­കാ­നു­ള്ള ചാ­ലു­ക­ളു­ടെ മൂ­ടി­യൊ­രെ­ണ്ണം തു­റ­ന്നു കി­ട­ന്ന­തു­കൊ­ണ്ടാ­യി­രു­ന്നു. നാ­ലാം തവണ മെ­ഴ്സി­ഡ­സ് 
SLS AMG ട്രാ­ക്ക് നി­യ­ന്ത്രി­ക്കാ­നാ­യി എത്തി­യ­ത്, ലോ­ട്ട­സി­ന്റെ യാ­നോ ട്രൂ­ലി ഹി­സ്പാ­നി­ക് റേ­സി­ങ് ടീ­മി­ന്റെ ഇന്ത്യന്‍ 
ഡ്രൈ­വര്‍ കരണ്‍ ചന്ദോ­ക്കി­ന്റെ മു­ക­ളില്‍ കൂ­ടി കയ­റി മറി­ഞ്ഞ­തി­നാ­ണ്.

­ട്രാ­ക്കി­നു­ള്ളില്‍­ക്കൂ­ടി ഓവര്‍­ടേ­ക്ക് ചെ­യ്യാ­നു­ള്ള ശ്ര­മ­ത്തി­നി­ട­യില്‍ ബാ­ലന്‍­സ് നഷ്ട­പ്പെ­ട്ട ട്രൂ­ലി­യു­ടെ കാര്‍ ചന്ദോ­ക്കി­ന്റെ 
കാ­റി­ന്റെ മു­ക­ളില്‍­കൂ­ടി മറി­യു­ക­യാ­യി­രു­ന്നു. തക്ക­സ­മ­യ­ത്ത് തല താ­ഴ്ത്താന്‍ ചന്ദോ­ക്കി­നു തോ­ന്നി­യി­രു­ന്നി­ല്ലെ­ങ്കില്‍ 
അപ­ക­ട­വാര്‍­ത്ത­യ്ക്കു­പ­ക­രം അദ്ദേ­ഹ­ത്തി­ന്റെ മര­ണ­വാര്‍­ത്ത നമു­ക്കു വാ­യി­ക്കേ­ണ്ടി­വ­ന്നേ­നെ. റേ­സി­ന്റെ എഴു­പ­ത്തി­നാ­ലാം 
ലാ­പ്പില്‍ നട­ന്ന ഈ അപ­ക­ടം ഇനി അത്ഭു­ത­ങ്ങ­ളൊ­ന്നും ട്രാ­ക്കില്‍ കാ­ണി­ല്ലെ­ന്നേ­ക­ദേ­ശം ഉറ­പ്പാ­ക്കി. പക്ഷേ, അപ്പോ­ഴും 
വി­ട്ടു­കൊ­ടു­ക്കാന്‍ തയ്യാ­റാ­കാ­തി­രു­ന്ന മൈ­ക്കേല്‍ ഷൂ­മാ­ക്കര്‍ അവ­സാ­ന­നി­മി­ഷം എല്ലാ­വ­രെ­യും അമ്പ­ര­പ്പി­ച്ച് 
അലോണ്‍­സൊ­യെ മറി­ക­ട­ന്നു­.

ഈ വി­ജ­യ­ത്തോ­ടെ ടീം മേ­റ്റ് വെ­റ്റ­ലി­നൊ­പ്പം ചാ­മ്പ്യന്‍­ഷി­പ്പ് പോ­രാ­ട്ട­ത്തില്‍ മു­ന്നി­ട്ടു­നില്‍­ക്കി­ന്ന വെ­ബ്ബ­റി­ന് 
വെ­ല്ലു­വി­ളി­യു­യര്‍­ത്താന്‍ വെ­റ്റ­ലി­ന് ഒരി­ക്ക­ലും സാ­ധി­ച്ചി­ല്ല. മാ­ത്ര­വു­മ­ല്ല, കു­ബി­ത്സ­യ­യില്‍ നി­ന്ന് ആവ­ശ്യ­ത്തി­നു 
സമ്മര്‍­ദ്ദ­ത്തി­ലു­മാ­യി­രു­ന്നു വെ­റ്റല്‍. ഒരു ചെ­റിയ പി­ഴ­വു­പോ­ലും വരു­ത്താ­തെ മൂ­ന്നു സേ­ഫ്റ്റി­കാ­റു­ക­ളില്‍ നി­ന്നും രക്ഷ­പ്പെ­ട്ട 
വെ­ബ്ബര്‍ സ്പെ­യി­നി­ലെ തന്റെ ഫോം തു­ട­രു­ക­യാ­യി­രു­ന്നു­വെ­ന്നു പറ­യ­ണം. എന്നാല്‍ റേ­സി­ലെ താ­രം പി­റ്റ് ലേ­നില്‍ നി­ന്ന് 
റേ­സ് ആരം­ഭി­ച്ച്, തു­ട­ക്ക­ത്തില്‍­ത്ത­ന്നെ ആവേ­ശ­ക­ര­മായ മറി­ക­ട­ക്ക­ലു­ക­ളി­ലൂ­ടെ­യും പി­റ്റ് സ്റ്റോ­പ്പെ­ടു­ക്കാ­തെ, മു­ഴു­വന്‍ റേ­സും 
ഒരു ടയ­റില്‍ തീര്‍­ത്ത തന്ത്ര­ത്തി­ലൂ­ടെ­യും ആറാ­മ­ത് ഫി­നി­ഷ് ചെ­യ്ത ഫെ­റാ­രി­യു­ടെ ഫെര്‍­ണാ­ണ്ടോ അലോണ്‍­സോ­യാ­ണ്. 
(ഏ­താ­ണ്ട് മു­ഴു­വന്‍ റേ­സും, ആദ്യ യെ­ല്ലോ­ഫ്ലാ­ഗ് വന്ന­പ്പോള്‍ പി­റ്റ് ചെ­യ്ത് സോ­ഫ്റ്റ് ടയ­റു­കള്‍ മാ­റ്റി ഓപ്ഷന്‍ ടയ­റു­കള്‍ 
എടു­ക്കു­ക­യാ­ണ് അലോണ്‍­സോ ചെ­യ്ത­ത്). രണ്ടാ­യി­ര­ത്തി­യാ­റി­ലെ ഷൂ­മാ­ക്ക­റു­ടെ മൊ­ണാ­കൊ പ്ര­ക­ട­ന­ത്തെ ഓര്‍­മ്മി­പ്പി­ച്ചു 
ഇത്. അന്ന്, അലോണ്‍­സോ­യു­ടെ യോ­ഗ്യ­താ­ലാ­പ്പ് അല­ങ്കോ­ല­മാ­ക്കാന്‍ കാര്‍ ട്രാ­ക്കി­നു വി­ല­ങ്ങ­നെ­യി­ട്ട ഷൂ­മാ­ക്ക­റി­നെ, 
പി­ഴ­ചു­മ­ത്തി പി­റ്റ് ലേ­നി­ലെ­ത്തി­ക്കു­ക­യാ­യി­രു­ന്നു. ആവേ­ശ­ക­ര­മായ ഒരു റേ­സി­ലൂ­ടെ ഷു­മാ­ക്കര്‍ അന്ന് അഞ്ചാ­മ­ത് ഫി­നി­ഷ് 
ചെ­യ്തു.

­നി­ല­വി­ലെ ചാ­മ്പ്യന്‍ മക്‌­ലാ­ര­ന്റെ ജെന്‍­സണ്‍ ബട്ട­ണ് മൊ­ണാ­കൊ­യില്‍ കാ­ര്യ­ങ്ങള്‍ അത്ര ശു­ഭ­ക­ര­മ­ല്ലാ­യി­രു­ന്നു. 
ചാ­മ്പ്യന്‍­ഷി­പ്പ് പട്ടി­ക­യില്‍ എഴു­പ­തു പോ­യി­ന്റോ­ടെ ഒന്നാ­മ­നാ­യെ­ത്തിയ ബട്ടണ്‍ മട­ങ്ങു­മ്പോള്‍ നാ­ലാ­മ­താ­ണ്. മൂ­ന്നാം 
ലാ­പ്പില്‍ എന്‍­ജിന്‍ തക­രാ­റു­മൂ­ലം പു­റ­ത്തു­പോ­കേ­ണ്ടി­വ­ന്ന ബട്ട­നെ, റെ­ഡ്ബു­ള്ളി­ന്റെ ഡ്രൈ­വര്‍­മാ­രും (78 പോ­യി­ന്റു വീ­തം­),
ഫെര്‍­ണാ­ണ്ടോ അലോണ്‍­സോ­യു­മാ­ണ് (75 പോ­യി­ന്റ്) മറി­ക­ട­ന്ന­ത്. ഫെ­റാ­രി­യു­ടെ ഫെ­ലി­പെ മസ്സ 61 പോ­യി­ന്റു­മാ­യും, 
മൊ­ണാ­കൊ­യി­ലെ മൂ­ന്നാം സ്ഥാ­ന­ക്കാ­രന്‍ റെ­നോ­യു­ടെ റോ­ബര്‍­ട്ട് കു­ബി­ത്സ­യും, മക്‌­ലാ­ര­ന്റെ 2008­ലെ ലോ­ക­ചാ­മ്പ്യന്‍ 
ലൂ­യി­സ് ഹാ­മില്‍­ട്ട­ണും 59 പോ­യി­ന്റു­മാ­യും ബട്ട­ണു തൊ­ട്ടു­പി­റ­കി­ലു­മു­ണ്ട്. ആദ്യ ഏട്ടു­സ്ഥാ­ന­ക്കാ­രെ വെ­റും 25 പോ­യി­ന്റ് 
വേര്‍­തി­രി­ക്കു­ന്ന പട്ടിക ഇപ്പോ­ഴും ഇതൊ­രു തു­റ­ന്ന പോ­രാ­ട്ട­മാ­ണെ­ന്നു വ്യ­ക്ത­മാ­ക്കു­ന്നു. മെ­ഴ്സി­ഡ­സി­ന്റെ നി­കൊ 
റോ­സ്ബര്‍­ഗാ­ണ് എട്ടാ­മ­ത്. ഒന്‍­പ­താ­മ­ത്, മൈ­ക്കല്‍ ഷൂ­മാ­ക്ക­റും, പത്താ­മ­ത് ഫോ­ഴ്സ് ഇന്ത്യ­യു­ടെ അഡ്രി­യാന്‍ സു­ടി­ലാ­ണ്.

­ടീ­മു­ക­ളു­ടെ കാ­ര്യ­ത്തില്‍ റെ­ഡ്ബുള്‍ 156 പോ­യി­ന്റോ­ടെ രണ്ടാം­സ്ഥാ­ന­ത്തു­ള്ള ഫെ­റാ­രി­യേ­ക്കാള്‍ 22 പോ­യി­ന്റി­നു 
മു­ന്നി­ലാ­ണ്. മൂ­ന്നാം­സ്ഥാ­ന­ത്ത് നില്‍­ക്കു­ന്ന മക്‌­ലാ­ര­നും ഫെ­റാ­രി­യും തമ്മില്‍ വെ­റും 5 പോ­യി­ന്റ് വ്യ­ത്യാ­സ­മേ­യു­ള്ളൂ. 
എന്നാല്‍ മെ­ഴ്സി­ഡ­സ് 78 പോ­യി­ന്റു­മാ­യി ബഹു­ദൂ­രം പി­ന്നി­ലാ­ണ്. അഞ്ചാം സ്ഥാ­ന­ത്തു­ള്ള റെ­നോ കു­ബി­ത്സ­യു­ടെ 
ബല­ത്തില്‍ 65 പോ­യി­ന്റു­മാ­യി, മെ­ഴ്സി­ഡ­സി­ന് വെ­ല്ലു­വി­ളി­യു­യര്‍­ത്തു­ന്നു. വ്യ­ക്ത­മായ മധ്യ­നിര പ്ര­ക­ട­ന­വു­മാ­യി ഫോ­ഴ്സ് 
ഇന്ത്യ 30 പോ­യി­ന്റോ­ടെ ആറാ­മ­താ­ണ്.

അ­ടു­ത്ത­റേ­സി­നു വലിയ കാ­ല­താ­മ­സ­മി­ല്ലാ­ത്ത­തി­നാല്‍ (­മേ­യ് അവ­സാ­ന­വാ­രം തുര്‍­ക്കി­യില്‍) വലിയ മാ­റ്റ­ങ്ങ­ളൊ­ന്നും 
പ്ര­തീ­ക്ഷി­ക്കേ­ണ്ട­തി­ല്ല. എങ്കി­ലും തന്റെ വെ­ല്ലു­വി­ളി മധ്യ­നി­ര­യില്‍ നി­ന്ന് മുന്‍ നി­ര­യി­ലേ­ക്കെ­ത്തി­ക്കാന്‍ മൈ­ക്കേല്‍ ഷൂ­മാ­ക്കര്‍ 
ആവേ­ശ­പൂര്‍­വ്വം ശ്ര­മി­ക്കു­ന്ന­തും, ആദ്യ ഏട്ടു­സ്ഥാ­ന­ത്തില്‍ ഓരോ­റേ­സി­ലും മാ­റ്റ­ങ്ങള്‍ പ്ര­തീ­ക്ഷി­ക്കാ­മെ­ന്ന­തും തുര്‍­ക്കി­യില്‍ 
ആവേ­ശ­മു­ണര്‍­ത്തും.

­പിന്‍­കു­റി­പ്പ്: ഏഴു റേ­സു­ക­ളു­മാ­യി 1950­ലാ­ണ് ആദ്യ­ഫോര്‍­മു­ല­വണ്‍ ചാ­മ്പ്യന്‍­ഷി­പ്പ് തു­ട­ങ്ങു­ന്ന­ത്. കൃ­ത്യം പറ­ഞ്ഞാല്‍ 
1950 മേ­യ് 13­ന് ഇം­ഗ്ല­ണ്ടി­ലെ സില്‍­വര്‍­സ്റ്റോ­ണില്‍. 2010 റേ­സ് കല­ണ്ട­റില്‍ അന്നു­ണ്ടാ­യി­രു­ന്ന­തില്‍ നാ­ലു 
ട്രാ­ക്കു­ക­ളില്‍ ഇന്നും ചാ­മ്പ്യന്‍­ഷി­പ്പ് മത്സ­ര­ങ്ങള്‍ നട­ക്കു­ന്നു­ണ്ട്. മൊ­ണ്ടേ കാര്‍­ലോ, സില്‍­വര്‍­സ്റ്റോണ്‍, മോണ്‍­സ 
(ഇ­റ്റാ­ലി­യന്‍), സ്പാ ­(­ബെല്‍­ജി­യന്‍) എന്നി­വ­യാ­ണ­ത്. അറു­പ­തു വര്‍­ഷ­ങ്ങള്‍­ക്കു ശേ­ഷം ഏറെ­മാ­റ്റ­ങ്ങ­ളു­മാ­യി എഫ് 
വണ്‍ പ്ര­യാ­ണം തു­ട­രു­മ്പോള്‍, ഇന്നും യൂ­റോ­പ്യന്‍ ടീ­മു­ക­ളാ­ണ് ചാ­മ്പ്യന്‍­ഷി­പ്പി­നെ നി­യ­ന്ത്രി­ക്കു­ന്ന­തെ­ന്ന­ത് മറ്റൊ­രു 
സത്യം­.

(18 May 2010)\footnote{http://malayal.am/വിനോദം/കായികം/5513/അപകടങ്ങളുടെ-മോണ്ടേകാര്‍ലോ}

\newpage
