\secstar{ടയറുകള്‍ കളിനിയന്ത്രിച്ച കാനഡ ഗ്രാന്‍പ്രി}
\vskip 2pt

­ട്രാ­ക്കി­ന്റെ പ്ര­ത്യേ­ക­ത­കൊ­ണ്ട് ആവേ­ശ­ക­ര­മാ­യി മാ­റിയ റേ­സാ­യി­രു­ന്നു കാ­ന­ഡ­യി­ലേ­ത്. മോണ്‍­ട്രി­യാ­ലി­ലെ ഗി­ല്ലി­സ് 
വി­ലെ­ന്യ­വേ സര്‍­ക്യൂ­ട്ടില്‍ നട­ന്ന പോ­രാ­ട്ടം ആവേ­ശ­ക­ര­മാ­യ­ത് ട്രാ­ക്കി­ന്റെ പ്ര­ത­ല­വു­മാ­യി യോ­ജി­ച്ചു പോ­കു­ന്ന ടയ­റു­കള്‍ 
തി­ര­ഞ്ഞെ­ടു­ക്കു­ന്ന­തില്‍ ടീ­മു­കള്‍ പരാ­ജ­യ­പ്പെ­ട്ട­തി­നാ­ലാ­ണ്. അത് പതി­വി­നു വി­പ­രീ­ത­മാ­യി എല്ലാ­വ­രും രണ്ടു പി­റ്റ് 
സ്റ്റോ­പ്പെ­ങ്കി­ലും എടു­ക്കാന്‍ കാ­ര­ണ­മാ­യി. ടയ­റു­കള്‍ തി­ര­ഞ്ഞെ­ടു­ക്കു­ന്ന­തില്‍ പി­ഴ­വു­പ­റ്റിയ ­മെ­ഴ്സി­ഡ­സ് യോ­ഗ്യ­താ റൌ­ണ്ടില്‍ 
മോ­​­ശ­മാ­യി­പ്പോ­യി. മക്‌­ലാ­ര­ന്റെ ഹാ­മില്‍­ട്ടണ്‍ പോള്‍ നേ­ടി റെ­ഡ്ബു­ള്ളി­ന്റെ കു­ത്തക അവ­സാ­നി­പ്പി­ക്കു­ക­യും ചെ­യ്തു. 
എന്താ­യാ­ലും പ്രാ­ക്റ്റീ­സ്/­യോ­ഗ്യ­താ റൌ­ണ്ടു­ക­ളില്‍ കണ്ട അപ്ര­വ­ച­നീ­യത റേ­സി­ലും വന്ന­തോ­ടെ­യാ­ണ് മത്സ­രം നന്നാ­യ­ത്.

­പോള്‍ നേ­ടിയ ഹാ­മില്‍­ട്ടണ്‍ തന്നെ റേ­സ് ഒന്നാ­മ­നാ­യി ഫി­നി­ഷ് ചെ­യ്തെ­ങ്കി­ലും മുന്‍ റേ­സു­ക­ളില്‍ നി­ന്നും വി­പ­രീ­ത­മാ­യി,
ആദ്യാ­വ­സാ­നം രണ്ടും മൂ­ന്നും സ്ഥാ­ന­ക്കാ­രില്‍ നി­ന്ന് നല്ല സമ്മര്‍­ദ്ദ­മാ­യി­രു­ന്നു നേ­രി­ട്ട­ത്. രണ്ടാ­മ­താ­യി യോ­ഗ്യ­ത­നേ­ടി­യെ­ങ്കി­ലും
ഗി­യര്‍ ബോ­ക്സ് മാ­റ്റി­വ­ച്ച­തി­നാല്‍ ഗ്രി­ഡ്ഡില്‍ റെ­ഡ്ബു­ള­ളി­ന്റെ ­മാര്‍­ക് വെ­ബ്ബര്‍ എട്ടാ­മ­താ­യാ­ണ് തു­ട­ങ്ങി­യ­ത്. എന്നാല്‍ 
ആദ്യ­ലാ­പ്പില്‍ ഫെ­ലി­പെ മസ്സ­യും വി­റ്റാന്‍­ടോ­ണി­യോ ലി­യു­സ്സി­യും തമ്മി­ലു­ണ്ടായ ഉര­സല്‍ രണ്ടു­കാ­റു­ക­ളെ­യും 
പി­റ്റി­ലെ­ത്തി­ച്ച­ത് ലീ­ഡര്‍ ടേ­ബി­ളി­ലും മാ­റ്റ­ങ്ങള്‍ വരു­ത്തി. അത്യു­ഗ്രന്‍ സ്റ്റാര്‍­ട്ടി­ലൂ­ടെ ഷു­മാ­ക്ക­റും ആദ്യ പത്തി­ലി­ടം കണ്ടെ­ത്തി­.

എ­ന്നാല്‍ ടയ­റു­കള്‍ വി­ചാ­രി­ച്ച­ത്ര നി­ല­നില്‍­ക്കാ­ഞ്ഞ­ത്, ഏഴാം ലാ­പ്പില്‍­ത്ത­ന്നെ റെ­ഗു­ലര്‍ പി­റ്റ് സ്റ്റോ­പ്പു­കള്‍ എടു­ക്കു­ന്ന 
കാ­ഴ്ച­യാ­ണ് സമ്മാ­നി­ച്ച­ത്. അഞ്ചാം­ലാ­പ്പില്‍ വെ­ബ്ബര്‍ ബട്ട­ണെ മറി­ക­ട­ന്ന് മുന്‍ നി­ര­യി­ലെ­ത്തി. എന്നാല്‍ യോ­ഗ്യ­താ 
റൌ­ണ്ടില്‍ സോ­ഫ്റ്റ് ടയ­റു­കള്‍ പരീ­ക്ഷി­ച്ച ടീ­മു­കള്‍­ക്ക് നി­ര­നി­ര­യാ­യി പി­റ്റു ചെ­യ്യേ­ണ്ട അവ­സ്ഥ­യു­ണ്ടാ­ക്കി (യോ­ഗ്യ­താ 
റൌ­ണ്ടില്‍ മൂ­ന്നാം പാ­ദ­ത്തി­ലെ­ത്തിയ ഡ്രൈ­വര്‍­മാര്‍ അതേ ടയ­റില്‍ വേ­ണം റേ­സ് തു­ട­ങ്ങാന്‍). റേ­സി­ന്റെ തു­ട­ക്ക­ത്തില്‍ 
ഒരു സമ­യം റെ­ഡ്ബു­ള്ളു­കള്‍ ഒന്നും രണ്ടും സ്ഥാ­ന­ത്തും ഷു­മാ­ക്കര്‍ മൂ­ന്ന­മ­തു­മാ­യി­രു­ന്നു­.

ആ­ദ്യ പി­റ്റ് സ്റ്റോ­പ്പു­കള്‍ കഴി­ഞ്ഞ­പ്പോള്‍ ചി­ത്ര­മാ­കെ മാ­റി. സോ­ഫ്റ്റ് ടയ­റു­ക­ളി­ലേ­ക്കു മാ­റിയ വെ­റ്റല്‍ ഒരു­പാ­ടു പി­ന്നി­ലേ­ക്കു 
പോ­യ­പ്പോള്‍, മി­ഡ്ഫീല്‍­ഡി­ലെ പൊ­രി­ഞ്ഞ പോ­രാ­ട്ടം വെ­ബ്ബ­റി­നു വി­ന­യാ­യി. കു­ബി­ത്സ­യും, ടോ­റോ റോ­സോ­യു­ടെ 
ബ്യു­യെ­മി­യും, ഫോ­ഴ്സ് ഇന്ത്യ­യു­ടെ സു­ട്ടി­ലും കാ­ര്യ­ങ്ങള്‍ കഷ്ട­മാ­ക്കി. ആദ്യ പി­റ്റ് സ്റ്റോ­പ്പില്‍ വച്ച് അലോണ്‍­സൊ ഹാ­മില്‍­ട്ട­ണെ
മറി­ക­ട­ന്നെ­ങ്കി­ലും വേ­ഗം തന്നെ അവ­സാന സ്ട്രൈ­റ്റില്‍ വച്ച് പൊ­സി­ഷന്‍ തി­രി­ച്ചു പി­ടി­ച്ചു. റെ­ഡ്ബു­ള്ളു­ക­ളു­ടെ പി­ഴ­വു­കള്‍ 
മു­ത­ലാ­ക്കി ഈ സമ­യം കൊ­ണ്ട് ബട്ടണ്‍ മൂ­ന്നാ­മ­തെ­ത്തു­ക­യും ചെ­യ്തു. പി­ന്നെ ഹാ­മില്‍­ട്ട­ണി­നു മേല്‍ സമ്മര്‍­ദ്ദം 
ചെ­ലു­ത്തു­ന്ന­തി­നോ­ടൊ­പ്പം തന്നെ ബട്ട­ണില്‍ നി­ന്നും സമ്മര്‍­ദ്ദ­ത്തി­ലാ­വു­ക­യും ചെ­യ്തു അലോണ്‍­സൊ. അവ­സാ­നം 
കരണ്‍ ചന്ദോ­ക്കി­ന്റെ ഹി­സ്പാ­നി­ക് റേ­സി­ങ് ടീം കാ­റി­നു പി­ന്നില്‍ പെ­ട്ടു­പോയ അവ­സ­രം നോ­ക്കി ബട്ടണ്‍ രണ്ടാം സ്ഥാ­നം 
പി­ടി­ച്ചെ­ടു­ക്കു­ക­യും ചെ­യ്തു­.

ഒ­ന്നാം പി­റ്റ് സ്റ്റോ­പ്പ് വരെ നല്ല പ്ര­ക­ട­നം കാ­ഴ്ച­വെ­ച്ച ഷു­മാ­ക്കര്‍ പി­റ്റ് സ്റ്റോ­പ്പി­നു ശേ­ഷം കു­ബി­ത്സ­യു­ടെ ആക്ര­മ­ണ­ത്തി­ന്റെ 
ഫല­മാ­യി ട്രാ­ക്കില്‍ നി­ന്നും മാ­റി­പ്പോ­കേ­ണ്ടി വന്നു. തത്കാ­ലം സ്ഥാ­നം സം­ര­ക്ഷി­ക്കാ­നാ­യെ­ങ്കി­ലും ഉടന്‍­ത­ന്നെ, 
ഉര­സ­ലി­ന്റെ ഫല­മാ­യി ഒരു പി­റ്റ് സ്റ്റോ­പ്പി­നു നിര്‍­ബ­ന്ധി­ത­നാ­വു­ക­യും റേ­സില്‍ പി­ന്നി­ലാ­യി­പ്പോ­വു­ക­യും ചെ­യ്ത. പി­ന്നീ­ട് 
സ്ഥാ­നം മെ­ച്ച­പ്പെ­ടു­ത്താ­നാ­യെ­ങ്കി­ലും ഒരി­ക്ക­ലും മുന്‍­നി­ര­യി­ലെ­ത്താ­നാ­യി­ല്ല.

അ­വ­സാ­ന­ലാ­പ്പു­ക­ളില്‍ മോ­ശം ടയ­റു­ക­ളി­ലാ­യി­പ്പോയ ഷു­മാ­ക്കര്‍ ഫോ­ഴ്സ് ഇന്ത്യ­യു­ടെ കാ­റു­ക­ളില്‍ നി­ന്നും നി­ര­ന്തര 
ആക്ര­മ­ണ­ത്തി­ലു­മാ­യി­രു­ന്നു. രണ്ടു­മൂ­ന്നു ലാ­പ്പു­കള്‍ പി­ടി­ച്ചു­നി­ന്നെ­ങ്കി­ലും അവ­സാ­നം രണ്ടു ഫോ­ഴ്സ് ഇന്ത്യ­കള്‍­ക്കും മു­ന്നില്‍ 
കീ­ഴ­ട­ങ്ങി. ആദ്യ­ലാ­പ്പു­ക­ളി­ലെ അസ്ഥി­രത മു­ത­ലെ­ടു­ത്ത ടോ­റോ റൊ­സോ­യു­ടെ ­ബ്യു­യെ­മി­ അവ­സാ­നം വരെ 
ഏതാ­ണ്ടൊ­ക്കെ പൊ­സി­ഷന്‍ നി­ല­നിര്‍­ത്തു­ക­യും ചെ­യ്തു. വെ­റും അഞ്ച് റി­ട്ട­യര്‍­മെ­ന്റു­കള്‍ മാ­ത്രം നട­ന്ന റേ­സ് 
സ്ഥി­ര­ത­യു­ടെ കാ­ര്യ­ത്തില്‍ ടീ­മു­കള്‍­ക്ക് ആശ്വാ­സ­മാ­യി­ക്കാ­ണ­ണം. റി­ട്ട­യര്‍­മെ­ന്റ്/ആ­ക്സി­ഡെ­ന്റ് തു­ടര്‍­ക്ക­ഥ­യാ­ക്കിയ 
ബ്രൂ­ണോ സെ­ന്ന­യും യാ­നോ ട്രൂ­ലി­യും ഇത്ത­വ­ണ­യും മു­ഴു­വന്‍ റേ­സും തീര്‍­ത്തി­ല്ല. പെ­ഡ്രോ ഡി ലാ റൊ­സ­യു­ടെ 
റി­ട്ട­യര്‍­മെ­ന്റ് സൌ­ബ­റി­ന്റെ ­ഫെ­റാ­രി­ എന്‍­ജി­നു­മാ­യു­ള്ള പ്ര­ശ്ന­ങ്ങള്‍ ഇനി­യും തീര്‍­ന്നി­ല്ലെ­ന്നു വ്യ­ക്ത­മാ­ക്കി­.

­പ­തി­വില്‍ നി­ന്നും വ്യ­ത്യ­സ്ത­മാ­യി ഹാര്‍­ഡ് ടയ­റു­കള്‍ നല്ല പെര്‍­ഫോര്‍­മന്‍­സ് കാ­ഴ്ച­വ­യ്ക്കു­ക­യും രണ്ടു­ത­രം ടയ­റു­ക­ളും 
നിര്‍­ബ­ന്ധ­മാ­യി ഉപ­യോ­ഗി­ക്ക­ണ­മെ­ന്ന നി­യ­മം ഡ്രൈ­വര്‍­മാ­രെ വല­യ്ക്കു­ക­യും ചെ­യ്ത­ത് കാ­ണി­കള്‍­ക്ക് ആവേ­ശ­ക­ര­മായ 
പോ­രാ­ട്ട­ങ്ങള്‍ ട്രാ­ക്കി­ലൊ­രു­ക്കി. പതി­വി­ല്ലാ­തെ മദ്ധ്യ­നി­ര­യി­ലും മുന്‍­നി­ര­യി­ലും ഒരു­പോ­ലെ പോ­രാ­ട്ട­ങ്ങ­ളും പൊ­സി­ഷന്‍ 
മാ­റ­ലു­ക­ളും നട­ന്ന­ത് റേ­സ് ആദ്യാ­ന്തം ആവേ­ശ­ക­ര­മാ­ക്കി. ­ക­നേ­ഡി­യന്‍ ഗ്രാന്‍­പ്രീ­ കി­രീ­ട­പ്പോ­രാ­ട്ട­ങ്ങ­ളില്‍ വലിയ 
മാ­റ്റ­ങ്ങ­ളൊ­ന്നും വരു­ത്തി­ല്ലെ­ങ്കി­ലും ഒരു 5-വേ ചാ­മ്പ്യന്‍­ഷി­പ്പ് പോ­രാ­ട്ട­ത്തി­നു­ള്ള കാ­ഹ­ള­ങ്ങ­ളാ­ണ് അണി­യ­റ­യില്‍ നി­ന്നും 
മു­ഴ­ങ്ങു­ന്ന­ത്. പാര്‍­ട്ടി സ്പോ­യി­ലേ­ഴ്സാ­യി റൊ­സ്ബര്‍­ഗും ­(74),­ കു­ബി­ത്സ­യും ­(73),­ മ­സ്സ­യും ­(67) ഒപ്പ­ത്തി­നൊ­പ്പ­മു­ണ്ട്. 
ഇവര്‍­ക്ക് പോ­ഡി­യ­ങ്ങ­ളും ഒന്നാം സ്ഥാ­ന­ങ്ങ­ളും വരും റേ­സു­ക­ളില്‍ നേ­ടാ­നാ­വു­മെ­ങ്കില്‍ തീര്‍­ച്ച­യാ­യും അത്ഭു­ത­ങ്ങള്‍­ക്കി­നി­യും
പഴു­തു­ണ്ട് (ഒ­രു പക്ഷേ 2007 നേ­ക്കാ­ളും മി­ക­ച്ച ഫി­നി­ഷി­നു വരെ­യും­).

­ഞാ­യ­റാ­ഴ്ച­ത്തെ റേ­സി­നു ശേ­ഷം ­(13 ജൂണ്‍) 109 പേ­ാ­യി­ന്റു­മാ­യി ഹാ­മില്‍­ട്ട­ണാ­ണ് ഒന്നാ­മ­ത്. ടീം മേ­റ്റ് ബട്ടണ്‍ വെ­റും 
മൂ­ന്നു പോ­യി­ന്റ് വ്യ­താ­സ­ത്തില്‍ രണ്ടാ­മ­തും മാര്‍­ക് വെ­ബ്ബര്‍ 103 പോ­യി­ന്റു­മാ­യി മൂ­ന്നാ­മ­തു­മാ­ണ്. അലോണ്‍­സൊ­യും (94) വെ­റ്റ­ലു ­(90)­ മാ­ണ് നാ­ലും അഞ്ചും സ്ഥാ­ന­ങ്ങ­ളില്‍. ടീ­മു­ക­ളു­ടെ കാ­ര്യ­ത്തില്‍ മക്‌­ലാ­രന്‍ 215 പോ­യി­ന്റു­മാ­യി 
റെ­ഡ്ബു­ള്ളില്‍ നി­ന്നും 22 പോ­യി­ന്റ് മു­ന്നി­ലാ­ണ്. ഫെ­റാ­രി 161 പോ­യി­ന്റു­മാ­യി മൂ­ന്നാ­മ­താ­ണ്.

­ചി­ര­പ­രി­ചി­ത­മായ യൂ­റോ­പ്യന്‍ ട്രാ­ക്കു­ക­ളില്‍ നട­ക്കു­ന്ന റേ­സു­ക­ളാ­ണ് ഇനി വരും വാ­ര­ങ്ങ­ളില്‍. മധ്യ­നിര ടീ­മു­ക­ളായ ഫോ­ഴ്സ്
ഇന്ത്യ­യും റെ­നോ­യും കഴി­ഞ്ഞ വര്‍­ഷം ഈ ട്രാ­ക്കു­ക­ളില്‍ മി­ക­ച്ച പ്ര­ക­ട­ന­മാ­ണ് കാ­ഴ്ച­വ­ച്ച­ത്. എന്നാല്‍ ബ്രാ­വ്‌­ണി­ന്റെ 
പ്ര­ക­ട­നം അത്ര മെ­ച്ച­വു­മാ­യി­രു­ന്നി­ല്ല. മി­ക­ച്ച കാ­റ­ല്ലെ­ങ്കി­ലും, പ്ര­ക­ട­ന­ങ്ങള്‍ ശരാ­ശ­രി­മാ­ത്ര­മാ­ണെ­ങ്കി­ലും ഷൂ­മാ­ക്ക­റി­ന് 
എല്ലാ­വ­രും ട്രാ­ക്കില്‍ നല്‍­കു­ന്ന ബഹു­മാ­ന­വും ഓരോ പോ­യി­ന്റി­നും വേ­ണ്ടി­യു­ള്ള പോ­രാ­ട്ട­ങ്ങ­ളും മെ­ഴ്സി­ഡ­സി­ന് എന്നും 
മു­തല്‍­ക്കൂ­ട്ടാ­ണ്. റൊ­സ്ബര്‍­ഗ് ആദ്യ­റേ­സു­ക­ളി­ലെ തന്റെ നി­ല­വാ­ര­ത്തി­ലേ­ക്ക് തി­രി­ച്ചെ­ത്തു­ക­യാ­ണെ­ങ്കില്‍, കി­രീ­ടം 
നേ­ടാ­നാ­യി­ല്ലെ­ങ്കി­ലും പല­രു­ടെ­യും കഞ്ഞി­യില്‍ പാ­റ്റ­യാ­വാന്‍ ഇപ്പോ­ഴും കെല്‍­പ്പു­ള്ള ടീ­മാ­ണ് മെ­ഴ്സി­ഡ­സ്. വി­ല്യം­സും, 
സൌ­ബ­റും, ടോ­റോ റൊ­സൊ­യും പ്ര­ക­ട­നം മെ­ച്ച­പ്പെ­ടു­ത്തി­യ­തും മധ്യ­നിര പോ­രാ­ട്ട­ങ്ങള്‍ മുന്‍­നിര മത്സ­ര­ങ്ങ­ളേ­ക്കാള്‍ 
ആവേ­ശ­ക­ര­മാ­വാ­നു­ള്ള സാ­ധ്യ­ത­യി­ലേ­ക്കാ­ണ് വി­രല്‍ ചൂ­ണ്ടു­ന്ന­ത്.

­പല ടീ­മു­ക­ളും അടു­ത്ത സീ­സ­ണി­ലെ കാ­റി­ന്റെ കാ­ര്യം പറ­ഞ്ഞു തു­ട­ങ്ങി­യി­ട്ടു­ണ്ടെ­ങ്കി­ലും അടു­ത്ത­മാ­സം അവ­സാ­ന­ത്തോ­ടെ 
മാ­ത്ര­മേ ഈ സീ­സ­ണി­ലെ വി­ക­സ­ന­ങ്ങ­ളെ­ക്കു­റി­ച്ച് അന്തിമ തീ­രു­മാ­ന­ത്തി­ലെ­ത്തു­ക­യു­ള്ളൂ. മുന്‍­നി­ര­ടീ­മു­കള്‍ മു­ഴു­വ­നാ­യും 
അടു­ത്ത സീ­സ­ണില്‍ കേ­ന്ദ്രീ­ക­രി­ക്കാ­നു­ള്ള സാ­ധ്യത തു­ച്ഛ­മാ­ണ്. മാ­ത്ര­മ­ല്ല, ഈ സീ­സ­ണി­ലെ പോ­രാ­ട്ടം കടു­ത്ത­താ­യ­ത് 
തീ­രു­മാ­ന­ങ്ങ­ളെ സ്വാ­ധീ­നി­ക്കാ­നു­മി­ട­യു­ണ്ട്. എന്നാല്‍ മധ്യ­നിര ടീ­മു­കള്‍ പല­തും കൂ­ടു­തല്‍ റി­സോ­ഴ്സു­കള്‍ അടു­ത്ത സീ­സ­ണി­നു 
വേ­ണ്ടി മാ­റ്റി­വ­യ്ക്കാ­നാ­ണ് സാ­ധ്യ­ത. അത് അവ­സാന രണ്ടു­മാ­സ­ങ്ങ­ളി­ലെ പോ­രാ­ട്ട­ങ്ങ­ളെ വി­പ­രീ­ത­മാ­യി സ്വാ­ധീ­നി­ച്ചേ­ക്കാം­.

(15 June 2010)\footnote{http://malayal.am/വിനോദം/കായികം/6079/ടയറുകള്‍-കളിനിയന്ത്രിച്ച-കാനഡ-ഗ്രാന്‍പ്രി}

\newpage
