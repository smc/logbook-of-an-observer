\secstar{ടയറുകള്‍ കളിനിയന്ത്രിച്ച കാനഡ ഗ്രാന്‍പ്രി}
\vskip 2pt

ട്രാക്കിന്റെ പ്രത്യേകതകൊണ്ടു് ആവേശകരമായി മാറിയ റേസായിരുന്നു കാനഡയിലേതു്. മോണ്‍ട്രിയാലിലെ ഗില്ലിസ് 
വിലെന്യവേ സര്‍ക്യൂട്ടില്‍ നടന്ന പോരാട്ടം ആവേശകരമായതു് ട്രാക്കിന്റെ പ്രതലവുമായി യോജിച്ചുപോകുന്ന ടയറുകള്‍ 
തിരഞ്ഞെടുക്കുന്നതില്‍ ടീമുകള്‍ പരാജയപ്പെട്ടതിനാലാണു്. അതു് പതിവിനുവിപരീതമായി എല്ലാവരും രണ്ടു പിറ്റ് 
സ്റ്റോപ്പെങ്കിലും എടുക്കാന്‍ കാരണമായി. ടയറുകള്‍ തിരഞ്ഞെടുക്കുന്നതില്‍ പിഴവുപറ്റിയ മെഴ്സിഡസ് യോഗ്യതാ റൌണ്ടില്‍ 
മോ​ശമായിപ്പോയി. മക്‌ലാരന്റെ ഹാമില്‍ട്ടണ്‍ പോള്‍ നേടി റെഡ്ബുള്ളിന്റെ കുത്തക അവസാനിപ്പിക്കുകയും ചെയ്തു. 
എന്തായാലും പ്രാക്റ്റീസ്/യോഗ്യതാ റൌണ്ടുകളില്‍ കണ്ട അപ്രവചനീയത റേസിലും വന്നതോടെയാണു് മത്സരം നന്നായതു്.

പോള്‍ നേടിയ ഹാമില്‍ട്ടണ്‍തന്നെ റേസ് ഒന്നാമനായി ഫിനിഷ് ചെയ്തെങ്കിലും മുന്‍റേസുകളിനിന്നും വിപരീതമായി,
ആദ്യാവസാനം രണ്ടും മൂന്നും സ്ഥാനക്കാരിനിന്നു് നല്ല സമ്മര്‍ദ്ദമായിരുന്നു നേരിട്ടതു്. രണ്ടാമതായി യോഗ്യതനേടിയെങ്കിലും
ഗിയര്‍ ബോക്സ് മാറ്റിവച്ചതിനാല്‍ ഗ്രിഡ്ഡില്‍ റെഡ്ബുളളിന്റെ മാര്‍ക് വെബ്ബര്‍ എട്ടാമതായാണു് തുടങ്ങിയത്. എന്നാല്‍ 
ആദ്യലാപ്പില്‍ ഫെലിപെ മസ്സയും വിറ്റാന്‍ടോണിയോ ലിയുസ്സിയും തമ്മിലുണ്ടായ ഉരസല്‍ രണ്ടുകാറുകളെയും 
പിറ്റിലെത്തിച്ചത് ലീഡര്‍ ടേബിളിലും മാറ്റങ്ങള്‍ വരുത്തി. അത്യുഗ്രന്‍ സ്റ്റാര്‍ട്ടിലൂടെ ഷുമാക്കറും ആദ്യപത്തിലിടം കണ്ടെത്തി.

എന്നാല്‍ ടയറുകള്‍ വിചാരിച്ചത്ര നിലനില്‍ക്കാഞ്ഞതു്, ഏഴാംലാപ്പില്‍ത്തന്നെ റെഗുലര്‍ പിറ്റ് സ്റ്റോപ്പുകള്‍ എടുക്കുന്ന 
കാഴ്ചയാണു് സമ്മാനിച്ചത്. അഞ്ചാംലാപ്പില്‍ വെബ്ബര്‍ ബട്ടണെ മറികടന്നു് മുന്‍നിരയിലെത്തി. എന്നാല്‍ യോഗ്യതാ 
റൌണ്ടില്‍ സോഫ്റ്റ് ടയറുകള്‍ പരീക്ഷിച്ച ടീമുകള്‍ക്കു് നിരനിരയായി പിറ്റുചെയ്യേണ്ട അവസ്ഥയുണ്ടാക്കി. (യോഗ്യതാ 
റൌണ്ടില്‍ മൂന്നാംപാദത്തിലെത്തിയ ഡ്രൈവര്‍മാര്‍ അതേ ടയറില്‍ വേണം റേസ് തുടങ്ങാന്‍.) റേസിന്റെ തുടക്കത്തില്‍ 
ഒരു സമയം റെഡ്ബുള്ളുകള്‍ ഒന്നും രണ്ടും സ്ഥാനത്തും ഷുമാക്കര്‍ മൂന്നാമതുമായിരുന്നു.

ആദ്യ പിറ്റ് സ്റ്റോപ്പുകള്‍ കഴിഞ്ഞപ്പോള്‍ ചിത്രമാകെ മാറി. സോഫ്റ്റ് ടയറുകളിലേക്കു മാറിയ വെറ്റല്‍ ഒരുപാടു പിന്നിലേക്കു 
പോയപ്പോള്‍, മിഡ്ഫീല്‍ഡിലെ പൊരിഞ്ഞപോരാട്ടം വെബ്ബറിനു വിനയായി. കുബിത്സയും ടോറോ റോസോയുടെ 
ബ്യുയെമിയും ഫോഴ്സ് ഇന്ത്യയുടെ സുട്ടിലും കാര്യങ്ങള്‍ കഷ്ടമാക്കി. ആദ്യ പിറ്റ് സ്റ്റോപ്പില്‍വച്ചു് അലോണ്‍സൊ ഹാമില്‍ട്ടണെ
മറികടന്നെങ്കിലും വേഗംതന്നെ അവസാന സ്ട്രൈറ്റില്‍വച്ചു് പൊസിഷന്‍ തിരിച്ചുപിടിച്ചു. റെഡ്ബുള്ളുകളുടെ പിഴവുകള്‍ 
മുതലാക്കി ഈ സമയംകൊണ്ടു് ബട്ടണ്‍ മൂന്നാമതെത്തുകയും ചെയ്തു. ഹാമില്‍ട്ടണിനുമേല്‍ സമ്മര്‍ദ്ദം 
ചെലുത്തുന്നതിനോടൊപ്പംതന്നെ ബട്ടണില്‍നിന്നും സമ്മര്‍ദ്ദത്തിലാവുകയും ചെയ്തു അലോണ്‍സൊ. അവസാനം 
കരണ്‍ ചന്ദോക്കിന്റെ ഹിസ്പാനിക് റേസിങ് ടീം കാറിനു പിന്നില്‍ പെട്ടുപോയ അവസരം നോക്കി ബട്ടണ്‍ രണ്ടാംസ്ഥാനം 
പിടിച്ചെടുക്കുകയും ചെയ്തു.

ഒന്നാം പിറ്റ് സ്റ്റോപ്പുവരെ നല്ല പ്രകടനം കാഴ്ചവെച്ച ഷുമാക്കര്‍ പിറ്റ് സ്റ്റോപ്പിനുശേഷം കുബിത്സയുടെ ആക്രമണത്തിന്റെ 
ഫലമായി ട്രാക്കില്‍നിന്നും മാറിപ്പോകേണ്ടിവന്നു. തത്കാലം സ്ഥാനം സംരക്ഷിക്കാനായെങ്കിലും, ഉടന്‍തന്നെ 
ഉരസലിന്റെ ഫലമായി ഒരു പിറ്റ് സ്റ്റോപ്പിനു നിര്‍ബന്ധിതനാവുകയും റേസില്‍ പിന്നിലായിപ്പോവുകയും ചെയ്ത. പിന്നീടു് 
സ്ഥാനം മെച്ചപ്പെടുത്താനായെങ്കിലും ഒരിക്കലും മുന്‍നിരയിലെത്താനായില്ല.

അവസാനലാപ്പുകളില്‍ മോശം ടയറുകളിലായിപ്പോയ ഷുമാക്കര്‍ ഫോഴ്സ് ഇന്ത്യയുടെ കാറുകളില്‍നിന്നും നിരന്തര 
ആക്രമണത്തിലുമായിരുന്നു. രണ്ടുമൂന്നു ലാപ്പുകള്‍ പിടിച്ചുനിന്നെങ്കിലും അവസാനം രണ്ടു് ഫോഴ്സ് ഇന്ത്യകള്‍ക്കും പിന്നില്‍ 
കീഴടങ്ങി. ആദ്യലാപ്പുകളിലെ അസ്ഥിരത മുതലെടുത്ത ടോറോ റൊസോയുടെ ബ്യുയെമി അവസാനംവരെ 
ഏതാണ്ടൊക്കെ പൊസിഷന്‍ നിലനിര്‍ത്തുകയും ചെയ്തു. വെറും അഞ്ചു് റിട്ടയര്‍മെന്റുകള്‍ മാത്രം നടന്ന റേസ് 
സ്ഥിരതയുടെ കാര്യത്തില്‍ ടീമുകള്‍ക്കു് ആശ്വാസമായിക്കാണണം. റിട്ടയര്‍മെന്റ്/ആക്സിഡെന്റ് തുടര്‍ക്കഥയാക്കിയ 
ബ്രൂണോ സെന്നയും യാനോ ട്രൂലിയും ഇത്തവണയും മുഴുവന്‍ റേസും തീര്‍ത്തില്ല. പെഡ്രോ ഡി ലാ റൊസയുടെ 
റിട്ടയര്‍മെന്റ് സൌബറിന്റെ ഫെറാരി എന്‍ജിനുമായുള്ള പ്രശ്നങ്ങള്‍ ഇനിയും തീര്‍ന്നില്ലെന്നു വ്യക്തമാക്കി.

പതിവിനിന്നും വ്യത്യസ്തമായി ഹാര്‍ഡ് ടയറുകള്‍ നല്ല പെര്‍ഫോമന്‍സ് കാഴ്ചവയ്ക്കുകയും, രണ്ടുതരം ടയറുകളും 
നിര്‍ബന്ധമായി ഉപയോഗിക്കണമെന്ന നിയമം ഡ്രൈവര്‍മാരെ വലയ്ക്കുകയും ചെയ്തതു് കാണികള്‍ക്കു് ആവേശകരമായ 
പോരാട്ടങ്ങള്‍ ട്രാക്കിലൊരുക്കി. പതിവില്ലാതെ മദ്ധ്യനിരയിലും മുന്‍നിരയിലും ഒരുപോലെ പോരാട്ടങ്ങളും പൊസിഷന്‍ 
മാറലുകളും നടന്നതു് റേസ് ആദ്യന്തം ആവേശകരമാക്കി. കനേഡിയന്‍ ഗ്രാന്‍പ്രീ വലിയ മാറ്റങ്ങളൊന്നും 
കിരീടപ്പോരാട്ടങ്ങളില്‍ വരുത്തില്ലെങ്കിലും ഒരു 5-വേ ചാമ്പ്യന്‍ഷിപ്പ് പോരാട്ടത്തിനുള്ള കാഹളങ്ങളാണു് അണിയറയില്‍ മുഴങ്ങുന്നതു്. 
പാര്‍ട്ടി സ്പോയിലേഴ്സായി റൊസ്ബര്‍ഗും (74), കുബിത്സയും (73), മസ്സയും (67) ഒപ്പത്തിനൊപ്പമുണ്ടു്. 
ഇവര്‍ക്കു് പോഡിയങ്ങളും ഒന്നാംസ്ഥാനങ്ങളും വരും റേസുകളില്‍ നേടാനാവുമെങ്കില്‍ തീര്‍ച്ചയായും അത്ഭുതങ്ങള്‍ക്കിനിയും
പഴുതുണ്ടു്. ഒരു പക്ഷേ 2007 നേക്കാളും മികച്ച ഫിനിഷിനു വരെയും.

ഞായറാഴ്ചത്തെ റേസിനു ശേഷം (13 ജൂണ്‍) 109 പോയിന്റുമായി ഹാമില്‍ട്ടണാണു് ഒന്നാമതു്. ടീം മേറ്റ് ബട്ടണ്‍ വെറും 
മൂന്നു പോയിന്റ് വ്യതാസത്തില്‍ രണ്ടാമതും മാര്‍ക് വെബ്ബര്‍ 103 പോയിന്റുമായി മൂന്നാമതുമാണു്. അലോണ്‍സൊയും (94) വെറ്റലു (90) മാണു് 
നാലും അഞ്ചും സ്ഥാനങ്ങളില്‍. ടീമുകളുടെ കാര്യത്തില്‍ മക്‌ലാരന്‍ 215 പോയിന്റുമായി 
റെഡ്ബുള്ളില്‍ നിന്നും 22 പോയിന്റ് മുന്നിലാണു്. ഫെറാരി 161 പോയിന്റുമായി മൂന്നാമതാണു്.

ചിരപരിചിതമായ യൂറോപ്യന്‍ ട്രാക്കുകളില്‍ നടക്കുന്ന റേസുകളാണു് ഇനി വരുംവാരങ്ങളില്‍. മധ്യനിരടീമുകളായ ഫോഴ്സ്
ഇന്ത്യയും റെനോയും കഴിഞ്ഞവര്‍ഷം ഈ ട്രാക്കുകളില്‍ മികച്ച പ്രകടനമാണു് കാഴ്ചവച്ചതു്. എന്നാല്‍ ബ്രാവ്‌ണിന്റെ 
പ്രകടനം അത്ര മെച്ചവുമായിരുന്നില്ല. മികച്ച കാറല്ലെങ്കിലും, പ്രകടനങ്ങള്‍ ശരാശരിമാത്രമാണെങ്കിലും, ഷൂമാക്കറിനു് 
എല്ലാവരും ട്രാക്കില്‍ നല്‍കുന്ന ബഹുമാനവും ഓരോ പോയിന്റിനും വേണ്ടിയുള്ള പോരാട്ടങ്ങളും മെഴ്സിഡസിനു് എന്നും 
മുതല്‍ക്കൂട്ടാണു്. റൊസ്ബര്‍ഗ് ആദ്യറേസുകളിലെ തന്റെ നിലവാരത്തിലേക്കു് തിരിച്ചെത്തുകയാണെങ്കില്‍, കിരീടം 
നേടാനായില്ലെങ്കിലും പലരുടെയും കഞ്ഞിയില്‍ പാറ്റയാവാന്‍ ഇപ്പോഴും കെല്‍പ്പുള്ള ടീമാണു് മെഴ്സിഡസ്. വില്യംസും 
സൌബറും ടോറോ റൊസൊയും പ്രകടനം മെച്ചപ്പെടുത്തിയതു് മദ്ധ്യനിരപോരാട്ടങ്ങള്‍ മുന്‍നിരമത്സരങ്ങളേക്കാള്‍ 
ആവേശകരമാവാനുള്ള സാധ്യതയിലേക്കാണു് വിരല്‍ ചൂണ്ടുന്നതു്.

പല ടീമുകളും അടുത്ത സീസണിലെ കാറിന്റെ കാര്യം പറഞ്ഞുതുടങ്ങിയിട്ടുണ്ടെങ്കിലും അടുത്തമാസം അവസാനത്തോടെ 
മാത്രമേ ഈ സീസണിലെ വികസനങ്ങളെക്കുറിച്ചു് അന്തിമ തീരുമാനത്തിലെത്തുകയുള്ളൂ. മുന്‍നിരടീമുകള്‍ മുഴുവനായും 
അടുത്ത സീസണില്‍ കേന്ദ്രീകരിക്കാനുള്ള സാധ്യത തുച്ഛമാണു്. മാത്രമല്ല, ഈ സീസണിലെ പോരാട്ടം കടുത്തതായതു് 
തീരുമാനങ്ങളെ സ്വാധീനിക്കാനുമിടയുണ്ടു്. മദ്ധ്യനിരടീമുകള്‍ പലതും കൂടുതല്‍ റിസോഴ്സുകള്‍ അടുത്ത സീസണിനുവേണ്ടി 
മാറ്റിവയ്ക്കാനാണു് സാധ്യത. അതു് അവസാന രണ്ടുമാസങ്ങളിലെ പോരാട്ടങ്ങളെ വിപരീതമായി സ്വാധീനിച്ചേക്കാം.

(15 June 2010)\footnote{http://malayal.am/വിനോദം/കായികം/6079/ടയറുകള്‍-കളിനിയന്ത്രിച്ച-കാനഡ-ഗ്രാന്‍പ്രി}

\newpage
