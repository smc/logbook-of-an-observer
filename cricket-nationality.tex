\secstar{ക്രിക്കറ്റ്, ദേശീയത, പണം - ചില നിരീക്ഷണങ്ങള്‍}
\vskip 2pt

ക്ഷമിക്കണം ബാബുരാജ് ഇവിടെയെഴുതിയ\footnote{\url{http://puthunireekshanam.blogspot.com/2008/05/blog-post.html}} 
ലേഖനത്തിന് മറുപടിയായി എഴുതിത്തുടങ്ങിയതാണ്. എഴുതിയെഴുതി, ലേഖനത്തിന്റെ വിഷയത്തില്‍ നിന്നും കുറച്ചകന്നു 
പോയി എന്നു തോന്നിയതിനാല്‍ ഇവിടെക്കുറിക്കുന്നു. ക്രിക്കറ്റിനെക്കുറിച്ചുള്ള ചില നിരീക്ഷണങ്ങള്‍.

കെറി പാര്‍ക്കര്‍ ക്രിക്കറ്റിലേക്ക് നിറവും പണവും കൊണ്ടുവന്നപ്പോള്‍ തുടങ്ങിയ മാറ്റങ്ങളുടെ വികാസമാണു് 20-20യും, 
ലോകത്തിന്റെ വേഗമനുസരിച്ച് കളിയുടെ വേഗം വര്‍ദ്ധിപ്പിച്ചുകൊണ്ടുള്ള പരീക്ഷണം. വാരാന്ത്യങ്ങളില്‍ നടക്കുന്ന 
രണ്ടുമണിക്കൂര്‍ പ്രീമിയര്‍ ലീഗ് പോരാട്ടങ്ങളും, മൂന്നു മണിക്കൂറിലേറെ നീളാത്ത വേഗത്തിന്റെ പോരാട്ടങ്ങളും, 
രാത്രിയിലെത്തുന്ന എന്‍.ബി.എ. യുദ്ധങ്ങളും, ദിനം മുഴുവന്‍ നീളുന്ന ക്രിക്കറ്റിന് സ്വീകരണമുറിയില്‍ വെല്ലുവിളിയുയര്‍ത്തുന്നു 
എന്ന തിരിച്ചറിഞ്ഞതിന്റെയും പുതിയ വിപണി സാദ്ധ്യതകളുടെയും ആകെത്തുകയാണ് 20-20 ക്രിക്കറ്റ്. 
ദേശീയതയും പണത്തിനുള്ള ഉപകരണം മാത്രമായ മേധാവികള്‍ക്ക് ഈ പുതിയ രീതി ഒരു വലിയ ലോട്ടറിയായിരുന്നു.

പിന്നെ, വമ്പന്‍ ക്ലബ്ബുകളും മറ്റും ഫുട്‌ബോള്‍ നിലങ്ങള്‍ വാഴുമ്പോള്‍ ക്രിക്കറ്റില്‍ അതു പാടില്ലെന്ന നിലപാടിന് 
പ്രധാനകാരണം ഒരു വലിയ തെറ്റിദ്ധാരണയാണ്. ഇന്ത്യയുടെ പേരില്‍ കളിക്കാനിറങ്ങുന്നവര്‍ പ്രധിനിധീകരിക്കുന്നത് 
യാതൊരു പൊതുബാദ്ധ്യതയുമില്ലാത്ത ഒരു ക്ലബ്ബ് കൂട്ടായ്മയെയാണെന്നുള്ളത് ഭൂരിഭാഗത്തിനും അറിയില്ല. ``ടീം ഇന്ത്യ`` എന്നത് ''ടീം ബി സി സി ഐ`` 
മാത്രമാണെന്ന് സുപ്രീം കോടതി വരെ വ്യക്തമാക്കിയതാണ്(ക്ഷമിക്കണം, ഇവിടെ ചെറിയൊരു പിശകുപറ്റിയോ എന്നൊരു സംശയം, 
ബോര്‍ഡ് സുപ്രീം കോടതിയില്‍ അങ്ങനെ വാദിക്കുകയും, സുപ്രീം കോടതി ശരി വയ്ക്കുകയും ചെയ്തു എന്നാണ് തോന്നുന്നത്. 
ഇതും\footnote{\url{http://www.thehindubusinessline.com/2004/10/01/stories/2004100103330400.htm}} 
ഇതും\footnote{\url{http://www.hinduonnet.com/thehindu/2005/02/03/stories/2005020308411200.htm}} 
വായിച്ചിട്ട് എനിക്കങ്ങനെയാണു തോന്നിയത്.). ബി സി സി ഐ എന്ന ക്ലബ്ബിന്റെ ടീം ദേശീയ ടീമാണെന്നുള്ള 
തെറ്റിദ്ധാരണയാണ് പുതിയ 20-20 മാമാങ്കങ്ങള്‍ ദേശീയതയുടെ പേരില്‍ എതിര്‍ക്കുന്നവര്‍ക്കുള്ളത്.

ആഗോളതലത്തില്‍ത്തന്നെ വെറുമൊരു സാമ്പത്തിക കൂട്ടായ്മയായോ മറ്റോ ക്രിക്കറ്റിനെ കാണാവുന്നതാണ്. 
ഐ സി സി അടക്കം എല്ലാ ബോര്‍ഡുകളുടെയും പ്രധാനലക്ഷ്യം സാമ്പത്തികമാണ്. മാത്രമല്ല, ഒരു ക്രിക്കറ്റ് ബോഡിയും 
കൃത്യമായി ഒരു രാഷ്ട്രത്തെ പ്രധിനിധീകരിക്കുന്നില്ല. ലോകത്തിലെ മുഴുവന്‍ ക്രിക്കറ്റ് മാമാങ്കങ്ങളും സത്യത്തില്‍ കുറെ 
ക്ലബ്ബുകളുടെ ഏറ്റുമുട്ടലുകള്‍ മാത്രമാണ്. പിന്നെ, ദേശീയത നല്ലവണ്ണം വിറ്റഴിയുന്ന ഒരുല്‍പ്പന്നമായതുകൊണ്ട് വ്യാപകമായി 
അതുപയോഗിക്കുന്നെന്നുമാത്രം. ദേശീയതയെ ചുരുങ്ങിയ നിലയില്‍ ഉപയോഗിച്ച് ക്രിക്കറ്റ് അതിന്റെ യഥാര്‍ത്ഥ മുഖം 
വെളിവാക്കമ്പോള്‍ അത് പ്രോത്സാഹിപ്പിക്കപ്പെടണം. മുഖംമൂടികള്‍ വലിച്ചെറിയാനും പരീക്ഷണങ്ങള്‍ നടത്താനും 
തയ്യാറായാല്‍ ദേശീയതയുടെ പേരില്‍ കബളിപ്പിക്കപ്പെടുന്ന ജനതയെ ഓര്‍ത്തെങ്കിലും നമ്മള്‍ സത്യങ്ങള്‍ അംഗീകരിക്കാന്‍ 
തയ്യാറാവണം.

ഇന്ത്യക്കുവേണ്ടി ക്രിക്കറ്റ് കളിക്കുന്നവന്‍ സത്യത്തില്‍ പ്രീമിയര്‍ ലീഗില്‍ ഒരു ടീമിനു വേണ്ടി കളിക്കുന്നവരില്‍ നിന്നും 
വ്യത്യസ്തനൊന്നുമല്ല. താന്‍ കരാറൊപ്പിട്ടിട്ടുള്ള ബോര്‍ഡിനേയെ പ്രധിനിധീകരിക്കൂ എന്ന നിബന്ധനയുണ്ടെന്നതൊഴിച്ചാല്‍. 
പിന്നെ, പണത്തിനും പകരം കളിക്കാരെ കൈമാറ്റം ചെയ്യുന്ന രീതി ഐ സി സി നിയമവിധേയമാക്കിയിട്ടുമില്ല. 
അത് ദേശീയത എന്ന ഉല്‍പ്പന്നത്തെ സംരക്ഷിക്കാന്‍ വേണ്ടിയാണ്. ക്ലബ്ബുകള്‍ തമ്മിലുള്ള പോരാട്ടങ്ങളെ 
ദേശീയപോരാട്ടങ്ങളായി വ്യവസ്ഥമാറ്റാതെ യഥാര്‍ത്ഥത്തില്‍ അവതരിപ്പിക്കുമ്പോള്‍ അതിനെ ദേശീയ ബോധമുള്ളവര്‍ 
പിന്തുണക്കുകയാണ് വേണ്ടതെന്നാണെന്റെ അഭിപ്രായം.

\subsection*{പ്രതികരണങ്ങള്‍}
\begin{enumerate}
 \item{റോബി}

ഈ കുറിപ്പിനു നന്ദി. 'ടീം ഇന്ത്യ' ഇന്ത്യന്‍ ടീമാണെന്നൊരു തെറ്റിദ്ധാരണ എനിക്കുമുണ്ടായിരുന്നു. 
ആ സുപ്രീം കോടതി പരാമര്‍ശത്തിന്റെ ലിങ്ക് വല്ലതുമുണ്ടോ? ഇനി ഇക്കാര്യം ആരോടെങ്കിലും 
പറയേണ്ടി വന്നാല്‍ വസ്തുനിഷ്ഠമായി പറയാമല്ലോ.

 \item{ദസ്തക്കിര്‍}

ഓഫിന് മാപ്പ്: അരുന്ധതി റോയുടെ പ്രസിദ്ധമായ 'come september speech'ല്‍ നമ്മുടെയൊക്കെ കപട 
ദേശീയതയെക്കുറിച്ചുള്ള ഒരു പ്രസ്ഥാവന ഇവിടെ ഉദ്ധരിക്കുന്നു. "An 'anti-national' is a person who is against his/her 
own nation and, by inference, is pro some other one. But it isn't necessary to be 'anti-national' 
to be deeply suspicious of all nationalism, to be anti-nationalism. Nationalism of one kind or 
another was the cause of most of the genocide of the twentieth century. 
Flags are bits of coloured cloth that governments use first to shrink-wrap peoples' 
minds and then as ceremonial shrouds to bury the dead. When independent, thinking people 
(and here I do not include the corporate media) begin to rally under flags, when writers, 
painters, musicians, film-makers suspend their judgment and blindly yoke their art to the service of the 'Nation', 
it's time for all of us to sit up and worry. In India we saw it happen soon after the Nuclear tests in 
1998 and during the Kargil War against Pakistan in 1999. In the United States we saw it during 
the Gulf War and we see it now, during the 'War against Terror'. 
That blizzard of Made-in-China American flags."

 \item{Don (ഡോണ്‍)}

അത് ശരി. പക്ഷെ ഭാരത്, ഇന്ത്യന്‍, ഓഫ് ഇന്ത്യ എന്നീ പേരു പൊതുമേഖലാ സ്ഥാപനങ്ങള്‍ മാത്രമെ ഉപയോഗിക്കാവൂ 
എന്നൊരു നിയമമില്ലേ? അപ്പോള്‍ BCCI എന്ന പേരു നിയമ വിരുദ്ധമല്ലേ?
\\
\\
 \item{ബാബുരാജ് ഭഗവതി}

ശ്രീ ജിനീഷ് കെ.ജെ., താങ്കള്‍ ആ ആശയത്തെ വികസിപ്പിച്ചിരിക്കുന്നു. പുതിയ വിവരങ്ങള്‍ക്ക് നന്ദി. താങ്കളുടെ വാദം ഈ 
വിഷയത്തെ പുതിയ ഉയരത്തിലെത്തിച്ചു. അഴീക്കോടിന്റെ പ്രസ്താവനയിലെ മണ്ടത്തരം കൂടുതല്‍ വെളിവാക്കപ്പെടുകയും 
ചെയ്തിരിക്കുന്നു. ലേഖനം തയ്യാറാക്കിയപ്പോഴും ബിസിസിഐ കുറിച്ച് മറിച്ച് ഞാന്‍ ചിന്തിച്ചിരുന്നില്ല. സുപ്രീം കോടതി വിധി 
നെറ്റില്‍ ലഭ്യമാണോ?

 \item{jinsbond007}

ക്ഷമിക്കണം റോബി, ബാബുരാജ്, എന്റെ കയ്യില്‍ ലിങ്കൊന്നുമില്ല. പക്ഷെ ആ പരാമര്‍ശത്തിനു ശേഷമാണ് തോല്‍ക്കുന്ന ഇന്ത്യന്‍ ടീം, 
മാദ്ധ്യമങ്ങള്‍ക്ക് ബി സി സി ഐ യുടെ പതിനൊന്നു കളിക്കാര്‍ മാത്രമായത്. പിന്നെ, BCCI എന്നതിന്റെ പൂര്‍ണ്ണരൂപം, 
Board of Control for Cricket in India എന്നാണ്.

 \item{jinsbond007}

സുപ്രീം കോടതി പരാമര്‍ശത്തിന്റെ കാര്യത്തില്‍ ഞാന്‍ കരുതിയത് മുഴുവന്‍ ശരിയല്ല എന്നു തോന്നുന്നു. ബ്രാക്കറ്റില്‍ 
ഡിസ്‌ക്ലൈമറും ലിങ്കും ചേര്‍ത്തിട്ടുണ്ട്.
    
\end{enumerate}

\newpage
