\secstar{ആസ്റ്റെറിക്സിന്റെ സാഹസിക കഥകള്‍}
\vskip 2pt

ലോ­ക­മെ­മ്പാ­ടു­മു­ള്ള കു­ട്ടി­ക­ളു­ടെ­യും മു­തിര്‍­ന്ന­വ­രു­ടെ­യും പ്രീ­തി പി­ടി­ച്ചു പറ്റി­യ­വ­യാ­ണു് "ആ­സ്റ്റെ­റി­ക്സി­ന്റെ സാ­ഹ­സിക 
കഥ­കള്‍ (Adventures of Asterix the Gaul)".  ഫ്രാ­ങ്കോ-ബെല്‍­ജി­യന്‍ പാ­ര­മ്പ­ര്യ­ത്തി­ലു­ള്ള കോ­മി­ക്കു­ക­ളില്‍ 
(ഇം­ഗ്ലീ­ഷു­കാര്‍­ക്കി­തു ഗ്രാ­ഫി­ക് നോ­വ­ലു­ക­ളാ­ണു്) എറ്റ­വും ജന­പ്രീ­തി­യു­ള്ള­വ­യി­ലൊ­ന്നാ­ണി­തു്. വില്‍­പ്പ­ന­ക്ക­ണ­ക്കു­കള്‍ പ്ര­കാ­രം 
സൃ­ഷ്ടാ­ക്ക­ളായ റെ­നെ ഗോ­സി­ന്നി­യും ആല്‍­ബെര്‍­ട്ട് ഉദേര്‍­സോ­യും ഫ്രാന്‍­സി­നു പു­റ­ത്തു ഏറ്റ­വും ജന­പ്രീ­തി­യു­ള്ള ഫ്ര­ഞ്ച് 
എഴു­ത്തു­കാ­രാ­ണെ­ന്നു പറ­യു­മ്പോള്‍­ത്ത­ന്നെ പ്ര­ചാ­ര­ത്തി­ന്റെ വലി­പ്പം ഊഹി­ക്കാ­മ­ല്ലോ.

ഇ­തു­വ­രെ 34 പു­സ്ത­ക­ങ്ങള്‍ ആസ്റ്റെ­റി­ക്സി­ന്റെ കഥ­ക­ളു­മാ­യി പു­റ­ത്തി­റ­ങ്ങി­യി­ട്ടു­ണ്ട്. 1959 ല്‍ പൈ­ലെ­റ്റെ മാ­ഗ­സി­നില്‍ 
പര­മ്പ­ര­യാ­യി പ്ര­സി­ദ്ധീ­ക­രി­ച്ചു തു­ട­ങ്ങിയ ആസ്റ്റെ­റി­ക്സ്, പു­സ്ത­ക­മാ­യി ആദ്യം പു­റ­ത്തി­റ­ങ്ങു­ന്ന­തു് 1961­ലാ­ണു്. അതി­നു 
ശേ­ഷം 77ല്‍ ഗോ­സി­ന്നി­യു­ടെ മര­ണം വരെ ഏതാ­ണ്ടെ­ല്ലാ വര്‍­ഷ­വും ഒന്നെ­ന്ന കണ­ക്കില്‍ പു­സ്ത­ക­ങ്ങള്‍ ഇറ­ങ്ങി­യി­ട്ടു­ണ്ട്. 
അതു­വ­രെ കഥ­യും ആശ­യ­വും ഗോ­സി­ന്നി­യു­ടെ വക­യും വര ഉദേര്‍­സോ­യു­ടേ­തു­മാ­യി­രു­ന്നു. ഗോ­സി­ന്നി­യു­ടെ മര­ണ­ശേ­ഷം 
പത്തു പു­സ്ത­ക­ങ്ങള്‍ ഉദേര്‍­സോ തന്നെ വര­യ്ക്കു­ക­യും എഴു­തു­ക­യും ചെ­യ്ത് പ്ര­സി­ദ്ധീ­ക­രി­ച്ചി­ട്ടു­ണ്ടു്.

ആ­സ്റ്റെ­റി­ക്സ് കോ­മി­ക്കി­ന്റെ പശ്ചാ­ത്ത­ലം ഫ്ര­ഞ്ച് പ്ര­വ­ശ്യ­യായ അര്‍­മോ­റി­ക്ക­യി­ലു­ള്ള പേ­രി­ല്ലാ­ത്ത ഒരു ഗ്രാ­മ­മാ­ണു്. 
റോ­മന്‍ അധി­നി­വേ­ശ­ത്തെ വെര്‍­സിന്‍­ഗെ­റ്റോ­റി­ക്സി­ന്റെ കീ­ഴ­ട­ങ്ങ­ലി­നു ശേ­ഷ­വും പ്ര­തി­രോ­ധി­ക്കു­ന്ന­വ­രാ­ണു് ഗ്രാ­മ­വാ­സി­കള്‍ 
(ജൂ­ലി­യ­സ് സീ­സ­റെ എതിര്‍­ക്കു­ക­യും അവ­സാ­നം കീ­ഴ­ട­ക്ക­പ്പെ­ട്ട് റോ­മില്‍ വധ­ശി­ക്ഷ­യ്ക്കു വി­ധേ­യ­നാ­ക്ക­പ്പെ­ടു­ക­യും ചെ­യ്ത 
ഒരു യഥാര്‍­ത്ഥ ചരി­ത്ര കഥാ­പാ­ത്ര­മാ­ണു് വെര്‍സിന്‍ഗെറ്റോറിക്സ്). അതി­ന­വ­രെ സഹാ­യി­ക്കു­ന്ന­താ­വ­ട്ടെ ഗ്രാമ ­
മ­ന്ത്ര­വാ­ദി­ (druid) ആയ ഗെ­റ്റാ­ഫി­ക്സി­ന്റെ അമാ­നു­ഷിക ശക്തി നല്‍­കു­ന്ന മരു­ന്നും­.

­നാ­യ­ക­നായ ആ­സ്റ്റെ­റി­ക്സ് കര്‍­മ്മം കൊ­ണ്ടു് ഒരു യോ­ദ്ധാ­വാ­ണെ­ങ്കി­ലും ഒരു യോ­ദ്ധാ­വി­ന്റെ ശരീ­ര­മൊ­ന്നു­മ­ല്ല അദ്ദേ­ഹ­ത്തി­നു്. 
കു­റി­യ­വ­നായ ആസ്റ്റെ­റി­ക്സ് പ്ര­ശ്ന­ങ്ങ­ളെ നേ­രി­ടു­ന്ന­തും പരി­ഹ­രി­ക്കു­ന്ന­തും ബു­ദ്ധി­യു­പ­യോ­ഗി­ച്ചാ­ണു്. ആസ്റ്റെ­റി­ക്സി­ന്റെ 
ശാ­രീ­രി­ക­മായ പോ­രാ­യ്മ­ക­ളെ പരി­ഹ­രി­ക്കു­ന്ന­തു് ഉറ്റ കൂ­ട്ടു­കാ­ര­നും ഭീ­മാ­കാ­ര­നും ചെ­റു­പ്പ­ത്തി­ലെ അത്ഭുത മരു­ന്നി­ന്റെ 
കല­ത്തില്‍ വീ­ണ­തു കൊ­ണ്ടു് സ്ഥി­ര­മാ­യി അതി­ന്റെ ശക്തി­യു­ള്ള­വ­നു­മായ ഒബ്ലി­ക്സാ­ണു്. ഗ്രാ­മ­ത്തി­ലെ ഏക മെന്‍­ഹിര്‍ 
ക്വാ­റി­യു­ട­മ­യാ­ണു് ഒബ്ലി­ക്സ്.

%Oblix.jpg

­ശ­ക്ത­നും കൂ­ട്ടു­കാ­ര­നു വേ­ണ്ടി ജീ­വന്‍ കള­യു­ന്ന­വ­നു­മാ­ണെ­ങ്കി­ലും ഒ­ബ്ലി­ക്സ് കാ­ര്യ­ങ്ങ­ളെ­പ്പ­റ്റി ആലോ­ച­ന­യൊ­ന്നു­മി­ല്ലാ­ത്ത 
കൂ­ട്ട­ത്തി­ലാ­ണു്. ദി­വ­സ­വും രണ്ടു­മൂ­ന്നു പൊ­രി­ച്ച കാ­ട്ടു­പ­ന്നി­ക­ളും (wild boar) കൈ­ത്ത­രി­പ്പു തീര്‍­ക്കാന്‍ ഇട­യ്ക്കു ചില റോ­മന്‍ 
പട്ടാ­ള­ക്കാ­രെ­യും കി­ട്ടി­യാല്‍ മൂ­പ്പര്‍ ഹാ­പ്പി­യാ­ണു്. പില്‍­ക്കാല കോ­മി­ക്കു­ക­ളില്‍ അര്‍­മോ­റി­ക്ക­യി­ലെ നാ­ലു ക്യാ­മ്പു­ക­ളി­ലെ 
നി­യ­മ­നം റോ­മന്‍ പട്ടാ­ള­ത്തില്‍ ഒരു ശി­ക്ഷ­യോ­ള­മെ­ത്തിയ കാ­ല­ത്തു്, "ന­ല്ല പട്ടാ­ള­ക്കാ­രെ ഇങ്ങോ­ട്ട­യ­ക്കാ­തെ നമ്മ­ളെ 
പ്ര­കോ­പി­പ്പി­ച്ച് ഇവി­ടു­ത്തെ സമാ­ധാ­നാ­ന്ത­രീ­ക്ഷം തകര്‍­ക്കാന്‍ ശ്ര­മി­ക്കു­ക­യാ­ണു സീ­സ­റും സെ­ന­റ്റും" എന്നു വരെ ഒരി­ട­ത്തു് 
ഒബ്ലി­ക്സ് പരാ­തി പറ­യു­ന്നു­ണ്ടു്.

ആ­കെ ഗോ­ളില്‍ കീ­ഴ­ട­ങ്ങാ­തെ നില്‍­ക്കു­ന്ന ഈ ഒരേ­യൊ­രു ഗ്രാ­മ­ത്തി­നു ­സീ­സര്‍ ഉപ­രോ­ധം തീര്‍­ത്താ­ണു് മട­ങ്ങി­യ­തു്. 
നാ­ലു സൈ­നിക ക്യാ­മ്പു­ക­ളാ­ണു് ഈ ചെ­റു കട­ലോര ഗ്രാ­മ­ത്തി­നു ഉപ­രോ­ധം തീര്‍­ത്തി­രു­ന്ന­തു്. അക്വേ­റി­യം, ടോ­ടോ­റം, 
ലൌ­ഡാ­നും, കോ­മ്പെന്‍­ഡി­യം ക്യാ­മ്പു­ക­ളാ­ണു് അവ.

ആ­ദ്യ കോ­മി­ക്കു­ക­ളില്‍ വള­രെ സീ­രി­യ­സ്സ് ആയി­ത്ത­ന്നെ പ്ര­തി­രോ­ധ­ത്തി­ന്റെ അവ­സാ­ന­ക­ണ്ണി­യെ ഇല്ലാ­താ­ക്കാന്‍ 
ശ്ര­മി­ക്കു­ന്ന സൈ­നിക മേ­ധാ­വി­ക­ളെ­യും സൈ­നി­ക­രെ­യു­മാ­ണു് നമ്മള്‍ കാ­ണു­ന്ന­തു്. എന്നാല്‍ കാ­ലം ചെ­ല്ലും തോ­റും 
സീ­സര്‍ വരെ പല കാ­ര്യ­ങ്ങ­ളി­ലും ഗ്രാ­മ­വാ­സി­ക­ളോ­ടു സന്ധി ചെ­യ്യു­ക­യും സഹാ­യം സ്വീ­ക­രി­ക്കു­ക­യും (Asterix the
Legionary, Asterix and Son തു­ട­ങ്ങി­യവ ഉദാ­ഹ­ര­ണം) ചെ­യ്യു­മ്പോള്‍ സൈ­നി­ക­രു­ടെ മനോ­ഭാ­വ­വും മാ­റു­ന്നു­ണ്ടു്.

­പു­തിയ ദേ­ശ­ങ്ങള്‍ കീ­ഴ­ട­ക്കു­ന്ന സാ­മ്രാ­ജ്യ­ത്തി­നു് പഴയ അധി­നി­വേ­ശ­ങ്ങ­ളി­ലെ പ്ര­ശ്ന­ങ്ങള്‍ താ­ര­മേ­ന്യ നി­സ്സാ­ര­വും 
ആഭ്യ­ന്ത­ര­വു­മാ­കു­ന്ന­തി­ന്റെ പ്ര­ത്യ­ക്ഷ ഉദാ­ഹ­ര­ണ­മാ­യാ­ണു് ഇതു എടു­ത്തു കാ­ണി­ച്ചി­രി­ക്കു­ന്ന­തു്. ഒരി­ട­ത്തു സീ­സര്‍ തന്നെ 
കു­ഴി­മ­ടി­യ­നും മദ്യ­പ­നു­മായ ഒരു പട­യാ­ളി­യെ ഒരു പാ­ഠം പഠി­പ്പി­യ്ക്കാ­നാ­യി പി­രി­ഞ്ഞു പോ­കല്‍ ബോ­ണ­സ്സാ­യി ഈ ഗ്രാ­മം 
എഴു­തി­ക്കൊ­ടു­ക്കു­ന്നു­ണ്ടു്. ഇത്ത­ര­ത്തില്‍ നി­സ്സാ­ര­രും സാ­മ്രാ­ജ്യ­ത്തി­നു അഭി­മാ­ന­ക്ഷ­ത­മ­ല്ലാ­തെ വലിയ ദോ­ഷ­മി­ല്ലാ­ത്ത­തു­മായ 
ചെ­റു­ത്തു നില്‍­പ്പു­ക­ളോ­ടു­ള്ള അധി­കാ­രി­ക­ളു­ടെ മനോ­ഭാ­വ­ത്തെ വള­രെ വ്യ­ക്ത­മാ­യും സര­സ­മാ­യും ചി­ത്രീ­ക­രി­ച്ചി­ട്ടു­ണ്ടു് 
ഗോ­സി­ന്നി­യും ഉദേര്‍­സോ­യും­.

50 വര്‍­ഷ­ത്തി­നു­ള്ളില്‍ ആസ്റ്റെ­റി­ക്സും ഒബ്ലി­ക്സു­മ­ട­ക്കം കോ­മി­ക്കി­ലെ എല്ലാ കഥാ­പാ­ത്ര­ങ്ങ­ളും വര­യി­ലും ആശ­യ­ത്തി­ലും സ്വ­ന്തം 
വ്യ­ക്തി­ത്വ­വും വ്യ­ക്ത­ത­യും നേ­ടി­യെ­ടു­ത്തു­വെ­ന്നു പറ­യാം. ആസ്റ്റെ­റി­ക്സും ഒബ്ലി­ക്സും ഗെ­റ്റാ­ഫി­ക്സു­മ­ല്ലാ­തെ ഒരു പി­ടി കഥാ­പാ­ത്ര­ങ്ങള്‍ 
വേ­റെ­യു­മു­ണ്ടു് കോ­മി­ക്കില്‍. ഗ്രാ­മ­മു­ഖ്യന്‍ വൈ­റ്റല്‍ സ്റ്റാ­റ്റി­സ്റ്റി­ക്സും, ഭാ­ര്യ­യും ലു­റ്റേ­ഷ്യ(­പാ­രീ­സ­ട­ങ്ങു­ന്ന പ്ര­വ­ശ്യ)­ക്കാ­രി­യു­മായ 
ഇമ്പെ­ടി­മെ­ന്റ­യും, മീന്‍ കച്ച­വ­ട­ക്കാ­രന്‍ അണ്‍­ഹൈ­ജെ­നി­ക്സും ഭാ­ര്യ ബാ­ക്റ്റീ­ര­യ­യും, കൊ­ല്ലന്‍ ഫു­ള്ളി­ഓ­ട്ടോ­മാ­റ്റി­ക്സും ഭാ­ര്യ­യും, 
ഗ്രാ­മ­ത്തി­ലെ പ്ര­ധാ­ന­വ­യ­സ്സ­നായ ജെ­റി­യാ­ട്രി­ക്സും അയാ­ളു­ടെ ചെ­റു­പ്പ­ക്കാ­രി­യായ ഭാ­ര്യ­യും, ഗ്രാ­മ­ത്തി­ന്റെ ഗാ­യ­കന്‍ 
കാ­ക­ഫോ­ണി­ക്സ്, എന്നി­വ­രെ കൂ­ടാ­തെ ജൂ­ലി­യ­സ് സീ­സ­റും ഒരു പ്ര­ധാന കഥാ­പാ­ത്ര­മാ­ണു്.

%ആ­സ്റ്റെ­റി­ക്സ് കഥാ­പാ­ത്ര­ങ്ങള്‍

­ലു­റ്റേ­ഷ്യ­യില്‍ വെ­ച്ച് ഒബ്ലി­ക്സി­ന്റെ കൂ­ടെ­ക്കു­ടിയ ഡോ­ഗ്മാ­റ്റി­ക്സ് എന്ന വളര്‍­ത്തു­നാ­യ­യും എന്നും ഗ്രാ­മ­ത്തെ കൂ­കി­യു­ണര്‍­ത്തു­ന്ന 
പൂ­വന്‍­കോ­ഴി­യും ആവര്‍­ത്തി­യ്ക്കു­ന്ന മൃഗ കഥാ­പാ­ത്ര­ങ്ങ­ളാ­ണു്. മാ­ത്ര­മ­ല്ല സ്ഥി­ര­മാ­യി ഫ്രേ­മു­ക­ളില്‍ ആവര്‍­ത്തി­യ്ക്ക­പ്പെ­ടു­ന്ന 
സാ­ന്നി­ധ്യ­മാ­ണു് കോ­ഴി­കള്‍. ഗ്രാ­മ­ത്തി­ലെ എന്തു പ്ര­ധാന സം­ഭ­വ­ത്തി­ന്റെ ഫ്രേ­മി­ലും ഒരു കോ­ഴി­യെ­യെ­ങ്കി­ലും ഉദേര്‍­സോ 
ഉള്‍­പ്പെ­ടു­ത്തി­യി­ട്ടു­ണ്ടാ­കും. ഗോ­ളി­ന്റെ ചി­ഹ്ന­മാ­ണു കോ­ഴി എന്ന­തു മാ­ത്ര­മാ­ണോ ഇതി­നു കാ­ര­ണം? ഉദേര്‍­സോ­യു­ടെ തന്നെ 
വാ­ക്കു­ക­ളില്‍ കോ­ഴി­യെ വര­യ്ക്കാന്‍ തനി­യ്ക്കി­ഷ്ട­മാ­യ­തു കൊ­ണ്ടാ­ണെ­ന്നൊ­രു ഒഴു­ക്കന്‍ വി­ശ­ദീ­ക­ര­ണ­മാ­ണു നമു­ക്കു കി­ട്ടി­യി­ട്ടു­ള്ള­തു്.

ആ­ദ്യ­ക­ഥ­ക­ളില്‍ അത്ഭു­ത­മ­രു­ന്നി­ന്റെ സഹാ­യ­ത്തോ­ടെ ലക്ഷ്യം സാ­ധി­ച്ചു വരു­ന്ന ആസ്റ്റെ­റി­ക്സും ഒബ്ലി­ക്സും എന്ന 
ഇതി­വൃ­ത്ത­ത്തില്‍­ത്ത­ന്നെ കി­ട­ന്നു കറ­ങ്ങിയ കഥ­കള്‍ പി­ന്നീ­ടു വ്യ­ത്യ­സ്ത ഇതി­വൃ­ത്ത­ങ്ങ­ളും ആഖ്യാ­ന­ങ്ങ­ളും തേ­ടി­ത്തു­ട­ങ്ങി. 
'ആ­സ്റ്റെ­റി­ക്സ് ആന്‍­ഡ് ദ ബി­ഗ് ഫൈ­റ്റി­'­ലാ­ണു് അത്ഭുത മരു­ന്നി­ല്ലാ­തെ­ത­ന്നെ ലക്ഷ്യം കാ­ണു­ന്ന രീ­തി­യില്‍ ആദ്യം 
കഥ­യ­വ­സാ­നി­ക്കു­ന്ന­തു്. പി­ന്നീ­ട് "ആ­സ്റ്റെ­റി­ക്സ് ഇന്‍ ബ്രി­ട്ട­ണി­ലും" ഇതാ­വര്‍­ത്തി­യ്ക്ക­പ്പെ­ട്ടു­.

­പി­ന്നീ­ടു പല കോ­മി­ക്കു­ക­ളി­ലും ലോ­ക­വി­ഷ­യ­ങ്ങ­ളും ആഖ്യാ­ന­ങ്ങ­ളു­ടെ ഭാ­ഗ­മാ­യി. അത്‌­ല­റ്റി­ക്സില്‍ ഉത്തേ­ജക 
മരു­ന്നു­പ­യോ­ഗം വ്യാ­പ­ക­മാ­യ­തി­നെ പു­രാ­തന ഒളി­മ്പി­ക്സി­നെ കൂ­ട്ടു പി­ടി­ച്ചാ­ണു് ചി­ത്രീ­ക­രി­ച്ചി­രി­ക്കു­ന്ന­തു്. കമ്പോ­ള­ത്തി­നേ­യും 
ബൂര്‍­ഷ്വാ­സി­യെ­യും തൊ­ഴി­ലാ­ളി­വര്‍­ഗ്ഗ­സ­മ­ര­ത്തെ­യും ആഗോ­ള­വ­ത്ക­ര­ണ­ത്തേ­യും വി­വി­ധ­ങ്ങ­ളായ സാ­മ്പ­ത്തിക 
ശാ­സ്ത്ര­പ­ഠ­ന­ത്തി­ന്റെ മെ­ക്ക­ക­ളെ­യും പ്ര­തി­നി­ധീ­ക­രി­ച്ചു "ഒ­ബ്ലി­ക്സ് ആന്‍­ഡ് കോ­"­യി­ലെ അന്താ­രാ­ഷ്ട്ര മെന്‍­ഹിര്‍ മാര്‍­ക്ക­റ്റും 
സ്വ­ദേ­ശി മെന്‍­ഹിര്‍ നിര്‍­മ്മാ­താ­ക്ക­ളു­ടെ റോ­മന്‍ റോ­ഡ് ഉപ­രോ­ധ­വും ലാ­റ്റിന്‍ സ്കൂള്‍ ഓഫ് ഇക്ക­ണോ­മി­ക്സില്‍ പഠി­ച്ച കയ­സ് 
പ്ര­പോ­സ്റ്റ­റ­സും ഒക്കെ. ഴാ­ക് ഷി­റാ­ക്കി­ന്റെ നേ­തൃ­ത്വ­ത്തില്‍ ഫ്രാന്‍­സില്‍ സ്ഥാ­ന­മേ­റ്റേ­ടു­ത്ത സര്‍­ക്കാ­രി­ന്റെ 
നട­പ­ടി­കള്‍­ക്കെ­തി­രെ­യു­ള്ള ഒരു വി­മര്‍­ശ­ന­മാ­യി­രു­ന്നു ഇതെ­ന്നു ചില വ്യാ­ഖ്യാ­താ­ക്ക­ളു­ടെ പക്ഷം­.

­ന­ഗ­ര­വ­ത്ക­ര­ണ­ത്തി­ലൂ­ടെ ഗ്രാ­മ­വാ­സി­ക­ളെ നശി­പ്പി­യ്ക്കാന്‍ നോ­ക്കു­ന്ന­തും, മൂ­ല­ധ­നം ജന­ങ്ങള്‍­ക്കി­ട­യി­ലെ സ്വാ­ഭാ­വിക 
സു­ഹൃ­ത്ബ­ന്ധ­ങ്ങ­ളെ ഉല­യ്ക്കു­ന്ന­തും അതീ­വ­സ­ര­സ­മാ­യാ­ണെ­ങ്കി­ലും ചി­ത്രീ­ക­രി­ച്ച ഗോ­സി­ന്നി­യും ഉദേര്‍­സോ­യും 
വി­ക­സ­ന­ത്തില്‍ നഷ്ട­പ്പെ­ടു­ന്ന നന്മ­ക­ളെ­പ്പ­റ്റി ആകു­ല­രാ­യി­രു­ന്നു­വെ­ന്നു വ്യ­ക്തം. ഫ്രാന്‍­സില്‍ നല്ല വേ­രോ­ട്ട­മു­ള്ള 
ഫെ­മി­നി­സ­ത്തേ­യും വി­ഷ­യ­മാ­ക്കു­ന്നു­ണ്ടു് ഉദേര്‍­സോ­.

ആ­സ്റ്റെ­റി­ക്സ് കഥ­ക­ളില്‍ ഒരു­പാ­ടെ­ണ്ണം യാ­ത്ര­ക­ളാ­ണു്. അവ­യില്‍ മാ­ത്രം പ്ര­ത്യ­ക്ഷ­പ്പെ­ടു­ന്ന ചില കഥാ­പാ­ത്ര­ങ്ങ­ളു­മു­ണ്ടു്. 
അറ­ബ് വം­ശ­ജ­നായ നാ­വി­ക­നും കച്ച­വ­ട­ക്കാ­ര­നു­മായ എക്ക­ണോ­മി­ക്രൈ­സി­സ്, കടല്‍­ക്കൊ­ള്ള­ക്കാ­രു­ടെ സം­ഘം 
തു­ട­ങ്ങി­യ­വര്‍. ഇതില്‍ കടല്‍­ക്കൊ­ള്ള­ക്കാ­രു­ടെ സം­ഘ­ത്തി­ന്റെ പേ­ടി­സ്വ­പ്ന­മാ­യി­മാ­റു­ന്നു­ണ്ടു് ആസ്റ്റെ­റി­ക്സും ഒബ്ലി­ക്സും­.

ആ­ദ്യ­കാ­ല­ത്തു യാ­ത്ര­കള്‍ യൂ­റോ­പ്പി­ലെ വി­വി­ധ­ദേ­ശ­ങ്ങ­ളി­ലും (ബെല്‍­ജി­യം­,­ബ്രി­ട്ടന്‍,­സ്വി­റ്റ്സര്‍­ലാന്‍­ഡ്,­റോം­,), ഫ്രാന്‍­സി­ന്റെ 
വി­വിധ പ്ര­വി­ശ്യ­ക­ളി­ലും (കോര്‍­സി­ക്ക, ലു­റ്റേ­ഷ്യ, ബാന്‍­ക്വേ­യില്‍ മു­ഴു­വന്‍ ഫ്ര­ഞ്ച് പ്ര­വ­ശ്യ­ക­ളി­ലും പര്യ­ട­നം നട­ത്തു­ന്നു­ണ്ടു്) 
ഒതു­ങ്ങി­യി­രു­ന്നു. കടല്‍ മാര്‍­ഗ്ഗ­വും കര­മാര്‍­ഗ്ഗ­വും നട­ന്ന യാ­ത്ര­കള്‍ പല­തും മറ്റു­പ്ര­ദേ­ശ­ങ്ങ­ളി­ലെ 
അധി­നി­വേ­ശ­ക്കാര്‍­ക്കെ­തി­രെ സമാ­ന­മായ ചെ­റു­ത്തു നില്‍­പ്പു­ക­ളില്‍ പങ്കാ­ളി­ക­ളാ­വാ­നോ സഹാ­യ­ത്തി­നോ ആയി­രു­ന്നു­.

­പി­ന്നീ­ടു് വൈ­ക്കി­ങ്ങു­ക­ളു­ടെ പിന്‍­പ­റ്റി ഗ്രീന്‍­ലാന്‍­ഡി­ലും, അമേ­രി­ക്ക­യി­ലും, ഇന്ത്യ­യി­ലും, അറേ­ബ്യ­യി­ലും ഒക്കെ­യാ­യി യാ­ത്ര­കള്‍ വി­ക­സി­ക്കു­ന്നു­ണ്ടു്. ഇവി­ടെ­യൊ­ക്കെ അധി­നി­വേ­ശ­ത്തോ­ട­ല്ല പു­തിയ ജീ­വി­ത­രീ­തി­ക­ളോ­ടും വി­ല്ലന്‍­മാ­രോ­ടു­മാ­ണു് 
ആസ്റ്റെ­റി­ക്സി­നും ഒബ്ലി­ക്സി­നും എതി­രി­ടേ­ണ്ടി­വ­രു­ന്ന­തു്.

­ര­ണ്ടു ആഫ്രി­ക്കന്‍ യാ­ത്ര­ക­ളു­ള്ള­തു് ഒന്നു് ഈജി­പ്റ്റില്‍ ക്ലി­യോ­പാ­ട്ര­യ്ക്കു സീ­സ­റി­നു വേ­ണ്ടി കൊ­ട്ടാ­രം പണി­യാന്‍ 
സഹാ­യി­ക്കാ­നും മറ്റൊ­ന്നു് റോ­മന്‍ പട്ടാ­ള­ത്തി­ലേ­ക്കു ഡ്രാ­ഫ്റ്റ് ചെ­യ്യ­പ്പെ­ട്ട ട്രാ­ജി­ക്‌­ണോ­മി­ക്സി­നെ രക്ഷി­ക്കാന്‍ 
വേ­ണ്ടി­യു­ള്ള­തു­മാ­ണു്. രണ്ടാ­മ­ത്തെ കഥ­യില്‍ റോ­മന്‍ പട്ടാ­ള­ത്തി­ന്റെ അതി­രു കവി­ഞ്ഞ അച്ച­ട­ക്ക­ത്തെ 
അട­ച്ചാ­ക്ഷേ­പി­ക്കു­ന്നു­ണ്ടു്. അവ­സാ­നം സീ­സര്‍­ക്കു് പോം­പി­യു­ടെ മേല്‍ വി­ജ­യം നേ­ടി­ക്കൊ­ടു­ത്താ­ണു് ആസ്റ്റെ­റി­ക്സും ഒബ്ലി­ക്സും 
മട­ങ്ങു­ന്ന­തു്.

­വി­വി­ധ­രാ­ജ്യ­ങ്ങ­ളില്‍ അവി­ടു­ത്തു­കാ­രു­ടെ സ്വ­ഭാ­വ­ങ്ങ­ളും പ്ര­ത്യേ­ക­ത­ക­ളും മു­ത­ലാ­ക്കി­ക്കൊ­ണ്ടും എടു­ത്തു 
കാ­ണി­ച്ചു­കൊ­ണ്ടു­മാ­ണു് ഹാ­സ്യ­മു­ണ്ടാ­ക്കു­ന്ന­തു്. സ്വി­സ്സു­കാ­രു­ടെ സമാ­ധാ­ന­പ്രി­യ­ത­യും സമ­യ­നി­ഷ്ഠ­യും സ്വി­സ്സ് ബാ­ങ്കും 
വരെ പരാ­മര്‍­ശ­വി­ധേ­യ­മാ­കു­മ്പോള്‍ ലു­റ്റേ­ഷ്യ­യില്‍ തി­ര­ക്കും ബഹ­ള­ങ്ങ­ളും കോര്‍­സി­ക്ക­യില്‍ മടി­യും അവ­സാ­നി­ക്കാ­ത്ത 
കു­ടും­ബ­വ­ഴ­ക്കും മറ്റു­മാ­ണു് വി­ഷ­യം. ബ്രി­ട്ട­നി­ലെ­ത്തു­മ്പോള്‍ ഫു­ട്ബാള്‍ മു­തല്‍ തീന്‍­മേ­ശ­മ­ര്യാ­ദ­വ­രെ വി­ഷ­യ­മാ­കു­ന്നു­ണ്ടു്.

­യൂ­റോ­പ്പി­ലെ യാ­ത്ര­ക­ളി­ലെ നി­താ­ന്ത­സാ­ന്നി­ധ്യ­മാ­ണു് റോ­മന്‍ സാ­ങ്കേ­തി­ക­ത­യും റോ­ഡു­ക­ളും പശ്ചാ­ത്ത­ല­വി­ക­സ­ന­വും മറ്റും. 
സ്പീ­ഡ് കൂ­ടി­യ­തി­നു ഒരു യു­വാ­വി­ന്റെ രഥ­ത്തി­നു ഫൈ­ന­ടി­ക്കു­ന്ന ഓഫീ­സ­റും, മൊ­ബൈല്‍ രഥ­റി­പ്പ­യ­റും കു­തി­ര­യ്ക്കു പു­ല്ലും 
വൈ­ക്കോ­ലും നല്‍­കു­ന്ന ബങ്കും എല്ലാം യൂ­റോ­പ്യന്‍ ഹൈ­വേ­ക­ളേ­യും അവി­ടു­ത്തെ സം­ഭ­വ­ങ്ങ­ളേ­യും പു­രാ­ത­ന­കാ­ല­ത്തേ­യ്ക്കു 
പറി­ച്ചു നട്ട­തു­മാ­ത്ര­മാ­ണു്.

­ക­ഥ­ക­ളി­ലെ­ല്ലാം ആസ്റ്റെ­റി­ക്സാ­ണു് നാ­യ­ക­നെ­ങ്കി­ലും പഞ്ച് ഡയ­ലോ­ഗു­കള്‍ പല­തും ഒബ്ലി­ക്സി­ന്റേ­താ­ണു്. "ഈ റോ­മാ­ക്കാര്‍­ക്കു 
ഭ്രാ­ന്താ­ണു്" (these romans are crazy) എന്ന ഡയ­ലോ­ഗാ­ണു് ഇതില്‍ ഏറ്റ­വും പ്ര­സി­ദ്ധം. ഇതു പി­ന്നീ­ടു് മറ്റു പല 
ദേ­ശ­ക്കാ­രെ­യും പറ്റി­യും ഒബ്ലി­ക്സ് പറ­യു­ന്നു­ണ്ടു്. ആല്‍­പ്സ് കയ­റി­യി­റ­ങ്ങു­മ്പോള്‍ കള്ള­ടി­ച്ച് മത്താ­യി ഉറ­ങ്ങി­പ്പോയ 
ഒബ്ലി­ക്സി­നോ­ടു് ­സ്വി­റ്റ്സര്‍­ലാന്‍­ഡ് എങ്ങ­നെ­യു­ണ്ടാ­യി­രു­ന്നെ­ന്ന ചോ­ദ്യ­ത്തി­നു "ഫ്ലാ­റ്റ്" എന്നൊ­റ്റ­വാ­ക്കില്‍ ഉത്ത­രം 
നല്‍­കു­ന്നു­ണ്ടു് ഒബ്ലി­ക്സ്. അതു­പോ­ലെ ഒരു സ്ത്രീ ഗ്രാ­മ­ത്തി­ന്റെ ഗാ­യി­ക­യാ­യി കാ­ക­ഫോ­ണി­ക്സി­നു പക­രം എത്തി­യ­പ്പോള്‍ 
അതില്‍ അസ്വാ­ഭാ­വി­ക­ത­യൊ­ന്നും ഒബ്ലി­ക്സി­നി­ല്ല. അവര്‍ ആണു­ങ്ങ­ളു­ടെ മാ­തി­രി ബീ­ച്ച­സ്സ് ധരി­ച്ചി­രു­ന്ന­തും ഒബ്ലി­ക്സി­നു 
പ്ര­ശ്ന­മ­ല്ല, പക­രം കു­റു­കെ­യു­ള്ള വര­ക­ള­ല്ല നീ­ളന്‍ വര­ക­ളാ­ണു് തടി­കു­റ­ച്ചു കാ­ണി­ക്കു­ന്ന­തെ­ന്നു ഇവര്‍­ക്ക­റി­യി­ല്ലേ എന്നു 
പറ­ഞ്ഞു ആര്‍­ത്തു ചി­രി­ക്കു­ക­യാ­ണു് മൂ­പ്പര്‍. ഇത്ത­ര­ത്തില്‍ സന്ദര്‍­ഭം വി­ശ­ക­ല­നം ചെ­യ്തു് എഴു­താ­നാ­ണെ­ങ്കില്‍ മു­പ്പ­ത്തി­നാ­ലു 
പു­സ്ത­ക­ത്തി­നും ഓരോ­ന്നി­നൊ­ന്നു വച്ചെ­ന്ന നി­ല­യില്‍ ലേ­ഖ­ന­ങ്ങ­ളെ­ഴു­താന്‍ മാ­ത്ര­മു­ണ്ടു്.

ആ­സ്റ്റെ­റി­ക്സ് പര­മ്പ­ര­യെ­പ്പ­റ്റി പല പഠ­ന­ങ്ങ­ളും നട­ക്കു­ക­യും പല­രും വി­വി­ധ­ങ്ങ­ളായ വി­ശ­ക­ല­ന­ങ്ങ­ളും പല കോ­ണില്‍ നി­ന്നും
നട­ത്തു­ക­യും ചെ­യ്തി­ട്ടു­ണ്ടു്. ഇത്ത­ര­ത്തില്‍ ഏറ്റ­വും പ്ര­സി­ദ്ധ­വും സമ­ഗ്ര­വു­മായ ഒരു പഠ­നം "ദ കപ്ലീ­റ്റ് ഗൈ­ഡ് റ്റു ആസ്റ്റെ­റി­ക്സ്"
എന്ന പേ­രില്‍ പീ­റ്റര്‍ കെ­സ്സ്ലര്‍ പ്ര­സി­ദ്ധീ­ക­രി­ച്ച­താ­ണു്. ഇന്നി­പ്പോള്‍ ആസ്റ്റെ­റി­ക്സെ­ന്ന­തു ഫ്ര­ഞ്ച് ചെ­റു­ത്തു നില്‍­പ്പി­ന്റെ 
അട­യാ­ള­മാ­ണു്. അതോ­ടൊ­പ്പം തന്നെ മി­ല്യണ്‍ കണ­ക്കി­നു ഡോ­ളര്‍ മൂ­ല്യ­മു­ള്ള വ്യ­വ­സാ­യ­വും. പു­തിയ സി­നി­മ­ക­ളും മറ്റും
ആസ്റ്റെ­റി­ക്സി­നെ ആസ്പ­ദ­മാ­ക്കി വരു­ന്നു­ണ്ടു്. അടു­ത്ത­തു ത്രീ ഡി സാ­ങ്കേ­തിക വി­ദ്യ ഉപ­യോ­ഗി­ച്ചാ­കു­മെ­ന്ന­റി­യു­ന്നു. 
പു­സ്ത­ക­ങ്ങ­ളെ­ല്ലാം തന്നെ (എ­റ്റ­വും അവ­സാ­ന­മി­റ­ങ്ങിയ ഗോള്‍­ഡന്‍ ബു­ക്കു് ഇന്ത്യ­യില്‍ ഇതു­വ­രെ ലഭ്യ­മാ­യി­ട്ടി­ല്ല) 
ആഗോ­ള­മാര്‍­ക്ക­റ്റില്‍ ലഭ്യ­മാ­ണു്.

(6 January 2011)\footnote{http://malayal.am/വിനോദം/കാര്‍ട്ടൂണ്‍/9562/ആസ്റ്റെറിക്സിന്റെ-സാഹസിക-കഥകള്‍}

\newpage
