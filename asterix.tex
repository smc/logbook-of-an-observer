\secstar{ആസ്റ്റെറിക്സിന്റെ സാഹസിക കഥകള്‍}
\vskip 2pt

ലോകമെമ്പാടുമുള്ള കുട്ടികളുടെയും മുതിര്‍ന്നവരുടെയും പ്രീതി പിടിച്ചുപറ്റിയവയാണു് "ആസ്റ്റെറിക്സിന്റെ സാഹസിക 
കഥകള്‍ (Adventures of Asterix the Gaul)".  ഫ്രാങ്കോ-ബെല്‍ജിയന്‍ പാരമ്പര്യത്തിലുള്ള കോമിക്കുകളില്‍ 
(ഇംഗ്ലീഷുകാര്‍ക്കിതു് ഗ്രാഫിക് നോവലുകളാണു്) എറ്റവും ജനപ്രീതിയുള്ളവയിലൊന്നാണിതു്. വില്‍പ്പനക്കണക്കുകള്‍ പ്രകാരം 
സൃഷ്ടാക്കളായ റെനെ ഗോസിന്നിയും ആല്‍ബെര്‍ട്ടു് ഉദേര്‍സോയും ഫ്രാന്‍സിനു പുറത്തു് ഏറ്റവും ജനപ്രീതിയുള്ള ഫ്രഞ്ചു് 
എഴുത്തുകാരാണെന്നു പറയുമ്പോള്‍ത്തന്നെ പ്രചാരത്തിന്റെ വലിപ്പം ഊഹിക്കാമല്ലോ.

ഇതുവരെ 34 പുസ്തകങ്ങള്‍ ആസ്റ്റെറിക്സിന്റെ കഥകളുമായി പുറത്തിറങ്ങിയിട്ടുണ്ടു്. 1959 ല്‍ പൈലെറ്റെ മാഗസിനില്‍ 
പരമ്പരയായി പ്രസിദ്ധീകരിച്ചു തുടങ്ങിയ ആസ്റ്റെറിക്സ്, പുസ്തകമായി ആദ്യം പുറത്തിറങ്ങുന്നതു് 1961ലാണു്. അതിനു 
ശേഷം 77ല്‍ ഗോസിന്നിയുടെ മരണംവരെ ഏതാണ്ടെല്ലാ വര്‍ഷവും ഒന്നെന്ന കണക്കില്‍ പുസ്തകങ്ങള്‍ ഇറങ്ങിയിട്ടുണ്ടു്. 
അതുവരെ കഥയും ആശയവും ഗോസിന്നിയുടെ വകയും വര ഉദേര്‍സോയുടേതുമായിരുന്നു. ഗോസിന്നിയുടെ മരണശേഷം 
പത്തു പുസ്തകങ്ങള്‍ ഉദേര്‍സോ തന്നെ വരയ്ക്കുകയും എഴുതുകയും ചെയ്തു് പ്രസിദ്ധീകരിച്ചിട്ടുണ്ടു്.

ആസ്റ്റെറിക്സ് കോമിക്കിന്റെ പശ്ചാത്തലം ഫ്രഞ്ച് പ്രവിശ്യയായ അര്‍മോറിക്കയിലുള്ള പേരില്ലാത്ത ഒരു ഗ്രാമമാണു്. 
റോമന്‍ അധിനിവേശത്തെ വെര്‍സിന്‍ഗെറ്റോറിക്സിന്റെ കീഴടങ്ങലിനു ശേഷവും പ്രതിരോധിക്കുന്നവരാണു് ഗ്രാമവാസികള്‍.
(ജൂലിയസ് സീസറെ എതിര്‍ക്കുകയും അവസാനം കീഴടക്കപ്പെട്ടു് റോമില്‍ വധശിക്ഷയ്ക്കു വിധേയനാക്കപ്പെടുകയും ചെയ്ത 
ഒരു യഥാര്‍ത്ഥ ചരിത്രകഥാപാത്രമാണു് വെര്‍സിന്‍ഗെറ്റോറിക്സ്). അതിനവരെ സഹായിക്കുന്നതാവട്ടെ ഗ്രാമ 
മന്ത്രവാദി (druid) ആയ ഗെറ്റാഫിക്സിന്റെ അമാനുഷികശക്തി നല്‍കുന്ന മരുന്നും.

നായകനായ ആസ്റ്റെറിക്സ് കര്‍മ്മംകൊണ്ടു് ഒരു യോദ്ധാവാണെങ്കിലും ഒരു യോദ്ധാവിന്റെ ശരീരമൊന്നുമല്ല അദ്ദേഹത്തിനു്. 
കുറിയവനായ ആസ്റ്റെറിക്സ് പ്രശ്നങ്ങളെ നേരിടുന്നതും പരിഹരിക്കുന്നതും ബുദ്ധിയുപയോഗിച്ചാണു്. ആസ്റ്റെറിക്സിന്റെ 
ശാരീരികമായ പോരായ്മകളെ പരിഹരിക്കുന്നതു് ഉറ്റ കൂട്ടുകാരനും ഭീമാകാരനും ചെറുപ്പത്തിലേ അത്ഭുതമരുന്നിന്റെ 
കലത്തില്‍ വീണതുകൊണ്ടു് സ്ഥിരമായി അതിന്റെ ശക്തിയുള്ളവനുമായ ഒബ്ലിക്സാണു്. ഗ്രാമത്തിലെ ഏക മെന്‍ഹിര്‍ 
ക്വാറിയുടമയാണു് ഒബ്ലിക്സ്.

%Oblix.jpg

ശക്തനും കൂട്ടുകാരനുവേണ്ടി ജീവന്‍ കളയുന്നവനുമാണെങ്കിലും ഒബ്ലിക്സ് കാര്യങ്ങളെപ്പറ്റി ആലോചനയൊന്നുമില്ലാത്ത 
കൂട്ടത്തിലാണു്. ദിവസവും രണ്ടുമൂന്നു പൊരിച്ച കാട്ടുപന്നികളും (wild boar) കൈത്തരിപ്പു തീര്‍ക്കാന്‍ ഇടയ്ക്കു് ചില റോമന്‍ 
പട്ടാളക്കാരെയും കിട്ടിയാല്‍ മൂപ്പര്‍ ഹാപ്പിയാണു്. പില്‍ക്കാല കോമിക്കുകളില്‍ അര്‍മോറിക്കയിലെ നാലു ക്യാമ്പുകളിലെ 
നിയമനം റോമന്‍ പട്ടാളത്തില്‍ ഒരു ശിക്ഷയോളമെത്തിയ കാലത്തു്, "നല്ല പട്ടാളക്കാരെ ഇങ്ങോട്ടയക്കാതെ നമ്മളെ 
പ്രകോപിപ്പിച്ചു് ഇവിടുത്തെ സമാധാനാന്തരീക്ഷം തകര്‍ക്കാന്‍ ശ്രമിക്കുകയാണു സീസറും സെനറ്റും" എന്നുവരെ ഒരിടത്തു് 
ഒബ്ലിക്സ് പരാതി പറയുന്നുണ്ടു്.

ആകെ ഗോളില്‍ കീഴടങ്ങാതെ നില്‍ക്കുന്ന ഈ ഒരേയൊരു ഗ്രാമത്തിനു സീസര്‍ ഉപരോധം തീര്‍ത്താണു് മടങ്ങിയതു്. 
നാലു സൈനികക്യാമ്പുകളാണു് ഈ ചെറു കടലോരഗ്രാമത്തിനു ഉപരോധം തീര്‍ത്തിരുന്നതു്. അക്വേറിയം, ടോടോറം, 
ലൗഡാനും, കോമ്പെന്‍ഡിയം ക്യാമ്പുകളാണു് അവ.

ആദ്യ കോമിക്കുകളില്‍ വളരെ സീരിയസ്സ് ആയിത്തന്നെ പ്രതിരോധത്തിന്റെ അവസാനകണ്ണിയെ ഇല്ലാതാക്കാന്‍ 
ശ്രമിക്കുന്ന സൈനികമേധാവികളെയും സൈനികരെയുമാണു് നമ്മള്‍ കാണുന്നതു്. എന്നാല്‍ കാലംചെല്ലുംതോറും 
സീസര്‍ പലകാര്യങ്ങളിലും ഗ്രാമവാസികളോടു സന്ധിചെയ്യുകയും സഹായം സ്വീകരിക്കുകയും (Asterix the
Legionary, Asterix and Son തുടങ്ങിയവ ഉദാഹരണം) ചെയ്യുമ്പോള്‍ സൈനികരുടെ മനോഭാവവും മാറുന്നുണ്ടു്.

പുതിയ ദേശങ്ങള്‍ കീഴടക്കുന്ന സാമ്രാജ്യത്തിനു് പഴയ അധിനിവേശങ്ങളിലെ പ്രശ്നങ്ങള്‍ താരമേന്യ നിസ്സാരവും 
ആഭ്യന്തരവുമാകുന്നതിന്റെ പ്രത്യക്ഷ ഉദാഹരണമായാണു് ഇതു എടുത്തു കാണിച്ചിരിക്കുന്നതു്. ഒരിടത്തു് സീസര്‍തന്നെ 
കുഴിമടിയനും മദ്യപനുമായ ഒരു പടയാളിയെ പാഠം പഠിപ്പിക്കാനായി പിരിഞ്ഞുപോകല്‍ ബോണസ്സായി ഈ ഗ്രാമം 
എഴുതിക്കൊടുക്കുന്നുണ്ടു്. ഇത്തരത്തില്‍ നിസ്സാരരും സാമ്രാജ്യത്വത്തിനു് അഭിമാനക്ഷതമല്ലാതെ വലിയ ദോഷമില്ലാത്തതുമായ 
ചെറുത്തുനില്‍പ്പുകളോടുള്ള അധികാരികളുടെ മനോഭാവത്തെ വളരെ വ്യക്തമായും സരസമായും ചിത്രീകരിച്ചിട്ടുണ്ടു് 
ഗോസിന്നിയും ഉദേര്‍സോയും.

50 വര്‍ഷത്തിനുള്ളില്‍ ആസ്റ്റെറിക്സും ഒബ്ലിക്സുമടക്കം കോമിക്കിലെ എല്ലാ കഥാപാത്രങ്ങളും വരയിലും ആശയത്തിലും സ്വന്തം 
വ്യക്തിത്വവും വ്യക്തതയും നേടിയെടുത്തുവെന്നു പറയാം. ആസ്റ്റെറിക്സും ഒബ്ലിക്സും ഗെറ്റാഫിക്സുമല്ലാതെ ഒരുപിടി കഥാപാത്രങ്ങള്‍ 
വേറെയുമുണ്ടു് കോമിക്കില്‍. ഗ്രാമമുഖ്യന്‍ വൈറ്റല്‍ സ്റ്റാറ്റിസ്റ്റിക്സും, ഭാര്യയും ലുറ്റേഷ്യ (പാരീസടങ്ങുന്ന പ്രവശ്യ) ക്കാരിയുമായ 
ഇമ്പെടിമെന്റയും, മീന്‍കച്ചവടക്കാരന്‍ അണ്‍ഹൈജെനിക്സും ഭാര്യ ബാക്റ്റീരയയും, കൊല്ലന്‍ ഫുള്ളിഓട്ടോമാറ്റിക്സും ഭാര്യയും, 
ഗ്രാമത്തിലെ പ്രധാനവയസ്സനായ ജെറിയാട്രിക്സും അയാളുടെ ചെറുപ്പക്കാരിയായ ഭാര്യയും, ഗ്രാമത്തിന്റെ ഗായകന്‍ 
കാകഫോണിക്സ് എന്നിവരെ കൂടാതെ ജൂലിയസ് സീസറും ഒരു പ്രധാന കഥാപാത്രമാണു്.

%ആസ്റ്റെറിക്സ് കഥാപാത്രങ്ങള്‍

ലുറ്റേഷ്യയില്‍വച്ചു് ഒബ്ലിക്സിന്റെ കൂടെക്കൂടിയ ഡോഗ്മാറ്റിക്സ് എന്ന വളര്‍ത്തുനായയും, എന്നും ഗ്രാമത്തെ കൂകിയുണര്‍ത്തുന്ന 
പൂവന്‍കോഴിയും ആവര്‍ത്തിക്കുന്ന മൃഗകഥാപാത്രങ്ങളാണു്. മാത്രമല്ല, സ്ഥിരമായി ഫ്രേമുകളില്‍ ആവര്‍ത്തിയ്ക്കപ്പെടുന്ന 
സാന്നിധ്യമാണു് കോഴികള്‍. ഗ്രാമത്തിലെ എന്തു പ്രധാനസംഭവത്തിന്റെ ഫ്രേമിലും ഒരു കോഴിയെയെങ്കിലും ഉദേര്‍സോ 
ഉള്‍പ്പെടുത്തിയിട്ടുണ്ടാകും. ഗോളിന്റെ ചിഹ്നമാണു് കോഴി എന്നതു മാത്രമാണോ ഇതിനു കാരണം? ഉദേര്‍സോയുടെ തന്നെ 
വാക്കുകളില്‍ കോഴിയെ വരയ്ക്കാന്‍ തനിക്കിഷ്ടമായതുകൊണ്ടാണെന്നൊരു ഒഴുക്കന്‍ വിശദീകരണമാണു് നമുക്കു കിട്ടിയിട്ടുള്ളതു്.

ആദ്യകഥകളില്‍ അത്ഭുതമരുന്നിന്റെ സഹായത്തോടെ ലക്ഷ്യം സാധിച്ചുവരുന്ന 'ആസ്റ്റെറിക്സും ഒബ്ലിക്സും' എന്ന 
ഇതിവൃത്തത്തില്‍ത്തന്നെ കിടന്നുകറങ്ങിയ കഥകള്‍ പിന്നീടു് വ്യത്യസ്ത ഇതിവൃത്തങ്ങളും ആഖ്യാനങ്ങളും തേടിത്തുടങ്ങി. 
'ആസ്റ്റെറിക്സ് ആന്‍ഡ് ദ ബിഗ് ഫൈറ്റി'ലാണു് അത്ഭുത മരുന്നില്ലാതെതന്നെ ലക്ഷ്യംകാണുന്ന രീതിയില്‍ ആദ്യം 
കഥയവസാനിക്കുന്നതു്. പിന്നീടു് "ആസ്റ്റെറിക്സ് ഇന്‍ ബ്രിട്ടണിലും" ഇതാവര്‍ത്തിക്കപ്പെട്ടു.

പിന്നീടു് പല കോമിക്കുകളിലും ലോകവിഷയങ്ങള്‍ ആഖ്യാനങ്ങളുടെ ഭാഗമായി. അത്‌ലറ്റിക്സില്‍ ഉത്തേജക 
മരുന്നുപയോഗം വ്യാപകമായതിനെ പുരാതന ഒളിമ്പിക്സിനെ കൂട്ടുപിടിച്ചാണു് ചിത്രീകരിച്ചിരിക്കുന്നതു്. കമ്പോളത്തിനേയും 
ബൂര്‍ഷ്വാസിയെയും തൊഴിലാളിവര്‍ഗ്ഗസമരത്തെയും ആഗോളവത്കരണത്തേയും വിവിധങ്ങളായ സാമ്പത്തിക 
ശാസ്ത്രപഠനത്തിന്റെ മെക്കകളെയും പ്രതിനിധീകരിച്ചു് "ഒബ്ലിക്സ് ആന്‍ഡ് കോ"യിലെ അന്താരാഷ്ട്ര മെന്‍ഹിര്‍ മാര്‍ക്കറ്റും, 
സ്വദേശി മെന്‍ഹിര്‍ നിര്‍മ്മാതാക്കളുടെ റോമന്‍ റോഡ് ഉപരോധവും, ലാറ്റിന്‍ സ്കൂള്‍ ഓഫ് ഇക്കണോമിക്സില്‍ പഠിച്ച കയസ് 
പ്രപോസ്റ്ററസും ഒക്കെ ഴാക് ഷിറാക്കിന്റെ നേതൃത്വത്തില്‍ ഫ്രാന്‍സില്‍ സ്ഥാനമേറ്റേടുത്ത സര്‍ക്കാരിന്റെ 
നടപടികള്‍ക്കെതിരെയുള്ള ഒരു വിമര്‍ശനമായിരുന്നു എന്നു് ചില വ്യാഖ്യാതാക്കളുടെ പക്ഷം.

നഗരവത്കരണത്തിലൂടെ ഗ്രാമവാസികളെ നശിപ്പിയ്ക്കാന്‍ നോക്കുന്നതും, മൂലധനം ജനങ്ങള്‍ക്കിടയിലെ സ്വാഭാവിക 
സുഹൃത്ബന്ധങ്ങളെ ഉലയ്ക്കുന്നതും, അതീവസരസമായാണെങ്കിലും ചിത്രീകരിച്ച ഗോസിന്നിയും ഉദേര്‍സോയും 
വികസനത്തില്‍ നഷ്ടപ്പെടുന്ന നന്മകളെപ്പറ്റി ആകുലരായിരുന്നുവെന്നു വ്യക്തം. ഫ്രാന്‍സില്‍ നല്ല വേരോട്ടമുള്ള 
ഫെമിനിസത്തേയും വിഷയമാക്കുന്നുണ്ടു് ഉദേര്‍സോ.

ആസ്റ്റെറിക്സ് കഥകളില്‍ ഒരുപാടെണ്ണം യാത്രകളാണു്. അവയില്‍ മാത്രം പ്രത്യക്ഷപ്പെടുന്ന ചില കഥാപാത്രങ്ങളുമുണ്ടു്. 
അറബ് വംശജനായ നാവികനും കച്ചവടക്കാരനുമായ എക്കണോമിക്രൈസിസ്, കടല്‍ക്കൊള്ളക്കാരുടെ സംഘം 
തുടങ്ങിയവര്‍. ഇതില്‍ കടല്‍ക്കൊള്ളക്കാരുടെ സംഘത്തിന്റെ പേടിസ്വപ്നമായിമാറുന്നുണ്ടു് ആസ്റ്റെറിക്സും ഒബ്ലിക്സും.

ആദ്യകാലത്തു് യാത്രകള്‍ യൂറോപ്പിലെ വിവിധദേശങ്ങളിലും (ബെല്‍ജിയം, ബ്രിട്ടന്‍, സ്വിറ്റ്സര്‍ലാന്‍ഡ്, റോം), ഫ്രാന്‍സിന്റെ 
വിവിധ പ്രവിശ്യകളിലും (കോര്‍സിക്ക, ലുറ്റേഷ്യ, ബാന്‍ക്വേ)ഒതുങ്ങിയിരുന്നു. കടമാര്‍ഗ്ഗവും കരമാര്‍ഗ്ഗവും നടന്ന യാത്രകള്‍ പലതും മറ്റുപ്രദേശങ്ങളിലെ 
അധിനിവേശക്കാര്‍ക്കെതിരെ സമാനമായ ചെറുത്തുനില്‍പ്പുകളില്‍ പങ്കാളികളാവാനോ സഹായത്തിനോ ആയിരുന്നു.

പിന്നീടു് വൈക്കിങ്ങുകളുടെ പിന്‍പറ്റി ഗ്രീന്‍ലാന്‍ഡിലും അമേരിക്കയിലും ഇന്ത്യയിലും അറേബ്യയിലും ഒക്കെയായി 
യാത്രകള്‍ വികസിക്കുന്നുണ്ടു്. ഇവിടെയൊക്കെ അധിനിവേശത്തോടല്ല പുതിയ ജീവിതരീതികളോടും 
വില്ലന്‍മാരോടുമാണു് ആസ്റ്റെറിക്സിനും ഒബ്ലിക്സിനും എതിരിടേണ്ടിവരുന്നതു്.

രണ്ടു് ആഫ്രിക്കന്‍ യാത്രകളുള്ളതില്‍ ഒന്നു് ഈജിപ്റ്റില്‍ സീസറിനുവേണ്ടി ക്ലിയോപാട്രയ്ക്കു് കൊട്ടാരം പണിയാന്‍ 
സഹായിക്കാനും, മറ്റൊന്നു് റോമന്‍ പട്ടാളത്തിലേക്കു് ഡ്രാഫ്റ്റ്ചെയ്യപ്പെട്ട ട്രാജിക്‌ണോമിക്സിനെ രക്ഷിക്കാന്‍ 
വേണ്ടിയുള്ളതുമാണു്. രണ്ടാമത്തെ കഥയില്‍ റോമന്‍ പട്ടാളത്തിന്റെ അതിരുകവിഞ്ഞ അച്ചടക്കത്തെ 
അടച്ചാക്ഷേപിക്കുന്നുണ്ടു്. അവസാനം സീസര്‍ക്കു് പോംപിയുടെമേല്‍ വിജയം നേടിക്കൊടുത്താണു് ആസ്റ്റെറിക്സും ഒബ്ലിക്സും 
മടങ്ങുന്നതു്.

വിവിധരാജ്യങ്ങളില്‍ അവിടുത്തുകാരുടെ സ്വഭാവങ്ങളും പ്രത്യേകതകളും മുതലാക്കിക്കൊണ്ടും എടുത്തുകാണിച്ചുകൊണ്ടുമാണു് 
ഹാസ്യമുണ്ടാക്കുന്നതു്. സ്വിസ്സുകാരുടെ സമാധാനപ്രിയതയും സമയനിഷ്ഠയും സ്വിസ്സു് ബാങ്കും 
വരെ പരാമര്‍ശവിധേയമാകുമ്പോ,ള്‍ ലുറ്റേഷ്യയില്‍ തിരക്കും ബഹളങ്ങളും, കോര്‍സിക്കയില്‍ മടിയും അവസാനിക്കാത്ത 
കുടുംബവഴക്കും മറ്റുമാണു് വിഷയം. ബ്രിട്ടനിലെത്തുമ്പോള്‍ ഫുട്ബാള്‍ മുതല്‍ തീന്‍മേശ മര്യാദവരെ വിഷയമാകുന്നുണ്ടു്.

യൂറോപ്പിലെ യാത്രകളിലെ നിതാന്തസാന്നിധ്യമാണു് റോമന്‍ സാങ്കേതികതയും റോഡുകളും പശ്ചാത്തലവികസനവും മറ്റും. 
സ്പീഡ് കൂടിയതിനു് ഒരു യുവാവിന്റെ രഥത്തിനു് ഫൈനടിക്കുന്ന ഓഫീസറും, മൊബൈല്‍ രഥ റിപ്പയറും, കുതിരയ്ക്കു പുല്ലും 
വൈക്കോലും നല്‍കുന്ന ബങ്കും എല്ലാം യൂറോപ്യന്‍ ഹൈവേകളേയും അവിടുത്തെ സംഭവങ്ങളേയും പുരാതനകാലത്തേയ്ക്കു 
പറിച്ചു നട്ടതുമാത്രമാണു്.

കഥകളിലെല്ലാം ആസ്റ്റെറിക്സാണു് നായകനെങ്കിലും പഞ്ച് ഡയലോഗുകള്‍ പലതും ഒബ്ലിക്സിന്റേതാണു്. "ഈ റോമാക്കാര്‍ക്കു 
ഭ്രാന്താണു്" (These romans are crazy) എന്ന ഡയലോഗാണു് ഇതില്‍ ഏറ്റവും പ്രസിദ്ധം. ഇതു പിന്നീടു് മറ്റു പല 
ദേശക്കാരെപ്പറ്റിയും ഒബ്ലിക്സ് പറയുന്നുണ്ടു്. ആല്‍പ്സ് കയറിയിറങ്ങുമ്പോള്‍ കള്ളടിച്ചു് മത്തായി ഉറങ്ങിപ്പോയ 
ഒബ്ലിക്സിനോടു് സ്വിറ്റ്സര്‍ലാന്‍ഡ് എങ്ങനെയുണ്ടായിരുന്നെന്ന ചോദ്യത്തിനു "ഫ്ലാറ്റ്" എന്നൊറ്റവാക്കില്‍ ഉത്തരം 
നല്‍കുന്നുണ്ടു് ഒബ്ലിക്സ്. അതുപോലെ ഒരു സ്ത്രീ ഗ്രാമത്തിന്റെ ഗായികയായി കാകഫോണിക്സിനു പകരം എത്തിയപ്പോള്‍ 
അതില്‍ അസ്വാഭാവികതയൊന്നും ഒബ്ലിക്സിനില്ല. അവര്‍ ആണുങ്ങളുടെ മാതിരി ബീച്ചസ്സ് ധരിച്ചിരുന്നതും ഒബ്ലിക്സിനു 
പ്രശ്നമല്ല, പകരം, കുറുകെയുള്ള വരകളല്ല നീളന്‍ വരകളാണു് തടികുറച്ചു കാണിക്കുന്നതെന്നു് ഇവര്‍ക്കറിയില്ലേ എന്നു 
പറഞ്ഞു് ആര്‍ത്തു ചിരിക്കുകയാണു് മൂപ്പര്‍. ഇത്തരത്തില്‍ സന്ദര്‍ഭം വിശകലനംചെയ്തു് എഴുതാനാണെങ്കില്‍ മുപ്പത്തിനാലു 
പുസ്തകത്തിനും ഓരോന്നിനൊന്നുവച്ചെന്ന നിലയില്‍ ലേഖനങ്ങളെഴുതാന്‍ മാത്രമുണ്ടു്.

ആസ്റ്റെറിക്സ് പരമ്പരയെപ്പറ്റി പല പഠനങ്ങളും നടക്കുകയും, പലരും വിവിധങ്ങളായ വിശകലനങ്ങള്‍ പല കോണില്‍നിന്നു
നടത്തുകയും ചെയ്തിട്ടുണ്ടു്. ഇത്തരത്തില്‍ ഏറ്റവും പ്രസിദ്ധവും സമഗ്രവുമായ ഒരു പഠനം "ദ കംപ്ലീറ്റ ഗൈഡ് റ്റു ആസ്റ്റെറിക്സ്"
എന്ന പേരില്‍ പീറ്റര്‍ കെസ്സ്ലര്‍ പ്രസിദ്ധീകരിച്ചതാണു്. ഇന്നിപ്പോള്‍ ആസ്റ്റെറിക്സെന്നതു ഫ്രഞ്ച് ചെറുത്തുനില്‍പ്പിന്റെ 
അടയാളമാണു്. അതോടൊപ്പംതന്നെ മില്യണ്‍ കണക്കിനു ഡോളര്‍ മൂല്യമുള്ള വ്യവസായവും. പുതിയ സിനിമകളും മറ്റും
ആസ്റ്റെറിക്സിനെ ആസ്പദമാക്കി വരുന്നുണ്ടു്. അടുത്തതു ത്രീ ഡി സാങ്കേതികവിദ്യ ഉപയോഗിച്ചാകുമെന്നറിയുന്നു. 
പുസ്തകങ്ങളെല്ലാം തന്നെ (എറ്റവും അവസാനമിറങ്ങിയ ഗോള്‍ഡന്‍ ബുക്കു് ഇന്ത്യയില്‍ ഇതുവരെ ലഭ്യമായിട്ടില്ല) 
ആഗോളമാര്‍ക്കറ്റില്‍ ലഭ്യമാണു്.

\hspace*{2em}(6 January, 2011)\footnote{http://malayal.am/വിനോദം/കാര്‍ട്ടൂണ്‍/9562/ആസ്റ്റെറിക്സിന്റെ-സാഹസിക-കഥകള്‍}

\newpage
