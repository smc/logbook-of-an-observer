\begin{english}
\secstar{Memories of a Brother}
I still remember the incident when I met Jinesh for the first time. I was working as a lecturer in MES, during that time (in 2005). Once there was a tutorial and hands on session organized by the 'Open source community' of the college. Since I didn’t know anything about this, I went to attend the session. During the hands on, the tutor was telling the commands very fast, I was struggling. One of the volunteers of the program approached me, and patiently helped me to complete that hands on session. He impressed me a lot at that time itself. And the boy was one other than Jinesh.
\begin{center}
---- 
\end{center}
After he joined in IIIT, Hyderabad, and started working in Malayalam OCR, (I was also working on the same project), we spent lots of time spent together. We had quiet a few journeys together to attend OCR workshops. His passion towards the language Malayalam and the development of Malayalam computing was quite a lot.
 
The research areas, he worked on are mainly,
\begin{itemize}
\item The automatic annotation of Malayalam books - from words to character to symbol level.
\item Identification of broken characters,
\item Little bit of post-processing side of character recognition.
\item and more.
\end{itemize}
\begin{center}
---- 
\end{center}
He had a clear idea about what he wants to be. That’s why he came and join in IIIT Hyderabad, even though he had a job offer in hand at that time. And he used to like the environment of the college a lot, which made him decide to convert to Ph.D. under Jawahar sir, which didn’t happen though.
\begin{center}
---- 
\end{center}
One interesting thing that I remember about Jinesh was his passion towards food. Even though the mess food in IIIT, is not that great, he used to pick up a good amount of food and eat it completely. He never used to waste any food saying that its not tasty. This is a habit that I really appreciate (since I cannot follow).
\begin{center}
---- 
\end{center}
But his sleeping hours had varied a lot, during his stay in IIIT. Since the college never had a strict time-table, some times, some times, he used to follow, US timings, and sometimes, UK timings, for days and months. which I believe now, as one of the reason to spoil his health.
\begin{center}
---- 
\end{center}
Before, the disease is identified, quite a long time, I was suffering with leg pain and all. When I met him  last time in IIIT Hyderabad, he was showing me a blood test report. He asked me "See this particular value, this is varying a lot from what is needed. Looks like there is some problem". After that, when I called him after 2 weeks, he was in Vellore.

Neeba Rajesh
\end{english}
\newpage
