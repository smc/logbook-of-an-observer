\begin{english}
\secstar{Memories of a Brother}
I still remember the incident when I met Jinesh for the first time. I was working as a lecturer in MES, during that time (in 2005). Once, there was a tutorial and hands on session organized by the 'Open source community' of the college. Since I didn’t know anything about this, I went to attend the session. During the hands on, the tutor was telling the commands very fast and I struggled to keep up. One of the volunteers of the program approached me, and patiently helped me to complete that hands on session. He impressed me a lot at that instant itself. And the boy was none other than Jinesh himself.
\begin{center}
\line(1,0){20} 
\end{center}
After he joined in IIIT, Hyderabad, and started working in Malayalam OCR, (I was also working on the same project), we spent a lot of time together. We had quiet a few journeys together to attend OCR workshops. His passion towards the language Malayalam and the development of Malayalam computing was remarkable.
 
The research areas that he worked on are mainly:
\begin{itemize}
\item The automatic annotation of Malayalam books - from words to character to symbol level.
\item Identification of broken characters.
\item Little bit of post-processing side of character recognition.
\item and more.
\end{itemize}
\begin{center}
\line(1,0){20} 
\end{center}
He had a clear idea about what he wants to be. That’s why he came and joined in IIIT Hyderabad, even though he had a job offer in hand at that time. Moreover, he used to like the environment of the college a lot, which made him decide to convert to Ph.D. under Jawahar sir, which unfortunately, didn’t happen.
\begin{center}
\line(1,0){20} 
\end{center}
One interesting thing that I remember about Jinesh was his passion towards food. Even though the mess food in IIIT was not that great, he used to pick up a good amount of food and eat it completely. He never used to waste any food saying that its not tasty. This is one of his habits that I really admire (since I cannot follow).
\begin{center}
\line(1,0){20} 
\end{center}
However, his sleeping hours had varied a lot during his stay in IIIT. Since the college never had a strict time-table, sometimes, he used to follow US timings, and at other times, UK timings, for days and months, which I believe now, was one of the reasons that spoiled his health.
\begin{center}
\line(1,0){20} 
\end{center}
Before the disease was identified, for quite a long time, I was suffering from leg pain and all. When I met him last time in IIIT Hyderabad, he was showing me a blood test report. He told me, "See this particular value. This is varying a lot from what is needed. Looks like there is some problem." After that, when I called him after 2 weeks, he was in Vellore.

\begin{flushright}Neeba Rajesh\end{flushright}
\end{english}
\newpage
