\secstar{എന്തുകൊണ്ടു് പത്രങ്ങള്‍ സോഷ്യല്‍ മീഡിയയെ പേടിക്കുന്നു?}
\vskip 2pt

\begin{framed}
അച്ചടിമലയാളം നാടുകടത്തിയ ലേഖനമാണിതു്. മലയാളത്തിലെ ഒരു പ്രമുഖ വാരിക ആവശ്യപ്പെട്ടതനുസരിച്ചു് ജിനേഷ് 
തയ്യാറാക്കിനല്‍കിയ വിശകലനാത്മകമായ ഈ കുറിപ്പു് പിന്നീടവര്‍ പ്രസിദ്ധീകരിക്കേണ്ടതില്ലെന്നു് തീരുമാനിക്കുകയായിരുന്നു.
 ഡിജിറ്റല്‍ വെര്‍ച്വല്‍ ബന്ധങ്ങളിലെ സാമൂഹികത പരമ്പരാഗതമാദ്ധ്യമങ്ങളേക്കാള്‍ സുതാര്യതയും സാമീപ്യതയും സ്വീകാര്യതയും 
നേടുന്നതെങ്ങനെയെന്നും അതിനോടു മുഖ്യധാരാമാദ്ധ്യമങ്ങള്‍ പ്രതികരിക്കുന്നതെങ്ങനെയെന്നും പരിശോധിക്കുന്ന ലേഖനം 
ജിനേഷിന്റെ മരണശേഷം സുഹൃത്തുക്കള്‍ചേര്‍ന്നു് പ്രസിദ്ധീകരിക്കുന്നു.
\end{framed}


രണ്ടാംതലമുറ വെബ്ബ് സങ്കേതങ്ങള്‍ അഥവാ വെബ്ബ് 2.0 സങ്കേതങ്ങള്‍ മലയാളിയുടെ സാംസ്കാരികമണ്ഡലത്തിനു് 
കഴിഞ്ഞ കുറച്ചുവര്‍ഷങ്ങളായി വിലപ്പെട്ട സംഭാവനകള്‍തന്നെ നല്‍കിയിട്ടുണ്ടു്. സര്‍ഗ്ഗധനനും സര്‍ഗ്ഗാസ്വാദകനും സര്‍വ്വോപരി രാഷ്ട്രീയക്കാരനുമായ 
മലയാളിയുടെ വീര്‍പ്പുമുട്ടലിനു് ഇവയൊരു പരിഹാരം നല്‍കുകയായിരുന്നുവെന്നു പറയാം.

അച്ചടിയുടെ വായനാലോകം തന്നെയാണിന്നും വലുതു്. മാത്രമല്ല "മുഖ്യധാര" എന്ന നിലയില്‍ അഭിപ്രായരൂപീകരണവും 
കേന്ദ്രീകരണവും ഇന്നും മലയാളിക്കിടയില്‍ പരമ്പരാഗത മാദ്ധ്യമങ്ങളാണു് നടത്തുന്നതും. പുതിയ വെബ്ബ് സങ്കേതങ്ങളുടെ ഭാഗമായി
 ധാരാളം 'അവനവന്‍ പ്രസാധകന്‍ ' സംരംഭങ്ങളും സൗഹൃദക്കൂട്ടായ്മകളും വളര്‍ന്നുവന്നപ്പോള്‍ പരമ്പരാഗതമാദ്ധ്യമങ്ങളും 
വെറുതെയിരുന്നില്ല. അച്ചടിത്താളുകളില്‍ ഇടംകൊടുത്തു് അവര്‍ 'സങ്കേതങ്ങള്‍ക്കു് ' പ്രചാരം നല്‍കി. എന്നാല്‍ സാങ്കേതികവിദ്യയുടെ
 പ്രാധിനിധ്യത്തിന്റെ കാര്യത്തില്‍ ചില നിയന്ത്രണങ്ങള്‍ അവര്‍ ശരിക്കും പാലിച്ചു.

അവനവന്‍ പ്രസാധകസംരംഭങ്ങളും സൗഹൃദക്കൂട്ടായ്മകളും നിമിഷംപ്രതി സല്ലപിക്കാനാവുന്ന പൊതുഫോറങ്ങളും ഉറക്കെയുള്ള 
ആത്മഗതത്തിന്റെ വേദികളും എല്ലാം അടങ്ങിയതാണു സങ്കേതങ്ങള്‍. ഇതില്‍ അവനവന്‍ പ്രസാധകസംരഭങ്ങളും സൗഹൃദങ്ങളും
 മാത്രമാണു് അച്ചടിത്താളുകളില്‍ ഇടംനേടിയതു്. അവനവന്‍ പ്രസാധകസംരംഭങ്ങളിലെ സര്‍ഗ്ഗാത്മകമായ ഇടപെടലുകള്‍ എല്ലാ 
ആവേശത്തോടുംകൂടി ഉള്‍ക്കൊണ്ടപ്പോള്‍, തഴയപ്പെട്ടതു് പരമ്പരാഗതമായി വ്യവസ്ഥാപിതമാദ്ധ്യമങ്ങളിലൂടെ മാത്രം നടന്നിട്ടുള്ള 
ചര്‍ച്ച/പ്രവര്‍ത്തന/പ്രതികരണ സംരംഭങ്ങളാണു്. ബ്ലോഗുകളെന്ന വെബ്‌ലോഗുകളിലെ കവിതകളും കഥകളും 
അനുഭവക്കുറിപ്പുകളുമെല്ലാം ചര്‍ച്ചയ്ക്കു വിധേയമായി. ബ്ലോഗുകളില്‍ നിന്നെടുക്കുന്ന നല്ലതെന്നു തങ്ങള്‍ കരുതുന്ന കൃതികള്‍ 
പ്രസിദ്ധീകരിക്കാന്‍ ചില വിഭാഗങ്ങള്‍തന്നെ തുടങ്ങി. എന്നാല്‍ ഈ തഴുകല്‍ ലഭിച്ചതു് വിശാലമായ വെബ്ബിലെ പൊതുവെ 
നിരുദ്രവപരമെന്നു പറയാവുന്ന കൃതികള്‍ക്കുമാത്രമാണു്.

അതുപോലെ, വെബ്ബ് 2.0 സങ്കേതങ്ങളുടെ അപാരമായ സാമീപ്യതയുടെയും വേഗതയുടെയും സാധ്യതകള്‍ മുതലെടുത്തുകൊണ്ടുവന്ന
 സോഷ്യല്‍ നെറ്റ്‌വര്‍ക്കിങ് സംരംഭങ്ങളും അവഗണനയാണു നേരിട്ടതു്. ഓര്‍ക്കുട്ട്, ഫേസ്ബുക്ക് പോലുള്ള സൈറ്റുകളിലെ മലയാളി
 സാന്നിധ്യവും പ്രവര്‍ത്തനങ്ങളും തീരെ രേഖപ്പെടുത്താതിരിക്കുകയല്ല മാദ്ധ്യമങ്ങള്‍ ചെയ്തതു്. വിവരസമ്പാദനത്തിനും 
വിതരണത്തിനും പുതിയ മാനങ്ങള്‍ നിശ്ചയിച്ചുകൊണ്ടിരിക്കുന്ന ഈ ഇടങ്ങളെ സൗഹൃദവും സല്ലാപവും നേരംപോക്കും മാത്രം 
ലക്ഷ്യമാക്കുന്ന സ്ഥലങ്ങളായി അവര്‍ രേഖപ്പെടുത്തുകയാണു്. വാര്‍ത്തയുടെയും വിവരങ്ങളുടെയും മൊത്തവിതരണക്കാര്‍ തങ്ങളാണെന്നു
 കരുതുന്ന ഒരു മാദ്ധ്യമസമൂഹത്തില്‍നിന്നും ഇതിനപ്പുറം പ്രതീക്ഷിക്കാന്‍ വയ്യതാനും.

ഭൂമിശാസ്ത്രപരമായി അകന്നുകഴിയുന്ന സമാനമനസ്കരുടെ സമാഗമത്തിനും സംവാദത്തിനും യോജിച്ചുള്ള പ്രവര്‍ത്തനങ്ങള്‍ക്കും 
സുഭദ്രമായൊരു അടിത്തറയാണു് സോഷ്യല്‍ നെറ്റ്‌വര്‍ക്കിങ് സൈറ്റുകളിലൂടെ വളര്‍ന്നതു്. പരസ്പരസഹകരണത്തിനും 
പ്രവര്‍ത്തനത്തിനും പല വേദികളും സോഷ്യല്‍ നെറ്റ്‌വര്‍ക്കിങ് സൈറ്റുകള്‍ക്കു മുന്‍പും (ഇപ്പോഴും) ഉപയോഗിക്കപ്പെടുന്നുണ്ടെങ്കിലും, 
ഇന്ററാക്റ്റീവു് വെബ്ബിന്റെ സാധ്യതകള്‍ പരമാവധി ഉപയോഗിക്കുന്ന ഈ സൈറ്റുകള്‍ സാങ്കേതികവിദ്യയുടെ വിടവുകളെ ശരിക്കും 
ഒഴിവാക്കുന്നവയാണു്.

ഇത്തരം അപാരസാധ്യതകളുള്ള ഒരു സംവിധാനവും, അതില്‍ ചെറുതല്ലാത്ത ഒരു സാന്നിധ്യവുമായി മലയാളി ഇരിക്കുമ്പോഴും 
മാദ്ധ്യമങ്ങള്‍ക്കു് ഇവ വെറും കുട്ടിക്കളിയാവുന്നതെന്തുകൊണ്ടാണു്? അടയാളപ്പെടുത്തുന്നഇടങ്ങള്‍ പലപ്പോഴും 
അപര്യാപ്തമാവുന്നതെന്തുകൊണ്ടാണു്? പല ഇടപെടലുകളും സാധ്യതകളും വിവരവിനിമയത്തില്‍ത്തന്നെ വിപ്ലവം സൃഷ്ടിക്കുമ്പോഴും 
മുഖ്യധാരാസമൂഹത്തെ അവര്‍ ഭയപ്പെടുന്നതെന്തുകൊണ്ടാണു്?

നല്ലൊരു വായനക്കാരനായ മലയാളിയുടെ നല്ലതും ചീത്തയും കാലാകാലങ്ങളായി നിശ്ചയിച്ചിരുന്നതു് അവന്റെ വായനാശീലങ്ങളാണു്.
 അച്ചടിമാദ്ധ്യമത്തിലെ എഴുത്തുകളും ചര്‍ച്ചകളുമായിരുന്നു അവനു വെളിച്ചം നല്‍കിയിരുന്നതു്. എന്നാല്‍ ഇന്നത്തെ 
യുവസമൂഹത്തിന്റെ പ്രധാന ആശയ/വിവര വിനിമയ അടിത്തറ രണ്ടാംതലമുറ വെബ്ബ്സങ്കേതങ്ങളാണു്. മൊബൈലുകളിലെ 
വിവിധ സോഷ്യല്‍ നെറ്റ്‌വര്‍ക്കിങ് പ്രയോഗങ്ങള്‍വഴി അവര്‍ എന്നും ബന്ധിതരാണു്. സ്വന്തം സ്റ്റാറ്റസ് കൃത്യമായി ലോകത്തെ 
അറിയിക്കാനും തനിക്കു പ്രതികരിക്കണമെന്നു തോന്നുന്ന വിഷയത്തില്‍ പ്രതികരിക്കാനും പ്രതിഷേധങ്ങളില്‍ പങ്കെടുക്കാനും അവനു 
നിമിഷങ്ങളേ വേണ്ടു.

അതിരുകളില്ലാത്ത വെബ്ബിന്റെ സാധ്യതകള്‍ കൃത്യമായി ഉപയോഗിക്കുന്നവരാണു് ഇന്നത്തെ യുവത്വം. "ബര്‍ക്കാഗേറ്റി"ലും, 
"കോമണ്‍വെല്‍ത്തു്" ഗെയിംസ് അഴിമതിയുടെ കാര്യത്തിലും, ട്വിറ്ററിലും ബസ്സിലും ഫേസ്ബുക്കിലും മറ്റും പറന്നുനടന്ന സന്ദേശങ്ങള്‍ 
മതി പ്രതികരണശേഷി നശിച്ചവരല്ല ഇതെന്നു മനസ്സിലാക്കാന്‍. ഈ തിളക്കുന്ന ചോരയെ ശക്തമായ ചില ചാലുകളിലൂടെ തിരിച്ചുവിടാന്‍ 
ശേഷിയുള്ളതാണു് സോഷ്യല്‍ നെറ്റ്‌വര്‍ക്കിങ് സൈറ്റുകളുടെ സാന്നിധ്യം.

Like it മാദ്ധ്യമങ്ങളേക്കാളും, സുഹൃത്തുക്കളെ വിലമതിക്കുന്നവന്റെ തലമുറയില്‍ വിവരങ്ങളും ആശയങ്ങളും പ്രചരപ്പിക്കാനായി 
സ്റ്റാറ്റസ് സന്ദേശങ്ങള്‍ ഉപയോഗിക്കുന്ന "സാമീപ്യത"യുടെ മനഃശ്ശാസ്ത്രം പ്രധാനമാണു്. സുഹൃത്തുക്കളുടെ സ്റ്റാറ്റസ് മെസ്സേജുകളിലൂടെ 
വായിക്കുന്ന വാര്‍ത്തകളും ആശയങ്ങളും മൂന്നാമതൊരു മാദ്ധ്യമത്തിലൂടെ അറിയുന്നതിനേക്കാള്‍ വിശ്വാസ്യതയുള്ളതാവുന്നു. 
പലപ്പോഴും കാര്യമറിയാനായി വാര്‍ത്തയുടെ വിശദാംശങ്ങളിലേക്കെത്താനും ഇതു പ്രേരിപ്പിക്കുന്നു. പരമ്പരാഗതമായി 
മാദ്ധ്യമങ്ങള്‍ക്കു കഴിയാതിരുന്ന ഒരുകാര്യം, വാര്‍ത്തയും വിവരങ്ങളും മറ്റാരുടേതോ എന്നതിനുപകരം സ്വന്തംകാര്യം എന്നായി
 അവതരിപ്പിക്കാനുള്ള അവസരം ഇവിടെ സോഷ്യല്‍ നെറ്റ്‌വര്‍ക്കിങ്ങിന്റെ സങ്കേതങ്ങള്‍ നല്‍കുന്നു.

വെബ്ബ്സങ്കേതങ്ങളിലെ തിരച്ചില്‍സൗകര്യവും വിവരത്തിന്റെ അനന്തമായ സംരക്ഷണവും, അധികാരരൂപങ്ങളുടെ പരിധിക്കുമപ്പുറം
 ഒരു സ്വതന്ത്രതയുടെയും നിഷ്പക്ഷതയുടെയും പരിവേഷം അവയ്ക്കു നല്‍കിയിട്ടുണ്ടു്. പരമ്പരാഗതമായ അധികാരരൂപങ്ങളെയോ 
ദേശരാഷ്ട്രനിയമങ്ങളേയോ വെബ്ബ് മാനിക്കുന്നില്ലെന്നതു് കാലങ്ങള്‍ക്കുമുമ്പേ ഒരു തലവേദനയായി ഭരണകൂടങ്ങള്‍ 
കണ്ടിരുന്നു. വെബ്ബിലെ സ്വകാര്യതയുടെ നിയമങ്ങള്‍ ആദ്യകാല സാമ്പത്തിക കുറ്റകൃത്യങ്ങളെ നിയന്ത്രിക്കാനായി 
കൊണ്ടുവന്നതായിരുന്നുവെങ്കില്‍ (അമേരിക്കന്‍ നിയമങ്ങള്‍), അവ പലപ്പോഴും അധികാരത്തിന്റെ മുഷ്കിനെതിരെ സമരങ്ങള്‍ 
നയിക്കാന്‍ വ്യക്തികള്‍ക്കും പ്രസ്ഥാനങ്ങള്‍ക്കും കരുത്തേകുന്നതും നാം കണ്ടു.

വിവരങ്ങള്‍ എല്ലാവര്‍ക്കും അറിയാനുള്ളതാണെന്നു പറഞ്ഞുകൊണ്ടു് വിക്കിലീക്സ് നടത്തിയ വെളിപ്പെടുത്തലുകള്‍ ഭരണകൂടങ്ങളെ 
ഞെട്ടിച്ചതു് പല കാരണങ്ങള്‍ കൊണ്ടാണു്. ജനങ്ങള്‍ തങ്ങളെഴുതിയ സന്ദേശങ്ങള്‍ വായിച്ചു് വിപ്ലവം നടത്തിക്കളയുമെന്നതിനേക്കാളും,
 ഇത്രയും വലിയ വിവരസഞ്ചയം സുരക്ഷിതമല്ലെന്നതു് തങ്ങളുടെ വിവരശേഖരശേഷിയെ ബാധിക്കുമെന്ന ആശങ്കയും, എത്ര വലിയ 
ശേഖരവും ഈ വിവരവിസ്ഫോടനത്തിന്റെ കാലത്തു് എല്ലാക്കാലവും ലഭ്യവും തിരയാവുന്നതും ആണെന്നുള്ള ബോധവുമാണു്. വാക്കും
 പ്രവൃത്തിയും തമ്മിലുള്ള അന്തരം എത്രയും ചുരുക്കുകമാത്രമാണു് മുന്നോട്ടുള്ള വഴിയെന്നൊരു താക്കീതായാണു് പലരും ഇതിനെ 
കണ്ടതു്. ദേശരാഷ്ട്രങ്ങളുടെ നിലനില്‍പ്പും, ജനാധിപത്യത്തിനുമപ്പുറം ഭരണം മാത്രം മതിയെന്നും മുറവിളിക്കുന്നവര്‍ ഇതിനെ ദണ്ഡനം 
കൊണ്ടാണു് നേരിട്ടതു്.

ഭരണകൂടത്തിനും അധികാരത്തിനും, എന്തിനു് മൂലധനത്തിനുവരെ വ്യക്തമായ നിയന്ത്രണമില്ലാത്ത വെബ്ബിലെ സമരങ്ങളെ 
ഇന്ത്യയടക്കമുള്ള ദേശരാഷ്ട്രങ്ങള്‍ ബാലിശമായ സെന്‍സര്‍ഷിപ്പു് നിയമങ്ങള്‍ കൊണ്ടാണു് നേരിടുന്നതു്. പ്രത്യക്ഷത്തില്‍ 
നിയമപരമായിത്തന്നെ, ഐ.ടി. ആക്റ്റു പ്രകാരം തീര്‍ത്തും സ്വകാര്യമായ ഇ-മെയ്ല്‍ സന്ദേശങ്ങള്‍ പോലും വാറന്റില്ലാതെ 
പരിശോധിക്കാനാവുന്ന വകുപ്പുകളുണ്ടു്. ദേശരക്ഷയുടെയും തീവ്രവാദവിരുദ്ധതയുടെയും പേരില്‍ പലരുടെയും പ്രാഥമിക 
ആശയവിനിമയസംവിധാനംവരെ ശക്തമായ നിരീക്ഷണസംവിധാനങ്ങള്‍ക്കു കീഴില്‍ കൊണ്ടുവരാനുള്ള ശേഷിയാണു് 
ഭരണകൂടത്തിനിപ്പോഴുള്ളതു്.

പാര്‍ശ്വവത്കരിക്കപ്പെടുന്ന ജനങ്ങളുടെ അവകാശപ്പോരാട്ടങ്ങള്‍ വെബ്ബിന്റെ ലോകത്തു് വ്യക്തമായി അടയാളപ്പെടുത്തിയിട്ടുണ്ടു്. 
ഒറീസ്സാകണ്‍സേണ്‍സും മണിപ്പൂരിലെ പ്രശ്നങ്ങളും ബിനായക് സെന്നിന്റെ വിമോചനപ്പോരാട്ടവും മുതല്‍ കര്‍ണ്ണാടകാ 
സര്‍ക്കാരിന്റെ ബീഫ് നിരോധനനിയമവും വരെ ഇങ്ങനെ നെറ്റിസണുകള്‍ സ്വയം അടയാളപ്പെടുത്തിയ പോരാട്ടങ്ങളാണു്. 
ശ്രീരാം സേനയുടെ നേതാവു് പ്രമോദ് മുത്തലിക്കിനു് "പിങ്ക്" ചഡ്ഡികള്‍ അയച്ചുകൊടുത്ത സംഭവവും മറക്കാനാവില്ല. 
ഇവിടെയെല്ലായിടത്തും, ഭരണകൂടത്തിന്റെയും അധികാരത്തിന്റെയും വളഞ്ഞവഴികളെ പ്രതിരോധിക്കാന്‍ വെബ്ബിന്റെ വേഗവും 
അനന്തമായ ഓര്‍മ്മയുമാണു് തുണയായതു്.

വികസനത്തിന്റെ പേരുപറഞ്ഞും, തെറ്റായ കണക്കുകളും തെളിവുകളും നിരത്തിയും വാദിക്കുന്നവര്‍ക്കു് മുന്‍കാലങ്ങളില്‍നിന്നും 
വ്യത്യസ്തമായി തിരിച്ചും വിചാരണയ്ക്കൊരിടം വെബ്ബൊരുക്കുന്നു. അതിലുപരി, ഈ വിവരങ്ങളെ അതേവേഗത്തില്‍ സൗഹൃദങ്ങളെ
 മുതലെടുത്തുകൊണ്ടു് സമൂഹത്തിലെത്തിക്കാന്‍ സോഷ്യല്‍ നെറ്റ്‌വര്‍ക്കിങ്ങിന്റെ സാധ്യതകള്‍ക്കാവുന്നു. ഇതു് ജനാധിപത്യത്തിന്റെ 
പേരില്‍ സുതാര്യഭരണത്തിനുപകരം മാദ്ധ്യമങ്ങളേയും വിവിധ മൂലധനശക്തികളെയും കൂട്ടുപിടിച്ചു് അധികാരം കൈയ്യാളുന്നവനു് 
ബുദ്ധിമുട്ടുണ്ടാക്കുന്നുണ്ടു്.

ഇന്നും ബഹുഭൂരിപക്ഷം ജനങ്ങള്‍ക്കിടയിലും ആശയപ്രചരണത്തിനും ഏകോപനത്തിനും വെബ്ബിനു കഴിയുന്നില്ലെങ്കിലും, എല്ലാ 
പ്രശ്നങ്ങളെയും അന്താരാഷ്ട്രമെന്നും അന്തര്‍രാഷ്ട്രമെന്നും വേര്‍തിരിവില്ലാതെ അടയാളപ്പെടുത്തുന്നതിലും, മുന്‍പു് പ്രക്ഷോഭങ്ങളുടെ 
ഭാഗഭാക്കാവുന്നതില്‍ ബുദ്ധിമുട്ടുണ്ടായിരുന്ന ഒരു വിഭാഗത്തിന്റെ നിതാന്തസാന്നിധ്യവും ജാഗ്രതയും ഉറപ്പാക്കുന്നതിലും ശരിക്കും ഒരു 
വിപ്ലവമാണു് സോഷ്യല്‍ നെറ്റ്‌വര്‍ക്കിങ് സങ്കേതങ്ങള്‍ കൊണ്ടുവന്നതു്.

എന്നാല്‍ കേരളത്തിലെ വിദ്യാലയങ്ങളിലും കോളേജുകളിലുമടക്കം നടക്കുന്നതെന്താണു്? ഭരണകൂടം നിയമംകൊണ്ടു 
നടപ്പിലാക്കാന്‍ ശ്രമിക്കുന്ന സെന്‍സര്‍ഷിപ്പു്, വിഭവത്തിന്റെമേലുള്ള പ്രത്യേക അധികാരമുപയോഗിച്ചു അവിടെ നടപ്പാക്കുന്നു. 
സൗഹൃദങ്ങളും അവയില്‍നിന്നും വികസിക്കുന്ന/വികസിക്കാവുന്ന പുരോഗമനപരമായ സാമൂഹ്യമനസ്ഥിതിയുള്ള വിദ്യാര്‍ത്ഥിയും 
അവിടെ സ്വീകാര്യനല്ല. പകരം വിവരശേഖരണത്തിനും ആശയവിപുലീകരണത്തിനുമുള്ള മാര്‍ഗ്ഗങ്ങളെല്ലാം 
കൊട്ടിയടയ്ക്കപ്പെടുന്നു. വിനിമയത്തിന്റെയും കൂട്ടായ്മയുടെയും പുതുമാര്‍ഗ്ഗങ്ങളോടു് ഭരണവര്‍ഗ്ഗത്തിന്റെ അസ്കിതയും, 
പ്രാഥമികവിവരസംഭരണവിതരണസംവിധാനമെന്ന തങ്ങളുടെ സ്ഥാനത്തിനു ചെറുതായെങ്കിലും കിട്ടുന്ന കൊട്ടുകളും 
മാത്രമാണോ ഈ സെന്‍സര്‍ഷിപ്പിനു പിന്നില്‍?

വാര്‍ത്തയുടെ മൊത്തവിതരണക്കാര്‍ ചമയുന്നവര്‍ക്കു് ഈയടുത്തകാലത്തായി കാലവും കണക്കും തെറ്റിയ വാര്‍ത്തകളുടെ പേരില്‍ 
ഒരുപാടു വിമര്‍ശനം നവമാദ്ധ്യമങ്ങളില്‍നിന്നും നേരിടേണ്ടിവന്നിട്ടുണ്ടു്. പ്രസിദ്ധീകരിക്കുന്ന വാര്‍ത്തകളുടെ പ്രൂഫ് വായിച്ചുനോക്കാനും, 
പലപ്പോഴും ആദ്യം വാര്‍ത്തകൊടുക്കാനുള്ള വ്യഗ്രതയില്‍ വിശദാംശങ്ങള്‍ ഉറപ്പാക്കാനും തയ്യാറാവാത്ത മാദ്ധ്യമമുഷ്കിനെ 
കണക്കറ്റ പരിഹാസംകൊണ്ടും വ്യക്തമായ വിവരണങ്ങള്‍കൊണ്ടുമാണു് വെബ് ലോകം നേരിട്ടതു്. രൂപയുടെ ചിഹ്നംചേര്‍ത്ത ഫോണ്ടു് 
ഡിസൈന്‍ചെയ്ത പയ്യന്‍മാരുടെ അവകാശവാദം മുതല്‍ "ഹനാന്‍" എന്ന അത്ഭുത ബാലികയുടെ കഥവരെ ഇങ്ങനെ പലപ്പോഴായി 
പൊളിച്ചടുക്കപ്പെട്ടതാണു്.

വിവര​ശേഖരണം വിരല്‍ത്തുമ്പിനകത്താണെന്നും വിഷയസ്വാധീനമില്ലെങ്കില്‍ റിപ്പോര്‍ട്ടു് ചെയ്യരുതെന്നും വ്യക്തമായ താക്കീതാണു് 
മാദ്ധ്യമങ്ങള്‍ക്കു് വെബ്ബിലെ വേദികള്‍ നല്‍കിയതു്. അതിനുംപുറമെ, ഈ ലോകത്തിലെ നിയമങ്ങള്‍ ഞങ്ങളൊഴികെ 
ബാക്കിയെല്ലാവര്‍ക്കും ബാധകമാണെന്നവിധത്തില്‍ പ്രവര്‍ത്തിച്ചിരുന്നവര്‍ക്കും പ്രതിഷേധങ്ങള്‍ നേരിടേണ്ടിവരികയുണ്ടായി.
 പകര്‍പ്പവകാശംമൂലം സംരക്ഷിക്കപ്പെട്ട ചിത്രങ്ങള്‍ ഉപയോഗിച്ചതിനും, പലപ്പോഴും ബ്ലോഗുകളിലും മറ്റും പ്രസിദ്ധീകരിച്ച കൃതികള്‍ 
യാതൊരു കടപ്പാടും വിവരവുമില്ലാതെ പ്രസിദ്ധീകരിച്ചതിനും മുന്‍നിരമാദ്ധ്യമങ്ങളടക്കം പ്രതിക്കൂട്ടിലാവുന്നതും കണ്ടു.

തങ്ങള്‍ കാലങ്ങളായി വിഡ്ഢികളെന്നു കരുതിയവര്‍ക്കു് പ്രതികരിക്കാന്‍ വേദികളും സങ്കേതങ്ങളും ലഭിച്ചാല്‍ എത്രമാത്രം 
പരുങ്ങലിലായിരിക്കും തങ്ങളുടെ നില എന്നു് മാദ്ധ്യമങ്ങള്‍ തിരിച്ചറിയുകയായിരുന്നു. വിവരവിതരണസംഭരണരംഗത്തെ 
കുത്തക നിലനിര്‍ത്താന്‍ ചുരുങ്ങിയകാലത്തേക്കെങ്കിലും സാങ്കേതികവിദ്യയുടെ ഒഴുക്കിനെ വ്യക്തമായ പ്രചരണങ്ങളിലൂടെ 
തടയേണ്ടിവരുമെന്നു് അവര്‍ തീരുമാനിച്ചിട്ടുണ്ടെങ്കില്‍ അത്ഭുതപ്പെടാനില്ല. സാങ്കേതികവിദ്യ കൊണ്ടുവരുന്ന അതിസുതാര്യതയും 
വേഗവും സഹിക്കാത്ത ഭരണകൂടത്തിന്റെ കലവറയില്ലാത്ത പിന്തുണയും അവര്‍ക്കു ലഭിച്ചു കാണണം.

ഭരണകൂടത്തിന്റെയും മാദ്ധ്യമങ്ങളുടെയും അപ്രമാദിത്വത്തിനുനേരെ ഉയര്‍ത്തിയ വെല്ലുവിളികള്‍ സാധ്യതകളുടെ ഒരു കൂട്ടമാണു് സമൂഹത്തിനു 
മുന്നില്‍ തുറന്നുകൊടുത്തതു്. നേരത്തെ സൂചിപ്പിച്ചതുപോലെ ഇന്നത്തെ യുവാക്കള്‍ വര്‍ത്തമാനപത്രത്തേക്കാളും ഭരണകൂടത്തേക്കാളും 
വിവരങ്ങള്‍ക്കു് അനോണിമസ് വെബ്ബിന്റെ സഹായം തേടുന്നവരാണു്. മുമ്പുള്ള തലമുറകള്‍ക്കില്ലാതിരുന്ന, അപരനിലേക്കെത്തിച്ചേരാനും 
നേരിട്ടു് വിവരങ്ങളറിയാനുമുള്ള സംവിധാനങ്ങളവനുണ്ടു്. അധികാരരൂപങ്ങള്‍ വകയിരുത്തുന്ന കോളങ്ങള്‍ക്കപ്പുറം അപരന്റേയും 
പാര്‍ശ്വവത്കൃതന്റേയും വിവരങ്ങള്‍ അവനിന്നു ലഭ്യമാണു്.

പാടിപ്പഴകിയ ദേശഭക്തിയുടെയും ദേശരാഷ്ട്രങ്ങളുടെ ശാശ്വതമായ നിലനില്‍പ്പിന്റേയും മറ്റും ഭാഷ്യങ്ങളില്‍ ചാലിച്ച വാക്‌ധോരണികള്‍ 
അവനെ പഴയപോലെ സംതൃപ്തനാക്കുന്നില്ല. കാരണം ദേശരാഷ്ട്രങ്ങളുടെ കരുത്തു് വഴിഞ്ഞൊഴുകുന്ന ഇന്നത്തെ ലോകത്തു് 
ദേശങ്ങളോടു ബന്ധിക്കപ്പെടാനാവാത്തവന്റെ ആക്രോശങ്ങളും രോദനങ്ങളും അവനിലേക്കെത്തുന്നുണ്ടു്. രാഷ്ട്രീയവും ദേശീയവും
മാത്രമല്ല പ്രശ്നങ്ങള്‍, പരമപ്രധാനമായി മനുഷ്യത്വപരമായതാണെന്നുള്ള തിരിച്ചറിവിലേക്കു് അവന്‍ ചുവടുവച്ചു കയറുകയും ചെയ്യുന്നുണ്ടു്.
 അരുന്ധതി റോയ് പൊതുമണ്ഡലത്തില്‍ മനുഷ്യത്വപരമായ അഭിപ്രായപ്രകടനം നടത്തിയതിനു് ആക്രമണത്തിനു് വിധേയയായപ്പോഴും, 
തെഹല്‍ക്കയുടെ ഷാഹിനയുടെമേല്‍ ഗുരുതരമായ തീവ്രവാദക്കുറ്റം ചുമത്തപ്പെട്ടപ്പോഴും മാദ്ധ്യമങ്ങളേക്കാളും ഭരണകൂടത്തേക്കാളും 
യുവാക്കള്‍ വിശ്വസിച്ചതു് "അനോണിമസ്സ്" വെബ്ബിന്റെ നിഷ്പക്ഷതയായിരുന്നു.

സുതാര്യതയുടെയും സാമീപ്യതയുടെയും സ്വീകാര്യതയുടെയും പുതിയ വിനിമയപാതകള്‍ വെട്ടിത്തുറക്കുകയും ഇനിയും അനേകായിരം 
സാധ്യതകള്‍ അവശേഷിപ്പിക്കുകയും ചെയ്യുന്ന സങ്കേതങ്ങളെ ഇത്തരത്തില്‍ ഒതുക്കിക്കളയുന്നതു് അധികാരരൂപങ്ങളുടെ മാത്രം 
കളിയാണോ? വളര്‍ന്നുവരുന്ന തലമുറയുടെമേലുള്ള അധികാരത്തിന്റെ നിയന്ത്രണം നഷ്ടമാകുന്നതും, അറിവിന്റേയും അധികാരത്തിന്റേയും
 ജനാധിപത്യത്തിനുമപ്പുറമുള്ള യഥാര്‍ത്ഥവികേന്ദ്രീകരണവും വിതരണവും സുതാര്യമായി നടപ്പായിക്കൊണ്ടിരിക്കുന്ന സാധ്യതകളും 
അത്തരമൊരു ശ്രമത്തിലെത്തിച്ചിരിക്കാം. എന്നാല്‍ ചിന്തിക്കുന്നവന്റെ സമൂഹമായ കേരളത്തില്‍ അധികാരത്തിന്റെ കോട്ടകളില്‍നിന്നും 
മുഖ്യധാരാമാദ്ധ്യമങ്ങളില്‍നിന്നും വരുന്ന അജണ്ടകള്‍ എതിര്‍പ്പുകള്‍ കൂടാതെ നടപ്പാക്കാനാവുമെന്നതു് വെറും മിഥ്യാധാരണയാണു്. ‌

വികസനത്തിന്റെ പേരുപറഞ്ഞു് പാരിസ്ഥിതികവും മാനുഷികവുമായ പരിഗണനകളെ കാറ്റില്‍പ്പറത്താന്‍ ശ്രമിച്ചപ്പോഴൊക്കെ 
ശക്തമായ ജനകീയസമരങ്ങളുമായി അധികാരത്തെ തറപറ്റിച്ചവരാണു് കേരളീയര്‍. സൈലന്റ് വാലിയിലും പൂയ്യംകുട്ടിയിലും
 മുത്തങ്ങയിലുമെല്ലാം പാര്‍ശ്വവത്കരിക്കപ്പെടുന്ന താത്പര്യങ്ങളെ അടയാളപ്പെടുത്താനും ഏറ്റെടുക്കാനും തയ്യാറുള്ള ഒരു 
സമൂഹത്തെയാണു നാം കണ്ടതു്. എന്നാല്‍ കഴിഞ്ഞ കുറച്ചുകാലങ്ങളായി മലയാളിയുടെ അപരനോടുള്ള പൊതുബോധത്തിനും 
വിശ്വാസത്തിനും ഇടിവുതട്ടിയിട്ടില്ലേ എന്നതൊരു സംശയമാണു്. ഒരുപക്ഷേ വര്‍ത്തമാനമലയാളത്തിന്റെ സാമ്പത്തിക-സാംസ്കാരിക 
നട്ടെല്ലായ പ്രവാസത്തിന്റെ അനുഭവങ്ങളില്‍ നിന്നുമുള്ള ചൂടായിരിക്കാം ഇതിനു കാരണം.

കൂടുതല്‍ തന്നിലേക്കൊതുങ്ങുന്നതാണു് സൗകര്യമെന്നു തിരിച്ചറിയുന്ന മലയാളി ഒരു പുതിയ കാഴ്ചയല്ല. അടിമുടി നഗരപ്രതീതിയുള്ള
 ജീവിതം ഗ്രാമങ്ങളിലേക്കുപോലും ഒഴുകുമ്പോള്‍ അപരനോടുള്ള സഹകരണത്തിനുപകരം അപരരോടുള്ള അകലം അളവുകോലാവുന്ന 
നാഗരികജീവിതരീതി മലയാളിയെ ഗ്രസിച്ചില്ലെങ്കിലേ അത്ഭുതമുള്ളൂ. എന്നാല്‍ അതു് വ്യക്തമായ ചില അജണ്ടകളോടുകൂടി ഭരണകൂടത്തിനും
 നിക്ഷിപ്തതാല്‍പ്പര്യക്കാര്‍ക്കും തന്‍കാര്യം നടപ്പാക്കാന്‍ അവസരമേകുന്നു. വിദ്യാഭ്യാസത്തിന്റെ പേരുപറഞ്ഞു് സൗഹൃദക്കൂട്ടായ്മകളെ 
വിലക്കുന്ന പൊതുസമൂഹം എന്താണു് വിദ്യാഭ്യാസമെന്നു തിരിച്ചറിയുന്നതില്‍പ്പോലും പരാജയപ്പെടുന്നു.

Digital bomb കളെ പൊതുവേ സമൂഹത്തില്‍ നല്ല വേരോട്ടമുള്ള വിവരസാങ്കേതികവിദ്യാധിഷ്ഠിത പ്രവര്‍ത്തനങ്ങള്‍ നടത്തുന്ന 
പ്രസ്ഥാനങ്ങളും കൂട്ടായ്മകളും പോലും സംശയത്തോടെയാണു് വീക്ഷിക്കുന്നതു്. സ്വതന്ത്ര മലയാളം കമ്പ്യൂട്ടിങ്ങു് കൂട്ടായ്മയേയും മലയാളം 
വിക്കിപ്പീഡിയാ സംരംഭത്തേയും സ്വതന്ത്ര സോഫ്റ്റ്‌വെയര്‍ പ്രസ്ഥാനങ്ങളേയുമെല്ലാം സംശയത്തോടെ മാത്രമേ പൊതുസമൂഹം കാണുന്നുള്ളൂ.
സുഗമമായ ഒരുപാതയില്‍ സ്വാതന്ത്ര്യത്തിന്റേയും പങ്കുവയ്ക്കലിന്റേയും സന്ദേശങ്ങളുമായി വരുന്നതും, അതിനോടു തന്റെ
 അടുത്തതലമുറ സഹകരിക്കുന്നതും സമൂഹത്തിനിഷ്ടപ്പെടുന്നില്ല. ആ പൊതുബോധത്തോടു കൂട്ടുചേര്‍ന്നാണു് അധികാരരൂപങ്ങള്‍ 
സോഷ്യല്‍ നെറ്റ്‌വര്‍ക്കിങ്ങിലൂടെയുള്ള ആശയവിനിമയത്തിന്റെ പുതിയ സാധ്യതകളെ ഒരു തലമുറയില്‍നിന്നും അകറ്റി നിര്‍ത്താന്‍ 
ശ്രമിക്കുന്നതു്.

തന്റേതു മാത്രമെന്ന ബോധത്തിനുമപ്പുറം അപരനെക്കുറിച്ചു് ചിന്തിക്കാനും അവനെ മനസ്സിലാക്കാനും വിപ്ലവകരമായ 
മാര്‍ഗ്ഗങ്ങള്‍ തുറന്നുതരുന്ന സോഷ്യല്‍ നെറ്റ്‌വര്‍ക്കിങ്ങിന്റെ സാധ്യതകളെ അടിച്ചിരുത്തേണ്ടതു് പ്രതിലോമശക്തികളുടെ
 ആവശ്യമാണു്. അതിനവര്‍ എല്ലാ ആയുധങ്ങളും പ്രയോഗിക്കുന്നുമുണ്ടു്. സെന്‍ഷര്‍ഷിപ്പിന്റേതായ അധികാരത്തിന്റെ വഴിയും, 
വിമര്‍ശനങ്ങളോടും പൊതുതാത്പര്യങ്ങളോടും ആഭിമുഖ്യം പ്രകടിപ്പിക്കുന്നതിനെ എതിര്‍ത്തുകൊണ്ടുള്ള സാമൂഹ്യഅക്രമങ്ങളും, 
അഭിപ്രായരൂപീകരണത്തില്‍ മാദ്ധ്യമങ്ങളുടെ പങ്കു് കൗശലപരമായി ഉപയോഗിച്ചും ഈ സാധ്യതകളെ തകിടംമറിക്കാന്‍ 
ശ്രമിക്കുന്നുണ്ടു്. വരുംതലമുറകളുടെ ആശയവിവരശേഖരത്തിന്റേയും ആവിഷ്കാരത്തിന്റെയും അടിത്തറയാവാന്‍ കെല്‍പ്പുള്ള ഒരു 
മാദ്ധ്യമമാണു് ഇത്തരത്തില്‍ അവഗണന നേരിടുന്നതു്.

ചെറുതല്ലാത്ത വെബ്ബ് ഇടപെടലുകള്‍വഴി സമൂഹത്തിലും മലയാളഭാഷയുടെതന്നെ ഡിജിറ്റല്‍ വഴികളിലും വ്യക്തമായ മുദ്രപതിപ്പിച്ച 
കൂട്ടായ്മകളെയാണു് പ്രതിലോമകരമാണോ എന്നു് സംശയിച്ചു് സമൂഹം പ്രതിക്കൂട്ടിലാക്കാന്‍ നോക്കുന്നതു്. അധികാരത്തിന്റെ 
മണ്ടത്തരങ്ങളോടു കലഹിക്കാനും, ബദലുകള്‍ നിര്‍ദ്ദേശിക്കാനും, പലപ്പോഴും സുദൃഢമായ ആശയങ്ങള്‍ മുന്നോട്ടു വയ്ക്കാനും 
ഈ കൂട്ടായ്മകള്‍ക്കായിട്ടുണ്ടു്. എന്നാല്‍ വ്യവസ്ഥിതിയ്ക്കു പുറത്താണു് നില്‍പ്പെന്നതിനാല്‍ പല ഇടപെടലുകളും ലക്ഷ്യത്തിലെത്താതെ
 പോകുകയും ചെയ്യുന്നു.

വെബ്ബ് അധിഷ്ഠിത കൂട്ടായ്മകളുടെ പ്രവര്‍ത്തനമികവു മാത്രം കൈമുതലായി പ്രശ്നങ്ങളില്‍ ഭാഗഭാക്കാവാന്‍ ഇന്നും അവര്‍ക്കു 
ബുദ്ധിമുട്ടുകളുണ്ടു്. എന്നാല്‍പ്പോലും പ്രത്യക്ഷ ഫോറങ്ങളില്‍ത്തന്നെ സമൂഹവുമായി സംവദിക്കാന്‍ അവര്‍ക്കിന്നു സാധിക്കുന്നു 
എന്നതു് ഒരു മികവാണു്. ഇതിനുവേണ്ടി പുരോഗമനപരമായ സാമൂഹ്യആശയങ്ങളുടെ വക്താക്കളെന്ന ലേബലിനേക്കാളും 
സാങ്കേതികവിദ്യയുടെ വക്താക്കളെന്ന ലേബലുപയോഗിക്കേണ്ടിവരുന്നുവെന്ന അവസ്ഥയാണു് മാറേണ്ടതു്. അതിനു കഴിഞ്ഞാല്‍ 
മാത്രമേ, അപരനെ തിരിച്ചറിയാനും ബഹുമാനിക്കാനും കഴിയുന്ന ഒരു സമൂഹത്തെ വളര്‍ത്തുന്നതിനായി ഈ സങ്കേതങ്ങളെ
 ഉപയോഗിക്കാന്‍ കഴിയൂ.

\hspace*{2em}(Oct 8, 2011)\footnote{http://malayal.am//വാര്‍ത്ത/വിശകലനം/13042/എന്തുകൊണ്ടു്-പത്രങ്ങള്‍-സോഷ്യല്‍-മീഡിയയെ-പേടിക്കുന്നു}
\newpage
