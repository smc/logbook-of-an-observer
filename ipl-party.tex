\secstar{ഐപിഎല്‍ ആഫ്റ്റര്‍ മാച്ച് പാര്‍ട്ടി}
\enlargethispage{2\baselineskip}
%\vskip 2pt

‌\begin{framed}
"ഐപിഎല്‍ ടീമുകളുടെ രണ്ടുവര്‍ഷത്തെ ചരിത്രം പരിശോധിച്ചു തുടങ്ങിയതാണു് പരമ്പര. ആദ്യഭാഗത്തില്‍ ഐപിഎല്‍ 
ഉണ്ടാക്കാനിടയായ സാഹചര്യം വിലയിരുത്തിയെങ്കില്‍ ,രണ്ടും മൂന്നും ഭാഗങ്ങള്‍ വിവിധടീമുകളുടെ സ്ട്രാറ്റജിയും നയങ്ങളും 
ലക്ഷ്യങ്ങളുമാണു് വിലയിരുത്തിയതു്. ഇന്ത്യന്‍ സ്പോര്‍ട്സ് വ്യവസായത്തിനു് ഐപിഎല്‍ നല്‍കിയ ഏറ്റവും 
പ്രധാന സംഭാവനയെക്കൂടി വിലയിരുത്തി ഈ പരമ്പര അവസാനിക്കുകയാണു്."
\end{framed}

%{\vskip 4pt}

ഏറ്റവും വലിയ സംഭാവനയെന്തെന്നുള്ള ചോദ്യത്തിനു് പലരും പല ഉത്തരങ്ങളും നല്‍കുമായിരിക്കും. എന്റെ കണക്കില്‍, 
കായികവിനോദവ്യവസായത്തില്‍ കായികവിനോദമാണു് വ്യവസായവത്കരിക്കപ്പെടുന്നതു്. ജനകീയ കായികരൂപങ്ങളിലെ 
വിനോദം മൈതാനത്തിലെ കളിയില്‍നിന്നു് ഏറെയൊന്നും മുന്‍പോട്ടു പോയിട്ടില്ല. വ്യവസായവത്കരിക്കുമ്പോള്‍ ഇതൊരു 
പ്രശ്നമാണു്, കാരണം കളി കാണാന്‍ മാത്രമായി വരുന്ന സ്റ്റേഡിയത്തിലെ കാണികളും, ടെലിവിഷന്‍ പ്രേക്ഷകരും അതില്‍നിന്നുള്ള വരുമാനവും പരിധിയുള്ളതാണു്. അതിനാല്‍ത്തന്നെ കമ്പോളത്തില്‍ എല്ലാവര്‍ക്കുംവേണ്ട വര്‍ഷാവര്‍ഷം പുതുക്കിയ രണ്ടക്ക വളര്‍ച്ചാനിരക്ക് 
(ഇക്കൊല്ലത്തെ വളര്‍ച്ചാനിരക്ക് കഴിഞ്ഞകൊല്ലത്തേക്കാള്‍ കുറഞ്ഞാല്‍ പോലും കമ്പോളം 
വേവലാതിപ്പെടും :)) എന്നത് ഒരു ഉട്ടോപ്യയായി മാറും. ഐപിഎല്‍ ഫ്രാഞ്ചൈസികള്‍ ലിസ്റ്റ് ചെയ്യാന്‍ (അതുവഴി കൂടുതല്‍ 
പണം സ്വരൂപിക്കാനും) കഷ്ടപ്പെടും. അതിനു് ഐപിഎല്‍ കണ്ട പ്രതിവിധി, കായികവിനോദത്തിലെ വിനോദത്തിനെ ഒന്നുകൂടി വിപുലമാക്കി, ഗ്രൗണ്ടില്‍ നടക്കുന്ന കായികമത്സരവുമായി യാതൊരു ബന്ധവുമില്ലാതാക്കുകയായിരുന്നു.

പണം ചെലവാക്കുന്ന ആളുകളുടെ സെഗ്‌മെന്റ് എടുത്തു നോക്കിയാല്‍, ഏറ്റവും വലിയ ധൂര്‍ത്തന്‍മാര്‍ 'യങ് അര്‍ബന്‍ 
മിഡില്‍ക്ലാസ് ' ആണെന്നുകാണാം. അവരെ മുഴുവന്‍ ഉള്‍ക്കൊള്ളാന്‍ ഇന്ത്യന്‍ ക്രിക്കറ്റ് സ്റ്റേഡിയങ്ങള്‍ക്കു കഴിയില്ല. മാത്രമല്ല, 
ക്രിക്കറ്റിനോടു് ഭ്രാന്തമായ ആവേശമില്ലാത്ത, പ്രീമിയര്‍ ലീഗും എന്‍ബിഎയും ഫോര്‍മുല വണ്ണും പിന്തുടരുന്ന ഒരു വലിയ വിഭാഗം 
അവര്‍ക്കിടയിലുണ്ടു്. പലപ്പോഴും, ഈ സെഗ്‌മെന്റിലെ ഏറ്റവും സമ്പന്നവിഭാഗം ഇവരാണുതാനും. ഇവര്‍ 
പണമൊഴുക്കിത്തുടങ്ങിയാലെ, നിശ്ചിതവരുമാനത്തില്‍നിന്നും എക്സ്‌പൊണെന്‍ഷ്യല്‍ രീതിയില്‍ വളരാന്‍ ഐപിഎല്ലിനു 
കഴിയൂ. അതിനാല്‍ അവരുടെ പാര്‍ട്ടിസമയങ്ങള്‍ക്കും കൂടി സമാനമായാണു് ഇക്കൊല്ലത്തെ ഐപിഎല്‍ മാച്ചുകള്‍ 
നിശ്ചയിച്ചിരുന്നത്.

%image courtesy: http://blogs.rediff.com/aashirvaad09/

പ്രവര്‍ത്തിദിനങ്ങളില്‍ ഒരു കളി, വൈകി 8 മണിക്കു തുടങ്ങുന്നു. വാരാന്ത്യങ്ങളില്‍ രണ്ടു കളി, ഒന്നു നാലുമണിക്കും മറ്റേതു് 
എട്ടുമണിക്കും. നട്ടുച്ചയ്ക്കു കളിനടത്തിയാലും ഗ്രൗണ്ടു് നിറയുന്ന ഇന്ത്യയില്‍, ഇതു് പ്രധാനമായും പാര്‍ട്ടിപ്രേക്ഷകരെ ലക്ഷ്യം 
വച്ചാണെന്നുള്ളതു വ്യക്തം (ടെലിവിഷനില്‍ പ്രൈം ടൈം ആണതു്, കുടുംബകലഹം ഉണ്ടാക്കാന്‍ പോന്ന കാര്യം!).

ഇതിനോടൊപ്പം തന്നെ, എന്റര്‍ടൈന്‍മെന്റ് സ്പോര്‍ട്സ് ഡയറക്റ്റുമായി സഹകരിച്ചു് റോയല്‍ ചാലഞ്ചര്‍ സ്പോര്‍ട്സ് 
മിതമായ പരസ്യങ്ങളുമായി പബ്ബുകള്‍ക്കും സ്പോര്‍ട്സ് ബാറുകള്‍ക്കും നല്‍കിയ ഉഗ്രന്‍ ഫീഡും കണക്കിലെടുക്കണം. 
ടെലിവിഷന്‍ പ്രേക്ഷകന്റെ ഒരു ഭാഗമായി പബ്ബ്/സ്പോര്‍ട്സ് ബാര്‍ പ്രേക്ഷകരെ കാണാതെ, പ്രത്യേകമായിത്തന്നെ 
പരിഗണിച്ചിരുന്നു എന്നാണിതു കാണിക്കുന്നതു്. ക്രിക്കറ്റിനൊപ്പം പൂളും, ബൗളിങ്ങും, ഹിപു് ഹോപ്പും, പിന്നെ മല്യയുടെ മദ്യവും. 
ഇത്രയും ക്രിക്കറ്റിനെ ടൗണിലെ അടിച്ചുപൊളി പിള്ളാരുടെ ഡെയ്‌ലി റൊട്ടീനില്‍ ഉള്‍പ്പെടുത്താനുള്ള കളികള്‍. ഇവിടെ 
പ്രധാനമായും മെട്രോകളിലെയും രണ്ടാംനിര നഗരങ്ങളിലെയും അപ്പര്‍ മിഡില്‍ക്ലാസ്സിലെ ക്രിക്കറ്റ് അലര്‍ജിക്കാരെയാണു് 
ലക്ഷ്യംവച്ചതു്. നിറഞ്ഞ പബ്ബുകള്‍ ഇതൊരു വന്‍വിജയമായിരുന്നുവെന്നതിനു തെളിവാണു്.

%image courtesy: http://bollywoodnewsstories.blogspot.com/2010/03/ipl-signature-after-match-party-at-ub.html

കളിനടക്കുന്ന നഗരങ്ങളില്‍ മത്സരത്തിനുശേഷം നടക്കുന്ന പാര്‍ട്ടികളും ഫാഷന്‍ ഷോകളും ലക്ഷ്യംവയ്ക്കുന്നതു് അതിസമ്പന്നരുടെ 
സോഷ്യല്‍ ലൈഫില്‍ ക്രിക്കറ്റിനു് ഇടം നല്‍കുകയെന്നാണു്. ക്രിക്കറ്റ് മൈതാനത്തിന്റെ ഒരു മുക്കില്‍ 
തുള്ളിച്ചാടാനെന്ന പേരില്‍ കൊണ്ടുവരുന്ന ചിയര്‍ഗേള്‍സും, ഈ പാര്‍ട്ടികളില്‍ കുറച്ചു ചിയര്‍ എക്സ്ട്രാ കൊണ്ടുവരാനുള്ളതാണു്. 
ഇന്ത്യന്‍ പാര്‍ട്ടിസര്‍ക്കിളിലെ, ഹൂസ് ഹൂ ആയ ഷാരൂഖ്-ഗൗരി ഖാന്‍, വിജയ് മല്ല്യ, ഷെട്ടി സിസ്റ്റേഴ്സ്, പ്രീതി സിന്റ, 
നിത-മുകേഷ് അംബാനി, ഇവരുടെയൊക്കെ പാര്‍ട്ടികളില്‍ ക്ഷണിക്കപ്പെട്ടാല്‍ അതു നല്‍കുന്ന സോഷ്യല്‍ മൈലേജ് ഈ 
പാര്‍ട്ടികളെ ഗംഭീരമാക്കുന്നു. ഇപ്രാവശ്യം പാര്‍ട്ടികളുടെ അതിപ്രസരം കാരണം പല ക്രിക്കറ്റര്‍മാരും, 'പ്ലീസ് ഇന്നെന്നെ 
ഒഴിവാക്കൂ' എന്നു പറഞ്ഞതായും കേള്‍ക്കുന്നുണ്ടു്. പണ്ടു് "പൂച്ചക്കൊരു മൂക്കുത്തിയിലെ" സുകുമാരിയെപ്പോലെ ക്രിക്കറ്റ് 
കാണുന്നവരുമായി ഒരു രാത്രിമുഴുവന്‍ ചിലവഴിക്കുന്നതോര്‍ക്കുമ്പോള്‍ കളിക്കാര്‍ക്കു മുട്ടിടിക്കുന്നുണ്ടാകും! തമാശ ഒഴിവാക്കിയാല്‍, 
കാലങ്ങളായി ജനലക്ഷത്തിന്റെ കളി എന്നു പറഞ്ഞു് ക്രിക്കറ്റിനെ ഒഴിവാക്കിയിരുന്നവരെക്കൂടി പ്രധാന 
പ്രേക്ഷകരാക്കിയെടുക്കുന്നതിനാണു് ഈ ക്രിക്കറ്റ് വിനോദത്തില്‍നിന്നും ക്രിക്കറ്റ് ഒഴിവാക്കിയ പരിപാടി സഹായിച്ചതു്, 
അതിലൂടെ കോടികളുടെ വരുമാനവും.

ഐപിഎല്‍ ഇന്ത്യന്‍ സ്പോര്‍ട്സ് വ്യവസായത്തിനു നല്‍കിയ ഏറ്റവും വലിയ സംഭാവന ഇതാണു്. ഒരു ജനകീയ കായികരൂപമായതിനാല്‍ 
ക്രിക്കറ്റ് മെട്രോകളിലെ ഉപരിവര്‍ഗ്ഗ പാര്‍ട്ടിസര്‍ക്കിളുകളില്‍ നേരിട്ടിരുന്ന അയിത്തം ഒഴിവാക്കാന്‍ 
ഐപിഎല്ലിനു കഴിഞ്ഞു. അതുവഴി, പതിന്മടങ്ങു വരുമാനവും. ജനകീയകായികവിനോദത്തെ എക്സ്‌ക്ലൂസിവിറ്റിയുടെ ലോകത്തു് 
പ്രതിഷ്ഠിക്കുന്നതെങ്ങനെയെന്നാണു് ഐപിഎല്‍ കാണിച്ചു തന്നതു്.

\hspace*{2em}(13 May, 2010)\footnote{http://malayal.am/പലവക/പരമ്പര/ബിസിനസ്-ലീഗ്/5426/ഐപിഎല്‍-ആഫ്റ്റര്‍-മാച്ചു്-പാര്‍ട്ടി}

\newpage
