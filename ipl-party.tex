\secstar{ഐപിഎല്‍ ആഫ്റ്റര്‍ മാച്ച് പാര്‍ട്ടി}
\vskip 2pt

‌\begin{quotation}
``ഐ­പി­എല്‍ ടീ­മു­ക­ളു­ടെ രണ്ടു വര്‍­ഷ­ത്തെ ചരി­ത്രം പരി­ശോ­ധി­ച്ച് തു­ട­ങ്ങി­യ­താ­ണ് പര­മ്പ­ര. ആദ്യഭാഗത്തില്‍ ഐ­പി­എല്‍ 
ഉണ്ടാ­ക്കാ­നി­ട­യായ സാ­ഹ­ച­ര്യം വി­ല­യി­രു­ത്തി­യെ­ങ്കില്‍ രണ്ടും മൂന്നും ഭാ­ഗ­ങ്ങള്‍ വി­വിധ ടീ­മു­ക­ളു­ടെ സ്ട്രാ­റ്റ­ജി­യും നയ­ങ്ങ­ളും 
ലക്ഷ്യ­ങ്ങ­ളു­മാ­ണ് വി­ല­യി­രു­ത്തി­യ­ത്. ഐപി­എല്‍ ഇന്ത്യന്‍ സ്പോര്‍­ട്സ് വ്യ­വ­സാ­യ­ത്തി­ന് നല്‍­കിയ ഏറ്റ­വും 
പ്ര­ധാ­ന­സം­ഭാ­വ­ന­യെ­ക്കൂ­ടി വി­ല­യി­രു­ത്തി ഈ ­പ­ര­മ്പ­ര അവ­സാ­നി­ക്കു­ക­യാ­ണ്.''
\end{quotation}

{\vskip 12pt}

ഏ­റ്റ­വും വലിയ സം­ഭാ­വ­ന­യെ­ന്തെ­ന്നു­ള്ള ചോ­ദ്യ­ത്തി­ന് പല­രും പല ഉത്ത­ര­ങ്ങ­ളും നല്‍­കു­മാ­യി­രി­ക്കും. എന്റെ കണ­ക്കില്‍, 
കാ­യിക വി­നോ­ദ­വ്യ­വ­സാ­യ­ത്തില്‍ കാ­യിക വി­നോ­ദ­മാ­ണ് വ്യ­വ­സാ­യ­വ­ത്ക­രി­ക്ക­പ്പെ­ടു­ന്ന­ത്. ജന­കീയ കാ­യിക രൂ­പ­ങ്ങ­ളി­ലെ 
­വി­നോ­ദം­ മൈ­താ­ന­ത്തി­ലെ കളി­യില്‍­നി­ന്ന് ഏറെ­യൊ­ന്നും മുന്‍­പോ­ട്ടു പോ­യി­ട്ടി­ല്ല. വ്യ­വ­സാ­യ­വ­ത്ക­രി­ക്കു­മ്പോള്‍ ഇതൊ­രു 
പ്ര­ശ്ന­മാ­ണ്, കാ­ര­ണം കളി­കാ­ണാന്‍ മാ­ത്ര­മാ­യി വരു­ന്ന സ്റ്റേ­ഡി­യ­ത്തി­ലെ കാ­ണി­ക­ളും, ­ടെ­ലി­വി­ഷന്‍ പ്രേ­ക്ഷ­ക­രും അതില്‍ 
നി­ന്നു­ള്ള വരു­മാ­ന­വും പരി­ധി­യു­ള്ള­താ­ണ്. അതി­നാല്‍­ത്ത­ന്നെ കമ്പോ­ള­ത്തില്‍ എല്ലാ­വര്‍­ക്കും വേ­ണ്ട വര്‍­ഷാ­വര്‍­ഷം പു­തു­ക്കിയ 
രണ്ട­ക്ക വളര്‍­ച്ചാ­നി­ര­ക്ക് (ഇ­ക്കൊ­ല്ല­ത്തെ വളര്‍­ച്ചാ­നി­ര­ക്ക് കഴി­ഞ്ഞ­കൊ­ല്ല­ത്തേ­ക്കാള്‍ കു­റ­ഞ്ഞാല്‍ പോ­ലും കമ്പോ­ളം 
വേ­വ­ലാ­തി­പ്പെ­ടും :)) എന്ന­ത് ഒരു ഉട്ടോ­പ്യ­യാ­യി മാ­റും. ഐപി­എല്‍ ഫ്രാ­ഞ്ചൈ­സി­കള്‍ ലി­സ്റ്റ് ചെ­യ്യാന്‍ (അ­തു­വ­ഴി കൂ­ടു­തല്‍ 
പണം സ്വ­രൂ­പി­ക്കാ­നും) കഷ്ട­പ്പെ­ടും. അതി­ന് ഐപി­എല്‍ കണ്ട പ്ര­തി­വി­ധി, കാ­യിക വി­നോ­ദ­ത്തി­ലെ വി­നോ­ദ­ത്തി­നെ ഒന്നു 
കൂ­ടി വി­പു­ല­മാ­ക്കി, ഗ്രൌ­ണ്ടില്‍ നട­ക്കു­ന്ന കാ­യി­ക­മ­ത്സ­ര­വു­മാ­യി യാ­തൊ­രു ബന്ധ­വു­മി­ല്ലാ­താ­ക്കു­ക­യാ­യി­രു­ന്നു­.

­പ­ണം ചെ­ല­വാ­ക്കു­ന്ന ആളു­ക­ളു­ടെ സെ­ഗ്മെ­ന്റ് എടു­ത്തു നോ­ക്കി­യാല്‍, ഏറ്റ­വും വലിയ ധൂര്‍­ത്തന്‍­മാര്‍ 'യ­ങ് അര്‍­ബന്‍ 
മി­ഡില്‍­ക്ലാ­സ്' ആണെ­ന്നു­കാ­ണാം. അവ­രെ മു­ഴു­വന്‍ ഉള്‍­ക്കൊ­ള്ളാന്‍ ഇന്ത്യന്‍ ­ക്രി­ക്ക­റ്റ് സ്റ്റേ­ഡി­യ­ങ്ങള്‍­ക്കു കഴി­യി­ല്ല. മാ­ത്ര­മ­ല്ല, 
ക്രി­ക്ക­റ്റി­നോ­ട് ഭ്രാ­ന്ത­മായ ആവേ­ശ­മി­ല്ലാ­ത്ത, പ്രീ­മി­യര്‍ ലീ­ഗും, എന്‍­ബി­എ­യും, ഫോര്‍­മുല വണ്ണും പി­ന്തു­ട­രു­ന്ന ഒരു വലിയ വി­ഭാ­ഗം 
അവര്‍­ക്കി­ട­യി­ലു­ണ്ട്. പല­പ്പോ­ഴും, ഈ സെ­ഗ്മെ­ന്റി­ലെ ഏറ്റ­വും സമ്പ­ന്ന വി­ഭാ­ഗം ഇവ­രാ­ണു­താ­നും. ഇവര്‍ 
പണ­മൊ­ഴു­ക്കി­ത്തു­ട­ങ്ങി­യാ­ലെ, നി­ശ്ചി­ത­വ­രു­മാ­ന­ത്തില്‍ നി­ന്നും എക്സ്‌­പൊ­ണെന്‍­ഷ്യല്‍ രീ­തി­യില്‍ വള­രാന്‍ ഐപി­എ­ല്ലി­നു 
കഴി­യൂ. അതി­നാല്‍ അവ­രു­ടെ ­പാര്‍­ട്ടി­ സമ­യ­ങ്ങള്‍­ക്കും കൂ­ടി സമാ­ന­മാ­യാ­ണ് ഇക്കൊ­ല്ല­ത്തെ ഐപി­എല്‍ മാ­ച്ചു­കള്‍ 
നി­ശ്ച­യി­ച്ചി­രു­ന്ന­ത്.

%image courtesy: http://blogs.rediff.com/aashirvaad09/

­പ്ര­വര്‍­ത്തി­ദി­ന­ങ്ങ­ളില്‍ ഒരു കളി, വൈ­കി 8 മണി­ക്കു തു­ട­ങ്ങു­ന്നു. വാ­രാ­ന്ത്യ­ങ്ങ­ളില്‍ രണ്ടു കളി ഒന്നു നാ­ലു­മ­ണി­ക്കും മറ്റേ­ത് 
എട്ടു­മ­ണി­ക്കും. നട്ടു­ച്ച­യ്ക്കു കളി­ന­ട­ത്തി­യാ­ലും ഗ്രൌ­ണ്ട് നി­റ­യു­ന്ന ഇന്ത്യ­യില്‍, ഇത് പ്ര­ധാ­ന­മാ­യും പാര്‍­ട്ടി പ്രേ­ക്ഷ­ക­രെ ലക്ഷ്യം 
വച്ചാ­ണെ­ന്നു­ള്ള­തു വ്യ­ക്തം (ടെ­ലി­വി­ഷ­നില്‍ പ്രൈം ടൈം ആണ­ത്, കു­ടും­ബ­ക­ല­ഹം ഉണ്ടാ­ക്കാന്‍ പോ­ന്ന കാ­ര്യം­!).

ഇ­തി­നോ­ടൊ­പ്പം തന്നെ, എന്റര്‍­ടൈന്‍­മെ­ന്റ് സ്പോര്‍­ട്സ് ഡയ­റ­ക്റ്റു­മാ­യി സഹ­ക­രി­ച്ച് റോ­യല്‍ ചാ­ല­ഞ്ചര്‍ സ്പോര്‍­ട്സ് 
മി­ത­മായ പര­സ്യ­ങ്ങ­ളു­മാ­യി പബ്ബു­കള്‍­ക്കും സ്പോര്‍­ട്സ് ബാ­റു­കള്‍­ക്കും നല്‍­കിയ ഉഗ്രന്‍ ഫീ­ഡും കണ­ക്കി­ലെ­ടു­ക്ക­ണം. 
ടെ­ലി­വി­ഷന്‍ പ്രേ­ക്ഷ­ക­ന്റെ ഒരു ഭാ­ഗ­മാ­യി പബ്ബ്/­സ്പോര്‍­ട്സ് ബാര്‍ പ്രേ­ക്ഷ­ക­രെ കാ­ണാ­തെ, പ്ര­ത്യേ­ക­മാ­യി­ത്ത­ന്നെ 
പരി­ഗ­ണി­ച്ചി­രു­ന്നു എന്നാ­ണി­തു കാ­ണി­ക്കു­ന്ന­ത്. ക്രി­ക്ക­റ്റി­നൊ­പ്പം, പൂ­ളും, ബൌ­ളി­ങ്ങും, ഹി­പ് ഹോ­പ്പും, പി­ന്നെ മല്യ­യു­ടെ മദ്യ­വും. 
ഇത്ര­യും ക്രി­ക്ക­റ്റി­നെ ടൌ­ണി­ലെ അടി­ച്ചു­പൊ­ളി പി­ള്ളാ­രു­ടെ ഡെ­യ്‌­ലി റൊ­ട്ടീ­നില്‍ ഉള്‍­പ്പെ­ടു­ത്താ­നു­ള്ള കളി­കള്‍. ഇവി­ടെ 
പ്ര­ധാ­ന­മാ­യും മെ­ട്രോ­ക­ളി­ലെ­യും രണ്ടാം നിര നഗ­ര­ങ്ങ­ളി­ലെ­യും അപ്പര്‍ മി­ഡില്‍ ക്ലാ­സ്സി­ലെ, ക്രി­ക്ക­റ്റ് അലര്‍­ജി­ക്കാ­രെ­യാ­ണ് 
ലക്ഷ്യം വച്ച­ത്. നി­റ­ഞ്ഞ പബ്ബു­കള്‍ ഇതൊ­രു വന്‍ വി­ജ­യ­മാ­യി­രു­ന്നു­വെ­ന്ന­തി­നു തെ­ളി­വാ­ണ്.

%image courtesy: http://bollywoodnewsstories.blogspot.com/2010/03/ipl-signature-after-match-party-at-ub.html

­ക­ളി നട­ക്കു­ന്ന നഗ­ര­ങ്ങ­ളില്‍ മത്സ­ര­ത്തി­നു ശേ­ഷം നട­ക്കു­ന്ന പാര്‍­ട്ടി­ക­ളും ഫാ­ഷന്‍ ഷോ­ക­ളും ലക്ഷ്യം വയ്ക്കു­ന്ന­ത് അതി 
സമ്പ­ന്ന­രു­ടെ സോ­ഷ്യല്‍ ലൈ­ഫില്‍ ക്രി­ക്ക­റ്റി­നു ഇടം നല്‍­കു­ക­യെ­ന്നാ­ണ്. ക്രി­ക്ക­റ്റ് മൈ­താ­ന­ത്തി­ന്റെ ഒരു­മു­ക്കില്‍ 
തു­ള്ളി­ച്ചാ­ടാ­നെ­ന്ന പേ­രില്‍ കൊ­ണ്ടു­വ­രു­ന്ന ചി­യര്‍­ഗേള്‍­സും, ഈ പാര്‍­ട്ടി­ക­ളില്‍ കു­റ­ച്ചു ചി­യര്‍ എക്സ്ട്രാ കൊ­ണ്ടു­വ­രാ­നു­ള്ള­താ­ണ്. 
ഇന്ത്യന്‍ പാര്‍­ട്ടി സര്‍­ക്കി­ളി­ലെ, ഹൂ­സ് ഹൂ ആയ ഷാ­രൂ­ഖ്-ഗൌ­രി ഖാന്‍, വി­ജ­യ് മല്ല്യ, ഷെ­ട്ടി സി­സ്റ്റേ­ഴ്സ്, പ്രീ­തി സി­ന്റ, 
നിത-മു­കേ­ഷ് അം­ബാ­നി, ഇവ­രു­ടെ­യൊ­ക്കെ പാര്‍­ട്ടി­ക­ളില്‍ ക്ഷ­ണി­ക്ക­പ്പെ­ട്ടാല്‍ അതു നല്‍­കു­ന്ന സോ­ഷ്യല്‍ മൈ­ലേ­ജ് ഈ 
പാര്‍­ട്ടി­ക­ളെ ഗം­ഭീ­ര­മാ­ക്കു­ന്നു. ഇപ്രാ­വ­ശ്യം പാര്‍­ട്ടി­ക­ളു­ടെ അതി­പ്ര­സ­രം കാ­ര­ണം പല ക്രി­ക്ക­റ്റര്‍­മാ­രും, 'പ്ലീ­സ് ഇന്നെ­ന്നെ 
ഒഴി­വാ­ക്കൂ' എന്നു പറ­ഞ്ഞ­താ­യും കേള്‍­ക്കു­ന്നു­ണ്ട്. പണ്ട് "പൂ­ച്ച­ക്കൊ­രു മൂ­ക്കു­ത്തി­യി­ലെ" സു­കു­മാ­രി­യെ­പ്പോ­ലെ ക്രി­ക്ക­റ്റ് 
കാ­ണു­ന്ന­വ­രു­മാ­യി ഒരു രാ­ത്രി­മു­ഴു­വന്‍ ചി­ല­വ­ഴി­ക്കു­ന്ന­തോര്‍­ക്കു­മ്പോള്‍ കളി­ക്കാര്‍­ക്കു മു­ട്ടി­ടി­ക്കു­ന്ന­താ­കും! തമാശ ഒഴി­വാ­ക്കി­യാല്‍, 
കാ­ല­ങ്ങ­ളാ­യി, ജന­ല­ക്ഷ­ത്തി­ന്റെ കളി എന്നു പറ­ഞ്ഞ് ക്രി­ക്ക­റ്റി­നെ ഒഴി­വാ­ക്കി­യി­രു­ന്ന­വ­രെ­ക്കൂ­ടി പ്ര­ധാന 
പ്രേ­ക്ഷ­ക­രാ­ക്കി­യെ­ടു­ക്കു­ന്ന­തി­നാ­ണ് ഈ ക്രി­ക്ക­റ്റ് വി­നോ­ദ­ത്തില്‍ നി­ന്നും ക്രി­ക്ക­റ്റ് ഒഴി­വാ­ക്കിയ പരി­പാ­ടി സഹാ­യി­ച്ച­ത്, 
അതി­ലൂ­ടെ കോ­ടി­ക­ളു­ടെ വരു­മാ­ന­വും­.

ഐ­പി­എല്‍ ഇന്ത്യന്‍ സ്പോര്‍­ട്സ് വ്യ­വ­സാ­യ­ത്തി­നു നല്‍­കിയ ഏറ്റ­വും വലിയ സം­ഭാ­വന ഇതാ­ണ്. ഒരു ജന­കീയ കാ­യിക 
രൂ­പ­മാ­യ­തി­നാല്‍ ക്രി­ക്ക­റ്റ് മെ­ട്രോ­ക­ളി­ലെ ഉപ­രി­വര്‍­ഗ്ഗ പാര്‍­ട്ടി സര്‍­ക്കി­ളു­ക­ളില്‍ നേ­രി­ട്ടി­രു­ന്ന അയി­ത്തം ഒഴി­വാ­ക്കാന്‍ 
ഐ പി എല്ലി­നു കഴി­ഞ്ഞു. അതു­വ­ഴി, പതി­ന്മ­ട­ങ്ങു വരു­മാ­ന­വും. ജന­കീയ കാ­യിക വി­നോ­ദ­ത്തെ എക്സ്‌­ക്ലൂ­സി­വി­റ്റി­യു­ടെ ലോ­ക­ത്ത് 
പ്ര­തി­ഷ്ഠി­ക്കു­ന്ന­തെ­ങ്ങ­നെ­യെ­ന്നാ­ണ് ഐപി­എല്‍ കാ­ണി­ച്ചു തന്ന­ത്.

(13 May 2010)\footnote{http://malayal.am/പലവക/പരമ്പര/ബിസിനസ്-ലീഗ്/5426/ഐപിഎല്‍-ആഫ്റ്റര്‍-മാച്ച്-പാര്‍ട്ടി}

\newpage
