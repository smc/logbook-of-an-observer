\secstar{വലന്‍സിയയിലെ അപകടവും വെബ്ബറിന്റെ രക്ഷപ്പെടലും}
\vskip 2pt

ആദ്യപത്തില്‍ സീസണിലെ പതിവില്‍നിന്നും വ്യത്യസ്തമായി ചില പുതുമുഖങ്ങളെ കണ്ട റേസായിരുന്നു വലന്‍സിയയിലേതു്. 
യൂറോപ്യന്‍ ഗ്രാന്‍പ്രീ എന്ന പേരുമായി, ഫോര്‍മുല വണ്‍ ചാമ്പ്യന്‍ഷിപ്പിന്റെ ഒന്‍പതാം റൌണ്ടു് ഇക്കഴിഞ്ഞ ഞായറാഴ്ച (27
ജൂണ്‍) സ്പെയിനിലെ വലന്‍സിയയിലെ തെരുവുകളില്‍ പൂര്‍ത്തിയായപ്പോള്‍ ലോകചാമ്പ്യന്‍ഷിപ്പ് പോരാട്ടങ്ങള്‍ മക്‌ലാരന്‍,
റെഡ്ബുള്‍ ടീമുകളിലേക്കൊതുങ്ങുന്ന കാഴ്ചയാണു് കാണാനാവുന്നതു്. കാനഡയില്‍ കാഴ്ചവെച്ച പ്രകടനത്തിനിന്നും 
ഒരുപാടു മുന്നോട്ടുപോകാന്‍ വില്യംസ്, സൌബര്‍,ടോറോ റോസൊ ടീമുകള്‍ക്കായതു്, മധ്യനിര പോരാട്ടം വരുംനാളുകളില്‍
ശക്തമാകുമെന്നും വ്യക്തമാക്കി.

ടയറുകള്‍ കുഴപ്പങ്ങളുണ്ടാക്കിയ യോഗ്യതാ റൌണ്ടായിരുന്നു വലന്‍സിയയിലേതു്. ശരിയായ ടയര്‍ സ്ട്രാറ്റജിയിലൂടെ മക്‌ലാരന്‍
റെഡ്ബുള്‍ ടീമുകള്‍ മുന്‍നിരയിലെത്തിയപ്പോള്‍, റെഡ്ബുള്‍ സെബാസ്റ്റ്യന്‍ വെറ്റലിലൂടെ സീസണിലെ എട്ടാമത്തെ പോള്‍ 
ഉറപ്പാക്കി. ഹാമില്‍ട്ടണ്‍ മൂന്നാമതെത്തിയെങ്കിലും, ഒന്നാം നമ്പര്‍ കാറില്‍ ഏഴാമതെത്താനെ നിലവിലെ ചാമ്പ്യന്‍ ജെന്‍സണ്‍
ബട്ടണു കഴിഞ്ഞുള്ളൂ. ഫെറാരികള്‍ നാലും അഞ്ചും സ്ഥാനത്തും കുബിത്സ ആറാമതുമെത്തി. എന്നാല്‍ യോഗ്യതാ റൌണ്ടിന്റെ
അത്ഭുതമായതു്, സീസണില്‍ തീരെ നിറംമങ്ങിപ്പോയ വില്യംസ് കാറുകള്‍ ഒരേസമയത്തോടെ എട്ടും ഒന്‍പതും 
സ്ഥാനങ്ങളിലെത്തിയതാണു്. ടയറുകള്‍ ചതിച്ച മെഴ്സിഡസും ഫോഴ്സ് ഇന്ത്യയും പതിവിനു വിപരീതമായി, യോഗ്യതാ റൌണ്ടിന്റെ
മൂന്നാംപാദം കാണാതെ പുറത്തായി.

റേസില്‍ അത്യുഗ്രന്‍ ഒരു സ്റ്റാര്‍ട്ടിലൂടെ ഹാമില്‍ട്ടണ്‍ വെബ്ബറെ മറികടന്നു (ടയറുകള്‍ തമ്മിലുരസിയില്ലായിരുന്നുവെങ്കില്‍ 
വെറ്റലിനേയും രണ്ടാം വളവിനടുത്തുവച്ചു് മറികടക്കുമായിരുന്നു). ട്രാക്കിലെ പൊസിഷന്‍ നഷ്ടപ്പെട്ട വെബ്ബര്‍ 
ആദ്യലാപ്പുകഴിഞ്ഞപ്പോള്‍ ഏഴാമതായി. കാനഡയിലെ അത്രയും മികച്ചതല്ലെങ്കിലും കുഴപ്പമില്ലാത്ത ഒരു സ്റ്റാര്‍ട്ടിലൂടെ 
ഷുമാക്കര്‍ പതിനൊന്നാമതെത്തിയപ്പോള്‍, കൂട്ടുകാരന്‍ റൊസ്ബര്‍ഗ് മോശമായിപ്പോയി. യാനോ ട്രൂലി അഞ്ചു് ലാപ്പു 
കഴിഞ്ഞപ്പോള്‍ത്തന്നെ രണ്ടു് പിറ്റ് സ്റ്റോപ്പുകളെടുത്തു് ലോട്ടസിന്റെ റിലയബിലിറ്റിയെക്കുറിച്ചൊരു സൂചന നല്‍കി. 
വെബ്ബറാകട്ടെ എട്ടാം ലാപ്പില്‍ പിറ്റ് ചെയ്തു് ടയറുകള്‍ മാറ്റി മറ്റു കാറുകള്‍ (നിയമപ്രകാരം റേസില്‍ ഓപ്ഷന്‍ ടയറുകളും 
ഹാര്‍ഡ് ടയറുകളും നിര്‍ബന്ധമായും ഉപയോഗിച്ചിരിക്കണം) പിറ്റു ചെയ്യുമ്പോള്‍ പൊസിഷന്‍ തിരിച്ചുപിടിക്കാനുള്ള ശ്രമം 
നടത്തി. എന്നാല്‍ വളരെ മോശം ഒരു പിറ്റ് സ്റ്റോപ്പിലൂടെ ട്രാക്കിലുള്ള മുന്‍തൂക്കം നഷ്ടമാവുകയാണു് ചെയ്തതു്. 
തൊട്ടതെല്ലാം പിഴച്ച വെബ്ബര്‍ പത്താം ലാപ്പില്‍ ലോട്ടസിന്റെ ഹൈക്കി കൊവലായ്‌നെനുമായി കൂട്ടിയിടിച്ചു് പുറത്തുപോവുകയും ചെയ്തു. 
ഒരു സാധാരണ മറികടക്കലിനിടയില്‍ വെബ്ബറെ പ്രതിരോധിക്കാന്‍ ശ്രമിച്ച ലോട്ടസിന്റെ പിന്നില്‍ത്തട്ടി 
റെഡ്ബുള്‍ ട്രാക്കില്‍ ശരിക്കും തലകുത്തിമറിയുകതന്നെയായിരുന്നു (വീഡിയോ കാണുക). അത്ഭുതകരമായാണു് 
തകര്‍ന്നുപോയ റെഡ്ബുള്‍ കാറില്‍നിന്നു് മാര്‍ക് വെബ്ബര്‍ യാതൊരു പരിക്കുമില്ലാതെ രക്ഷപ്പെട്ടതു്.

ഇതായിരുന്നു, അല്ലെങ്കില്‍ വിരസമെന്നു പറയാവുന്ന റേസിലെ ടേണിങ് പോയിന്റ്. അപകടത്തിനുശേഷം സേഫ്റ്റികാര്‍ 
വരുമെന്നുറപ്പായതോടെ എല്ലാ മുന്‍നിരകാറുകളും ഒന്നിനുപിറകേ ഒന്നായി പിറ്റ് ചെയ്തു് ഹാര്‍ഡ് ടയറുകളിലേക്കുമാറി. 
സേഫ്റ്റികാറിനു പിന്നില്‍ ഫോര്‍മേഷന്‍ നടക്കുന്നതിനു മുന്‍പുതന്നെ, പിറ്റെടുത്തു് പൊസിഷന്‍ നിലനിര്‍ത്താനുള്ള 
ശ്രമത്തിന്റെ ഭാഗമായിരുന്നു ഇതു്. ഇതിന്റെ ഫലമായി, ഷുമാക്കറും കൊബിയാഷിയും മുന്‍നിരയിലെത്തുകയും ചെയ്തു. 
എന്നാല്‍ ശരിക്കും മുന്നിലോടുകയും ഹാമില്‍ട്ടണു കനത്ത വെല്ലുവിളിയുയര്‍ത്തുകയും ചെയ്തിരുന്ന ഫെറാരികള്‍ ഇവിടെ 
സേഫ്റ്റികാറിനു പിന്നില്‍പ്പെട്ടുപോയി. സേഫ്റ്റികാറിനെ അവഗണിച്ച ഹാമില്‍ട്ടണാകട്ടെ ഒരു അഞ്ചു സെക്കന്റ് ഡ്രൈവു് ത്രൂ 
പെനാല്‍ട്ടിയുമായി രക്ഷപ്പെടുകയും ചെയ്തു. സേഫ്റ്റികാര്‍ നിയമങ്ങള്‍ക്കു് ശരിക്കും കനത്ത വിലകൊടുക്കേണ്ടിവന്നതു് 
മെഴ്സിഡസാണു്. പിറ്റ്ലേനില്‍ റെഡ് ലൈറ്റ് കിട്ടിയ ഷുമാക്കര്‍ മൂന്നാമതുനിന്നു് പത്തൊന്‍പതാമനായാണു് പുറത്തെത്തിയതു്. 
പിന്നീടു് സോഫ്റ്റ് ടയറുകളെ മാറ്റാനായി ഒന്നുകൂടി പിറ്റ് ചെയ്തു് ഷുമാക്കര്‍ ഇരുപത്തിയൊന്നാമതായി. എന്നാല്‍ പിറ്റ് 
സ്റ്റോപ്പു് അവസാനംവരെ എടുക്കാതിരുന്ന കൊബിയാഷി ഏതാണ്ടു് റേസിന്റെ അവസാനംവരെ മൂന്നാമതായിരുന്നു. 
പിന്നീടു് പിറ്റ് ചെയ്തു് ഓപ്ഷന്‍ ടയറുകളിലേക്കുമാറി ഏഴാമതായി ഫിനിഷ് ചെയ്തു.

വെബ്ബറിന്റെ അപകടവും തുടര്‍ന്നുണ്ടായ ബഹളവും ശരിക്കും മുതലാക്കിയത് മധ്യനിര ടീമുകളാണു്. വില്യംസിന്റെ 
ബാരിക്കെല്ലോയും റെനോയുടെ കുബിത്സയും ടോറോ റോസൊയുടെ ബ്യയെമിയും ഫോഴ്സ് ഇന്ത്യയുടെ സുട്ടിലും 
അഞ്ചുമുതല്‍ എട്ടുവരെ സ്ഥാനങ്ങളിലെത്തുകയും, ഏതാണ്ടു് അവസാനംവരെ നിലനിര്‍ത്തുകയും ചെയ്തു. (പിന്നീടു് 
യെല്ലോ ഫ്ലാഗ് നിയമങ്ങളെ അവഗണിച്ചെന്നു പറഞ്ഞു് ഇവര്‍ക്കെല്ലാം പെനാല്‍ട്ടിയും ലഭിച്ചു.) വില്യംസിന്റെ നികൊ 
ഹള്‍ക്കന്‍ബര്‍ഗ് റിട്ടയര്‍ചെയ്തതും, സൌബറിന്റെ പെഡ്രോ ഡി ലാ റൊസയ്ക്കു് പെനാല്‍ട്ടി കിട്ടിയതും നികൊ 
റൊസ്ബര്‍ഗിനു് ഒരു ആശ്വാസ പത്താംസ്ഥാനം നല്‍കി. ഇവിടെ ഒന്നാമതായി ഫിനിഷ് ചെയ്തെങ്കിലും സെബാസ്റ്റ്യന്‍ 
വെറ്റല്‍ ചാമ്പ്യഷിപ്പു് പോരാട്ടത്തില്‍ 115 പോയിന്റുമായി മൂന്നാമതാണു്. മക്‌ലാരന്റെ ഹാമില്‍ട്ടണ്‍ 127 പോയിന്റുമായി 
ഒന്നാമതും, വെറും ആറുപോയിന്റു വ്യത്യാസത്തില്‍ ബട്ടണ്‍ രണ്ടാമതുമാണു്. വലന്‍സിയയില്‍ പോയിന്റൊന്നും 
നേടിയില്ലെങ്കിലും വെബ്ബര്‍ 103 പോയിന്റുമായി നാലാമതുണ്ടു്. അലൊണ്‍സോ 98 പോയിന്റുമായി അഞ്ചാമതാണു്. 
കണ്‍സ്ട്രക്റ്ററുമാരുടെ പോരാട്ടത്തില്‍ മക്‌ലാരന്‍ (248) തന്നെയാണു മുന്നില്‍. കനത്ത വെല്ലുവിളിയുമായി റെഡ്ബുള്‍ 
തൊട്ടുപിറകിലുണ്ടു് (218). എന്നാല്‍ മൂന്നാമതുള്ള ഫെറാരിയ്ക്ക് ഇപ്പോഴത്തെ പ്രകടനത്തില്‍നിന്നും 
ഒരുപാടു മുന്നോട്ടുപോയെ മതിയാകു.

പോയിന്റ് നിലയില്‍നിന്നും ഇതുവരെയുള്ള റേസ് അനുഭവങ്ങളില്‍നിന്നും മനസ്സിലാകുന്നതു്, മുനിരയേക്കാള്‍ കനത്ത 
പോരാട്ടം മധ്യനിരയിലാകുമെന്നാണു്. വില്യംസ്, സൌബര്‍ ടോറോ റോസോ ടീമുകള്‍ കൂടി കരുത്തറിയിച്ചു കഴിഞ്ഞതോടെ
വരുംആഴ്ചകളില്‍ യൂറോപ്പിലെ ട്രാക്കുകളില്‍ തീപാറുമെന്നുറപ്പിക്കാം.വലന്‍സിയയിലെ റേസിനിടയില്‍ മെഴ്സിഡസ് 
കാറുകളുടെ ബ്രേക്കുകള്‍ അമിതമായി ചൂടായിരുന്നതു്, ഈ സീസണിലെ റിലയബിലിറ്റി പ്രശ്നങ്ങളില്‍നിന്നും അവര്‍ ഇനിയും 
മുക്തരായിട്ടില്ലെന്നതിനു തെളിവായി.

ഫോര്‍മുല വണ്‍ ലീഡര്‍ ബോര്‍ഡില്‍ മുന്നിലുള്ള മക്‌ലാരന്റെയും ഹാമില്‍ട്ടണിന്റേയും, നിലവിലെ ചാമ്പ്യന്‍ ബട്ടണിന്റേയും
ഹോം റേസാണു് ജൂലൈ രണ്ടാംവാരത്തില്‍. ഒരുപാടു കനത്തപോരാട്ടങ്ങള്‍ക്കു വേദിയായിട്ടുള്ള സില്‍വര്‍സ്റ്റോണ്‍ 
ഇത്തവണയും നിരാശരാക്കില്ലെന്നു പ്രതീക്ഷിക്കാം.

\hspace*{2em}(30 June, 2010)\footnote{http://malayal.am/വിനോദം/കായികം/6467/വലന്‍സിയയിലെ-അപകടവും-വെബ്ബറിന്റെ-രക്ഷപ്പെടലും}

\newpage
