\secstar{Hospital Log 5}
\vskip 2pt
\begin{english}
\date{Friday November 19 2010}

It has been 10 days since I got discharged after my first chemotherapy. I was supposed to get admitted last Sunday (14th). However, India being a country of a billion people, it has less than the adequate medical facilities to cater to all of them. Hence, getting a bed in an hospital which excels in treatment for these kind of serious diseases was a headache. I am hoping to get a bed sometime this weekend! To add more to the news, my sis flew from the US to be with me for 10 days. She will be here till the 30th of this month. 

Today, she asked me something like, "Why don't you write down your experiences as an ALL patient?". I guess she meant something like a memoir. She knows that I write certain columns and stuff. She added, "I like to write, but I can't. If you can and you like do it, that's good".

I didn't tell her that I was scribbling down something like this. Instead, I asked her, "What should I write?". ``The problem'', I said, ``Is that, cancer patient memoirs with stories of recovery is a cliché these days. It won't be of interest to any reader and no one will like to publish it either''.

Then she said, ``The things you write now cater to a very specific set of people, and still, you have readers. So, there will be someone or the other to read this too''. I replied, ``A book of 100 pages of content means almost 25000 words. Where am I going to get that much content from?''. 

Her answer was interesting. She said, ``Write about our village and our life there''. She might have meant how our upbringing affected our life as well. But again that is a cliché. There is no novelty factor in the content there to give out any new thoughts.

I don't know why I want to write novel stuff. I know that most of the things I have written in these pages might have found its place in someone else's mind and pen too. The main reason for that is, the kind of novelty we believe to exist (something new and afresh), actually doesn't. Especially in writings of my kind, it is quite difficult to account for the novelty factor. Sometimes trivial matters gain the status of novelty. I will take you through my writings to show you why there is no novelty in my content. 

The content I wrote so far was read by a doctor. He told me that it is in line with a normal cancer patient's thoughts. So, is there any justice in me seeking novelty? Or is there any good if the content I create ends up being just another cliché because I am a cancer patient? The fact is, I haven't written a single thing in any of these pages which was not there in my mind for the past few years. At certain places, I used the latest evidences or proof of concepts I had. Then some realizations or understandings which I scribbled down and analyzed, those  which have got nothing to do with me being a cancer patient. Because that is something which I all the time. But still, whatever I write from hospital or during the course of my treatment will have the label of, "notes of a cancer patient". It is again the system which I refer to always. The one in which our minds and that too by large in society, like to put people into boxes to understand them better. Though my expressions are not just my reactions to the current situation, people like to see it so since its easy to analyze and ignore them that way.

\section{Me and my writings}

What I write is by no means a creative content. It never contains any decorative remarks or in-depth detailing. It is more or less an analysis, or my perspectives on issues, or my responses or reactions. Some of it are pure explanations of my experiments (as papers) or trouble shooting (blog posts). None of that has place for much creativity. Those follow a standard format like, presenting the problem, my views and logic, and analysis done. I applied the same format in these notes as well. Nothing more. First one was regarding my thoughts on some common question I get, its analysis and my perspective. Next one was on my family and friends, a detailed analysis and retrospection of how good my social assets are. The third one was again my thoughts about me, how my life has contrasting elements and how my existence defies certain odds, which, I think is thoroughly selfish but in no way anything creative. May be it includes retrospection, but the thing is, I've already done these retrospection a long time ago itself.

If I analyze my writings, I have written on a long range of subjects. Interestingly, I did not write much about technology, which is my line of work. I use my grip over faculties of observation and logical analysis to analyze different scenarios. Mostly, I review social and political situations. Another area where I put a lot of my time into, is sports. It seems I was the only regular formula-one analyst in Malayalam when I wrote. I also had spent some time writing about IPL. I have written a couple of critiques too. One of them (which I like most) is on media. Especially on diminishing influence of Editors in today's media and its effect in quality of the content produced. But everywhere, if I look back, I just applied the logic and observations with help of my a priori knowledge. My a priori knowledge is from my reading and experiences, which is not vast, but was enough to change a lot of my perspectives. On top, some of my acquaintances gave me new outlooks and first hand experiences on a variety of situations. 

As I mentioned somewhere before, I do have a lot of friends from different spheres of life. That does help me a lot in comprehending different situations. During my earlier days, I did not know anything about how news content was created/edited or anything of the sort. I got a lot of insight into this matter from my journalist friends. I also got the chance to interact and listen to first hand opinions of some of the respected intellectuals of Kerala by being part of some discussion groups. It made me aware about the fact that, behind the veil of the public figure, everyone in the world is more or less a normal human being. With all the prejudice as well as one's own conventions and opinions in different matters. The only difference is that some are aware of the audience while most of us are not. Some are aware to the extent that they try to keep the prejudices in check when they have to comment on a related subject. Recently one of them was angry when the discussions in a so called ``closed group'' came up in a public forum. However the thing is, the group itself is of 500+ members consisting of some of the lead journalists and subject experts. When you talk in this big a forum of journalists, I believe that the first thing which should be there in anybody's mind is that it is as good (or bad?) as a public forum.

I do have my own perspectives and opinions on many subjects. These are formed on the basis of my observations and understandings. I usually express them whenever possible and write articles on them as well. I do talk about these and take part in discussions on these subjects in various off/online forums. However, one thing I do care about, is being aware of my audience. Even then, many times when I write in discussion forums, I have been warned for the length of my posts. Not only that, but sometimes people often call it illegible (just like my handwriting) also. I once had to write two more long posts to explain one that I had written before. :) Other than being aware of the audience, what I usually try to look for is the simplest human factors. Most of the times, when something becomes news, we assume everything associated with it is acting out of place. Just because something is out of place, all details associated with it won't go crazy. Hence I just try to think through situations as a normal human would do.

So, my writings are simply nothing more than the most realistic (in my perspective) explanations to scenarios considering trivial factors. It is a little different because I am aware of my audience too. Other than that, there aren't much novel thoughts and ideas created by me. Yeah, I do channel thoughts of my readers to some other streams than the obvious one. But it is never any new channel. It is always through some path which were ignored by mainstream analysts or media because of their affiliations or biases. I just do what any honest critique or commentator does. Make the field balanced and in the process, very rarely do I have to be innovative, thanks to the inverted logics in the prejudices of our society.

\section{Novelty, publishing and more}

I started the note talking about novelty and my quest for novelty in the content. But after a retrospection of my writings, my approach to problems and the logics I apply, it seems, my writings are not so novel. My only novel contribution in it is putting it together. I should say, I am doing the role of a packager than creator/developer. At least in case of the kind of stuff I write, I believe there is no claim for creation of novel content. May be there is space for a novelty factor in packaging it all together. This is quite similar to the novelty factor I get for applying techniques from other domains in my own. The novelty in that process is actually for the thought to use the technology in a different domain, by being able to see some underlying similarities between the different domains. Anyway, I am not sure how novel it is when you point out the trivial facts in some problems. :) 

It is quite confusing. We have seen a lot of discussions and debates on protection for creative content from what many call `piracy'. People were called pirates for making copyrighted content available over World Wide Web. Interestingly, it was not the content generators who raised objections. But, the owners of the copyright -- publishing houses/record labels/studios -- were in forefront of the assault. They raised arguments of losing revenue considering each download as a lost purchase. They were never interested in checking the details of increase in audience and effective use of technology in content distribution. For them, in age of information explosion, the distribution of content should follow rules dating back to 18th and 19th century. 

Their concern, as they put it was, ``survival of the original content generators''. They argued that rewards of content generators will be hit if open access advocates were allowed to continue the distribution over WWW. The argument of rewarding the novelty factor in the generated content was the one which is still raised at a lot of forums. It even includes questions of survival and attraction of new talent. It seems people like to go with the argument that, discarding all the technological advances and possible new markets, we need to stick to age old rules of publishing that may be as old as Gutenberg himself. Amazingly, people support it because none favor a change in status qua. Tomorrow, if they become a writer, what will they do!

I used to believe that there is novelty involved in content generated. But, when I think about it now after a retrospection of the content I generated so far, there is not much which I can call as my own. Most of it is someone else's. My novel contribution most of the time is, the packaging. It might be different for creative work. If the argument is, reward is for the thought and initiative to packaging in the way I did, I am skeptical so as to how they ensure the cut of original content owners were served. Or is it like, only the packaging that is worth a reward and not the content as such? The answer, and quite truly too is that, it is practically impossible to always reward the content generators the way they are supposed to (or the way system claim it will). Then why all the fuss about rewarding novelty? Reward according to their worth. New methods to evaluate the worthiness may have to be developed, but I believe that is quite necessary in a business which is not ready to get out of their profit making ways back from 1800s and seeks legal protection for that citing the novelty factor and plight of content generators.   

\section{Half baked analysis and Questions!}

I do accept that analysing only my writings won't be the best method to trash the entire idea of generation of novel content. But, it is quite true that new ideas are hard to come by. Most of the times, ideas tagged new are just repackaged versions of existing ones. Still it is quite true that some 20 articles I have written should not be the base for the conclusion that there is no novelty in content. But I do believe it is quite right to raise the question, "How much of it is novel". On top, "while you reward packagers, what is the plan for original content generators?", should be examined as well. Another question is of effectively tapping into the new markets and opportunities in distribution, brought forth by advancements in technology.
\end{english}
\newpage
