\begin{english}
\section*{Free Software (Swathanthra Software)}

“Free software” is a matter of liberty, not price. To understand the concept, you should think of “free” as in “free speech”, not as in “free beer”.

\textbf{Free software is a matter of the users' freedom to run, copy, distribute, study, change and improve the software.} More precisely, it refers to four kinds of freedom, for the users of the software:
\begin{enumerate}
 \itemsep0em
 \item The freedom to run the program, for any purpose (freedom 0).
 \item The freedom to study how the program works, and adapt it to your needs (freedom 1). Access to the source code is a precondition for this.
 \item The freedom to redistribute copies so you can help your neighbor (freedom 2).
 \item The freedom to improve the program, and release your improvements to the public, so that the whole community benefits (freedom 3). Access to the source code is a precondition for this.
\end{enumerate}

Learn more about Free Software at \mbox{\url{www.gnu.org}}
\begin{center}
\line(2,0){150}
\end{center}
\end{english}
