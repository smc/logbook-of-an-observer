\secstar{ക്രിക്കറ്റ് കുടത്തിലെ ഭൂതം}
\vskip 2pt

‌\begin{quotation}
­``വി­വാ­ദ­ത്തി­ന്റെ കൂ­ട്ടു­കാ­ര­നാ­യി മാ­റിയ കു­ട്ടി­ക്ക്രി­ക്ക­റ്റി­ന്റെ അന്താ­രാ­ഷ്ട്ര പതി­പ്പില്‍ നി­ന്ന് 'ഇ­ന്ത്യന്‍ സേ­ന' പു­റ­ത്താ­യ­ത് 
ഇന്ന­ലെ­യാ­ണ്. അപ്പോ­ഴും ഇവി­ടെ ഇന്ത്യ­യില്‍ ഐ­പി­എല്‍ വി­വാ­ദ­ത്തി­ന്റെ അല­യൊ­ലി­കള്‍ അട­ങ്ങി­യി­ട്ടി­ല്ല. 
കാ­ര­ണം­കാ­ണി­ക്കല്‍ നോ­ട്ടീ­സ് ലഭി­ച്ച ഐപി­എ­ല്ലി­ന്റെ മുന്‍ കമ്മീ­ഷ­ണര്‍ ലളി­ത് മോ­ഡി­ക്ക് മറു­പ­ടി പറ­യാ­നു­ള്ള 
സമ­യം ദീര്‍­ഘി­പ്പി­ച്ച് വി­വാ­ദ­ത്തി­ന്റെ ആയു­സ്സു­നീ­ട്ടു­ക­യാ­ണ് ബി­സി­സി­ഐ. ഇന്ത്യന്‍ ­ക്രി­ക്ക­റ്റ് ലീ­ഗ് എന്ന നവീന 
ആശ­യ­ത്തെ കട്ടെ­ടു­ത്ത് ഇന്ത്യന്‍ പ്രീ­മി­യര്‍ ലീ­ഗ് ആയി അവ­ത­രി­പ്പി­ച്ച ­ല­ളി­ത് മോ­ഡി­ ഒടു­വില്‍ കാ­യി­ക­രം­ഗ­ത്തെ 
അധോ­ലോ­ക­വാ­ഴ്ച­യു­ടെ അട­യാ­ള­മാ­യി മാ­റു­ന്നു. മൂ­ന്നു­കൊ­ല്ലം പി­ന്നി­ട്ട ഐപി­എ­ല്ലി­ന്റെ കഥ­കള്‍ അപ­സര്‍­പ്പ­ക­ക­ഥ­ക­ളെ 
പോ­ലും വെ­ല്ലും. ചതി­യു­ടെ­യും പക­യു­ടെ­യും അദ്ധ്യാ­യ­ങ്ങള്‍ നി­റ­ച്ച ആദ്യ രണ്ടു­വര്‍­ഷ­ത്തെ ഐപി­എല്‍ കാ­ലം 
അന്വേ­ഷി­ക്കു­ക­യാ­ണി­വി­ടെ­.  മല­യാ­ളം ന്യൂ­സ് പോര്‍­ട്ട­ലി­നു വേ­ണ്ടി ജി­നേ­ഷ് കെ­ജെ തയ്യാ­റാ­ക്കി­യ­ത്.''
\end{quotation}

{\vskip 12pt}

ഐ­പി­എ­ല്ലി­ന്റെ ചരി­ത്രം അന്വേ­ഷി­ക്കു­മ്പോള്‍ ചെ­ന്നെ­ത്തു­ന്ന­ത് സു­ഭാ­ഷ് ചന്ദ്ര­യു­ടെ എസ്സെല്‍ ഗ്രൂ­പ്പില്‍ പെ­ട്ട സീ ടെ­ലി­ഫി­ലിം­സ് 
2003 മു­തല്‍ ഇന്ത്യ­യി­ലെ ക്രി­ക്ക­റ്റ് സം­പ്രേ­ക്ഷ­ണാ­വ­കാ­ശ­ത്തി­നാ­യി നട­ത്തിയ ലേ­ല­യു­ദ്ധ­ങ്ങ­ളി­ലും നി­യ­മ­യു­ദ്ധ­ങ്ങ­ളി­ലു­മാ­ണ്. 
പല­പ്പോ­ഴും പൊ­തു­ജ­ന­ത്തെ അമ്പ­ര­പ്പി­ക്കു­ന്ന നട­പ­ടി­ക്ര­മ­ങ്ങ­ളി­ലൂ­ടെ സം­പ്രേ­ക്ഷ­ണാ­വ­കാ­ശം കി­ട്ടാ­ക്ക­നി­യാ­യ­പ്പോള്‍ ­
സു­ഭാ­ഷ് ചന്ദ്ര അസാ­ധ്യ­മായ ഒരു സാ­ഹ­സ­ത്തി­നു മു­തിര്‍­ന്നു. ബി­സി­സി­ഐ­യി­ലെ 'ക­ടല്‍ കി­ഴ­വന്‍­മാ­രു­ടെ' സം­ഘ­ത്തി­ന് 
ഒരു കോര്‍­പ്പ­റേ­റ്റ് ബദല്‍ എന്ന സ്വ­പ്ന­ത്തി­നു നി­റം പക­രാന്‍ ശ്ര­മി­ച്ചു. ഇതു ചെ­യ്യു­മ്പോള്‍ ഇന്ത്യ­യി­ലെ ക്രി­ക്ക­റ്റ് ഭര­ണം 
നേ­രെ­യാ­ക്കി­യെ­ടു­ക്ക­ണം എന്നൊ­രു­ദ്ദേ­ശം ചന്ദ്ര­യു­ടെ സ്വ­പ്ന­ത്തില്‍ പോ­ലു­മു­ണ്ടാ­യി­രു­ന്നു എന്നു തോ­ന്നു­ന്നി­ല്ല. ഒരു പക്ഷേ 
ടെ­ലി­വി­ഷന്‍ സം­പ്രേ­ഷ­ണാ­വ­കാ­ശ­ങ്ങ­ളു­ടെ വില്‍­പ്പ­ന­യി­ലൂ­ടെ ഫോര്‍­മുല വണ്ണി­ന്റെ അവ­സാ­ന­വാ­ക്കാ­യി മാ­റിയ ബെര്‍­ണി 
എക്‌­ലെ­സ്റ്റോ­ണാ­യി­രു­ന്നി­രി­ക്ക­ണം ചന്ദ്ര­യു­ടെ (പി­ന്നീ­ട് ലളി­ത് മോ­ഡി­യു­ടെ­യും) പ്ര­ചോ­ദ­നം­.

എ­ന്താ­യാ­ലും എന്റെ ചാ­ന­ലു­കള്‍­ക്ക് കാ­ണി­ക്കാന്‍ അവര്‍ ക്രി­ക്ക­റ്റ് തരു­ന്നി­ല്ല, അതു­കൊ­ണ്ട് ഞാന്‍ സ്വ­ന്ത­മാ­യി 
ക്രി­ക്ക­റ്റ് മത്സ­ര­ങ്ങള്‍ സം­ഘ­ടി­പ്പി­ക്കാന്‍ പോ­കു­ന്നു എന്ന് ­കെ­റി പാര്‍­ക്കര്‍ ശൈ­ലി­യില്‍ പറ­ഞ്ഞ് ചന്ദ്ര തു­റ­ന്നു­വി­ട്ട ഇന്ത്യന്‍ 
ക്രി­ക്ക­റ്റ് ലീ­ഗ് ഭൂ­തം ലോക ക്രി­ക്ക­റ്റി­ന്റെ മേ­ലാ­ളന്‍­മാ­രു­ടെ ഉറ­ക്കം കെ­ടു­ത്താന്‍ വലിയ താ­മ­സ­മു­ണ്ടാ­യി­ല്ല. ഐസി­എ­ല്ലു­മാ­യി 
സഹ­ക­രി­ക്കു­ന്ന എല്ലാ­വര്‍­ക്കും വി­ല­ക്കേര്‍­പ്പെ­ടു­ത്തി ­ബി­സി­സി­ഐ­ നയം വ്യ­ക്ത­മാ­ക്കി. ഇന്ത്യന്‍ ബോര്‍­ഡി­ന്റെ മണി­പ­വ­റി­നു 
മു­മ്പില്‍ ഐസി­സി­യും മറ്റു ബോര്‍­ഡു­ക­ളും മു­ട്ടു­മ­ട­ക്കി. ലീ­ഗു­മാ­യി സഹ­ക­രി­ക്കു­ന്ന എല്ലാ­വര്‍­ക്കും അം­ഗീ­കൃത വേ­ദി­ക­ളില്‍ നി­ന്നും 
വി­ല­ക്കു വന്നു.

­പ­ക്ഷെ, അവ­സ­ര­ങ്ങ­ളെ­ക്കാ­ളേ­റെ ഉദ്യോ­ഗാര്‍­ത്ഥി­ക­ളു­ള്ള ഇന്ത്യന്‍ വ്യ­വ­സ്ഥി­തി­ക്കു­ള്ളില്‍ നി­ന്നും ആറു തര­ക്കേ­ടി­ല്ലാ­ത്ത 
­ടീ­മു­ക­ളെ ഉണ്ടാ­ക്കാന്‍ ഐസി­എ­ല്ലി­നു സാ­ധി­ച്ചു. കപില്‍ ദേ­വി­ന്റെ­യും ഡീന്‍ ജോണ്‍­സി­ന്റെ­യും മറ്റും മേല്‍­നോ­ട്ട­ത്തില്‍ 
ആവേ­ശ­ക­ര­മായ ഒരു സീ­സണ്‍ സം­ഘ­ടി­പ്പി­ക്കാന്‍ ചന്ദ്ര­യ്ക്കാ­യി. ചന്ദ്ര­യെ തളര്‍­ത്താന്‍ തങ്ങ­ളാ­ലാ­വും വി­ധം ബി­സി­സിഐ 
ശ്ര­മി­ച്ചു. എങ്കി­ലും ആദ്യ സീ­സണ്‍ കഴി­ഞ്ഞ­പ്പോള്‍ ബി­സി­സി­ഐ­യെ­ക്കൊ­ണ്ട് സ്വ­ന്തം 20-20 ലീ­ഗ് പ്ര­ഖ്യാ­പി­ക്കാന്‍ ചന്ദ്ര­യു­ടെ 
സാ­ഹ­സ­ത്തി­നു സാ­ധി­ച്ചു­.

%ഇ­ന്ത്യന്‍ ക്രി­ക്ക­റ്റ് ലീ­ഗി­ന്റെ സ്ഥാ­പ­ക­നായ 
%<a href=­സീ ടി­വി­ ഉടമ സു­ഭാ­ഷ് ചന്ദ്ര" title="­സു­ഭാ­ഷ് ചന്ദ്ര" style="margin-top:7px;margin-bottom:7px;" height="280" width="400" />

%(image courtesy: forbes)

ഐപിഎല്‍ പ്ര­ഖ്യാ­പി­ച്ചു കഴി­ഞ്ഞ­പ്പോള്‍ മാ­ധ്യ­മ­ങ്ങ­ളു­ടെ ഒരു പ്ര­ധാന ആശ­യ­മാ­യി­രു­ന്നു മാ­റ്റു­ര­ച്ചു­നോ­ക്കല്‍. ഐപി­എല്‍ 
ജേ­താ­വും ഐ­സി­എല്‍ ജേ­താ­വും തമ്മി­ലൊ­രു­മ­ത്സ­രം. തങ്ങള്‍­ക്കു യാ­തൊ­രു പ്ര­ശ്ന­വു­മി­ല്ലെ­ന്നു കപില്‍ പറ­ഞ്ഞെ­ങ്കി­ലും 
വി­മ­ത­രോ­ട് യാ­തൊ­രു ഒത്തു­തീര്‍­പ്പു­മി­ല്ലെ­ന്ന് ബി­സി­സിഐ തീര്‍­ത്തു പറ­ഞ്ഞു. അതോ­ടെ ഐസി­എ­ല്ലി­ന്റെ നാ­ളു­കള്‍ 
എണ്ണ­പ്പെ­ട്ടു കഴി­ഞ്ഞു­വെ­ന്ന് ചന്ദ്ര­യ്ക്ക് ബോ­ദ്ധ്യം വന്നി­ട്ടു­ണ്ടാ­വ­ണം. അദ്ദേ­ഹം കോ­ട­തി­യി­ലും, ഐസി­സി­യി­ലും ഹര്‍­ജി­കള്‍ 
നല്‍­കി. അടു­ത്ത­വര്‍­ഷം കൂ­ടു­തല്‍ വി­പു­ല­മാ­യി സം­ഘ­ടി­പ്പി­ക്കാന്‍ ശ്ര­മി­ക്കു­ക­യും ചെ­യ്തു. ഒന്നില്‍ കൂ­ടു­തല്‍ ടൂര്‍­ണ്ണ­മെ­ന്റു­ക­ളും 
ദേ­ശീ­യ­ത­യു­ടെ ചാ­യ­വും ചേര്‍­ത്ത് കൊ­ഴു­പ്പു­കൂ­ട്ടാ­നു­മു­ള്ള ശ്ര­മം, ഐസി­എല്‍ വി­ട്ടു വരാന്‍ താല്‍­പ­ര്യ­മു­ള്ള­വര്‍­ക്ക് മാ­പ്പു 
നല്‍­കാ­നു­ള്ള ബി­സി­സി­ഐ. തീ­രു­മാ­ന­ത്തോ­ടെ അവ­സാ­നി­ച്ചു. അങ്ങ­നെ സു­ഭാ­ഷ് ചന്ദ്ര­യു­ടെ ബി­സി­സി­ഐ­യു­മാ­യു­ള്ള 
പോ­രാ­ട്ടം കെ­റി പാര്‍­ക്ക­റു­ടേ­തി­നു സമാ­ന­മാ­യി ബി­സി­സി­ഐ­യു­ടെ അന്തി­മ­വി­ജ­യ­ത്തില്‍ കലാ­ശി­ച്ചു.

­ഡോ­ള­റു­കള്‍ പറ­ന്നു നട­ന്ന ഐപി­എല്‍ ലേ­ല­ത്തില്‍ മും­ബൈ, ബാം­ഗ്ലൂര്‍, ഹൈ­ദ­രാ­ബാ­ദ് ടീ­മു­ക­ളു­ടെ പത്തു വര്‍­ഷ­ത്തെ 
അവ­കാ­ശ­ത്തി­ന് നാ­നൂ­റു കോ­ടി­ക്കു മു­ക­ളില്‍ കൊ­ടു­ക്കാന്‍ മുന്‍­നിര ലി­സ്റ്റ­ഡ് കമ്പ­നി­കള്‍ തയ്യാ­റാ­യി. ബി­സി­സി­ഐ­യു­ടെ 
മു­ഖ­മു­ദ്ര­യായ അതാ­ര്യ നയ­ങ്ങ­ളു­ടെ പ്ര­തീ­ക­മാ­യി, ഐപി­എല്‍ ഭര­ണ­സ­മി­തി­യം­ഗ­മായ എന്‍ ശ്രീ­നി­വാ­സന്‍ ചെ­ന്നൈ 
ടീ­മി­നു­ട­മ­യാ­യി. ഒരു കണ­ക്കി­നു പറ­ഞ്ഞാല്‍ ശ്രീ­നി­വാ­സന്‍ ചെ­ന്നൈ ടീ­മി­ന്റെ അവ­കാ­ശം ശ്രീ­നി­വാ­സ­നു തന്നെ വി­റ്റു.

ഇ­തി­നി­ട­യില്‍ (ലേ­ല­ത്തി­നും മു­മ്പ്) മറ്റൊ­രു കാ­ര്യം നട­ന്നി­രു­ന്നു, ഐപി­എ­ല്ലി­നു ജീ­വന്‍ നല്‍­കാന്‍ ഏറ്റ­വു­മ­ധി­കം പ്ര­വര്‍­ത്തി­ച്ച 
ബി­സി­സിഐ വൈ­സ് പ്ര­സി­ഡ­ന്റ് ലളി­ത് മോ­ഡി­യെ സ്ഥി­ര­ത­യ്ക്കു­വേ­ണ്ടി അഞ്ചു വര്‍­ഷ­ത്തേ­ക്ക് ചെ­യര്‍­മാ­നും കമ്മീ­ഷ­ണ­റു­മാ­യി 
നി­യ­മി­ച്ചു. പി­ന്നെ മാ­ദ്ധ്യ­മ­ങ്ങള്‍ ഊഹ­ങ്ങള്‍ കൊ­ണ്ടും വി­ശ­ക­ല­നം കൊ­ണ്ടും നി­റ­ഞ്ഞു. ഇന്ത്യന്‍ വി­പ­ണി സ്പോര്‍­ട്സ് 
എന്റര്‍­ടൈന്‍­മെ­ന്റ് കമ്പോ­ള­ത്തി­നു തു­റ­ന്നു കി­ട്ടാന്‍ പോ­കു­ന്ന­തി­ലെ സന്തോ­ഷ­ത്തി­ലാ­യി­രു­ന്നു. ഇന്ത്യന്‍ ക്രി­ക്ക­റ്റ് ബോര്‍­ഡ് 
സാ­മ്പ­ത്തിക സു­താ­ര്യ­ത­യി­ലേ­ക്ക് വയ്ക്കു­ന്ന ആദ്യ ചു­വ­ടു­ക­ളാ­യി വരെ വി­ശ­ക­ലന വി­ദ­ഗ്ദര്‍ ശു­ഭാ­പ്തി വി­ശ്വാ­സം പ്ര­ക­ടി­പ്പി­ച്ചു.

­ടീ­മു­ക­ളു­ടെ സാ­മ്പ­ത്തിക കാ­ര്യ­ങ്ങ­ളെ­പ്പ­റ്റി­വ­ന്ന ലേ­ഖ­ന­ങ്ങ­ളി­ലെ ഉള്ള­ട­ക്കം പല­പ്പോ­ഴും ഒരേ അഭി­പ്രാ­യ­ങ്ങ­ളാ­യി­രു­ന്നു പ്ര­ക­ടി­പ്പി­ച്ച­ത്. 
അവ­യെ­ക്കു­റി­ച്ച് അടുത്ത ലേ­ഖ­ന­ത്തില്‍.\footnote{അംബാനി മുതല്‍ മല്യ വരെ}

(20 April 2010)\footnote{http://malayal.am/പലവക/പരമ്പര/ബിസിനസ്-ലീഗ്/5365/ക്രിക്കറ്റ്-കുടത്തിലെ-ഭൂതം}

\newpage
