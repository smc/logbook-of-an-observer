\secstar{ക്രിക്കറ്റ് കുടത്തിലെ ഭൂതം}
\enlargethispage{2\baselineskip}
%\vskip 2pt

‌\begin{framed}
"വിവാദത്തിന്റെ കൂട്ടുകാരനായി മാറിയ കുട്ടിക്രിക്കറ്റിന്റെ അന്താരാഷ്ട്രപതിപ്പില്‍നിന്നു് 'ഇന്ത്യന്‍ സേന' പുറത്തായതു് 
ഇന്നലെയാണു്. അപ്പോഴും ഇവിടെ ഇന്ത്യയില്‍ ഐപിഎല്‍ വിവാദത്തിന്റെ അലയൊലികള്‍ അടങ്ങിയിട്ടില്ല. 
കാരണംകാണിക്കല്‍ നോട്ടീസ് ലഭിച്ച ഐപിഎല്ലിന്റെ മുന്‍ കമ്മീഷണര്‍ ലളിത് മോഡിക്കു് മറുപടി പറയാനുള്ള 
സമയം ദീര്‍ഘിപ്പിച്ചു് വിവാദത്തിന്റെ ആയുസ്സുനീട്ടുകയാണു് ബിസിസിഐ. ഇന്ത്യന്‍ ക്രിക്കറ്റ് ലീഗ് എന്ന നവീന 
ആശയത്തെ കട്ടെടുത്തു് ഇന്ത്യന്‍ പ്രീമിയര്‍ ലീഗ് ആയി അവതരിപ്പിച്ച ലളിത് മോഡി ഒടുവില്‍ കായികരംഗത്തെ 
അധോലോകവാഴ്ചയുടെ അടയാളമായി മാറുന്നു. മൂന്നുകൊല്ലം പിന്നിട്ട ഐപിഎല്ലിന്റെ കഥകള്‍ 
അപസര്‍പ്പകകഥകളെപോലും വെല്ലും. ചതിയുടെയും പകയുടെയും അദ്ധ്യായങ്ങള്‍ നിറച്ച ആദ്യ രണ്ടുവര്‍ഷത്തെ ഐപിഎല്‍ കാലം 
അന്വേഷിക്കുകയാണിവിടെ.  മലയാളം ന്യൂസ് പോര്‍ട്ടലിനു വേണ്ടി ജിനേഷ് കെജെ തയ്യാറാക്കിയത്."
\end{framed}

%{\vskip 4pt}

ഐപിഎല്ലിന്റെ ചരിത്രം അന്വേഷിക്കുമ്പോള്‍ ചെന്നെത്തുന്നതു് സുഭാഷ് ചന്ദ്രയുടെ എസ്സെല്‍ ഗ്രൂപ്പില്‍ പെട്ട സീ ടെലിഫിലിംസ് 
2003 മുതല്‍ ഇന്ത്യയിലെ ക്രിക്കറ്റ് സംപ്രേക്ഷണാവകാശത്തിനായി നടത്തിയ ലേലയുദ്ധങ്ങളിലും നിയമയുദ്ധങ്ങളിലുമാണു്. 
പലപ്പോഴും പൊതുജനത്തെ അമ്പരപ്പിക്കുന്ന നടപടിക്രമങ്ങളിലൂടെ സംപ്രേക്ഷണാവകാശം കിട്ടാക്കനിയായപ്പോള്‍ 
സുഭാഷ് ചന്ദ്ര അസാധ്യമായ ഒരു സാഹസത്തിനു മുതിര്‍ന്നു. ബിസിസിഐയിലെ 'കടല്‍ കിഴവന്‍മാരുടെ' സംഘത്തിനു് 
ഒരു കോര്‍പ്പറേറ്റ് ബദല്‍ എന്ന സ്വപ്നത്തിനു നിറംപകരാന്‍ ശ്രമിച്ചു. ഇതു ചെയ്യുമ്പോള്‍ ഇന്ത്യയിലെ ക്രിക്കറ്റ് ഭരണം 
നേരെയാക്കിയെടുക്കണം എന്നൊരുദ്ദേശം ചന്ദ്രയുടെ സ്വപ്നത്തില്‍പോലുമുണ്ടായിരുന്നു എന്നു തോന്നുന്നില്ല. ഒരു പക്ഷേ 
ടെലിവിഷന്‍ സംപ്രേഷണാവകാശങ്ങളുടെ വില്‍പ്പനയിലൂടെ ഫോര്‍മുല വണ്ണിന്റെ അവസാനവാക്കായി മാറിയ ബെര്‍ണി 
എക്‌ലെസ്റ്റോണായിരുന്നിരിക്കണം ചന്ദ്രയുടെ (പിന്നീടു് ലളിത് മോഡിയുടെയും) പ്രചോദനം.

എന്തായാലും എന്റെ ചാനലുകള്‍ക്കു് കാണിക്കാന്‍ അവര്‍ ക്രിക്കറ്റ് തരുന്നില്ല, അതുകൊണ്ടു് ഞാന്‍ സ്വന്തമായി 
ക്രിക്കറ്റ് മത്സരങ്ങള്‍ സംഘടിപ്പിക്കാന്‍ പോകുന്നു എന്നു് കെറി പാര്‍ക്കര്‍ ശൈലിയില്‍ പറഞ്ഞു് ചന്ദ്ര തുറന്നുവിട്ട ഇന്ത്യന്‍ 
ക്രിക്കറ്റ് ലീഗ് ഭൂതം ലോകക്രിക്കറ്റിന്റെ മേലാളന്‍മാരുടെ ഉറക്കംകെടുത്താന്‍ വലിയ താമസമുണ്ടായില്ല. ഐസിഎല്ലുമായി 
സഹകരിക്കുന്ന എല്ലാവര്‍ക്കും വിലക്കേര്‍പ്പെടുത്തി ബിസിസിഐ നയം വ്യക്തമാക്കി. ഇന്ത്യന്‍ ബോര്‍ഡിന്റെ മണിപവറിനു 
മുമ്പില്‍ ഐസിസിയും മറ്റു ബോര്‍ഡുകളും മുട്ടുമടക്കി. ലീഗുമായി സഹകരിക്കുന്ന എല്ലാവര്‍ക്കും അംഗീകൃതവേദികളില്‍നിന്നും 
വിലക്കുവന്നു.

പക്ഷെ, അവസരങ്ങളെക്കാളേറെ ഉദ്യോഗാര്‍ത്ഥികളുള്ള ഇന്ത്യന്‍ വ്യവസ്ഥിതിക്കുള്ളില്‍നിന്നും ആറു് തരക്കേടില്ലാത്ത 
ടീമുകളെ ഉണ്ടാക്കാന്‍ ഐസിഎല്ലിനു സാധിച്ചു. കപില്‍ ദേവിന്റെയും ഡീന്‍ ജോണ്‍സിന്റെയും മറ്റും മേല്‍നോട്ടത്തില്‍ 
ആവേശകരമായ ഒരു സീസണ്‍ സംഘടിപ്പിക്കാന്‍ ചന്ദ്രയ്ക്കായി. ചന്ദ്രയെ തളര്‍ത്താന്‍ തങ്ങളാലാവുംവിധം ബിസിസിഐ 
ശ്രമിച്ചു. എങ്കിലും ആദ്യ സീസണ്‍ കഴിഞ്ഞപ്പോള്‍ ബിസിസിഐയെക്കൊണ്ടു് സ്വന്തം 20-20 ലീഗ് പ്രഖ്യാപിക്കാന്‍ ചന്ദ്രയുടെ 
സാഹസത്തിനു സാധിച്ചു.

%ഇന്ത്യന്‍ ക്രിക്കറ്റ് ലീഗിന്റെ സ്ഥാപകനായ 
%<a href=സീ ടിവി ഉടമ സുഭാഷ് ചന്ദ്ര" title="സുഭാഷ് ചന്ദ്ര" style="margin-top:7px;margin-bottom:7px;" height="280" width="400" />

%(image courtesy: forbes)

ഐപിഎല്‍ പ്രഖ്യാപിച്ചുകഴിഞ്ഞപ്പോള്‍ മാധ്യമങ്ങളുടെ ഒരു പ്രധാന ആശയമായിരുന്നു മാറ്റുരച്ചുനോക്കല്‍. ഐപിഎല്‍ 
ജേതാവും ഐസിഎല്‍ ജേതാവും തമ്മിലൊരുമത്സരം. തങ്ങള്‍ക്കു യാതൊരു പ്രശ്നവുമില്ലെന്നു് കപില്‍ പറഞ്ഞെങ്കിലും 
വിമതരോടു് യാതൊരു ഒത്തുതീര്‍പ്പുമില്ലെന്നു് ബിസിസിഐ തീര്‍ത്തുപറഞ്ഞു. അതോടെ ഐസിഎല്ലിന്റെ നാളുകള്‍ 
എണ്ണപ്പെട്ടു കഴിഞ്ഞുവെന്നു് ചന്ദ്രയ്ക്കു് ബോദ്ധ്യംവന്നിട്ടുണ്ടാവണം. അദ്ദേഹം കോടതിയിലും ഐസിസിയിലും ഹര്‍ജികള്‍ 
നല്‍കി. അടുത്തവര്‍ഷം കൂടുതല്‍ വിപുലമായി സംഘടിപ്പിക്കാന്‍ ശ്രമിക്കുകയും ചെയ്തു. ഒന്നില്‍കൂടുതല്‍ ടൂര്‍ണ്ണമെന്റുകളും 
ദേശീയതയുടെ ചായവും ചേര്‍ത്തു് കൊഴുപ്പുകൂട്ടാനുള്ള ശ്രമം, ഐസിഎല്‍ വിട്ടുവരാന്‍ താല്‍പര്യമുള്ളവര്‍ക്കു് മാപ്പുനല്‍കാനുള്ള 
ബിസിസിഐ തീരുമാനത്തോടെ അവസാനിച്ചു. അങ്ങനെ സുഭാഷ് ചന്ദ്രയുടെ ബിസിസിഐയുമായുള്ള 
പോരാട്ടം കെറി പാര്‍ക്കറുടേതിനു സമാനമായി ബിസിസിഐയുടെ അന്തിമവിജയത്തില്‍ കലാശിച്ചു.

ഡോളറുകള്‍ പറന്നുനടന്ന ഐപിഎല്‍ ലേലത്തില്‍ മുംബൈ, ബാംഗ്ലൂര്‍, ഹൈദരാബാദ് ടീമുകളുടെ പത്തുവര്‍ഷത്തെ 
അവകാശത്തിനു് നാനൂറു കോടിക്കുമുകളില്‍ കൊടുക്കാന്‍ മുന്‍നിര ലിസ്റ്റഡ് കമ്പനികള്‍ തയ്യാറായി. ബിസിസിഐയുടെ 
മുഖമുദ്രയായ അതാര്യനയങ്ങളുടെ പ്രതീകമായി, ഐപിഎല്‍ ഭരണസമിതിയംഗമായ എന്‍ ശ്രീനിവാസന്‍ ചെന്നൈ 
ടീമിനുടമയായി. ഒരു കണക്കിനു പറഞ്ഞാല്‍ ചെന്നൈ ടീമിന്റെ അവകാശം ശ്രീനിവാസന്‍ ശ്രീനിവാസനുതന്നെ വിറ്റു.

ഇതിനിടയില്‍, ലേലത്തിനും മുമ്പു്, മറ്റൊരുകാര്യം നടന്നിരുന്നു. ഐപിഎല്ലിനു ജീവന്‍നല്‍കാന്‍ ഏറ്റവുമധികം പ്രവര്‍ത്തിച്ച 
ബിസിസിഐ വൈസ് പ്രസിഡന്റ് ലളിത് മോഡിയെ സ്ഥിരതയ്ക്കുവേണ്ടി അഞ്ചുവര്‍ഷത്തേക്കു് ചെയര്‍മാനും കമ്മീഷണറുമായി 
നിയമിച്ചു. മാദ്ധ്യമങ്ങള്‍ ഊഹങ്ങള്‍കൊണ്ടും വിശകലനംകൊണ്ടും നിറഞ്ഞു. ഇന്ത്യന്‍ വിപണി സ്പോര്‍ട്സ് 
എന്റര്‍ടൈന്‍മെന്റ് കമ്പോളത്തിനു തുറന്നുകിട്ടാന്‍ പോകുന്നതിലെ സന്തോഷത്തിലായിരുന്നു. ഇന്ത്യന്‍ ക്രിക്കറ്റ് ബോര്‍ഡ് 
സാമ്പത്തികസുതാര്യതയിലേക്കുവയ്ക്കുന്ന ആദ്യചുവടുകളായിവരെ വിശകലനവിദഗ്ദ്ധര്‍ ശുഭാപ്തിവിശ്വാസം പ്രകടിപ്പിച്ചു.

\hspace*{2em}(10 May, 2010)\footnote{http://malayal.am/പലവക/പരമ്പര/ബിസിനസ്-ലീഗ്/5365/ക്രിക്കറ്റ്-കുടത്തിലെ-ഭൂതം}

\newpage
