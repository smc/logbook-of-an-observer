\secstar{Hospital Log 2}
\vskip 2pt
\begin{english}
\subsection*{Monday November 01 2010}

Today is the 11\textsuperscript{th} day of my first chemo cycle. Still going good and other
than some instances of vomiting and loss of interest in food, the days are okay.
Thankfully, no major infections or after effects of chemo showed up. 
Don't know how the medicine is doing, but I guess by next week the second 
round of chemo should start (if platelet count reaches the desired level). 
Like that, I have to do 8 cycles and usually it goes on for 4-6 months. 
This is the info from the medical side.

On the personal front, I finished reading 2 novels and one book by Vinay 
Lal on the global political scenario. Now reading the famous Edward P. Said's 
"Orientalism". My cousin was asking whether I wanted "Glimpses of world history" 
to read. I actually have no idea about what that is. Before starting to read such a book, 
a lot of things are there to be considered including, how much time you can devote to 
think about it. 

One of my cousins got me a Tata photon+. I took a 1GB plan. I may have to rethink 
about the plans after the first month. It enhances my connectivity but also brings 
in the danger of keeping me out of other activities.

During the last few days, I had a large set of visitors. Apart from my father's brothers,
my aunt and uncle also visited. They were here for a day. Then two friends from IIIT 
came with news, books and a lot more. Two more are coming on Wednesday. My cousin visited 
twice. Two friends from Kuttippuram came. Everyone wants to see the sick and 
show their moral support, while I'm more interested in poaching for their blood and 
platelets :) 

\subsection*{Thoughts on my Friends}

Some of my friends are keenly interested in helping me financially and so is one of my professors. 
My father might have some difficulty in getting the 10-15 lakhs required for the initial stage 
but, that doesn't mean he is totally in need of money right now. For me, someone who was making 
rs 6000 per month, I spent all that money during the last three years. Mostly on dresses, food, party, 
flight tickets home (I fly every time I go home), maintaining a nice ponytail with extravagant trips to the best 
saloons in the city such that many people at nice restaurants in the city knew me by appearance, I should say. 
For someone extravagant like that, it is amazing that I do have friends who wants to put in money 
for my costly treatment and to help getting money from endowments. 

I don't really know whether I should claim it or not. I was and am someone who doesn't care much 
for money. Someone who is not used to giving it when I have it. However, fortunately, I did have friends who could give me 
some when I was in need. First of all, I don't know whether it is quite right to make a financial 
relation out of a more emotionally and personally knit social relation. I never cared for financials, 
but made sure I never crossed my reserve. On top people say that there are always strings attached with 
money, which I would like to avoid as much as possible. On top being a rational human being, I would 
say, it doesn't make much sense for me to give financial aid to someone who can afford top medical facilities of the country along with VIP facilities. 

May be my friends think that I am not going to make money for a little while on the pretext of 
the disease (I lost a year and haven't stopped thinking about PhD). My father being a retired man, 
they might be thinking it is their duty to help out a friend in need. After all, some wise people said 
long before "a friend in need is a friend indeed". They might be confused on how to be friendly 
towards a guy, who, with this kind of a serious illness, was not in much of any emotional distress but was full of energy 
and optimism and on top, was someone with a philosophical and rational edge over them. I am a kind of socially
introvert guy, who likes to be in a closed space rather than in a party. But, being in company of friends, 
I just like to be one of them and have a good time of what I get. I should say, I try to make the
most out of what I have got. To experience things to the best and thereby making sure, I am not missing any details. 

\subsection*{Who are my friends?}

May be that attitude gave me access to a varied and interesting group of individuals. Many of them, 
my very good friends and well wishers. I do have company of intellectuals who tries to debate, see 
the other perspectives, form perspectives, avoid prejudices and on top, see logic in situations. I 
have friends among people who call these intellectuals as dreamers and dismiss them as waste. I party 
with them during IPL, New Year, Diwali, or any day for that matter. We watch premier league matches 
and formula one races together, discuss food and restaurants, music and live like any other. They do 
know that I have a good say and company in an intellectual gang too (not in IIIT however. There I still keep a low
profile,. Being a new guy, I have a few friends and just started discussing some initiatives in Exact
Humanities dept. with some friends. May be by next year, I will get a chance there too :)). I do 
have access to the so called geeky hacker community for having a niche of my work on in Indic 
language computing and being a avid fan and user of GNU/Linux. Though not much active for the past 
few years, I do have my presence in the Indian Free Software community, courtesy of friends I made in 
due course. I still involve so much in the activities of a free software group I revived back in 
college in keeping it active. I would say I have a good friend in the journalist community who, wholeheartedly, provides me access, knowledge and space for my thoughts and perspectives.  


All these gives me access to a varied set of friends and community of people who are all, I say, 
belong to either middle class or upper middle class. I don't have much friends in the strata above 
that (the one of socialites/celebrities). However, I should say that some of my friends in partying circles 
were good enough to get me a glimpse or two of what was happening there (with help of their friends). 
It would be wrong of me to say that I don't have friends in the lower strata of community. I did my primary education 
in an aided Malayalam medium school and there I have a lot friends who are either self employed 
or managed to lift themselves up from the poverty they had, through hard work (only handful of my 
primary friends managed to get through SSLC. Although most of my classmates were Muslims, I don't 
remember any of them getting through). Most of my female primary classmates were married by the time 
I finished my 12\textsuperscript{th}. Very few of them even thought about studying till 10\textsuperscript{th}. Even on the male side, 
very few were interested in studies. So, for a class topping studious guy, I was more than a friend 
to them. More like I was someone to most of them to tell their children that we were in the same school :) Still 
I meet them occasionally (we don't cross paths much as I visit my village only once in a while 
and most of them will be associated with some function or the other). Then I moved to JNV\footnote{Jawahar Navodaya Vidyalaya}. There we rarely discussed financials, but I do remember many of my friends had
parents with govt. jobs. Some of them self employed. Some of them a little more sophisticated jobs
like teachers. Some were hard laborers who tilled their land. But all of us were good in something 
or other. We were good in
studies, were disciplined well by the residential establishment, leading us to the prime goal of 
being a professional (doctor/engineer). If you can't become a Doctor or an Engineer, education can provide something else too. I should say, most of us did well. Many are engineers and doctors. Some are nurses. 
Some are in service sector. Some disappointed, may be didn't get the system of real world outside 
the system of JNV. Handful of us are in academia too. 

Then the 4 years in a self financing Engineering college in Malappuram where most of the students 
are wards of NRIs, exposed me to a totally different group of people including real street gangs. :) 
It gave me a glimpse of what campus politics have boiled down to. Why it is not going beyond the hefty 
gang wars between the muscles of some fighting minds, who don't really need the political affiliation 
to put up a fight (which was evident from the past and most recent fights in the campus). Most importantly, 
it gave me a home in almost all districts of Kerala where I can except a meal in a time of trouble. :) 
In Ernakulam, I have more than five spots where I can spend a night. In Trivandrum, not much but still 
a guy has settled down there for now. Calicut, I won't say how many will be happy to accommodate me. So is the case in Kollam, Aleppy, Kannur, Thalassery and even in Wayanad. I think I had the aura of a geek there.
Eccentric and intellectual at the same time. There is not much of a variety present in the campuses of Kerala
(though they are supposedly people whom we find in serious tech. depts of western universities). I
don't know where it placed me in the social strata of the college, but definitely not in the best lovable guy 
list (or even manly guy list) I suppose. I was the one who used to share the lunch of day
scholars (both boys and girls alike) or make them give me money for lunch to avoid a long walk to
hostel. I kept no shame in begging since my idea of 'financially sound' or 'manly' doesn't mean that I can't 
beg for 10 rs. On top of it all, for someone who does not play any games, though being able to give a hardcore lecture on soccer 
and the scores, being not much of an actual player of the game, not feisty and also not being the kind of person who keeps grit 
over things, all made me a different kind of guy and still a popular one at that I should say. :) Likewise, 
my 3 year stay in IIIT, gave me access to a different breed of people and virtually a bed in every 
city in the country.

\subsection*{Thoughts on what helped and how it helps in having so many friends}

I should say, having been fortunate enough to meet so many people and make friends with such a 
large cross section of society, gives you access to a lot of information too. May be my interest
in Sherlock Holmes from childhood and later in logics and computers, might have shot up my interest 
in both logic and observation. I should say it helps in my current field of study a lot too. I Don't 
know how many others, working among a similar crowd, had the kind of access and familiarity I had with
different groups of people. Usually, one likes to remain in their comfort zone. Being to a hostel 
when I was 10, my comfort zone was being alone. I am not a crowds man, not a public speaker or dancer 
or anything of the sort. However, may be good in parties and enjoying a good time with friends with those little anecdotes
and comments. Pretty good foodie and a nice food gives me a good time. Other than that, I belong to 
the arm chair, virtual world generation. Life for me without Internet access is a disaster. I do have
a GPRS connection and a netbook and kind of had an uninterrupted Internet connection for the last 3 years. I like to 
be in my room invisible to everyone but then again, I know that is not what life is about. 

I do go out, make friends with others and enjoy life to the fullest. It gives me the chance to understand 
and put to use the observations I made from texts and books. So, I should say I am comfortable
everywhere. I enjoy all that I take part in, make my contributions in any capacity, get embarrassed
sometimes and unlikely of my nature, I do even volunteer for some tasks. :) May be the idea that I
have a comfort zone might have left my mind when I went to boarding school. I might have started to anticipate the adaptation to 
unexpected situations and different kinds of people. On top, 
my philosophy of changing the numbers of life according to the numbers I get might have helped me. I 
expect ambiguity, abnormality and interestingly on top, a pattern in life. I do think and try to
understand about society and life at large. My little access to different circles of people gives 
me food for thought too. It also gives me experience in the so called etiquette of different 
groups of people. I get a chance to observe by being a part of the society, and be aware of what society 
is capable of. 

Being someone who likes to take decisions of his life in his own fashion, I sometimes end up in
loggerheads with my immediate circle of family and friends. The logics that I follow or pursue, mostly 
just don't convince them. But my experiences in interacting with a large section of people, thinking 
in different ways, and looking life at different angles is what usually helps in cooling the system.
Rational and logical arguments seldom work. So does stubbornness and arguments. Sometimes a smile or 
a little anecdote will convince the other better. Sometimes it is better to avoid confrontation, if at 
all possible. Sometimes, it is best that you be a part of the group, even though you are nothing of the 
sort (I was quite accepted by the all the so called political outfits of the campus). 

This gives quite an account of my friends. May be  about why I have so many friends and what I do get from 
them than the financial aid or real idea of friendship too. It is a learning experience to be in the company 
of so many people. Also, for people who love knowledge and understand it better than anything else, with a keen 
and trained (from JNV) thoughts and eyes on society, it is something precious I should say.   


\subsection*{Family}
   
Though I am not a filthy rich guy, I am quite sure that affording my treatment at its current cost is not so 
much of a hefty burden on my father than many other things. It will be more of a mental and physical 
burden for him than financial. He has brothers who are able to and are helping him in running his farm, 
his real estate investment etc. Also as an investment, he preferred liquidity over higher returns 
and that also might help him a little here. Amazingly, I do have friends, teachers and family to 
help me in anyway they can. I do know it makes me one of the few fortunate people with the same 
medical situation and it, I should say, humbles me. 

But it also gives me an idea of the society that I live in. The monetary help is always seen as the best 
and important way of showing support and easing the burden in case of expensive treatments. 
Strong family ties are a crucial matter when it comes to situations like this. Me and my father can 
manage in Vellore easily without having the slightest tension and even think about doing some work back in IIIT, only 
because I have two of my father's brothers looking after his properties and my mother in our 
village and another brother looking after our properties in town. A fourth brother who is a doctor, 
spend a week of his life troubleshooting my issue and still calls almost everyday to get the 
details of treatment and my status in case he needs to get involved. I do have doctors as my cousins 
who spend the best ties and knowledge they have to find out where I can get the best treatment in India. 

All this came from a strong loyalty to the idea of family -- for which even I should say I am
trying to contribute a little by efforts of building a family tree. It is a project this is halfway and I would 
like to complete it. When I'm in hospital, I can at least get it going by a small detail. A lot of
information is gathered by my father's elder brother but was not archived well in a software scheme 
-- I will try to do that and give it a shape as soon as I get hold of that data. The loyalty to 
family came from various points. First of all, all the stakeholders involved were hardworking and
financially stable and the matriarch (my father's mother) who ran the family for long time actually 
made sure that the details of money were correct and kept out equation wherever possible. 

There is enough and even more family troubles and issues in each of the atomic families. However, usually any 
big issues are put forward to something like a family meeting and I should say everyone is made 
to participate in that. Plus, they gather for Onam, Vishu, different poojas and other festivities 
every year. On top of their participation, they make sure people who can participate from next
generation is sure available and is on the scene. The sense of being a part of the family and all these 
people are our own (they are not cousins but brothers and sisters) is something they like us to
understand. That gives my sister the nerve to tell her nine year old son that he got 100s of 
cousins back in India. Seeing the American society, I don't think he might have been convinced 
and that put me into making the family tree. But I guess that his visits to India and participation in one or 
two family gatherings every time might have given him the idea. He might have seen and 
played with a lot of those 100s of cousins in the two month vacation he gets once in two years. 

All of my father's siblings have achieved their own level of self sufficiency. Interestingly none of 
them think alike. There used to be huge debates on any issue with people having different views and all of it gets
thoroughly debated out. Head of the family takes the decision in the end. May be this level of
involvement due to the diversity of opinions might be creating a chance of discussing out all 
possibilities of a problem. The kind of rationality obtained when things are debated well to 
understand its different perspectives. This is not to say that they are modernists or reforming, but instead, most 
of them are believers with different political mentalities and have kind of orthodox views in 
social relations. But, I should say, they understand change, they resist when I grow my hair or 
decide to make a choice to go into research and academics than the money spinning IT Galore. 
But they accept it. They know we have the power and will to defy them and proceed on our own 
if they are not cautious, which might not do much good for the family fabric that they have been protecting 
for the last 50+ years.

But the thing is, it created a loosely knit, yet loyal institution, out of a set of fragmented, diversely
thinking, atomic family entities. As a matter of fact, the idea of belonging is there for even people who don't really 
belong to the family (ones who are married off). Many of them prefer to spend their
festivities with us since we make the best out of it :) 

The idea of having an independent life out of society and family ties is encouraging and goes 
well with the liberal mind set that I have. I have managed to keep my own identity and made sure I'm the 
one taking decisions for me in life. Still somehow, I managed to keep the life alive with so many 
friends and even put them to good use. Even though I don't accept or rather, sometimes just make and do 
things which many in family don't like (small things like growing hair or big decisions like my 
stream of study), I managed being a part of the family. I do make sure I visit my family in every 
city I have been to. 

It is a kind of experiment. As I said before, the system likes to use it for your benefit. So the best 
thing to do is understand the system, keep it out of you but be inside it. Here I explained two 
large systems of society (friends and family). I managed to be a part of the system, at the same time 
kind of out of its reach. I tried to learn the system, using it to increase my leverage when I needed to have 
an interaction or when I come to loggerheads. I am not interested in, or believe in direct
confrontations. Very rarely do I end up in one. I like to maneuver through the system without 
alerting it. I think I did it right so far. :) If I publish this as it is, it might make a difficult 
and awkward situation, but then life is not always what you want. :)
\end{english}
\newpage 
