\section*{Hospital Log 3}
\vskip 2pt

Wednesday November 10 2010 
--------------------------

I was temporarily discharged from hospital on Monday 8th Nov. First round of my chemotherapy was over on Thursday and by Monday my counts started to increase. Doctors were happy with my health and progress, so they decided to discharge me temporarily for a week. Voila! i was out of hospital after a month. 

I was getting out of a room(a floor i should say) after a month. For 2 days i was in a 2nd floor ward, from third day onwards in a room on 8th floor of a building. At times i was free to roam around the floor corridors and i was allowed to look through the windows and gaze at people and rush outside. But i did not literally get out of a building for one month. Exactly speaking from 7th October to 8th November. But the problem was, i didn't feel it. I did not have any problem in being in a room for a month with my father. I did not feel happy or special to get out of that room. 

Now i can identify some reasons for that too. First of all, being confined to a room is not new for me. If you ask my friends or even my Professor, you will understand how much i love to be in my hostel room :) That might have helped me a little on enjoying being lonely. In fact if i am doing something, i like to be alone. I usually use a headphone in lab to work.  

Next is the communication and information explosion in digital world. Things i need to live is food,dress and medicines, hospital gives all these. Next issue is of getting engaged. I have 4 mobile connections, 2 internet connections one mobile with an internet connection, my netbook. All these devices keep me alive in the virtual world. I am connected 24/7 if i need(i actually put off computers to give some 2 hours for reading). These devices kept me posted, they gave me news from around the world. I get daily news from USA, Hyderabad, Kerala and sometimes Banglore and Chennai. I do work on my laptop related to variety of things ranging from Free Software Activities to Academic Research. I do watch Television but its not required for me, most of the time i do it to get sleep. If i am not in a mood to use computers, i use my mobile to connect to world, read my mails and talk to my friends. If i am not in a mood for that too, then i will go for old fashion reading and that too doesn't bore me much. On top when you are sick, hospital is more comforting. Doctors and Nurses who know what to do when something is wrong is near by.

Not getting out of a room for a month and undergoing a treatment for kind of serious anomaly in my body haven't thrown me out of my comfort zone is a soothing feeling. But when you think more about it, the idea seems like, my comfort zone is "being sick" though this is only serious illness i ever had :) I can do things i want to do, in the pace i want and i am doing it or more interested in doing it than normal. It is kind of frightening thought that you feel very much comfortable in an hospital bed. Yeah there are some reasons like, i didn't get any infections from my chemo, my body was good enough to sustain the medicines. No serious issues were other than some vomiting, loss of appetite, loose motion etc. But in all, even during the one week doctors waited for Doctor Vikram to come and decide on my treatment and i was actually free to do anything, i never had any thoughts of getting out of the hospital :) I was more than happy to be in the room and go along with the routine. 

Then when you think about it, i am still blessed with a lot of friends who are concerned for me. A big family, many of whom visited me in hospital during stay and those who can't visit actually called me to check out. I do have a lot of teachers who are concerned for me and ready to help in whatever way they can. My Prof. calls me once in a while and tells me not to worry about things in Hyderabad. Another Prof. from my UG times called me enquired details and told me to keep him posted(have to mail him :)). Friends from my school calls me once in a while and checks.

This puts two things in to my perspective. First is a realisation that i am a solid hardcore member of so called Playstation/Internet/armchair/beanbag generation(though i never owned a gaming console). For me the entire world is on finger tips and i like to get out of it whenever i please to do so. I like the power to switch off or log off from others. Being alone is not a problem as long as i am sure i can connect back to the world. But unlike others in this generation, my twitter/facebook usage is very minimal. I might have tweeted some 200 times so far and logs in facebook only once in a week. Its nothing like i don't want like facebook or twitter, basically i use twitter/buzz to update my status or give out news and very rarely i get something to update :) About facebook, huge volume of information just scares me! I use it once in a while to check on my friends, check their trip pictures, status etc.etc. but not on the level of reading everything on the wall. But, i do follow a lot of other details like dozens of mailing lists,blogs and portals to keep up to the world. 

There are lot of things people associate with "us(internet generation)". It includes  lack of social skills, inability to form real human relationships, difficulties in understanding human relations, lack of understanding of real world problems(some people call it "reset button syndrome" :)) to even lack of moral values(i don't really know how, may be all the fps is making us blood savvy). I should say these are exaggerations. Its not like, none in the internet generation can make sense of the world. I point to myself as a proof of concept. I am better off alone in front of a computer with access to wikipedia and google. But that doesn't mean i am good only there. We are very comfortable getting out of comfort zones too. We do make friends and most of the twitter/facebook/playstation generation do a lot of partying. Most of the page 3 figures are kind of people who can live in a room for a month without much trouble if they have a computer, mobile phone and internet connection. 

The idea of being a very social guy with lot friends from different groups and levels and a nice family fellow with lot loving relatives and someone who feels comfortable alone in a room with a pc and internet connection is kind of bizzare! But some how i am able to manage it. May be the contrasting images of me in different scenarios is a suggestion that the real me is not there in any of these broad pictures. But thats usually the pet idea for grand unified theoriests. I am more of a believer in existance of multiple instances. 

I firmly believe that it is natural for people to behave differently in different social scenarios to avail maximum leverage for themselves(everyone is selfish). I do believe its natural for me to be comfortable in a crowded dance floor listening to head banging metal or a quiet bar/restaurant with friends or in a closed room alone with a computer and internet. Multiple comfort zones and equally enjoying them might be possible only for people of this generation i guess. The ones who change themselves from a geeky researcher to hard core metal fan in seconds and combine the contrasting persona well so that he will listen to heavy metal while writing a paper on advanced pattern recognition. 

People might say its bad to be inconsistant and without a personal identity. But what i would say is, multiple identities are natural. They are generated and nourished as a part of our ability to adapt to situations. It is unwise to expect everyone to follow the defined roles of the society. People tend to follow patterns and groups. Most of the time identity of the person is merged with the identity of the group he belongs. But that helps in bringing out his personal identity since within the group identity of the group is irrelevant. For people who belong to a large variety of groups like me, there are visible contrasting identities on play. Within a group even the identities from other group becomes your personal identity. So, ultimately when someone defines me, its a perspective of him about me from the identies he know. I think its true for all human beings. Everyone has different identites in different groups and one's perspective on other is just an opinion of the group from which he knows the other. In conclusion, what i wanted to say is, its wrong to say someone belongs to a group just because he is comfortable with a group or he is part of a group. He might belong to another contrasting group too. It is very much possible that he can be very much comfortable there. People can be equally comfortable in two contrasting scenarios :)

\newpage 
