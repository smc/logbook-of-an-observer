\secstar{Hospital Log 3}
\vskip 2pt

\subsection*{Wednesday November 10 2010}

I was temporarily discharged from hospital on Monday, the 8\textsuperscript{th} of November. First round of my chemotherapy 
was over on Thursday and by Monday, my counts started to increase. Doctors were happy with my health and 
progress, so they decided to discharge me temporarily for a week. Voila! I was out of hospital after a month. 

I was getting out of a room (a floor I should say) after a month. For 2 days I was in a 2\textsuperscript{nd} floor ward and from 
the third day onwards in a room on 8th floor of a building. At times I was free to roam around the floor corridors 
and I was allowed to look through the windows and gaze at the people and the rush outside. But I did not literally 
get out of a building for one month. To be specific, from 7th October to 8th November. But the problem was, I didn't 
feel it. I did not have any problem being in a room for a month with my father. I did not feel happy or special to 
get out of that room. 

Now I can identify some reasons for that too. First of all, being confined to a room is not new for me. If you ask 
my friends or even my Professor, you will understand how much I love to be in my hostel room :) That might 
have helped me a little in enjoying being lonely. In fact if I am doing something, I like to be alone. I usually use a 
headphone in lab to work.  

Next is the communication and information explosion in digital world. Things I need to live is food, dress and 
medicines. The hospital gives all these. Next issue is of getting engaged. I have 4 mobile connections, 2 internet 
connections, one mobile with an internet connection and my netbook. All these devices keep me alive in the virtual 
world. I can be connected 24/7 if I want (I actually put off computers to give some 2 hours for reading). These 
devices kept me posted, they gave me news from around the world. I get daily news from USA, Hyderabad, Kerala 
and sometimes Banglore and Chennai. I do work on my laptop related to variety of things ranging from Free Software
Activities to Academic Research. I do watch Television but its not required for me. Most of the time I do it to get 
sleep. If I am not in a mood to use computers, I use my mobile to connect to the world, to read my mails and to talk 
to my friends. If I am not in a mood for that too, then I would go for the old fashion reading and that too does not 
get me bored much. On top, when you are sick, hospital is more comforting. Doctors and Nurses who know what 
to do when something is wrong is nearby.

Not getting out of a room for a month and undergoing a treatment for a kind of serious anomaly in my body not 
having thrown me out of my comfort zone is a soothing feeling. But when you think more about it, the idea seems 
like, my comfort zone is "being sick" though this is the only serious illness I ever had :) I can do things I want to 
do in the pace I want. Moreover I was doing it and I was more interested in doing it than I normally was. It is a 
kind of frightening thought that you feel very much comfortable in a hospital bed. Yeah there are some reasons 
like, I didn't get any infections from my chemo and my body was good enough to sustain the medicines. No serious 
issues were there other than some vomiting, loss of appetite, loose motion etc. But all in all, even during the one 
week doctors waited for Doctor Vikram to come and decide on my treatment, I was actually free to do anything. 
I never had any thoughts of getting out of the hospital :) I was more than happy to be in the room and go along 
with the routine. 

When you think about it, I am still blessed with a lot of friends who are concerned for me. A big family, many of 
whom visited me in hospital during my stay and those who couldn't visit actually called me to check up on me. 
I do have a lot of teachers who are concerned for me and ready to help in whatever way they can. My Professor 
calls me once in a while and tells me not to worry about things in Hyderabad. Another Professor from my UG times 
called me to enquire about the details and told me to keep him posted (have to mail him :)). Friends from my 
school calls me once in a while and enquires about my health.

This puts two things into my perspective. First is a realization that I am a solid hardcore member of the so called 
Playstation/Internet/armchair/beanbag generation (though I never owned a gaming console). For me, the entire 
world is there on my finger tips and I like to get out of it whenever I pleased to do so. I like the power to switch off 
or log off from others. Being alone is not a problem as long as I am sure I can connect back to the world. But unlike 
others in this generation, my twitter/facebook usage is very minimal. I might have tweeted some 200 times so far 
and get on facebook only once in a week. Itis nothing like ``I don't want/like facebook or twitter''. Basically, I use 
twitter/buzz to update my status or give out news and very rarely do I get something to update :) About facebook, 
huge volume of information just scares me! I use it once in a while to check on my friends, check their trip pictures, 
status, etc., but not on the level of reading everything on the wall. But I do follow a lot of other details like dozens 
of mailing lists, blogs and portals to keep up to date with the world. 

There are lot of things people associate with "us (Internet generation)". It includes lack of social skills, inability to 
form real human relationships, difficulties in understanding human relations, lack of understanding of real world 
problems (some people call it "reset button syndrome" :)) to even lack of moral values (I don't really know how. 
May be all the fps is making us blood savvy). I should say these are exaggerations. Its not like, no one in the 
Internet generation can make sense of the world. I point to myself as a proof of concept. I am better off alone in 
front of a computer with access to wikipedia and google. But that doesn't mean I am good only there. We are very 
comfortable getting out of comfort zones too. We do make friends and most of the twitter/facebook/playstation 
generation do a lot of partying. Most of the page 3 figures are kind of people who can live in a room for a month 
without much trouble if they have a computer, mobile phone and internet connection. 

The idea of being a very social guy with a lot of friends from different groups and levels, and a nice family fellow 
with lots loving relatives, and someone who feels comfortable being alone in a room with a pc and internet 
connection is kind of bizzare! But some how I am able to manage it. May be the contrasting images of me in 
different scenarios is a suggestion that the real me is not there in any of these broad pictures. But that is usually 
the pet idea for grand unified theorists. I am more of a believer in existence of multiple instances. 

I firmly believe that it is natural for people to behave differently in different social scenarios to avail maximum 
leverage for themselves (everyone is selfish). I do believe it is natural for me to be comfortable in a crowded 
dance floor listening to head banging metal or a quiet bar/restaurant with friends or in a closed room alone with 
a computer and internet. Multiple comfort zones and equally enjoying them might be possible only for people of 
this generation I guess. The kind of people who change themselves from a geeky researcher to a hard core 
metal fan in seconds and combine the contrasting personalities well so that he listens to heavy metal while 
writing a paper on advanced pattern recognition. 

People might say it is bad to be inconsistent and to be without a personal identity. But what I would say is, 
multiple identities are natural. They are generated and nourished as a part of our ability to adapt to situations. 
It is unwise to expect everyone to follow the defined roles of the society. People tend to follow patterns and 
groups. Most of the time identity of the person is merged with the identity of the group he belongs. But that 
helps in bringing out his personal identity since within the group, identity of the group is irrelevant. For people 
who belong to a large variety of groups like me, there are visible contrasting identities on play. Even within a 
group, the identities from other groups becomes your personal identity. So, ultimately when someone defines me, 
it will be based on his perspective about me from his knowledge regarding the various identities that I possess. 
I think it is true for all human beings. Everyone has different identites in different groups and one's perspective on 
another is just an opinion of the group from which he knows the other to be from. In conclusion, what I wanted to 
say is, it is wrong to say someone belongs to a group just because he is comfortable with a group or he is a 
part of it. He might belong to another contrasting group too. It is very much possible that he can be comfortable 
there too. People can be equally comfortable in two contrasting scenarios :)

\newpage 
