\newpage
\begin{english}
\secstar{Conference on ICTs for Differently-Abled/Underprivileged in Education, Employment and Entrepreneurship}

It has been 10 days after 1st National Conference on ICTs for Diffrently-Abled/Underprivileged in Education, Employment and Entrepreneurship wound up at Layola College, Chennai. Since the Conference organizers Layola College and Center for Internet and Society didn’t bother to inform the participants or the world on what happened at the conference, I am writing down on what happened at a conference which could have made a difference in the field.

Since the conference addresses the need of accessibility and accessible technologies, let me start with a look into existing situation in the field. The cues of using different Information Communication Technologies for the empowerment of differently-abled has its root in successful implementation of these in Latin languages. Major communication technologies in use for improving accessibility range from Text to speech engines, Automatic Speech recognizer, Screen Readers etc. Open standard, including web standards play an important role in improving accessibility. Some other technologies like character recognition, have some roles in the field though not as direct as this. In the context of Indic Languages, Localisation , local language computing and localised assistive technologies also play a key role in increasing the accessibility of 1/6th of world population

The primary problem for lack of reach of advances of technology to the differentially-abled is penetration of technology. So, we should consider all those who work on improving the penetration of the technology as the primary component of any related discussions. For example, any text speech system is of no use along with a screen reader, if we don’t have enough screens which are localized to read. Internationalization of existing modules, so that it can have localized versions is also an important aspect. Along with many others the lack of localized screens remained an important reason for penetration of technology among the under privileged. In an Indian context, development of any useful system is heavily dependent on the availability of useful interfaces. Which adds another technology group working in localization/internationalization arena to the whole set of technology providers.

There are groups and/or individuals who had demonstrated and are working hard to introduce FOSS based solutions for the accessibility problems in India. Starting from cost advantage to sustainable development of the technology is the main reason for these demands for Open Standards and FOSS based solutions. Primary organizer of the conference is even a good advocate of open standards.

More over, if you consider ICT development in technology related to Indian Languages, development efforts are on for last 30 years. But, a useful system is not yet in place. Among many other factors, loss of existing knowledge is an important reason for this(precisely duplication of efforts). This could have been avoided if these systems were Free and Open Systems(there are many more reasons to it, but this is one important factor). Govt. is spending a lot of money on developing different systems for Indian Languages ranging from Robust OCRs/Handwriting recognition systems/Text to speech converters etc. Differently-abled communities are one of the direct benefactors of most of these technologies. To ensure sustainable development and use of technology, it was decided that these projects will be developed and implemented as FOSS solutions. But, later govt. added “Intellectual Property” as a measure of evaluating the project and this just negated the effects of development of sustainable and accessible commercial models on top of the Research and Development went into it. Fear of dying the technology developed along with termination of project is now again looks at the ones who continue to develop it.

In case of India, most of the interesting work came out/released so far on accessibility are done by non state actors. Though there are lot of organizations in the field and many issues are to be addressed from state policy to access to the technologies, co-ordination efforts were limited. So a National Level Conference with dedicated tracks on academics/technology, Workshops for differentially-abled, workshops for building accessible systems(web developers), workshops for NGOs including a tracks on Policy discussions and panel discussions is always expected to establish the much needed coordination and a communication pipeline between different players. In case of India, inclusion of the state and non-state actors in localization efforts and push to FOSS technologies is inevitable for a sustainable model. When it is conducted by credible organizations like Center for Internet and Society who support Open Standards and work for a govt. level policy, naturally we expect the conference to make a mark on the front.

Since the technologies for improving accessibility in Indian Languages is still in its infancy, it is expected from the conference to bring the different actors in the field together. On the contrary, due to faults in organizational activities and negligence from organizers, lot of names were missing from participants. I should say no organization who work FOSS related technologies for improving accessibility were represented. In short words, what could have become an important stepping stone in addressing issues in using ICTs for differently-abled, ended up as place to finalize details of pilot implementation of govt. developed technology at some centers(for which i don’t think such a conference was not required!). Though organizers had a chance of making a difference related to bringing all people together, if they plan a conference round table with out an agenda, i do have serious reservations about what they wanted from the conference.

Just as an example, I will take the OCR round table. They practically invited only OCR Consortium \footnote{\url{http://ocr.cdacnoida.in/}} and Debayan Banerjee \footnote{\url{http://debayan.wordpress.com/}} from technology developers side. That too most of the consortium members got invite on the first day of conference(they were even made to walk around for their TA). Only Debayan was there to represent all OCR developers outside consortium. About the proceedings of the Round Table, practically consortium presented details of the OCR project(including technological challenges, what we have achieved and small example of how on basis of Gurumukhi along with a demo). I still didn’t understand how they failed to invite everyone else! Even for Debayan(i think he got a chance to learn a lot) but for organizers i have no idea what they gained from him. They didn’t even give a chance to him to explain what he did. Interestingly, the organizers had no idea of what should be done. There was no agenda for round table. Mr. Deependra from National Association for Blind took the pain to note down the action points and decisions. Apart from finalizing the pilot implementation of consortium’s OCR at selected centers it was decided to start a mailing list of all round table members for future communication. Practically consortium didn’t gain much other than announcing to lot other people that we exist. Debayan might have gained a lot after the interaction with all the Professors. The ones who represented NAB and Access India also gained a little. Nothing was there in plate for anybody else.

The decision for a pilot implementation of consortium developed technology at the centers of NAB(?) was not a direct result of the conference or round table. Due to pressure from govt. to see its technology field tested somewhere, talks were on with NAB for sometime on possibilities. The conference just gave the final touches to the decisions. If you look from that angle, i am not quite sure what the organizers did in the whole round table exercise. Since they didn’t care to set an agenda for the meeting, i don’t think they have anything to their credit on this whole exercise.

Interestingly, everyone was interested in trying out the technology, in public or exclusive and unfortunately, govt. policies related to “Intellectual Property” had some effect there. Still i do believe the things we have in consortium product is very minimal and before concentrating on producing on top of it, we should concentrate on developing the existing technology. I believe we could only have an OCR frame work, not a complete OCR solution which will satisfy everyone. For different purposes, we may have to tune the components in framework. In that case, a tunable system framework should be the final target. The current development is nowhere near that target. As it was correctly explained in Round Table, currently it will be able to help only in conversions of large books. Only later revisions can handle the bigger challenges. Even the questions on time line of the project were not addressable because the consortium will exist only till the govt. supports it. But i would say, at least in these discussions, the consortium was happy to give away content if govt. allowed. They could be later addressed if we are able to address govt. policy. Since Center for Internet and Society stands for Open Standards and access to information and all, they could have taken up the responsibility of taking up the initiative of pressurizing the govt. to release the technology,data and code related to the consortium project. Unfortunately, the only representative of CIS(a lady, Ms.Nirmita) opened her mouth only when Mr. Deependra asked her some thing about coordinating the effort of consortium and NAB(i think, not sure). I would say CIS missed a chance there on making some difference on the scene, here they just helped finalizing some discussions which was on for a long time.

Third day i attended only for very less time. Interestingly, the round table was driven here not even by consortium, but a lady from SigmaSciTech. She had a wrapper for a proprietary sound API and practically she wanted all TTS engines to comply her standards. She was asking whether Dhwani will work with SAFA. Though people were pretty happy with Santhosh’s system, it was interesting to see, how the discussions were directed. I had to leave in between so didn’t get to know what finally happened.

In short, the conference just did the opposite of what it should have done. It helped in accelerating some processes, which I think might have happened even if they didn’t intervene. It did nothing on policy level, no decisions on lobbying govt. for freeing up its technologies related to accessibility(at the least). More over, it gave chance for some to play the big daddy or mommy in the field and dictate to people who work on the field on what to do(thankfully, at least consortium had enough thinking heads to counter it). This was a good opportunity for a lot of people to make a mark, and I do believe they indeed made a mark in their progress card, just that it is in red.

\subsection*{Comments}
\begin{itemize}
\item sankarshan Says:
December 14, 2009 at 4:43 am

Thank you for the detail in your post. My conclusions after hearing about the conference from various people has not been favorable. Is it possible for you to follow this blog up with a post on what should be an agenda if it were being organized by parties serious about addressing the issues ?
    \begin{itemize}
    \item jinsbond007 Says:
    December 14, 2009 at 6:52 am

    On the agenda i can surely write down my bits. But for me, organizing a conference on topics like ICTs for differently-abled should be done very carefully. When i first heard about conference, i checked the cis website. The aims, themes and focus statements about the conference given there are excellent. But once we list these things out, then an agenda of items to be prepared so as to keep the focus of entire conference on the themes and aims. That is some thing which needs serious thoughts and inputs. I can just give the things come to my mind, may be like a draft. It should be debated and discussed to get to the final shape. More over my expertise is mainly on technical side, i can comment on what could have been done there. On other tracks i am not an expert and it will be difficult to comment on.
        \begin{itemize}
	\item sankarshan Says:
        December 14, 2009 at 6:59 am

        It would be good to have a technical perspective on the agenda. I am not making a request for you to “organize”, my request was whether it was possible for you to spend some time drawing an agenda that makes sense and, can achieve things.
	\end{itemize}
    \end{itemize}

\item jinsbond007 Says:
December 14, 2009 at 7:12 am

I can surely spend time on drawing a rough sketch.
It might take sometime since, its hard for me to find time right now.

    \begin{itemize}
    \item sankarshan Says:
    December 14, 2009 at 7:23 am

    That is perfectly fine. If you could do up a draft and blog about it, it would be of immense help.
	\begin{itemize}
        \item jinsbond007 Says:
        December 14, 2009 at 8:07 am

        I will try to do a draft up and blog as soon as i have something. But as i said before, i can’t guarantee anything soon
	\end{itemize}
    \end{itemize}
\end{itemize}
\end{english}
