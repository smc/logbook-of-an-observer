\secstar{ക്രിക്കറ്റ്, ദേശീയത, പണം, ഐ.പി.എല്‍. എതിര്‍ക്കപ്പെടേണമോ?}
\vskip 2pt


രാംകുമാറിന്റെ പോസ്റ്റിനു\footnote{\url{http://ramakumarr.blogspot.com/2010/04/blog-post.html}} മറുപടിയായി എഴുതിയതാണ്. പല ഭാഗങ്ങളായി വിശദമായി എഴുതിയ ഒരു പോസ്റ്റാണത്, ഞാന്‍ അതിലെ ഒരുഭാഗത്തിനു മാത്രം എഴുതുന്ന മറുപടിയാണിത്. അത് പോസ്റ്റിലെ മറ്റു വിഷയങ്ങളുടെ പ്രാധാന്യം കുറയ്ക്കരുതെന്നു കരുതി ഇവിടെയിടുന്നു.

ഇന്ത്യയില്‍ ഒരുപക്ഷേ ഏറ്റവും ജനകീയമായ കായികവിനോദമാണ് ക്രിക്കറ്റ്. അതിനാല്‍ തന്നെ,ഇന്ത്യന്‍ കായികവിനോദവ്യവസായത്തിന്റെ ആണിക്കല്ലും. ഇത്രയേറെ ജനകീയവും ലാഭകരവുമായ ഒരു വ്യവസായത്തിന്റെ മൊത്തമായുള്ള അവകാശം കുത്തകവത്കരിക്കപ്പെട്ടതാണ്. ബോര്‍ഡ് ഫോര്‍ കണ്‍ട്രോള്‍ ഓഫ് ക്രിക്കറ്റ് ഇന്‍ ഇന്ത്യ എന്ന ബി. സി.സി. ഐ. യ്ക്കാണ് ഇന്ത്യയിലെ ക്രിക്കറ്റ് നടത്തിപ്പിന്റെ ചുമതല. ക്രിക്കറ്റിന്റെ പ്രചാരണത്തിനും നടത്തിപ്പിനും പ്രോത്സാഹനത്തിനും അവരെഴുക്കുന്ന "വിയര്‍പ്പിനു" പകരമായി ഈയടുത്തകാലം വരെ നികുതിയിളവുകളും ലോകക്രിക്കറ്റില്‍ രാജ്യത്തെ പ്രതിനിധീകരിച്ച് പങ്കെടുക്കാനുള്ള അവകാശവും കാലാകാലങ്ങളായി സര്‍ക്കാര്‍ അവര്‍ക്കു നല്‍കിപ്പോന്നിരുന്നു.

ഈ ബി.സി.സി.ഐ. എന്ന പ്രാദേശിക ക്ലബ്ബ് കൂട്ടായ്മയുടെ ഭരണമാകട്ടെ കാലങ്ങളായി പത്രത്താളുകളിലിടം പിടിക്കുന്ന തൊഴുത്തില്‍കുത്തുകളുടെ കഥയാണ്. വമ്പന്‍ രാഷ്ട്രീയ നേതാക്കളും (ശരദ് പവാര്‍, അരുണ്‍ ജെയ്റ്റ്ലി, മാധവറാവു സിന്ധ്യ, നരേന്ദ്ര മോഡി), ബിസിനസ്സുകാരും (എന്‍. ശ്രീനിവാസന്‍, ലളിത് മോഡി, ജഗ്‌മോഹന്‍ ഡാല്‍മിയ, ചിരായു അമീന്‍) എല്ലാം ചേര്‍ന്ന ഒരു പവര്‍ കണ്‍സോര്‍ഷ്യമാണിതെന്നു പറയുന്നതാവും ശരി. ഹിമാചല്‍ പ്രദേശ് ക്രിക്കറ്റ് അസോസിയേഷനെ വിലക്കെടുത്താണ് ഇപ്പോഴത്തെ വിവാദതാരമായ ലളിത് മോഡി ആദ്യമായി ബി.സി.സി.ഐ.യ്ക്കുള്ളിലെത്തുന്നത്. ഇങ്ങനെ അടിമുടി അഴിമതിയില്‍ കുളിച്ചു നില്‍ക്കുന്നതെങ്കിലും, കൃത്യമായുണ്ടാകുന്ന തിരഞ്ഞെടുപ്പുകള്‍ കാര്യങ്ങള്‍ ഭദ്രമാണെന്നൊരു തോന്നല്‍ ജനങ്ങള്‍ക്കു നല്‍കിയിരുന്നു. അതുകൊണ്ടുതന്നെ, ഇന്ത്യന്‍ ദേശീയതയെ സ്വാര്‍ത്ഥ ബിസിനസ്സ് താല്‍പ്പര്യങ്ങള്‍ക്കുപയോഗിക്കുന്ന ലാഭേച്ഛയുള്ള സംഘടനാ സംവിധാനമാണിതെന്ന കാര്യം സമര്‍ത്ഥമായി മറച്ചു വയ്ക്കപ്പെട്ടു.

ഐ.പി.എല്‍. തുടങ്ങാന്‍ തീരുമാനിച്ചതോടെ ബി.സി.സി.ഐ. ഒരേ സമയം അതിന്റെ കപടദേശീയതയുടെ മുഖംമൂടി ഭാഗികമായെങ്കിലും കീറീക്കളയുകയും, പണക്കൊതിയുടെയും സ്വാര്‍ത്ഥ ബിസിനസ്സ് താല്‍പ്പര്യങ്ങളുടെയും കുടം തുറക്കുകയാണു ചെയ്തത്. ഐ.സി.എല്ലിനെതിരായ നീക്കങ്ങള്‍, ഒരു റെഗുലേറ്ററേക്കാള്‍ ഒരു കുത്തകയുടെ കുപ്പായമാണ് തങ്ങള്‍ക്കുള്ളതെന്ന് അവര്‍ തുറന്നു സമ്മതിയ്ക്കുകയായിരുന്നു. കാലങ്ങളായി ക്രിക്കറ്റിനെ നിയന്ത്രിച്ചിരുന്നവരുടെ അല്ലെങ്കില്‍ ഇന്ത്യന്‍ ക്രിക്കറ്റിന്റെ തനതു സാമ്പത്തിക സ്വഭാവമാണ് വെളിപ്പെട്ടതെന്നും പറയാം.

ഇപ്പോള്‍ മൂന്നാം വര്‍ഷത്തിലെത്തിനില്‍ക്കുന്ന ഐ.പി.എല്‍. മാമാങ്കം വര്‍ഷങ്ങളായി ബി.സി.സി.ഐ.യില്‍ നടന്നു വന്നിരുന്ന സ്വജന പക്ഷപാതിത്വത്തിന്റെയും സ്വര്‍ത്ഥതാല്‍പ്പര്യങ്ങളുടെയും ഇരയായി ഇരുട്ടില്‍ തപ്പുകയാണ്. ഇതിനു പകരം ഒന്നാംതരം നടത്തിപ്പിലൂടെ ഐ.പി.എല്‍. നല്ല പേരുണ്ടാക്കിയിരുന്നെങ്കിലും, സമീപഭാവിയില്‍ത്തന്നെ (പത്തോ പന്ത്രണ്ടോ വര്‍ഷങ്ങള്‍ക്കുള്ളില്‍) ബി.സി.സി.ഐ. യെ വിഴുങ്ങാന്‍ മാത്രം സാമ്പത്തിക വളര്‍ച്ച ഐ.പി.എല്‍. നേടിയേനെ. പൊതു വിപണിയില്‍ സജീവമായ കമ്പനികളുടെയും വ്യക്തികളുടെയും വിശ്വാസതയുടെ പുറത്ത് കൂടുതല്‍ സുതാര്യത കൈവരിക്കാനുള്ള സാധ്യതയും അസ്ഥാനത്തായിരുന്നില്ല (മറിച്ചു സംഭവിക്കാനുള്ള സാധ്യതയാണു പക്ഷെ കൂടുതല്‍). അങ്ങനെ ഇത്രയും കാലം ഇന്ത്യയുടെയും ഇന്ത്യയിലെ ജനങ്ങളുടെയും പേരില്‍ ക്രിക്കറ്റ് നിയന്ത്രിച്ചിരുന്ന സാമ്പത്തിക രാഷ്ട്രീയ ദല്ലാളുമാര്‍ നേരിട്ട് അവകാശം നേടി സ്വന്തം മുഖം വെളിവാക്കാനുള്ള സാധ്യതയാണ് ഐ.പി.എല്‍. കൊണ്ടുവന്നത്. ക്രിക്കറ്റ് ദേശീയതയുടെ പേരില്‍ കബളിപ്പിക്കപ്പെട്ടിരുന്ന പൊതുജനത്തിന് സത്യം മനസ്സിലാക്കാനുള്ള അവസരം.

ഇന്ത്യന്‍ ക്രിക്കറ്റ് പൊതുജനങ്ങള്‍ തിരിച്ചുപിടിക്കേണ്ട യാതൊരു ആവശ്യവുമില്ല, കാരണം അതൊരിക്കലും പൊതുജനങ്ങളുടേതായിരുന്നില്ല തന്നെ. ഇപ്പോള്‍ ഐ.പി.എല്‍. സാമ്പത്തികവും സംഘടനാപരവുമായ വിവദങ്ങളില്‍പ്പെട്ടുഴലുന്നത് പൊതുജനത്തെ സംബന്ധിച്ച നല്ലതാണ്. കൂടുതല്‍ സാമ്പത്തിക അച്ചടക്കത്തോടെ വളരാന്‍ അത് ഐ.പി.എല്ലെന്ന ബ്രാന്‍ഡിനെ സഹായിച്ചേക്കും. ഇന്ത്യന്‍ ക്രിക്കറ്റിന്റെ യാഥാര്‍ത്ഥ്യം മനസ്സിലാക്കുന്നതിന് അവരെ അതു കൂടുതല്‍ സഹായിക്കും. ഒരു പക്ഷേ ഭാവിയില്‍ മൂലധനം നിയന്ത്രിക്കുന്നതും പ്രതിഭനിയന്ത്രിക്കുന്നതുമായി പ്രൊഫഷണലെന്നും അമേച്വറെന്നും ക്രിക്കറ്റിനെ വേര്‍ത്തിരിക്കാനും ഇതുപകരിച്ചേക്കും.
\newpage
