\secstar{സഹിഷ്ണുത എന്ന മിത്തു്}

\vskip 2pt

അസഹിഷ്ണുതയുടെ വിവിധ അദ്ധ്യായങ്ങള്‍ കണ്‍മുന്നില്‍ വിരിയുമ്പോള്‍ പലപ്പോഴും ജനമനസ്സുകളിലും ചായക്കടസംവാദങ്ങളിലും 
സാമാന്യവത്കരണങ്ങളായി പരിണമിക്കുന്ന സംഭാഷണങ്ങളില്‍ ഒരുപാടു വിലയിരുത്തലുകള്‍ നടക്കാറുണ്ടു്. അമിത സാമാന്യവത്കരണത്തിനുള്ള
 വ്യഗ്രതയില്‍ 'മതവിശ്വാസങ്ങള്‍'ക്കനുസരിച്ചു് മനുഷ്യന്റെ സഹിഷ്ണുതയില്‍ വ്യക്തമായ മാറ്റങ്ങള്‍ വരാറുണ്ടെന്നൊരു വിധിയും കല്‍പ്പിക്കാറുണ്ടു്. 
 ചായക്കടസംവാദങ്ങളിലും സുഹൃദ്‌വേദികളിലും, വ്യക്തിപരമായ അഭിപ്രായം എന്നു പേരിട്ടുകേള്‍പ്പിക്കുന്ന കെട്ടുകഥകള്‍ക്കും കേട്ടുകേള്‍വികള്‍ക്കും
  തുല്യമായ ഇത്തരം സാമാന്യവത്കരണങ്ങള്‍ പിന്നീടു് ഈ വിധികളെത്തന്നെ തെളിവുകളായെടുത്തു് സാമൂഹ്യസത്യങ്ങളുടെ 
  മേലങ്കിയണിയുമ്പോള്‍ സാമൂഹ്യവിപത്തായിമാറുകയാണു ചെയ്യുന്നത്.

വിവിധമതവിശ്വാസങ്ങള്‍ പുലര്‍ത്തുന്നവര്‍ എത്രമാത്രം സഹിഷ്ണുക്കളും മറ്റുള്ളവരുടെ വിശ്വാസത്തെ ബഹുമാനിക്കുന്നവരുമാണെന്നറിയണമെങ്കില്‍, 
ആസൂത്രിതമായും അല്ലാതെയും ഈ രാജ്യത്തും ലോകത്തും നടക്കുന്ന കൂട്ടക്കൊലകളുടെ കണക്കുകളെടുത്തുനോക്കിയാല്‍ മതിയാകും. 
രാഷ്ട്രീയാധികാരം കയ്യേറിയവര്‍ക്കുനേരെ ഭീഷണിയുയര്‍ത്തുന്നുവെന്നുമുതല്‍, ഭൂരിപക്ഷത്തിന്റെ വിനോദത്തിനുവേണ്ടിവരെ അന്യവിശ്വാസക്കാര്‍ 
പലകാലത്തായി കൊലചെയ്യപ്പെട്ടിട്ടുണ്ടു്. പലപ്പോഴും ഒരു വിശ്വാസക്കാര്‍ അമിതമായി സഹിഷ്ണുക്കളായി മുദ്രകുത്തപ്പെടുന്നതു് സാധാരണമാണുതാനും. 
പക്ഷേ ഇതും മതവിശ്വാസങ്ങളുമായി വലിയ ബന്ധമൊന്നുമുണ്ടാകണമെന്നില്ല. അവിശ്വാസിയെ അന്യവത്കരിക്കുന്നതിനു്, അതുപോലെ 
വിവിധസംഭവങ്ങളെ സ്വന്തം വിശ്വാസത്തോടുള്ള കടന്നു കയറ്റമായിക്കാണുന്നതിനു്, സാമൂഹ്യവും രാഷ്ട്രീയപരവുമായ കാരണങ്ങളാണു് കൂടുതലും.

സമാധാനം ലോകസന്ദേശമാക്കുന്ന, ഹിംസ എന്നതു് കൊടുപാപമായ ജൈനമതക്കാര്‍ക്കും പല ബുദ്ധവിശ്വാസികള്‍ക്കും 
ഇതരവിശ്വാസങ്ങളെ ബഹുമാനിക്കാനുള്ള കെല്‍പ്പു് കുറവാണു്. അതുപോലെ ഹിന്ദുക്കളുടെ സഹിഷ്ണുതയുടെ ആധാരം പലപ്പോഴും 
വിചിത്രമാണു്. ഒരു ഹിന്ദു യുവാവു് അന്യമതക്കാരിയെ വിവാഹം കഴിച്ചാലോ, അല്ലെങ്കില്‍ ഹിന്ദുയുവതി അന്യമതസ്ഥനെ വിവാഹം 
കഴിച്ചാലോ ഉയരാന്‍ സാധ്യതയുള്ള മുറുമുറുപ്പുകളെക്കാള്‍ ശക്തമായ എതിര്‍പ്പുകള്‍ താഴ്‌ന്ന ജാതിക്കാരനെ/കാരിയെ വിവാഹം 
കഴിക്കുമ്പോള്‍ ഉയര്‍ന്നേക്കാം. കാരണം, ഹിന്ദു എന്ന സ്വത്വത്തേക്കാള്‍ ജാതീയമായ സ്വത്വം ശക്തമായതിനാലാണത്. 
ഇനിയിപ്പോള്‍ സ്വന്തം ജാതിയിലെത്തന്നെ ഒരാളെ ഇഷ്ടപ്പെട്ടു വിവാഹം കഴിക്കാന്‍ തീരുമാനിച്ചാലും അതു സ്വന്തം 
കുടുംബത്തിനപമാനമായിത്തോന്നിയാല്‍ മരണശിക്ഷവിധിക്കുന്നവരും ഹിന്ദുക്കളുടെയിടയിലുണ്ടു്.

കുടുംബത്തിനപമാനമാകാനുള്ള കാരണങ്ങളന്വേഷിച്ചാല്‍, സര്‍വ്വംസഹിഷ്ണുക്കളായ സനാതനഹിന്ദുക്കളെ ലോകത്തിലെത്തന്നെ
 ഏറ്റവും വലിയ അസഹിഷ്ണുക്കളായിക്കാണേണ്ടിവരും. താന്‍ മുറുകെപിടിക്കുന്ന വിശ്വാസങ്ങളെ എതിര്‍ക്കാന്‍ ആര്‍ക്കും അവകാശമില്ലെന്നുള്ളതിലും 
 കവിഞ്ഞു്, താന്‍ അംഗീകരിക്കാത്ത വിശ്വാസങ്ങള്‍ സ്വീകരിക്കുന്നവര്‍ മരണശിക്ഷ അര്‍ഹിക്കുന്നവരാണെന്നുള്ള 
 ചിന്തകളിലെത്തിനില്‍ക്കുന്ന സഹിഷ്ണുത.

അപ്പോള്‍ ഒരു വിഭാഗം സഹിഷ്ണുക്കളാണെന്നുള്ള പ്രചാരത്തിന്റെ അടിസ്ഥാനമെന്താണു്? വെറുമൊരു കണ്‍കെട്ടുവിദ്യമാത്രമാണിതു്. 
പരസ്പരം വിശ്വാസങ്ങളെ ബഹുമാനിക്കാനാവുന്ന ഒരു സമൂഹം നവോത്ഥാനശ്രമങ്ങളുടെ ഭാഗമായി വളര്‍ന്നുവന്നിരുന്നു. ഈ സാമൂഹ്യ 
നവോത്ഥാനശ്രമങ്ങള്‍ സ്വാതന്ത്ര്യസമരത്തിനും മുമ്പു് വേരുള്ളവയാണു്. മാത്രമല്ല, പുരോഗമന ചിന്താഗതിക്കാരായ ഭരണാധികാരികളുടെ
 ശ്രമഫലമായി, ആരോഗ്യപരമായ സാമൂഹ്യവളര്‍ച്ചനേടാനുള്ള ശ്രമങ്ങളുടെ ഭാഗമായി പരസ്പരബഹുമാനത്തിന്റെ പാഠങ്ങള്‍ സമൂഹത്തില്‍ 
 വേരൂന്നുകയും ചെയ്തു. ഇങ്ങനെ, അയല്‍ക്കാരന്റെ വിശ്വാസങ്ങളെ ബഹുമാനിക്കാന്‍ ശീലിച്ചിരുന്ന ഒരു സമൂഹത്തെ, നിങ്ങള്‍ 
 അപരന്റെ വിശ്വാസങ്ങളെ സഹിക്കുകയാണെന്നു പഠിപ്പിച്ചു തുടങ്ങുന്നതു് ഭൂരിപക്ഷ വര്‍ഗ്ഗീയതയുടെ വക്താക്കളാണു്.

പരസ്പരം ബഹുമാനിക്കുകയും സഹായിക്കുകയും ചെയ്യുന്നതു് ശീലമാക്കിയ ഒരു സമൂഹത്തില്‍ അപരന്‍ നിന്റെ സഹിഷ്ണുതയെ 
മുതലെടുക്കുകയാണെന്നുള്ള പ്രചരണം അസഹിഷ്ണുതയുടെ വിത്തുകള്‍ പാകി. ന്യൂനപക്ഷ വര്‍ഗ്ഗീയതയുടെ വക്താക്കളുടെ ഇടപെടലാകട്ടെ, 
ഇതിനു് ആക്കം കൂട്ടിയതേയുള്ളൂ. ഇല്ലാത്ത നിന്റെ 'ക്ഷമയുടെ നെല്ലിപ്പലക' നീ കണ്ടുകഴിഞ്ഞുവെന്നു രണ്ടുകൂട്ടരേയും വിശ്വസിപ്പിക്കാനായ
 വര്‍ഗ്ഗീയവാദികള്‍ക്കാവട്ടെ, അവര്‍ക്കു് വേണ്ടതു കിട്ടുകയും ചെയ്തു: രാഷ്ട്രീയനേതൃത്വവും വിലപേശല്‍ ശേഷിയും. പുരോഗമനപരമായ
  ഒരു സമൂഹത്തില്‍ തികച്ചും സ്വാഭാവികമായ പരസ്പരബഹുമാനത്തെ കൃത്രിമമായ സഹിഷ്ണുതയായി തെറ്റിദ്ധരിപ്പിച്ചു് രാഷ്ട്രീയമായി 
  നേട്ടം ലക്ഷ്യമിട്ടവരുടെ ഇരകള്‍ മാത്രമാണു് സഹിഷ്ണുതാവാദവുമായി രംഗത്തെത്തുന്നവര്‍.

നമ്മള്‍ പകര്‍ന്നുകൊടുക്കേണ്ടതും അളക്കേണ്ടതും സഹിഷ്ണുതയുടെ പാഠങ്ങളല്ല, പരസ്പരബഹുമാനത്തിന്റെ പാഠങ്ങളാണു്. സഹിഷ്ണുതയുടെ 
പാഠങ്ങള്‍ക്കുള്ള പ്രശ്നമെന്തെന്നാല്‍, ഒരു പരിധിക്കപ്പുറം ഒരാളുടെയും സംയമനം കാത്തുസൂക്ഷിക്കാന്‍ അന്യനോടുള്ള സഹിഷ്ണുത 
അവനെ സഹായിക്കില്ല എന്നുള്ളതാണു്. എന്നാല്‍ പരസ്പരബഹുമാനത്തിന്റേയും തിരച്ചറിയലിന്റേയും പാഠങ്ങള്‍ ബന്ധങ്ങള്‍ക്കു് കൂടുതല്‍ ആഴം നല്‍കുന്നു. മറ്റൊരുതരത്തില്‍ പറഞ്ഞാല്‍, മറ്റുവിശ്വാസങ്ങള്‍ക്കു്  ബഹുമാനം കൊടുത്തുകൊണ്ടു് അവരെയും സമൂഹത്തിന്റെ 
ഭാഗമായിക്കണ്ടു് ജീവിക്കാനുള്ള കഴിവാണു് അളക്കേണ്ടതു്. ഇതും മതവിശ്വാസവുമായി യാതൊരു ബന്ധവുമുണ്ടാകണമെന്നില്ല. ഇതു് 
പലപ്പോഴും ഒരു പ്രദേശത്തെ വിശ്വാസിസമൂഹത്തിന്റെ രാഷ്ട്രീയവും സാമൂഹ്യവുമായ കാഴ്ചപ്പാടുകളുമായാണു് കൂടുതല്‍ ബന്ധപ്പെട്ടിരിക്കുന്നത്.

ഹിന്ദുക്കളോളം തന്നെ മുസ്ലിങ്ങളും/ക്രിസ്ത്യാനികളും ഉള്ള പ്രദേശങ്ങളില്‍ ജനിച്ചുവളര്‍ന്നവര്‍ക്കു്, പലപ്പോഴും പരസ്പരം മനസ്സിലാക്കാന്‍ 
കൂടുതല്‍ അവസരം ലഭിക്കും. അതുവഴി സാമൂഹ്യപരമായി അടുത്തറിയാനും സാധിക്കും. എന്നാല്‍ ഒരു പ്രത്യേക വിഭാഗം ഭൂരിപക്ഷമായ
 പ്രദേശത്തു്, രണ്ടു വിഭാഗങ്ങളും പരസ്പരം അടുത്തു് മനസ്സിലാക്കുന്നതു് അപൂര്‍വ്വമായിരിക്കും. അതിനാല്‍ത്തന്നെ, പരസ്പരം വിശ്വാസങ്ങളെ
  ബഹുമാനിക്കാനുള്ള ശേഷി അവര്‍ ആര്‍ജ്ജിക്കുന്നുമില്ല.

ഇതു വിശ്വാസത്തിന്റെ പ്രശ്നത്തേക്കാളും, സാമൂഹ്യപരമായ അന്യവത്കരണത്തിന്റേയും അന്യന്റെ സംസ്കാരത്തെ മനസ്സിലാക്കാന്‍ 
കഴിയാത്തതിന്റേയും പ്രശ്നങ്ങളാണു്. കേരളത്തില്‍ ഒരു പരിധിവരെ അന്യവത്കരണം ഇല്ലാതിരുന്നതിന്റെ കാരണം, 
യൂണിഫോറങ്ങളില്ലാത്ത സര്‍ക്കാര്‍ പ്രൈമറി വിദ്യാലയങ്ങളില്‍ ഒരുമിച്ചു പഠിച്ചുവളര്‍ന്ന തലമുറകളാണു്. ജാതി/മത സംഘടനകളുടെ 
പ്രൈമറിവിദ്യാലയങ്ങള്‍ സാര്‍വത്രികമാകുന്നതു് ഈ അന്യവത്കരണത്തിനു് ആക്കം കൂട്ടുന്നുണ്ടാകണം.

ഇത്തരത്തില്‍ വളര്‍ന്നുവരുന്ന, അല്ലെങ്കില്‍ നിലനില്‍ക്കുന്ന പരസ്പരബഹുമാനത്തിന്റെ പാഠങ്ങളെ അങ്ങനെതന്നെ കണ്ടു് അവയെ
 പരിപോഷിപ്പിക്കാനാണു് ശ്രമിക്കേണ്ടതു്, ഒപ്പം അന്യവത്കരണശ്രമങ്ങളെ ചെറുക്കാനും. അല്ലാതെ, സഹിഷ്ണുതയായി തെറ്റിദ്ധരിക്കുകയും, 
 പിന്നീടു് വ്യക്തമായ രാഷ്ട്രീയ അജണ്ടകളോടുകൂടിയ ഭൂരിപക്ഷവര്‍ഗ്ഗീയതയുടെ വക്താക്കള്‍ അക്രമമഴിച്ചുവിടുമ്പോള്‍ അവയെ സഹിഷ്ണുതയുടെ 
 പരിധികഴിഞ്ഞതായി വ്യാഖ്യാനിച്ചു് തൃപ്തിയടയുകയുമല്ല വേണ്ടതു്. അതുപോലെ, ന്യൂനപക്ഷവര്‍ഗ്ഗീയതയുടെ അന്യവത്കരണശ്രമങ്ങളുടെ 
 ഇരകളാവുന്നവരെ അസിഷ്ണുക്കളായ വിശ്വാസത്തിന്റെ വക്താക്കളാക്കാന്‍ ധൃതിപിടിക്കുന്നതിനു മുമ്പേ, അന്യവത്കരണശ്രമങ്ങളെ 
 ചെറുത്തു് മുഖ്യധാരയിലേക്കു് ഇവരേയുംകൂടി കൊണ്ടുവരാനാണു് ശ്രമിക്കേണ്ടതു്.

സഹിഷ്ണുതയുടെ പാഠങ്ങളും കണക്കുകളും നിരത്തുന്നതിനുപകരം നമുക്കു് പരസ്പരബഹുമാനത്തിന്റെ പാഠങ്ങള്‍ പഠിപ്പിക്കാം. 
'നിനക്കു അവനെ സഹിച്ചുവേണം ജീവിക്കാന്‍' എന്നതിനു പകരം 'നീ അവനേയും ബഹുമാനിക്കുക' എന്നു പഠിപ്പിക്കാം.

\begin{flushright}(Jul 10, 2010)\footnote{http://malayal.am/വാര്‍ത്ത/വിശകലനം/6671/സഹിഷ്ണുത-എന്ന-മിത്തു്}\end{flushright}
\newpage
