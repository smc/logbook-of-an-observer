\secstar{സഹിഷ്ണുത എന്ന മിത്ത്}

\vskip 2pt

അ­സ­ഹി­ഷ്ണു­ത­യു­ടെ വി­വിധ അദ്ധ്യാ­യ­ങ്ങള്‍ കണ്‍­മു­ന്നില്‍ വി­രി­യു­മ്പോള്‍ പല­പ്പോ­ഴും ജന­മ­ന­സ്സു­ക­ളി­ലും ചാ­യ­ക്കട സം­വാ­ദ­ങ്ങ­ളി­ലും 
സാ­മാ­ന്യ­വ­ത്ക­ര­ണ­ങ്ങ­ളാ­യി പരി­ണ­മി­ക്കു­ന്ന സം­ഭാ­ഷ­ണ­ങ്ങ­ളില്‍ ഒരു­പാ­ടു വി­ല­യി­രു­ത്ത­ലു­കള്‍ നട­ക്കാ­റു­ണ്ട്. അമിത സാ­മാ­ന്യ­വ­ത്ക­ര­ണ­ത്തി­നു­ള്ള
 വ്യ­ഗ്ര­ത­യില്‍ 'മ­ത­വി­ശ്വാ­സ­ങ്ങള്‍'­ക്ക­നു­സ­രി­ച്ചു മനു­ഷ്യ­ന്റെ സഹി­ഷ്ണു­ത­യില്‍ വ്യ­ക്ത­മായ മാ­റ്റ­ങ്ങള്‍ വരാ­റു­ണ്ടെ­ന്നൊ­രു വി­ധി­യും കല്‍­പ്പി­ക്കാ­റു­ണ്ട്. 
 ചാ­യ­ക്ക­ട­സം­വാ­ദ­ങ്ങ­ളി­ലും സു­ഹൃ­ദ്‌­വേ­ദി­ക­ളി­ലും വ്യ­ക്തി­പ­ര­മായ അഭി­പ്രാ­യം എന്നു പേ­രി­ട്ടു­കേള്‍­പ്പി­ക്കു­ന്ന കെ­ട്ടു­ക­ഥ­കള്‍­ക്കും കേ­ട്ടു­കേള്‍­വി­കള്‍­ക്കും
  തു­ല്യ­മായ ഇത്ത­രം സാ­മാ­ന്യ­വ­ത്ക­ര­ണ­ങ്ങള്‍ പി­ന്നീ­ട് ഈ വി­ധി­ക­ളെ­ത്ത­ന്നെ തെ­ളി­വു­ക­ളാ­യെ­ടു­ത്ത് സാ­മൂ­ഹ്യ­സ­ത്യ­ങ്ങ­ളു­ടെ 
  മേ­ല­ങ്കി­യ­ണി­യു­മ്പോള്‍ സാ­മൂ­ഹ്യ­വി­പ­ത്താ­യി­മാ­റു­ക­യാ­ണു ചെ­യ്യു­ന്ന­ത്.

­വി­വധ മത­വി­ശ്വാ­സ­ങ്ങള്‍ പു­ലര്‍­ത്തു­ന്ന­വര്‍ എത്ര­മാ­ത്രം സഹി­ഷ്ണു­ക്ക­ളും, മറ്റു­ള്ള­വ­രു­ടെ വി­ശ്വാ­സ­ത്തെ ബഹു­മാ­നി­ക്കു­ന്ന­വ­രു­മാ­ണെ­ന്ന­റി­യ­ണ­മെ­ങ്കില്‍, 
ആസൂ­ത്രി­ത­മാ­യും അല്ലാ­തെ­യും ഈ രാ­ജ്യ­ത്തും ലോ­ക­ത്തും നട­ക്കു­ന്ന കൂ­ട്ട­ക്കൊ­ല­ക­ളു­ടെ കണ­ക്കു­ക­ളെ­ടു­ത്തു­നോ­ക്കി­യാല്‍ മതി­യാ­കും. 
രാ­ഷ്ട്രീ­യാ­ധി­കാ­രം കയ്യേ­റി­യ­വര്‍­ക്കു നേ­രെ ഭീ­ഷ­ണി­യു­യര്‍­ത്തു­ന്നു­വെ­ന്നു­മു­തല്‍, ഭൂ­രി­പ­ക്ഷ­ത്തി­ന്റെ വി­നോ­ദ­ത്തി­നു വേ­ണ്ടി­വ­രെ അന്യ­വി­ശ്വാ­സ­ക്കാര്‍ 
പല­കാ­ല­ത്താ­യി കൊ­ല­ചെ­യ്യ­പ്പെ­ട്ടി­ട്ടു­ണ്ട്. പല­പ്പോ­ഴും ഒരു വി­ശ്വാ­സ­ക്കാര്‍ അമി­ത­മാ­യി സഹി­ഷ്ണു­ക്ക­ളാ­യി മു­ദ്ര­കു­ത്ത­പ്പെ­ടു­ന്ന­ത് സാ­ധാ­ര­ണ­മാ­ണു­താ­നും. 
പക്ഷേ ഇതും മത­വി­ശ്വാ­സ­ങ്ങ­ളു­മാ­യി വലിയ ബന്ധ­മൊ­ന്നു­മു­ണ്ടാ­ക­ണ­മെ­ന്നി­ല്ല. അവി­ശ്വാ­സി­യെ അന്യ­വ­ത്ക­രി­ക്കു­ന്ന­തി­ന്, അതു­പോ­ലെ 
വി­വിധ സം­ഭ­വ­ങ്ങ­ളെ സ്വ­ന്തം വി­ശ്വാ­സ­ത്തോ­ടു­ള്ള കട­ന്നു കയ­റ്റ­മാ­യി­ക്കാ­ണു­ന്ന­തി­ന്, സാ­മൂ­ഹ്യ­വും രാ­ഷ്ട്രീ­യ­പ­ര­വു­മായ കാ­ര­ണ­ങ്ങ­ളാ­ണ് കൂ­ടു­ത­ലും­.

­സ­മാ­ധാ­നം ലോ­ക­സ­ന്ദേ­ശ­മാ­ക്കു­ന്ന, ഹിംസ എന്ന­ത് കൊ­ടു­പാ­പ­മാ­യ, ജൈ­ന­മ­ത­ക്കാര്‍­ക്കും പല ബു­ദ്ധ­വി­ശ്വാ­സി­കള്‍­ക്കും 
ഇത­ര­വി­ശ്വാ­സ­ങ്ങ­ളെ ബഹു­മാ­നി­ക്കാ­നു­ള്ള കെല്‍­പ്പും കു­റ­വാ­ണ്. അതു­പോ­ലെ ഹി­ന്ദു­ക്ക­ളു­ടെ സഹി­ഷ്ണു­ത­യു­ടെ ആധാ­രം പല­പ്പോ­ഴും 
വി­ചി­ത്ര­മാ­ണ്. ഒരു ഹി­ന്ദു യു­വാ­വ് അന്യ­മ­ത­ക്കാ­രി­യെ വി­വാ­ഹം കഴി­ച്ചാ­ലോ അല്ലെ­ങ്കില്‍ ഹി­ന്ദു­യു­വ­തി അന്യ­മ­ത­സ്ഥ­നെ വി­വാ­ഹം 
കഴി­ച്ചാ­ലോ ഉയ­രാന്‍ സാ­ധ്യ­ത­യു­ള്ള മു­റു­മു­റു­പ്പു­ക­ളെ­ക്കാള്‍ ശക്ത­മായ എതിര്‍­പ്പു­കള്‍ താ­ഴ്‌­ന്ന ജാ­തി­ക്കാ­ര­നെ­/­കാ­രി­യെ വി­വാ­ഹം 
കഴി­ക്കു­മ്പോള്‍ ഉയര്‍­ന്നേ­ക്കാം. കാ­ര­ണം, ഹി­ന്ദു എന്ന സ്വ­ത്വ­ത്തേ­ക്കാള്‍ കൂ­ടു­തല്‍ ജാ­തീ­യ­മായ സ്വ­ത്വം അവി­ടെ ശക്ത­മാ­യ­തി­നാ­ലാ­ണ­ത്. 
ഇനി­യി­പ്പോള്‍ സ്വ­ന്തം ജാ­തി­യി­ലെ­ത്ത­ന്നെ ഒരാ­ളെ ഇഷ്ട­പ്പെ­ട്ടു വി­വാ­ഹം കഴി­ക്കാന്‍ തീ­രു­മാ­നി­ച്ചാ­ലും അതു സ്വ­ന്തം 
കു­ടും­ബ­ത്തി­ന­പ­മാ­ന­മാ­യി­ത്തോ­ന്നി­യാല്‍ മര­ണ­ശി­ക്ഷ­വി­ധി­ക്കു­ന്ന­വ­രും ഹി­ന്ദു­ക്ക­ളു­ടെ­യി­ട­യി­ലു­ണ്ട്.

­കു­ടും­ബ­ത്തി­ന­പ­മാ­ന­മാ­കാ­നു­ള്ള കാ­ര­ണ­ങ്ങ­ള­ന്വേ­ഷി­ച്ചാല്‍, സര്‍­വ്വം­സ­ഹി­ഷ്ണു­ക്ക­ളായ സനാ­തന ഹി­ന്ദു­ക്ക­ളെ ലോ­ക­ത്തി­ലെ­ത്ത­ന്നെ
 ഏറ്റ­വും വലിയ അസ­ഹി­ഷ്ണു­ക്ക­ളാ­യി­ക്കാ­ണേ­ണ്ടി­വ­രും. താന്‍ മു­റു­കെ പി­ടി­ക്കു­ന്ന വി­ശ്വാ­സ­ങ്ങ­ളെ എതിര്‍­ക്കാന്‍ ആര്‍­ക്കും അവ­കാ­ശ­മി­ല്ലെ­ന്നു­ള്ള­തി­ലും 
 കവി­ഞ്ഞ്, താന്‍ അം­ഗീ­ക­രി­ക്കാ­ത്ത വി­ശ്വാ­സ­ങ്ങള്‍ സ്വീ­ക­രി­ക്കു­ന്ന വേ­ണ്ട­പ്പെ­ട്ട­വര്‍ മര­ണ­ശി­ക്ഷ അര്‍­ഹി­ക്കു­ന്ന­വ­രാ­ണെ­ന്നു­ള്ള 
 ചി­ന്ത­ക­ളി­ലെ­ത്തി­നില്‍­ക്കു­ന്ന സഹി­ഷ്ണു­ത.

അ­പ്പോള്‍ ഒരു വി­ഭാ­ഗം സഹി­ഷ്ണു­ക്ക­ളാ­ണെ­ന്നു­ള്ള പ്ര­ചാ­ര­ത്തി­ന്റെ അടി­സ്ഥാ­ന­മെ­ന്താ­ണ്? വെ­റു­മൊ­രു കണ്‍­കെ­ട്ടു­വി­ദ്യ­മാ­ത്ര­മാ­ണി­ത്. 
പര­സ്പ­രം വി­ശ്വാ­സ­ങ്ങ­ളെ ബഹു­മാ­നി­ക്കാ­നാ­വു­ന്ന ഒരു സമൂ­ഹം നവോ­ത്ഥാ­ന­ശ്ര­മ­ങ്ങ­ളു­ടെ ഭാ­ഗ­മാ­യി വളര്‍­ന്നു വന്നി­രു­ന്നു. ഈ സാ­മൂ­ഹ്യ 
നവോ­ത്ഥാ­ന­ശ്ര­മ­ങ്ങള്‍ സ്വാ­ത­ന്ത്ര്യ സമ­ര­ത്തി­നും മു­മ്പ് വേ­രു­ള്ള­വ­യാ­ണ്. മാ­ത്ര­മ­ല്ല, പു­രോ­ഗ­മന ചി­ന്താ­ഗ­തി­ക്കാ­രായ ഭര­ണാ­ധി­കാ­രി­ക­ളു­ടെ
 ശ്ര­മ­ഫ­ല­മാ­യി ആരോ­ഗ്യ­പ­ര­മായ സാ­മൂ­ഹ്യ­വ­ളര്‍­ച്ച­നേ­ടാ­നു­ള്ള ശ്ര­മ­ങ്ങ­ളു­ടെ ഭാ­ഗ­മാ­യി പര­സ്പ­ര­ബ­ഹു­മാ­ന­ത്തി­ന്റെ പാ­ഠ­ങ്ങള്‍ സമൂ­ഹ­ത്തില്‍ 
 വേ­രൂ­ന്നു­ക­യും ചെ­യ്തു. ഇങ്ങ­നെ, അയല്‍­ക്കാ­ര­ന്റെ വി­ശ്വാ­സ­ങ്ങ­ളെ ബഹു­മാ­നി­ക്കാന്‍ ശീ­ലി­ച്ചി­രു­ന്ന ഒരു സമൂ­ഹ­ത്തെ, നി­ങ്ങള്‍ 
 അപ­ര­ന്റെ വി­ശ്വാ­സ­ങ്ങ­ളെ സഹി­ക്കു­ക­യാ­ണെ­ന്നു പഠി­പ്പി­ച്ചു തു­ട­ങ്ങു­ന്ന­ത്, ഭൂ­രി­പ­ക്ഷ വര്‍­ഗ്ഗീ­യ­ത­യു­ടെ വക്താ­ക്ക­ളാ­ണ്.

­പ­ര­സ്പ­രം ബഹു­മാ­നി­ക്കു­ക­യും, സഹാ­യ­ങ്ങള്‍ ചെ­യ്യു­ക­യും ചെ­യ്യു­ന്ന­ത് ശീ­ല­മാ­ക്കിയ ഒരു സമൂ­ഹ­ത്തില്‍ അപ­രന്‍ നി­ന്റെ സഹി­ഷ്ണു­ത­യെ 
മു­ത­ലെ­ടു­ക്കു­ക­യാ­ണെ­ന്നു­ള്ള പ്ര­ച­ര­ണം അസ­ഹി­ഷ്ണു­ത­യു­ടെ വി­ത്തു­കള്‍ പാ­കി. ന്യൂ­ന­പ­ക്ഷ വര്‍­ഗ്ഗീ­യ­ത­യു­ടെ വക്താ­ക്ക­ളു­ടെ ഇട­പെ­ട­ലാ­ക­ട്ടെ, 
ഇതി­ന് ആക്കം കൂ­ട്ടി­യ­തേ­യു­ള്ളൂ. ഇല്ലാ­ത്ത നി­ന്റെ 'ക്ഷ­മ­യു­ടെ നെ­ല്ലി­പ്പ­ല­ക' നീ കണ്ടു­ക­ഴി­ഞ്ഞു­വെ­ന്നു രണ്ടു­കൂ­ട്ട­രേ­യും വി­ശ്വ­സി­പ്പി­ക്കാ­നായ
 വര്‍­ഗ്ഗീ­യ­വാ­ദി­കള്‍­ക്കാ­വ­ട്ടെ, അവര്‍­ക്കു വേ­ണ്ട­തു കി­ട്ടു­ക­യും ചെ­യ്തു: രാ­ഷ്ട്രീയ നേ­തൃ­ത്വ­വും വി­ല­പേ­ശല്‍ ശേ­ഷി­യും. പു­രോ­ഗ­മ­ന­പ­ര­മായ
  ഒരു സമൂ­ഹ­ത്തില്‍ തി­ക­ച്ചും സ്വാ­ഭാ­വി­ക­മായ പര­സ്പര ബഹു­മാ­ന­ത്തെ, കൃ­ത്രി­മ­മായ സഹി­ഷ്ണു­ത­യാ­യി തെ­റ്റി­ദ്ധ­രി­പ്പി­ച്ച് രാ­ഷ്ട്രീ­യ­മാ­യി 
  നേ­ട്ടം ലക്ഷ്യ­മി­ട്ട­വ­രു­ടെ ഇര­കള്‍ മാ­ത്ര­മാ­ണ്, സഹി­ഷ്ണു­താ­വാ­ദ­വു­മാ­യി രം­ഗ­ത്തെ­ത്തു­ന്ന­വര്‍.

­ന­മ്മള്‍ പകര്‍­ന്നു കൊ­ടു­ക്കേ­ണ്ട­തും അള­ക്കേ­ണ്ട­തും സഹി­ഷ്ണു­ത­യു­ടെ പാ­ഠ­ങ്ങ­ള­ല്ല, പര­സ്പ­ര­ബ­ഹു­മാ­ന­ത്തി­ന്റെ പാ­ഠ­ങ്ങ­ളാ­ണ്. സഹി­ഷ്ണു­ത­യു­ടെ 
പാ­ഠ­ങ്ങള്‍­ക്കു­ള്ള പ്ര­ശ്ന­മെ­ന്തെ­ന്നാല്‍, ഒരു പരി­ധി­ക്ക­പ്പു­റം ഒരാ­ളു­ടെ­യും സം­യ­മ­നം കാ­ത്തു­സൂ­ക്ഷി­ക്കാന്‍ അന്യ­നോ­ടു­ള്ള ­സ­ഹി­ഷ്ണു­ത 
അവ­നെ സഹാ­യി­ക്കി­ല്ല. എന്നാല്‍ പര­സ്പ­ര­ബ­ഹു­മാ­ന­ത്തി­ന്റേ­യും തി­ര­ച്ച­റി­യ­ലി­ന്റേ­യും പാ­ഠ­ങ്ങള്‍ ബന്ധ­ങ്ങള്‍­ക്ക് കൂ­ടു­തല്‍ ആഴം നല്‍­കു­ന്നു­.

­മ­റ്റൊ­രു തര­ത്തില്‍ പറ­ഞ്ഞാല്‍, മറ്റു വി­ശ്വാ­സ­ങ്ങള്‍­ക്കും ഒരു പരി­ധി­വ­രെ ബഹു­മാ­നം കൊ­ടു­ത്തു കൊ­ണ്ട് അവ­രെ­യും സമൂ­ഹ­ത്തി­ന്റെ 
ഭാ­ഗ­മാ­യി­ക്ക­ണ്ട് ജീ­വി­ക്കാ­നു­ള്ള കഴി­വാ­ണ് അള­ക്കേ­ണ്ട­ത്. ഇതും മത­വി­ശ്വാ­സ­വു­മാ­യി യാ­തൊ­രു ബന്ധ­വു­മു­ണ്ടാ­ക­ണ­മെ­ന്നി­ല്ല. ഇതു 
പല­പ്പോ­ഴും ഒരു പ്ര­ദേ­ശ­ത്തെ വി­ശ്വാ­സി­സ­മൂ­ഹ­ത്തി­ന്റെ രാ­ഷ്ട്രീ­യ­വും­,­സാ­മൂ­ഹ്യ­വു­മായ കാ­ഴ്ച­പ്പാ­ടു­ക­ളു­മാ­യാ­ണ് കൂ­ടു­തല്‍ ബന്ധ­പ്പെ­ട്ടി­രി­ക്കു­ന്ന­ത്.

­ഹി­ന്ദു­ക്ക­ളോ­ളം തന്നെ മു­സ്ലി­ങ്ങ­ളും­/­ക്രി­സ്ത്യാ­നി­ക­ളും ഉള്ള പ്ര­ദേ­ശ­ങ്ങ­ളില്‍ ജനി­ച്ചു വളര്‍­ന്ന­വര്‍­ക്ക്, പല­പ്പോ­ഴും പര­സ്പ­രം മന­സ്സി­ലാ­ക്കാന്‍ 
കൂ­ടു­തല്‍ അവ­സ­രം ലഭി­ക്കും. അതു­വ­ഴി സാ­മൂ­ഹ്യ­പ­ര­മാ­യി അടു­ത്ത­റി­യാ­നും സാ­ധി­ക്കും. എന്നാല്‍ ഒരു പ്ര­ത്യേക വി­ഭാ­ഗം ഭൂ­രി­പ­ക്ഷ­മായ
 പ്ര­ദേ­ശ­ത്ത്, രണ്ടു വി­ഭാ­ഗ­ങ്ങ­ളും പര­സ്പ­രം അടു­ത്ത് മന­സ്സി­ലാ­ക്കു­ന്ന­ത് അപൂര്‍­വ്വ­മാ­യി­രി­ക്കും. അതി­നാല്‍­ത്ത­ന്നെ, പര­സ്പ­രം വി­ശ്വാ­സ­ങ്ങ­ളെ
  ബഹു­മാ­നി­ക്കാ­നു­ള്ള ശേ­ഷി അവര്‍ ആര്‍­ജ്ജി­ക്കു­ന്നു­മി­ല്ല.

ഇ­തു വി­ശ്വാ­സ­ത്തി­ന്റെ പ്ര­ശ്ന­ത്തേ­ക്കാ­ളും സാ­മൂ­ഹ്യ­പ­ര­മായ അന്യ­വ­ത്ക­ര­ണ­ത്തി­ന്റേ­യും അന്യ­ന്റെ സം­സ്കാ­ര­ത്തെ മന­സ്സി­ലാ­ക്കാന്‍ 
കഴി­യാ­ത്ത­തി­ന്റേ­യും പ്ര­ശ്ന­ങ്ങ­ളാ­ണ്. കേ­ര­ള­ത്തില്‍ ഒരു പരി­ധി­വ­രെ അ­ന്യ­വ­ത്ക­ര­ണം­ ഇല്ലാ­തി­രു­ന്ന­തി­ന്റെ കാ­ര­ണം, 
യൂ­ണി­ഫോ­റ­ങ്ങ­ളി­ല്ലാ­ത്ത സര്‍­ക്കാര്‍ പ്രൈ­മ­റി വി­ദ്യാ­ല­യ­ങ്ങ­ളില്‍ ഒരു­മി­ച്ചു പഠി­ച്ചു വളര്‍­ന്ന തല­മു­റ­ക­ളാ­ണ്. ജാ­തി­/­മത സം­ഘ­ട­ന­ക­ളു­ടെ 
പ്രൈ­മ­റി­വി­ദ്യാ­ല­യ­ങ്ങള്‍ സാര്‍­വ­ത്രി­ക­മാ­കു­ന്ന­ത്, ഈ അന്യ­വ­ത്ക­ര­ണ­ത്തി­ന് ആക്കം കൂ­ട്ടു­ന്നു­ണ്ടാ­ക­ണം­.

ഇ­ത്ത­ര­ത്തില്‍ വളര്‍­ന്നു വരു­ന്ന അല്ലെ­ങ്കില്‍ നി­ല­നില്‍­ക്കു­ന്ന പര­സ്പ­ര­ബ­ഹു­മാ­ന­ത്തി­ന്റെ പാ­ഠ­ങ്ങ­ളെ, അങ്ങ­നെ­ത്ത­ന്നെ കണ്ട് അവ­യെ
 പരി­പോ­ഷി­പ്പി­ക്കാ­നാ­ണ് ശ്ര­മി­ക്കേ­ണ്ട­ത്, ഒപ്പം അന്യ­വ­ത്ക­ര­ണ­ശ്ര­മ­ങ്ങ­ളെ ചെ­റു­ക്കാ­നും. അല്ലാ­തെ, സഹി­ഷ്ണു­ത­യാ­യി തെ­റ്റി­ദ്ധ­രി­ക്കു­ക­യും, 
 പി­ന്നീ­ട് വ്യ­ക്ത­മായ രാ­ഷ്ട്രീയ അജ­ണ്ട­ക­ളോ­ടു­കൂ­ടിയ ഭൂ­രി­പ­ക്ഷ വര്‍­ഗ്ഗീ­യ­ത­യു­ടെ വക്താ­ക്കള്‍ അക്ര­മ­മ­ഴി­ച്ചു­വി­ടു­മ്പോള്‍ അവ­യെ സഹി­ഷ്ണു­ത­യു­ടെ 
 പരി­ധി­ക­ഴി­ഞ്ഞ­താ­യി വ്യാ­ഖ്യാ­നി­ച്ച് തൃ­പ്തി­യ­ട­യു­ക­യു­മ­ല്ല വേ­ണ്ട­ത്. അതു­പോ­ലെ, ന്യൂ­ന­പ­ക്ഷ വര്‍­ഗ്ഗീ­യ­ത­യു­ടെ അന്യ­വ­ത്ക­രണ ശ്ര­മ­ങ്ങ­ളു­ടെ 
 ഇര­ക­ളാ­വു­ന്ന­വ­രെ അസി­ഷ്ണു­ക്ക­ളായ വി­ശ്വാ­സ­ത്തി­ന്റെ വക്താ­ക്ക­ളാ­ക്കാന്‍ ധൃ­തി­പി­ടി­ക്കു­ന്ന­തി­നു മു­മ്പേ, അന്യ­വ­ത്ക­ര­ണ­ശ്ര­മ­ങ്ങ­ളെ 
 ചെ­റു­ത്ത് മു­ഖ്യ­ധാ­ര­യി­ലേ­ക്ക് ഇവ­രേ­യും കൂ­ടി­കൊ­ണ്ടു­വ­രാ­നാ­ണ് ശ്ര­മി­ക്കേ­ണ്ട­ത്.

­സ­ഹി­ഷ്ണു­ത­യു­ടെ പാ­ഠ­ങ്ങ­ളും കണ­ക്കു­ക­ളും നി­ര­ത്തു­ന്ന­തി­നു പക­രം നമു­ക്ക് പര­സ്പര ബഹു­മാ­ന­ത്തി­ന്റെ പാ­ഠ­ങ്ങള്‍ പഠി­പ്പി­ക്കാം. 
'നി­ന­ക്കു അവ­നെ സഹി­ച്ചു വേ­ണം ജീ­വി­ക്കാന്‍' എന്ന­തി­നു പക­രം 'നീ അവ­നേ­യും ബഹു­മാ­നി­ക്കു­ക' എന്നു പഠി­പ്പി­ക്കാം­.

(Jul 10, 2010)\footnote{http://malayal.am/വാര്‍ത്ത/വിശകലനം/6671/സഹിഷ്ണുത-എന്ന-മിത്ത്}
\newpage
