\section*{സാരിയെക്കുറിച്ച് എന്റെ വിനീത അഭിപ്രായം}
\vskip 2pt


ഞാന്‍ ഈ കുറിപ്പ് എഴുതുന്നത് വിഷ്ണുപ്രസാദിന്റെ പോസ്റ്റും\footnote{http://prathibasha.blogspot.com/2007/10/blog-post_13.html} അതിലെ മറ്റു കണ്ണികളും അവിടെയുള്ള ചര്‍ച്ചകളും കണ്ടാണ്.

സാരിക്ക് പുതുതലമുറ (ഇപ്പൊ സാരിയുടുത്ത് തുടങ്ങുന്നവരുടെ തലമുറ) കൊടുക്കുന്ന സ്ഥാനം ഞാന്‍ വലുതായി എന്ന് സ്വയവും മറ്റുള്ളവരെയും തോന്നിപ്പിക്കാനുള്ള ഒരു വസ്ത്രം എന്ന നിലയിലാണെന്നാണ് എന്റെ തോന്നല്‍. ചില സംഭവങ്ങളിലൂടെ വ്യക്തമാക്കാന്‍ ശ്രമിക്കാം. ഓണത്തിന് വീട്ടീപ്പോവാന്‍ കഴിയാതിരുന്ന എന്നെ ഒരു അനിയത്തിക്കുട്ടി ഫോണ്‍ വിളിച്ച് പറഞ്ഞു, അവള്‍ സാരിയാണ് ഉടുത്തതെന്ന്, അവള്‍ക്ക് സാരി മുതിര്‍ന്നവരുടെ കൂട്ടത്തിലേക്കുള്ള ഒരു ചവിട്ടു പടിയാണ്. സ്കൂളില്‍ പഠിപ്പിക്കാന്‍ പോയ എന്റെ ക്ലാസ്മേറ്റ് ആദ്യദിവസം ചുരിദാര്‍ ഇട്ടു ചെന്നപ്പോള്‍ കുട്ടികള്‍ക്ക് തമാശ, പിറ്റേന്ന് സാരിയും ഉടുത്ത് ചെന്നപ്പോള്‍ എവിടെനിന്നല്ലാതെ ബഹുമാനം, അവിടെ സാരി മുതിര്‍ന്ന സ്ത്രീയുടെ പരിവേഷം നല്‍കുന്നു. വെറും അഞ്വരമീറ്റര്‍ തുണിക്ക് ഇത്രയും മാറ്റങ്ങള്‍ മനുഷ്യമനസ്സില്‍ വരുത്താന്‍ കഴിയുമെങ്കില്‍ അത് ചില്ലറയല്ല എന്നാണ്എന്റെ വിനീത അഭിപ്രായം. ഞാന്‍ സാരി ഉടുക്കുന്നു അല്ലെങ്കില്‍ എനിക്ക് സാരി ഉടുക്കാനറിയാം എന്ന് എന്നോടു പറഞ്ഞ ഓരോ പെങ്കുട്ടിയും അത് ഒരു പൊതു വസ്ത്രം ധരിക്കാനറിയാം എന്നതിനേക്കാളുപരി, ihaveaskill എന്ന രീതിയിലാണ് എന്നോടു പറഞ്ഞിട്ടുള്ളത്. അഞ്ചരമീറ്റര്‍ തുണി അഴിഞ്ഞു വീഴാതെ ധരിച്ച് സ്വതന്ത്രമായി നടക്കുക എന്നത് ഒരു കഴിവ് തന്നെയാണ്.

പിന്നെ ഞാന്‍ കണ്ടറിഞ്ഞിടത്തോളം, സാരി സ്ഥിരമായി ഉടുക്കാന്‍ താല്‍പ്പര്യമുള്ളവര്‍ പുതുതലമുറയില്‍ ഇല്ല എന്നു പറയാം, ഒരു സെറിമോണിയല്‍ സ്റ്റാറ്റസ് ആണ് എല്ലാവര്‍ക്കും സാരിയോടുള്ളത്. അത്ര തന്നെ മതി എന്നാണ് എന്റെ അഭിപ്രായവും. അല്ലാതെ “സാരിയുടുക്കാനറിയാത്തവര്‍ മലയാളി മങ്കയാവില്ല” എന്നത് വരട്ടു തത്വവാദം എന്ന ഗണത്തില്‍ പെടുത്താനാണെനിക്കിഷ്ടം.

സാരി ധരിക്കാനറിയുന്നവര്‍ ധരിക്കട്ടെ, പക്ഷെ അതൊരിക്കലും ഒരു രീതിയിലും അവശ്യ യോഗ്യതയാവരുത്. സാരി ധരിക്കില്ലെങ്കിലും നന്നായി പഠിപ്പിക്കാനറിയുന്ന ഒരു സ്ത്രീയെ നിങ്ങള്‍ക്ക് ടീച്ചറാവാനുള്ള യോഗ്യതയില്ല എന്നു പറഞ്ഞ് തിരിച്ചയക്കുന്നത് പിന്തിരിപ്പന്‍ നയമാണ്. സാരിയുടെ പ്രധാന യോഗ്യത എന്നു ഞാന്‍ പറയുക, ഒരേ സമയം executive ഉം traditional ഉം ആയ ഒരു വസ്ത്രം എന്നതാണ്. മുണ്ടുടുത്ത പുരുഷന്‍മാര്‍ സ്വീകരിക്കപ്പെടാത്ത സ്ഥലങ്ങളില്‍ പ്പോലും സാരിയുടുത്ത സ്ത്രീകള്‍ സ്വീകരിക്കപ്പെടും.

(Oct 21, 2007)
\newpage
