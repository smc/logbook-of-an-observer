\secstar{നന്ദി മൈക്രോസോഫ്റ്റ്... നന്ദി മമ്മൂട്ടി...}
\enlargethispage{3\baselineskip}
%\vskip 2pt

മൈക്രോസോഫ്റ്റും മമ്മൂട്ടിയും e-literacy പരിപാടിയ്ക്കുവേണ്ടി കൈകോര്‍ക്കാന്‍ ആലോചിക്കുന്നു. 
സാമൂഹ്യബോധത്തിന്റെ പേരില്‍ കൊക്കൊകോള പരസ്യത്തില്‍നിന്നു പിന്‍വാങ്ങാന്‍ ധൈര്യം കാണിച്ച മമ്മൂട്ടി, 
സോഫ്റ്റ്‌വെയര്‍ സ്വാതന്ത്ര്യത്തിന്റെ പ്രാധാന്യം മനസ്സിലാക്കാത്തതുകൊണ്ടാണു് ഇത്തരമൊരു തീരുമാനത്തിലെത്തിയതെന്നു 
കരുതുന്നു. സമൂഹത്തിന്റെ ഉന്നമനത്തിനും സംരക്ഷണത്തിനും സ്വാതന്ത്ര്യം അത്യന്താപേക്ഷിതമാണെന്നും, അതു് ഭരണഘടന 
നല്‍കുന്ന അവകാശങ്ങള്‍ക്കുമപ്പുറം അറിവിന്റെ സ്വാതന്ത്ര്യത്തോടെയേ സാധ്യമാവൂ എന്നും നാം ഓരോരുത്തരും 
മനസ്സിലാക്കേണ്ടതാണു്.

ആധുനികലോകത്തെ അറിവിന്റെ രൂപമായ വിവരസാങ്കേതികവിദ്യയില്‍, വിവരവും വിദ്യയും സങ്കേതവും സമൂഹത്തില്‍നിന്നും അകറ്റി, 
വാണിജ്യവല്‍ക്കരണത്തിനും വിപണത്തിനും ശ്രമിക്കുന്ന കുത്തകളുമായി കൈകോര്‍ത്തു് എല്ലാ ജനവിഭാഗങ്ങളെയും 
സാക്ഷരരാക്കാന്‍ കഴിയില്ല. പകരം സമൂഹത്തെ ഒന്നടങ്കം ചില വമ്പന്‍മാര്‍ക്കു വിധേയരായി നിര്‍ത്താനെ അതുപകരിക്കൂ. 
സമൂഹത്തിന്റെ ഉന്നമനത്തിനു വേണ്ടതു് സാമൂഹ്യനീതിയും സ്വാതന്ത്ര്യവും ഉറപ്പാക്കുകയും പ്രോത്സാഹിപ്പിക്കുകയും 
ചെയ്യുന്ന സ്വതന്ത്ര സോഫ്റ്റ്‌വേറുകളാണു്. സമൂഹത്തിന്റെ ആരോഗ്യകരമായ നിലനില്‍പ്പിനാവശ്യമായ, പങ്കുവയ്ക്കലിന്റെയും 
പഠനത്തിന്റെയും പരിഷ്കരണത്തിന്റെയും പാഠങ്ങള്‍ ഉറപ്പാക്കുന്ന സ്വതന്ത്രസങ്കേതങ്ങളേക്കാള്‍ മറ്റു സങ്കേതങ്ങള്‍ എങ്ങനെ 
സാമൂഹ്യ ഉന്നമനത്തിനു് സഹായകമാവും? അറിവിനെപ്പോലും വിപണിയിലെ ആയുധമാക്കുന്ന വൃത്തികെട്ട വില്‍പ്പനതന്ത്രങ്ങളെ 
എതിര്‍ത്തു തോല്‍പ്പിക്കേണ്ടതു് സമൂഹമാണു്. സാമൂഹ്യവികസനത്തിന്റെയും വിദ്യാഭ്യാസത്തിന്റെയും പേരുപറഞ്ഞു് 
ജനങ്ങളെ ഒന്നടങ്കം തങ്ങളുടെ അടിമകളാക്കാനുള്ള ശ്രമങ്ങളെ ഒറ്റക്കെട്ടായി പൊതുസമൂഹം ചെറുത്തു തോല്‍പ്പിക്കണം. 
സമൂഹത്തെ സ്വാധീനിക്കാന്‍ കഴിവുള്ള വ്യക്തികളേയും സ്ഥാപനങ്ങളെയും കൂട്ടുപിടിക്കാനുള്ള കുത്സിതശ്രമത്തെ 
തകര്‍ക്കുകയും വേണം.

നേര്‍ക്കുനേര്‍ നിന്നുള്ള വിപണിയുദ്ധത്തില്‍ കാലിടറിത്തുടങ്ങിയതും മത്സരത്തിന്റെ ആധിക്യവുമാണു്, standardization ന്റെയും 
വിദ്യാഭ്യാസപ്രവര്‍ത്തനങ്ങളുടെയും രൂപത്തിലേക്കു് വില്‍പ്പനതന്ത്രങ്ങളെ മാറ്റിയെഴുതാന്‍ കുത്തകകളെ പ്രേരിപ്പിക്കുന്നതു്. 
 സൗജന്യവിദ്യാഭ്യാസത്തിലൂടെ സമൂഹത്തെ സ്വാതന്ത്ര്യത്തില്‍ നിന്നകറ്റാനുള്ള ശ്രമമാണു് ഇത്തരം സാക്ഷരതാ 
പ്രവര്‍ത്തനങ്ങളിലൂടെ നടത്തുന്നതു്. ഇതിനെതിരെ പ്രതികരിക്കേണ്ടതു് നമ്മളോരോരുത്തരുമടങ്ങുന്ന സമൂഹമാണു്.

\vskip 2pt

പ്രതികരിക്കാന്‍ താങ്കള്‍ ആഗ്രഹിക്കുന്നുവെങ്കില്‍, അത്, 
ഈ കത്തിലൊരൊപ്പിട്ടു്\footnote{\url{http://fci.wikia.com/wiki/Open_Letter_To_Mammooty}} തുടങ്ങൂ. 
സ്വാതന്ത്ര്യത്തിന്റെ സന്ദേശം ജനങ്ങളിലേക്കെത്തിക്കാന്‍ വീണ്ടുമൊരവസരം തന്നതിനു് മൈക്രോസോഫ്റ്റിനും മമ്മൂട്ടിക്കും നന്ദി.

\begin{flushright}(April 18, 2008)\end{flushright}
\newpage
