\secstar{Hospital Log 7}
\vskip 2pt
\begin{english}

It is the 6\textsuperscript{th} of June and in Kerala, it is raining like hell it seems. Anyway, in Tamil Nadu, it seems they are having a calm and quiet summer. I never liked the change of seasons although I knew they were inevitable. Just like changing the place of your operation. For the last few years, I had bases in many different places. However, leukemia made Vellore my base for the last 8 months. I was admitted on the 7\textsuperscript{th} of October last year. So now it has been exactly 8 months. I did see different lives and people during these days, all of them nurses, doctors, patients and relatives. May be a very few other hospital staff too. Many know me by my masked appearance by now, and in turn I know many of them too. The friend circle did grow and I should say I have some good friends among them. From nurses who come from OPD, to those in wards and doctors who are PG registrars to consultants and department heads. I do turn many heads because of a simple reason. When I get admitted, I usually go there with a laptop and most of the times, many see books as niche or weird. 

Very recently, I got a chance to catch up with one of my classmates from JNV. Though we sat in the same class for, kind of 7 years, I couldn't describe her as a friend. Only during the last 3-4 days did we get to know of each other really. While we were catching up, she kind of gave me a very good advice. :) Not to have complex words and ideas in my world and to see everything as simple and plain. Though I don't agree my thoughts and ideas are complex just because it doesn't align with the perspective of the majority, the thing she said just before struck me even more. It was, ``this is not a friendly conversation anymore". I might have pushed her to the intellectual limits or the black truth part, I don't know. I was just arguing (which I have been doing so well for some years) about our school not being progressive. It might have hurt her pride and good memories she cherished of being a JNV-ite.

Interestingly, she was not the first one to tell me to shut my mouth. Many of my classmates from my JNV days bite me back when I tell them about the gap JNV created among its male and female population. For them, its not friendly talk. I am someone who have been called a male chauvinist and many other names in different forums for just raising simple queries. I'm quite sure that people who call me so wouldn't have read J.Devika, C.S. Chandrika, Kadeeja Mumtaz or Urvasi Bhutalia as much as I have. Many of them might even agree with a B.S. Sherin version of having no individuality for women or women are beyond individuals when they are collectively addressed with family ties. But none who might have read `The Other Side of Silence' by Urvashi Bhutalia would agree with it. She clearly shows how women were arm twisted during partition years and how much women activists considered the opinion of women back then. Still, when I tell my friends that it was bad for our school when they don't allow us to even communicate freely with our classmates, it is me who gets arm-twisted. :) I guess even JNV has got holy status at least in the minds of ex JNV-ites. 

Or may be its my problem. I might be out of place. The niche and weird character that doctors and nurses see in me might be true. But I believe the thoughts and ideas I developed and words I write as this guy is far more better. For me, they are progressive, they are liberal. May be society does not have a place for niche and weird. You might get into the stereotype intellectual. I tried to beat that appearance by growing a long beard and hair at same time and carefully seeing to wear shorts and t-shirts. But it might not work always. People like to compartmentalize. The most recent complaint I heard of compartmentalization was from the National Award winning actor, Salim Kumar. Even though he was talking about the trends in cinema field, what he said holds true for any field. It is a part and parcel of the simplistic view my friend suggested to me instead of the complex words and ideas. She is someone who is trained to nurse ailing patients and thus, knows the complex anatomy and physiology of human beings are not anywhere near simple. I don't know why she then wants the thoughts of human beings to be simplistic. 

The answer, I think, lies in the fact that simple doesn't really mean simple, but accepted world view. From her perspective and many others' I know during the years, it is not right to fight the wrongs in the world. Some time fighting wrong just gets us being portrayed as proponents of wrong, though we were never anything like that. :)
\end{english}
\newpage
