\secstar{ചില സാമൂഹ്യശാസ്ത്ര നിരീക്ഷണങ്ങളും അഭിപ്രായങ്ങളും}
\vskip 2pt

സാധാരണ ഒരു വിഷയത്തില്‍ രണ്ടു് പോസ്റ്റ് പോയിട്ടു് ഒരു പോസ്റ്റുപോലും ഇടാത്ത ഞാന്‍ ഇങ്ങനെയെഴുതുന്നതു് 
വല്ല ഹിഡന്‍ അജണ്ടയും വച്ചുകൊണ്ടാണോ എന്നു ചോദിച്ചാല്‍ ചില കാര്യങ്ങളൊക്കെ ചര്‍ച്ചചെയ്തു് കാണണമെന്ന 
ഒരു പ്രത്യക്ഷ അജണ്ട ഉണ്ടെന്നാണു് മറുപടി. ചോദിക്കേണ്ട പല ചോദ്യങ്ങളും ചോദിക്കാതിരിക്കുകയും, ആവശ്യമില്ലാത്ത 
ചോദ്യങ്ങളും ചര്‍ച്ചകളും നടത്തുകയുമാണു് വര്‍ത്തമാനകാല മാധ്യമങ്ങളുടെയും രാഷ്ട്രീയത്തിന്റയും ഒരു രീതി. 
പ്രസക്തമായ ഈ വിഷയത്തിലും അതങ്ങനെത്തന്നെയാണെന്നു തോന്നുന്നു. ബ്ലോഗില്‍നടന്ന ചര്‍ച്ചകളില്‍പോലും 
മറ്റാരോ ഉണ്ടാക്കിയിട്ട ഒരു അജണ്ടയിന്‍മേല്‍ ചര്‍ച്ച തുടങ്ങി, അതേ ട്രാക്കിലൂടെ പോകുന്ന ഒരു തോന്നല്‍. വേണ്ട 
പലകാര്യങ്ങളും ചര്‍ച്ചചെയ്യാതിരിക്കാന്‍ വേണ്ടിയാണോ ഇത്തരം ഒരു വിവാദം എന്നൊരു തോന്നല്‍ പലരും 
പ്രകടിപ്പിച്ചു് കണ്ടെങ്കിലും ആരും അതു കാര്യമായി വിശകലനംചെയ്തു കണ്ടില്ല.

പഠിക്കുന്ന കുട്ടികളുടെ തലത്തിലേക്കിറങ്ങിച്ചെന്നു് പ്രശ്നത്തെ അപഗ്രഥിക്കാന്‍ ചെറിയ ശ്രമമേ കണ്ടുള്ളു. കുട്ടികള്‍ 
ഇക്കാര്യം അവരുടേതായ റിസോഴ്സുകളില്‍നിന്നും മനസ്സിലാക്കണം എന്നതാണു് പുതിയരീതിയുടെ സ്വഭാവം. 
പുസ്തകങ്ങളേക്കാളും കുട്ടികള്‍ ആശ്രയിക്കേണ്ടതു് മാതാപിതാക്കളും അദ്ധ്യാപകരും മുതിര്‍ന്നപൌരന്മാരുമടങ്ങുന്ന 
റിസോഴ്സിനെയാണു്. അവിടെ പ്രധാന ചുമതല വഹിക്കേണ്ടത് അദ്ധ്യാപകരും. കുട്ടികളെ ഇത്തരം ഒരു രീതിയില്‍ 
അവരുടെ അപഗ്രഥനശേഷിയെ തിരിച്ചറിയാനും, ചെറിയ നിഷ്പക്ഷമായ വിവരണങ്ങള്‍ നല്‍കാനും, 
കാര്യങ്ങളേയും കാരണങ്ങളേയും തിരിച്ചറിയാനും, വസ്തുതകള്‍ മനസ്സിലാക്കാനും സഹായിക്കേണ്ടതും, നാനാവിധത്തിലുള്ള 
കഴിവുകളെ അളന്നു് മാര്‍ക്കു കൊടുക്കേണ്ടതും അദ്ധ്യാപകനാണു്. ലഘുവിവരണങ്ങളും ചോദ്യങ്ങളും അടങ്ങിയ തീരെ 
ലളിതമായ പാഠപുസ്തകങ്ങള്‍ക്കു് പഠനത്തില്‍ രണ്ടാംസ്ഥാനമാണുള്ളത്. അവയുടെ അദ്ധ്യാപന വ്യാഖ്യാനത്തിനും, 
അദ്ധ്യാപനത്തിന്റെ ചുവടുപിടിച്ചു് കുട്ടി നടത്തുന്ന അന്വേഷണങ്ങള്‍ക്കുമാണു് പ്രഥമസ്ഥാനം.

പാഠത്തെക്കാളും ഗുരുവാണു് വിദ്യാര്‍ത്ഥികളെ കൂടുതല്‍ സ്വാധീനിക്കാന്‍ പോകുന്നതു്. ഇത്രയും പ്രൊഫഷനലായ, 
ഇത്രയും എഫേര്‍ട്ടു് എടുക്കാന്‍ താത്പര്യമുള്ള ഗുരുജനങ്ങളിന്നുണ്ടോ, ആവോ! ഉണ്ടെങ്കില്‍ത്തന്നെയും എല്ലാ 
സര്‍ക്കാര്‍/എയ്ഡഡ് സ്കൂളിലും കാണുമോ? മാത്രമല്ല, തങ്ങളുടെ ജോലിഭാരം വര്‍ദ്ധിക്കുന്നതില്‍ അദ്ധ്യാപകര്‍ 
ഇതുവരെ ഉത്ക്കണ്ഠയൊന്നും പ്രകടിപ്പിച്ചുകാണാത്തതുകൊണ്ടു് അവര്‍ക്കു കാര്യംതന്നെ കൃത്യമായി 
മനസ്സിലായിട്ടില്ലെന്നാണെനിക്കു തോന്നുന്നത്. കമ്യൂണിസ്റ്റു ബുദ്ധിജീവികള്‍ എന്തെങ്കിലും ഹിഡന്‍ അജണ്ട 
നടപ്പാക്കുന്നുണ്ടെങ്കില്‍ അതു അദ്ധ്യാപക സമൂഹത്തിലൂടെയായിരിക്കണം

\subsection*{പ്രതികരണങ്ങള്‍}

\begin{enumerate}

\item{അടകോടന്‍}

എല്ലാവിധ ക്രിമിനല്‍ സ്വഭാവങ്ങളും സമൂഹത്തില്‍ നാള്‍ക്കുനാള്‍ വര്‍ദ്ധിച്ചുകൊണ്ടിരിക്കുന്നു, 
അതില്‍ മതമില്ലാത്തവനും മതമുള്ളവനും കണക്കാണു്. അതുകൊണ്ടു് പാഠപുസ്തകത്തില്‍ ശാസ്ത്രത്തോടൊപ്പം സംസ്കാരവും 
പഠിപ്പിക്കട്ടെ. അതില്‍ 'മതമില്ലാത്ത ജീവന്‍' പ്രത്യേകമായി പഠിപ്പിക്കേണ്ട ആവശ്യവുമില്ല, സാമൂഹ്യമായ മാറ്റങ്ങള്‍ക്കനുസരിച്ചു് 
എല്ലായിടത്തും സ്വയം മാറ്റം വരും.

\item{അങ്കിള്‍}

ജാതി ആവശ്യമില്ലെന്നു പഠിച്ചുവരുന്ന ജീവനോടു് ജാതി പറഞ്ഞാലേ ആനുകൂല്യങ്ങള്‍ നല്‍കൂ എന്നു പറഞ്ഞാല്‍ അവനു 
ആശയകുഴപ്പം ഉണ്ടാകില്ലേ? മറ്റൊരിടത്തു് ഞാനിതവതരിപ്പിച്ചപ്പോള്‍, ഇതു് രണ്ടും രണ്ടായിട്ടു് കാണണമെന്നാണു് എന്നെ 
ഉപദേശിച്ചതു്.

\item{Rajeeve Chelanat}

"പ്രശ്നത്തെ പഠിക്കുന്ന കുട്ടികളുടെ തലത്തിലേക്കിറങ്ങിച്ചെന്നു് അപഗ്രഥിക്കാന്‍ ചെറിയ ശ്രമമേ കണ്ടുള്ളൂ". അത്രയെങ്കിലും 
കണ്ടുവല്ലോ ജിന്‍സ്. നന്നായി. അജണ്ടയും വ്യക്തം. "ഹിഡന്‍ അജണ്ട എന്നത്, അങ്ങനെ ഒന്നുണ്ടു് എന്നറിയുന്നവനു 
പോലും പ്രത്യക്ഷത്തില്‍ ദൃശ്യമാവില്ല". അത് ദൃശ്യമാകാന്‍ പിന്നെയെന്താണു് സാര്‍ ഒരു വഴി. ഒന്നു പറഞ്ഞുതരൂ.

\item{jinsbond007}

അങ്കിളേ, ഈ പറഞ്ഞ പാഠം തന്നെ വേറൊരു രീതിയില്‍ പഠിപ്പിച്ചാല്‍, കുട്ടിയ്ക്കു് മനസ്സിലാവുന്നതു് മതവും ജാതിയും 
ഒഴിവാക്കാനാവാത്തതാണെന്നാവും. അങ്ങനെയും വ്യാഖ്യാനിക്കാനും, കുട്ടി ചിന്തിച്ചു് അങ്ങനെയെത്തിയാല്‍ തിരുത്താന്‍ 
പാടില്ലെന്നുമുള്ള ഒരു രീതി ഉപയോഗിച്ചാല്‍. ജാതിയുടെയും മതത്തിന്റെയും പേരിലുള്ള ആനുകൂല്യങ്ങള്‍ 
അവകാശമായി നല്‍കപ്പെട്ടതാണെങ്കിലും, അതു ഉപയോഗിക്കുന്നതിനു ഒരു മാനുഷികവശം ഉണ്ടാക്കാന്‍ ഇത്തരം 
ഒരു പാഠം സഹായിച്ചാല്‍ നല്ലതല്ലെ? പട്ടികജാതി സംവരണം നേടിയ കുറെ കുട്ടികളെ എനിക്ക് നേരിട്ടറിയാവുന്നതു കൊണ്ടു 
പറഞ്ഞതാ. പിന്നെ രാജിവു്, ഹിഡന്‍ എന്ന വാക്കിന്റെ അര്‍ത്ഥം തന്നെ ഒളിച്ചുവയ്ക്കപ്പെട്ടതു് എന്നല്ലെ? 
വല്ലതും ഒളിച്ചുവയ്ക്കപ്പെട്ടിട്ടുണ്ടു് എന്നു് തറപ്പിച്ചു പറയാന്‍ ഞാന്‍ ത്രികാല ജ്ഞാനിയൊന്നുമല്ല, എനിക്കു തോന്നിയ ഒരു വേര്‍ഷന്‍ 
എഴുതി എന്നുമാത്രം. എന്തെങ്കിലും ഹിഡന്‍ അജണ്ട ഉണ്ടു്/ഇല്ല എന്ന ഒരാരോപണം ഒരിക്കലൂം തെളിയിക്കാന്‍ 
കഴിയാത്ത ഒന്നാണു്. അതു് പുറത്തുവരാനുള്ള ഒരേയൊരു സാധ്യത അങ്ങനെചെയ്ത അരെങ്കിലും വ്യക്തമായി 
സമ്മതിച്ചു വെളിപ്പെടുത്തുമ്പോള്‍ മാത്രമാണു്. അവിടെപ്പോലും ആധികാരിതയെ സംബന്ധിച്ചു് പ്രശ്നങ്ങള്‍ ഉയര്‍ന്നുവന്നേക്കാം. 
ഒരു ഹിഡന്‍ അജണ്ട എന്നതു് വെറും ഒരു ആരോപണമായി ഉന്നയിക്കാമെന്നു മാത്രമേ ഞാന്‍ പറഞ്ഞുള്ളു. അജണ്ട 
എന്നതു് ഒരു കാര്യപരിപാടിയല്ലെ, അതു മറച്ചുവച്ചു നടത്തുമ്പോളല്ലെ ഹിഡന്‍ അജണ്ട ആവുന്നതു് എന്നു കരുതിയാണു് 
ഞാന്‍ എഴുതിയതു്. മറച്ചുവച്ചു് ഒരു കാര്യപരിപാടി നടത്തുമ്പോള്‍, അതു് എങ്ങനെ ഏതു രീതിയില്‍ നടത്തും? 
മറ്റുപലതിന്റെയും പേരില്‍. അത്തൊരമൊരു അജണ്ട തെളിയിക്കപ്പെടണമെങ്കില്‍, പരസ്യമായി ഇതു നടത്തപ്പെട്ടു 
എന്നു സംശയിക്കപ്പെട്ട കാര്യപരിപാടികളൊക്കെ വ്യക്തമായി ഹിഡന്‍ അജണ്ട എന്ന സാധനത്തിന്റെ ലക്ഷ്യങ്ങളാണു് 
അല്ലെങ്കില്‍ അതുമാത്രമാണു് നിറവേറ്റിയിരുന്നത് എന്നു തെളിയിക്കണം. അതു് വളരെ ദുഷ്കരവും, എനിക്കറിയാവുന്നിടത്തോളം 
നടപ്പാക്കിയവരുടെ സഹായമില്ലാതെ തെളിയിക്കല്‍ അസാദ്ധ്യവുമാണു്. ഞാന്‍ എന്നെത്തന്നെ വിശദീകരിച്ചു എന്നു 
കരുതട്ടെ. പിന്നെ, പഠിക്കുന്ന കുട്ടിയേക്കാളും എനിക്ക് സംശയം പഠിപ്പിക്കുന്നവരെപ്പറ്റിയാണു്. ഇന്നത്തെ ക്ലാസ് മുറികള്‍ 
ഞാന്‍ പഠിച്ച കാലത്തെയല്ല എന്നു ഞാന്‍ മനസ്സിലാക്കുന്നു. അതെങ്ങനെയായിരുന്നു എന്നു ഞാന്‍ തൊട്ടുമുമ്പത്തെ 
പോസ്റ്റില്‍ എഴുതിയിട്ടുണ്ടു്. എന്റെ ആശയങ്ങള്‍ക്കു് വ്യക്തതയില്ലെങ്കില്‍ ഞാന്‍ തന്നെയാണുത്തരവാദി. 
പോയന്റ് ടു പോയന്റ് ക്ലിയര്‍ എന്ന രീതിയില്‍ ടെക്നിക്കല്‍ പേപ്പറായി എഴുതിയാല്‍ അതെന്റെ ഒരുപാടു സമയം 
കാര്‍ന്നുതിന്നും. അതിനാല്‍ ദയവായി ക്ഷമിക്കുക. ഒറ്റയിരുപ്പിനു് എനിക്കു് മനസ്സില്‍ തോന്നിയതെല്ലാം എഴുതി, വീണ്ടും 
വ്യക്തമായി വായിച്ചുപോലും നോക്കാതെയാണു് ഞാനിതു് പ്രസിദ്ധീകരിച്ചതു്. അതിന്റെ തെറ്റും കുറ്റങ്ങളും പൊറുക്കുക.

\end{enumerate}

\newpage
