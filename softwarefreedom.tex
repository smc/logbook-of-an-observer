\secstar{സോഫ്റ്റ്‌വെയര്‍ സ്വാതന്ത്ര്യം}
\vskip 2pt

സന്തോഷിന്റെ ബ്ലോഗില്‍\footnote{\url{http://santhoshspeaking.blogspot.com/2008/01/blog-post_21.html}} നടന്ന ചര്‍ച്ചയോടും, 
അനിവറിന്റെ വിശകലനത്തോടും\footnote{\url{http://replyspot.blogspot.com/2008/01/blog-post_24.html}} ചേര്‍ത്തു വായിക്കാന്‍ 
എന്റെ ചില നിരീക്ഷണങ്ങള്‍.

എനിക്ക് പൈറേറ്റ് (pirate\footnote{\url{http://en.wiktionary.org/wiki/pirate}} - കടല്‍കൊള്ളക്കാരന്‍) 
എന്നുപയോഗിക്കുന്നതിനോടുതന്നെ എതിര്‍പ്പാണ്. pirates of silicon valley എന്ന ചലചിത്രം തുറന്നു കാട്ടുന്ന 
കൊള്ളക്കാരുടെ ചിത്രവുമായി താരതമ്യം ചെയ്യുമ്പോള്‍ രണ്ടുകാലിലും മന്തുള്ളവന്‍ ഒരു കാലിലുള്ളവനെ 'മന്താ' 
എന്നു വിളിക്കുന്നതിലെ സുഖമില്ലായ്മ അനുഭവപ്പെടുന്നു. ആ വാക്കിന്റെ നിര്‍വചനത്തില്‍‌പ്പെടാത്ത എത്ര പേരുണ്ടെന്നതും 
അതുപയോഗിക്കുമ്പോള്‍ ആലോചിക്കണം. 'പാപം ചെയ്യാത്തവന്‍ കല്ലെറിയട്ടെ' എന്നു തീരുമാനിച്ചു നടപ്പാക്കിയാല്‍ 
അവശേഷിക്കുന്ന ജനക്കൂട്ടം നാമമാത്രമായിരിക്കും.

വില കുറഞ്ഞ മാര്‍ക്കറ്റിങ്ങ് തന്ത്രങ്ങളൂം അടവു നയങ്ങളൂം ഉപയോഗിച്ച് സ്വന്തം പ്രോഡക്ട് വാങ്ങാന്‍ നിര്‍ബന്ധിക്കുകയും,
തലതിരിഞ്ഞ കരാറുകളിലൂടെയും അവിശുദ്ധ കൂട്ടുകെട്ടുകളിലൂടെയും ഉപയോക്താവിന്റെ സ്വാതന്ത്ര്യത്തില്‍ കൈകടത്തുകയും 
ചെയ്യുന്നു എന്നതാണ് മൈക്രോസോഫ്റ്റും മറ്റനേകം കുത്തകകളും ചെയ്യുന്ന ആദ്യ criminal offense 
(ഒരു ചെറിയ കൈകടത്തലിന്റെ ഉദാഹരണം\footnote{\url{http://chithrangal.blogspot.com/2008/01/blog-post_24.html}}). 
അതാരും വലിയ പ്രശ്നമായിക്കാണുന്നില്ല (കാണാറില്ല), എല്ലാവരും പറയും അത് വെന്‍ഡറുടെ സ്വാതന്ത്ര്യം എന്ന്. 
വെന്‍ഡര്‍ എടുക്കുന്ന സ്വാതന്ത്ര്യം ഉപയോക്താവിന്റെ സ്വാതന്ത്ര്യത്തില്‍ ചവിട്ടി അലറിവിളിച്ചാണ് എന്ന് ആരും 
പ്രശ്നമായിക്കാണാറില്ല. വിപണിയില്‍ വെന്‍ഡര്‍ക്കല്ലല്ലോ കസ്റ്റമര്‍ക്കല്ലെ സ്വാതന്ത്ര്യം വേണ്ടത് എന്ന ചോദ്യം പലര്‍ക്കും 
രസിക്കാറുപോലുമില്ല.

പകരം, ഡ്രൈവറുകളും മറ്റു ഹാര്‍ഡ്‌വെയര്‍ സപ്പോര്‍ട്ട് സങ്കേതങ്ങളും നല്‍കേണ്ടുന്ന വെന്‍ഡര്‍ അതു നല്‍കുന്നില്ലെങ്കില്‍ അതിന് 
വിഘാതം യൂസര്‍ക്കു നല്‍കുന്ന സ്വാതന്ത്ര്യമാണെന്നു വരെ പറഞ്ഞു കളയും. നൂറും ഇരുനൂറും നാലായിരവും അയ്യായിരവും 
ലക്ഷങ്ങളും കൊടുത്തു വാങ്ങുന്ന ഉപകരണങ്ങള്‍ നിങ്ങള്‍ക്ക് ഇഷ്ടമുള്ള സിസ്റ്റത്തില്‍ ഉപയോഗിക്കാനുള്ള സ്വാതന്ത്ര്യം വെന്‍ഡര്‍ 
തരാത്തതെന്ത് എന്നു ചിന്തിക്കാതെ, നമുക്ക് അവശ്യമായ സ്വതന്ത്ര്യത്തെക്കുറിച്ച് ബോധവാനാവാതെ, സോഫ്റ്റ്‌വെയറില്‍ 
സാധാരണക്കാരന് എന്തിന് സ്വാതന്ത്ര്യം എന്ന് ചോദിക്കുകയും മൈക്രോസോഫ്റ്റിനും മറ്റു കുത്തകകള്‍ക്കും അവരുടെ വഴി 
എന്നു പറയുകയും ചെയ്യുന്നത് എനിക്ക് പിടികിട്ടുന്നില്ല. അതും ഇപ്പറയുന്ന സംവിധാനങ്ങള്‍ നിര്‍മ്മിക്കാന്‍ വളരെ 
എളുപ്പമുള്ള സങ്കേതങ്ങളാണെന്നു വരുമ്പോള്‍ (പലരും സ്വതന്ത്ര സോഫ്റ്റ്‌വെയറായി ഡ്രൈവറുകള്‍ തരാത്തത് അവരുപയോഗിക്കുന്ന 
എതിരാളി പേറ്റന്റ് ചെയ്ത സങ്കേതം തിരിച്ചറിയപ്പെടുമെന്നുള്ളതു കൊണ്ടാണത്രേ.). 
സ്വാതന്ത്ര്യത്തിന്റെ ആവശ്യത്തെപ്പറ്റി ചോദിച്ചാല്‍ ഈ ലേഖനത്തില്‍\footnote{\url{http://www.gnu.org/philosophy/right-to-read.html}} 
RMS വിവരിക്കുന്ന കാലം വിദൂരമല്ലെന്ന് നിയമങ്ങളുടെ ഊരാക്കുടുക്കുകള്‍ വ്യക്തമായി അറിയാവുന്നവര്‍ക്ക് മനസ്സിലാവും. 
ഇത്തരം ഒരു അവസ്ഥ സംജാതമാക്കാന്‍ മാത്രം പ്രശ്നമുള്ള കരാറുകളാണ്, നാം ഓരോ തവണയും EULA യില്‍ I Agree 
അമര്‍ത്തുമ്പോള്‍ ഒപ്പു വയ്ക്കുന്നത്. നിയമങ്ങള്‍ വ്യക്തമായി നടപ്പാക്കണം എന്ന് പറയുന്നവര്‍ എത്ര പ്രാവശ്യം ഈ കരാറുകള്‍ 
ലംഘിച്ചിട്ടുണ്ട് എന്ന് ആലോചിക്കുക. സ്വാതന്ത്ര്യത്തിന്റെ വില തിരിച്ചറിയാന്‍ അതുമതിയാവും.

"യൂസര്‍ ഫ്രന്റ്‌ലിനസ്സ്" എന്ന പദം പലപ്പോഴും ഒരു പ്രധാന പ്രശ്നമാണ്. ഉപയോഗിക്കാനുള്ള നിര്‍‌ദ്ദേശങ്ങള്‍ നല്‍കുന്ന കാര്യത്തില്‍ 
കുറച്ചു കാലം വരെ സ്വതന്ത്ര സങ്കേതങ്ങള്‍ പിന്നിലായിരുന്നു. എന്നാല്‍ ഇന്ന് ബ്ലോഗുകളും മറ്റു സ്വതന്ത്രമാധ്യമങ്ങളും ഒരു google 
തിരച്ചിലിനപ്പുറത്തേക്ക് കാര്യങ്ങള്‍ എത്തിച്ചിരിക്കുന്നു. ഉപയോക്താവിന് കുത്തകളില്‍ നിന്നോ സ്വതന്ത്ര സങ്കേതങ്ങളില്‍ നിന്നോ 
ഏതു വേണമെങ്കിലും എടുക്കാം. പക്ഷേ നമ്മള്‍ വായിക്കാതെ വിടുന്ന രണ്ടു കൂട്ടരുടെയും പരമപ്രധാനമായ അനുമതിപത്രവും 
പകര്‍പ്പകാശവും വായിച്ച് മനസ്സിലാക്കി വേണമെന്ന് മാത്രം. ഇപ്പോഴും ഉപയോക്താവിന്റെ സ്വാതന്ത്ര്യത്തെ അംഗീകരിക്കാന്‍ മടി 
കാണിക്കുന്ന വെന്‍ഡറെ അതിനു നിര്‍ബന്ധിക്കുകയാണ് നമുക്ക് ചെയ്യാവുന്ന എളുപ്പമുള്ള കാര്യം.

മൈക്രോസോഫ്റ്റിനെ എതിരിടുകയല്ല, മറിച്ച് സ്വാതന്ത്ര്യത്തിനു വേണ്ടി നിലകൊള്ളുകയാണ് Free Software Foundation 
ചെയ്യുന്നതെന്നു വ്യക്തമാക്കിയ ഒരു ലേഖനത്തിലെ കമന്റുകള്‍ മുഴുവന്‍ മൈക്രോസോഫ്റ്റിനെയും ആന്റിപൈറസി റെയ്ഡിനേയും 
ചുറ്റിപ്പറ്റിയായതില്‍ എനിക്ക് ഇപ്പോഴും അത്ഭുതമുണ്ട്. അതും പല നല്ല ചര്‍ച്ചകളിലും പങ്കെടുത്തു കണ്ട മുഖങ്ങളാവുമ്പോള്‍.


(January 24, 2008)
\newpage
