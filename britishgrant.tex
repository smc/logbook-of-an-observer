\secstar{ആവേശം അലകളുയര്‍ത്തിയ ബ്രിട്ടീഷ് ഗ്രാന്‍പ്രീ}
\vskip 2pt

2010ലെ ഫോര്‍മുല വണ്‍ ചാമ്പ്യന്‍ഷിപ്പിലെ പത്താമതു് റേസാണു് കഴിഞ്ഞ ഞായറാഴ്ച (11 ജൂലൈ) ബ്രിട്ടണിലെ സില്‍വര്‍സ്റ്റോണ്‍ പാര്‍ക്കില്‍ 
നടന്നതു്. മത്സരരംഗത്തുള്ള പന്ത്രണ്ടു ടീമുകളില്‍ ഭൂരിഭാഗത്തിന്റേയും ഹോം റേസായിരുന്നു സില്‍വര്‍സ്റ്റോണിലേതു്. ഇറ്റലിയിനിന്നുള്ള ഫെറാരിയും ടോറോ റോസോയും, സ്പെയിനിനിന്നും പ്രവര്‍ത്തിക്കുന്ന ഹിസ്പാനിക് റേസിങ് ടീമും, സ്വിസ്സര്‍ലാന്‍ഡില്‍നിന്നും പ്രവര്‍ത്തിക്കുന്ന ബിഎംഡബ്ല്യൂ സൗബറുമാണു് ഇതിനപവാദം.

ഹോം റേസായതിന്റെ വീറും വാശിയുമാണോ എന്തോ, ഈ സീസണിലെ ഏറ്റവും നല്ല റേസായിരുന്നു ബ്രിട്ടണില്‍ കണ്ടതു്.   
ഇന്ധനം നിറയ്ക്കുന്നതിനു് വിലക്കേര്‍പ്പെടുത്തിയതിനുശേഷം ട്രാക്കില്‍നിന്നും അപ്രത്യക്ഷമായിരുന്ന ശക്തമായ മത്സരങ്ങളും കനത്ത
 പോരാട്ടങ്ങളും ധാരാളമായിരുന്നു ബ്രിട്ടണിലെ ട്രാക്കില്‍. ഈ സീസണില്‍ ഇത്തരം മത്സരം കണ്ടതു് അപകടങ്ങളുടെ പരമ്പര 
 തന്നെയുണ്ടായ മോണ്ടേ കാര്‍ലോയിലും ടയറുകള്‍ ചതിച്ച കാനഡയിലും മാത്രമാണു്. എന്നാല്‍ അപകടങ്ങള്‍ വളരെ കുറവും, 
 നല്ല പ്രതലത്തില്‍ നടന്ന മത്സരവുമായിട്ടും സില്‍വര്‍സ്റ്റോണിലേതു് നല്ല ഒരു പോരാട്ടം തന്നെയായിരുന്നു.

വെള്ളിയാഴ്ച പുതിയ ഡിഫ്യൂസറൊക്കെ പരീക്ഷിച്ചു് ആത്മവിശ്വാസം കാണിച്ചെങ്കിലും വേഗത്തില്‍ വന്ന കുറവു്, വേഗംതന്നെ പഴയ 
ഡിസൈനിലേക്കു മടങ്ങാന്‍ ചാമ്പ്യന്‍ഷിപ്പില്‍ മുന്നിട്ടുനില്‍ക്കുന്ന മക്‌ലാരന്‍ തീരുമാനിച്ചിടത്തുനിന്നാണു് ബ്രിട്ടണിലെ ബഹളങ്ങള്‍ 
തുടങ്ങുന്നതു്. അതിശക്തമായ ഒരു പോരാട്ടത്തില്‍ റെഡ്ബുള്‍ പതിവുപോലെ പോള്‍ നേടി. കഴിഞ്ഞ കുറേ റേസുകളായി ദൗര്‍ഭാഗ്യം 
വേട്ടയാടിക്കൊണ്ടിരുന്ന റൊസ്ബര്‍ഗ്, റെഡ്ബുള്ളുകള്‍ക്കും അലോണ്‍സൊയുടെ ഫെറാരിയ്ക്കും ഹാമില്‍ട്ടണും പിറകില്‍ അഞ്ചാമതെത്തി. 
എന്നാല്‍ ടയറുകള്‍ ഉപയോഗിക്കുന്നതില്‍ പിഴവുപറ്റിയ കൂട്ടുകാരന്‍ ഷുമാക്കര്‍ പത്താമതേ എത്തിയുള്ളൂ. ബാരിക്കെല്ലോയുടെ വില്യംസും 
പെഡ്രോ ഡി ലാ റോസയുടെ സൗബറും വലന്‍സിയയിലെ തങ്ങളുടെ ഫോം നിലനിര്‍ത്തിക്കൊണ്ടു് ഗ്രിഡ്ഡില്‍ എട്ടും ഒന്‍പതും 
സ്ഥാനങ്ങളിലെത്തിയപ്പോള്‍ നഷ്ടമുണ്ടായതു് നമ്മുടെ സ്വന്തം ഫോഴ്സ് ഇന്ത്യക്കാണു്. യോഗ്യതാറൗണ്ടിന്റെ മൂന്നാംപാദത്തിലെത്താന്‍ 
ഇപ്രാവശ്യവും രണ്ടു ഫോഴ്സ് ഇന്ത്യകള്‍ക്കുമായില്ല. യോഗ്യതാറൗണ്ട് കഴിഞ്ഞപ്പോള്‍ മക്‌ലാരനു് ആശങ്കകള്‍ നല്‍കിക്കൊണ്ടു് നിലവിലെ 
ചാമ്പ്യന്‍ ജന്‍സണ്‍ ബട്ടണ്‍ രണ്ടാമതായിരുന്ന വില്യംസിനും സൗബറിനും പിറകില്‍ പതിനാലാമതായാണു് ഗ്രിഡ്ഡിലെത്തിയതു്. രണ്ടാമത്തെ 
ഫോഴ്സ് ഇന്ത്യയില്‍ വിറ്റാന്‍ടോണിയോ ലിയുസ്സി പതിനഞ്ചാമതായി യോഗ്യത നേടിയെങ്കിലും, നികൊ ഹള്‍ക്കന്‍ബര്‍ഗിന്റെ ഫ്ലൈയിങ് 
ലാപ്പില്‍ ഇടങ്കോലിട്ടെന്നു പറഞ്ഞു്  5 സ്ഥാനം പിഴമേടിച്ചു. ഇത്രയുമായിരുന്നു ശനിയാഴ്ചത്തെ വിശേഷങ്ങളെങ്കില്‍, ഞായറാഴ്ച ഭാഗ്യത്തിന്റെ 
കാറ്റു് ഗതിമാറിവീശിയെന്നു പറയണം.

പോളില്‍ റേസാരംഭിച്ച വെറ്റല്‍ മോശം സ്റ്റാര്‍ട്ടും, മസ്സയുടെ ഫെറാരിയുമായുള്ള ഉരസലും, അതുവഴി ഒരു പിറ്റ്സ്റ്റോപ്പും എല്ലാമായി 
ആദ്യലാപ്പുകഴിഞ്ഞപ്പോള്‍ ഇരുപത്തിമൂന്നാമതായിരുന്നു. (മസ്സയായിരുന്നു ഇരുപത്തിനാലാമതു്.) ഈ ബഹളത്തിനിടയില്‍ സ്ഥാനം 
മെച്ചപ്പെടുത്തിയെങ്കിലും മുന്‍നിരയില്‍ ഹാമില്‍ട്ടണും വെബ്ബറും കനത്ത പോരാട്ടത്തിലായിരുന്നു. റെഡ്ബുള്‍ ഡ്രൈവര്‍മാരിനിന്നും
 ചാമ്പ്യന്‍ഷിപ്പില്‍ കടുത്തമത്സരം നേരിടുന്നതിനാല്‍ ഹാമില്‍ട്ടണ്‍ വിട്ടുകൊടുക്കാന്‍ യാതൊരു പരിപാടിയുമില്ലായിരുന്നു. അവര്‍ക്കുപിന്നില്‍ 
 എതാണ്ടു് മൂന്നുമിനിട്ടു പിറകിലായി കുബിത്സയും റൊസ്ബര്‍ഗും അലോണ്‍സൊയും തമ്മിലായിരുന്നു അടുത്ത പോരാട്ടം. 
 ആദ്യപത്തുലാപ്പുകളില്‍ വിര്‍ജിനിന്റെ ലൂകാസ് ഡി ഗ്രാസ്സി റിട്ടയര്‍ചെയ്തതല്ലാതെ വേറെ പ്രത്യേകിച്ചു് വിശേഷമൊന്നുമുണ്ടായില്ല. 
 എന്നാല്‍ ടയറുകളുടെ കാര്യത്തില്‍ പിഴച്ച ഷുമാക്കര്‍ക്കു് റേസ് തന്ത്രത്തിലും പതിവില്ലാതെ പിഴയ്ക്കുന്നതു് ബ്രിട്ടനിലെ കാഴ്ചയായി. 
 പത്താംലാപ്പില്‍ ആദ്യ റഗുലര്‍ പിറ്റെടുത്ത ഷുമാക്കര്‍ പ്രതീക്ഷിച്ചതിനു വിരുദ്ധമായി മദ്ധ്യനിരയിലെ ട്രാഫിക്കിന്റെ ഒത്തനടുവില്‍ 
 തിരിച്ചെത്തി. എന്നാല്‍ തൊട്ടുപിന്നാലെ പിറ്റുചെയ്ത സൗബറിന്റെ കൊബിയാഷിയാകട്ടെ ഒന്നാന്തരമൊരു പിറ്റ് സ്റ്റോപ്പിലൂടെ 
 ഷുമാക്കറിനു മുന്നില്‍ കടക്കുകയും ചെയ്തു. ഷുമാക്കറിന്റെ 'ദൗര്‍ഭാഗ്യം' ഇവിടം കൊണ്ടവസാനിച്ചില്ല.

ആദ്യ പിറ്റില്‍ പിഴച്ച കുബിത്സയുടെയും അലോണ്‍സൊയുടെയും ചെലവില്‍ മൂന്നാമതോടിയിരുന്ന ബട്ടണ്‍ ഇരുപത്തിരണ്ടാം ലാപ്പില്‍ 
പിറ്റെടുത്തപ്പോള്‍ റൊസ്ബര്‍ഗ് ശരിക്കും പോഡിയം മണത്തു തുടങ്ങിയിരുന്നു. അതിനുമുമ്പു് അല്‍ഗ്യുസാരിയുടെ ടോറോ റോസോയിനിന്നും കടുത്ത സമ്മര്‍ദ്ദത്തിലായിരുന്ന അലോണ്‍സൊ ഒരു കോര്‍ണര്‍ ഒഴിവാക്കി കുബിത്സയെ മറികടന്നിരുന്നു. അതിനു് അതിഭയങ്കര 
വിലയാണു് മുന്‍ലോകചാമ്പ്യന്‍ കൊടുക്കേണ്ടിവന്നതു്. ഒരു ഡ്രൈവ് ത്രൂ പെനാല്‍ട്ടി കിട്ടിയെങ്കിലും അതെടുക്കാനാവുന്നതിനുമുമ്പു് 
പെഡ്രോ ഡി ലാ റോസയുടെ സൗബറിന്റെ കാറില്‍നിന്നും ഇളകിവീണ ഭാഗങ്ങള്‍ ട്രാക്കില്‍ വീണുകിടക്കുന്നതുകൊണ്ടു് സേഫ്റ്റികാര്‍ 
ട്രാക്കിലെത്തി. അതോടെ യെല്ലോ ഫ്ലാഗിനുശേഷം ഉടനെത്തന്നെ (ശരിക്കും പൊസിഷന്‍ മെച്ചപ്പെടുത്താവുന്ന അവസ്ഥയില്‍) 
ഡ്രൈവ് ത്രൂ എടുക്കണമെന്നായി അലോണ്‍സൊയുടെ അവസ്ഥ. അതു് അലോണ്‍സോയുടെ റേസിന്റെ വിധീയെഴുതിയെന്നു വേണമെങ്കില്‍ പറയാം.

സേഫ്റ്റികാര്‍ മാറിയ ഉടനെയുണ്ടായ ഒരു കൂട്ടപ്പൊരിച്ചിലും, രണ്ടു് ലോട്ടസുകളും കൂടി ബാരിക്കെല്ലോയെയും കൊബിയാഷിയെയും മറച്ചതും
 മുതലാക്കി ബട്ടണ്‍ നാലാംസ്ഥാനം പിടിച്ചെടുത്തു. വേഗത്തിന്റെ കാര്യത്തില്‍ ബട്ടന്റെ മക്‌ലാരന്റെയൊപ്പമെത്തില്ലെങ്കിലും ട്രാക്കിലുള്ള 
 മുന്‍തൂക്കം അവസാനംവരെ കാത്തുസൂക്ഷിക്കാന്‍ റൊസ്ബര്‍ഗിനായി.

എന്നാല്‍ ഈ സമയംകൊണ്ടു് തന്റെ വേഗത്തിനുമുമ്പില്‍ ഒന്നുമല്ലാതിരുന്ന പിന്‍നിര കാറുകളെയെല്ലാം തട്ടിമാറ്റി വെറ്റല്‍ 
മദ്ധ്യനിരയിലെത്തിയിരുന്നു. ഈ സമയം ഏഴാംസ്ഥാനത്തു് ബാരിക്കെല്ലോയുടെ വില്യംസിന്റെയും കൊബിയാഷിയുടെ സൗബറിന്റേയും 
പുറകില്‍ ഓടിക്കൊണ്ടിരുന്ന ഷൂമാക്കറിനു് വീണ്ടും കഷ്ടകാലം തുടങ്ങി. ആദ്യം അഡ്രിയാന്‍ സുട്ടിലിന്റെ ഫോഴ്സിന്ത്യയുടെ ആക്രമണത്തില്‍ 
വേഗംതന്നെ നിലംപരിശായ ഷുമാക്കര്‍ തന്റെ പഴയ പ്രതാപമെല്ലാംപോയ ഒരു കാരണവരുടെ അവസ്ഥയിലായപ്പോഴാണു് മറ്റൊരു
 യുവജര്‍മനില്‍നിന്നു് ശക്തമായ ആക്രമണത്തിലാവുന്നതു്. എന്നാല്‍ വെറ്റലിനോടും വേഗംതന്നെ ഷുമാക്കര്‍ കീഴടങ്ങി. അഡ്രയാന്‍ 
 സുട്ടില്‍ ഏതാണ്ടു് അവസാനംവരെ വെറ്റലിനെ തന്റെ പിന്നില്‍ തളച്ചിട്ട വിരുതുകൂടി കാണുമ്പോഴാണു് ഷുമാക്കര്‍ എത്ര വെല്ലുവിളി 
 ട്രാക്കില്‍ ഉയര്‍ത്തുന്നുവെന്നു നമ്മള്‍ സംശയിക്കുന്നത്. അവസാനലാപ്പുകളില്‍ ഏഴ്, എട്ടു് സ്ഥാനങ്ങള്‍ക്കുവേണ്ടി ശക്തമായ 
 മത്സരമായിരുന്നു ട്രാക്കില്‍ നടന്നതു്. നാലു ജര്‍മ്മന്‍ ഡ്രൈവര്‍മാര്‍ (വെറ്റല്‍, സുട്ടില്‍, ഷുമാക്കര്‍, ഹള്‍ക്കൈന്‍ബര്‍ഗ്) തങ്ങളുടെ ഇംഗ്ലീഷ് 
 ടീമുകള്‍ക്കുവേണ്ടി ട്രാക്കില്‍ ഏറ്റുമുട്ടുന്ന കാഴ്ച ശരിക്കും ഒരു വിരുന്നുതന്നെയായിരുന്നു.

കാര്യങ്ങളിങ്ങനെയൊക്കെയാണെങ്കിലും ചാമ്പ്യഷിപ്പ് പോരാട്ടങ്ങള്‍ മക്‌ലാരനിലേക്കും (279) റെഡ്ബുളളിലേക്കും (249) ഒതുങ്ങുന്ന 
കാഴ്ചയാണു കാണുന്നതു്. 145 പോയിന്റുമായി ലൂയിസ് ഹാമില്‍ട്ടണാണു് മുന്നില്‍. 133 പോയിന്റുമായി നിലവിലെ ചാമ്പ്യനും സഹ മക്‌ലാരന്‍ 
ഡ്രൈവറുമായ ബട്ടണ്‍ രണ്ടാമതാണു്. മൂന്നാമതു് റെഡ്ബുള്ളിന്റെ മാര്‍ക് വെബ്ബറും (128) നാലാമതു് (121) രണ്ടാമത്തെ റെഡ്ബുള്‍ 
ഡ്രൈവര്‍ വെറ്റലുമാണു്. പക്ഷെ, കഴിഞ്ഞ കുറെ റേസുകളായി തുടര്‍ന്നുവരുന്ന മദ്ധ്യനിരയിലെ പോരാട്ടം ശക്തമായിക്കൊണ്ടിരിക്കുകയാണു്.
 വരുംയൂറോപ്യന്‍ റേസുകളില്‍ അതു് ശക്തമാകുമെന്നു് നമുക്കു് കൃത്യമായി പ്രതീക്ഷിക്കാം. എന്തായാലൂം മെഴ്സിഡസ് അവരുടെ 
 അടുത്തകൊല്ലത്തെ കാറിനെക്കുറിച്ചു് ഇപ്പോള്‍ത്തന്നെ ആലോചിച്ചു തുടങ്ങിയിരിക്കുമെന്നു മാത്രം ഊഹിക്കാം. :)

വില്യംസിന്റെയും സൗബറിന്റേയും കാറുകള്‍ മദ്ധ്യനിരയിലെ തിരക്കേറ്റിയപ്പോള്‍ വലഞ്ഞതു് മെഴ്സിഡസും റെനോയും മാത്രമല്ല. 
സീസണിലെ തുടക്കംമുതലേ അംഗീകൃത മദ്ധ്യനിരടീമുകളെന്ന പദവിക്കുവേണ്ടി പോരാടുന്ന ഫോഴ്സ് ഇന്ത്യയും ടോറോ റോസോയുമാണു്. 
എന്തായാലും, കോസ്‌വര്‍ത്തു് എന്‍ജിനുകളല്ല മറ്റു ടീമുകളെയൊന്നും പിന്നോട്ടടിപ്പിക്കുന്നതെന്നു് വില്യംസിന്റെ കഴിഞ്ഞ രണ്ടു റേസിലെ 
പ്രകടനത്തില്‍നിന്നും ശരിക്കും വ്യക്തമായി. അതുപോലെ ഫെറാരിയുടെ എന്‍ജിന്‍ സൗബറിനു ഇപ്പോഴും പ്രശ്നങ്ങളുണ്ടാക്കുന്നതു് 
എന്‍ജിന്‍ സപ്ലയറെന്ന നിലയില്‍ ഫെറാരിയ്ക്കൊരു നല്ല വാര്‍ത്തയല്ല. പ്രത്യേകിച്ചും മെഴ്സിഡസ് ശക്തമായ വെല്ലുവിളിയുയര്‍ത്തുമ്പോള്‍.

എന്തായാലും അടുത്താഴ്ചനടക്കുന്ന ജര്‍മന്‍ ഗ്രാന്‍പ്രീ ഒരുപാടു മുന്‍നിര ഡ്രൈവര്‍മാര്‍ക്കു് ഹോം റേസാണു്. ഈ സീസണിലിന്നുവരെ 
ഒരു ഡ്രൈവറും ഹോം റേസില്‍ ഒന്നാമനായിട്ടില്ല. കളം നിറഞ്ഞുനില്‍ക്കുന്ന ജര്‍മന്‍ ഡ്രൈവര്‍മാര്‍ പതിവിനു വ്യത്യാസം വരുത്തുമോ 
എന്നു കണ്ടറിയാം.

\hspace*{2em}(Jul 15, 2010)\footnote{http://malayal.am/വിനോദം/കായികം/6772/ആവേശം-അലകളുയര്‍ത്തിയ-ബ്രിട്ടീഷ്-ഗ്രാന്‍പ്രീ}
\newpage
