\secstar{ആവേശം അലകളുയര്‍ത്തിയ ബ്രിട്ടീഷ് ഗ്രാന്‍പ്രീ}
\vskip 2pt

2010­ലെ ­ഫോര്‍­മുല വണ്‍ ചാ­മ്പ്യന്‍­ഷി­പ്പി­ലെ പത്താ­മ­ത് റേ­സാ­ണ് കഴി­ഞ്ഞ ഞാ­യ­റാ­ഴ്ച(11 ജൂ­ലൈ) ബ്രി­ട്ട­ണി­ലെ സില്‍­വര്‍­സ്റ്റോണ്‍­പാര്‍­ക്കില്‍ 
നട­ന്ന­ത്. മത്സ­ര­രം­ഗ­ത്തു­ള്ള പന്ത്ര­ണ്ടു ടീ­മു­ക­ളില്‍ ഭൂ­രി­ഭാ­ഗ­ത്തി­ന്റേ­യും ഹോം റേ­സാ­യി­രു­ന്നു സില്‍­വര്‍­സ്റ്റോ­ണി­ലേ­ത്. ഇറ്റ­ലി­യില്‍ 
നി­ന്നു­ള്ള ഫെ­റാ­രി­യും ടോ­റോ റോ­സോ­യും, സ്പെ­യി­നില്‍ നി­ന്നും പ്ര­വര്‍­ത്തി­ക്കു­ന്ന ഹി­സ്പാ­നി­ക് റേ­സി­ങ് ടീ­മും, സ്വി­സ്സര്‍­ലാന്‍­ഡില്‍ 
നി­ന്നും പ്ര­വര്‍­ത്തി­ക്കു­ന്ന ­ബി­എം­ഡ­ബ്ല്യൂ­ സൌ­ബ­റു­മാ­ണ് ഇതി­ന­പ­വാ­ദം.

­ഹോം റേ­സാ­യ­തി­ന്റെ വീ­റും വാ­ശി­യു­മാ­ണോ എന്തോ, ഈ സീ­സ­ണി­ലെ ഏറ്റ­വും നല്ല റേ­സാ­യി­രു­ന്നു ബ്രി­ട്ട­ണില്‍ കണ്ട­ത്.  
ഇന്ധ­നം നി­റ­യ്ക്കു­ന്ന­തി­ന് വി­ല­ക്കേര്‍­പ്പെ­ടു­ത്തി­യ­തി­നു ശേ­ഷം ട്രാ­ക്കില്‍ നി­ന്നും അപ്ര­ത്യ­ക്ഷ­മാ­യി­രു­ന്ന ശക്ത­മായ മത്സ­ര­ങ്ങ­ളും കന­ത്ത
 പോ­രാ­ട്ട­ങ്ങ­ളും ധാ­രാ­ള­മാ­യി­രു­ന്നു ബ്രി­ട്ട­ണി­ലെ ട്രാ­ക്കില്‍. ഈ സീ­സ­ണില്‍ ഇത്ത­രം മത്സ­രം കണ്ട­ത് അപ­ക­ട­ങ്ങ­ളു­ടെ പര­മ്പര 
 തന്നെ­യു­ണ്ടായ മോ­ണ്ടേ കാര്‍­ലോ­യി­ലും ടയ­റു­കള്‍ ചതി­ച്ച കാ­ന­ഡ­യി­ലും മാ­ത്ര­മാ­ണ്. എന്നാല്‍ അപ­ക­ട­ങ്ങള്‍ വള­രെ കു­റ­വും, 
 നല്ല പ്ര­ത­ല­ത്തില്‍ നട­ന്ന മത്സ­ര­വും ആയി­ട്ടും സില്‍­വര്‍­സ്റ്റോ­ണി­ലേ­ത് നല്ല ഒരു പോ­രാ­ട്ടം തന്നെ­യാ­യി­രു­ന്നു­.

­വെ­ള്ളി­യാ­ഴ്ച പു­തിയ ഡി­ഫ്യൂ­സര്‍ ഒക്കെ പരീ­ക്ഷി­ച്ച് ആത്മ­വി­ശ്വാ­സം കാ­ണി­ച്ചെ­ങ്കി­ലും വേ­ഗ­ത്തില്‍ വന്ന കു­റ­വ്, വേ­ഗം തന്നെ പഴയ 
ഡി­സൈ­നി­ലേ­ക്കു മട­ങ്ങാന്‍ ചാ­മ്പ്യന്‍­ഷി­പ്പില്‍ മു­ന്നി­ട്ടു­നില്‍­ക്കു­ന്ന മക്‌­ലാ­രന്‍ തീ­രു­മാ­നി­ച്ചി­ട­ത്തു­നി­ന്നാ­ണ് ബ്രി­ട്ട­ണി­ലെ ബഹ­ള­ങ്ങള്‍ 
തു­ട­ങ്ങു­ന്ന­ത്. അതി­ശ­ക്ത­മായ ഒരു പോ­രാ­ട്ട­ത്തില്‍ ­റെ­ഡ്ബുള്‍ പതി­വു­പോ­ലെ പോള്‍ നേ­ടി. കഴി­ഞ്ഞ കു­റേ റേ­സു­ക­ളാ­യി ദൌര്‍­ഭാ­ഗ്യം 
വേ­ട്ട­യാ­ടി­ക്കൊ­ണ്ടി­രു­ന്ന റൊ­സ്ബര്‍­ഗ്, റെ­ഡ്ബു­ളു­കള്‍­ക്കും അലോണ്‍­സൊ­യു­ടെ ഫെ­റാ­രി­യ്ക്കും ഹാ­മില്‍­ട്ട­ണും പി­റ­കില്‍ അഞ്ചാ­മ­തെ­ത്തി. 
എന്നാല്‍ ടയ­റു­കള്‍ ഉപ­യോ­ഗി­ക്കു­ന്ന­തില്‍ പി­ഴ­വു പറ്റിയ കൂ­ട്ടു­കാ­രന്‍ ഷു­മാ­ക്കര്‍ പത്താ­മ­തെ എത്തി­യു­ള്ളൂ. ബാ­രി­ക്കെ­ല്ലോ­യു­ടെ വി­ല്യം­സും 
പെ­ഡ്രോ ഡി ലാ റോ­സ­യു­ടെ സൌ­ബ­റും വലന്‍­സി­യ­യി­ലെ തങ്ങ­ളു­ടെ ഫോം നി­ല­നിര്‍­ത്തി­ക്കൊ­ണ്ട് ഗ്രി­ഡ്ഡില്‍ എട്ടും ഒന്‍­പ­തും 
സ്ഥാ­ന­ങ്ങ­ളി­ലെ­ത്തി­യ­പ്പോള്‍ നഷ്ട­മു­ണ്ടാ­യ­ത് നമ്മു­ടെ സ്വ­ന്തം ഫോ­ഴ്സ് ഇന്ത്യ­ക്കാ­ണ്. യോ­ഗ്യ­താ റൌ­ണ്ടി­ന്റെ മൂ­ന്നാം പാ­ദ­ത്തി­ലെ­ത്താന്‍ 
ഇപ്രാ­വ­ശ്യ­വും രണ്ടു ഫോ­ഴ്സ് ഇന്ത്യ­കള്‍­ക്കു­മാ­യി­ല്ല. യോ­ഗ്യ­താ റൌ­ണ്ട് കഴി­ഞ്ഞ­പ്പോള്‍ മക്‌­ലാ­ര­ന് ആശ­ങ്ക­കള്‍ നല്‍­കി­ക്കൊ­ണ്ട് നി­ല­വി­ലെ 
ചാ­മ്പ്യന്‍ ­ജന്‍­സണ്‍ ബട്ടണ്‍ രണ്ടാ­മ­ത്തെ വി­ല്യം­സി­നും സൌ­ബ­റി­നും പി­റ­കില്‍ പതി­നാ­ലാ­മ­താ­യാ­ണ് ഗ്രി­ഡ്ഡി­ലെ­ത്തി­യ­ത്. രണ്ടാ­മ­ത്തെ 
ഫോ­ഴ്സ് ഇന്ത്യ­യില്‍ ­വി­റ്റാന്‍­ടോ­ണി­യോ ലി­യു­സ്സി­ പതി­ന­ഞ്ചാ­മ­താ­യി യോ­ഗ്യത നേ­ടി­യെ­ങ്കി­ലും നി­കൊ ഹള്‍­ക്കന്‍­ബര്‍­ഗി­ന്റെ ഫ്ലൈ­യി­ങ് 
ലാ­പ്പില്‍ ഇട­ങ്കോ­ലി­ട്ടെ­ന്നു പറ­ഞ്ഞ് 5 സ്ഥാ­നം പി­ഴ­മേ­ടി­ച്ചു. ഇത്ര­യു­മാ­യി­രു­ന്നു ശനി­യാ­ഴ്ച­ത്തെ വി­ശേ­ഷ­ങ്ങ­ളെ­ങ്കില്‍, ഞാ­യ­റാ­ഴ്ച ഭാ­ഗ്യ­ത്തി­ന്റെ 
കാ­റ്റ് ഗതി­മാ­റി വീ­ശി­യെ­ന്നു പറ­യ­ണം­.

­പോ­ളില്‍ റേ­സാ­രം­ഭി­ച്ച വെ­റ്റല്‍ മോ­ശം സ്റ്റാര്‍­ട്ടൂം മസ്സ­യു­ടെ ഫെ­റാ­രി­യു­മാ­യു­ള്ള ഉര­സ­ലും അതു­വ­ഴി­ഒ­രു പി­റ്റ്സ്റ്റോ­പ്പും എല്ലാ­മാ­യി 
ആദ്യ­ലാ­പ്പു­ക­ഴി­ഞ്ഞ­പ്പോള്‍ ഇരു­പ­ത്തി­മൂ­ന്നാ­മ­താ­യി­രു­ന്നു­(­മ­സ്സ­യാ­യി­രു­ന്നു ഇരു­പ­ത്തി­നാ­ലാ­മ­ത്). ഈ ബഹ­ള­ത്തി­നി­ട­യില്‍ സ്ഥാ­നം 
മെ­ച്ച­പ്പെ­ടു­ത്തി­യെ­ങ്കി­ലും മുന്‍­നി­ര­യില്‍ ഹാ­മില്‍­ട്ട­ണും വെ­ബ്ബ­റും കന­ത്ത പോ­രാ­ട്ട­ത്തി­ലാ­യി­രു­ന്നു. റെ­ഡ്ബുള്‍ ഡ്രൈ­വര്‍­മാ­രില്‍ നി­ന്നും
 ചാ­മ്പ്യന്‍­ഷി­പ്പില്‍ കടു­ത്ത മത്സ­രം നേ­രി­ടു­ന്ന­തി­നാല്‍ ഹാ­മില്‍­ട്ടണ്‍ വി­ട്ടു­കൊ­ടു­ക്കാന്‍ യാ­തൊ­രു പരി­പാ­ടി­യു­മി­ല്ലാ­യി­രു­ന്നു. അവര്‍­ക്കു­പി­ന്നില്‍ 
 എതാ­ണ്ട് മൂ­ന്നു­മി­നി­ട്ടു പി­റ­കി­ലാ­യി കു­ബി­ത്സ­യും, റൊ­സ്ബര്‍­ഗും അലോണ്‍­സൊ­യും തമ്മി­ലാ­യി­രു­ന്നു അടു­ത്ത പോ­രാ­ട്ടം. 
 ആദ്യ­പ­ത്തു­ലാ­പ്പു­ക­ളില്‍ വിര്‍­ജി­നി­ന്റെ ­ലൂ­കാ­സ് ഡി ഗ്രാ­സ്സി­ റി­ട്ട­യര്‍ ചെ­യ്ത­ത­ല്ലാ­തെ വേ­റെ പ്ര­ത്യേ­കി­ച്ച് വി­ശേ­ഷ­മൊ­ന്നു­മു­ണ്ടാ­യി­ല്ല. 
 എന്നാല്‍ ടയ­റു­ക­ളു­ടെ കാ­ര്യ­ത്തില്‍ പി­ഴ­ച്ച ഷു­മാ­ക്കര്‍­ക്ക് റേ­സ് തന്ത്ര­ത്തി­ലും പതി­വി­ല്ലാ­തെ പി­ഴ­യ്ക്കു­ന്ന­ത് ബ്രി­ട്ട­നി­ലെ കാ­ഴ്ച­യാ­യി. 
 പത്താം ലാ­പ്പില്‍ ആദ്യ റഗു­ലര്‍ പി­റ്റെ­ടു­ത്ത ഷു­മാ­ക്കര്‍ പ്ര­തീ­ക്ഷി­ച്ച­തി­നു വി­രു­ദ്ധ­മാ­യി മദ്ധ്യ­നി­ര­യി­ലെ ട്രാ­ഫി­ക്കി­ന്റെ ഒത്ത നടു­വില്‍ 
 തി­രി­ച്ചെ­ത്തി. എന്നാല്‍ തൊ­ട്ടു പി­ന്നാ­ലെ പി­റ്റു ചെ­യ്ത സൌ­ബ­റി­ന്റെ കൊ­ബി­യാ­ഷി­യാ­ക­ട്ടെ ഒന്നാ­ന്ത­ര­മൊ­രു പി­റ്റ് സ്റ്റോ­പ്പി­ലൂ­ടെ 
 ഷു­മാ­ക്ക­റി­നു മു­ന്നില്‍ കട­ക്കു­ക­യും ചെ­യ്തു. ഷു­മാ­ക്ക­റി­ന്റെ 'ദൌര്‍­ഭാ­ഗ്യം' ഇവി­ടം കൊ­ണ്ട­വ­സാ­നി­ച്ചി­ല്ല.

ആ­ദ്യ പി­റ്റില്‍ പി­ഴ­ച്ച കു­ബി­ത്സ­യു­ടെ­യും അലോണ്‍­സൊ­യു­ടെ­യും ചെ­ല­വില്‍ മൂ­ന്നാ­മ­തോ­ടി­യി­രു­ന്ന ബട്ടണ്‍ ഇരു­പ­ത്തി­ര­ണ്ടാം ലാ­പ്പില്‍ 
പി­റ്റെ­ടു­ത്ത­പ്പോള്‍ ­റൊ­സ്ബര്‍­ഗ് ശരി­ക്കും പോ­ഡി­യം മണ­ത്തു തു­ട­ങ്ങി­യി­രു­ന്നു. അതി­നു മു­മ്പ് അല്‍­ഗ്യു­സാ­രി­യു­ടെ ടോ­റോ റോ­സോ­യില്‍ 
നി­ന്നും കടു­ത്ത സമ്മര്‍­ദ്ദ­ത്തി­ലാ­യി­രു­ന്ന അലോണ്‍­സൊ ഒരു കോര്‍­ണര്‍ ഒഴി­വാ­ക്കി കു­ബി­ത്സ­യെ മറി­ക­ട­ന്നി­രു­ന്നു. അതി­ന് അതി­ഭ­യ­ങ്കര 
വി­ല­യാ­ണ് മുന്‍ ലോ­ക­ചാ­മ്പ്യന്‍ കൊ­ടു­ക്കേ­ണ്ടി­വ­ന്ന­ത്. ഒരു ഡ്രൈ­വ് ത്രൂ പെ­നാല്‍­ട്ടി കി­ട്ടി­യെ­ങ്കി­ലും അതെ­ടു­ക്കാ­നാ­വു­ന്ന­തി­നു മു­മ്പ് 
പെ­ഡ്രോ ഡി ലാ റോ­സ­യു­ടെ സൌ­ബ­റി­ന്റെ കാ­റില്‍ നി­ന്നും ഇള­കി വീണ ഭാ­ഗ­ങ്ങള്‍ ട്രാ­ക്കില്‍ വീ­ണു കി­ട­ക്കു­ന്ന­തു കൊ­ണ്ട് സേ­ഫ്റ്റി­കാര്‍ 
ട്രാ­ക്കി­ലെ­ത്തി. അതോ­ടെ യെ­ല്ലോ ഫ്ലാ­ഗി­നു ശേ­ഷം ഉട­നെ­ത്ത­ന്നെ (ശ­രി­ക്കും പൊ­സി­ഷന്‍ മെ­ച്ച­പ്പെ­ടു­ത്താ­വു­ന്ന അവ­സ്ഥ­യില്‍) 
ഡ്രൈ­വ് ത്രൂ എടു­ക്ക­ണ­മെ­ന്നാ­യി അലോണ്‍­സൊ­യു­ടെ അവ­സ്ഥ. അത് അലോണ്‍­സോ­യു­ടെ റേ­സി­ന്റെ വി­ധീ­യെ­ഴു­തി­യെ­ന്നു വേ­ണ­മെ­ങ്കില്‍ പറ­യാം­.

­സേ­ഫ്റ്റി­കാര്‍ മാ­റിയ ഉട­നെ­യു­ണ്ടായ ഒരു കൂ­ട്ട­പ്പൊ­രി­ച്ചി­ലും, രണ്ട് ലോ­ട്ട­സു­ക­ളും കൂ­ടി ബാ­രി­ക്കെ­ല്ലോ­യെ­യും കൊ­ബി­യാ­ഷി­യേ­യും മറ­ച്ച­തും
 മു­ത­ലാ­ക്കി ബട്ടണ്‍ നാ­ലാം സ്ഥാ­നം പി­ടി­ച്ചെ­ടു­ത്തു. വേ­ഗ­ത്തി­ന്റെ കാ­ര്യ­ത്തില്‍ ബട്ട­ന്റെ മക്‌­ലാ­ര­ന്റെ­യൊ­പ്പ­മെ­ത്തി­ലെ­ങ്കി­ലും ട്രാ­ക്കി­ലു­ള്ള 
 മുന്‍­തൂ­ക്കം അവ­സാ­നം വരെ കാ­ത്തു സൂ­ക്ഷി­ക്കാന്‍ റൊ­സ്ബര്‍­ഗി­നാ­യി­.

എ­ന്നാല്‍ ഈ സമ­യം കൊ­ണ്ട് തന്റെ വേ­ഗ­ത്തി­നു­മു­മ്പില്‍ ഒന്നു­മ­ല്ലാ­തി­രു­ന്ന പിന്‍­നിര കാ­റു­ക­ളെ­യെ­ല്ലാം തട്ടി­മാ­റ്റി വെ­റ്റല്‍ 
മദ്ധ്യ­നി­ര­യി­ലെ­ത്തി­യി­രു­ന്നു. ഈ സമ­യം ഏഴാം സ്ഥാ­ന­ത്ത് ബാ­രി­ക്കെ­ല്ലോ­യു­ടെ വി­ല്യം­സി­ന്റെ­യും കൊ­ബി­യാ­ഷി­യു­ടെ സൌ­ബ­റി­ന്റേ­യും 
പു­റ­കില്‍ ഓടി­ക്കൊ­ണ്ടി­രു­ന്ന ഷൂ­മാ­ക്ക­റി­ന് വീ­ണ്ടും കഷ്ട­കാ­ലം തു­ട­ങ്ങി. ആദ്യം അഡ്രി­യാന്‍ സു­ട്ടി­ലി­ന്റെ ഫോ­ഴ്സി­ന്ത്യ­യു­ടെ ആക്ര­മ­ണ­ത്തില്‍ 
വേ­ഗം തന്നെ നി­ലം പരി­ശായ ഷു­മാ­ക്കര്‍ തന്റെ പഴ­യ­പ്ര­താ­പ­മെ­ല്ലാം പോയ ഒരു കാ­ര­ണ­വ­രു­ടെ അവ­സ്ഥ­യി­ലാ­യ­പ്പോ­ഴാ­ണ് മറ്റൊ­രു
 യുവ ജര്‍­മ­നില്‍ നി­ന്ന് ശക്ത­മായ ആക്ര­മ­ണ­ത്തി­ലാ­വു­ന്ന­ത്. എന്നാല്‍ വെ­റ്റ­ലി­നോ­ടും വേ­ഗം തന്നെ ഷു­മാ­ക്കര്‍ കീ­ഴ­ട­ങ്ങി. അഡ്ര­യാന്‍ 
 സു­ട്ടില്‍ ഏതാ­ണ്ട് അവ­സാ­നം വരെ വെ­റ്റ­ലി­നെ തന്റെ പി­ന്നില്‍ തള­ച്ചി­ട്ട വി­രു­തു കൂ­ടി കാ­ണു­മ്പോ­ഴാ­ണ് ഷു­മാ­ക്കര്‍ എത്ര വെ­ല്ലു­വി­ളി 
 ട്രാ­ക്കില്‍ ഉയര്‍­ത്തു­ന്നു­വെ­ന്നു നമ്മള്‍ സം­ശ­യി­ക്കു­ന്ന­ത്. അവ­സാ­ന­ലാ­പ്പു­ക­ളില്‍ ഏഴ്,എ­ട്ട് സ്ഥാ­ന­ങ്ങള്‍­ക്കു വേ­ണ്ടി ശക്ത­മായ 
 മത്സ­ര­മാ­യി­രു­ന്നു ട്രാ­ക്കില്‍ നട­ന്ന­ത്. നാ­ലു ജര്‍­മ്മന്‍ ഡ്രൈ­വര്‍­മാര്‍ (വെ­റ്റല്‍,­സു­ട്ടില്‍,­ഷു­മാ­ക്കര്‍,­ഹള്‍­ക്കൈന്‍­ബര്‍­ഗ്) തങ്ങ­ളു­ടെ ഇം­ഗ്ലീ­ഷ് 
 ടീ­മു­കള്‍­ക്കു വേ­ണ്ടി ട്രാ­ക്കില്‍ ഏറ്റു­മു­ട്ടു­ന്ന കാ­ഴ്ച ശരി­ക്കും ഒരു വി­രു­ന്നു തന്നെ­യാ­യി­രു­ന്നു­.

­കാ­ര്യ­ങ്ങ­ളി­ങ്ങ­നെ­യൊ­ക്കെ­യാ­ണെ­ങ്കി­ലും ചാ­മ്പ്യ­ഷി­പ്പ് പോ­രാ­ട്ട­ങ്ങള്‍ മക്‌­ലാ­ര­നി­ലേ­ക്കും­(279), റെ­ഡ്ബു­ള­ളി­ലേ­ക്കും­(249) ഒതു­ങ്ങു­ന്ന 
കാ­ഴ്ച­യാ­ണു കാ­ണു­ന്ന­ത്. 145 പോ­യി­ന്റു­മാ­യി ലൂ­യി­സ് ഹാ­മില്‍­ട്ട­ണാ­ണു­മു­ന്നില്‍, 133 പോ­യി­ന്റു­മാ­യി നി­ല­വി­ലെ ചാ­മ്പ്യ­നും സഹ­മ­ക്‌­ലാ­രന്‍ 
ഡ്രൈ­വ­റു­മായ ബട്ടണ്‍ രണ്ടാ­മ­താ­ണ്. മൂ­ന്നാ­മ­ത് റെ­ഡ്ബു­ള്ളി­ന്റെ മാര്‍­ക് വെ­ബ്ബ­റും­(128) നാ­ലാ­മ­ത്(121) രണ്ടാ­മ­ത്തെ റെ­ഡ്ബുള്‍ 
ഡ്രൈ­വര്‍ വെ­റ്റ­ലു­മാ­ണ്. പക്ഷെ, കഴി­ഞ്ഞ കു­റെ റേ­സു­ക­ളാ­യി തു­ടര്‍­ന്നു വരു­ന്ന മദ്ധ്യ­നി­ര­യി­ലെ പോ­രാ­ട്ടം ശക്ത­മാ­യി­ക്കൊ­ണ്ടി­രി­ക്കു­ക­യാ­ണ്.
 വരും യൂ­റോ­പ്യന്‍ റേ­സു­ക­ളില്‍ അത് ശക്ത­മാ­കു­മെ­ന്ന് നമു­ക്ക് കൃ­ത്യ­മാ­യൂ­ഹി­ക്കാം. എന്താ­യാ­ലൂം ­മെ­ഴ്സി­ഡ­സ് അവ­രു­ടെ 
 അടു­ത്ത­കൊ­ല്ല­ത്തെ കാ­റി­നെ­ക്കു­റി­ച്ച് ഇപ്പോള്‍­ത്ത­ന്നെ ആലോ­ചി­ച്ചു തു­ട­ങ്ങി­യി­രി­ക്കു­മെ­ന്നു മാ­ത്രം ഊഹി­ക്കാം. :)

­വി­ല്യം­സി­ന്റെ­യും സൌ­ബ­റി­ന്റേ­യും കാ­റു­കള്‍ മദ്ധ്യ­നി­ര­യി­ലെ തി­ര­ക്കേ­റ്റി­യ­പ്പോള്‍ വല­ഞ്ഞ­ത് മെ­ഴ്സി­ഡ­സും റെ­നോ­യും മാ­ത്ര­മ­ല്ല. 
സീ­സ­ണി­ലെ തു­ട­ക്കം­മു­ത­ലേ അം­ഗീ­കൃത മദ്ധ്യ­നിര ടീ­മു­ക­ളെ­ന്ന പദ­വി­ക്കു വേ­ണ്ടി പോ­രാ­ടു­ന്ന ഫോ­ഴ്സ് ഇന്ത്യ­യും ടോ­റോ റോ­സോ­യു­മാ­ണ്. 
എന്താ­യാ­ലൂം, കോ­സ്‌­വര്‍­ത്ത് എന്‍­ജി­നു­ക­ള­ല്ല മറ്റു­ടീ­മു­ക­ളെ­യെ­ാ­ന്നും പി­ന്നോ­ട്ട­ടി­പ്പി­ക്കു­ന്ന­തെ­ന്ന് വി­ല്യം­സി­ന്റെ കഴി­ഞ്ഞ രണ്ടു റേ­സി­ലെ 
പ്ര­ക­ട­ന­ത്തില്‍ നി­ന്നും ശരി­ക്കും വ്യ­ക്ത­മാ­യി. അതു­പോ­ലെ ഫെ­റാ­രി­യു­ടെ എന്‍­ജിന്‍ സൌ­ബ­റി­നു ഇപ്പോ­ഴും പ്ര­ശ്ന­ങ്ങ­ളു­ണ്ടാ­ക്കു­ന്ന­ത് 
എന്‍­ജിന്‍ സപ്ല­യ­റെ­ന്ന നി­ല­യില്‍ ഫെ­റാ­രി­യ്ക്കൊ­രു നല്ല വാര്‍­ത്ത­യ­ല്ല. പ്ര­ത്യേ­കി­ച്ചും മെ­ഴ്സി­ഡ­സ് ശക്ത­മായ വെ­ല്ലു­വി­ളി­യു­യര്‍­ത്തു­മ്പോള്‍.

എ­ന്താ­യാ­ലും അടു­ത്താ­ഴ്ച­ന­ട­ക്കു­ന്ന ജര്‍­മന്‍ ഗ്രാന്‍­പ്രീ ഒരു­പാ­ടു മുന്‍­നിര ഡ്രൈ­വര്‍­മാര്‍­ക്ക് ഹോം റേ­സാ­ണ്. ഈ സീ­സ­ണി­ലി­ന്നു­വ­രെ 
ഒരു ഡ്രൈ­വ­റും ഹോം റേ­സില്‍ ഒന്നാ­മ­നാ­യി­ട്ടി­ല്ല. കളം നി­റ­ഞ്ഞു നില്‍­ക്കു­ന്ന ജര്‍­മന്‍ ഡ്രൈ­വര്‍­മാര്‍ പതി­വി­നു വ്യ­ത്യാ­സം വരു­ത്തു­മോ 
എന്നു കണ്ട­റി­യാം­.

(Jul 15, 2010)\footnote{http://malayal.am/വിനോദം/കായികം/6772/ആവേശം-അലകളുയര്‍ത്തിയ-ബ്രിട്ടീഷ്-ഗ്രാന്‍പ്രീ}
\newpage
