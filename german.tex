\secstar{ജര്‍മന്‍ ഗ്രാന്‍പ്രീയില്‍ ഫെറാരിയുടെ തിരിച്ചുവരവു്}
\vskip 2pt

25 ജൂലൈ ഞായറാഴ്ച ജര്‍മനിയിലെ ഹോക്കന്‍ഹൈമിലെ സര്‍ക്യൂട്ടില്‍ നടന്ന പതിനൊന്നാംപാദം ഫെറാരിയുടെ 
തിരിച്ചുവരവുകൊണ്ടാണു് ശ്രദ്ധേയമായതു്. സീസണിലെ ആദ്യറേസായിരുന്ന ബഹ്റൈന്‍ ഗ്രാന്‍പ്രീക്കുശേഷം 
ജര്‍മനിയിലാണു് വമ്പന്‍മാരായ ഫെറാരി ഒരു 1-2 പോഡിയം ഫിനിഷ് കരസ്ഥമാക്കിയതു്. ഫെറാരിയുടെ 
വിജയത്തിളക്കത്തില്‍ ചെറിയ കരിനിഴല്‍ വീഴ്‌ത്തിയതു്, അലോണ്‍സൊക്കു് ഒന്നാംസ്ഥാനം കിട്ടാന്‍വേണ്ടി ഫെലിപെ 
മസ്സ വഴിയൊഴിഞ്ഞുകൊടുത്തു എന്നുകണ്ടു് എഫ്.ഐ.എ. ഫെറാരിയ്ക്കു് ഒരു ലക്ഷം ഡോളര്‍ പിഴയിടുകയും, കൂടുതല്‍ 
ശിക്ഷയുടെ കാര്യം തീരുമാനിക്കാനായി വേള്‍ഡ് മോട്ടോര്‍സ്പോര്‍ട്സ് കൗണ്‍സിലിനു വിടുകയും ചെയ്തതാണു്.

ഫെറാരിയുടെ ആരാധകര്‍ക്ക് ആഹ്ലാദിക്കാന്‍ ധാരാളം ഇടനല്‍കിയെങ്കിലും, നാലു് ശക്തരായ ജര്‍മന്‍ ഡ്രൈവര്‍മാര്‍ 
അണിനിരന്ന റേസ് ജര്‍മനിക്കു് വലിയ ആഹ്ലാദമൊന്നും നല്‍കിയില്ല. പോള്‍ നേടുകയും ഫെറാരികളുടെ പിറകില്‍ 
മൂന്നാംസ്ഥാനം നേടുകയും ചെയ്ത സെബാസ്റ്റ്യന്‍ വെറ്റല്‍ മാത്രമാണു് ജര്‍മനിക്കു് ആശ്വാസമായതു്. മെഴ്സിഡസിന്റെ 
ജര്‍മന്‍ ഡ്രൈവര്‍മാര്‍ സീസണിലെ തങ്ങളുടെ പതിവുതുടര്‍ന്നപ്പോള്‍, ഫോഴ്സ് ഇന്ത്യയുടെ ഏറ്റവും മോശം 
റേസുകളിലൊന്നായിരുന്നു ഇതു്. ലിയുസ്സി യോഗ്യതാറൗണ്ടില്‍ കാറിന്റെ നിയന്ത്രണം നഷ്ടപ്പെട്ടു് പുറത്തിരുന്നുവെങ്കില്‍ 
ഗിയര്‍ബോക്സ് മാറ്റിവച്ചതിനു് അഞ്ചു് സ്ഥാനം ഗ്രിഡില്‍ പെനാല്‍ട്ടിയുമായാണു് സുട്ടില്‍ തുടങ്ങിയതു്. മാത്രമല്ല, 
റേസിനിടയില്‍ പിറ്റില്‍വച്ചു് രണ്ടു ഡ്രൈവര്‍മാരുടെയും ടയറുകള്‍ മാറിപ്പോയതിനു് ഫോഴ്സ് ഇന്ത്യക്കു് പിഴയും ലഭിച്ചു. 
ഇന്ത്യന്‍ ആരാധകരുടെ മറ്റൊരു പ്രതീക്ഷയായ കരണ്‍ ചന്ദോക്കിനാവട്ടെ ജര്‍മനിയില്‍ ഹിസ്പാനിക് റേസിങ് ടീം 
അവസരം നല്‍കിയതുമില്ല.

കഴിഞ്ഞ കുറെ റേസിലെ പതിവില്‍നിന്നും വിപരീതമായി, ഇത്തവണ ആദ്യ പത്തുസ്ഥാനങ്ങളെല്ലാംതന്നെ 
സ്വന്തമാക്കിയതു് മുന്‍നിര ടീമുകളാണു്. ഒന്നും രണ്ടും ഫെറാരി, മൂന്നും ആറും റെഡ്ബുള്‍, നാലും അഞ്ചും മക്‌ലാരന്‍, ഏഴും 
പത്തും റെനോ, എട്ടും ഒന്‍പതും മെഴ്സിഡസ്. കഴിഞ്ഞ റേസുകളില്‍ മികച്ച പ്രകടനം കാഴ്ചവെച്ച വില്യംസും സൗബറും 
ടോറോ റോസോയും മറ്റും കുറച്ചുമങ്ങിപ്പോയെന്നുവേണമെങ്കില്‍ പറയാം. സീസണിന്റെ തുടക്കംമുതലേ മധ്യനിരയിലെ 
ശക്തമായ സാന്നിധ്യമായിരിക്കുകയും, യൂറോപ്പില്‍ പോഡിയംവരെ നേടാന്‍ സാധ്യതകല്‍പ്പിക്കപ്പെടുകയും ചെയ്തിരുന്ന 
ഫോഴ്സ് ഇന്ത്യയാകട്ടെ ജര്‍മനിയില്‍ തീരെ മങ്ങിപ്പോയി.

ശക്തമായ ഒരു സ്റ്റാര്‍ട്ടിലൂടെ ഫെറാരിയുടെ ഫെലിപെ മസ്സ വെറ്റലിനെ മറികടന്നുവെങ്കിലും രണ്ടാമതുണ്ടായിരുന്ന 
അലോണ്‍സൊയെ ചെറുതായി ഒന്നു തടുക്കാന്‍ ജര്‍മനു കഴിഞ്ഞു. പക്ഷെ അധികം വൈകാതെതന്നെ മസ്സ, 
അലോണ്‍സൊ, വെറ്റല്‍ എന്നായി നില. കഴിഞ്ഞമത്സരങ്ങളില്‍നിന്നും വ്യത്യസ്തമായി, മക്‌ലാരന്‍ കാറുകള്‍ക്കു് വെറ്റലിന്റെ 
മുകളില്‍ സമ്മര്‍ദ്ദം ചെലുത്താന്‍ സാധിച്ചില്ല. ആദ്യലാപ്പില്‍ ഗ്രിഡ്ഡിന്റെ മധ്യത്തിലുണ്ടായ അപകടം ടോറോ റോസോയുടെ 
സെബാസ്റ്റ്യന്‍ ബ്യയെമിയുടെ റേസിനു് വിരാമമിടുകയും ഒട്ടേറേ കാറുകള്‍ക്കു് പിറ്റ് ലേനിലേക്കു് ഒരു ട്രിപ്പ് സമ്മാനിക്കുകയും 
ചെയ്തു. ലോട്ടസിനെ വിടാതെ പിന്തുടരുന്ന സ്ഥിരതയില്ലായ്മ ഇത്തവണ ട്രൂലിയുടെ ഗിയര്‍ബോക്സിനെയാണു് ഇരയാക്കിയതു്. 
മൂന്നാം ലാപ്പില്‍ ട്രൂലിയുടെ റേസ് അവസാനിച്ചു.

പതിമൂന്നാം ലാപ്പുമുതല്‍ മുന്‍നിരകാറുകള്‍ പിറ്റ് ചെയ്തു് തുടങ്ങി. പതിനഞ്ചാം ലാപ്പില്‍ പിറ്റ് ചെയ്ത റേസ് ലീഡര്‍ മസ്സ 
രണ്ടാമനായാണു് തിരിച്ചുകയറിയത്. ഇരുപത്തിയാറാം ലാപ്പില്‍ ബട്ടന്‍ പിറ്റു ചെയ്യുന്നതുവരെ രണ്ടാംസ്ഥാനത്തുണ്ടായിരുന്ന 
മസ്സ അത്ഭുതങ്ങള്‍ക്കൊന്നും ഇടനല്‍കാതെ വീണ്ടും റേസ് ലീഡറായി. പിന്നീടു് മുപ്പത്തിയഞ്ചാം 
ലാപ്പുവരെ പ്രത്യേകിച്ചൊന്നും സംഭവിച്ചില്ല. സൗബറിന്റെ പെഡ്രോ ഡി ലാ റോസ ടീം മേറ്റ് കമുയി 
കൊബിയാഷിയേക്കാളും കേമനാണു് താനെന്നു കാണിക്കാനെന്നോണം ട്രാക്കില്‍ അഗ്രസ്സീവായി പെരുമാറിയതു് 
ആരാധകര്‍ക്ക് ആഹ്ലാദം പകര്‍ന്ന കാഴ്ചയായിരുന്നു.

ഹള്‍ക്കെന്‍ബെര്‍ഗിനെ മറികടന്നു് ഏഴാംസ്ഥാനത്തെത്തിയെങ്കിലും പോരാട്ടം 
മുന്‍നിരകാറുകളിലേക്കെത്തിക്കാനാവാത്തതിനാല്‍ അവിടംകൊണ്ടു് തൃപ്തിപ്പെടേണ്ടിവന്നു. ഇതേസമയം മുന്‍നിരയില്‍ 
മസ്സ അലോണ്‍സൊയില്‍നിന്നും ശക്തമായ സമ്മര്‍ദ്ദം നേരിടുന്നുണ്ടായിരുന്നു. എങ്കിലും പിന്നിടു് നാല്‍പ്പത്തിയൊന്‍പതാം 
ലാപ്പിലാണു് പിഴയ്ക്കിടയാക്കിയ റേഡിയോ നിര്‍ദ്ദേശവും മറികടക്കലുമുണ്ടായതു്.

അന്‍പത്തിരണ്ടാം ലാപ്പില്‍ പെഡ്രോ ഡി ലാ റോസ പിറ്റ് ചെയ്തു് സോഫ്റ്റ് ഓപ്ഷന്‍ ടയറുകളില്‍ പുറത്തുവന്നു. ഉടന്‍തന്നെ 
വില്യംസുകളോടു് ശക്തമായ പോരാട്ടവും തുടങ്ങി. അന്‍പത്തിയെട്ടാം ലാപ്പായപ്പോഴേക്കും റോസയുടെ പോരാട്ടം 
ബാരിക്കെല്ലോയുടെ വില്യംസിനോടായിരുന്നു. എന്നാല്‍ അറുപതാം ലാപ്പില്‍ തനിക്കും വില്യംസിനും ഇടയില്‍ ചാടിയ 
കൊവാലെയ്നന്റെ രണ്ടാം ലോട്ടസിനോടുരസി റോസയ്ക്കു് വീണ്ടും പിറ്റു ചെയ്യേണ്ടിവന്നു. അതോടെ ലോട്ടസിന്റെ ജര്‍മ്മന്‍ 
ഗ്രാന്‍പ്രീയ്ക്ക് തിരശ്ശീലവീഴുകയും ചെയ്തു. അവസാന ലാപ്പുകളില്‍ ശക്തമായ പോരാട്ടമായിരുന്നു മുന്‍നിരയില്‍ മസ്സയും 
വെറ്റലും തമ്മില്‍ നടന്നതു്.

എന്തായാലും ജര്‍മന്‍ ഗ്രാന്‍പ്രീയും ഹോം റേസില്‍ ഒരു ജര്‍മനെ ഒന്നാമതെത്തിക്കാതെ സീസണിന്റെ റെക്കോര്‍ഡ് കാത്തു 
സൂക്ഷിച്ചു. ജര്‍മനിയില്‍ ശക്തമായി സാന്നിധ്യമറിയിച്ചെങ്കിലും കിരീടപോരാട്ടത്തില്‍ മക്‌ലാരനില്‍ (300) നിന്നും 92 
പോയിന്റ് പിറകിലാണു് ഫെറാരി. റെഡ്ബുള്ളാകട്ടെ 28 പോയിന്റ് പിറകിലും. ഡ്രൈവര്‍മാരുടെ പോരാട്ടത്തില്‍ 
ഹാമില്‍ട്ടണ്‍ 157 പോയിന്റുമായി ഇപ്പോഴും മുന്നിലാണു്. ബ്രിട്ടനിലേക്കാളും തന്റെ നില അദ്ദേഹം മെച്ചപ്പെടുത്തുകയും 
ചെയ്തു. എന്നാല്‍ വെറ്റല്‍ (136) സ്വന്തം ടീം മേറ്റ് വെബ്ബറിനൊപ്പത്തിനൊപ്പമാണു്.

ആഗസ്റ്റ് ഒന്നിനാണു് ഹംഗേറിയന്‍ ഗ്രാന്‍പ്രീ. ബാക്ക് ടു ബാക്ക് റേസായതിനാല്‍ കൂടുതല്‍ വലിയ മാറ്റങ്ങളൊന്നും 
ടീമുകളില്‍നിന്നും പ്രതീക്ഷിക്കേണ്ടതില്ല. എന്നാല്‍ ഡ്രൈവേഴ്സ് ചാമ്പ്യന്‍ഷിപ്പില്‍ ശക്തമായ പോരാട്ടം നടക്കുന്നതും ഫെറാരി 
താളം കണ്ടെത്തിയതും മക്‌ലാരന്‍ റെഡ്ബുള്‍ ടീമുകളെ വിഷമിപ്പിക്കും. മെഴ്സിഡസ് ഈ വര്‍ഷത്തെ പോരാട്ടം 
അവസാനിപ്പിച്ചു് അടുത്തവര്‍ഷത്തെ കാറില്‍ ശ്രദ്ധകേന്ദ്രീകരിക്കാന്‍ തീരുമാനിച്ചാല്‍ വീണ്ടും ടീമുകള്‍ സമ്മര്‍ദ്ദത്തിലാകും.
എന്തായാലും വരുംനാളുകളിലെ യൂറോപ്യന്‍ റേസുകളും തീപാറുന്നവയായിരിക്കുമെന്നു് നമുക്കുറപ്പിക്കാം.

\begin{flushright}(28 July, 2010)\footnote{http://malayal.am/വിനോദം/കായികം/7023/ജര്‍മന്‍-ഗ്രാന്‍പ്രീയില്‍-ഫെറാരിയുടെ-തിരിച്ചുവരവ്}\end{flushright}

\newpage
