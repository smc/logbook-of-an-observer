\secstar{ജര്‍മന്‍ ഗ്രാന്‍പ്രീയില്‍ ഫെറാരിയുടെ തിരിച്ചുവരവ്}
\vskip 2pt

25 ജൂ­ലൈ ഞാ­യ­റാ­ഴ്ച ജര്‍­മ­നി­യി­ലെ ഹോ­ക്കന്‍­ഹൈ­മി­ലെ സര്‍­ക്യൂ­ട്ടില്‍ നട­ന്ന പതി­നൊ­ന്നാം പാ­ദം ഫെ­റാ­രി­യു­ടെ 
തി­രി­ച്ചു വര­വു­കൊ­ണ്ടാ­ണ് ശ്ര­ദ്ധേ­യ­മാ­യ­ത്. സീ­സ­ണി­ലെ ആദ്യ­റേ­സാ­യി­രു­ന്ന ബഹ്റൈന്‍ ഗ്രാന്‍­പ്രീ­ക്കു ശേ­ഷം 
ജര്‍­മ­നി­യി­ലാ­ണ് വമ്പന്‍­മാ­രായ ­ഫെ­റാ­രി­ ഒരു 1-2 പോ­ഡി­യം ഫി­നി­ഷ് കര­സ്ഥ­മാ­ക്കി­യ­ത്. ഫെ­റാ­രി­യു­ടെ 
വി­ജ­യ­ത്തി­ള­ക്ക­ത്തില്‍ ചെ­റിയ കരി­നി­ഴല്‍ വീ­ഴ്‌­ത്തി­യ­ത്, അലോണ്‍­സൊ­ക്ക് ഒന്നാം സ്ഥാ­നം കി­ട്ടാന്‍ വേ­ണ്ടി ­ഫെ­ലി­പെ 
മസ്സ വഴി­യൊ­ഴി­ഞ്ഞു­കൊ­ടു­ത്തു എന്നു­ക­ണ്ട് എഫ്. ഐ. ഏ. ഫെ­റാ­രി­യ്ക്ക് ഒരു ലക്ഷം ഡോ­ളര്‍ പി­ഴ­യി­ടു­ക­യും, കൂ­ടു­തല്‍ 
ശി­ക്ഷ­യു­ടെ കാ­ര്യം തീ­രു­മാ­നി­ക്കാ­നാ­യി വേള്‍­ഡ് മോ­ട്ടോര്‍­സ്പോര്‍­ട്സ് കൌണ്‍­സി­ലി­നു വി­ടു­ക­യും ചെ­യ്ത­താ­ണ്.

­ഫെ­റാ­രി­യു­ടെ ആരാ­ധ­കര്‍­ക്ക് ആഹ്ലാ­ദി­ക്കാന്‍ ധാ­രാ­ളം ഇട­നല്‍­കി­യെ­ങ്കി­ലും, നാ­ലു ശക്ത­രായ ജര്‍­മന്‍ ഡ്രൈ­വര്‍­മാര്‍ 
അണി­നി­ര­ന്ന റേ­സ് ജര്‍­മ­നി­ക്ക് വലിയ ആഹ്ലാ­ദ­മൊ­ന്നും നല്‍­കി­യി­ല്ല. പോള്‍ നേ­ടു­ക­യും ഫെ­റാ­രി­ക­ളു­ടെ പി­റ­കില്‍ 
മൂ­ന്നാം സ്ഥാ­നം നേ­ടു­ക­യും ചെ­യ്ത ­സെ­ബാ­സ്റ്റ്യന്‍ വെ­റ്റല്‍ മാ­ത്ര­മാ­ണ് ജര്‍­മ­നി­ക്ക് ആശ്വാ­സ­മാ­യ­ത്. മെ­ഴ്സി­ഡ­സി­ന്റെ 
ജര്‍­മന്‍ ഡ്രൈ­വര്‍­മാര്‍ സീ­സ­ണി­ലെ തങ്ങ­ളു­ടെ പതി­വു­തു­ടര്‍­ന്ന­പ്പോള്‍, ഫോ­ഴ്സ് ഇന്ത്യ­യു­ടെ ഏറ്റ­വും മോ­ശം 
റേ­സു­ക­ളി­ലൊ­ന്നാ­യി­രു­ന്നു ഇത്. ലി­യു­സ്സി യോ­ഗ്യ­താ­റൌ­ണ്ടില്‍ കാ­റി­ന്റെ നി­യ­ന്ത്ര­ണം നഷ്ട­പ്പെ­ട്ട് പു­റ­ത്തി­രു­ന്നു­വെ­ങ്കില്‍ 
ഗി­യര്‍­ബോ­ക്സ് മാ­റ്റി വച്ച­തി­ന് അഞ്ചു സ്ഥാ­നം ഗ്രി­ഡില്‍ പെ­നാല്‍­ട്ടി­യു­മാ­യാ­ണ് സു­ട്ടില്‍ തു­ട­ങ്ങി­യ­ത്. മാ­ത്ര­മ­ല്ല, 
റേ­സി­നി­ട­യില്‍ പി­റ്റില്‍ വച്ച് രണ്ടു ഡ്രൈ­വര്‍­മാ­രു­ടെ­യും ടയ­റു­കള്‍­മാ­റി­പ്പോ­യ­തി­ന് ഫോ­ഴ്സ്ഇ­ന്ത്യ­ക്ക് പി­ഴ­യും ലഭി­ച്ചു. 
ഇന്ത്യന്‍ ആരാ­ധ­ക­രു­ടെ മറ്റൊ­രു പ്ര­തീ­ക്ഷ­യായ കരണ്‍ ചന്ദോ­ക്കി­നാ­വ­ട്ടെ ജര്‍­മ­നി­യില്‍ ഹി­സ്പാ­നി­ക് റേ­സി­ങ് ടീം 
അവ­സ­രം നല്‍­കി­യ­തു­മി­ല്ല.

­ക­ഴി­ഞ്ഞ കു­റെ റേ­സി­ലെ പതി­വില്‍ നി­ന്നും വി­പ­രീ­ത­മാ­യി, ഇത്ത­വണ ആദ്യ പത്തു സ്ഥാ­ന­ങ്ങ­ളെ­ല്ലാം­ത­ന്നെ 
സ്വ­ന്ത­മാ­ക്കി­യ­ത് മുന്‍­നിര ടീ­മു­ക­ളാ­ണ്. ഒന്നും രണ്ടും ഫെ­റാ­രി, മൂ­ന്നും ആറും റെ­ഡ്ബുള്‍, നാ­ലും അഞ്ചും മക്‌­ലാ­രന്‍, ഏഴും 
പത്തും ­റെ­നോ­, എട്ടും ഒന്‍­പ­തും മെ­ഴ്സി­ഡ­സ്. കഴി­ഞ്ഞ റേ­സു­ക­ളില്‍ മി­ക­ച്ച പ്ര­ക­ട­നം കാ­ഴ്ച­വെ­ച്ച വി­ല്യം­സും സൌ­ബ­റും 
ടോ­റോ റോ­സോ­യും മറ്റും കു­റ­ച്ചു­മ­ങ്ങി­പ്പോ­യെ­ന്നു­വേ­ണ­മെ­ങ്കില്‍ പറ­യാം. സീ­സ­ണി­ന്റെ തു­ട­ക്കം മു­ത­ലേ മധ്യ­നി­ര­യി­ലെ 
ശക്ത­മായ സാ­ന്നി­ധ്യ­മാ­യി­രി­ക്കു­ക­യും, യൂ­റോ­പ്പില്‍ പോ­ഡി­യം വരെ നേ­ടാന്‍ സാ­ധ്യ­ത­കല്‍­പ്പി­ക്ക­പ്പെ­ടു­ക­യും ചെ­യ്തി­രു­ന്ന 
ഫോ­ഴ്സ് ഇന്ത്യ­യാ­ക­ട്ടെ ജര്‍­മ­നി­യില്‍ തീ­രെ മങ്ങി­പ്പോ­യി­.

­ശ­ക്ത­മായ ഒരു സ്റ്റാര്‍­ട്ടി­ലൂ­ടെ ഫെ­റാ­രി­യു­ടെ ഫെ­ലി­പെ മസ്സ വെ­റ്റ­ലി­നെ മറി­ക­ട­ന്നു­വെ­ങ്കി­ലും രണ്ടാ­മ­തു­ണ്ടാ­യി­രു­ന്ന 
അലോണ്‍­സൊ­യെ ചെ­റു­താ­യി ഒന്നു തടു­ക്കാന്‍ ജര്‍­മ­നു കഴി­ഞ്ഞു. പക്ഷെ അധി­കം വൈ­കാ­തെ തന്നെ നില മസ്സ, 
അലോണ്‍­സൊ, വെ­റ്റല്‍ എന്നാ­യി. കഴി­ഞ്ഞ­മ­ത്സ­ര­ങ്ങ­ളില്‍ നി­ന്നും വ്യ­ത്യ­സ്ത­മാ­യി, മക്‌­ലാ­രന്‍ കാ­റു­കള്‍­ക്ക് വെ­റ്റ­ലി­ന്റെ 
മു­ക­ളില്‍ സമ്മര്‍­ദ്ദം ചെ­ലു­ത്താന്‍ സാ­ധി­ച്ചി­ല്ല. ആദ്യ ലാ­പ്പില്‍ ഗ്രി­ഡ്ഡി­ന്റെ മധ്യ­ത്തി­ലു­ണ്ടായ അപ­ക­ടം ടോ­റോ റോ­സോ­യു­ടെ 
സെ­ബാ­സ്റ്റ്യന്‍ ബ്യ­യെ­മി­യു­ടെ റേ­സി­ന് വി­രാ­മ­മി­ടു­ക­യും ഒട്ടേ­റേ കാ­റു­കള്‍­ക്ക് പി­റ്റ്ലേ­നി­ലേ­ക്ക് ഒരു ട്രി­പ്പ് സമ്മാ­നി­ക്കു­ക­യും 
ചെ­യ്തു. ലോ­ട്ട­സി­നെ വി­ടാ­തെ പി­ന്തു­ട­രു­ന്ന സ്ഥി­ര­ത­യി­ല്ലാ­യ്മ ഇത്ത­വണ ട്രൂ­ലി­യു­ടെ ഗി­യര്‍­ബോ­ക്സി­നെ­യാ­ണ് ഇര­യാ­ക്കി­യ­ത്. 
മൂ­ന്നാം ലാ­പ്പില്‍ ട്രൂ­ലി­യു­ടെ റേ­സ് അവ­സാ­നി­ച്ചു­.

­പ­തി­മൂ­ന്നാം ലാ­പ്പു­മു­തല്‍ മുന്‍­നി­ര­കാ­റു­കള്‍ പി­റ്റ് ചെ­യ്ത് തു­ട­ങ്ങി. പതി­ന­ഞ്ചാം ലാ­പ്പില്‍ പി­റ്റ് ചെ­യ്ത റേ­സ് ലീ­ഡര്‍ മസ്സ 
രണ്ടാ­മ­നാ­യാ­ണ് തി­രി­ച്ചു കയ­റി­യ­ത്. ഇരു­പ­ത്തി­യാ­റാം ലാ­പ്പില്‍ ബട്ടന്‍ പി­റ്റു ചെ­യ്യു­ന്ന­തു­വ­രെ രണ്ടാം 
സ്ഥാ­ന­ത്തു­ണ്ടാ­യി­രു­ന്ന മസ്സ അത്ഭു­ത­ങ്ങള്‍­ക്കൊ­ന്നും ഇട­നല്‍­കാ­തെ വീ­ണ്ടും റേ­സ് ലീ­ഡ­റാ­യി. പി­ന്നീ­ട് മു­പ്പ­ത്തി­യ­ഞ്ചാം 
ലാ­പ്പു­വ­രെ പ്ര­ത്യേ­കി­ച്ചൊ­ന്നും സം­ഭ­വി­ച്ചി­ല്ല. സൌ­ബ­റി­ന്റെ ­പെ­ഡ്രോ ഡി ലാ റോ­സ ടീം മേ­റ്റ് കമു­യി 
കൊ­ബി­യാ­ഷി­യേ­ക്കാ­ളും കേ­മ­നാ­ണു താ­നെ­ന്നു കാ­ണി­ക്കാ­നെ­ന്നോ­ണം ട്രാ­ക്കില്‍ അഗ്ര­സ്സീ­വാ­യി പെ­രു­മാ­റി­യ­ത് 
ആരാ­ധ­കര്‍­ക്ക് ആഹ്ലാ­ദം പകര്‍­ന്ന കാ­ഴ്ച­യാ­യി­രു­ന്നു­.

­ഹള്‍­ക്കെന്‍­ബെര്‍­ഗി­നെ മറി­ക­ട­ന്ന് ഏഴാം സ്ഥാ­ന­ത്തെ­ത്തി­യെ­ങ്കി­ലും പോ­രാ­ട്ടം 
മുന്‍­നി­ര­കാ­റു­ക­ളി­ലേ­ക്കെ­ത്തി­ക്കാ­നാ­വാ­ത്ത­തി­നാല്‍ അവി­ടം കൊ­ണ്ട് തൃ­പ്തി­പ്പെ­ടേ­ണ്ടി­വ­ന്നു. ഇതേ­സ­മ­യം മുന്‍­നി­ര­യില്‍ 
മസ്സ അലോണ്‍­സൊ­യില്‍­നി­ന്നും ശക്ത­മായ സമ്മര്‍­ദ്ദം നേ­രി­ടു­ന്നു­ണ്ടാ­യി­രു­ന്നു. എങ്കി­ലും പി­ന്നി­ട് നാല്‍­പ്പ­ത്തി­യൊന്‍­പ­താം 
ലാ­പ്പി­ലാ­ണ് പി­ഴ­യ്ക്കി­ട­യാ­ക്കിയ റേ­ഡി­യോ നിര്‍­ദ്ദേ­ശ­വും മറി­ക­ട­ക്ക­ലു­മു­ണ്ടാ­യ­ത്.

അന്‍­പ­ത്തി­ര­ണ്ടാം ലാ­പ്പില്‍ പെ­ഡ്രോ ഡി ലാ റോസ പി­റ്റ് ചെ­യ്ത് സോ­ഫ്റ്റ് ഓപ്ഷന്‍ ടയ­റു­ക­ളില്‍ പു­റ­ത്തു­വ­ന്നു. ഉടന്‍­ത­ന്നെ, 
വി­ല്യം­സു­ക­ളോ­ട് ശക്ത­മായ പോ­രാ­ട്ട­വും തു­ട­ങ്ങി. അന്‍­പ­ത്തി­യെ­ട്ടാം ലാ­പ്പാ­യ­പ്പോ­ഴേ­ക്കും റോ­സ­യു­ടെ പോ­രാ­ട്ടം 
ബാ­രി­ക്കെ­ല്ലോ­യു­ടെ വി­ല്യം­സി­നോ­ടാ­യി­രു­ന്നു. എന്നാല്‍ അറു­പ­താം ലാ­പ്പില്‍ തനി­ക്കും വി­ല്യം­സി­നും ഇട­യില്‍ ചാ­ടിയ 
കൊ­വാ­ലെ­യ്ന­ന്റെ രണ്ടാം ലോ­ട്ട­സി­നോ­ടു­ര­സി റോ­സ­യ്ക്ക് വീ­ണ്ടും പി­റ്റു ചെ­യ്യേ­ണ്ടി­വ­ന്നു. അതോ­ടെ ലോ­ട്ട­സി­ന്റെ ജര്‍­മ്മന്‍ 
ഗ്രാന്‍­പ്രീ­യ്ക്ക് തി­ര­ശ്ശീ­ല­വീ­ഴു­ക­യും ചെ­യ്തു. അവ­സാന ലാ­പ്പു­ക­ളില്‍ ശക്ത­മായ പോ­രാ­ട്ട­മാ­യി­രു­ന്നു മുന്‍­നി­ര­യില്‍ മസ്സ­യും 
വെ­റ്റ­ലും തമ്മില്‍ നട­ന്ന­ത്.

എ­ന്താ­യാ­ലും ജര്‍­മന്‍ ഗ്രാന്‍­പ്രീ­യും ഹോം റേ­സില്‍ ഒരു ജര്‍­മ­നെ ഒന്നാ­മ­തെ­ത്തി­ക്കാ­തെ സീ­സ­ണി­ന്റെ റെ­ക്കോര്‍­ഡ് കാ­ത്തു 
സൂ­ക്ഷി­ച്ചു. ജര്‍­മ­നി­യില്‍ ശക്ത­മാ­യി സാ­ന്നി­ധ്യ­മ­റി­യി­ച്ചെ­ങ്കി­ലും കി­രീ­ട­പോ­രാ­ട്ട­ത്തില്‍ മക്‌­ലാ­ര­നില്‍ (300) നി­ന്നും 92 
പോ­യി­ന്റ് പി­റ­കി­ലാ­ണ് ഫെ­റാ­രി. റെ­ഡ്ബു­ളാ­ക­ട്ടെ 28 പോ­യി­ന്റ് പി­റ­കി­ലും. ഡ്രൈ­വര്‍­മാ­രു­ടെ പോ­രാ­ട്ട­ത്തില്‍ 
ഹാ­മില്‍­ട്ടണ്‍ 157 പോ­യി­ന്റു­മാ­യി ഇപ്പോ­ഴും മു­ന്നി­ലാ­ണ്. ബ്രി­ട്ട­നി­ലേ­ക്കാ­ളും തന്റെ നില അദ്ദേ­ഹം മെ­ച്ച­പ്പെ­ടു­ത്തു­ക­യും 
ചെ­യ്തു. എന്നാല്‍ വെ­റ്റല്‍ (136) സ്വ­ന്തം ടീം മേ­റ്റ് വെ­ബ്ബ­റി­നൊ­പ്പ­ത്തി­നൊ­പ്പ­മാ­ണ്.

ആ­ഗ­സ്റ്റ് ഒന്നി­നാ­ണ് ഹം­ഗേ­റി­യന്‍ ഗ്രാന്‍­പ്രീ. ബാ­ക്ക് ടു ബാ­ക്ക് റേ­സാ­യ­തി­നാല്‍ കൂ­ടു­തല്‍ വലിയ മാ­റ്റ­ങ്ങ­ളൊ­ന്നും 
ടീ­മു­ക­ളില്‍ നി­ന്നും പ്ര­തീ­ക്ഷി­ക്കേ­ണ്ട­തി­ല്ല. എന്നാല്‍ ഡ്രൈ­വേ­ഴ്സ് ചാ­മ്പ്യന്‍­ഷി­പ്പില്‍ ശക്ത­മായ പോ­രാ­ട്ടം നട­ക്കു­ന്ന­തും ഫെ­റാ­രി 
താ­ളം കണ്ടെ­ത്തി­യ­തും മക്‌­ലാ­രന്‍ ­റെ­ഡ്ബുള്‍ ടീ­മു­ക­ളെ വി­ഷ­മി­പ്പി­ക്കും. ­മെ­ഴ്സി­ഡ­സ് ഈ വര്‍­ഷ­ത്തെ പോ­രാ­ട്ടം 
അവ­സാ­നി­പ്പി­ച്ച് അടു­ത്ത വര്‍­ഷ­ത്തെ കാ­റില്‍ ശ്ര­ദ്ധ­കേ­ന്ദ്രീ­ക­രി­ക്കാന്‍ തീ­രു­മാ­നി­ച്ചാല്‍ വീ­ണ്ടും ടീ­മു­കള്‍ സമ്മര്‍­ദ്ദ­ത്തി­ലാ­കും.
എന്താ­യാ­ലും വരും നാ­ളു­ക­ളി­ലെ യൂ­റോ­പ്യന്‍ റേ­സു­ക­ളും തീ­പാ­റു­ന്ന­വ­യാ­യി­രി­ക്കു­മെ­ന്ന് നമു­ക്കു­റ­പ്പി­ക്കാം­.

(28 July 2010)\footnote{http://malayal.am/വിനോദം/കായികം/7023/ജര്‍മന്‍-ഗ്രാന്‍പ്രീയില്‍-ഫെറാരിയുടെ-തിരിച്ചുവരവ്}

\newpage
