\secstar{Again on student politics, freedom of expression and interestingly my views about women}
\vskip 2pt

In Facebook group of MESCE, there was an interesting discussion on an age old topic. 
Fatima Ali (B Arch, student at MES I guess) started it. She wanted our opinion on whether 
you like to be in a college like MES, laden with strikes, gang wars and not much cultural 
activities, but authorities are questioned from time to time. Or in a college at Bangalore 
or somewhere, where no one questions authorities and in return they get to conduct some 
extravagant events and activities. As someone who values freedom over anything, my 
choice was obviously the former one, stating there are better alternatives with better 
democratic atmosphere (with both inclusive and representative democracy).

About the kind of politics in college, I had written two long articles (here\footnote{gang war 1 [TODO]} 
and here\footnote{gang war 2 [TODO]}). The first one had a fierce discussion in the comment section. 
Then my experience in IIIT gave me a new outlook of student politics and democracy in academic institutions. 
In any system, it completely depends upon the leaders. No one groomed the next generation or batch. If there 
was someone who was naturally capable of and interested in taking the lead, the whole community would 
benefit. Else the ones who represented the students simply gained something for themselves.

Anyway, that was not what I was going to write here. In the whole discussion, only a handful of people 
expressed their views. Interestingly (or as usual) the only lady in the whole discussion was Fatima. Later 
Karthika also joined. When Fatima wrote a concluding note, I innocently asked, ``Fatima, interestingly, you 
were the only girl here, are you people afraid to express yourself or what?''. I was actually interested to 
know whether all the ladies in the group (both alumni and current students of MES) were not interested 
and believed that it is up to them to express their views on politics. The result, I was accused of being 
``Male Chauvinist''. I have been called many names during discussions in many forums. I have been 
part of internet and real world forums which discussed a variety of problems including the plight of 
women. I do have a variety of friends from a various sections of society and it is the first time one of 
them is accusing me of being a male chauvinist.

Anyway, whether am I a male chauvinist or not is not worth a discussion. Though to make my part clear, 
I should state that I have respect for and belief in all women. As a matter of fact, in many occasions I 
had to oppose the prejudiced remarks about them. Once I had an argument with my father about political 
awareness among female student population (later there was one about geographical awareness). I was 
the one who defended the girls against his prejudiced, society imposed “ignorant female” thesis. So, 
it was natural for me to be curious to know whether there is any truth in the thesis put forward by my father 
(and majority of the society). A reply by Fatima does imply that, they have their own opinions and they do 
think and try to understand situations (it seems that they just don’t like to put it up in public forums).

The cause of my father’s thesis might be the society imposed restrictions on women to get into public 
discussions (when you are not part of discussions, you don’t have opinions. It is an age old method of 
suppression). Then later, women themselves started restricting themselves from being a part of messy 
discussions to avoid being tagged. I think Sreebala K. Menon once wrote about this in Mathrubhumi 
(society tagging females who take their own path in life or express their opinion on worldly issues). 
When you look around, females who (tried to) express their views and take a stand in issues were 
attacked more fiercely than their male colleagues. Just think about Arundhathi Roy in recent Kashmir 
issue. There were 3 people in the dias and only her remarks were taken up by Times Now. If you read 
her interviews in Thehelka (there are many on different issues), you will understand how good an intellectual 
she is and how much research and analysis has been done by her before expressing her opinion on a 
matter (still we accuse her of reacting spontaneously to hot issues).

In case of MES, KBLH is a jail. Parents feel their children are safe in jail, where in fact, they might be 
getting frustrated. If people are going to say we are happy in KBLH, I would suggest them to look up 
Stockholm Syndrome. When there is a strike in college, the entire female fraternity of the college is 
stripped of a chance of being part of it (they are in fact jailed). Most of them don't get to see 
what happens or understand what is the point of the agitation. So it will be interesting to hear 
their point of view on the matter which was under discussion on that thread. On top, I was interested 
to know whether the same atmosphere of tension which kept a large portion of females out of 
political discussions in campus was keeping them away from that thread. Now I understand that 
being curious about the opinions, inadequate representation and participation of females in a discussion 
is a good enough ground to be tagged as a male chauvinist.

If, as Fatima says, a lot of women can express their views freely on her wall, why don't they do that in 
the thread in MESCE group? Why don't they pick fights with us and enlighten us with their perspectives 
and visions? I do like to know these and that is not because I am a male chauvinist or sympathiser 
for the cause of female empowerment. But as someone who knows there is huge potential which is 
going to be wasted just because people are not confident enough or not ready to break some unwritten 
rules in the society. Social conventions are not forever. They are supposed to be rewritten after a period of 
time. I had actively written about my views on society. I dislike society for the restrictions it has put on 
my thought and the prejudices it has against anybody and everybody. I learned to fight it and I would like 
more people to fight that and overcome social inertia to make our system a dynamic one.

Here I rest my case!

\newpage
