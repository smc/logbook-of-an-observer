\section*{Again on student politics, freedom of expression and interestingly my views about women}
\vskip 2pt

In facebook group of MESCE, there was an interesting discussion of an age old topic. A Fatima Ali
(B Arch. Student at MES i guess) started it. She wanted our opinion on whether you like to be in a 
college like MES laden with strikes and gang wars and not much cultural activities as a result of 
it but authorities are questioned from time to time. Or in a college in Bangalore or somewhere, 
where none questions the authorities and in return they get to conduct some extravagant events and 
activities. As someone who values freedom over anything, my choice was obviously former stating there are 
better alternatives with better democratic atmosphere(with both inclusive and representative democracy).

About the kind of politics in college, i had written two long articles. The first one has a fierce discussion 
in the comment section. Then my experience in IIIT gave me a new outlook for student politics and democracy in 
academic institutions. In any system, what i found was it all depended on leaders. None groomed the next generation 
or batch. If there was someone who is naturally capable and interested in taking the lead, the whole community benefits. 
Else the ones who represent the students simply gains something for themselves.

Anyway, that was not what was going to write here. In the whole discussion only a handful of people expressed their views. 
Interestingly(or as usual) the only lady in the whole discussion was Fatima. Later Karthika also joined it. 
When Fatima wrote a concluding note, i innocently asked, ``fatima, interestingly, you were the only girl here, 
are you people afraid to express yourself or what?''. I was actually interested to know whether all the 
ladies in the group(both alumni and current students of MES) are not interested or believe its up to them 
to express their views on politics. The result, i was accused of being ``Male Chauvinist''. 
I have been called many names during discussions in many forums. I have been part of internet and real 
world forums which discussed a variety of problems including plight of women. 
I do have a variety of friends from a various sections of society and it is the first time one of 
them is accusing me of being a male chauvinist

Anyway, am I a male chauvinist or not is not worth a discussion. Though to make my part clear, 
i should state that i have respect and belief in all ladies. As a matter of fact in many occasions i had to 
oppose the prejudiced remarks about them. Once I had an argument with my father about political awareness 
among female student population(later there was one about geographical awareness  ). 
I was the one who defended the girls against his prejudiced, society imposed ``ignorant female'' thesis. 
So, it was natural for me to be curious to know whether there is any truth in the thesis put forward by my 
father(and majority of the society). A reply by Fatima does imply that they have their own opinions and they 
do think and try to understand situations(it seems they just don't like to put it up in public forums).

The cause of my father's thesis might be the society imposed restrictions on females to get into public 
discussions(when you are not part of discussions, you don't have opinions, its an age old method of suppression  ). 
Then later, females themselves restricting themselves from being a part of messy discussions to avoid being tagged. 
I think Sreebala K. Menon once wrote about this in Mathrubhumi(society tagging females who take their own path in life 
or express their opinion on worldly issues). When you look around, females who (tried to) express their views and take a 
stand in issues were attacked more fiercely than their male colleagues. Just think about Arundhathi Roy in 
recent Kashmir issue. There were 3 people in the dias, only her remarks were taken up by Times Now. 
If you read her interviews in Thehelka(there are many on different issues) you will understand how good an 
intellectual is she and how much research and analysis is done by her before expressing her opinion on a 
matter(still we accuse her of reacting spontaneously to hot issues).

In case of MES, KBLH is a jail. Parents feel their children are safe in jail, where in fact they might be 
getting frustrated. If people are going to say we are happy in KBLH, i would suggest them to look up 
Stockholm Syndrome. When there is a strike in college, the entire female fraternity of the college is stripped of a 
chance of being part of it(they are in fact jailed). Most of them didn't get to see what happens or understand 
whats the point of the agitation. So it will be interesting to hear their point of view on the matter which was 
under discussion on that thread. On top i was interested to know whether the same atmosphere of tension which kept a 
large portion of females out of political discussions in campus was keeping them away from that thread. 
Now i understood that being curious about the opinions and inadequate representation and participation of 
females in a discussion is good enough ground to be tagged as a male chauvinist.

If as Fatima says a lot of females can express their views freely on her wall, why not they do 
that in the thread in MESCE group? Why not they pick up fights with us and enlighten us 
with their perspectives and visions? I do like to know these and thats not because i am a male chauvinist 
or sympathiser for the cause of female empowerment. But as someone who know there is huge potential which going 
to be wasted just because people are not confident enough or not ready to break some unwritten rules in the society. 
Social conventions are not forever, they are supposed to be rewritten after a period of time. 
I had actively written about my views on society. I dislike society for the restrictions it put on my thought and 
the prejudices it has against anybody and everybody. I learned to fight it, i like more people to fight that and 
over come social inertia to make our system a dynamic one.

Here I rest my case!


\newpage
