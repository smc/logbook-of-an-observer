\secstar{അപ്രത്യക്ഷമാകുന്ന എഡിറ്റോറിയല്‍ ഡെസ്ക്}
\vskip 2pt

ജ­നാ­ധി­പ­ത്യ­ത്തി­ന്റെ നി­ല­നില്‍­പ്പി­നും മനു­ഷ്യാ­വ­കാ­ശ­ങ്ങ­ളു­ടെ പരി­ര­ക്ഷ­യ്ക്കും സ്വ­ത­ന്ത്ര­വും നി­ഷ്പ­ക്ഷ­വു­മായ 
മാ­ദ്ധ്യ­മ­ങ്ങ­ളു­ണ്ടാ­വേ­ണ്ട­ത് അത്യ­ന്താ­പേ­ക്ഷി­ത­മാ­ണ്. "സ്വ­ത­ന്ത്ര­വും നി­ഷ്പ­ക്ഷ­വു­മായ മാ­ദ്ധ്യ­മ­ങ്ങള്‍" എന്ന­തി­ലെ 
സ്വാ­ത­ന്ത്ര്യ­മെ­ന്ന ഭാ­ഗ­ത്തി­ന് കൂ­ടു­തല്‍ ഊന്നല്‍ കൊ­ടു­ക്കു­ക­യും, നി­ഷ്പ­ക്ഷത എന്ന­ത് പല­പ്പോ­ഴും ഒരു ജല­രേ­ഖ­യാ­വു­ക­യും 
ചെ­യ്യു­ന്ന­ത് ഇന്ന­ത്തെ മാ­ദ്ധ്യ­മ­ലോ­ക­ത്ത് സാ­ധാ­ര­ണ­മാ­ണ്. പ്ര­ത്യ­ക്ഷ അജ­ണ്ട­ക­ളോ­ടെ­യോ വ്യ­ക്ത­മായ 
ചാ­യ്‌­വു­ക­ളോ­ടെ­യോ രാ­ഷ്ട്രീ­യ/­മ­ത/­സാ­മൂ­ഹ്യ സം­ഘ­ട­ന­ക­ളു­ടെ ജി­ഹ്വ­ക­ളാ­യി ധാ­രാ­ളം മാ­ദ്ധ്യ­മ­ങ്ങള്‍ പ്ര­വര്‍­ത്തി­ക്കു­ന്നു. 
നി­ഷ്പ­ക്ഷ പ്ര­വര്‍­ത്ത­ന­മെ­ന്ന­തി­നേ­ക്കാ­ളും മു­ഖ്യ­ധാ­ര­യില്‍ പി­ന്ത­ള്ള­പ്പെ­ട്ടു­പോ­കു­ന്ന കാ­ഴ്ച­പ്പാ­ടു­ക­ളെ പൊ­തു­സ­മൂ­ഹ­ത്തില്‍ 
ചര്‍­ച്ച­യ്ക്കു വയ്ക്കുക എന്ന­താ­ണ് ഇവ­രു­ടെ പ്ര­ധാന അ­ജ­ണ്ട.

­വി­ദ്യാ­ഭ്യാ­സ­പ­ര­മാ­യി മു­ന്നോ­ക്കം നില്‍­ക്കു­ന്ന സമൂ­ഹ­ങ്ങ­ളില്‍ പൊ­തു­ജ­നാ­ഭി­പ്രായ രൂ­പീ­ക­ര­ണ­ത്തി­ന് മാ­ദ്ധ്യ­മ­ങ്ങള്‍­ക്കു­ള്ള 
സ്വാ­ധീ­നം അള­ക്കാ­നാ­വാ­ത്ത­താ­ണ്. അതു­മൂ­ലം ഭര­ണ­ത്തി­ന്റെ ചക്രം തി­രി­ക്കു­ന്ന­വര്‍­ക്ക് പല­പ്പോ­ഴും മാ­ദ്ധ്യ­മ­ങ്ങ­ളെ 
പ്രീ­തി­പ്പെ­ടു­ത്തേ­ണ്ടു­ന്ന­ത് ഒരു ആവ­ശ്യ­മാ­കു­ന്നു. ഇത്ത­ര­ത്തില്‍ സ്വ­ന്തം മാ­ദ്ധ്യ­മ­ങ്ങള്‍ ആരം­ഭി­ക്കാന്‍ പണ­മു­ള്ള­വ­രു­ടേ­യും 
അധി­കാ­ര­മു­ള്ള­വ­രു­ടേ­യും മാ­ത്രം സ്വ­ര­ങ്ങള്‍ വഴി പൊ­തു­ജ­നാ­ഭി­പ്രാ­യ­രൂ­പീ­ക­ര­ണം നട­ത്ത­പ്പെ­ടു­ന്ന­ത് തട­യാ­നാ­ണ് 
മാ­ദ്ധ്യ­മ­ങ്ങ­ളും മാ­ദ്ധ്യ­മ­പ്ര­വര്‍­ത്ത­ക­രും സ്വ­യം ഒരു പെ­രു­മാ­റ്റ­ച്ച­ട്ടം രൂ­പീ­ക­രി­ക്ക­ണ­മെ­ന്നു പറ­യു­ന്ന­ത്.

­വാര്‍­ത്ത­കള്‍ വസ്തു­താ­ടി­സ്ഥാ­ന­ത്തി­ലു­ള്ള ­വി­വ­ര­ണ­ങ്ങള്‍ മാ­ത്ര­മാ­വു­ക­യും, മാ­ദ്ധ്യമ അജ­ണ്ട­കള്‍ 
വാര്‍­ത്ത­ക­ളോ­ടു­നു­ബ­ന്ധി­ച്ചു­ള്ള അവ­ലോ­ക­ന­ങ്ങ­ളോ, വി­ശ­ക­ല­ങ്ങ­ളോ, നി­രീ­ക്ഷ­ണ­ങ്ങ­ളോ, അഭി­മു­ഖ­ങ്ങ­ളോ വഴി 
രേ­ഖ­പ്പെ­ടു­ത്തു­ക­യും ചെ­യ്യുക എന്ന­താ­ണ് സാ­മ്പ്ര­ദാ­യി­ക­മാ­യി അം­ഗീ­ക­രി­ച്ചി­ട്ടു­ള്ള രീ­തി. സ്കൂ­പ്പു­ക­ളി­ലോ, 
വെ­ളി­പ്പെ­ടു­ത്ത­ലു­ക­ളി­ലോ മാ­ദ്ധ്യ­മ­ങ്ങ­ളു­ടെ നി­ഗ­മ­ന­ങ്ങള്‍ സ്ഥാ­നം പി­ടി­ക്കു­ന്നു­ണ്ടെ­ങ്കില്‍ അതി­നെ നി­ഗ­മ­ന­ങ്ങ­ളാ­യി­ത്ത­ന്നെ 
കാ­ണി­ക്കു­ന്ന­തും പതി­വാ­ണ്. മാ­ത്ര­മ­ല്ല, പ്ര­സി­ദ്ധീ­ക­രി­ക്കു­ന്ന ഏതൊ­രു വാര്‍­ത്ത­യ്ക്കും (എ­ന്തി­നും) നി­യ­മ­പ­ര­മാ­യും 
ധാര്‍­മ്മി­ക­പ­ര­മാ­യും മാ­ദ്ധ്യ­മ­സ്ഥാ­പ­ന­ങ്ങള്‍ ഉത്ത­ര­വാ­ദി­യു­മാ­ണ് (അ­വ­ന­വന്‍ പ്ര­സാ­ധ­ക­നാ­വു­ന്ന ബ്ലോ­ഗു­കള്‍­ക്കും 
പോര്‍­ട്ട­ലു­കള്‍­ക്കും ഇവ ബാ­ധ­ക­മാ­ണ്).

­ന്യൂ­സു­ക­ളി­ലൂ­ടെ പ്ര­ത്യേക അജ­ണ്ട­കള്‍­ക്ക് പ്ര­ച­ര­ണം കൊ­ടു­ക്കാന്‍ മാ­ദ്ധ്യ­മ­ങ്ങള്‍ സ്വീ­ക­രി­ക്കു­ന്ന എളു­പ്പ­വ­ഴി ഈ 
അതിര്‍­വ­ര­മ്പു­ക­ളെ ഒഴി­വാ­ക്കു­ക­യാ­ണ്. പല­രും ഒരു­പ­ടി­കൂ­ടി കട­ന്ന് വാര്‍­ത്ത­കള്‍ തന്നെ സൃ­ഷ്ടി­ക്കു­ക­യും ചെ­യ്യു­ന്നു. ഇത് 
വേ­ണ­മെ­ന്നു വച്ച് വാര്‍­ത്ത­കള്‍ വള­ച്ചൊ­ടി­ക്കു­ന്ന­വ­രു­ടെ കഥ­ക­ളാ­ണ്.

­വി­ശ­ക­ല­ന­വും അവ­ലോ­ക­ന­വും പല­പ്പോ­ഴും പത്ര­ങ്ങ­ളില്‍ പ്ര­ധാന സ്ഥാ­നം നേ­ടാ­റു­ണ്ട്. പല­പ്പോ­ഴും അന്വേ­ഷ­ണാ­ത്മക 
പര­മ്പ­ര­കള്‍ എഴു­ത­പ്പെ­ടു­ന്ന­ത് വി­ശ­ക­ല­ന­ങ്ങ­ളാ­യി­ട്ടാ­യി­രി­ക്കും. ഇവി­ടെ­യാ­ണ് വി­വ­ര­മു­ള്ള പത്ര­പ്ര­വര്‍­ത്ത­കര്‍ ഇന്ന് 
കു­റ­ഞ്ഞു­വ­രി­ക­യാ­ണെ­ന്നു­ള്ള­തി­ന്റെ സൂ­ച­ന­കള്‍ കാ­ണാ­വു­ന്ന­ത്. ഗം­ഭീ­ര­മാ­യി ഫീല്‍­ഡ് റി­പ്പോര്‍­ട്ടി­ങ് ചെ­യ്യാന്‍ 
കഴി­വു­ള്ള­വര്‍ പല­പ്പോ­ഴും വി­ശ­ക­ല­ന­ത്തി­ലും അതി­നോ­ട­നു­ബ­ന്ധി­ച്ച ചില സാ­മാ­ന്യ­നി­യ­മ­ങ്ങ­ളി­ലും അജ്ഞ­രാ­യി­രി­ക്കും. 
അത് അപൂര്‍­ണ്ണ­വും അപ­ക്വ­വു­മായ നി­ഗ­മ­ന­ങ്ങ­ളി­ലാ­യി­രി­ക്കും പല­പ്പോ­ഴും എത്തി­ക്കു­ന്ന­ത്.

­വി­ശ­ക­ല­ന­ത്തി­നു വേ­ണ്ട പ്രാ­ഥ­മിക വി­വ­ര­ങ്ങള്‍ ശേ­ഖ­രി­ക്കു­ന്ന­തില്‍ തു­ട­ങ്ങി, എറര്‍ മാര്‍­ജിന്‍ എന്ന വാ­ക്കു­പോ­ലും 
കേള്‍­ക്കാ­ത്ത­വര്‍ സ്റ്റാ­റ്റി­സ്റ്റി­ക്കല്‍ അനാ­ലി­സി­സ് നട­ത്തി­യാല്‍ വരു­ന്ന കു­റ­വു­ക­ളും, ഏതൊ­ക്കെ പാ­രാ­മീ­റ്റ­റു­കള്‍ 
മാ­റു­ന്ന­തു­കൊ­ണ്ടാ­ണ് വ്യ­ത്യാ­സ­ങ്ങള്‍ കാ­ണു­ന്ന­തെ­ന്ന കാ­ര്യ­ത്തില്‍ മുന്‍­വി­ധി­കള്‍ നി­ഗ­മ­ന­ങ്ങ­ളെ ബാ­ധി­ക്കു­ന്ന­തും വരെ 
അപ­ക്വ­മായ സാ­മാ­ന്യ­വ­ത്ക­ര­ണ­ത്തി­ന് (immature generalization) കാ­ര­ണ­മാ­കാ­റു­ണ്ട്. ഇത്ത­രം പാ­തി­വെ­ന്ത 
റി­പ്പോര്‍­ട്ടു­കള്‍ അവ കൈ­വ­യ്ക്കു­ന്ന വി­ഷ­യ­ങ്ങള്‍­ക്ക­നു­സ­രി­ച്ച് പല­പ്പോ­ഴും സമൂ­ഹ­ത്തില്‍ അകാ­ര­ണ­മായ ഭയ­ങ്ങ­ളും 
മുന്‍­വി­ധി­ക­ളും രൂ­പ­പ്പെ­ടു­ത്താ­നും കാ­ര­ണ­മാ­കു­ന്നു­.

­വി­ശ­ക­ല­ന­മെ­ന്ന­ത് വ്യ­ക്ത­മായ ചട്ട­ക്കൂ­ടു­ക­ളു­ള്ള ഒരു സങ്കേ­ത­മാ­ണെ­ന്ന­ത് മന­സ്സി­ലാ­ക്കാ­തി­രി­ക്കു­ക­യും, തന്റെ ഫീല്‍­ഡ് 
­റി­പ്പോര്‍­ട്ട് വി­ശ­ക­ല­ന­ത്തെ സഹാ­യി­ക്കാ­നു­ള്ള പല റി­സോ­ഴ്സു­ക­ളി­ലൊ­ന്നു മാ­ത്ര­മാ­ണെ­ന്നു തി­രി­ച്ച­റി­യാ­തി­രി­ക്കു­ക­യും 
ചെ­യ്യു­ന്ന­താ­യി­രി­ക്ക­ണം ഇത്ത­രം അപ­ക­ട­ങ്ങ­ളി­ലേ­ക്കെ­ത്തി­ക്കു­ന്ന­ത്. വസ്തു­താ­ധി­ഷ്ഠിത റി­പ്പോര്‍­ട്ടു­ക­ളില്‍ നി­ന്നും 
വ്യ­ക്ത­മായ അക­ലം വി­ശ­ക­ല­ന­ങ്ങള്‍­ക്കും അവ­ലോ­ക­ന­ങ്ങള്‍­ക്കു­മു­ണ്ടെ­ന്ന് തി­രി­ച്ച­റി­യേ­ണ്ട­ത് അവ­ശ്യ­മാ­ണ്. ലൈ­വാ­യി 
വാര്‍­ത്താ­വ­താ­ര­കന്‍ ചോ­ദ്യ­ങ്ങ­ളി­ലൂ­ടെ അവ­ശ്യ ഡാ­റ്റ­കള്‍ ശേ­ഖ­രി­ച്ചും പ്ര­ധാന പ്ര­തി­ക­ര­ണ­ങ്ങള്‍ പങ്കു­വെ­ച്ചും മറ്റും ന്യൂ­സ് 
റൂ­മി­നു­ള്ളില്‍ മി­നി­റ്റു­കള്‍­ക്കു­ള്ളില്‍ നി­ഗ­മ­ന­ങ്ങ­ളി­ലെ­ത്തു­ന്ന ഇക്കാ­ല­ത്ത് ഇതു പ്ര­ത്യേ­കം പ്ര­സ്താ­വ്യ­മാ­ണ്. 'ഇ­തു­വ­രെ അറി­വായ 
വി­വ­ര­ങ്ങള്‍ വച്ച്,' എന്ന് ഡി­സ്‌­ക്ലൈ­മര്‍ ചേര്‍­ക്കാന്‍ പോ­ലും പല­രും മടി­ക്കാ­റു­ണ്ടി­ന്ന്.‌

ആ­ദ്യം പത്ര­ങ്ങള്‍ ചില സാ­മ്പ്ര­ദാ­യിക നി­യ­മ­ങ്ങള്‍ ഒഴി­വാ­ക്കി­ക്കൊ­ണ്ട് സമൂ­ഹ­മ­ന­സ്സില്‍ അജ­ണ്ട­കള്‍ ഒളി­ച്ചു­ക­ട­ത്തു­ന്ന 
ഗീ­ബല്‍­സി­യന്‍ (ഹി­റ്റ്ല­റു­ടെ പ്ര­ചാ­ര­ണ­മ­ന്ത്രി­യാ­യി­രു­ന്ന ഗീ­ബല്‍­സാ­ണ് ഈ രീ­തി വള­രെ ഫല­പ്ര­ദ­മാ­യി പരീ­ക്ഷി­ച്ച­ത്) 
രീ­തി­യെ­ക്കു­റി­ച്ചും രണ്ടാ­മ­ത് വി­ശ­ക­ല­ന­മെ­ന്ന ചാ­രു­ക­സേല പ്ര­വര്‍­ത്ത­നം, ഫീല്‍­ഡ് റി­പ്പോര്‍­ട്ടി­ങ്ങി­നു കൊ­ടു­ക്കു­ന്ന 
അമി­ത­പ്രാ­ധാ­ന്യ­ത്തില്‍ വശ­ത്തേ­ക്കൊ­തു­ങ്ങി­പ്പോ­കു­ക­യും അതു­വ­ഴി പല അപ­ക­ട­ക­ര­മായ സാ­മാ­ന്യ­വ­ത്ക­ര­ണ­ങ്ങ­ളും 
നി­ഗ­മ­ന­ങ്ങ­ളാ­യി തെ­ളി­വോ­ടെ അച്ച­ടി­മ­ഷി­പു­ര­ളു­ക­യും ചെ­യ്യു­ന്ന­തി­നെ­ക്കു­റ­ച്ചു­മാ­ണ് പറ­ഞ്ഞ­ത്. ഇനി പറ­യാന്‍ പോ­കു­ന്ന­ത്, 
വാ­യ­ന­ക്കാ­രെ­ന്തു­വാ­യി­ക്ക­ണ­മെ­ന്നു തീ­രു­മാ­നി­ക്കു­ന്ന എഡി­റ്റോ­റി­യല്‍ ബോര്‍­ഡി­ന്റെ നീ­ല­പ്പെന്‍­സി­ലു­ക­ളു­ടെ 
(ക­ട­പ്പാ­ട്: തി­രു­ത്ത്, എം­.എ­സ്.­മാ­ധ­വന്‍) പക്ഷ­ഭേ­ദ­ത്തെ­പ്പ­റ്റി­യാ­ണ്.

ഏ­തു­ത­രം വാര്‍­ത്ത­കള്‍ തി­ര­സ്ക­രി­ക്ക­ണ­മെ­ന്ന­തി­ലോ അച്ച­ടി­മ­ഷി­പു­ര­ള­ണ­മെ­ന്ന­തി­ലോ പത്ര­ത്തി­ന് നയ­ങ്ങ­ളും 
കാ­ഴ്ച­പ്പാ­ടു­ക­ളും കാ­ണും. ഒരേ വി­ഷ­യ­ത്തില്‍ ചില പക്ഷ­ങ്ങ­ളു­ടെ വാ­ദ­ങ്ങള്‍­ക്ക് കൂ­ടു­തല്‍ പ്രാ­ധാ­ന്യം കൊ­ടു­ക്കു­ന്ന­തും 
സാ­ധാ­ര­ണം മാ­ത്ര­മാ­ണ്. പ്ര­തി­പ­ക്ഷ­സ്വ­ര­ങ്ങള്‍ വേ­റെ വല്ല­വ­രും കേള്‍­പ്പി­ച്ചോ­ളും. എന്നാല്‍, അച്ച­ടി­മ­ഷി­പു­ര­ളു­ന്ന 
വാര്‍­ത്ത­കള്‍ ഏതു­വി­ധ­മാ­യി­ക്കോ­ട്ടെ കു­റ്റ­മ­റ്റ­താ­യി­രി­ക്ക­ണ­മെ­ന്ന­ത് സാ­മാ­ന്യ­നി­യ­മം മാ­ത്ര­മാ­ണ്. പ്ര­ത്യേ­കി­ച്ചും 
പത്ര­ങ്ങ­ളു­ടെ തന്നെ ഭാ­ഷ­യില്‍ അവ­രു­ടെ വാ­യ­ന­ക്കാ­രില്‍ പല­രും ഈ വി­ഷ­യ­ങ്ങ­ളില്‍ 
അഗാ­ധ­പാ­ണ്ഡി­ത്യ­മി­ല്ലാ­ത്ത­വ­രോ, പത്രം പറ­ഞ്ഞ­തു­കൊ­ണ്ട് ഇത് ശരി­യാ­യി­രി­ക്കു­മെ­ന്നു കരു­തു­ന്ന­വ­രു­മാ­യ­തു­കൊ­ണ്ട്.

ഇ­ത്ത­ര­ത്തില്‍ ഇന്ത്യന്‍ മാ­ദ്ധ്യ­മ­ങ്ങ­ളില്‍ എഡി­റ്റര്‍­മാ­രു­ടെ കസേ­ര­ക­ളി­ലി­രി­ക്കു­ന്ന­വര്‍ തങ്ങ­ളു­ടെ ജോ­ലി കൃ­ത്യ­മാ­യി 
ചെ­യ്യാ­ത്ത­തു­കാ­ര­ണം പല വാര്‍­ത്ത­ക­ളും വാര്‍­ത്ത­ക­ളു­ടെ തലം വി­ട്ട് മാ­ജി­ക്കല്‍ റി­യ­ലി­സ­ത്തി­ന്റെ തല­ത്തി­ലെ­ത്താ­റു­ണ്ട്. 
ടെ­ക്നോ­ള­ജി സം­ബ­ന്ധ­മായ വാര്‍­ത്ത­ക­ളും വി­ല­യി­രു­ത്ത­ലു­ക­ളു­മാ­ണ് ഇവ­യില്‍ പ്ര­ധാ­നം. പല­പ്പോ­ഴും കൌ­തു­ക­വാര്‍­ത്ത­ക­ളും 
ചില സ്പോര്‍­ട്സ് വാര്‍­ത്ത­ക­ളും ഇത്ത­ര­ത്തി­ലാ­വാ­റു­ണ്ട്. ഇതില്‍ രസ­ക­ര­മായ കാ­ര്യം, പല­പ്പോ­ഴും 
സം­ഭ­വി­ക്കു­ന്ന­തെ­ന്തെ­ന്നാല്‍ ഈ പ്രോ­ഡ­ക്റ്റ് അവ­ത­രി­പ്പി­ച്ച­പ്പോ­ഴോ അല്ലെ­ങ്കില്‍ അതി­ന്റെ അവ­ലോ­ക­ന­ത്തി­നാ­യി 
റി­പ്പോര്‍­ട്ടര്‍­മാര്‍ സമീ­പി­ച്ച­പ്പോ­ഴോ കമ്പ­നി­കള്‍ ഊന്നല്‍ കൊ­ടു­ത്ത കാ­ര്യ­ങ്ങള്‍­ക്കു പക­രം മറ്റു­പ­ല­തു­മാ­യി­രി­ക്കും 
റി­പ്പോര്‍­ട്ടര്‍­മാര്‍ മന­സ്സി­ലാ­ക്കു­ന്ന­ത്.

­നേ­രി­ട്ടു­ക­ണ്ട് മന­സ്സി­ലാ­ക്കിയ ഒര­നു­ഭ­വം പറ­യ­ട്ടെ. NVIDIA Tesla എന്ന പേ­രില്‍ CUDA അടി­സ്ഥാ­ന­മാ­ക്കി ഒരു 
സൂ­പ്പര്‍ കമ്പ്യൂ­ട്ടി­ങ് കഴി­വു­ക­ളു­ള്ള പ്ലാ­റ്റ്ഫോം വി­ക­സി­പ്പി­ച്ചി­രു­ന്നു. NVIDIA സി­.ഇ­.ഓ.­യും സ്ഥാ­പ­ക­നു­മായ ജെന്‍ സുന്‍ 
ഹ്യ­യാ­ങ് 2008 നവം­ബ­റില്‍ ഇന്ത്യ സന്ദര്‍­ശി­ച്ച­പ്പോള്‍ IIIT, Hyderabadല്‍ വച്ചാ­ണ് അത് (ഇ­ന്ത്യ­യില്‍) 
അവ­ത­രി­പ്പി­ച്ച­ത്. ഈ ഡിഗ്ഗ് 
ലി­ങ്ക്\footnote{http://digg.com/news/story/NDTV_kills_nvidia_tesla_with_stupid_reporting}
കണ്ടാല്‍ മന­സ്സി­ലാ­വും എന്‍­ഡി­ടി­വി­യു­ടെ റി­പ്പോര്‍­ട്ടര്‍ ഇക്കാ­ര്യം 
മന­സ്സി­ലാ­ക്കി­യ­തെ­ങ്ങ­നെ­യാ­ണെ­ന്ന്. അതി­നു താ­ഴെ കമ­ന്റു­ക­ളില്‍ ഹി­ന്ദു­വി­ന്റെ കവ­റേ­ജും കൊ­ടു­ത്തി­ട്ടു­ണ്ട്.

­വ­സ്തു­താ­പ­ര­മായ പി­ഴ­വു­കള്‍­മു­തല്‍, ­ടെ­ക്നോ­ള­ജി­ റി­പ്പോര്‍­ട്ട് ചെ­യ്യു­ന്ന നമ്മു­ടെ മല­യാ­ളം പത്ര­ങ്ങ­ളി­ലെ യു­വ­ര­ക്തം 
പി­ന്തു­ട­രു­ന്ന "പാ­തി­വെ­ന്ത മന­സ്സി­ലാ­ക്ക­ലു­ക­ളെ പഞ്ച­സാ­ര­പൊ­തി­ഞ്ഞ് അവ­ത­രി­പ്പി­ക്കു­ന്ന" പരി­പാ­ടി­യും കൂ­ടി­യാ­യ­പ്പോള്‍ 
ചു­ക്ക്, ചു­ണ്ണാ­മ്പി­നു­മ­പ്പു­റം വേ­റെ­യെ­ന്തൊ­ക്കെ­യോ ആയി. റി­പ്പോര്‍­ട്ട് ചെ­യ്യാന്‍ വന്ന കൊ­ച്ചി­നു NVIDIA CEO­യു­ടെ 
അമേ­രി­ക്കന്‍ ഉച്ചാ­ര­ണം മന­സ്സി­ലാ­കാ­ഞ്ഞ­തോ, വി­ഷ­യ­പ­രി­ജ്ഞാ­നം കമ്മി­യാ­യ­തോ, എഴു­തി­യെ­ടു­ത്ത­ത് പി­ന്നെ 
വാ­യി­ച്ച­പ്പോള്‍ തല­തി­രി­ഞ്ഞു­പോ­യ­തോ ഒക്കെ­യാ­കാം. എങ്കി­ലും മി­നി­മം NVIDIA­യു­ടെ വെ­ബ്സൈ­റ്റില്‍ പോ­യി Tesla 
എന്ന പ്രോ­ഡ­ക്റ്റി­നു കീ­ഴില്‍ എഴു­തി­യ­തൊ­ക്കെ­ത്ത­ന്നെ­യാ­ണോ തന്റെ റി­പ്പോര്‍­ട്ട­റും എഴു­തി­യ­ത് എന്ന് നോ­ക്കാ­നെ­ങ്കി­ലും 
തോ­ന്നു­ന്ന ഒരു എ­ഡി­റ്റര്‍ NDTV­യു­ടെ ടെ­ക്നോ­ള­ജി ഡെ­സ്കില്‍ ഉണ്ടാ­യി­രു­ന്നെ­ങ്കില്‍ ഇത്ര­മാ­ത്രം 
നാ­ണ­ക്കേ­ടു­ണ്ടാ­കി­ല്ലാ­യി­രു­ന്നു. അതി­നു­ശേ­ഷ­വും മുന്‍­പും ഞാന്‍ NDTV­യു­ടെ ടെ­ക്നോ­ള­ജി വാര്‍­ത്ത­ക­ളൊ­ന്നും വാ­യി­ക്കാ­റി­ല്ല, 
ആ ­വാര്‍­ത്ത ഇപ്പോ­ഴും ആ തെ­റ്റു­ക­ളോ­ടെ­ല്ലാം കൂ­ടി അവി­ടെ­ത്ത­ന്നെ കി­ട­ക്കു­ന്ന­തു­കൊ­ണ്ട്, അവ­രു­ടെ നയ­ങ്ങ­ളൊ­ന്നും 
മാ­റി­യി­ട്ടി­ല്ലെ­ന്നു കരു­തു­ന്നു­.

ഈ രീ­തി­യി­ലു­ള്ള റി­പ്പോര്‍­ട്ടു­കള്‍ പല­പ്പോ­ഴും അന്താ­രാ­ഷ്ട്ര­ത­ല­ത്തില്‍ നമു­ക്ക് മാ­ന­ക്കേ­ടു­മാ­ത്ര­മാ­ണു­ണ്ടാ­ക്കാ­റ്. ഇന്ത്യ­യി­ലെ 
റി­പ്പോര്‍­ട്ടര്‍­മാ­രു­ടെ അത്യു­ത്സാ­ഹം കാ­ര­ണം, പത്ര­ങ്ങ­ളില്‍ റി­പ്പോര്‍­ട്ട് ചെ­യ്യ­പ്പെ­ടു­ന്ന തദ്ദേ­ശ­പ­ര­മാ­യി വി­ക­സി­പ്പി­ച്ച നൂ­തന 
വി­ദ്യ­ക­ളെ മൂ­ന്നു­പ്രാ­വ­ശ്യം ഇരു­ത്തി­വാ­യി­ക്കു­ക­യും നാ­ലാ­ളോ­ടു ചോ­ദി­ച്ചു ഉറ­പ്പു­വ­രു­ത്തി­യും മാ­ത്ര­മേ വി­ശ്വ­സി­ക്കാ­വൂ 
എന്ന­നി­ല­യാ­ണ്. ഇത്ത­ര­ത്തില്‍ ചു­ക്കും ചു­ണ്ണാ­മ്പും തി­രി­ച്ച­റി­യാ­ത്ത റി­പ്പോര്‍­ട്ടു­കള്‍ പത്ര/­ടെ­ലി­വി­ഷന്‍ മുന്‍­നി­ര­ക്കാ­രു­ടെ 
പോര്‍­ട്ട­ലു­ക­ളില്‍ മാ­ത്ര­മ­ല്ല, താ­ര­മേ­ന്യ പു­തിയ ന്യൂ­സ് പോര്‍­ട്ട­ലു­ക­ളി­ലും കാ­ണാ­റു­ണ്ട്. പക്ഷേ അവ­രു­ടെ ഒരു ഗു­ണം, തെ­റ്റ് 
ചൂ­ണ്ടി­ക്കാ­ണി­ച്ചു കൊ­ടു­ത്താല്‍ ക്ഷമ ചോ­ദി­ക്കാ­നും തി­രു­ത്താ­നും തയ്യാ­റാ­കു­മെ­ന്ന­താ­ണ്. NDTV­യെ ഇവി­ടെ­യൊ­രു 
സാ­മ്പി­ളാ­യി മാ­ത്രം കാ­ണി­ച്ച­താ­ണ്. ഇത്ത­രം തല­തി­രി­ഞ്ഞ റി­പ്പോര്‍­ട്ടി­ങ് എല്ലാ ഇന്ത്യന്‍ മാ­ദ്ധ്യ­മ­ങ്ങ­ളി­ലും ഏതാ­ണ്ട് ഒരേ 
അള­വില്‍­ത്ത­ന്നെ കണ്ടി­ട്ടു­ണ്ട്.

ഈ പ്ര­ശ്ന­ങ്ങ­ളൊ­ക്കെ ശക്ത­മാ­യ, അല്ലെ­ങ്കില്‍ ലോ­ജി­ക്ക­ലാ­യി ചി­ന്തി­ക്കു­ക­യെ­ങ്കി­ലും ചെ­യ്യു­ന്ന ഒരു എഡി­റ്റോ­റി­യല്‍ 
സം­ഘ­വും റി­പ്പോര്‍­ട്ടര്‍­മാ­രു­മി­ല്ലാ­ത്ത­തി­ന്റേ­താ­ണെ­ങ്കില്‍, തീര്‍­ത്തും വ്യ­ത്യ­സ്ത­മായ പൂര്‍­ണ്ണ അവ­ഗ­ണ­ന­യു­ടെ കണ­ക്കു­ക­ളും 
പല­പ്പോ­ഴും പത്ര­ങ്ങ­ളില്‍ കാ­ണാ­റു­ണ്ട്. നി­രു­ത്ത­വാ­ദ­പ­ര­മായ കാ­റോ­ട്ട­മ­ത്സ­ര­ങ്ങ­ളു­ടെ റി­പ്പോര്‍­ട്ടി­ങ് ഒരു­ദാ­ഹ­ര­ണം. അതു 
ചൂ­ണ്ടി­ക്കാ­ണി­ച്ചാല്‍ പല­പ്പോ­ഴും കാ­ര­ണ­ങ്ങ­ളാ­യി പറ­യു­ന്ന­ത്, വേ­ണ്ട­ത്ര വാ­യ­ന­ക്കാ­രി­ല്ലാ­ത്ത­തു­കൊ­ണ്ടാ­ണെ­ന്നാ­ണ് 
(ഈ­യ­ടു­ത്ത്, അമൃ­ത­യി­ലോ മറ്റോ ഒരു വാ­രാ­ന്ത്യ സ്പോര്‍­ട്സ് റൌ­ണ്ട­പ്പി­ലോ മറ്റോ തര­ക്കേ­ടി­ല്ലാ­തെ ഗ്രാന്‍­പ്രീ­കള്‍ 
റി­പ്പോര്‍­ട്ട് ചെ­യ്തു­ക­ണ്ടു­).

എ­ല്ലാ പ്ര­മുഖ മല­യാ­ളം പത്ര­ങ്ങ­ളി­ലും നല്ല വാ­ഹ­ന­റി­വ്യൂ­ക­ളും, അനു­ബ­ന്ധ­വാര്‍­ത്ത­ക­ളും കാ­ണാം. മാ­ത്ര­മ­ല്ല, വള­രെ­ക്കു­റ­ച്ചു 
തെ­റ്റു­കള്‍ മാ­ത്ര­മേ, ധാ­രാ­ളം സാ­ങ്കേ­തിക വി­വ­ര­ങ്ങ­ളെ പരാ­മര്‍­ശി­ച്ചു­കൊ­ണ്ടെ­ഴു­തു­ന്ന ഈ റി­പ്പോര്‍­ട്ടു­ക­ളില്‍ കാ­ണാ­റു­ള്ളു. 
വി­വ­ര­മു­ള്ള റി­പ്പോര്‍­ട്ടര്‍­മാ­രു­ടേ­യും എഡി­റ്റര്‍­മാ­രു­ടേ­യും സാ­ന്നി­ധ്യ­മാ­യി­രി­ക്കാം കാ­ര­ണം. ഇത്ര­യും നല്ല റി­പ്പോര്‍­ട്ടു­കള്‍ 
പ്ര­സി­ദ്ധീ­ക­രി­ക്കാ­നാ­വു­ന്നു­ണ്ടെ­ങ്കില്‍ അവര്‍­ക്ക് സാ­ധാ­ര­ണ­ഗ­തി­യില്‍ തെ­റ്റു­ക­ളി­ല്ലാ­തെ റേ­സ് റി­പ്പോര്‍­ട്ടു­ക­ളും എഴു­താന്‍ 
കഴി­യേ­ണ്ട­താ­ണ് (തീര്‍­ച്ച­യാ­യും വസ്തു­താ­പ­ര­മായ പി­ഴ­വു­ക­ളെ ഒഴി­വാ­ക്കാ­നാ­വും­). സ്പോര്‍­ട്സ് സെ­ക്ഷ­നി­ലെ ന്യൂ­സ് എന്ന 
നി­ല­യില്‍ കൈ­കാ­ര്യം ചെ­യ്യാന്‍ നില്‍­ക്കാ­തെ, ഇത്ത­രം ന്യൂ­സു­ക­ളില്‍ ഇന്റര്‍­നെ­റ്റി­ന്റേ­യോ, പത്ര­ത്തില്‍­ത്ത­ന്നെ­യു­ള്ള 
ഓട്ടോ­മോ­ട്ടീ­വ് സെ­ക്ഷ­ന്റേ­യോ സഹാ­യം വെ­രി­ഫി­ക്കേ­ഷ­നു വേ­ണ്ടി­യെ­ങ്കി­ലും ഉപ­യോ­ഗി­ച്ചാല്‍­ത്ത­ന്നെ, പി­ഴ­വു­കള്‍ 
ഒഴി­വാ­ക്കാ­നാ­വും­.

ഇ­ത്ത­രം കാ­ര്യ­ങ്ങള്‍ പത്ര­പ്ര­വര്‍­ത്ത­നം ബി­രു­ദ/­ഡി­പ്ലോമ കോ­ഴ്സു­ക­ളാ­യി പഠി­പ്പി­ക്കു­ന്ന­വര്‍ അവ­രു­ടെ സി­ല­ബ­സ്സില്‍ 
ഉള്‍­പ്പെ­ടു­ത്തി­യി­ട്ടു­ണ്ടോ എന്ന­റി­യി­ല്ല. ഇല്ലെ­ങ്കില്‍ ഇത്ത­രം കാ­ര്യ­ങ്ങള്‍ കൂ­ടി വാര്‍­ത്ത­കള്‍ എഴു­താ­നും തി­രു­ത്താ­നും 
പഠി­പ്പി­ക്കു­ന്ന കൂ­ട്ട­ത്തില്‍ പഠി­പ്പി­ച്ചാല്‍ നന്നാ­യി­രി­ക്കും. കാ­ണാ­പ്പാ­ഠം പഠി­ച്ച് പരീ­ക്ഷ പാ­സാ­യി പത്ര­പ്ര­വര്‍­ത്ത­ക­രാ­കു­ന്ന­വര്‍
തങ്ങള്‍­ക്ക് അജ്ഞാ­ത­മായ വി­ഷ­യ­ങ്ങ­ളില്‍ ചെ­റി­യൊ­രു പഠ­ന­മെ­ങ്കി­ലും കൂ­ടാ­തെ ആധി­കാ­രിക റി­പ്പോര്‍­ട്ടു­കള്‍ 
എഴു­തി­വി­ടു­ന്ന­ത് കു­റ­യാ­നും, ഏതു വി­ഷ­യ­വും എഡി­റ്റ് ചെ­യ്യു­ന്ന­തില്‍ ഡെ­സ്ക് ജോ­ലി­ക്കാര്‍ കൂ­ടു­തല്‍ ശ്ര­ദ്ധ­വ­യ്ക്കു­ന്ന­തി­നും 
ഇത്ത­രം കു­റി­പ്പു­ക­ളെ­ങ്കി­ലും സഹാ­യ­ക­മാ­വു­മെ­ന്നു കരു­തു­ന്നു­.

­വ­സ്തു­താ­പ­ര­മായ ഒരു പി­ഴ­വ്, ശക്ത­മായ വി­ഷ­യ­ങ്ങള്‍ കൈ­കാ­ര്യം ചെ­യ്യു­ന്ന റി­പ്പോര്‍­ട്ടി­ന്റെ മു­ഴു­വന്‍ 'ഇ­ന്റ­ഗ്രി­റ്റി­'­യേ­യും 
സം­ശ­യ­ത്തി­ന്റെ നി­ഴ­ലി­ലാ­ക്കു­മെ­ന്നു­ള്ള മന­സ്സി­ലാ­ക്ക­ലെ­ങ്കി­ലും ഉണ്ടെ­ങ്കില്‍ പകു­തി കാ­ര്യ­ങ്ങള്‍ ശരി­യാ­വു­മെ­ന്നു തോ­ന്നു­ന്നു. 
മി­ക­ച്ച പത്ര­പ്ര­വര്‍­ത്ത­ന­ത്തി­നു­ള്ള അവാര്‍­ഡ് നേ­ടു­ന്ന­വര്‍­ക്കു­പോ­ലും പത്ര­പ്ര­വര്‍­ത്ത­ന­ത്തി­ന് മി­ക­ച്ച ഭാ­ഷ­യു­ടെ­യും 
ഘട­ന­യു­ടെ­യു­മ­പ്പു­റ­ത്ത്, വേ­റെ­യും തല­ങ്ങ­ളു­ണ്ടെ­ന്നു­ള്ള തി­രി­ച്ച­റി­വി­ല്ലെ­ന്ന­തി­ന് അത്ര പഴ­യ­ത­ല്ലാ­ത്ത ചില 
പത്ര­വാര്‍­ത്ത­കള്‍ സാ­ക്ഷി­ക­ളാ­ണ്.

(5 August 2010)\footnote{http://malayal.am/വാര്‍ത്ത/മീഡിയ-സ്കാന്‍/7249/അപ്രത്യക്ഷമാകുന്ന-എഡിറ്റോറിയല്‍-ഡെസ്ക്}

\newpage
