\secstar{അപ്രത്യക്ഷമാകുന്ന എഡിറ്റോറിയല്‍ ഡെസ്ക്}
\vskip 2pt

ജനാധിപത്യത്തിന്റെ നിലനില്‍പ്പിനും മനുഷ്യാവകാശങ്ങളുടെ പരിരക്ഷയ്ക്കും, സ്വതന്ത്രവും നിഷ്പക്ഷവുമായ 
മാദ്ധ്യമങ്ങളുണ്ടാവേണ്ടതു് അത്യന്താപേക്ഷിതമാണു്. 'സ്വതന്ത്രവും നിഷ്പക്ഷവുമായ മാദ്ധ്യമങ്ങ' എന്നതിലെ 
സ്വാതന്ത്ര്യമെന്ന ഭാഗത്തിനു് കൂടുതല്‍ ഊന്നല്‍കൊടുക്കുകയും, നിഷ്പക്ഷത എന്നതു് പലപ്പോഴും ഒരു ജലരേഖയാവുകയും 
ചെയ്യുന്നതു് ഇന്നത്തെ മാദ്ധ്യമലോകത്തു് സാധാരണമാണു്. പ്രത്യക്ഷ അജണ്ടകളോടെയോ വ്യക്തമായ 
ചായ്‌വുകളോടെയോ രാഷ്ട്രീയ/മത/സാമൂഹ്യ സംഘടനകളുടെ ജിഹ്വകളായി ധാരാളം മാദ്ധ്യമങ്ങള്‍ പ്രവര്‍ത്തിക്കുന്നു. 
നിഷ്പക്ഷപ്രവര്‍ത്തനമെന്നതിനേക്കാളും മുഖ്യധാരയില്‍ പിന്തള്ളപ്പെട്ടുപോകുന്ന കാഴ്ചപ്പാടുകളെ പൊതുസമൂഹത്തില്‍ 
ചര്‍ച്ചയ്ക്കുവയ്ക്കുക എന്നതാണു് ഇവരുടെ പ്രധാന അജണ്ട.

വിദ്യാഭ്യാസപരമായി മുന്നോക്കംനില്‍ക്കുന്ന സമൂഹങ്ങളില്‍ പൊതുജനാഭിപ്രായരൂപീകരണത്തിനു് മാദ്ധ്യമങ്ങള്‍ക്കുള്ള 
സ്വാധീനം അളക്കാനാവാത്തതാണു്. അതുമൂലം ഭരണത്തിന്റെ ചക്രം തിരിക്കുന്നവര്‍ക്കു് പലപ്പോഴും മാദ്ധ്യമങ്ങളെ 
പ്രീതിപ്പെടുത്തേണ്ടതു് ഒരു ആവശ്യമാകുന്നു. ഇത്തരത്തില്‍ സ്വന്തംമാദ്ധ്യമങ്ങള്‍ ആരംഭിക്കാന്‍ കഴിവുള്ള പണമുള്ളവരുടേയും 
അധികാരമുള്ളവരുടേയും മാത്രം സ്വരങ്ങള്‍വഴി പൊതുജനാഭിപ്രായരൂപീകരണം നടത്തപ്പെടുന്നതു് തടയാനാണു് 
മാദ്ധ്യമങ്ങളും മാദ്ധ്യമപ്രവര്‍ത്തകരും സ്വയം ഒരു പെരുമാറ്റച്ചട്ടം രൂപീകരിക്കണമെന്നു പറയുന്നതു്.

വാര്‍ത്തകള്‍ വസ്തുതാടിസ്ഥാനത്തിലുള്ള വിവരണങ്ങള്‍ മാത്രമാവുകയും, മാദ്ധ്യമ അജണ്ടകള്‍ 
വാര്‍ത്തകളോടുനുബന്ധിച്ചുള്ള അവലോകനങ്ങളോ വിശകലങ്ങളോ നിരീക്ഷണങ്ങളോ അഭിമുഖങ്ങളോ വഴി 
രേഖപ്പെടുത്തുകയും ചെയ്യുക എന്നതാണു് സാമ്പ്രദായികമായി അംഗീകരിച്ചിട്ടുള്ള രീതി. സ്കൂപ്പുകളിലോ 
വെളിപ്പെടുത്തലുകളിലോ മാദ്ധ്യമങ്ങളുടെ നിഗമനങ്ങള്‍ സ്ഥാനംപിടിക്കുന്നുണ്ടെങ്കില്‍ അതിനെ നിഗമനങ്ങളായിത്തന്നെ 
കാണിക്കുന്നതും പതിവാണു്. മാത്രമല്ല, പ്രസിദ്ധീകരിക്കുന്ന ഏതൊരു വാര്‍ത്തയ്ക്കും (എന്തിനും) നിയമപരമായും 
ധാര്‍മ്മികപരമായും മാദ്ധ്യമസ്ഥാപനങ്ങള്‍ ഉത്തരവാദിയുമാണു്. (അവനവന്‍ പ്രസാധകനാവുന്ന ബ്ലോഗുകള്‍ക്കും 
പോര്‍ട്ടലുകള്‍ക്കും ഇവ ബാധകമാണു്.)

ന്യൂസുകളിലൂടെ പ്രത്യേക അജണ്ടകള്‍ക്കു് പ്രചരണം കൊടുക്കാന്‍ മാദ്ധ്യമങ്ങള്‍ സ്വീകരിക്കുന്ന എളുപ്പവഴി ഈ 
അതിര്‍വരമ്പുകളെ ഒഴിവാക്കുകയാണു്. പലരും ഒരുപടികൂടി കടന്നു് വാര്‍ത്തകള്‍തന്നെ സൃഷ്ടിക്കുകയും ചെയ്യുന്നു. 
വേണമെന്നുവച്ചു് വാര്‍ത്തകള്‍ വളച്ചൊടിക്കുന്നവരുടെ കഥകളാണിതു്.

വിശകലനവും അവലോകനവും പലപ്പോഴും പത്രങ്ങളില്‍ പ്രധാനസ്ഥാനം നേടാറുണ്ടു്. പലപ്പോഴും അന്വേഷണാത്മക 
പരമ്പരകള്‍ എഴുതപ്പെടുന്നതു് വിശകലനങ്ങളായിട്ടായിരിക്കും. ഇവിടെയാണു് വിവരമുള്ള പത്രപ്രവര്‍ത്തകര്‍ ഇന്നു് 
കുറഞ്ഞുവരികയാണെന്നുള്ളതിന്റെ സൂചനകള്‍ കാണാവുന്നത്. ഗംഭീരമായി ഫീല്‍ഡ് റിപ്പോര്‍ട്ടിങ് ചെയ്യാന്‍ 
കഴിവുള്ളവര്‍ പലപ്പോഴും വിശകലനത്തിലും അതിനോടനുബന്ധിച്ച ചില സാമാന്യനിയമങ്ങളിലും അജ്ഞരായിരിക്കും. 
അതു് അപൂര്‍ണ്ണവും അപക്വവുമായ നിഗമനങ്ങളിലായിരിക്കും പലപ്പോഴും എത്തിക്കുന്നതു്.

വിശകലനത്തിനുവേണ്ട പ്രാഥമികവിവരങ്ങള്‍ ശേഖരിക്കുന്നതില്‍ തുടങ്ങി, എറര്‍ മാര്‍ജിന്‍ എന്ന വാക്കുപോലും 
കേള്‍ക്കാത്തവര്‍ സ്റ്റാറ്റിസ്റ്റിക്കല്‍ അനാലിസിസ് നടത്തിയാല്‍ വരുന്ന കുറവുകളും, ഏതൊക്കെ പാരാമീറ്ററുകള്‍ 
മാറുന്നതുകൊണ്ടാണു് വ്യത്യാസങ്ങള്‍ കാണുന്നതെന്ന കാര്യത്തില്‍ മുന്‍വിധികള്‍ നിഗമനങ്ങളെ ബാധിക്കുന്നതും വരെ 
അപക്വമായ സാമാന്യവത്കരണത്തിനു് (immature generalization) കാരണമാകാറുണ്ടു്. ഇത്തരം പാതിവെന്ത 
റിപ്പോര്‍ട്ടുകള്‍ അവ കൈവയ്ക്കുന്ന വിഷയങ്ങള്‍ക്കനുസരിച്ചു് പലപ്പോഴും സമൂഹത്തില്‍ അകാരണമായ ഭയങ്ങളും 
മുന്‍വിധികളും രൂപപ്പെടുത്താനും കാരണമാകുന്നു.

വിശകലനമെന്നതു് വ്യക്തമായ ചട്ടക്കൂടുകളുള്ള ഒരു സങ്കേതമാണെന്നു് മനസ്സിലാക്കാതിരിക്കുകയും, ഫീല്‍ഡ് 
റിപ്പോര്‍ട്ടു് വിശകലനത്തെ സഹായിക്കാനുള്ള പല റിസോഴ്സുകളിലൊന്നു മാത്രമാണെന്നു തിരിച്ചറിയാതിരിക്കുകയും 
ചെയ്യുന്നതായിരിക്കണം ഇത്തരം അപകടങ്ങളിലേക്കെത്തിക്കുന്നതു്. വസ്തുതാധിഷ്ഠിത റിപ്പോര്‍ട്ടുകളിനിന്നും 
വ്യക്തമായ അകലം വിശകലനങ്ങള്‍ക്കും അവലോകനങ്ങള്‍ക്കുമുണ്ടെന്നു് തിരിച്ചറിയേണ്ടതു് അവശ്യമാണു്. ലൈവായി 
വാര്‍ത്താവതാരകന്‍ ചോദ്യങ്ങളിലൂടെ അവശ്യ ഡാറ്റകള്‍ ശേഖരിച്ചും പ്രധാന പ്രതികരണങ്ങള്‍ പങ്കുവെച്ചും മറ്റും ന്യൂസ് 
റൂമിനുള്ളില്‍ മിനിറ്റുകള്‍ക്കുള്ളില്‍ നിഗമനങ്ങളിലെത്തുന്ന ഇക്കാലത്തു് ഇതു് പ്രത്യേകം പ്രസ്താവ്യമാണു്. 'ഇതുവരെ അറിവായ 
വിവരങ്ങള്‍വച്ചു്' എന്നു് ഡിസ്‌ക്ലൈമര്‍ ചേര്‍ക്കാന്‍പോലും പലരും മടിക്കാറുണ്ടിന്നു്.‌

പത്രങ്ങള്‍ ചില സാമ്പ്രദായികനിയമങ്ങള്‍ ഒഴിവാക്കിക്കൊണ്ടു് സമൂഹമനസ്സില്‍ അജണ്ടകള്‍ ഒളിച്ചുകടത്തുന്ന 
ഗീബല്‍സിയന്‍ (ഹിറ്റ്ലറുടെ പ്രചാരണമന്ത്രിയായിരുന്ന ഗീബല്‍സാണു് ഈ രീതി വളരെ ഫലപ്രദമായി പരീക്ഷിച്ചതു്) 
രീതിയെക്കുറിച്ചും, വിശകലനമെന്ന ചാരുകസേല പ്രവര്‍ത്തനം ഫീല്‍ഡ് റിപ്പോര്‍ട്ടിങ്ങിനു കൊടുക്കുന്ന 
അമിതപ്രാധാന്യത്തില്‍ വശത്തേക്കൊതുങ്ങിപ്പോകുകവഴി അപകടകരമായ സാമാന്യവത്കരണങ്ങള്‍ 
നിഗമനങ്ങളായി അച്ചടിമഷിപുരളുകയും ചെയ്യുന്നതിനെക്കുറിച്ചുമാണു് പറഞ്ഞതു്. ഇനി പറയാന്‍ പോകുന്നതു്, 
വായനക്കാര്‍എന്തുവായിക്കണമെന്നു തീരുമാനിക്കുന്ന എഡിറ്റോറിയല്‍ ബോര്‍ഡിന്റെ നീലപ്പെന്‍സിലുകളുടെ 
(കടപ്പാടു്: തിരുത്തു്, എം.എസ്. മാധവന്‍) പക്ഷഭേദത്തെപ്പറ്റിയാണു്.

ഏതുതരം വാര്‍ത്തകള്‍ തിരസ്ക്കരിക്കണമെന്നതിലോ അച്ചടിമഷിപുരളണമെന്നതിലോ പത്രത്തിനു് നയങ്ങളും 
കാഴ്ചപ്പാടുകളും കാണും. ഒരേവിഷയത്തില്‍ ചില പക്ഷങ്ങളുടെ വാദങ്ങള്‍ക്കു് കൂടുതല്‍ പ്രാധാന്യം കൊടുക്കുന്നതും 
സാധാരണം മാത്രമാണു്. പ്രതിപക്ഷസ്വരങ്ങള്‍ വേറെവല്ലവരും കേള്‍പ്പിച്ചോളും. എന്നാല്‍ അച്ചടിമഷിപുരളുന്ന 
വാര്‍ത്തകള്‍ ഏതുവിധമായിക്കോട്ടെ കുറ്റമറ്റതായിരിക്കണമെന്നതു് സാമാന്യനിയമം മാത്രമാണു്. പ്രത്യേകിച്ചും 
പത്രങ്ങളുടെതന്നെ ഭാഷയില്‍ അവരുടെ വായനക്കാരില്‍ പലരും ഈ വിഷയങ്ങളില്‍ 
അഗാധപാണ്ഡിത്യമില്ലാത്തവരോ, പത്രം പറഞ്ഞതുകൊണ്ടു് ശരിയായിരിക്കുമെന്നു കരുതുന്നവരോ ആയതുകൊണ്ടു്.

ഇത്തരത്തില്‍ ഇന്ത്യന്‍ മാദ്ധ്യമങ്ങളില്‍ എഡിറ്റര്‍മാരുടെ കസേരകളിലിരിക്കുന്നവര്‍ തങ്ങളുടെ ജോലി കൃത്യമായി 
ചെയ്യാത്തതുകാരണം പല വാര്‍ത്തകളും വാര്‍ത്തകളുടെ തലംവിട്ടു് മാജിക്കല്‍ റിയലിസത്തിന്റെ തലത്തിലെത്താറുണ്ടു്. 
ടെക്നോളജി സംബന്ധമായ വാര്‍ത്തകളും വിലയിരുത്തലുകളുമാണു് ഇവയില്‍ പ്രധാനം. പലപ്പോഴും കൗതുകവാര്‍ത്തകളും 
ചില സ്പോര്‍ട്സ് വാര്‍ത്തകളും ഇത്തരത്തിലാവാറുണ്ടു്. പലപ്പോഴും സംഭവിക്കുന്ന രസകരമായ കാര്യം,  
ഒരു പ്രോഡക്റ്റ് അവതരിപ്പിച്ചപ്പോഴോ അല്ലെങ്കില്‍ അതിന്റെ അവലോകനത്തിനായി 
റിപ്പോര്‍ട്ടര്‍മാര്‍ സമീപിച്ചപ്പോഴോ കമ്പനികള്‍ ഊന്നല്‍കൊടുത്ത കാര്യങ്ങള്‍ക്കുപകരം മറ്റുപലതുമായിരിക്കും 
റിപ്പോര്‍ട്ടര്‍മാര്‍ മനസ്സിലാക്കുന്നതു് എന്നുള്ളതാണു്.

നേരിട്ടുകണ്ടു് മനസ്സിലാക്കിയ ഒരനുഭവം പറയട്ടെ.  Tesla എന്ന പേരില്‍ CUDA അടിസ്ഥാനമാക്കി 
സൂപ്പര്‍ കമ്പ്യൂട്ടിങ് കഴിവുകളുള്ള ഒരു പ്ലാറ്റ്ഫോം NVIDIA വികസിപ്പിച്ചിരുന്നു. NVIDIA സി.ഇ.ഒ.യും സ്ഥാപകനുമായ ജെന്‍ സുന്‍ 
ഹ്യയാങ് 2008 നവംബറില്‍ ഇന്ത്യ സന്ദര്‍ശിച്ചപ്പോള്‍ IIIT, Hyderabadല്‍ വച്ചാണു് അത് ഇന്ത്യയി 
അവതരിപ്പിച്ചത്. ഈ ഡിഗ്ഗ് 
ലിങ്ക്\footnote{\url{http://digg.com/news/story/NDTV_kills_nvidia_tesla_with_stupid_reporting}}
കണ്ടാല്‍ മനസ്സിലാവും എന്‍ഡിടിവിയുടെ റിപ്പോര്‍ട്ടര്‍ ഇക്കാര്യം 
മനസ്സിലാക്കിയതെങ്ങനെയാണെന്നു്. അതിനുതാഴെ കമന്റുകളില്‍ ഹിന്ദുവിന്റെ കവറേജും കൊടുത്തിട്ടുണ്ടു്.

വസ്തുതാപരമായ പിഴവുകള്‍മുതല്‍, ടെക്നോളജി റിപ്പോര്‍ട്ടുചെയ്യുന്ന നമ്മുടെ മലയാളം പത്രങ്ങളിലെ യുവരക്തം 
പിന്തുടരുന്ന 'പാതിവെന്ത മനസ്സിലാക്കലുകളെ പഞ്ചസാരപൊതിഞ്ഞു് അവതരിപ്പിക്കുന്ന' പരിപാടിയും കൂടിയായപ്പോള്‍ 
ചുക്കു് ചുണ്ണാമ്പിനുമപ്പുറം വേറെയെന്തൊക്കെയോ ആയി! റിപ്പോര്‍ട്ടുചെയ്യാന്‍വന്ന കൊച്ചിനു് NVIDIA CEOയുടെ 
അമേരിക്കന്‍ ഉച്ചാരണം മനസ്സിലാകാഞ്ഞതോ, വിഷയപരിജ്ഞാനം കമ്മിയായതോ, എഴുതിയെടുത്തതു് പിന്നെ 
വായിച്ചപ്പോള്‍ തലതിരിഞ്ഞുപോയതോ ഒക്കെയാകാം കാരണം. എങ്കിലും മിനിമം NVIDIAയുടെ വെബ്സൈറ്റില്‍ പോയി Tesla 
എന്ന പ്രോഡക്റ്റിനു കീഴില്‍ എഴുതിയതൊക്കെത്തന്നെയാണോ തന്റെ റിപ്പോര്‍ട്ടറും എഴുതിയതു് എന്നു് നോക്കാനെങ്കിലും 
തോന്നുന്ന ഒരു എഡിറ്റര്‍ NDTVയുടെ ടെക്നോളജി ഡെസ്കില്‍ ഉണ്ടായിരുന്നെങ്കില്‍ ഇത്രമാത്രം 
നാണക്കേടുണ്ടാകില്ലായിരുന്നു. അതിനുശേഷം ഞാന്‍ NDTVയുടെ ടെക്നോളജി വാര്‍ത്തകളൊന്നും വായിക്കാറില്ല, 
ആ വാര്‍ത്ത ഇപ്പോഴും ആ തെറ്റുകളോടെ അവിടെത്തന്നെ കിടക്കുന്നതുകൊണ്ടു് അവരുടെ നയങ്ങളൊന്നും 
മാറിയിട്ടില്ലെന്നു കരുതുന്നു.

ഈ രീതിയിലുള്ള റിപ്പോര്‍ട്ടുകള്‍ പലപ്പോഴും അന്താരാഷ്ട്രതലത്തില്‍ നമുക്കു് മാനക്കേടുമാത്രമാണുണ്ടാക്കാറു്. ഇന്ത്യയിലെ 
റിപ്പോര്‍ട്ടര്‍മാരുടെ അത്യുത്സാഹം കാരണം, പത്രങ്ങളില്‍ റിപ്പോര്‍ട്ടു് ചെയ്യപ്പെടുന്ന പ്രാദേശികമായി വികസിപ്പിച്ച നൂതനവിദ്യകളെ 
മൂന്നുപ്രാവശ്യം ഇരുത്തിവായിക്കുകയും നാലാളോടു ചോദിച്ചു് ഉറപ്പുവരുത്തിയും മാത്രമേ വിശ്വസിക്കാവൂ.
 ഇത്തരത്തില്‍ ചുക്കും ചുണ്ണാമ്പും തിരിച്ചറിയാത്ത റിപ്പോര്‍ട്ടുകള്‍ പത്ര/ടെലിവിഷന്‍ മുന്‍നിരക്കാരുടെ 
പോര്‍ട്ടലുകളില്‍ മാത്രമല്ല, പുതിയ ന്യൂസ് പോര്‍ട്ടലുകളിലും കാണാറുണ്ടു്. പക്ഷേ അവരുടെ ഒരു ഗുണം, തെറ്റു് 
ചൂണ്ടിക്കാണിച്ചുകൊടുത്താല്‍ ക്ഷമചോദിക്കാനും തിരുത്താനും തയ്യാറാകുമെന്നതാണു്. NDTVയെ ഇവിടെയൊരു 
സാമ്പിളായി മാത്രം കാണിച്ചതാണു്. ഇത്തരം തലതിരിഞ്ഞ റിപ്പോര്‍ട്ടിംഗ് എല്ലാ ഇന്ത്യന്‍ മാദ്ധ്യമങ്ങളിലും ഏതാണ്ടു് ഒരേ 
അളവില്‍ത്തന്നെ കണ്ടിട്ടുണ്ടു്.

ഈ പ്രശ്നങ്ങളൊക്കെ ശക്തമായ, അല്ലെങ്കില്‍ ലോജിക്കലായി ചിന്തിക്കുകയെങ്കിലും ചെയ്യുന്ന ഒരു എഡിറ്റോറിയല്‍ 
സംഘവും റിപ്പോര്‍ട്ടര്‍മാരുമില്ലാത്തതിന്റേതാണെങ്കില്‍, തീര്‍ത്തും വ്യത്യസ്തമായ പൂര്‍ണ്ണഅവഗണനയുടെ കണക്കുകളും 
പലപ്പോഴും പത്രങ്ങളില്‍ കാണാറുണ്ടു്. കാറോട്ടമത്സരങ്ങളുടെ നിരുത്തവാദപരമായ റിപ്പോര്‍ട്ടിംഗ് ഒരുദാഹരണം. അതു 
ചൂണ്ടിക്കാണിച്ചാല്‍ പലപ്പോഴും കാരണങ്ങളായി പറയുന്നതു്, വേണ്ടത്ര വായനക്കാരില്ലാത്തതുകൊണ്ടാണെന്നാണു്. 
(ഈയടുത്തു്, അമൃതയിലോ മറ്റോ ഒരു വാരാന്ത്യ സ്പോര്‍ട്സ് റൗണ്ടപ്പില്‍ തരക്കേടില്ലാതെ ഗ്രാന്‍പ്രീകള്‍ 
റിപ്പോര്‍ട്ടു് ചെയ്തുകണ്ടു.)

എല്ലാ പ്രമുഖ മലയാളം പത്രങ്ങളിലും നല്ല വാഹന റിവ്യൂകളും അനുബന്ധവാര്‍ത്തകളും കാണാം. മാത്രമല്ല, വളരെക്കുറച്ചു 
തെറ്റുകള്‍ മാത്രമേ ധാരാളം സാങ്കേതികവിവരങ്ങളെ പരാമര്‍ശിച്ചുകൊണ്ടെഴുതുന്ന ഈ റിപ്പോര്‍ട്ടുകളില്‍ കാണാറുള്ളു. 
വിവരമുള്ള റിപ്പോര്‍ട്ടര്‍മാരുടേയും എഡിറ്റര്‍മാരുടേയും സാന്നിധ്യമായിരിക്കാം കാരണം. ഇത്രയും നല്ല റിപ്പോര്‍ട്ടുകള്‍ 
പ്രസിദ്ധീകരിക്കാനാവുന്നുണ്ടെങ്കില്‍ അവര്‍ക്കു് സാധാരണഗതിയില്‍ തെറ്റുകളില്ലാതെ റേസ് റിപ്പോര്‍ട്ടുകളും എഴുതാന്‍ 
കഴിയേണ്ടതാണു്. (തീര്‍ച്ചയായും വസ്തുതാപരമായ പിഴവുകളെ ഒഴിവാക്കാനാവും.) ഇത്തരം ന്യൂസുകളില്‍ ഇന്റര്‍നെറ്റിന്റേയോ പത്രത്തില്‍ത്തന്നെയുള്ള 
ഓട്ടോമോട്ടീവു് സെക്‌ഷന്റേയോ സഹായം വെരിഫിക്കേഷനുവേണ്ടിയെങ്കിലും ഉപയോഗിച്ചാല്‍ത്തന്നെ പിഴവുകള്‍ 
ഒഴിവാക്കാനാവും.

ഇത്തരം കാര്യങ്ങള്‍ പത്രപ്രവര്‍ത്തനം ബിരുദ/ഡിപ്ലോമ കോഴ്സുകളായി പഠിപ്പിക്കുന്നവര്‍ അവരുടെ സിലബസ്സില്‍ 
ഉള്‍പ്പെടുത്തിയിട്ടുണ്ടോ എന്നറിയില്ല. ഇല്ലെങ്കില്‍ ഇത്തരം കാര്യങ്ങള്‍കൂടി വാര്‍ത്തകള്‍ എഴുതാനും തിരുത്താനും 
പഠിപ്പിക്കുന്ന കൂട്ടത്തില്‍ പഠിപ്പിച്ചാല്‍ നന്നായിരിക്കും. കാണാപ്പാഠം പഠിച്ചു് പരീക്ഷ പാസായി പത്രപ്രവര്‍ത്തകരാകുന്നവര്‍
തങ്ങള്‍ക്കു് അജ്ഞാതമായ വിഷയങ്ങളില്‍ ചെറിയൊരു പഠനമെങ്കിലും കൂടാതെ ആധികാരിക റിപ്പോര്‍ട്ടുകള്‍ 
എഴുതിവിടുന്നതു് ഒഴിവാക്കാനും, ഏതു വിഷയവും എഡിറ്റ് ചെയ്യുന്നതില്‍ ഡെസ്ക് ജോലിക്കാര്‍ കൂടുതല്‍ ശ്രദ്ധവയ്ക്കാനും 
ഇതു് സഹായകമാവുമെന്നു കരുതുന്നു.

വസ്തുതാപരമായ ഒരു പിഴവ്, ശക്തമായ വിഷയങ്ങള്‍ കൈകാര്യം ചെയ്യുന്ന റിപ്പോര്‍ട്ടിന്റെ മുഴുവന്‍ 'ഇന്റഗ്രിറ്റി'യേയും 
സംശയത്തിന്റെ നിഴലിലാക്കുമെന്നുള്ള മനസ്സിലാക്കലെങ്കിലും ഉണ്ടെങ്കില്‍ പകുതി കാര്യങ്ങള്‍ ശരിയാവുമെന്നു തോന്നുന്നു. 
മികച്ച പത്രപ്രവര്‍ത്തനത്തിനുള്ള അവാര്‍ഡ് നേടുന്നവര്‍ക്കുപോലും പത്രപ്രവര്‍ത്തനത്തിനു് മികച്ച ഭാഷയുടെയും 
ഘടനയുടെയുമപ്പുറത്തു്, വേറെയും തലങ്ങളുണ്ടെന്നുള്ള തിരിച്ചറിവില്ലെന്നതിനു് അത്ര പഴയതല്ലാത്ത ചില 
പത്രവാര്‍ത്തകള്‍ സാക്ഷികളാണു്.

\begin{flushright}(5 August, 2010)\footnote{http://malayal.am/വാര്‍ത്ത/മീഡിയ-സ്കാന്‍/7249/അപ്രത്യക്ഷമാകുന്ന-എഡിറ്റോറിയല്‍-ഡെസ്ക്}\end{flushright}

\newpage
