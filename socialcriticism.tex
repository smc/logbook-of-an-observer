\secstar{സാമൂഹ്യ വിമര്‍ശനത്തെക്കുറിച്ച് }

\vskip 2pt

ആരെങ്കിലും എന്തെങ്കിലും നടപടിയെയോ നയത്തേയോ വിമര്‍ശിക്കുമ്പോള്‍ സ്ഥിരമായി കേള്‍ക്കുന്നതാണു്, ബദലിന്റെ ചോദ്യം. 
നമ്മുടെ മനസ്സില്‍ പതിഞ്ഞുപോയ ചെറുപ്പം മുതലേ പഠിപ്പിച്ചു വച്ചിരിയ്ക്കുന്ന ഒരു കാര്യമാണതു്. എന്തെങ്കിലും കാര്യത്തിന്റെ നടത്തിപ്പില്‍ 
കാര്യമായ ദോഷം നിങ്ങള്‍ കാണുന്നുണ്ടെങ്കിലും അതിന്റെ നല്ല വശം മാത്രം കണ്ടു് അതിനെ അഭിനന്ദിക്കുക, നിങ്ങള്‍ക്കു പ്രവര്‍ത്തിച്ചു
 കാണിക്കാനാവുന്ന ഒരു ബദല്‍ നിര്‍ദ്ദേശിക്കാനില്ലെങ്കില്‍ ദോഷകരമായ വശങ്ങളെ കണ്ടില്ലെന്നു നടിച്ച്, ഇത്രയും ചെയ്ത നല്ല മനസ്സിനെ 
 അഭിനന്ദിക്കുക. ഇത്രയൊക്കെ നന്മ ചെയ്യുന്ന നല്ല മനസ്സിനെ കണ്ടുകൂടെ എന്ന ചോദ്യവും, ഇനി വിമര്‍ശനം പേടിച്ചു ആരും ഒന്നും ചെയ്യില്ല 
 എന്ന വായ്‌ത്താരിയും, വെറുതെയിരുന്നു കുറ്റം പറയുന്ന നേരം രണ്ടു കാര്യം ചെയ്തു കാണിക്കു് എന്ന വെല്ലുവിളിയും എല്ലാം വിമര്‍ശങ്ങളെ 
 കാത്തിരിക്കുന്ന സ്ഥിരം മറുപടികളാണു്.

ഒരു കാര്യം ചീത്തയാണെങ്കില്‍ അതു ചൂണ്ടിക്കാണിക്കും മുമ്പ് അതിനൊരു ബദലും ചൂണ്ടിക്കാണിക്കുന്ന ആള്‍ തന്നെ നിര്‍ദ്ദേശിക്കണം 
എന്നു പറയുന്നതു് തന്നെ സത്യത്തില്‍ മണ്ടത്തരമാണു്. പലപ്പോഴും തെറ്റുകള്‍ ചൂണ്ടിക്കാണിക്കുന്നവര്‍ ബദല്‍ നിര്‍ദ്ദേശിക്കാറുണ്ട്. 
അതിനു കിട്ടാറുള്ള മറുപടി, എങ്കില്‍ നിങ്ങളതൊന്നു ചെയ്തു കാണിക്കു ഞങ്ങള്‍ക്കു് സമയമില്ല എന്നാണ്. ബദലുകള്‍ ചര്‍ച്ച ചെയ്യാനുള്ള
 സന്നദ്ധത വളരെക്കുറച്ചു പേര്‍ മാത്രമേ കാണിക്കൂ.

കാരണം മറ്റൊന്നുമല്ല, തങ്ങള്‍ തുടങ്ങി വച്ച വിജയകരമായ ഒരു ഉദ്യമത്തില്‍ തങ്ങളെ നിശിതമായി വിമര്‍ശിച്ചവര്‍ക്കു പങ്കാളിത്തം
 നല്‍കുന്നതിലുള്ള വൈക്ലബ്യം. ചുരുക്കം ചിലര്‍ വിമര്‍ശനങ്ങളെ കാര്യമായി കാണുകയും, നിര്‍ദ്ദേശിക്കപ്പെട്ട ബദലുകള്‍ അവര്‍ 
 പരിഗണിക്കുകയും പിന്നീട് തള്ളിക്കളയുകയും ചെയ്തതാണെങ്കില്‍ അക്കാര്യം അറിയിക്കുകയും, ചൂണ്ടിക്കാട്ടിയ പ്രശ്നങ്ങളെ അവ അര്‍ഹിക്കുന്ന 
 ഗൌരവത്തോടെ സമീപിക്കുകയും ചെയ്യാറുണ്ടു് (വിമര്‍ശകന്‍ പ്രശ്നത്തിനു കൊടുക്കുന്ന മുന്‍ഗണനയാവണമെന്നില്ല ഇവരുടേത്).

ഏതാണ്ടു 90 ശതമാനം കേസുകളിലും വിമര്‍ശകന്‍ ചൂണ്ടിക്കാണിക്കുന്ന പ്രശ്നങ്ങള്‍ ഗൌരവമേറിയതാണെങ്കിലും ബദലുകള്‍ 
പ്രായോഗികമാകണമെന്നില്ല. അവ ഒരാളുടെ നിരീക്ഷണത്തില്‍ നിന്നും ഉരുത്തിരിഞ്ഞു വന്നവമാത്രമാണ്. 
സംരംഭം നടത്തുന്നവര്‍ ഒരുപാടു പഠനങ്ങളും മറ്റും നടത്തിയാകണം അവരുടെ വഴിതിരഞ്ഞെടുത്തിരിക്കുക, അതുകൊണ്ടുതന്നെ
 പ്രായോഗികതലത്തില്‍ വിമര്‍ശങ്ങളില്‍ ഉന്നയിക്കപ്പെടുന്ന പ്രശ്നങ്ങള്‍ ശ്രദ്ധയര്‍ഹിക്കുമ്പോള്‍ തന്നെ ബദലുകള്‍ മിക്കപ്പോഴും 
 സമൂഹത്തിന്റെ വായടപ്പിക്കാന്‍ വേണ്ടി മാത്രം നിര്‍ദ്ദേശിക്കപ്പെടുന്നവയുമാകും.

മാത്രമല്ല, ഒരു പ്രശ്നം പരിഹരിക്കാനുള്ള ശരിയായ വഴി എല്ലായ്പ്പോഴും, അതു കൃത്യമായി ബന്ധപ്പെട്ടവരുടെ ശ്രദ്ധയില്‍ 
കൊണ്ടുവരികയെന്നതാണ്. യോജിച്ച പരിഹാരം അവര്‍ കണ്ടെത്തിക്കോളും (കണ്ടെത്തണം). വേണമെങ്കില്‍ കൃത്യമായ ഇടവേളകളില്‍ 
കാര്യങ്ങള്‍ വിണ്ടും ഉന്നയിച്ച് അവ അധികാരികളുടെ മുന്‍ഗണനാ പട്ടികയില്‍ മുന്നില്‍ത്തന്നെ ഇടം നേടിക്കൊടുക്കയും ചെയ്യാം.

എങ്കിലും സമൂഹത്തിനു ഒരു വിമര്‍ശം കാമ്പുള്ളതായിത്തോന്നണമെങ്കില്‍ അതില്‍ ബദല്‍ നിര്‍ദ്ദേശങ്ങള്‍ വേണം.
 നിര്‍ദ്ദേശിക്കപ്പെട്ട ബദല്‍ നടപ്പാക്കത്തതിനു കാരണം ബോധിപ്പിക്കണം. പലപ്പോഴും ഇതു് മുമ്പേ ചെയ്തിട്ടുണ്ടാകും, എങ്കിലും പുതിയ 
 വിമര്‍ശത്തിന്റെയും പഠനത്തിന്റെയും അടിസ്ഥാനത്തില്‍ ഒന്നു കൂടി ചെയ്യണമെന്നു് സമൂഹം വാശിപിടിക്കുന്നതു് അപൂര്‍വ്വമൊന്നുമല്ല.

കൃത്യമായ പഠനങ്ങളുടെ പിന്‍ബലമില്ലാതെ നിര്‍ദ്ദേശിക്കപ്പെടുന്ന ബദലുകള്‍ സമൂഹത്തിനു ഗുണകരമായ ഒരു പദ്ധതിയുടെ
 നടത്തിപ്പിനെ ബാധിക്കുന്നു. പിന്നെ സമൂഹത്തിന്റെ വാശിയും ദേഷ്യവും സാധാരണ തിരിയുന്നതു് വിമര്‍ശകനിലേക്കാണ്, 
 ഏതു സമൂഹത്തിന്റെ അപ്രീതി ഭയന്നു് വേണ്ടത്ര തെളിവുകളോ പഠനങ്ങളോ നടത്താതെ അടിയന്തര പ്രാധാന്യമുള്ള പ്രശ്നങ്ങള്‍ 
 പരിഹരിക്കപ്പെടട്ടെ എന്നു കരുതിമാത്രം ഒരു ബദലും കൂട്ടിക്കെട്ടി വിമര്‍ശം പ്രസിദ്ധീകരിച്ചുവോ ആ സമൂഹത്തിന്റെ.

പ്രശ്നങ്ങളോടുകൂടിയാണെങ്കിലും സുഗമമായി നടന്നു പൊയ്ക്കൊണ്ടിരിക്കുന്ന ഒരു വ്യവസ്ഥയെ പരിഷ്കരിക്കാനാണു് വിമര്‍ശകന്‍ പ്രശ്നങ്ങള്‍ 
ചൂണ്ടിക്കാണിക്കുന്നത്. അതിനോട് സമൂഹം അസഹിഷ്ണുതയോടെ പ്രതികരിക്കുന്നതു് സാമൂഹികമായ ജഡത്വം (social inertia) മൂലമാണു്. 
അതുശരിയാക്കാനുള്ള മാര്‍ഗ്ഗം വെറും വിമര്‍ശനമല്ല, ശക്തമായ പ്രചരണപ്രവര്‍ത്തനങ്ങളിലൂടെയുള്ള ബോധവത്കരണമാണ്.

ഒരുരീതിയിലുള്ള സമരം, മേധാ പട്കറും, മയിലമ്മയും ഒക്കെ നടത്തിവന്നിരുന്ന (വരുന്ന) സമരം വിമര്‍ശനങ്ങളെ പ്രശ്നങ്ങളിലേക്കു 
ശ്രദ്ധ ക്ഷണിക്കാനുപയോഗിക്കാം. വളരെ വ്യക്തവും സുശക്തവുമായ തെളിവുകളുടെ പിന്‍ബലമുണ്ടെങ്കില്‍ ബദലുകളും നിര്‍ദ്ദേശിക്കാം.
 സമൂഹത്തിന്റെ അപ്രീതി ഭയന്നു് ആവശ്യമില്ലാത്തതൊന്നും കൂട്ടിച്ചേര്‍ക്കുകയോ, അവശ്യകാര്യങ്ങള്‍ വിട്ടുകളയുകയോ ചെയ്ത് വിമര്‍ശിക്കുന്നത്, 
 വിമര്‍ശിക്കാതിരിക്കുന്നതിനു തുല്യമാണു്. അതു സമൂഹത്തിലെ ജഡത്വത്തെ ശക്തിപ്പെടുത്തുക മാത്രമേയുള്ളൂ.

ഇതുവരെ സംരംഭങ്ങളെ വിമര്‍ശിക്കുന്നവരോടോ വിമര്‍ശനാത്മകമായി വിലയിരുത്തന്നവരോടോ സംരംഭകരും, മിക്കപ്പോഴും 
ഗുണഭോക്താക്കളായ സിവില്‍ സമൂഹവും എടുക്കുന്ന നിലപാടുകളെക്കുറിച്ചാണു് പറഞ്ഞതു്. വിമര്‍ശനം സംരംഭങ്ങളെപറ്റി മാത്രമല്ല ഉണ്ടാവാറ്.
 സാമൂഹിക/രാഷ്ട്രീയ/ഭരണ സ്ഥാപനങ്ങളുടെ നയങ്ങളെയോ, സമൂഹത്തിലെ വിവിധ കീഴ്‌വഴക്കങ്ങളെയോ, ഒക്കെ വിമര്‍ശനവിധേയമാക്കാറുണ്ട്. 
 പലപ്പോഴും ഇത്തരം വിമര്‍ശങ്ങളുന്നയിക്കുന്നവരോടു് മാദ്ധ്യമസ്ഥാപനങ്ങളടക്കമുള്ളവരുടെ (സാധാരണഗതിയില്‍ വലിയ വിമര്‍ശകര്‍ മാദ്ധ്യമങ്ങളാണു്)
  സ്ഥിരം ചോദ്യങ്ങള്‍ രണ്ടാണ്.

ഒന്നു് ബദലിനെ സംബന്ധിച്ചതാണു്. വലിയ സാമൂഹിക ചലനങ്ങളുണ്ടാക്കാന്‍ കഴിവുള്ള ഒരു സ്ഥാപനത്തിന്റെ നയം ചില ദോഷകരമായ 
പ്രത്യാഘാതങ്ങളുണ്ടാക്കാന്‍ പോന്നതാണു് എന്നു ചൂണ്ടിക്കാണിച്ചതിനാണു് ഈ ചോദ്യം എന്നോര്‍ക്കണം. ഇത്രയും വലിയ സ്ഥാപനത്തിന്റെ
 നയപരമായ കാര്യങ്ങളുടെ പ്രത്യാഘാതങ്ങളെക്കുറിച്ച് എന്തെങ്കിലും ധാരണകളുണ്ടെങ്കില്‍ അത്തരമൊരു ചോദ്യം മനസ്സില്‍ വരാനേ പാടില്ലാത്തതാണ്. 
 ഒരു വ്യക്തിയുടെ അഭിപ്രായങ്ങളിലൂടെ പരിഹരിക്കേണ്ടതല്ല ഈ പ്രശ്നങ്ങള്‍. പക്ഷെ, അതിനര്‍ത്ഥം തെറ്റുകള്‍ ചൂണ്ടിക്കാണിച്ചു കൊടുക്കാന്‍ വ്യക്തികളെ അനുവദിക്കരുതെന്നല്ല.

രണ്ടാമതു ചോദിക്കുന്ന ചോദ്യമാണു് ഏറ്റവും രസകരം. അതു പ്രസ്തുത സാമൂഹിക സ്ഥാപനത്തിന്റെ സേവനം ഉപയോഗിക്കുന്നതിനെപ്പറ്റിയാണു്.
 അത്രമാത്രം വിമര്‍ശനാത്മകമാണു് പ്രസ്തുത സ്ഥാപനത്തിന്റെ പ്രവൃത്തികളെങ്കില്‍ അതിനെ ഒഴിവാക്കി ബദലുകള്‍ തേടിക്കൂടെ എന്നാണു ചോദ്യം.
  കാര്യം പറഞ്ഞാല്‍, തീര്‍ത്തും ബാലിശവും രസകരവുമായ ചോദ്യം. അതു ചോദിക്കുന്നതു് ഉത്തരവാദപ്പെട്ട സാമൂഹിക വിമര്‍ശകരായി 
  സ്വയം മാറേണ്ട മാദ്ധ്യമപ്രവര്‍ത്തകരാവുമ്പോഴാണു് ഇതിലെ അപകടം.

ഈയടുത്തകാലത്തു് ഈ രണ്ടു ചോദ്യങ്ങളേയും നേരിടേണ്ടിവന്നതു് അരുന്ധതി റോയ് ആണു്. ദേശരാഷ്ട്രങ്ങളില്‍ പലപ്പോഴും 
പാര്‍ശ്വവത്കൃതര്‍ക്കു് നീതി കിട്ടുന്നില്ലെന്നു തുറന്നു പറഞ്ഞ അവര്‍ കാശ്മീരിലെ ജനങ്ങളുടെ സ്വയം നിര്‍ണ്ണയാവകാശത്തെപ്പറ്റിയും
 മനുഷ്യാവകാശങ്ങളെപ്പറ്റിയും സംസാരിച്ചപ്പോഴായിരുന്നു ഇതു്. ഒരു ദേശരാഷ്ട്രത്തിന്റെ സുരക്ഷയിലും പിന്തുണയിലുമിരുന്നാണു് 
 താന്‍ ഇതൊക്കെപ്പറയുന്നതെന്നു് അരുന്ധതി മറക്കരുതെന്നായിരുന്നു ഒരു വാരികയില്‍ വന്നതു്. അരുന്ധതിയുമായി മറ്റൊരു 
 വാരിക നടത്തിയ അഭിമുഖത്തിലാവട്ടെ, ബദലുകളുടെ ചോദ്യവും ഉന്നയിക്കപ്പെട്ടു.

ഈ ചോദ്യങ്ങള്‍ വരുന്നതു് ചില മുന്‍വിധികളില്‍ നിന്നാണു്. വിമര്‍ശങ്ങള്‍ വരുന്നതു് പ്രസ്തുത സ്ഥാപനമായോ വ്യവസ്ഥയുമായോ
 സംരംഭമായോ കടുത്ത എതിര്‍പ്പിലാണെങ്കില്‍ മാത്രമാണെന്നതാണൊന്നു്. മറ്റൊന്നു വിമര്‍ശനം മറ്റൊരു സമരമാര്‍ഗ്ഗം
  മാത്രമാണെന്ന തെറ്റിദ്ധാരണയാണു്. താന്‍ കൂടി ഭാഗമായ സമൂഹത്തിന്റെ ഉന്നമനത്തിനും സാമൂഹിക സ്ഥാപനങ്ങളുടെ നല്ല 
  നടത്തിപ്പിനും അവയുടെ നടത്തിപ്പിലോ നയങ്ങളിലോ ഉള്ള തെറ്റുകള്‍ പരിഹരിച്ച് മുന്നോട്ടു പോകണമെന്ന ആഗ്രഹം, അല്ലങ്കില്‍ 
  നടത്തിപ്പിലെ അപാകതകള്‍ പരിഹരിക്കപ്പെടണമെന്ന ആഗ്രഹം, വിമര്‍ശകര്‍ക്കുണ്ടാവുമെന്നു് പലര്‍ക്കും സ്വപ്നം പോലും കാണാന്‍ 
  കഴിയുന്നില്ല.

അതുകൊണ്ടു തന്നെയാണു് നമ്മളില്‍ പലര്‍ക്കും താന്‍ കാശ്മീര്‍ സ്വതന്ത്രമാക്കണമെന്നും ഇന്ത്യ വെട്ടിമുറിക്കണമെന്നുമല്ല 
വാദിക്കുന്നതെന്നും, ഇന്ത്യ എന്ന ദേശരാഷ്ട്രം അതിന്റെ ഭാഗമായികാണുന്ന കാശ്മീരിലെ ജനതയോടു ചെയ്തതു് / ചെയ്യുന്നതു് മാനുഷികപരമായി 
നീതിയല്ലെന്നും, അവരുടെ സ്വയം നിര്‍ണ്ണയാവകാശത്തെയും, മനുഷ്യാവകാശങ്ങളെയും മാനിക്കണമെന്നു് ആവശ്യപ്പെടുകയാണു് ചെയ്തതെന്നും
 അരുന്ധതി റോയ് പറയുന്നതു് ദഹിക്കാത്തതു്. അരുന്ധതി റോയ്, 'കാശ്മീരില്‍ ഇന്ത്യ പെരുമാറുന്നതു് ഒരു അധിനിവേശ ശക്തിയെപ്പോലെയാണെന്നു' 
 പറയുമ്പോള്‍ അവര്‍ ദേശദ്രോഹിയായാണു് മുദ്രകുത്തപ്പെടുന്നതു്. പക്ഷേ, അവര്‍ താന്‍ നേരിട്ടു കണ്ട തെളിവുകള്‍ കൊണ്ടു പറയുന്നതിനെ
  സാക്ഷ്യപ്പെടുത്തുമ്പോള്‍ അതു് ഒരു നീതിപൂര്‍വ്വ സമൂഹം പുലരുന്ന ജനാധിപത്യരാജ്യമെന്ന നിലയില്‍ ഇന്ത്യയെ മെച്ചപ്പെട്ട ഭാവിയിലേക്കു് 
  നയിക്കാനാണെന്നു മനസ്സിലാക്കാന്‍ പലര്‍ക്കും കഴിയാതെപോകുന്നതു്, വിമര്‍ശങ്ങളെക്കുറിച്ചുള്ള മുന്‍വിധികള്‍ കാരണമാണു്.  
  തന്നെ വിമര്‍ശിക്കുന്ന ആരേക്കാളും ഇന്ത്യയെ താന്‍ സ്നേഹിക്കുന്നെന്നും തന്റെ രാജ്യത്തില്‍ നീതിപൂര്‍വ്വക സമൂഹം പുലര്‍ന്നു കാണാനുള്ള 
  ആഗ്രഹമാണു തന്റെ വിമര്‍ശത്തിനു പിന്നിലെന്നും അവര്‍ പറയുമ്പോള്‍ അതു മനസ്സിലാക്കാന്‍ നമുക്കു കഴിയാത്തതും അതുകൊണ്ടു തന്നെ.

സാമൂഹ്യവിമര്‍ശനമെന്നതു് നീതിപൂര്‍വ്വകമായൊരു സമൂഹം വാര്‍ത്തെടുക്കാനുള്ള ശക്തമായ ആയുധമാണു്. ദേശരാഷ്ട്രങ്ങളിലെ 
അധികാരകേന്ദ്രങ്ങള്‍ തങ്ങളുടെ ജനതയിലെ പല വിഭാഗങ്ങളെയും പാര്‍ശ്വവത്കരിക്കുകയും അവഗണിക്കുകയും പലപ്പോഴും അവരുടെ 
മൌലികാവകാശങ്ങള്‍ പോലും കവര്‍ന്നെടുക്കുകയും ചെയ്യുമ്പോള്‍ സാമൂഹ്യ വിമര്‍ശനം നാം ഓരോരുത്തരുടെയും കടമയായി മാറുകയാണു്. 
എന്നാല്‍ വിമര്‍ശനം അവസാനമല്ല, അതൊരു ദീര്‍ഘമേറിയതും ദുര്‍ഘടം പിടിച്ചതുമായ പാതയുടെ തുടക്കം മാത്രമാണു്.

അധികാരത്തിനെതിരെയുള്ള സമരങ്ങളും, വിവിധ ബോധവത്കരണ പ്രചരണ പരിപാടികളും, എല്ലാം വിമര്‍ശങ്ങള്‍ക്കു പിറകേ വരണം.
 അവയിലും സജീവമായ ഇടപെടല്‍ വിമര്‍ശകരുടെ ഭാഗത്തുനിന്നുണ്ടാവണം. അനീതിയില്‍ നിന്നും നീതിപൂര്‍വ്വകമായ ബദലുകളിലേക്കു് 
 നയിക്കാന്‍ അവശ്യം വേണ്ട പൊതുശ്രദ്ധയും ചര്‍ച്ചകളും സ്വയം വിമര്‍ശനാത്മകമായി വിലയിരുത്തുന്ന സമൂഹത്തില്‍ വളരെ എളുപ്പം നടക്കും. 
 അതു സ്വാഭാവികമായ ജഡത്വം വെടിഞ്ഞു് ചലനാത്മകവും നീതിപൂര്‍വ്വകവുമായ ഒന്നായി മാറുന്നതിലേക്കു് സമൂഹത്തെ സഹായിക്കുകയും ചെയ്യും.

(Feb 15, 2011)\footnote{http://malayal.am/വാര്‍ത്ത/വിശകലനം/10054/സാമൂഹ്യ-വിമര്‍ശനത്തെക്കുറിച്ച്}
\newpage
