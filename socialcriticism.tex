\secstar{സാമൂഹ്യ വിമര്‍ശനത്തെക്കുറിച്ച് }

\vskip 2pt

ആ­രെ­ങ്കി­ലും എന്തെ­ങ്കി­ലും നട­പ­ടി­യെ­യോ നയ­ത്തേ­യോ വി­മര്‍­ശി­ക്കു­മ്പോള്‍ സ്ഥി­ര­മാ­യി കേള്‍­ക്കു­ന്ന­താ­ണു്, ബദ­ലി­ന്റെ ചോ­ദ്യം. 
നമ്മു­ടെ മന­സ്സില്‍ പതി­ഞ്ഞു­പോയ ചെ­റു­പ്പം മു­ത­ലേ പഠി­പ്പി­ച്ചു വച്ചി­രി­യ്ക്കു­ന്ന ഒരു കാ­ര്യ­മാ­ണ­തു്. എന്തെ­ങ്കി­ലും കാ­ര്യ­ത്തി­ന്റെ നട­ത്തി­പ്പില്‍ 
കാ­ര്യ­മായ ദോ­ഷം നി­ങ്ങള്‍ കാ­ണു­ന്നു­ണ്ടെ­ങ്കി­ലും അതി­ന്റെ നല്ല വശം മാ­ത്രം കണ്ടു് അതി­നെ അഭി­ന­ന്ദി­ക്കു­ക, നി­ങ്ങള്‍­ക്കു പ്ര­വര്‍­ത്തി­ച്ചു
 കാ­ണി­ക്കാ­നാ­വു­ന്ന ഒരു ബദല്‍ നിര്‍­ദ്ദേ­ശി­ക്കാ­നി­ല്ലെ­ങ്കില്‍ ദോ­ഷ­ക­ര­മായ വശ­ങ്ങ­ളെ കണ്ടി­ല്ലെ­ന്നു നടി­ച്ച്, ഇത്ര­യും ചെ­യ്ത നല്ല മന­സ്സി­നെ 
 അഭി­ന­ന്ദി­ക്കു­ക. ഇത്ര­യൊ­ക്കെ നന്മ ചെ­യ്യു­ന്ന നല്ല മന­സ്സി­നെ കണ്ടു­കൂ­ടെ എന്ന ചോ­ദ്യ­വും, ഇനി ­വി­മര്‍­ശ­നം­ പേ­ടി­ച്ചു ആരും ഒന്നും ചെ­യ്യി­ല്ല 
 എന്ന വാ­യ്‌­ത്താ­രി­യും, വെ­റു­തെ­യി­രു­ന്നു കു­റ്റം പറ­യു­ന്ന നേ­രം രണ്ടു കാ­ര്യം ചെ­യ്തു കാ­ണി­ക്കു് എന്ന വെ­ല്ലു­വി­ളി­യും എല്ലാം വി­മര്‍­ശ­ങ്ങ­ളെ 
 കാ­ത്തി­രി­ക്കു­ന്ന സ്ഥി­രം മറു­പ­ടി­ക­ളാ­ണു്.

ഒ­രു കാ­ര്യം ചീ­ത്ത­യാ­ണെ­ങ്കില്‍ അതു ചൂ­ണ്ടി­ക്കാ­ണി­ക്കും മു­മ്പ് അതി­നൊ­രു ബദ­ലും ചൂ­ണ്ടി­ക്കാ­ണി­ക്കു­ന്ന ആള്‍ തന്നെ നിര്‍­ദ്ദേ­ശി­ക്ക­ണം 
എന്നു പറ­യു­ന്ന­തു് തന്നെ സത്യ­ത്തില്‍ മണ്ട­ത്ത­ര­മാ­ണു്. പല­പ്പോ­ഴും തെ­റ്റു­കള്‍ ചൂ­ണ്ടി­ക്കാ­ണി­ക്കു­ന്ന­വര്‍ ബദല്‍ നിര്‍­ദ്ദേ­ശി­ക്കാ­റു­ണ്ട്. 
അതി­നു കി­ട്ടാ­റു­ള്ള മറു­പ­ടി, എങ്കില്‍ നി­ങ്ങ­ള­തൊ­ന്നു ചെ­യ്തു കാ­ണി­ക്കു ഞങ്ങള്‍­ക്കു് സമ­യ­മി­ല്ല എന്നാ­ണ്. ബദ­ലു­കള്‍ ചര്‍­ച്ച ചെ­യ്യാ­നു­ള്ള
 സന്ന­ദ്ധത വള­രെ­ക്കു­റ­ച്ചു പേര്‍ മാ­ത്ര­മേ കാ­ണി­ക്കൂ­.

­കാ­ര­ണം മറ്റൊ­ന്നു­മ­ല്ല, തങ്ങള്‍ തു­ട­ങ്ങി വച്ച വി­ജ­യ­ക­ര­മായ ഒരു ഉദ്യ­മ­ത്തില്‍ തങ്ങ­ളെ നി­ശി­ത­മാ­യി വി­മര്‍­ശി­ച്ച­വര്‍­ക്കു പങ്കാ­ളി­ത്തം
 നല്‍­കു­ന്ന­തി­ലു­ള്ള വൈ­ക്ല­ബ്യം. ചു­രു­ക്കം ചി­ലര്‍ വി­മര്‍­ശ­ന­ങ്ങ­ളെ കാ­ര്യ­മാ­യി കാ­ണു­ക­യും, നിര്‍­ദ്ദേ­ശി­ക്ക­പ്പെ­ട്ട ബദ­ലു­കള്‍ അവര്‍ 
 പരി­ഗ­ണി­ക്കു­ക­യും പി­ന്നീ­ട് തള്ളി­ക്ക­ള­യു­ക­യും ചെ­യ്ത­താ­ണെ­ങ്കില്‍ അക്കാ­ര്യം അറി­യി­ക്കു­ക­യും, ചൂ­ണ്ടി­ക്കാ­ട്ടിയ പ്ര­ശ്ന­ങ്ങ­ളെ അവ അര്‍­ഹി­ക്കു­ന്ന 
 ഗൌ­ര­വ­ത്തോ­ടെ സമീ­പി­ക്കു­ക­യും ചെ­യ്യാ­റു­ണ്ടു് (വി­മര്‍­ശ­കന്‍ പ്ര­ശ്ന­ത്തി­നു കൊ­ടു­ക്കു­ന്ന മുന്‍­ഗ­ണ­ന­യാ­വ­ണ­മെ­ന്നി­ല്ല ഇവ­രു­ടേ­ത്).

ഏ­താ­ണ്ടു 90 ശത­മാ­നം കേ­സു­ക­ളി­ലും വി­മര്‍­ശ­കന്‍ ചൂ­ണ്ടി­ക്കാ­ണി­ക്കു­ന്ന പ്ര­ശ്ന­ങ്ങള്‍ ഗൌ­ര­വ­മേ­റി­യ­താ­ണെ­ങ്കി­ലും ബദ­ലു­കള്‍ 
പ്രാ­യോ­ഗി­ക­മാ­ക­ണ­മെ­ന്നി­ല്ല. അവ ഒരാ­ളു­ടെ നി­രീ­ക്ഷ­ണ­ത്തില്‍ നി­ന്നും ഉരു­ത്തി­രി­ഞ്ഞു വന്ന­വ­മാ­ത്ര­മാ­ണ്. 
സം­രം­ഭം നട­ത്തു­ന്ന­വര്‍ ഒരു­പാ­ടു പഠ­ന­ങ്ങ­ളും മറ്റും നട­ത്തി­യാ­ക­ണം അവ­രു­ടെ വഴി­തി­ര­ഞ്ഞെ­ടു­ത്തി­രി­ക്കു­ക, അതു­കൊ­ണ്ടു­ത­ന്നെ
 പ്രാ­യോ­ഗി­ക­ത­ല­ത്തില്‍ വി­മര്‍­ശ­ങ്ങ­ളില്‍ ഉന്ന­യി­ക്ക­പ്പെ­ടു­ന്ന പ്ര­ശ്ന­ങ്ങള്‍ ശ്ര­ദ്ധ­യര്‍­ഹി­ക്കു­മ്പോള്‍ തന്നെ ബദ­ലു­കള്‍ മി­ക്ക­പ്പോ­ഴും 
 സമൂ­ഹ­ത്തി­ന്റെ വാ­യ­ട­പ്പി­ക്കാന്‍ വേ­ണ്ടി മാ­ത്രം നിര്‍­ദ്ദേ­ശി­ക്ക­പ്പെ­ടു­ന്ന­വ­യു­മാ­കും­.

­മാ­ത്ര­മ­ല്ല, ഒരു പ്ര­ശ്നം പരി­ഹ­രി­ക്കാ­നു­ള്ള ശരി­യായ വഴി എല്ലാ­യ്പ്പോ­ഴും, അതു കൃ­ത്യ­മാ­യി ബന്ധ­പ്പെ­ട്ട­വ­രു­ടെ ശ്ര­ദ്ധ­യില്‍ 
കൊ­ണ്ടു­വ­രി­ക­യെ­ന്ന­താ­ണ്. യോ­ജി­ച്ച പരി­ഹാ­രം അവര്‍ കണ്ടെ­ത്തി­ക്കോ­ളും (ക­ണ്ടെ­ത്ത­ണം­). വേ­ണ­മെ­ങ്കില്‍ കൃ­ത്യ­മായ ഇട­വേ­ള­ക­ളില്‍ 
കാ­ര്യ­ങ്ങള്‍ വി­ണ്ടും ഉന്ന­യി­ച്ച് അവ അധി­കാ­രി­ക­ളു­ടെ മുന്‍­ഗ­ണ­നാ പട്ടി­ക­യില്‍ മു­ന്നില്‍­ത്ത­ന്നെ ഇടം നേ­ടി­ക്കൊ­ടു­ക്ക­യും ചെ­യ്യാം­.

എ­ങ്കി­ലും സമൂ­ഹ­ത്തി­നു ഒരു വി­മര്‍­ശം കാ­മ്പു­ള്ള­താ­യി­ത്തോ­ന്ന­ണ­മെ­ങ്കില്‍ അതില്‍ ബദല്‍ നിര്‍­ദ്ദേ­ശ­ങ്ങള്‍ വേ­ണം.
 നിര്‍­ദ്ദേ­ശി­ക്ക­പ്പെ­ട്ട ബദല്‍ നട­പ്പാ­ക്ക­ത്ത­തി­നു കാ­ര­ണം ബോ­ധി­പ്പി­ക്ക­ണം. പല­പ്പോ­ഴും ഇതു് മു­മ്പേ ചെ­യ്തി­ട്ടു­ണ്ടാ­കും, എങ്കി­ലും പു­തിയ 
 വി­മര്‍­ശ­ത്തി­ന്റെ­യും പഠ­ന­ത്തി­ന്റെ­യും അടി­സ്ഥാ­ന­ത്തില്‍ ഒന്നു കൂ­ടി ചെ­യ്യ­ണ­മെ­ന്നു് ­സ­മൂ­ഹം­ വാ­ശി­പി­ടി­ക്കു­ന്ന­തു് അപൂര്‍­വ്വ­മൊ­ന്നു­മ­ല്ല.

­കൃ­ത്യ­മായ പഠ­ന­ങ്ങ­ളു­ടെ പിന്‍­ബ­ല­മി­ല്ലാ­തെ നിര്‍­ദ്ദേ­ശി­ക്ക­പ്പെ­ടു­ന്ന ബദ­ലു­കള്‍ സമൂ­ഹ­ത്തി­നു ഗു­ണ­ക­ര­മായ ഒരു പദ്ധ­തി­യു­ടെ
 നട­ത്തി­പ്പി­നെ ബാ­ധി­ക്കു­ന്നു. പി­ന്നെ സമൂ­ഹ­ത്തി­ന്റെ വാ­ശി­യും ദേ­ഷ്യ­വും സാ­ധാ­രണ തി­രി­യു­ന്ന­തു് വി­മര്‍­ശ­ക­നി­ലേ­ക്കാ­ണ്, 
 ഏതു സമൂ­ഹ­ത്തി­ന്റെ അപ്രീ­തി ഭയ­ന്നു് വേ­ണ്ട­ത്ര തെ­ളി­വു­ക­ളോ പഠ­ന­ങ്ങ­ളോ നട­ത്താ­തെ അടി­യ­ന്തര പ്രാ­ധാ­ന്യ­മു­ള്ള പ്ര­ശ്ന­ങ്ങള്‍ 
 പരി­ഹ­രി­ക്ക­പ്പെ­ട­ട്ടെ എന്നു കരു­തി­മാ­ത്രം ഒരു ബദ­ലും കൂ­ട്ടി­ക്കെ­ട്ടി വി­മര്‍­ശം പ്ര­സി­ദ്ധീ­ക­രി­ച്ചു­വോ ആ സമൂ­ഹ­ത്തി­ന്റെ­.

­പ്ര­ശ്ന­ങ്ങ­ളോ­ടു­കൂ­ടി­യാ­ണെ­ങ്കി­ലും സു­ഗ­മ­മാ­യി നട­ന്നു പൊ­യ്ക്കൊ­ണ്ടി­രി­ക്കു­ന്ന ഒരു വ്യ­വ­സ്ഥ­യെ പരി­ഷ്ക­രി­ക്കാ­നാ­ണു് വി­മര്‍­ശ­കന്‍ പ്ര­ശ്ന­ങ്ങള്‍ 
ചൂ­ണ്ടി­ക്കാ­ണി­ക്കു­ന്ന­ത്. അതി­നോ­ട് സമൂ­ഹം അസ­ഹി­ഷ്ണു­ത­യോ­ടെ പ്ര­തി­ക­രി­ക്കു­ന്ന­തു് സാ­മൂ­ഹി­ക­മായ ജഡ­ത്വം (social inertia) മൂ­ല­മാ­ണു്. 
അതു­ശ­രി­യാ­ക്കാ­നു­ള്ള മാര്‍­ഗ്ഗം വെ­റും വി­മര്‍­ശ­ന­മ­ല്ല, ശക്ത­മായ പ്ര­ച­ര­ണ­പ്ര­വര്‍­ത്ത­ന­ങ്ങ­ളി­ലൂ­ടെ­യു­ള്ള ബോ­ധ­വ­ത്ക­ര­ണ­മാ­ണ്.

ഒ­രു­രീ­തി­യി­ലു­ള്ള സമ­രം, മേ­ധാ പട്ക­റും, മയി­ല­മ്മ­യും ഒക്കെ നട­ത്തി­വ­ന്നി­രു­ന്ന (വ­രു­ന്ന) ­സ­മ­രം­ വി­മര്‍­ശ­ന­ങ്ങ­ളെ പ്ര­ശ്ന­ങ്ങ­ളി­ലേ­ക്കു 
ശ്ര­ദ്ധ ക്ഷ­ണി­ക്കാ­നു­പ­യോ­ഗി­ക്കാം. വള­രെ വ്യ­ക്ത­വും സു­ശ­ക്ത­വു­മായ തെ­ളി­വു­ക­ളു­ടെ പിന്‍­ബ­ല­മു­ണ്ടെ­ങ്കില്‍ ബദ­ലു­ക­ളും നിര്‍­ദ്ദേ­ശി­ക്കാം.
 സമൂ­ഹ­ത്തി­ന്റെ അപ്രീ­തി ഭയ­ന്നു് ആവ­ശ്യ­മി­ല്ലാ­ത്ത­തൊ­ന്നും കൂ­ട്ടി­ച്ചേര്‍­ക്കു­ക­യോ, അവ­ശ്യ­കാ­ര്യ­ങ്ങള്‍ വി­ട്ടു­ക­ള­യു­ക­യോ ചെ­യ്ത് വി­മര്‍­ശി­ക്കു­ന്ന­ത്, 
 വി­മര്‍­ശി­ക്കാ­തി­രി­ക്കു­ന്ന­തി­നു തു­ല്യ­മാ­ണു്. അതു സമൂ­ഹ­ത്തി­ലെ ജഡ­ത്വ­ത്തെ ശക്തി­പ്പെ­ടു­ത്തുക മാ­ത്ര­മേ­യു­ള്ളൂ­.

ഇ­തു­വ­രെ സം­രം­ഭ­ങ്ങ­ളെ വി­മര്‍­ശി­ക്കു­ന്ന­വ­രോ­ടോ വി­മര്‍­ശ­നാ­ത്മ­ക­മാ­യി വി­ല­യി­രു­ത്ത­ന്ന­വ­രോ­ടോ സം­രം­ഭ­ക­രും, മി­ക്ക­പ്പോ­ഴും 
ഗു­ണ­ഭോ­ക്താ­ക്ക­ളായ സി­വില്‍ സമൂ­ഹ­വും എടു­ക്കു­ന്ന നി­ല­പാ­ടു­ക­ളെ­ക്കു­റി­ച്ചാ­ണു് പറ­ഞ്ഞ­തു്. വി­മര്‍­ശ­നം സം­രം­ഭ­ങ്ങ­ളെ­പ­റ്റി മാ­ത്ര­മ­ല്ല ഉണ്ടാ­വാ­റ്.
 സാ­മൂ­ഹി­ക/­രാ­ഷ്ട്രീ­യ/­ഭ­രണ സ്ഥാ­പ­ന­ങ്ങ­ളു­ടെ നയ­ങ്ങ­ളെ­യോ, സമൂ­ഹ­ത്തി­ലെ വി­വിധ കീ­ഴ്‌­വ­ഴ­ക്ക­ങ്ങ­ളെ­യോ, ഒക്കെ വി­മര്‍­ശ­ന­വി­ധേ­യ­മാ­ക്കാ­റു­ണ്ട്. 
 പല­പ്പോ­ഴും ഇത്ത­രം വി­മര്‍­ശ­ങ്ങ­ളു­ന്ന­യി­ക്കു­ന്ന­വ­രോ­ടു് മാ­ദ്ധ്യ­മ­സ്ഥാ­പ­ന­ങ്ങ­ള­ട­ക്ക­മു­ള്ള­വ­രു­ടെ (സാ­ധാ­ര­ണ­ഗ­തി­യില്‍ വലിയ വി­മര്‍­ശ­കര്‍ മാ­ദ്ധ്യ­മ­ങ്ങ­ളാ­ണു്)
  സ്ഥി­രം ചോ­ദ്യ­ങ്ങള്‍ രണ്ടാ­ണ്.

ഒ­ന്നു് ബദ­ലി­നെ സം­ബ­ന്ധി­ച്ച­താ­ണു്. വലിയ സാ­മൂ­ഹിക ചല­ന­ങ്ങ­ളു­ണ്ടാ­ക്കാന്‍ കഴി­വു­ള്ള ഒരു സ്ഥാ­പ­ന­ത്തി­ന്റെ നയം ചില ദോ­ഷ­ക­ര­മായ 
പ്ര­ത്യാ­ഘാ­ത­ങ്ങ­ളു­ണ്ടാ­ക്കാന്‍ പോ­ന്ന­താ­ണു് എന്നു ചൂ­ണ്ടി­ക്കാ­ണി­ച്ച­തി­നാ­ണു് ഈ ചോ­ദ്യം എന്നോര്‍­ക്ക­ണം. ഇത്ര­യും വലിയ സ്ഥാ­പ­ന­ത്തി­ന്റെ
 നയ­പ­ര­മായ കാ­ര്യ­ങ്ങ­ളു­ടെ പ്ര­ത്യാ­ഘാ­ത­ങ്ങ­ളെ­ക്കു­റി­ച്ച് എന്തെ­ങ്കി­ലും ധാ­ര­ണ­ക­ളു­ണ്ടെ­ങ്കില്‍ അത്ത­ര­മൊ­രു ചോ­ദ്യം മന­സ്സില്‍ വരാ­നേ പാ­ടി­ല്ലാ­ത്ത­താ­ണ്. 
 ഒരു വ്യ­ക്തി­യു­ടെ അഭി­പ്രാ­യ­ങ്ങ­ളി­ലൂ­ടെ പരി­ഹ­രി­ക്കേ­ണ്ട­ത­ല്ല ഈ പ്ര­ശ്ന­ങ്ങള്‍. പക്ഷെ, അതി­നര്‍­ത്ഥം തെ­റ്റു­കള്‍ ചൂ­ണ്ടി­ക്കാ­ണി­ച്ചു കൊ­ടു­ക്കാന്‍ വ്യ­ക്തി­ക­ളെ അനു­വ­ദി­ക്ക­രു­തെ­ന്ന­ല്ല.

­ര­ണ്ടാ­മ­തു ചോ­ദി­ക്കു­ന്ന ചോ­ദ്യ­മാ­ണു് ഏറ്റ­വും രസ­ക­രം. അതു പ്ര­സ്തുത സാ­മൂ­ഹിക സ്ഥാ­പ­ന­ത്തി­ന്റെ സേ­വ­നം ഉപ­യോ­ഗി­ക്കു­ന്ന­തി­നെ­പ്പ­റ്റി­യാ­ണു്.
 അത്ര­മാ­ത്രം വി­മര്‍­ശ­നാ­ത്മ­ക­മാ­ണു് പ്ര­സ്തുത സ്ഥാ­പ­ന­ത്തി­ന്റെ പ്ര­വൃ­ത്തി­ക­ളെ­ങ്കില്‍ അതി­നെ ഒഴി­വാ­ക്കി ബദ­ലു­കള്‍ തേ­ടി­ക്കൂ­ടെ എന്നാ­ണു ചോ­ദ്യം.
  കാ­ര്യം പറ­ഞ്ഞാല്‍, തീര്‍­ത്തും ബാ­ലി­ശ­വും രസ­ക­ര­വു­മായ ചോ­ദ്യം. അതു ചോ­ദി­ക്കു­ന്ന­തു് ഉത്ത­ര­വാ­ദ­പ്പെ­ട്ട സാ­മൂ­ഹിക വി­മര്‍­ശ­ക­രാ­യി 
  സ്വ­യം മാ­റേ­ണ്ട മാ­ദ്ധ്യ­മ­പ്ര­വര്‍­ത്ത­ക­രാ­വു­മ്പോ­ഴാ­ണു് ഇതി­ലെ അപ­ക­ടം­.

ഈ­യ­ടു­ത്ത­കാ­ല­ത്തു് ഈ രണ്ടു ചോ­ദ്യ­ങ്ങ­ളേ­യും നേ­രി­ടേ­ണ്ടി­വ­ന്ന­തു് അ­രു­ന്ധ­തി റോ­യ് ആണു്. ദേ­ശ­രാ­ഷ്ട്ര­ങ്ങ­ളില്‍ പല­പ്പോ­ഴും 
പാര്‍­ശ്വ­വ­ത്കൃ­തര്‍­ക്കു് നീ­തി കി­ട്ടു­ന്നി­ല്ലെ­ന്നു തു­റ­ന്നു പറ­ഞ്ഞ അവര്‍ കാ­ശ്മീ­രി­ലെ ജന­ങ്ങ­ളു­ടെ സ്വ­യം നിര്‍­ണ്ണ­യാ­വ­കാ­ശ­ത്തെ­പ്പ­റ്റി­യും
 മനു­ഷ്യാ­വ­കാ­ശ­ങ്ങ­ളെ­പ്പ­റ്റി­യും സം­സാ­രി­ച്ച­പ്പോ­ഴാ­യി­രു­ന്നു ഇതു്. ഒരു ദേ­ശ­രാ­ഷ്ട്ര­ത്തി­ന്റെ സു­ര­ക്ഷ­യി­ലും പി­ന്തു­ണ­യി­ലു­മി­രു­ന്നാ­ണു് 
 താന്‍ ഇതൊ­ക്കെ­പ്പ­റ­യു­ന്ന­തെ­ന്നു് അരു­ന്ധ­തി മറ­ക്ക­രു­തെ­ന്നാ­യി­രു­ന്നു ഒരു വാ­രി­ക­യില്‍ വന്ന­തു്. അരു­ന്ധ­തി­യു­മാ­യി മറ്റൊ­രു 
 വാ­രിക നട­ത്തിയ അഭി­മു­ഖ­ത്തി­ലാ­വ­ട്ടെ, ബദ­ലു­ക­ളു­ടെ ചോ­ദ്യ­വും ഉന്ന­യി­ക്ക­പ്പെ­ട്ടു­.

ഈ ചോ­ദ്യ­ങ്ങള്‍ വരു­ന്ന­തു് ചില മുന്‍­വി­ധി­ക­ളില്‍ നി­ന്നാ­ണു്. വി­മര്‍­ശ­ങ്ങള്‍ വരു­ന്ന­തു് പ്ര­സ്തുത സ്ഥാ­പ­ന­മാ­യോ വ്യ­വ­സ്ഥ­യു­മാ­യോ
 സം­രം­ഭ­മാ­യോ കടു­ത്ത എതിര്‍­പ്പി­ലാ­ണെ­ങ്കില്‍ മാ­ത്ര­മാ­ണെ­ന്ന­താ­ണൊ­ന്നു്. മറ്റൊ­ന്നു വി­മര്‍­ശ­നം മറ്റൊ­രു സമ­ര­മാര്‍­ഗ്ഗം
  മാ­ത്ര­മാ­ണെ­ന്ന തെ­റ്റി­ദ്ധാ­ര­ണ­യാ­ണു്. താന്‍ കൂ­ടി ഭാ­ഗ­മായ സമൂ­ഹ­ത്തി­ന്റെ ഉന്ന­മ­ന­ത്തി­നും സാ­മൂ­ഹിക സ്ഥാ­പ­ന­ങ്ങ­ളു­ടെ നല്ല 
  നട­ത്തി­പ്പി­നും അവ­യു­ടെ നട­ത്തി­പ്പി­ലോ നയ­ങ്ങ­ളി­ലോ ഉള്ള തെ­റ്റു­കള്‍ പരി­ഹ­രി­ച്ച് മു­ന്നോ­ട്ടു പോ­ക­ണ­മെ­ന്ന ആഗ്ര­ഹം, അല്ല­ങ്കില്‍ 
  നട­ത്തി­പ്പി­ലെ അപാ­ക­ത­കള്‍ പരി­ഹ­രി­ക്ക­പ്പെ­ട­ണ­മെ­ന്ന ആഗ്ര­ഹം, വി­മര്‍­ശ­കര്‍­ക്കു­ണ്ടാ­വു­മെ­ന്നു് പലര്‍­ക്കും സ്വ­പ്നം പോ­ലും കാ­ണാന്‍ 
  കഴി­യു­ന്നി­ല്ല.

അ­തു­കൊ­ണ്ടു തന്നെ­യാ­ണു് നമ്മ­ളില്‍ പലര്‍­ക്കും താന്‍ ­കാ­ശ്മീര്‍ സ്വ­ത­ന്ത്ര­മാ­ക്ക­ണ­മെ­ന്നും ഇന്ത്യ വെ­ട്ടി­മു­റി­ക്ക­ണ­മെ­ന്നു­മ­ല്ല 
വാ­ദി­ക്കു­ന്ന­തെ­ന്നും, ഇന്ത്യ എന്ന ദേ­ശ­രാ­ഷ്ട്രം അതി­ന്റെ ഭാ­ഗ­മാ­യി­കാ­ണു­ന്ന കാ­ശ്മീ­രി­ലെ ജന­ത­യോ­ടു ചെ­യ്ത­തു് / ചെ­യ്യു­ന്ന­തു് മാ­നു­ഷി­ക­പ­ര­മാ­യി 
നീ­തി­യ­ല്ലെ­ന്നും, അവ­രു­ടെ സ്വ­യം നിര്‍­ണ്ണ­യാ­വ­കാ­ശ­ത്തെ­യും, മനു­ഷ്യാ­വ­കാ­ശ­ങ്ങ­ളെ­യും മാ­നി­ക്ക­ണ­മെ­ന്നു് ആവ­ശ്യ­പ്പെ­ടു­ക­യാ­ണു് ചെ­യ്ത­തെ­ന്നും
 അരു­ന്ധ­തി റോ­യ് പറ­യു­ന്ന­തു് ദഹി­ക്കാ­ത്ത­തു്. അരു­ന്ധ­തി റോ­യ്, 'കാ­ശ്മീ­രില്‍ ഇന്ത്യ പെ­രു­മാ­റു­ന്ന­തു് ഒരു അധി­നി­വേശ ശക്തി­യെ­പ്പോ­ലെ­യാ­ണെ­ന്നു' 
 പറ­യു­മ്പോള്‍ അവര്‍ ദേ­ശ­ദ്രോ­ഹി­യാ­യാ­ണു് മു­ദ്ര­കു­ത്ത­പ്പെ­ടു­ന്ന­തു്. പക്ഷേ, അവര്‍ താന്‍ നേ­രി­ട്ടു കണ്ട തെ­ളി­വു­കള്‍ കൊ­ണ്ടു പറ­യു­ന്ന­തി­നെ
  സാ­ക്ഷ്യ­പ്പെ­ടു­ത്തു­മ്പോള്‍ അതു് ഒരു നീ­തി­പൂര്‍­വ്വ സമൂ­ഹം പു­ല­രു­ന്ന ജനാ­ധി­പ­ത്യ­രാ­ജ്യ­മെ­ന്ന നി­ല­യില്‍ ഇന്ത്യ­യെ മെ­ച്ച­പ്പെ­ട്ട ഭാ­വി­യി­ലേ­ക്കു് 
  നയി­ക്കാ­നാ­ണെ­ന്നു മന­സ്സി­ലാ­ക്കാന്‍ പലര്‍­ക്കും കഴി­യാ­തെ­പോ­കു­ന്ന­തു്, വി­മര്‍­ശ­ങ്ങ­ളെ­ക്കു­റി­ച്ചു­ള്ള മുന്‍­വി­ധി­കള്‍ കാ­ര­ണ­മാ­ണു്.  
  തന്നെ വി­മര്‍­ശി­ക്കു­ന്ന ആരേ­ക്കാ­ളും ഇന്ത്യ­യെ താന്‍ സ്നേ­ഹി­ക്കു­ന്നെ­ന്നും തന്റെ രാ­ജ്യ­ത്തില്‍ നീ­തി­പൂര്‍­വ്വക സമൂ­ഹം പു­ലര്‍­ന്നു കാ­ണാ­നു­ള്ള 
  ആഗ്ര­ഹ­മാ­ണു തന്റെ വി­മര്‍­ശ­ത്തി­നു പി­ന്നി­ലെ­ന്നും അവര്‍ പറ­യു­മ്പോള്‍ അതു മന­സ്സി­ലാ­ക്കാന്‍ നമു­ക്കു കഴി­യാ­ത്ത­തും അതു­കൊ­ണ്ടു തന്നെ­.

­സാ­മൂ­ഹ്യ­വി­മര്‍­ശ­ന­മെ­ന്ന­തു് നീ­തി­പൂര്‍­വ്വ­ക­മാ­യൊ­രു സമൂ­ഹം വാര്‍­ത്തെ­ടു­ക്കാ­നു­ള്ള ശക്ത­മായ ആയു­ധ­മാ­ണു്. ദേ­ശ­രാ­ഷ്ട്ര­ങ്ങ­ളി­ലെ 
അധി­കാ­ര­കേ­ന്ദ്ര­ങ്ങള്‍ തങ്ങ­ളു­ടെ ജന­ത­യി­ലെ പല വി­ഭാ­ഗ­ങ്ങ­ളെ­യും പാര്‍­ശ്വ­വ­ത്ക­രി­ക്കു­ക­യും അവ­ഗ­ണി­ക്കു­ക­യും പല­പ്പോ­ഴും അവ­രു­ടെ 
മൌ­ലി­കാ­വ­കാ­ശ­ങ്ങള്‍ പോ­ലും കവര്‍­ന്നെ­ടു­ക്കു­ക­യും ചെ­യ്യു­മ്പോള്‍ സാ­മൂ­ഹ്യ വി­മര്‍­ശ­നം നാം ഓരോ­രു­ത്ത­രു­ടെ­യും കട­മ­യാ­യി മാ­റു­ക­യാ­ണു്. 
എന്നാല്‍ വി­മര്‍­ശ­നം അവ­സാ­ന­മ­ല്ല, അതൊ­രു ദീര്‍­ഘ­മേ­റി­യ­തും ദുര്‍­ഘ­ടം പി­ടി­ച്ച­തു­മായ പാ­ത­യു­ടെ തു­ട­ക്കം മാ­ത്ര­മാ­ണു്.

അ­ധി­കാ­ര­ത്തി­നെ­തി­രെ­യു­ള്ള സമ­ര­ങ്ങ­ളും, വി­വിധ ബോ­ധ­വ­ത്ക­രണ പ്ര­ച­രണ പരി­പാ­ടി­ക­ളും, എല്ലാം വി­മര്‍­ശ­ങ്ങള്‍­ക്കു പി­റ­കേ വര­ണം.
 അവ­യി­ലും സജീ­വ­മായ ഇട­പെ­ടല്‍ വി­മര്‍­ശ­ക­രു­ടെ ഭാ­ഗ­ത്തു­നി­ന്നു­ണ്ടാ­വ­ണം. അനീ­തി­യില്‍ നി­ന്നും നീ­തി­പൂര്‍­വ്വ­ക­മായ ബദ­ലു­ക­ളി­ലേ­ക്കു് 
 നയി­ക്കാന്‍ അവ­ശ്യം വേ­ണ്ട പൊ­തു­ശ്ര­ദ്ധ­യും ചര്‍­ച്ച­ക­ളും സ്വ­യം വി­മര്‍­ശ­നാ­ത്മ­ക­മാ­യി വി­ല­യി­രു­ത്തു­ന്ന സമൂ­ഹ­ത്തില്‍ വള­രെ എളു­പ്പം നട­ക്കും. 
 അതു സ്വാ­ഭാ­വി­ക­മായ ജഡ­ത്വം വെ­ടി­ഞ്ഞു് ചല­നാ­ത്മ­ക­വും നീ­തി­പൂര്‍­വ്വ­ക­വു­മായ ഒന്നാ­യി മാ­റു­ന്ന­തി­ലേ­ക്കു് സമൂ­ഹ­ത്തെ സഹാ­യി­ക്കു­ക­യും ചെ­യ്യും­.

(Feb 15, 2011)\footnote{http://malayal.am/വാര്‍ത്ത/വിശകലനം/10054/സാമൂഹ്യ-വിമര്‍ശനത്തെക്കുറിച്ച്}
\newpage
