\newpage
\secstar{മുഖക്കുറി}
വേണ്ടപ്പെട്ടവര്‍ വിട്ടുപോകുമ്പോഴാണു് ദുഃഖിതരാകുന്ന ബന്ധുക്കളും പിരിഞ്ഞുപോയവരുടെ നന്മകളും
കഴിവുകളും കൂടുതലായി തിരിച്ചറിയുന്നതു്. വേര്‍പിരിഞ്ഞ വ്യക്തിയുടെ സ്വഭാവത്തിലെ അസാധാരണതകളും വൈരുദ്ധ്യങ്ങളും ഒരുപോലെ
ചിന്താവിഷയമാകുന്നു. അവര്‍ പൊതുസമൂഹത്തിനു് നല്കിയ സംഭാവനകള്‍ ഓരോന്നായി ഓര്‍ത്തെടുക്കുന്നു. അങ്ങനെ ഒന്നാണു്
``നിരീക്ഷകന്റെ കുറിപ്പുകള്‍''. ബാക്കിവച്ചതു് മുഴുമിക്കുകയും തുടര്‍ച്ച ഉണ്ടാക്കുകയും ചെയ്യുന്നതിനു് ഇത്തരം പരിശ്രമങ്ങള്‍ സഹായകരമാകും.
നേരത്തു് വരികയും നേരത്തെ പോകുകയും ചെയ്യുന്നതു് ജ്ഞാനികളുടെ രീതിയാണു്. ജിനേഷിന്റെ വേര്‍പിരിയലിനു് ഒരു വര്‍ഷം പൂര്‍ത്തിയാകുന്നു.

സൌമ്യദീപ്തമായ പെരുമാറ്റത്തിലൂടെ ആളുകളെ ആകര്‍ഷിക്കുന്ന വ്യക്തിത്വമായിരുന്നു ജിനേഷിന്റെതു്. താന്‍ എന്താണു് ചെയ്യുന്നതെന്നും
എന്തിനുവേണ്ടിയാണു് ജീവിക്കുന്നതെന്നും നല്ലതുപോലെ നിശ്ചയമുണ്ടായിരുന്നു. തനിയ്ക്കു് സമൂഹം നല്കിയതിലധികം തിരിച്ചുനല്കണമെന്നു്
വിശ്വസിക്കുകയും അതിനായി പ്രയത്നിക്കുകയും ചെയ്തു. അതു് പറയുകയും അതിനായി തയ്യാറെടുക്കുകയും ചെയ്തു. അതിന്റെ സ്ഫുരണങ്ങള്‍
ജിനേഷിന്റെ കര്‍മ്മകാണ്ഡത്തിലുടനീളം ദൃശ്യവുമായിരുന്നു. അതുതന്നെയാണു് ജിനേഷിനു് ഒരു വലിയ സുഹൃത്-സഹപ്രവര്‍ത്തകസംഘത്തെ
നല്കിയതു്. മറ്റുള്ളവരുടെ ജീവിതത്തിനു് കൂടുതല്‍ നിറവും മണവും നല്‍കുക എന്ന ശ്രേഷ്ഠമായ ലക്ഷ്യത്തോടെ ആ ജീവിതം വെളിച്ചം വിതറി കടന്നുപോയി.

തെളിഞ്ഞ ചിന്തയും തെളിഞ്ഞ ഭാഷയും സ്വായത്തമാക്കിയ ജിനേഷ് സ്വതന്ത്ര സോഫ്റ്റ്‌വെയറും യുണികോഡും മാത്രമല്ല പഠനവിഷയമാക്കിയതു് 
പിണറായിയുടെ മക്കളുടെ വിദ്യാഭ്യാസം, ക്രിക്കറ്റ്, ദേശീയത, പണം, ഐ.പി.എല്‍, സാരി, കാറോട്ടം, ഫോര്‍മുല വണ്‍, അംബാനി, ആഫ്റ്റര്‍ 
മാച്ച് പാര്‍ട്ടി എന്നിങ്ങനെ വിവിധ വിഷയങ്ങളില്‍ വ്യാപരിക്കുകയും പ്രതികരിക്കുകയും ചെയ്തു. വര്‍ഷങ്ങള്‍ക്കുമുമ്പുതന്നെ ഇന്റര്‍നെറ്റിന്റെയും 
ബ്ലോഗിന്റെയും സാദ്ധ്യതകള്‍ ജിനേഷ് തിരിച്ചറിഞ്ഞു.  സ്പോര്‍ട്സും സാമൂഹ്യനീതിയും അദ്ദേഹത്തിന് ഒരുപോലെ വഴങ്ങുമായിരുന്നുവെന്നു് 
എഴുത്തുകളിലൂടെ വ്യക്തമാക്കി. ചിന്തകളിലും വാക്കുകളിലും ലാളിത്യം വച്ചുപുലര്‍ത്താന്‍ അദ്ദേഹം എപ്പോഴും നിഷ്കര്‍ഷിച്ചിരുന്നു. 

തേച്ചുമിനുക്കിയ ചിന്തയുടെ അവതരണമാണു് ജിനേഷിന്റെ മാനസികവ്യാപാരങ്ങളായ നിരീക്ഷകന്റെ കുറിപ്പുകള്‍. അതില്‍ കാല്പനിക കൌമാരത്തിന്റെ 
ചപലതകള്‍ അത്രമേല്‍ ദൃശ്യമല്ല. സുശിക്ഷിതമായ ആത്മസംയമനം മിക്കപ്പോഴും കാണാം. ഒരു ശാസ്ത്രജ്ഞന്റെ സൂക്ഷ്മതയോടെ, അവധാനതയോടെ 
കാര്യങ്ങളെയും സംഭവങ്ങളെയും നിരീക്ഷിക്കാനും സൌന്ദര്യാത്മകമായി രേഖപ്പെടുത്താനും അദ്ദേഹത്തിനു് സാധിച്ചിട്ടുണ്ടു്. സമൂഹത്തിലെ 
കൊള്ളരുതായ്മകളോടു് തീവ്രമായ പ്രതിഷേധം ഉള്ളിലുള്ളപ്പോഴും തികഞ്ഞ പ്രത്യാശ എഴുത്തില്‍ ദൃശ്യമാണു്. വിഷയങ്ങളിലെ വൈവിദ്ധ്യവും 
ഭാഷാപ്രയോഗവും ഭാവനാസമ്പന്നതയും ഈ കുറിപ്പുകളുടെ മുഖമുദ്രയാണു്. ജ്ഞാതരും അജ്ഞാതരുമായ വായനക്കാര്‍ക്കുമുന്നില്‍ നിരന്തരം 
വിവിധവിഷയങ്ങളില്‍ പ്രതികരിക്കുക എന്നത് അത്ര എളുപ്പമുള്ള കാര്യമല്ലല്ലോ.

ബുദ്ധിജീവികളുടെ പതിവുപ്രതിസന്ധികള്‍ ജിനേഷിനെ ഒരിക്കലും ബാധിച്ചില്ല. സ്വതന്ത്ര സോഫ്റ്റ്‌വെയറിന്റെ അന്തഃസത്ത ശരിയാംവണ്ണം 
ഉള്‍ക്കൊണ്ട ജിനേഷ് തന്റെ കര്‍മ്മമണ്ഡലങ്ങളുമായി അതിനെ ബന്ധിപ്പിച്ചു. പറയുന്നതുപോലെ ചെയ്യുന്ന, ചെയ്യുന്നതുപോലെ പറയുന്ന 
 അപൂര്‍വ്വമായിക്കൊണ്ടിരിക്കുന്ന രണ്ടു് ഗുണങ്ങളും ജിനേഷിനു് ഒരേസമയം ഉണ്ടായിരുന്നു. വിധിയും ജീവിതവും തമ്മിലുള്ള 
ഏകപക്ഷീയമായ മത്സരത്തില്‍ തോല്‍വി ഉറപ്പാക്കിക്കഴിഞ്ഞപ്പോഴും തന്റെ പ്രവര്‍ത്തനങ്ങള്‍ പരമാവധി തുടര്‍ന്നു.  ഈ ചെറിയവട്ടത്തില്‍ 
ജീവിച്ചുപോകുന്നവര്‍ക്കു് തന്റെ ജീവന്റെ അടയാളങ്ങള്‍ ഭൂമിയില്‍ രേഖപ്പെടുത്തുക എന്നതു് പ്രയാസകരമായ ഒന്നാണു്. എന്നാല്‍ ജിനേഷിനു് 
അതു് സാധിച്ചിരിയ്ക്കുന്നു. അതിന്റെ സാക്ഷ്യപത്രമാണു് ഈ സമാഹാരം.

ഈ കുറിപ്പുകളിലുടനീളം തെളിഞ്ഞുകാണുന്നതു് ജിനേഷിന്റെ ചിന്തയുടെ തെളിമയും സമഗ്രതയുമാണു്. ലളിതമാണു് വാചകങ്ങളും അതുപോലെ ഘടനയും. 
വളച്ചുകെട്ടലില്ലാത്ത പ്രയോഗങ്ങള്‍, ആഖ്യാനത്തിലെ തെളിമ, ശാസ്ത്രീയ അപഗ്രഥനം, ആധുനികശാസ്ത്രവും സാങ്കേതികവിദ്യയും സമന്വയിപ്പിച്ച 
ചിന്തകള്‍ - ഇതെല്ലാം ഈ എഴുത്തിന്റെ മുഖമുദ്രയാണു്. ജിനേഷിന്റെ എഴുത്തുകളുടെ സംയോജനത്തിനു് നേതൃത്വം നല്‍കിയ എല്ലാ സുമനസ്സുകള്‍ക്കും 
നല്ലതു്, നല്ലത്, നല്ലതു് മാത്രം നേരുന്നു. 

\begin{flushright}ജവഹര്‍ \\(ഇന്ത്യന്‍ ഇന്‍സ്റ്റിറ്റ്യൂട്ട് ഓഫ് ഇന്‍ഫോര്‍മേഷന്‍ ടെക്നോളജി ഹൈദരാബാദില്‍ അദ്ധ്യാപകന്‍) \end{flushright}
\newpage
