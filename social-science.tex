\secstar{എന്റെ ചില സാമൂഹ്യശാസ്ത്ര ചിന്തകള്‍}
\vskip 2pt

ഏഴാംക്ലാസ്സിലെ പാഠപുസ്തകത്തെപ്പറ്റി പണ്ടു് ഞാന്‍ ഏഴാംക്ലാസ്സില്‍ പഠിച്ച ഓര്‍മ്മവച്ചു് എഴുതുന്നത് ശരിയാണൊ എന്നറിയില്ല. കാരണം,
എനിക്കു് ഇപ്പോഴും കൃത്യമായി അറിയില്ല ഞാന്‍ ഏഴാംക്ലാസ് സാമൂഹ്യശാസ്ത്രത്തിലെന്താണു് പഠിച്ചതെന്നു്! കുറെ ഇന്ത്യന്‍ ഹിസ്റ്ററിയും 
(മെഡീവല്‍ ഇന്ത്യ ആണെന്നൊരോര്‍മ്മ), ജ്യോഗ്രഫിയും, പിന്നെക്കുറച്ചു് സിവിക് സ്റ്റഡീസും (ഫണ്ടമെന്റല്‍ ഡ്യൂട്ടിസ്, ഫണ്ടമെന്റല്‍ റൈറ്റ്സ്, 
ഭരണഘടന, ദേശീയനയങ്ങള്‍ അങ്ങനെയെന്തൊക്കെയോ). സി.ബി.എസ്.സി. സിലബസ്സില്‍ ഒരു സീരീസ്സായിട്ടാണു് ഇതൊക്കെ പഠിപ്പിച്ചിരുന്നതു്. 
ആറുമുതല്‍ പത്തുവരെയുള്ള ക്ലാസ്സുകളില്‍ കൃത്യമായി വിഭജിച്ചു് പുസ്തകങ്ങളുണ്ടായിരുന്നു. ആന്‍ഷ്യന്റ് ഇന്ത്യ, മെഡീവല്‍ ഇന്ത്യ, മെഡീവല്‍ വേള്‍ഡ്, 
മോഡേണ്‍ വേള്‍ഡ് എന്നൊക്കെ ഹിസ്റ്ററിയിലും, യൂറോപ്പും ആഫ്രിക്കയും അമേരിക്കയുമൊക്കെയായി ജ്യോഗ്രഫിയിലും അതങ്ങനെ പടര്‍ന്നുകിടന്നു. 
വേണമെന്നു വായിച്ചു് വിവാദമുണ്ടാക്കാനുള്ള ഒരു വെടിക്കുള്ള മരുന്നു് ആ പുസ്തകങ്ങളിലുണ്ടായിരുന്നു എന്നാണെന്റെ ചെറിയ ഓര്‍മ്മ. പുതിയ ബോധനരീതിയുടെ ഇന്ററാക്ടീവു് രീതിയിലുള്ളതാവാഞ്ഞതുകൊണ്ടും, പുസ്തകങ്ങളെഴുതിയവരും റിവ്യു ചെയ്തവരും അപാരബുദ്ധിജീവികളായതുകൊണ്ടും, ആര്‍ക്കും വായിച്ചു മനസ്സിലാവാഞ്ഞതുകൊണ്ടുമൊക്കെയായിരിക്കാം ആരും ഒന്നും പറയാഞ്ഞതു്. പിന്നെ, സോഷ്യല്‍ സയന്‍സ് ഭൂരിഭാഗം കുട്ടികളും പരീക്ഷ ജയിക്കാന്‍ വേണ്ടിമാത്രം പഠിച്ചിരുന്ന ഒരു വിഷയമായതുകൊണ്ടുമാകാം. എന്തായാലും, അദ്ധ്യാപകനോ അദ്ധ്യാപികയ്ക്കോ പ്രത്യേകിച്ചൊന്നും ചെയ്യാനില്ലാത്ത തരത്തില്‍ 
ആ പുസ്തകങ്ങളെല്ലാം ഒരു ഫാക്റ്റ് ഫയല്‍ മാത്രമായിരുന്നു. വിഷയത്തെ നേര്‍രേഖയിലൂടെ കാണിച്ചു് ഒരു വ്യതിചലനത്തിനും ഇടംകൊടുക്കാതെ 
പഠിപ്പിക്കാവുന്നവയായിരുന്നു ആ പാഠങ്ങള്‍. എന്നാല്‍, പുസ്തകം വായിച്ചു് ഒരഭിപ്രായം രൂപീകരിച്ചു് എഴുതാന്‍ പറഞ്ഞാല്‍, മതവിശ്വാസത്തെയും സിസ്റ്റങ്ങളെയും കൃത്യമായി എതിര്‍ക്കുന്ന, അല്ലെങ്കില്‍ ഓരോ മതത്തിന്റെയും ജനനത്തിന്റെ കാലഘട്ടവും നടത്തിയ രക്തച്ചൊരിച്ചിലും, അവ വളര്‍ത്തിയ സംസ്കാരവും വിശദമായി പുസ്തകം പ്രതിപാദിച്ചിരുന്നു. ഓരോ മതരീതികളെയും സംസ്കാരമായി എടുത്തുകാട്ടി, അവയുടെ അധഃപധനം വിവരിച്ചിരുന്ന രീതി വായിച്ചുകഴിഞ്ഞാല്‍, ഇന്നുള്ള സംവിധാനങ്ങള്‍ വെറും കെട്ടുകാഴ്ചകള്‍ മാത്രമാണെന്നു് മനസ്സിലാക്കാമായിരുന്നു. ഇസ്ലാമിന്റെയും ക്രിസ്ത്യാനിറ്റിയുടെയും ജുഡായിസത്തിന്റെയും ആദ്യകാലരീതികളും, പിന്നീടു് അവയില്‍ വന്നമാറ്റങ്ങളും എല്ലാം ഏതു പള്ളിയേയും പിടിച്ചുകുലക്കാന്‍ പറ്റിയ രീതിയില്‍ത്തന്നെ വേണമെങ്കില്‍ ഒരാള്‍ക്കു പഠിപ്പിക്കാനുള്ള വക ആ പുസ്തകങ്ങളിലുണ്ടായിരുന്നു. എന്നാല്‍, കുട്ടികള്‍ക്കുള്ള ചോദ്യങ്ങളും വര്‍ക്കുകളും, കാലഘട്ടങ്ങളെയും പ്രസ്ഥാനങ്ങളെയും സംബന്ധിച്ച പ്രബന്ധരചനയായിരുന്നുവെന്നുമാത്രം. 
വിരസമായ അക്കാദമിക് എഴുത്തിന്റെ ലോകമായിരുന്നു അന്നു് സാമൂഹ്യശാസ്ത്രം. പോരാഞ്ഞിട്ടു് കാണാപ്പാഠം പഠിച്ചു് പരീക്ഷയ്ക്കു് എഴുത്തും.

ഈ പുസ്തകം കണ്ടപ്പോള്‍ ആ ഭീകരമായ സാമൂഹ്യശാസ്ത്രപഠനത്തെ കുറിച്ചോര്‍ത്തു് എനിക്കൊരിത്തിരി സങ്കടംവന്നു. സാമൂഹ്യശാസ്ത്രത്തിലെ വിരസത ഒട്ടൊന്നൊഴിവായല്ലൊ എന്നൊരു സമാധാനവും. പക്ഷെ, ഞാന്‍ കണ്ട (വായിച്ച, പഠിച്ച എന്നൊക്കെ പറയാനെന്താമടി എന്നതിനു് എന്റെ സോഷ്യല്‍ സയന്‍സ് മാര്‍ക്കുകള്‍ മറുപടി പറയും) പുസ്തകങ്ങളിലില്ലാതിരുന്ന ഒരു കാര്യം ഇപ്പോ വന്നു. ഒരേ പുസ്തകം വിവിധ അദ്ധ്യാപകരുടെ കീഴില്‍ പഠിക്കുന്ന കുട്ടികള്‍, തങ്ങളുടെ കാഴ്വപ്പാടുകളെ തിരിച്ചറിയുന്നതിനേക്കാള്‍ അദ്ധ്യാപകന്റെ കാഴ്ചപ്പാടുകളെ മനസ്സിലാക്കുന്നതിനുള്ള ഒരു സാധ്യത, അല്ലെങ്കില്‍ അദ്ധ്യാപകനു് പഠനത്തില്‍ കൂടുതല്‍ ഇടപെടാനുള്ള സാഹചര്യം. പത്താംക്ലാസ് കഴിയുമ്പോള്‍ സാമൂഹ്യശാസ്ത്രത്തില്‍നിന്നു്, സമൂഹവീക്ഷണവും സമൂഹത്തിന്റെ രീതിശാസ്ത്രവും അഭ്യസിക്കാത്ത ഞങ്ങള്‍ക്കു പകരം, സ്വന്തം കുടുംബത്തില്‍നിന്നും അദ്ധ്യാപകരില്‍നിന്നും സമൂഹത്തില്‍നിന്നും പാഠപുസ്തകംവഴി ഒരു സമൂഹവീക്ഷണം കണ്ടെത്താനും, സ്വയം ഒരു രീതിശാസ്ത്രം (അവ അനുകരണമോ സ്വന്തമോ എന്നതു് 
ഓരോരുത്തര്‍ക്കനുസരിച്ചിരിക്കും) കൈമുതലായുള്ള ഒരു തലമുറ. അദ്ധ്യാപകനു് ഒരുപാടു് ഇടംനല്‍കുന്ന ഈ പുസ്തകങ്ങളിലൂടെ ഉണ്ടായേക്കാവുന്ന 
തിക്തഫലങ്ങളെ ഒഴിവാക്കാന്‍ സുസജ്ജമായ ഒരു അദ്ധ്യാപകസമൂഹം കേരളത്തിലുണ്ടാവണം.

സ്കൂള്‍കാലങ്ങളെക്കുറിച്ചിത്തിരി. ഞാന്‍ അഞ്ചാംക്ലാസ്സിനുശേഷം ബോര്‍ഡിങ്ങിലാണു് പഠിച്ചതു്. അവിടെ മതവിശ്വാസവും ജാതിയുമൊന്നും ഒരു കാര്യമല്ലായിരുന്നു. ജാതി ചോദിക്കരുതു് പറയരുതു് സ്റ്റൈലായിരുന്നു. അഞ്ചുവരെ പഠിച്ച സ്കൂളിലും, ഞങ്ങള്‍ക്കു് ജാതിയും മതവുമല്ലായിരുന്നു. വലിയ കാര്യം, വൈകുന്നേരത്തെ കളിയും കോപ്പി എഴുതലും സിനിമാക്കഥ പറയലുമൊക്കെത്തന്നെയായിരുന്നു. പിന്നെ, കൊച്ചുവഴക്കുകളും. അതിലൊന്നും ജാതിയും മതവുമല്ല ആശയം നല്‍കിയിരുന്നതു്, ജീവിതമായിരുന്നു. ഇന്നും വിദ്യാലയങ്ങളില്‍ അത്തരം അവസ്ഥയുണ്ടെങ്കില്‍ കുട്ടികള്‍ എന്തു സംഭവിച്ചാലും, ഇന്നുള്ളതുപോലെയൊക്കെത്തന്നെ വളര്‍ന്നുവന്നോളും. സ്കൂളില്‍ പഠിക്കുന്ന പാഠം എങ്ങനെയൊക്കെ സ്വാധീനിച്ചാലും ഒരു സാമൂഹ്യവിപ്ലവത്തിനുള്ള വഴിമരുന്നിടാന്‍ അതിനാവുമോ എന്നെനിക്കറിയില്ല. ഒരു പക്ഷേ, ഇപ്പോള്‍ കാണിക്കുന്ന 
പ്രതിഷേധപേക്കൂത്തുകള്‍ക്കു പകരം, വിമര്‍ശനവിധേയമായ പഠനത്തിനുശേഷം ഒരു നിലപാടെടുക്കണമെന്നെങ്കിലും കുട്ടികളെ ഉത്ബോധിപ്പിക്കാന്‍ പാഠങ്ങള്‍ക്കു കഴിയട്ടെ എന്നൊരു പ്രാര്‍ത്ഥന. കുട്ടികള്‍ സ്വയം തിരിച്ചറിയാനും മനസ്സിലാക്കാനും പ്രേരിപ്പിക്കുന്ന പാഠ്യപദ്ധതി നല്ലതുതന്നെ, പക്ഷെ, നേര്‍വഴിക്കു നയിക്കാന്‍, അല്ലെങ്കില്‍ സംശയങ്ങള്‍ നിവൃത്തിക്കാന്‍ വ്യക്തമായ സംവിധാനങ്ങളില്ലെങ്കില്‍ ഈ ഉദ്യമം ഒരു അരാജകസമൂഹത്തിന്റെ രൂപീകരണത്തെ ത്വരിതപ്പെടുത്തുമോ എന്നൊരു സംശയം!

\subsection*{പ്രതികരണങ്ങള്‍}

\begin{enumerate}
 \item{ഡാലി}

ജിന്‍സ് ഈ ബ്ലോഗ് (\url{http://scertkerala.wordpress.com/}) കണ്ടിരുന്നോ?  ജിന്‍സിന്റെ പോസ്റ്റ് 
ഇവിടെ (\url{http://scertkerala.wordpress.com/2008/06/27/13/}) ലിങ്ക് ചെയ്തിട്ടുണ്ടു്.
\item{jinsbond007}

ഡാലി ചേച്ചി, ബ്ലോഗ് ഞാന്‍ കണ്ടിരുന്നു. അവിടെയുള്ള ഏതാണ്ടെല്ലാ ചര്‍ച്ചകളും വായിക്കുകയും ചെയ്തു. 
എന്റെ ചില അഭിപ്രായങ്ങള്‍ ഞാന്‍ വേറൊരു പോസ്റ്റായി ഇട്ടിട്ടുണ്ടു്. പിന്നെ ചര്‍ച്ചകളൊക്കെ സംഗ്രഹിക്കാനുള്ള 
ഒരു ശ്രമവും നല്ലതാണു്.

\end{enumerate}

\newpage
