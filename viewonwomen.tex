\secstar{മലയാളിയുടെ പെണ്‍കാഴ്ചയെപ്പറ്റി...}
\vskip 2pt

നമത് വാഴ്വും കാലത്തിന്റേയും\footnote{\url{http://disorderedorder.blogspot.com/2008/01/blog-post_7837.html}}, 
അവിടെക്കണ്ട കൊച്ചുത്രേസ്യയുടെ കമന്റിന്റെയും വെളച്ചത്തില്‍ ഒരു കേരളീയ യുവാവിന്റെ കുറിപ്പു്. 
അന്നു് കൊച്ചുത്രേസ്യയെ തുറിച്ചുനോക്കിയവരില്‍ ഞാന്‍ ഉണ്ടാവാഞ്ഞതു് ഞാന്‍ കൊച്ചിയില്‍ ഇല്ലാതിരുന്നതു 
കൊണ്ടാണെന്നു് ആദ്യമേ പറയട്ടെ. എന്താച്ചാല്‍, സത്യത്തിന്റെ മുഖം വികൃതമാണെങ്കിലും 
തുറന്നു പറയുന്നതാണു് നല്ലതെന്നു് ഞാന്‍ പഠിച്ചുവരികയാണു്. എന്നെപ്പോലെ, വീട്ടിലെ പെണ്ണുങ്ങള്‍ ഉപദേശിച്ചു 
നന്നാക്കാന്‍ ശ്രമിച്ചു് പരാജയപ്പെട്ടു്, അല്ലെങ്കില്‍ അവരെ തെറ്റിദ്ധരിപ്പിച്ചു്, നാട്ടിലെ പെണ്ണുങ്ങള്‍ക്കു് പിടികൊടുക്കാതെ 
നടക്കുന്ന പയ്യന്‍മാരാവണം ആ കൂട്ടത്തിലേറെയും.

എന്താണു് ഞങ്ങളുടെ കാതലായ പ്രശ്നം എന്നൊന്നും ചോദിച്ചാല്‍ മറുപടി തരാന്‍ എനിക്കാവില്ല, കാരണം അങ്ങനെ ഒന്നില്ല 
എന്നതുതന്നെ. വിലങ്ങിട്ടു് നിര്‍ത്താന്‍ ഒരു 'ഗേള്‍ ഫ്രണ്ടോ', അല്ലെങ്കില്‍ ശക്തമായ ഒരു മാതൃസാന്നിദ്ധ്യമോ ഇല്ലാത്ത എല്ലാ 
കേരളീയ യുവാക്കളും എന്നെപ്പോലെത്തന്നെയാണു് എന്നാണു് എന്റെ ഇത്രയും നാളത്തെ അനുഭവസാക്ഷ്യം. ദൈവത്തിന്റെ 
സ്വന്തം നാട്ടില്‍ കാമഭ്രാന്തുമായി ജീവിക്കുന്നവരാണു് ഞങ്ങള്‍ എന്നു് എഴുതിത്തള്ളുന്നതിനുമുമ്പു്, എന്തു സാമൂഹിക സാഹചര്യമാണു് 
ഞങ്ങളെ ഇങ്ങനെയാക്കിത്തീര്‍ത്തതു് എന്നു് മനസ്സിലാക്കിത്തരൂ. ഒരു വിഭാഗം ആളുകള്‍ ഒരേതരത്തിലുള്ള 
ദുശ്ശീലങ്ങള്‍ക്കടിമകളാവുന്നുണ്ടെങ്കില്‍ പ്രശ്നം വ്യക്തിക്കല്ല സമൂഹത്തിനാണു് എന്നാണു് എനിക്കു തോന്നുന്നതു്. 
മദ്യത്തിനെതിരെ ഘോരഘോരം പ്രസംഗിച്ചു് രാത്രിയില്‍ നാല്‍ക്കാലില്‍ വീട്ടിലെത്തുന്ന, അല്ലെങ്കില്‍ വീട്ടിലെ അലമാരയില്‍നിന്നു് 
ബ്ലാക്ക് ലേബല്‍ വീശിയാല്‍ മാത്രം ഉറക്കംവരുന്ന സംസ്കാരത്തിന്റെ ബാക്കിപത്രമാണോ? അമിതമായ 
വിലക്കുകളിലൂടെ തെറ്റായ വിദ്വേഷങ്ങളും ആസക്തികളും വളര്‍ത്തുന്ന ആത്മീയ വിദ്യാഭ്യാസസ്ഥാപനങ്ങളുടെ 
സംഭാവനയോ? അതോ, എതിര്‍ലിംഗത്തിലെ ന്യൂനപക്ഷത്തിനെക്കണ്ടു് ഒരു വിഭാഗത്തിനെ മുഴുവന്‍ തെറ്റിദ്ധരിച്ചതോ? 
സമൂഹത്തിനെ കുറ്റം പറഞ്ഞു് സമാധാനിക്കുന്നതിലുപരി, ഏതേതു സാഹചര്യമാണു് മാറേണ്ടതു് എന്നു് മനസ്സിലാക്കാനാണു് 
എന്റെ ശ്രമം. സമൂഹത്തിന്റെ നെടുംതൂണു് വ്യക്തിയാണെന്നും, സമൂഹം മാറണമെങ്കില്‍ മാറേണ്ടതു് 
വ്യക്തിയാണെന്നും തിരിച്ചറിയുന്നു ഞാന്‍. അതുകൊണ്ടു്, ഒരു വ്യക്തിയെന്ന നിലയില്‍ മാറ്റേണ്ട 
ശീലങ്ങളെന്തൊക്കെയെന്നാണു് ഞാന്‍ അന്വേഷിക്കുന്നതും.

ആശങ്കകള്‍ പങ്കുവയ്ക്കുന്നതിനോടൊപ്പം പ്രശ്നത്തിന്റെ വേരുതേടിപ്പിടിച്ചു് ചികിത്സയ്ക്കു് ഒരു കുറിപ്പടിക്കുള്ള ആദ്യ 
ഉദ്യമമെങ്കിലും നടത്താനാണു് എനിക്കു് താത്പര്യം. ബൂലോകത്തിലെ സഹൃദയരെല്ലാരും സഹായത്തിനുണ്ടാവുമെന്നാണു് പ്രതീക്ഷ.

\hspace*{2em}(January 08, 2008)
\newpage
