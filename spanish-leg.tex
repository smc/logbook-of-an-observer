\secstar{സ്പാനിഷ് ലെഗ്ഗോടെ യൂറോപ്യന്‍ പാദത്തിനു് തുടക്കം}
\vskip 2pt

അങ്ങനെ അത്ഭുതങ്ങളൊന്നുമില്ലാതെ ഫോര്‍മുല വണ്‍ യൂറോപ്യന്‍പാദത്തിനു് തുടക്കമായി. ആദ്യന്തം 
വിരസമായ റേസിനൊടുവില്‍ റെഡ്ബുള്ളിന്റെ മാര്‍ക്ക് വെബ്ബര്‍ കരിയറിലെ മൂന്നാമതു് കിരീടം നേടി. 
ടയര്‍ പരിപാലിക്കുന്നതില്‍ പിഴവുവരുത്തിയ ഹാമില്‍ട്ടന്റേയും, കേടായ ബ്രേക്കുമായി മത്സരം 
പൂര്‍ത്തിയാക്കിയ റെഡ്ബുള്ളിന്റെതന്നെ സെബാസ്റ്റ്യന്‍ വെറ്റലിന്റേയും ചിലവില്‍ ഹോം റേസില്‍ 
അലോണ്‍സോ ഫെറാരിക്കുവേണ്ടി പതിനെട്ടു പോയിന്റു നേടി.

മുന്‍ റേസുകളില്‍ തന്റെ പഴയകാലത്തിന്റെ നിഴല്‍ മാത്രമായിരുന്ന മെഴ്സിഡസിന്റെ മൈക്കല്‍ ഷൂമാക്കര്‍ 
കാറില്‍ ചെറിയ മാറ്റങ്ങളുമായി വന്നു്, താനിപ്പോഴും ഒരങ്കത്തിനു് തയ്യാറാണെന്നു തെളിയിച്ചതാണു് 
വാര്‍ത്തകളില്‍ പ്രധാനം. പ്രാക്റ്റീസുകളിലും യോഗ്യതാ റൌണ്ടുകളിലും നല്ല പ്രകടനം കാഴ്ചവച്ച ഷുമാക്കര്‍ 
നാലാമതായാണു് റേസ് അവസാനിപ്പിച്ചത്. നാലു് മുന്‍നിര ടീമുകളില്‍ മെഴ്സിഡസ് വേഗത്തിന്റെ കാര്യത്തില്‍ 
ബഹുദൂരം പിന്നിലാണെന്ന കാര്യം സ്പെയിനില്‍ വ്യക്തമായി കാണാമായിരുന്നു. ഷുമാക്കറിന്റെ 
പരിചയസമ്പത്തൊന്നുമാത്രമാണു് ജെന്‍സണ്‍ ബട്ടന്റെയും ഫെലിപെ മസ്സയുടെയും അക്രമണങ്ങളില്‍നിന്നു് 
രക്ഷിച്ചതു്. തൊട്ടുമുമ്പിലെ വേഗമേറിയ ഫെറാരിയെ നേരിടുന്നതിനുപകരം, പിന്നിലെ വേഗമേറിയ കാറുകളെ 
തടഞ്ഞു് സ്ഥാനംനിലനിര്‍ത്താന്‍ നടത്തിയ ശ്രമം വിജയം കണ്ടെന്നു പറയാം. ടീം മേറ്റ് നികോ റൊസ്ബര്‍ഗ് 
പക്ഷെ മുന്‍ റേസുകളിനിന്നും വ്യത്യസ്തമായി വളരെ മങ്ങിയ പ്രകടനമാണു് കാഴ്ചവച്ചതു്. എട്ടമതായി തുടങ്ങി 
ഒരു ഘട്ടത്തില്‍ പതിനേഴാം സ്ഥാനംവരെ പോയ റൊസ്ബര്‍ഗ് പതിമൂന്നാമതായാണു് ഫിനിഷ് ചെയ്തതു്. 
റൊസ്ബര്‍ഗിനു് തൊട്ടതെല്ലാം പിഴച്ച വാരമാണെന്നു് വേണമെങ്കില്‍ പറയാം.

റെഡ്ബുള്ളിന്റെ രണ്ടാം ടീമായ ടോറോ റോസോയുടെ സെബാസ്റ്റ്യന്‍ ബുയെമിക്കും നിര്‍ഭാഗ്യങ്ങളുടെ റേസായിരുന്നു. 
ട്രാക്കില്‍ രണ്ടു പെനാല്‍ട്ടിയും പിഴച്ച സ്ട്രാറ്റജിയും, അവസാനം ഹൈഡ്രോളിക് സംവിധാനത്തിന്റെ പിഴവും ഈ യുവ 
സ്വിസ്സ് ഡ്രൈവറുടെ മറ്റൊരു റേസ് വാരംകൂടി അലങ്കോലമാക്കി. പിന്‍നിരയില്‍, ലോട്ടസിന്റെ ഹൈക്കി 
കൊവാലെയിനന്‍ ഗിയര്‍ ബോക്സ് പിഴവുകാരണം ട്രാക്കുകാണാതെ പിന്‍മാറിയെങ്കില്‍, ഹിസ്പാനിക് റേസിങ് ടീമിന്റെ 
ബ്രൂണോ സെന്നയുടെ (അന്തരിച്ച മുന്‍ചാമ്പ്യന്‍ അയര്‍ട്ടന്‍ സെന്നയുടെ അനന്തിരവന്‍) കരിയറിലെ ഏറ്റവും മോശം 
റേസ് വാരാന്ത്യങ്ങളിലൊന്നായിരുന്നു സ്പെയിനിലേതു്. ഒരു ലാപ്പുപോലും നീണ്ടില്ല സെന്നയുടെ പോരാട്ടം. ഇന്ത്യക്കാരനും 
ടീം മേറ്റുമായ കരണ്‍ ചന്ദോക് ഇരുപത്തിയേഴാം ലാപ്പുവരെ ശ്രമിച്ചുനോക്കിയെങ്കിലും ഇരുപതാമതായി റിട്ടയര്‍ ചെയ്തു. 
വിര്‍ജിന്റെ ലുകാസ് ഡി ഗ്രാസ്സി 62 ലാപ്പുകള്‍ പൂര്‍ത്തിയാക്കിയെങ്കില്‍, വിര്‍ജിന്‍ ടീം മേറ്റ് ടിമോ ഗ്ലോക്കും ലോട്ടസിന്റെ 
യാനോ ട്രൂലിയും 63 ലാപ്പുകള്‍ പൂര്‍ത്തിയാക്കി. ആക്സിഡന്റുമൂലം റേസ് അവസാനപ്പിച്ച മക്‌ലാരന്റെ ലൂയിസ് 
ഹാമില്‍ട്ടണും സാങ്കേതിക തകരാറുമൂലം അവസാനഘട്ടത്തില്‍ റേസ് നിര്‍ത്തിയ ഫോഴ്സ് ഇന്ത്യയുടെ വിറ്റാന്‍ടോണിയോ 
ലിയൂസ്സിയും 64 ലാപ്പുകള്‍ പൂര്‍ത്തിയാക്കിയാണു് വിരമിച്ചതെന്നറിയുമ്പോഴാണു് ഇവരുടെ പ്രകടനത്തിന്റെ നിലവാരം 
വ്യക്തമാവുക.

മുന്‍നിരയില്‍ പോരാട്ടങ്ങളൊക്കെ കുറവായിരുന്നുവെങ്കിലും, മധ്യനിരയില്‍ ചില ചെറിയ അങ്കങ്ങളൊക്കെയുണ്ടായിരുന്നു.
ഏതാണ്ടു് ഒരേവേഗതയുള്ള കാറുകളില്‍, ഫോഴ്സ് ഇന്ത്യയുടെ അഡ്രിയാന്‍ സുടിലും, റെനോയുടെ റോബര്‍ട്ടു് കുബിത്സയും 
തമ്മിലായിരുന്നു പ്രധാന പോരാട്ടം. തുടക്കത്തില്‍ സോബറിന്റെ കാമുയി കൊബിയാഷിയുമായി നടന്ന ഒരു ഉരസല്‍ മൂലം 
താളം നഷ്ടപ്പെട്ട കുബിത്സ ബാക്കി റേസ് മുഴുവന്‍ സുടിലിനെ മറികടക്കാനുള്ള ശ്രമത്തിലായിരുന്നു. ഇവര്‍ക്കു പിന്നിലായി, 
വില്യംസിന്റെ റൂബന്‍ ബാരിക്കെല്ലോ, ടോറോ റോസോയുടെ ജൈമി അല്‍ഗ്യുസാരി, റെനോയുടെ വിറ്റാലി പെട്രോവ് 
എന്നിവരും കൊബിയാഷിയും അവസാന രണ്ടു പോയിന്റുകള്‍ക്കായുള്ള പോരാട്ടങ്ങളിലായിരുന്നു.

ഈ ഞായറാഴ്വയാണു് (മേയ് 16) മൊണാകൊ ഗ്രാന്‍പ്രീ. പുതിയ അപ്ഗ്രേഡുകള്‍ക്കൊന്നും സമയമില്ലാത്തതിനാല്‍, 
ടീമുകള്‍ സ്ട്രാറ്റജി പ്ലാനിങ്ങില്‍ കൂടുതല്‍ ശ്രദ്ധകേന്ദ്രീകരിക്കാനാണു് സാധ്യത. ടയറുകളുടെ പരിപാലനം ഏറെ ആവശ്യമുള്ള 
ട്രാക്കാണു് മൊണാകൊയിലേതും. എങ്കിലും യോഗ്യതാ റൌണ്ടിലെ പ്രകടനവും ട്രാക്കിലെ സ്ഥിരതയുമാണു് ഇതുവരെ 
എല്ലാ റേസുകളിലും ജേതാക്കളെ നിശ്ചയിച്ചതു്. റോസബ്രാവ്ണും ഷുമാക്കറും പുതിയ വല്ല തന്ത്രങ്ങളുമായി ഇറങ്ങുമോ
എന്നതു് കാത്തിരുന്നു കാണേണ്ടതാണു്. ആദ്യ ഏഴു ഡ്രൈവര്‍മാര്‍ക്കിടയില്‍ വെറും 21 പോയിന്റ് വ്യത്യാസവും 
(70 പോയിന്റോടെ ബട്ടണ്‍ ഒന്നാമതും, 49 പോയിന്റോടെ മസ്സ ഏഴാമതും) ആദ്യ മൂന്നു ടീമുകള്‍ തമ്മില്‍ വെറും ആറു 
പോയിന്റിന്റെയും (മക്‌ലാരന്‍ 119, ഫെറാരി 116, റെഡ്ബുള്‍ 113) മാത്രം വ്യത്യസമുള്ളതു് ഇനിയും അത്ഭുതങ്ങള്‍ക്കു് 
സാധ്യതയൊരുക്കുന്നു. 24 പോയിന്റുമായി ഫോഴ്സ് ഇന്ത്യ ആറാമതാണു്.

വാല്‍ക്കഷണം: മക്‌ലാരന്റെ പ്ലാന്റില്‍ മെഴ്സിഡസ് ഉണ്ടാക്കുന്ന SLS AMG യുടെ പരസ്യത്തില്‍ ഇപ്പോള്‍ മൈക്കല്‍ 
ഷൂമാക്കറാണു്. മക്‌ലാരന്‍ പ്ലാന്റില്‍ മെഴ്സിഡസ് ഉണ്ടാക്കുന്ന അവസാന കാറാണിതു്. ദശകങ്ങള്‍ക്കുശേഷം (എന്റെ 
ഓര്‍മ്മ ശരിയാണെങ്കില്‍ മക്‌ലാരന്‍ എഫ് 1നു ശേഷം) മക്‌ലാരന്‍ വീണ്ടും റോഡ് കാറുകള്‍ നിര്‍മ്മിക്കാന്‍ 
പദ്ധതിയിട്ടിരിക്കുകയാണു്. MP4-12C എന്നു് പേരിട്ടിരിക്കുന്ന കാര്‍ 2011ല്‍ നിരത്തിലെത്തുമെന്നാണു് മക്‌ലാരന്‍ 
വൃത്തങ്ങള്‍ പറയുന്നതു്.

\begin{flushright}(11 May, 2010)\footnote{http://malayal.am/വിനോദം/കായികം/5473/സ്പാനിഷ്-ലെഗ്ഗോടെ-യൂറോപ്യന്‍-പാദത്തിനു്-തുടക്കം}\end{flushright}

\newpage
