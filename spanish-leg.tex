\secstar{സ്പാനിഷ് ലെഗ്ഗോടെ യൂറോപ്യന്‍ പാദത്തിന് തുടക്കം}
\vskip 2pt

അ­ങ്ങ­നെ അത്ഭു­ത­ങ്ങ­ളൊ­ന്നു­മി­ല്ലാ­തെ ­ഫോര്‍­മുല വണ്‍ യൂ­റോ­പ്യന്‍ പാ­ദ­ത്തി­ന് തു­ട­ക്ക­മാ­യി. ആദ്യ­ന്തം 
വി­ര­സ­മായ റേ­സി­നൊ­ടു­വില്‍ റെ­ഡ്ബു­ള്ളി­ന്റെ മാര്‍­ക്ക് വെ­ബ്ബര്‍ കരി­യ­റി­ലെ മൂ­ന്നാ­മ­ത് കി­രീ­ടം നേ­ടി. 
ടയര്‍ പരി­പാ­ലി­ക്കു­ന്ന­തില്‍ പി­ഴ­വു വരു­ത്തിയ ഹാ­മില്‍­ട്ട­ന്റേ­യും കേ­ടായ ബ്രേ­ക്കു­മാ­യി മത്സ­രം 
പൂര്‍­ത്തി­യാ­ക്കിയ റെ­ഡ്ബു­ള്ളി­ന്റെ തന്നെ സെ­ബാ­സ്റ്റ്യന്‍ വെ­റ്റ­ലി­ന്റേ­യും ചി­ല­വില്‍ ഹോം റേ­സില്‍ 
അലോണ്‍­സോ ഫെ­റാ­രി­ക്കു വേ­ണ്ടി പതി­നെ­ട്ടു പോ­യി­ന്റു നേ­ടി­.

­മുന്‍ റേ­സു­ക­ളില്‍ തന്റെ പഴ­യ­കാ­ല­ത്തി­ന്റെ നി­ഴല്‍ മാ­ത്ര­മാ­യി­രു­ന്ന മെ­ഴ്സി­ഡ­സി­ന്റെ മൈ­ക്കല്‍ ­ഷൂ­മാ­ക്കര്‍ 
കാ­റില്‍ ചെ­റിയ മാ­റ്റ­ങ്ങ­ളു­മാ­യി വന്ന്, താ­നി­പ്പോ­ഴും ഒര­ങ്ക­ത്തി­ന് തയ്യാ­റാ­ണെ­ന്നു തെ­ളി­യി­ച്ച­താ­ണ് 
വാര്‍­ത്ത­ക­ളില്‍ പ്ര­ധാ­നം. പ്രാ­ക്റ്റീ­സു­ക­ളി­ലും യോ­ഗ്യ­താ റൌ­ണ്ടു­ക­ളി­ലും നല്ല പ്ര­ക­ട­നം കാ­ഴ്ച വെ­ച്ച ഷു­മാ­ക്കര്‍ 
നാ­ലാ­മ­താ­യാ­ണ് റേ­സ് അവ­സാ­നി­പ്പി­ച്ച­ത്. നാ­ലു മുന്‍ നിര ടീ­മു­ക­ളില്‍ മെ­ഴ്സി­ഡ­സ് വേ­ഗ­ത്തി­ന്റെ കാ­ര്യ­ത്തില്‍ 
ബഹു­ദൂ­രം പി­ന്നി­ലാ­ണെ­ന്ന കാ­ര്യം സ്പെ­യി­നില്‍ വ്യ­ക്ത­മാ­യി കാ­ണാ­മാ­യി­രു­ന്നു. ഷു­മാ­ക്ക­റി­ന്റെ 
പരി­ച­യ­സ­മ്പ­ത്തൊ­ന്നു­മാ­ത്ര­മാ­ണ് ജെന്‍­സണ്‍ ബട്ട­ന്റെ­യും ഫെ­ലി­പെ മസ്സ­യു­ടെ­യും അക്ര­മ­ണ­ങ്ങ­ളില്‍ നി­ന്ന് 
രക്ഷി­ച്ച­ത്. തൊ­ട്ടു­മു­മ്പി­ലെ വേ­ഗ­മേ­റിയ ഫെ­റാ­രി­യെ നേ­രി­ടു­ന്ന­തി­നു പക­രം, പി­ന്നി­ലെ വേ­ഗ­മേ­റിയ കാ­റു­ക­ളെ 
തട­ഞ്ഞ് സ്ഥാ­നം നി­ല­നിര്‍­ത്താന്‍ നട­ത്തിയ ശ്ര­മം വി­ജ­യം കണ്ടെ­ന്നു പറ­യാം. ടീം മേ­റ്റ് നി­കോ റൊ­സ്ബര്‍­ഗ് 
പക്ഷെ മുന്‍ റേ­സു­ക­ളില്‍ നി­ന്നും വ്യ­ത്യ­സ്ത­മാ­യി വള­രെ മങ്ങിയ പ്ര­ക­ട­ന­മാ­ണ് കാ­ഴ്ച­വ­ച്ച­ത്. എട്ട­മ­താ­യി തു­ട­ങ്ങി 
ഒരു ഘട്ട­ത്തില്‍ പതി­നേ­ഴാം സ്ഥാ­നം വരെ പോയ റൊ­സ്ബര്‍­ഗ് പതി­മൂ­ന്നാ­മ­താ­യാ­ണ് ഫി­നി­ഷ് ചെ­യ്ത­ത്. 
റൊ­സ്ബര്‍­ഗി­ന് തൊ­ട്ട­തെ­ല്ലാം പി­ഴ­ച്ച വാ­ര­മാ­ണെ­ന്നു വേ­ണ­മെ­ങ്കില്‍ പറ­യാം.

­റെ­ഡ്ബു­ള്ളി­ന്റെ രണ്ടാം ടീ­മായ ടോ­റോ റോ­സോ­യു­ടെ സെ­ബാ­സ്റ്റ്യന്‍ ബു­യെ­മി­ക്കും നിര്‍­ഭാ­ഗ്യ­ങ്ങ­ളു­ടെ റേ­സാ­യി­രു­ന്നു. 
ട്രാ­ക്കില്‍ രണ്ടു പെ­നാല്‍­ട്ടി­യും പി­ഴ­ച്ച സ്ട്രാ­റ്റ­ജി­യും അവ­സാ­നം ഹൈ­ഡ്രോ­ളി­ക് സം­വി­ധാ­ന­ത്തി­ന്റെ പി­ഴ­വും, ഈ യുവ 
സ്വി­സ്സ് ഡ്രൈ­വ­റു­ടെ മറ്റൊ­രു റേ­സ് വാ­രം­കൂ­ടി അല­ങ്കോ­ല­മാ­ക്കി. പിന്‍ നി­ര­യില്‍, ലോ­ട്ട­സി­ന്റെ ഹൈ­ക്കി 
കൊ­വാ­ലെ­യി­നന്‍ ഗി­യര്‍ ബോ­ക്സ് പി­ഴ­വു­കാ­ര­ണം ട്രാ­ക്കു­കാ­ണാ­തെ പിന്‍­മാ­റി­യെ­ങ്കില്‍, ഹി­സ്പാ­നി­ക് റേ­സി­ങ് ടീ­മി­ന്റെ 
ബ്രൂ­ണോ സെ­ന്നയു­ടെ (അ­ന്ത­രി­ച്ച മുന്‍ ചാ­മ്പ്യന്‍ അയര്‍­ട്ടന്‍ സെ­ന്ന­യു­ടെ അന­ന്തി­ര­വന്‍)­ കരി­യ­റി­ലെ ഏറ്റ­വും മോ­ശം 
റേ­സ് വാ­രാ­ന്ത്യ­ങ്ങ­ളി­ലൊ­ന്നാ­യി­രു­ന്നു സ്പെ­യി­നി­ലേ­ത്. ഒരു­ലാ­പ്പു പോ­ലും നീ­ണ്ടി­ല്ല സെ­ന്ന­യു­ടെ പോ­രാ­ട്ടം. ഇന്ത്യ­ക്കാ­ര­നും 
ടീം മേ­റ്റു­മായ കരണ്‍ ചന്ദോ­ക് ഇരു­പ­ത്തി­യേ­ഴാം ലാ­പ്പു­വ­രെ ​ശ്ര­മി­ച്ചു നോ­ക്കി­യെ­ങ്കി­ലും ഇരു­പ­താ­മ­താ­യി റി­ട്ട­യര്‍ ചെ­യ്ത. 
വിര്‍­ജി­ന്റെ ലു­കാ­സ് ഡി ഗ്രാ­സ്സി 62 ലാ­പ്പു­കള്‍ പൂര്‍­ത്തി­യാ­ക്കി­യെ­ങ്കില്‍, വിര്‍­ജിന്‍ ടീം മേ­റ്റ് ടി­മോ ഗ്ലോ­ക്കും ലോ­ട്ട­സി­ന്റെ 
യാ­നോ ട്രൂ­ലി­യും 63 ലാ­പ്പു­കള്‍ പൂര്‍­ത്തി­യാ­ക്കി. ആക്സി­ഡ­ന്റു­മൂ­ലം റേ­സ് അവ­സാ­ന­പ്പി­ച്ച മക്‌­ലാ­ര­ന്റെ ലൂ­യി­സ് 
ഹാ­മില്‍­ട്ട­ണും സാ­ങ്കേ­തിക തക­രാ­റു­മൂ­ലം അവ­സാ­ന­ഘ­ട്ട­ത്തില്‍ റേ­സ് നിര്‍­ത്തിയ ഫോ­ഴ്സ് ഇന്ത്യ­യു­ടെ വി­റ്റാന്‍­ടോ­ണി­യോ 
ലി­യൂ­സ്സി­യും 64 ലാ­പ്പു­കള്‍ പൂര്‍­ത്തി­യാ­ക്കി­യാ­ണ് വി­ര­മി­ച്ച­തെ­ന്ന­റി­യു­മ്പോ­ഴാ­ണ് ഇവ­രു­ടെ പ്ര­ക­ട­ന­ത്തി­ന്റെ നി­ല­വാ­രം 
വ്യ­ക്ത­മാ­വു­ക.

­മുന്‍ നി­ര­യില്‍ പോ­രാ­ട്ട­ങ്ങ­ളൊ­ക്കെ കു­റ­വാ­യി­രു­ന്നു­വെ­ങ്കി­ലും, മധ്യ­നി­ര­യില്‍ ചില ചെ­റിയ അങ്ക­ങ്ങ­ളൊ­ക്കെ­യു­ണ്ടാ­യി­രു­ന്നു.
ഏതാ­ണ്ട് ഒരേ വേ­ഗ­ത­യു­ള്ള കാ­റു­ക­ളില്‍, ഫോ­ഴ്സ് ഇന്ത്യ­യു­ടെ അഡ്രി­യാന്‍ സു­ടി­ലും, റെ­നോ­യു­ടെ റോ­ബര്‍­ട്ട് കു­ബി­ത്സ­യും 
തമ്മി­ലാ­യി­രു­ന്നു പ്ര­ധാന പോ­രാ­ട്ടം. തു­ട­ക്ക­ത്തില്‍ സോ­ബ­റി­ന്റെ കാ­മു­യി കൊ­ബി­യാ­ഷി­യു­മാ­യി നട­ന്ന ഒരു ഉര­സല്‍ മൂ­ലം 
താ­ളം നഷ്ട­പ്പെ­ട്ട കു­ബി­ത്സ ബാ­ക്കി റേ­സ് മു­ഴു­വന്‍ സു­ടി­ലി­നെ മറി­ക­ട­ക്കാ­നു­ള്ള ശ്ര­മ­ത്തി­ലാ­യി­രു­ന്നു. ഇവര്‍­ക്കു പി­ന്നി­ലാ­യി, 
വി­ല്യം­സി­ന്റെ റൂ­ബന്‍ ബാ­രി­ക്കെ­ല്ലോ, ടോ­റോ റോ­സോ­യു­ടെ ജൈ­മി അല്‍­ഗ്യു­സാ­രി, റെ­നോ­യു­ടെ വി­റ്റാ­ലി പെ­ട്രോ­വ് 
എന്നി­വ­രും കൊ­ബി­യാ­ഷി­യും അവ­സാന രണ്ടു പോ­യി­ന്റു­കള്‍­ക്കാ­യു­ള്ള പോ­രാ­ട്ട­ങ്ങ­ളി­ലാ­യി­രു­ന്നു­.

ഈ ഞാ­യ­റാ­ഴ്വ­യാ­ണ് (മേ­യ് 16) മൊ­ണാ­കൊ ഗ്രാന്‍­പ്രീ. പു­തിയ അപ്ഗ്രേ­ഡു­കള്‍­ക്കൊ­ന്നും സമ­യ­മി­ല്ലാ­ത്ത­തി­നാല്‍, 
ടീ­മു­കള്‍ സ്ട്രാ­റ്റ­ജി പ്ലാ­നി­ങ്ങില്‍ കൂ­ടു­തല്‍ ശ്ര­ദ്ധ­കേ­ന്ദ്രീ­ക­രി­ക്കാ­നാ­ണ് സാ­ധ്യ­ത. ടയ­റു­ക­ളു­ടെ പരി­പാ­ല­നം ഏറെ ആവ­ശ്യ­മു­ള്ള 
ട്രാ­ക്കാ­ണ് മൊ­ണാ­കൊ­യി­ലേ­തും. എങ്കി­ലും യോ­ഗ്യ­താ റൌ­ണ്ടി­ലെ പ്ര­ക­ട­ന­വും ട്രാ­ക്കി­ലെ സ്ഥി­ര­ത­യു­മാ­ണ് ഇതു­വ­രെ 
എല്ലാ ­റേ­സു­ക­ളി­ലും ജേ­താ­ക്ക­ളെ നി­ശ്ച­യി­ച്ച­ത്. റോ­സ് ബ്രാ­വ്ണും ഷു­മാ­ക്ക­റും പു­തിയ വല്ല­ ത­ന്ത്ര­ങ്ങ­ളു­മാ­യി ഇറ­ങ്ങു­മോ
എന്ന­ത് കാ­ത്തി­രു­ന്നു കാ­ണേ­ണ്ട­താ­ണ്. ആദ്യ ഏഴു ഡ്രൈ­വര്‍­മാര്‍­ക്കി­ട­യില്‍ വെ­റും 21 പോ­യി­ന്റ് വ്യ­ത്യാ­സ­വും­ 
(70 പോ­യി­ന്റോ­ടെ ബട്ടണ്‍ ഒന്നാ­മ­തും, 49 പോ­യി­ന്റോ­ടെ മസ്സ ഏഴാ­മ­തും) ആദ്യ മൂ­ന്നു ടീ­മു­കള്‍ തമ്മില്‍ വെ­റും ആറു 
പോ­യി­ന്റി­ന്റെ­യും ­(­മ­ക്‌­ലാ­രന്‍ 119, ­ഫെ­റാ­രി­ 116, ­റെ­ഡ്ബുള്‍ 113) മാ­ത്രം വ്യ­ത്യ­സ­മു­ള്ള­ത് ഇനി­യും അത്ഭു­ത­ങ്ങള്‍­ക്ക് 
സാ­ധ്യ­ത­യൊ­രു­ക്കു­ന്നു. 24 പോ­യി­ന്റു­മാ­യി ഫോ­ഴ്സ് ഇന്ത്യ ആറാ­മ­താ­ണ്.

­വാല്‍­ക്ക­ഷ­ണം: മക്‌­ലാ­ര­ന്റെ പ്ലാ­ന്റില്‍ മെ­ഴ്സി­ഡ­സ് ഉണ്ടാ­ക്കു­ന്ന SLS AMG യു­ടെ പര­സ്യ­ത്തില്‍ ഇപ്പോള്‍ മൈ­ക്കല്‍ 
ഷൂ­മാ­ക്ക­റാ­ണ്. മക്‌­ലാ­രന്‍ പ്ലാ­ന്റില്‍ മെ­ഴ്സി­ഡ­സ് ഉണ്ടാ­ക്കു­ന്ന അവ­സാ­ന­കാ­റാ­ണി­ത്. ദശ­ക­ങ്ങള്‍­ക്കു ശേ­ഷം ­(എ­ന്റെ 
ഓര്‍­മ്മ ശരി­യാ­ണെ­ങ്കില്‍ മക്‌­ലാ­രന്‍ എഫ് 1നു ശേ­ഷം) മക്‌­ലാ­രന്‍ വീ­ണ്ടും റോ­ഡ് കാ­റു­കള്‍ നിര്‍­മ്മി­ക്കാന്‍ 
പദ്ധ­തി­യി­ട്ടി­രി­ക്കു­ക­യാ­ണ്. MP4-12C എന്ന് പേ­രി­ട്ടി­രി­ക്കു­ന്ന കാര്‍ 2011ല്‍ നി­ര­ത്തി­ലെ­ത്തു­മെ­ന്നാ­ണ് മക്‌­ലാ­രന്‍ 
വൃ­ത്ത­ങ്ങള്‍ പറ­യു­ന്ന­ത്.

(11 May 2010)\footnote{http://malayal.am/വിനോദം/കായികം/5473/സ്പാനിഷ്-ലെഗ്ഗോടെ-യൂറോപ്യന്‍-പാദത്തിന്-തുടക്കം}

\newpage
