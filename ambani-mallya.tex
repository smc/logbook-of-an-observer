\secstar{അംബാനി മുതല്‍ മല്യ വരെ}
\vskip 1pt

ഐപിഎല്‍ ടീമുകളുടെ സാമ്പത്തിക വിശകലനത്തില്‍ എറ്റവുമാദ്യം വരേണ്ടതു്, 2008ല്‍ നൂറു മില്യണ്‍ ഡോളറിനുമേല്‍ 
വിറ്റുപോയ ടീമുകളാണു്. മുംബൈ, ബാംഗ്ലൂര്‍, ഹൈദരാബാദ് ടീമുകളുടെ അവകാശമാണു് പത്തുവര്‍ഷത്തേക്കു് 
നൂറുമില്യണ്‍ ഡോളറിനുമേല്‍ തുകയ്ക്കു വിറ്റുപോയതു്. ഇതില്‍ ആശ്വാസകരമായ സംഗതി മൂന്നു ടീമും സ്വന്തമാക്കിയതു് 
ലിസ്റ്റഡ് കമ്പനികളാണെന്നതാണു്. മാത്രമല്ല, ഐപിഎല്‍ വഴി എങ്ങനെ പണമുണ്ടാക്കാമെന്നതിനും 
മുടക്കുമുതല്‍ തിരിച്ചുപിടിയ്ക്കുന്നതിനും ശക്തമായ ന്യായീകരണം മുംബൈ, ബാംഗ്ലൂര്‍ ടീമുടമകള്‍ക്കുണ്ടായിരുന്നു താനും. 

ഹൈദരാബാദ് ടീമിന്റെ സാമ്പത്തികനയങ്ങളെപ്പറ്റി വലിയ വിവരങ്ങളൊന്നും വെളിയില്‍ വന്നിട്ടില്ല. എങ്കിലും ഒരു വട്ടം 
ഐപിഎല്‍ ചാമ്പ്യന്‍മാരായതും ആന്ധ്രാപ്രദേശിലെ ആരാധകവൃന്ദവും ഉടമസ്ഥരായ ഡെക്കാണ്‍ ക്രോണിക്കിളിനു് 
വിനോദവ്യവസായത്തിലുള്ള താല്‍പ്പര്യങ്ങളുമാകണം അവരെ നയിച്ചതു്. എന്തായാലും ഈ ലക്കത്തില്‍ മുംബൈ, 
ബാംഗ്ലൂര്‍ ടീമുകളുടെ സമീപനത്തിലേക്കാണു് കൂടുതല്‍ ശ്രദ്ധകൊടുത്തിരിക്കുന്നതു്.

ബാംഗ്ലൂര്‍ നഗരത്തിനുവേണ്ടിയുള്ള ലേലത്തില്‍ വിജയം കണ്ടതു് ഇന്ത്യയിലെ ഒരു ഒന്നാംകിട സ്പോര്‍ട്സു് എന്റര്‍പ്രോണറായ 
വിജയ് മല്യയുടെ യുബി ഗ്രൂപ്പാണു്. മുംബൈ ടീമിനെ നേടിയതു് ഇന്ത്യയിലെ ഏറ്റവും വലിയ സ്വകാര്യ കമ്പനിയായ റിലയന്‍സു് 
ഇന്‍ഡസ്ട്രീസും.

%image courtesy: http://www.samaylive.com/english/sports/676462175.html

ക്രിക്കറ്റും റിലയന്‍സുമായുള്ള ബന്ധം 1987ല്‍ ഇംഗ്ലണ്ടിനുപുറത്തു് ഏകദിന ലോകകപ്പ് 
സംഘടിപ്പിക്കാന്‍ ഇന്ത്യന്‍ ബോര്‍ഡിനെ സഹായിച്ചതില്‍നിന്നു് തുടങ്ങുന്നു. കോണ്‍ഗ്രസ്സുകാരനായ മാധവറാവു സിന്ധ്യയുടെ ബലത്തിലാണു് 
അന്നു് റിലയന്‍സ് ലോകകപ്പ് സ്പോണ്‍സര്‍ ചെയ്തതെന്നു് ഒരു പറച്ചിലുണ്ടെങ്കിലും, നാലാം ലോകകപ്പിനു് റിലയന്‍സ് 
കപ്പ് എന്നു് പേരിടാന്‍ സഹായങ്ങള്‍ ചെയ്തവര്‍ പിന്നീടു് അത്രയ്ക്കു മഹാമനസ്കത കാട്ടിയില്ല എന്നതാണു് സത്യം.

1996ല്‍ ലോകകപ്പു് ഇന്ത്യയിലെത്തുമ്പോള്‍, പുകയിലയ്ക്കപ്പുറം പുതിയലോക സ്വപ്നം കണ്ടുതുടങ്ങിയ ഐടിസിയുടെ 
വില്‍സ് ബ്രാന്‍ഡാണു് ടൈറ്റില്‍ സ്പോണ്‍സേഴ്സായതു്. റിലയന്‍സ് ക്രിക്കറ്റിന്റെ ലോകത്തേയ്ക്കു് ആഘോഷപൂര്‍‌വ്വം 
എത്തിയതിനു് സാമ്പത്തിക വിശാരദന്മാര്‍ വിവിധകാരണങ്ങളാണു് നിരത്തിയതു്.

ആ വിശകലനങ്ങളുടെ രത്നച്ചുരുക്കം ഇതായിരുന്നു, ഏറ്റവും പതുക്കെ ലാഭമുണ്ടാക്കാന്‍ തുടങ്ങുന്ന ഫ്രാഞ്ചൈസി 
മുംബൈയായിരിക്കും. വര്‍ഷത്തില്‍ മൂന്നു മില്യണ്‍വരെയൊക്കെ നഷ്ടം മുംബൈ സഹിക്കും. കാരണമോ, 
ഉടമസ്ഥരായ മുകേഷ്-നിതാ അംബാനി ദമ്പതികള്‍ക്കു ലഭിക്കുന്ന ടിവി പ്രൈം ടൈമും സൗജന്യ പരസ്യവും.

%image courtesy: http://thecurrentaffairs.com/ipl-will-be-responsible-for-players-security-modi.html
%നിത അംബാനി ബോളിവുഡ് നടി കരീന കപൂറിനും സച്ചിന്‍ ടെണ്ടുല്‍ക്കറുടെ ഭാര്യ അഞ്ജലിക്കുമൊപ്പം" height="364" width="430" />

ഇന്ത്യയിലെ ഏറ്റവും പണക്കാരായ ദമ്പതികള്‍ക്കു്  തങ്ങള്‍ എത്രമാത്രം മിഡില്‍ ക്ലാസാണെന്നു 
കാണിക്കാന്‍ കിട്ടുന്ന സൗജന്യാവസരങ്ങള്‍ അവരുടെ കമ്പനികള്‍ക്കു ചെയ്യുന്ന ഗുണങ്ങള്‍ പലമടങ്ങാണു്. മാത്രമല്ല, 
ഐപിഎല്‍ ഹോം മത്സരങ്ങള്‍ നിതാ അംബാനി തന്റെ സാമൂഹ്യസേവന സന്നദ്ധത തുറന്നുകാണിച്ചു് ഉപയോഗിക്കുന്നു. 
വേറെ ഒരു മാധ്യമവും അംബാനി ദമ്പതികളുടെ സാമൂഹ്യസേവനത്തെ ഇത്രയും പ്രകീര്‍ത്തിച്ചിട്ടുണ്ടാവില്ലെന്നുള്ളതുതന്നെ 
അവര്‍ക്കു് ടീമില്‍ വരുന്ന നഷ്ടം നികത്തുന്നു. സൗജന്യവിലയ്ക്കു ലഭിച്ച ക്രിക്കറ്റ് ദൈവത്തെ നന്നായി ഉപയോഗിച്ചു് 
നഷ്ടം കുറയ്ക്കാന്‍ അവര്‍ക്കു് സാധിച്ചിരിക്കുന്നു. (സച്ചിനും സഹീറും ഹര്‍ബജനും അണിനിരന്ന 2009ല്‍ ഐഡിയ പരസ്യം ഉദാഹരണം.)

ഈ ഘടകങ്ങളിലൂന്നി സാമ്പത്തികലക്ഷ്യങ്ങള്‍ നിശ്ചയിച്ച മുംബൈ ടീം ആവറേജ് പ്രകടനം കാഴ്ചവച്ചാല്‍പ്പോലും ഉടമസ്ഥരെ 
ഭയപ്പെടുത്താന്‍മാത്രം പ്രശ്നങ്ങളുള്ള ഒന്നായിരുന്നില്ല. ഇതിനൊപ്പം ടീം മര്‍ച്ചന്‍ഡൈസ് വിപണികൂടി ചേര്‍ത്താല്‍ കിട്ടുന്ന 
ഫലം അമ്പരപ്പിച്ചില്ലെങ്കിലും ഒരിക്കലും നിരാശാജനകമല്ല. മാത്രമല്ല, മുംബൈ നഗരത്തെ പ്രതിനിധീകരിക്കുന്നതുകൊണ്ടു് ഒരു 
പരിധിവരെ നല്ല സ്പോണ്‍സര്‍മാരെ ആകര്‍ഷിക്കാനും ടീമിനു കഴിഞ്ഞു.

മുംബൈ ഏറ്റവും യാഥാസ്ഥിതികമായ രീതിയില്‍ ഒരു സ്പോര്‍ട്സ് ഫ്രാഞ്ചൈസി എങ്ങനെ നടത്താം എന്നാണു് പരീക്ഷിച്ചതു്. 
ആവശ്യമില്ലാത്ത റിസ്കുകള്‍ ഒഴിവാക്കി, സാമ്പത്തിക നഷ്ടംപോലും എല്ലായ്പ്പോഴും ചില സാമൂഹികനേട്ടങ്ങള്‍ തരുമെന്നുറപ്പാക്കി, 
വ്യക്തമായ പ്ലാനോടുകൂടി കളത്തിലിറങ്ങിയ അവസ്ഥ. അവരുടെ ടീം അടുത്ത പത്തുവര്‍ഷത്തേക്കു് ഒരിക്കല്‍പോലും ഐപിഎല്‍ 
ജേതാക്കളായില്ലെങ്കില്‍കൂടി, ഉണ്ടാകാവുന്ന സഞ്ചിതനഷ്ടം എങ്ങനെ മറ്റുവഴികളിലൂടെ പരിഹരിക്കാം എന്നതു് ആദ്യമേ മുംബൈയുടെ 
കണക്കുപുസ്തകങ്ങളില്‍ ഇടംപിടിച്ചിരിക്കാം എന്നാണു് പല വിശകലന വിദഗ്ദരുടെയും വാദം.

ബാംഗ്ലൂര്‍ ടീമിന്റെ കഥ കുറച്ചു് വ്യത്യസ്തമാണു്. റിലയന്‍സിനെപ്പോലെതന്നെ പബ്ലിക് ലിമിറ്റഡ് കമ്പനിയായ യുബി ഗ്രൂപ്പാണു് 
ബാംഗ്ലൂര്‍ ടീമിന്റെ ഉടമസ്ഥര്‍. പക്ഷെ പ്രധാന പ്രമോട്ടറായ വിജയ് മല്യ മുകേഷ്-നിതാ അംബാനി ദമ്പതികളില്‍നിന്നു് 
വ്യത്യസ്തമായി പ്രശസ്ത സ്പോര്‍ട്സ് ഇന്‍വെസ്റ്ററാണു്. ഫോഴ്സ് ഇന്ത്യ ഫോര്‍മുലാ വണ്‍ ടീമാണു് മല്യയുടെ ഒരു പ്രധാന 
സ്പോര്‍ട്സ് നിക്ഷേപം. മറ്റൊരു നിക്ഷേപം ഇന്ത്യന്‍ കുതിരയോട്ട രംഗത്തു് പുനെ സമ്രാട്ടുകളെ വെല്ലുവിളിക്കുന്ന കുതിരകളൂടെ 
ഉടമകളായ യുണൈറ്റഡ് റേസിങ് ആന്റ് ബ്ലഡ്സ്റ്റോക്ക് ബ്രീഡേഴ്സാണു്. ഇന്ത്യന്‍ ഫുട്ബാളില്‍ ഈസ്റ്റു് ബംഗാള്‍, 
മോഹന്‍ ബഗാന്‍ ടീമുകളുടെ ടൈറ്റില്‍ സ്പോണ്‍സറാണു് യുബി ഗ്രൂപ്പു്. മാത്രമല്ല ഈസ്റ്റു് ബംഗാള്‍ ക്ലബ്ബില്‍ അമ്പതു 
ശതമാനം ഓഹരിയും മല്യക്കു സ്വന്തമായുണ്ടു്.

%image courtesy: http://www.bollywoodraj.com/2010/04/deepika-padukone-with-siddharth-mallya.html

മല്യയുടെ സ്പോര്‍ട്സു് നിക്ഷേപങ്ങളുടെ ഒരു പ്രത്യേകതയെന്തെന്നാല്‍, അവയെല്ലാം യുബി ഗ്രൂപ്പിന്റെ പ്രധാന ബിസിനസ്സായ 
മദ്യക്കച്ചവടത്തെ പരിപോഷിപ്പിക്കുന്ന തരത്തിലാണു് പ്ലേസ് ചെയ്തിരിക്കുന്നതെന്നാണു്. ഇന്ത്യയില്‍ കുതിരയോട്ടവും പന്തയവും 
നടത്തുന്ന വലിയ പണക്കാരുടെയിടയില്‍ സ്വന്തം ബ്രാന്‍ഡുകളുടെ സ്വാധീനം വര്‍ധിപ്പിക്കാനാണു് മല്യ കുതിരകളെ 
ഉപയോഗിക്കുന്നതു്. ഫോര്‍മുല വണ്ണിന്റെ പ്രഭവസ്ഥാനമായ യൂറോപ്പില്‍ യുബി ഗ്രൂപ്പിന്റെ അടിത്തറ വിപുലീകരിക്കുന്നതിനാണു് 
ഫോഴ്സു് ഇന്ത്യ ടീമിനെ മല്യ ഉപയോഗിക്കുന്നതു്. ബംഗാളിലെ സാധാരണ കുടിയന്മാരാണു് കൊല്‍ക്കത്ത ടീമുകളിലൂടെ 
ലക്ഷ്യമാക്കിയതെങ്കില്‍, അഖിലേന്ത്യാതലത്തില്‍ ക്രിക്കറ്റു് ആരാധകരുടെ പ്രധാന ബ്രാന്‍ഡാവുന്നതിനുള്ള 
അടവായിരുന്നു ഐപിഎല്‍ ടീം. എന്റര്‍ടൈന്‍മൈന്റ് ആന്റ് സ്പോര്‍ട്സ് ഡയറക്റ്റുമായി സഹകരിച്ചു് റോയല്‍ ചാലഞ്ചേഴ്സ് സ്പോര്‍ട്സ് 
വിവിധനഗരങ്ങളിലെ ക്ലബ്ബുകളില്‍ നടത്തുന്ന ഐപിഎല്‍ രാവുകള്‍ ഈ സ്ട്രാറ്റജിയുടെ നേരുദാഹരണമാണു്.

അതുകൊണ്ടുതന്നെ പുറമേനിന്നുള്ള സ്പോണ്‍സര്‍മാരെ പ്രോത്സാഹിപ്പിക്കാത്ത മല്യയുടെ നയം കാര്യങ്ങള്‍ കൂടുതല്‍ 
വ്യക്തമാക്കുന്നു. അതിനാല്‍ റോയല്‍ ചാലഞ്ചേഴ്സ് ബാംഗ്ലൂര്‍ ടീം ഉണ്ടാക്കുന്ന ലാഭത്തില്‍ ഐപിഎല്‍ മാമാങ്കത്തില്‍നിന്നു് 
മല്യയുടെ മദ്യ ബ്രാന്റുകള്‍ ഉണ്ടാക്കുന്ന ലാഭം കൂടി ചേര്‍ക്കേണ്ടതാണു്. ഇത്തരത്തില്‍ ഐപിഎല്ലിലെ വില കൂടിയ 
രണ്ടു ടീമുകളും വ്യക്തമായ ബിസിനസ്സു് ലക്ഷ്യങ്ങളും കണക്കുകളുമായാണു് ആളുകളെ അമ്പരപ്പിക്കുന്നതെങ്കില്‍ നേരെ 
എതിര്‍ധ്രുവത്തില്‍ നില്‍ക്കുന്ന വമ്പന്‍മാരുമുണ്ടു്. അവരെക്കുറിച്ചു് അടുത്ത ലേഖനത്തില്‍.

\begin{flushright}(10 May, 2010)\footnote{http://malayal.am/പലവക/പരമ്പര/ബിസിനസ്-ലീഗ്/5378/അംബാനി-മുതല്‍-മല്യ-വരെ}\end{flushright}

\newpage
