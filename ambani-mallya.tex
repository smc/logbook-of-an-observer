\secstar{അംബാനി മുതല്‍ മല്യ വരെ}
\vskip 1pt

ഐ­പി­എല്‍ ടീ­മു­ക­ളു­ടെ സാ­മ്പ­ത്തിക വി­ശ­ക­ല­ന­ത്തില്‍ എറ്റ­വു­മാ­ദ്യം വരേ­ണ്ട­ത്, 2008ല്‍ നൂ­റു മി­ല്യണ്‍ ഡോ­ള­റി­നു മേല്‍ 
വി­റ്റു പോയ ടീ­മു­ക­ളാ­ണ്, മും­ബൈ, ബാം­ഗ്ലൂര്‍, ഹൈ­ദ­രാ­ബാ­ദ് ടീ­മു­ക­ളു­ടെ അവ­കാ­ശ­മാ­ണ് പത്തു വര്‍­ഷ­ത്തേ­ക്ക് 
നൂ­റു­മി­ല്യണ്‍ ഡോ­ള­റി­നു­മേല്‍ തു­ക­യ്ക്കു വി­റ്റു പോ­യ­ത്. ഇതില്‍ ആശ്വാ­സ­ക­ര­മായ സം­ഗ­തി മൂ­ന്നു ടീ­മും സ്വ­ന്ത­മാ­ക്കി­യ­ത് 
ലി­സ്റ്റ­ഡ് കമ്പ­നി­ക­ളാ­ണെ­ന്ന­താ­ണ്. മാ­ത്ര­മ­ല്ല, ഐ­പി­എല്‍ വഴി എങ്ങ­നെ പണ­മു­ണ്ടാ­ക്കാ­മെ­ന്ന­തി­നും, മു­ട­ക്കു­മു­തല്‍ തി­രി­ച്ചു 
പി­ടി­യ്ക്കു­ന്ന­തി­നും ശക്ത­മായ ന്യാ­യീ­ക­ര­ണം മും­ബൈ, ബാം­ഗ്ലൂര്‍ ടീ­മു­ട­മ­കള്‍­ക്കു­ണ്ടാ­യി­രു­ന്നു താ­നും.

­ഹൈ­ദ­രാ­ബാ­ദ് ടീ­മി­ന്റെ സാ­മ്പ­ത്തി­ക­ന­യ­ങ്ങ­ളെ­പ്പ­റ്റി വലിയ വി­വ­ര­ങ്ങ­ളൊ­ന്നും വെ­ളി­യില്‍ വന്നി­ട്ടി­ല്ല. എങ്കി­ലും ഒരു വട്ടം 
ഐപി­എല്‍ ചാ­മ്പ്യന്‍­മാ­രാ­യ­തും, ആന്ധ്രാ­പ്ര­ദേ­ശി­ലെ ആരാ­ധ­ക­വൃ­ന്ദ­വും ഉട­മ­സ്ഥ­രായ ഡെ­ക്കാണ്‍ ക്രോ­ണി­ക്കി­ളി­ന് 
വി­നോ­ദ­വ്യ­വ­സാ­യ­ത്തി­ലു­ള്ള താല്‍­പ്പ­ര്യ­ങ്ങ­ളു­മാ­ക­ണം അവ­രെ നയി­ച്ച­ത്. എന്താ­യാ­ലും ഈ ലക്ക­ത്തില്‍ മും­ബൈ, 
ബാം­ഗ്ലൂര്‍ ടീ­മു­ക­ളു­ടെ സമീ­പ­ന­ത്തി­ലേ­ക്കാ­ണ് കൂ­ടു­തല്‍ ശ്ര­ദ്ധ­കൊ­ടു­ത്തി­രി­ക്കു­ന്ന­ത്.

­ബാം­ഗ്ലൂര്‍ നഗ­ര­ത്തി­നു വേ­ണ്ടി­യു­ള്ള ലേ­ല­ത്തില്‍ വി­ജ­യം കണ്ട­ത് ഇന്ത്യ­യി­ലെ ഒരു ഒന്നാം­കിട സ്പോര്‍­ട്സ് എന്റര്‍­പ്രോ­ണ­റായ 
വി­ജ­യ് മല്യ­യു­ടെ യു­ബി ഗ്രൂ­പ്പാ­ണ്. മും­ബൈ ടീം നേ­ടി­യ­ത് ഇന്ത്യ­യി­ലെ ഏറ്റ­വും വലിയ സ്വ­കാ­ര്യ കമ്പ­നി­യായ ­റി­ല­യന്‍­സ് 
ഇന്‍­ഡ­സ്ട്രീ­സും­.

%image courtesy: http://www.samaylive.com/english/sports/676462175.html

­ക്രി­ക്ക­റ്റും റി­ല­യന്‍­സു­മാ­യു­ള്ള ബന്ധം 1987ല്‍ ഇന്ത്യന്‍ ബോര്‍­ഡി­നെ ഇം­ഗ്ല­ണ്ടി­നു പു­റ­ത്ത് ഏക­ദിന ലോ­ക­ക­പ്പ് 
സം­ഘ­ടി­പ്പി­ക്കാന്‍ സഹാ­യി­ച്ച­തില്‍ തു­ട­ങ്ങു­ന്നു. കോണ്‍­ഗ്ര­സ്സു­കാ­ര­നായ മാ­ധ­വ­റാ­വു സി­ന്ധ്യ­യു­ടെ ബല­ത്തി­ലാ­ണ് 
അന്ന് റി­ല­യന്‍­സ് ലോ­ക­ക­പ്പ് സ്പോണ്‍­സര്‍ ചെ­യ്ത­തെ­ന്ന് ഒരു പറ­ച്ചി­ലു­ണ്ടെ­ങ്കി­ലും, നാ­ലാം ലോ­ക­ക­പ്പി­ന് റി­ല­യന്‍­സ് 
കപ്പ് എന്ന് പേ­രി­ടാന്‍ മാ­ത്രം സഹാ­യ­ങ്ങള്‍ ചെ­യ്ത­വര്‍ പി­ന്നീ­ട് അത്ര­യ്ക്കു മഹാ­മ­ന­സ്കത കാ­ട്ടി­യി­ല്ല എന്ന­താ­ണ് സത്യം.

1996ല്‍ ലോ­ക­ക­പ്പ് ഇന്ത്യ­യി­ലെ­ത്തു­മ്പോള്‍, പു­ക­യി­ല­യ്ക്ക­പ്പു­റം പു­തിയ ലോക സ്വ­പ്നം കണ്ടു തു­ട­ങ്ങിയ ഐടി­സി­യു­ടെ 
വില്‍­സ് ബ്രാന്‍­ഡാ­ണ് ടൈ­റ്റില്‍ സ്പോണ്‍­സേ­ഴ്സാ­യ­ത്. ആ റി­ല­യന്‍­സ് ക്രി­ക്ക­റ്റി­ന്റെ ലോ­ക­ത്തേ­യ്ക്ക് ആഘോ­ഷ­പൂര്‍‌­വ്വം 
എത്തി­യ­തി­ന് സാ­മ്പ­ത്തിക വി­ശാ­ര­ദ­ന്മാര്‍ വി­വി­ധ­കാ­ര­ണ­ങ്ങ­ളാ­ണ് നി­ര­ത്തി­യ­ത്.

ആ വി­ശ­ക­ല­ന­ങ്ങ­ളു­ടെ രത്ന­ച്ചു­രു­ക്കം ഇതാ­യി­രു­ന്നു, ഏറ്റ­വും പതു­ക്കെ ലാ­ഭ­മു­ണ്ടാ­ക്കാന്‍ തു­ട­ങ്ങു­ന്ന ഫ്രാ­ഞ്ചൈ­സി 
മും­ബൈ­യാ­യി­രി­ക്കും. പി­ന്നെ, ഒരു വര്‍­ഷം മൂ­ന്നു മി­ല്യണ്‍ വരെ­യൊ­ക്കെ നഷ്ടം മും­ബൈ സഹി­ക്കും. കാ­ര­ണ­മോ, 
ഉട­മ­സ്ഥ­രായ മു­കേ­ഷ്, നി­താ അം­ബാ­നി ദമ്പ­തി­കള്‍­ക്കു ലഭി­ക്കു­ന്ന ടി­വി പ്രൈം ടൈ­മും സൌ­ജ­ന്യ പര­സ്യ­വും­.

%image courtesy: http://thecurrentaffairs.com/ipl-will-be-responsible-for-players-security-modi.html­
%നിത അം­ബാ­നി­ ബോ­ളി­വു­ഡ് നടി കരീന കപൂ­റി­നും സച്ചിന്‍ ടെ­ണ്ടുല്‍­ക്ക­റു­ടെ ഭാ­ര്യ അഞ്ജ­ലി­ക്കു­മൊ­പ്പം" height="364" width="430" />

ഇ­ന്ത്യ­യി­ലെ ഏറ്റ­വും പണ­ക്കാ­രായ ദമ്പ­തി­കള്‍­ക്ക് സൌ­ജ­ന്യ­മാ­യി തങ്ങള്‍ എത്ര­മാ­ത്രം മി­ഡില്‍ ക്ലാ­സാ­ണെ­ന്നു 
കാ­ണി­ക്കാന്‍ കി­ട്ടു­ന്ന അവ­സ­ര­ങ്ങള്‍ അവ­രു­ടെ കമ്പ­നി­കള്‍­ക്കു ചെ­യ്യു­ന്ന ഗു­ണ­ങ്ങള്‍ പല മട­ങ്ങാ­ണ്. മാ­ത്ര­മ­ല്ല, 
ഐപി­എല്‍ ഹോം മത്സ­ര­ങ്ങള്‍ നി­താ അം­ബാ­നി തന്റെ സാ­മൂ­ഹ്യ­സേ­വന സന്ന­ദ്ധത തു­റ­ന്നു കാ­ണി­ച്ച് ഉപ­യോ­ഗി­ക്കു­ന്നു. 
വേ­റെ ഒരു മാ­ധ്യ­മ­വും അം­ബാ­നി ദമ്പ­തി­ക­ളു­ടെ സാ­മൂ­ഹ്യ സേ­വ­ന­ത്തെ ഇത്ര­യും പ്ര­കീര്‍­ത്തി­ച്ചി­ട്ടു­ണ്ടാ­വി­ല്ലെ­ന്നു­ള്ള­തു­ത­ന്നെ 
അവര്‍­ക്കു ടീ­മില്‍ വരു­ന്ന നഷ്ടം നി­ക­ത്തു­ന്നു. അവര്‍­ക്കു സൌ­ജ­ന്യ­വി­ല­യ്ക്കു ലഭി­ച്ച ­ക്രി­ക്ക­റ്റ് ദൈ­വ­ത്തേ­യൂം നന്നാ­യി ഉപ­യോ­ഗി­ച്ച് 
നഷ്ടം കു­റ­യ്ക്കാന്‍ സാ­ധി­ച്ചി­രു­ന്നു (സ­ച്ചി­നും സഹീ­റും ഹര്‍­ബ­ജ­നും അണി­നി­ര­ന്ന 2009ല്‍ ഐഡിയ ­പ­ര­സ്യം­ ഉദാ­ഹ­ര­ണം­).

ഈ ഘട­ക­ങ്ങ­ളി­ലൂ­ന്നി സാ­മ്പ­ത്തിക ലക്ഷ്യ­ങ്ങള്‍ നി­ശ്ച­യി­ച്ച മും­ബൈ ടീം ആവ­റേ­ജ് പ്ര­ക­ട­നം കാ­ഴ്ച­വ­ച്ചാല്‍­പ്പോ­ലും ഉട­മ­സ്ഥ­രെ 
ഭയ­പ്പെ­ടു­ത്താന്‍­മാ­ത്രം പ്ര­ശ്ന­ങ്ങ­ളു­ള്ള ഒന്നാ­യി­രു­ന്നി­ല്ല. ഇതി­നൊ­പ്പം ടീം മര്‍­ച്ചന്‍­ഡൈ­സ് വി­പ­ണി കൂ­ടി ചേര്‍­ത്താല്‍ കി­ട്ടു­ന്ന 
ഫലം അമ്പ­ര­പ്പി­ച്ചി­ല്ലെ­ങ്കി­ലും ഒരി­ക്ക­ലും നി­രാ­ശാ­ജ­ന­ക­മ­ല്ല. മാ­ത്ര­മ­ല്ല, മും­ബൈ നഗ­ര­ത്തെ പ്ര­തി­നി­ധീ­ക­രി­ക്കു­ന്ന­തു­കൊ­ണ്ട് ഒരു 
പരി­ധി­വ­രെ നല്ല സ്പോണ്‍­സര്‍­മാ­രെ ആകര്‍­ഷി­ക്കാ­നും ടീ­മി­നു കഴി­ഞ്ഞു­.

­മും­ബൈ ഏറ്റ­വും യാ­ഥാ­സ്ഥി­തി­ക­മായ രീ­തി­യില്‍ ഒരു സ്പോര്‍­ട്സ് ഫ്രാ­ഞ്ചൈ­സി എങ്ങ­നെ നട­ത്താം എന്നാ­ണ് പരീ­ക്ഷി­ച്ച­ത്. 
ആവ­ശ്യ­മി­ല്ലാ­ത്ത റി­സ്കു­കള്‍ ഒഴി­വാ­ക്കി, എല്ലാ­യ്പ്പോ­ഴും സാ­മ്പ­ത്തിക നഷ്ടം­പോ­ലും ചില സാ­മൂ­ഹിക നേ­ട്ട­ങ്ങള്‍ തരു­മെ­ന്നു­റ­പ്പാ­ക്കി 
വ്യ­ക്ത­മായ പ്ലാ­നോ­ടു­കൂ­ടി കള­ത്തി­ലി­റ­ങ്ങിയ അവ­സ്ഥ. അവ­രു­ടെ ടീം അടു­ത്ത പത്തു വര്‍­ഷ­ത്തേ­ക്ക് ഒരി­ക്ക­ലും ഐപി­എല്‍ 
ജേ­താ­ക്ക­ളാ­യി­ല്ലെ­ങ്കില്‍ ഉണ്ടാ­കാ­വു­ന്ന സഞ്ചിത നഷ്ടം എങ്ങ­നെ മറ്റു വഴി­ക­ളി­ലൂ­ടെ പരി­ഹ­രി­ക്കാം എന്ന­ത് ആദ്യ­മേ മും­ബൈ­യു­ടെ 
കണ­ക്കു­പു­സ്ത­ക­ങ്ങ­ളില്‍ ഇടം­പി­ടി­ച്ചി­രി­ക്കാം എന്നാ­ണ് പല വി­ശ­ക­ലന വി­ദ­ഗ്ദ­രു­ടെ­യും വാ­ദം­.

­ബാം­ഗ്ലൂര്‍ ടീ­മി­ന്റെ കഥ കു­റ­ച്ചു വ്യ­ത്യ­സ്ഥ­മാ­ണ്. റി­ല­യന്‍­സി­നെ­പ്പോ­ലെ പബ്ലി­ക് ലി­മി­റ്റ­ഡ് കമ്പ­നി­യായ യു­ബി ഗ്രൂ­പ്പാ­ണ് 
ബാം­ഗ്ലൂര്‍ ടീ­മി­ന്റെ ഉട­മ­സ്ഥര്‍. പക്ഷെ പ്ര­ധാന പ്ര­മോ­ട്ട­റായ ­വി­ജ­യ് മല്യ മു­കേ­ഷ്-നി­താ അം­ബാ­നി ദമ്പ­തി­ക­ളില്‍ നി­ന്നു 
വ്യ­ത്യ­സ്ഥ­മാ­യി പ്ര­ശ­സ്ത സ്പോര്‍­ട്സ് ഇന്‍­വെ­സ്റ്റ­റാ­ണ്. ഫോ­ഴ്സ് ഇന്ത്യ ഫോര്‍­മു­ലാ വണ്‍ ടീ­മാ­ണ് മല്യ­യു­ടെ ഒരു പ്ര­ധാന 
സ്പോര്‍­ട്സ് നി­ക്ഷേ­പം. മറ്റൊ­രു നി­ക്ഷേ­പം ഇന്ത്യന്‍ കു­തി­ര­യോ­ട്ട രം­ഗ­ത്ത് പു­നെ സമ്രാ­ട്ടു­ക­ളെ വെ­ല്ലു­വി­ളി­ക്കു­ന്ന കു­തി­ര­ക­ളൂ­ടെ 
ഉട­മ­ക­ളായ യു­ണൈ­റ്റ­ഡ് റേ­സി­ങ് ആന്റ് ബ്ല­ഡ്സ്റ്റോ­ക്ക് ബ്രീ­ഡേ­ഴ്സാ­ണ്. ഇന്ത്യന്‍ ഫു­ട്ബാ­ളില്‍, ഈസ്റ്റ് ബം­ഗാള്‍, 
മോ­ഹന്‍ ബഗാന്‍ ടീ­മു­ക­ളു­ടെ ടൈ­റ്റില്‍ സ്പോണ്‍­സ­റാ­ണ് ­യു­ബി ഗ്രൂ­പ്പ്. മാ­ത്ര­മ­ല്ല ഈസ്റ്റ് ബം­ഗാള്‍ ക്ല­ബ്ബില്‍ അമ്പ­തു 
ശത­മാ­നം ഓഹ­രി­യും മല്യ­ക്കു സ്വ­ന്ത­മാ­യു­ണ്ട്.

%image courtesy: http://www.bollywoodraj.com/2010/04/deepika-padukone-with-siddharth-mallya.html

­മ­ല്യ­യു­ടെ സ്പോര്‍­ട്സ് നി­ക്ഷേ­പ­ങ്ങ­ളു­ടെ ഒരു പ്ര­ത്യേ­ക­ത­യെ­ന്തെ­ന്നാല്‍, അവ­യെ­ല്ലാം യു­ബി ഗ്രൂ­പ്പി­ന്റെ പ്ര­ധാന ബി­സി­ന­സ്സായ 
മദ്യ­ക്ക­ച്ച­വ­ട­ത്തെ പരി­പോ­ഷി­ക്കു­ന്ന തര­ത്തി­ലാ­ണ് പ്ലേ­സ് ചെ­യ്തി­രി­ക്കു­ന്ന­തെ­ന്നാ­ണ്. ഇന്ത്യ­യില്‍ കു­തി­ര­യോ­ട്ട­വും പന്ത­യ­വും 
നട­ത്തു­ന്ന വലിയ പണ­ക്കാ­രു­ടെ­യി­ട­യില്‍ സ്വ­ന്തം ബ്രാന്‍­ഡു­ക­ളു­ടെ സ്വാ­ധീ­നം വര്‍­ധി­പ്പി­ക്കാ­നാ­ണ് മല്യ കു­തി­ര­ക­ളെ 
ഉപ­യോ­ഗി­ക്കു­ന്ന­ത്. ഫോര്‍­മുല വണ്ണി­ന്റെ പ്ര­ഭ­വ­സ്ഥാ­ന­മായ യൂ­റോ­പ്പി­ലെ യു­ബി ഗ്രൂ­പ്പി­ന്റെ അടി­ത്തറ വി­പു­ലീ­ക­രി­ക്കു­ന്ന­തി­നാ­ണ് 
ഫോ­ഴ്സ് ഇന്ത്യ ടീ­മി­നെ മല്യ ഉപ­യോ­ഗി­ക്കു­ന്ന­ത്. ബം­ഗാ­ളി­ലെ സാ­ധാ­രണ കു­ടി­യ­ന്മാ­രാ­ണ് കൊല്‍­ക­ത്ത ടീ­മു­ക­ളി­ലൂ­ടെ 
ലക്ഷ്യ­മാ­ക്കി­യ­തെ­ങ്കില്‍, ഐപി­എല്‍ ടീം അഖി­ലേ­ന്ത്യാ­ത­ല­ത്തില്‍ ക്രി­ക്ക­റ്റ് ആരാ­ധ­ക­രു­ടെ പ്ര­ധാന ബ്രാന്‍­ഡാ­വു­ന്ന­തി­നു­ള്ള 
അട­വാ­യി­രു­ന്നു. എന്റര്‍­ടൈന്‍­മൈ­ന്റ് ആന്റ് സ്പോര്‍­ട്സ് ഡയ­റ­ക്റ്റു­മാ­യി സഹ­ക­രി­ച്ച് റോ­യല്‍ ചാ­ല­ഞ്ചേ­ഴ്സ് സ്പോര്‍­ട്സ് 
വി­വിധ നഗ­ര­ങ്ങ­ളി­ലെ ക്ല­ബ്ബു­ക­ളില്‍ നട­ത്തു­ന്ന ഐപി­എല്‍ രാ­വു­കള്‍ ഈ സ്ട്രാ­റ്റ­ജി­യു­ടെ നേ­രു­ദാ­ഹ­ര­ണ­മാ­ണ്.

അ­തു­കൊ­ണ്ടു തന്നെ പു­റ­മേ­നി­ന്നു­ള്ള സ്പോണ്‍­സര്‍­മാ­രെ പ്രോ­ത്സാ­ഹി­പ്പി­ക്കാ­ത്ത മല്യ­യു­ടെ നയം കാ­ര്യ­ങ്ങള്‍ കൂ­ടു­തല്‍ 
വ്യ­ക്ത­മാ­ക്കു­ന്നു. അതി­നാല്‍ റോ­യല്‍ ചാ­ല­ഞ്ചേ­ഴ്സ് ബാം­ഗ്ലൂര്‍ ടീം ഉണ്ടാ­ക്കു­ന്ന ലാ­ഭ­ത്തില്‍ ഐപി­എല്‍ മാ­മാ­ങ്ക­ത്തില്‍ 
നി­ന്ന് മല്യ­യു­ടെ മദ്യ ബ്രാ­ന്റു­കള്‍ ഉണ്ടാ­ക്കു­ന്ന ലാ­ഭം കൂ­ടി ചേര്‍­ക്കേ­ണ്ട­താ­ണ്. ഇത്ത­ര­ത്തില്‍ ഐപി­എ­ല്ലി­ലെ വില കൂ­ടിയ 
രണ്ടു ടീ­മു­ക­ളും വ്യ­ക്ത­മായ ബി­സി­ന­സ്സ് ലക്ഷ്യ­ങ്ങ­ളും കണ­ക്കു­ക­ളു­മാ­യാ­ണ് ആളു­ക­ളെ അമ്പ­ര­പ്പി­ക്കു­ന്ന­തെ­ങ്കില്‍ നേ­രെ 
എതിര്‍ ധ്രു­വ­ത്തില്‍ നില്‍­ക്കു­ന്ന വമ്പന്‍­മാ­രു­മു­ണ്ട്. അവ­രെ­ക്കു­റി­ച്ച് അടുത്ത ലേ­ഖ­ന­ത്തില്‍.
(10 May 2010)\footnote{http://malayal.am/പലവക/പരമ്പര/ബിസിനസ്-ലീഗ്/5378/അംബാനി-മുതല്‍-മല്യ-വരെ}

\newpage
