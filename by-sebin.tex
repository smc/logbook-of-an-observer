\newpage
\secstar{ജീവിക്കുവാനുള്ള കാരണങ്ങള്‍}
{\vskip 2pt}

എത്രതവണ ഈ കുറിപ്പെഴുതാനിരുന്നിട്ടു് കരഞ്ഞുവീര്‍ത്ത കണ്ണുകളുമായി എഴുന്നേറ്റുപോയെന്നറിയില്ല. ഏതു സ്വകാര്യദുഃഖത്തേയും
സ്റ്റോറിയായി മാത്രം കണ്ടുപരിചയമുള്ള ഒരു പ്രൊഫഷനില്‍ ഇങ്ങനെ സംഭവിക്കാന്‍ പാടുള്ളതല്ല. പക്ഷെ ജിനേഷിന്റെ കാര്യത്തില്‍ നിയമങ്ങള്‍ തെറ്റുന്നു.

ഇന്നലെ ഉച്ചയ്ക്കാണു് ജിനേഷിന്റെ വിയോഗമറിയുന്നതു്. വെല്ലൂര്‍ സിഎംസിയില്‍ ക്യാന്‍സറിനോടു് പൊരുതിത്തോറ്റ ഞങ്ങളുടെ 
പ്രിയപ്പെട്ട സഹോദരനു് ഇന്നു മണ്ണാര്‍ക്കാടു് അന്ത്യവിശ്രമമായി. ജിനേഷ് കാഞ്ഞിരങ്ങാട്ടില്‍ ജയരാമന്‍ എന്ന ജിന്‍സ്ബോണ്ടിനു്
മലയാളത്തിന്റെ കണ്ണീരില്‍ക്കുതിര്‍ന്ന യാത്രാമൊഴി.

അറിയില്ല, എന്തൊക്കെയാണെഴുതേണ്ടതെന്നു്. പലതരത്തിലുള്ള ബന്ധമായിരുന്നു എനിക്കു് ജിനേഷുമായി ഉണ്ടായിരുന്നതു്. 
സ്വതന്ത്ര മലയാളം കമ്പ്യൂട്ടിങ്ങിലെ സജീവാംഗം എന്നനിലയിലുള്ള ബന്ധമാണു് ആദ്യത്തേതു്. അതുവഴി ജിനേഷിന്റെ 
ബ്ലോഗിലേക്കും എത്തിപ്പെട്ടു. The log book of an observer എന്നായിരുന്നു അവന്റെ ബ്ലോഗിനു പേരിട്ടിരുന്നതു്. നിരീക്ഷകന്റെ 
ആ നാള്‍വഴിപ്പുസ്തകം ഇന്നു് ഇന്‍വൈറ്റഡ് റീഡേഴ്സിനുവേണ്ടി മാത്രം തുറന്നിട്ടിരിക്കയാണു്. അതിലെ കുറിപ്പുകള്‍ 
എവിടെപ്പോയോ എന്തോ... ബ്ലോഗ് വായനയിലൂടെ ശക്തമായ ബന്ധം പിന്നീടു് ഗാഢമായ സുഹൃദ്ബന്ധമായി മാറി.

മലയാളം എന്ന ഈ വെബ്സൈറ്റ് തുറന്നതോടെ തുടക്കംമുതല്‍തന്നെ ഞങ്ങളുടെ ടീമില്‍ ഒരംഗമായി ജിനേഷ് മാറി. 
മലയാളരാജ്യത്തിനുവേണ്ടി ഫോര്‍മുല വണ്‍ റേസുകള്‍ റിവ്യൂ ചെയ്തു. ഐപിഎല്‍ എന്ന കായികമാമാങ്കത്തിന്റെ വാതില്‍പ്പുറം 
കളികളെക്കുറിച്ചെഴുതി. ആസ്റ്റെറിക്സ് എന്ന കാര്‍ട്ടൂണ്‍ പരമ്പരയെക്കുറിച്ചും ദി ബിഗ് ബാങ് തിയറി എന്ന സിറ്റ്കോമിനെക്കുറിച്ചും 
സ്വത്വരാഷ്ട്രീയത്തെക്കുറിച്ചും മാദ്ധ്യമങ്ങളുടെ പരിണാമത്തെക്കുറിച്ചുമൊക്കെ ഒരേ തീവ്രതയോടെ ജിനേഷ് എഴുതി. തലമുടിയെക്കുറിച്ചു് 
ഒരുപന്യാസം എന്ന ഏറെക്കുറെ പേഴ്സണലായ കുറിപ്പും മലയാളത്തിലെഴുതി. ഇവിടെ ഞങ്ങള്‍ പ്രസിദ്ധീകരിക്കുന്ന സ്റ്റോറികളില്‍ 
ഫാക്ച്വല്‍ എറേഴ്സ് വരുമ്പോളെല്ലാം അതുതിരുത്താന്‍ ഓടിയെത്തി. എഴുതിയതിനേക്കാള്‍ കൂതല്‍ ജിമെയ്ല്‍ ചാറ്റുകളില്‍ 
പറഞ്ഞുതീര്‍ത്തു.

ഫിലോസഫി, പൊളിറ്റിക്സ്, സൊസൈറ്റി, സയന്‍സ്, കമ്പ്യൂട്ടേഷനല്‍ ലിംഗ്വിസ്റ്റിക്സ്, മെട്രോ സെക്‌ഷ്വാലിറ്റി എന്നിങ്ങനെ വിവിധവിഷയങ്ങളില്‍ 
പടര്‍ന്നുകിടന്നു, ഞങ്ങളുടെ സംഭാഷണങ്ങള്‍. രോഗക്കിടക്കയില്‍ അനാരോഗ്യത്തിന്റെ ഇടവേളകളില്‍ 
അല്‍പ്പാല്‍പ്പം സംസാരിക്കാറാവുമ്പോഴെല്ലാം ജിചാറ്റില്‍ വന്നുനിറഞ്ഞൂ, ജിനേഷ്. തന്നെ കാര്‍ന്നുതിന്നുന്ന ക്യാന്‍സറിനെക്കുറിച്ചു്, 
ക്യാന്‍സറിന്റെ വെളിച്ചത്തില്‍ ജീവിതത്തെക്കുറിച്ചു്, തന്റെ അടുത്ത ബഡ്ഡില്‍ കിടക്കുന്ന രോഗിണിയായ പെണ്‍കുട്ടിയെക്കുറിച്ചു് 
ഒക്കെ ഇംഗ്ലീഷില്‍ ഏതാനും കുറിപ്പുകളെഴുതി, അവ എനിക്കയച്ചുതന്നു. മലയാളത്തിലേക്കു് തീവ്രതകുറയാതെ പരിഭാഷചെയ്യാന്‍ 
അവ ബ്ലോഗിലൂടെ പരിചിതനായ ഡോ. സൂരജിനെ ഏല്‍പ്പിച്ചു. സൂരജുമായി തന്റെ മെഡിക്കല്‍ ഹിസ്റ്ററി പങ്കുവച്ചു. മരണമല്ലാതെ 
മറ്റൊരു രക്ഷാമാര്‍ഗ്ഗമില്ലാത്ത അര്‍ബുദവകഭേദമായിരുന്നു, ജിനേഷിന്റേതു്. അക്കാര്യം ഡോക്ടര്‍മാരില്‍നിന്നു് കൃത്യമായി 
ജിനേഷ് മനസ്സിലാക്കിയിരുന്നു. സൂരജും അതുതന്നെ എന്നോടുപറഞ്ഞപ്പോള്‍ ഞാന്‍ നീറിയ നീറ്റല്‍ ...

എന്നാല്‍ ഇതൊന്നും വലിയ കാര്യമല്ല എന്ന മട്ടിലായിരുന്നു, ജിനേഷ്. തന്റെ ആരോഗ്യത്തെക്കുറിച്ചും ചികിത്സയെക്കുറിച്ചും 
ചോദിക്കുന്ന സൈബര്‍ സുഹൃത്തുക്കള്‍ക്കു് ജിനേഷ് കാട്ടിക്കൊടുത്തിരുന്നതു് xkcdയിലെ രണ്ടു കാര്‍ട്ടൂണുകളായിരുന്നു 
(\url{http://xkcd.com/931/}  , \url{http://xkcd.com/938/}) എന്നു് അനിവര്‍ അരവിന്ദ് എസ്എംസി മെയിലിങ് ലിസ്റ്റില്‍ എഴുതിയ 
ചെറുകുറിപ്പില്‍\footnote{'Anivar's email to Swathanthra Malayalam Computing' എന്ന ലേഖനം കാണുക} അനുസ്മരിക്കുന്നു. 

നിര്‍ത്താത്ത പനിയും നടുവേദനയും കാലിനു കഴപ്പും ചുമയുമൊക്കെയായിട്ടാണു് ജിനേഷിനെ ആദ്യം പരിശോധനയ്ക്കു 
കൊണ്ടുപോകുന്നതു്. അകാരണമായ ക്ഷീണമായിരുന്നു മറ്റൊരു ലക്ഷണം. ആശുപത്രിയിലെത്തുമ്പോഴേക്കും ഈ 
കാരണങ്ങള്‍വച്ചു് ജിനേഷ് അര്‍ബുദം ഊഹിച്ചിരുന്നു. കണ്‍ഫേം ചെയ്തശേഷം അതേക്കുറിച്ചു് ഒരു ഗവേഷകനുമാത്രം 
കഴിയുന്ന നിര്‍മ്മമതയോടെ വായിച്ചുപഠിച്ചു. പുതിയപുതിയ മെഡിക്കല്‍ ജേണലുകളില്‍ അതേക്കുറിച്ചുള്ള വിവരങ്ങള്‍ പരതി 
താന്‍ പെട്ടിരിക്കുന്ന അവസ്ഥ കൃത്യമായി മനസ്സിലാക്കി.

ജിനേഷിനു് താന്‍ തിരിച്ചുവരില്ലെന്നു് അറിയാമായിരുന്നു. അന്ത്യശ്വാസംവരേയും അര്‍ബുദത്തെ അവഗണിച്ചും താന്‍ 
ചെയ്തുകൊണ്ടിരുന്ന കാര്യങ്ങളില്‍ വല്ലാത്തൊരാസക്തിയോടെ വ്യാപൃതനാവാന്‍ ആ തിരിച്ചറിവു് ജിനേഷിനു ബലം 
പകരുകയാണുണ്ടായതു്. മരണഭീതിയില്‍ തകര്‍ന്നില്ലാതെയാകുന്നതായിരുന്നില്ല, ഭാസുരമായ ഭാവിയെക്കുറിച്ചു മാത്രം 
ചിന്തിക്കുന്നതായിരുന്നു, ആ നിശ്ചയദാര്‍ഢ്യം. താന്‍ ചെയ്യുന്ന കാര്യങ്ങള്‍ തനിക്കു് ഉപകാരപ്പെട്ടില്ലെങ്കിലും മറ്റുള്ളവര്‍ക്കു് 
ഉപകാരപ്പെടുമെന്നു് ജിനേഷിനു് ഉറപ്പുണ്ടായിരുന്നു. അതില്‍ കണ്ടെത്തിയ ആനന്ദമാണു് വേദനയെ അല്‍പ്പാല്‍പ്പമെങ്കിലും 
മറികടക്കാന്‍ സഹായിച്ചതു്.

ഇടയ്ക്കു് ക്യാന്‍സര്‍ ശരീരത്തില്‍നിന്നു് പൂര്‍ണ്ണമായി മാറിയെന്നു പറഞ്ഞ ഘട്ടത്തില്‍ മാത്രമാണു് ജിനേഷ് 
സ്വസ്ഥജീവിതത്തിലേക്കുള്ള മടക്കം പ്രതീക്ഷിച്ചതു്. അപ്പോഴാകട്ടെ, ജീവിതാസക്തി അതിന്റെ എല്ലാത്തിളക്കത്തോടും 
വന്നുദിക്കുന്നതുകണ്ടു. രോഗമടങ്ങി പാലക്കാടു വീട്ടിലെത്തിയാലുടന്‍ കൂട്ടുകാരെക്കൂട്ടി ഒരു ഒത്തുകൂടല്‍ നടത്തുന്നതിനെക്കുറിച്ചുവരെ 
സെപ്തംബറില്‍ സംസാരിച്ചു.

കീമോയ്ക്കുശേഷമുള്ള ബലഹീനതയുടെ നാളുകളില്‍ കുഴല്‍ഭക്ഷണം മാത്രം ഉള്ളിലേക്കെടുക്കവേ കമ്പ്യൂട്ടര്‍ ഉപയോഗിക്കാനാവാതെ 
വന്നപ്പോഴും ഞങ്ങളോടൊക്കെ ഓരോ പുസ്തകത്തിന്റെ പേരുപറഞ്ഞു് അവ വാങ്ങി അയച്ചുതരാന്‍ ആവശ്യപ്പെട്ടു. ചരിത്രവും 
സമൂഹവുമായി ബന്ധപ്പെട്ട ഒട്ടേറെ പുസ്തകങ്ങളാണു് ഇക്കാലയളവില്‍ ജിനേഷ് കുടിച്ചുവറ്റിച്ചതു്. അവയില്‍ നിന്നെല്ലാം 
വ്യത്യസ്തമായി കസാന്‍ദ്സാക്കീസിന്റെ റിപ്പോര്‍ട്ട് ടു ഗ്രീക്കോ എന്ന ആത്മകഥാപുസ്തകത്തിലും മുങ്ങിത്താഴ്ന്നു. ഓരോ വായനയിലും 
പുതുമസമ്മാനിക്കുന്ന, ഓരോ അടരുകളിലും ജീവിതകാമന കലര്‍ന്നിരിക്കുന്ന ഇപ്പുസ്തകം വായിക്കാനെന്തേ ഇത്രയും താമസിച്ചൂ 
എന്നു് എന്നോടു് അത്ഭുതംകൂറി.   അതേ തീത്തിളക്കത്തോടെ ചിന്തയില്‍ വിസ്ഫോടനം സൃഷ്ടിച്ച തത്വചിന്തകരേയും ഉള്‍ക്കൊണ്ടു.

എഫ്ഇസി പോലെയുള്ള സൈബര്‍ ഇടങ്ങളില്‍ വല്ലപ്പോഴും മാത്രം വായ തുറന്നു. പറയാന്‍ കാര്യമില്ലാത്തതായിരുന്നില്ല, 
നിറ്റ്പിക്കിങ് തീരെ താത്പര്യമില്ലാതിരുന്നതാണു് ഈ ഒതുങ്ങലിനു കാരണം. തനിക്കു് പറയാനുള്ളതു് സുഹൃത്തുക്കളോടു 
പറഞ്ഞുതീര്‍ക്കുകയായിരുന്നു, ജിനേഷ്. എന്നിട്ടും ഫോര്‍ത്തു് എസ്റ്റേറ്റ് ക്രിട്ടിക്‍ എന്ന ഡിസ്കഷന്‍ ഗ്രൂപ്പില്‍ ജിനേഷ് എഴുതിയ 
എണ്ണിപ്പെറുക്കിയ മെയിലുകള്‍ സുചിന്തിതവും ആലോചനാമൃതവുമായ വാദമുഖങ്ങള്‍ക്കു കുന്തമുനകളായി. ലോ വെയ്സ്റ്റ് 
ജീന്‍സിനെക്കുറിച്ചും സിനിമാറ്റിക്‍ ഡാന്‍സിനെക്കുറിച്ചും ആഫ്റ്റര്‍ മാച്ചു് പാര്‍ട്ടികളെക്കുറിച്ചും യുഐഡിയെക്കുറിച്ചും ഒക്കെ 
ജിനേഷ് പങ്കുവച്ച അഭിപ്രായങ്ങള്‍ കണ്‍സര്‍വേറ്റീവു് മനോഘടനകളുടെ സ്വാസ്ഥ്യംകെടുത്തി. ഷാമ്പൂചെയ്തു മനോഹരമാക്കി 
സൂക്ഷിച്ചിരുന്ന തന്റെ നീളന്‍ മുടിയെക്കുറിച്ചും കീമോയുടെ ഫലമായി അതുകൊഴിഞ്ഞതിനെക്കുറിച്ചുമൊക്കെയെഴുതിയ 
ഹൃദയഭേദകമായ കുറിപ്പാണു് എഫ്ഇസിയില്‍ ജിനേഷ് അവസാനമായി എഴുതിയ മെയിലുകളില്‍ ഒന്നു്. യൂണിക്കോഡ് 
മെയിലിങ് ലിസ്റ്റിലും എസ്എംസി ലിസ്റ്റിലുമായി ചില്ലക്ഷരവിവാദം കൊടുമ്പിരിക്കൊണ്ട കാലത്തു് ജിനേഷ് എഴുതിയ വരികളും 
ഓര്‍ക്കാതെവയ്യ.

അതേ നിമിഷം, ജീവിച്ച 24 വര്‍ഷങ്ങളിലും അവനോടു് അടുത്തുപരിചയപ്പെടാന്‍ സാധിച്ച ഓരോരുത്തര്‍ക്കും അഭിമാനത്തോടെ 
ഓര്‍ക്കാനുള്ളതുമാത്രം മിച്ചംവച്ചു. ചൂടേറിയ സൈബര്‍ തര്‍ക്കങ്ങളില്‍ വല്ലപ്പോഴും തലയിട്ടു് അഭിപ്രായം പറഞ്ഞിരുന്ന ഈ 
ചെറുപ്പക്കാരനോടു് കാലുഷ്യം സൂക്ഷിക്കാന്‍ എതിരഭിപ്രായക്കാര്‍ക്കുപോലും കഴിഞ്ഞിരുന്നില്ല. അത്രയ്ക്കു സൗമ്യദീപ്തമായി 
സംഭാഷണങ്ങളെ ഒരുക്കിയെടുക്കാന്‍ ജിനേഷിനു കഴിഞ്ഞിരുന്നു.

മലയാളം കമ്പ്യൂട്ടിങ്ങിനു് ജിനേഷ് നല്‍കിയ സംഭാവനകളെ എണ്ണിപ്പറയാതെ ഈ കുറിപ്പു് അവസാനിപ്പിക്കാനാവില്ല. 
ഇന്ത്യന്‍ ഭാഷകള്‍ക്കായി സിഡാക്‍ വികസിപ്പിച്ച ഇന്‍സ്ക്രിപ്റ്റ് ഇന്‍പുട്ടു് മെഥേഡിന്റെയും ആംഗലേയത്തില്‍ മലയാളമെഴുതുന്ന 
മൊഴിയും സ്വനലേഖയും അടക്കമുള്ള phoenomic രീതികളുടെയും സങ്കലനമായി ഗ്നൂ ലിനക്സ് ഓപ്പറേറ്റിങ് സിസ്റ്റങ്ങളില്‍ ലഭ്യമായ 
ലളിത എന്ന ഇന്‍പുട്ടു് മെഥേഡ് ജിനേഷിന്റെ സൃഷ്ടിയാണു്. 2007ലെ ഗൂഗിള്‍ സമ്മര്‍ ഓഫ് കോഡ് പ്രോജക്ടിലേക്കു് എസ്എംസി 
തിരഞ്ഞെടുത്ത അഞ്ചുവിദ്യാര്‍ത്ഥികളില്‍ ഒരാളായാണു് ജിനേഷ് സ്വതന്ത്ര മലയാളം കമ്പ്യൂട്ടിങ്ങില്‍ സജീവമാകുന്നതു്. ഗ്നൂ 
പ്രോജക്ടിന്റെ താളുകള്‍ മലയാളത്തിലേക്കു പരിഭാഷപ്പെടുത്തുന്ന ടീമിന്റെ ചുമതല വഹിച്ചിരുന്ന ശ്യാം ആത്മഹത്യ 
ചെയ്തതിനെത്തുടര്‍ന്നു് ആ പ്രോജക്ടിന്റെ കണ്‍വീനറായും ജിനേഷ് പ്രവര്‍ത്തിച്ചു. ശ്യാം കോഡിനേറ്റ് ചെയ്ത പരിഭാഷകളെ 
പുസ്തകമാക്കുക എന്ന ലക്ഷ്യത്തോടെ അവ എഡിറ്റ് ചെയ്യുന്ന കര്‍ത്തവ്യവും ജിനേഷ് നിര്‍വ്വഹിച്ചു. കോ എഡിറ്ററാവാന്‍ 
എന്നോടാവശ്യപ്പെട്ടിരുന്നെങ്കിലും സ്വതഃസിദ്ധമായ മടിയും മറ്റുതിരക്കുകളും മൂലം എനിക്കു കഴിഞ്ഞതുമില്ല.

മലയാളത്തിലാരംഭിച്ചു് സന്തോഷ് തോട്ടിങ്ങലിന്റെയും വാസുദേവിന്റെയും മറ്റും നേതൃത്വത്തില്‍ ഇന്ത്യയിലും വിദേശത്തുമുള്ള ഒട്ടേറെ 
ഭാഷകളിലേക്കു് വളര്‍ന്ന ശില്‍പ്പ പ്രോജക്ടില്‍ തുടക്കം മുതല്‍തന്നെ ജിനേഷിന്റെ പങ്കാളിത്തമുണ്ടായിരുന്നു. ശില്‍പ്പയെക്കുറിച്ചു് 
ഒട്ടേറെ പ്രതീക്ഷകള്‍ വച്ചുപുലര്‍ത്തുന്നയാളെന്ന നിലയില്‍ ഈ പ്രോജക്ടില്‍ ജിനേഷ് അവതരിപ്പിച്ച മാറ്റങ്ങളും 
നിര്‍ദ്ദേശങ്ങളുമൊക്കെ എത്രമാത്രം വിലപ്പെട്ടതാണെന്നു തിരിച്ചറിയുന്നു.

ഹൈദരാബാദ് ഐഐഐടിയില്‍ ജിനേഷ് ഉള്‍പ്പെടുന്ന ഗവേഷകസംഘം മലയാളം ഒസിആര്‍ വികസിപ്പിക്കാനുള്ള 
പരിശ്രമത്തിലായിരുന്നു. കേന്ദ്രസര്‍ക്കാര്‍ സ്പോണ്‍സര്‍ ചെയ്യുന്ന ഈ പ്രോജക്ട് പ്രൊപ്രൈറ്ററി മോഡില്‍ നീങ്ങുന്നതിനെക്കുറിച്ചു് 
ഉള്ളുരുകുന്ന സങ്കടം ജിനേഷ് സൂക്ഷിച്ചു. അതിനെ മറികടക്കാന്‍ ടെസറാക്ട് ഒസിആറിനെ മലയാളം പഠിപ്പിക്കാനുള്ള സ്വതന്ത്ര 
സോഫ്റ്റ്‌വെയര്‍ പ്രോജക്ടിനുവേണ്ടിയും ജിനേഷ് കോഡെഴുതി. നിര്‍ഭാഗ്യവശാല്‍ ഇവയൊന്നും പൂര്‍ണ്ണമായും 
ഉപയോഗയുക്തമാകുന്ന അവസ്ഥയിലേക്കു് ഇനിയും എത്തിയിട്ടില്ല.

സാങ്കേതികവിദ്യയും ഭാഷയും തമ്മിലുള്ള സങ്കലനത്തില്‍ ഉടലെടുക്കുന്ന പ്രശ്നങ്ങള്‍ പരിഹരിക്കുന്നതില്‍ പ്രത്യേക ശ്രദ്ധ ജിനേഷ് 
അര്‍പ്പിച്ചിരുന്നു. മലയാളത്തില്‍ ഇന്റര്‍നാഷണല്‍ ഡൊമെയ്ന്‍ നെയിം (IDN) സ്റ്റാന്‍ഡേര്‍ഡ് സെറ്റ് ചെയ്യുന്നതിനു് സിഡാക്‍ 
സമര്‍പ്പിച്ച ഡ്രാഫ്റ്റിലെ പോരായ്മകള്‍ ചൂണ്ടിക്കാട്ടി എസ്എംസിക്കുവേണ്ടി തയ്യാറാക്കിയ ക്രിട്ടിക്‍ ഡോക്യുമെന്റ് 
എഴുതിയവരിലൊരാള്‍ ജിനേഷ് ആണു്. ഈ ക്രിട്ടിക്കിന്റെ വെളിച്ചത്തില്‍ സിഡാക്‍ മുന്‍കയ്യെടുത്തു് എസ്എംസി അടക്കമുള്ള 
വിവിധസംഘങ്ങളുമായി ഇതുസംബന്ധിച്ചു് തിരുവനന്തപുരത്തു് കഴിഞ്ഞമാസം കണ്‍സല്‍ട്ടേഷന്‍ സംഘടിപ്പിച്ചിരുന്നു.

ഇന്‍സ്ക്രിപ്റ്റ് കീബോര്‍ഡ് ലേഔട്ടു് കൂടുതല്‍ മെച്ചപ്പെടുത്താനുള്ള സിഡാക്കിന്റെ നിര്‍ദ്ദേശത്തില്‍ കടന്നുകൂടിയ ചില പാകപ്പിഴകള്‍ 
ചൂണ്ടിക്കാട്ടാനും ജിനേഷ് ആയിരുന്നു മുന്‍കയ്യെടുത്തതു്. ഈ ഡ്രാഫ്റ്റില്‍ പിന്നീടു് സിഡാക്‍ തിരുത്തല്‍ വരുത്തുകയുണ്ടായി. 
സെപ്തംബറില്‍ അവസാനമായി സംസാരിക്കുമ്പോള്‍ ജിനേഷ് എന്നോടു് ആവശ്യപ്പെട്ടതു്, ഈ പുതുക്കിയ ഡ്രാഫ്റ്റിന്റെ 
പശ്ചാത്തലത്തില്‍ ഉടനെതന്നെ ക്രിട്ടിക്‍ പുതുക്കണമെന്നും തനിക്കു് അതിനി സാധിക്കുമെന്നു കരുതുന്നില്ല എന്നുമാണു്. മരണം 
തൊട്ടുമുമ്പില്‍ കണ്ടുകൊണ്ടാണു് ഇതുപറഞ്ഞതെന്നു് ഇതെഴുതുമ്പോഴും വിശ്വസിക്കാനാവുന്നില്ല.

% താഴെ സൂചിപ്പിയ്ക്കുന്ന കുറിപ്പുകള്‍ കിട്ടാന്‍ വഴിയുണ്ടോ?
യുഐഡിയുടെ ഭാഗമായി നടപ്പാക്കുന്ന ബയോമെട്രിക്‍ ഐഡന്റിഫിക്കേഷനിലെ ചതിക്കുഴികള്‍ ചൂണ്ടിക്കാട്ടുന്ന ഒരു ലേഖനം 
തയ്യാറാക്കുവാന്‍ ഇതിനിടെ ജിനേഷ് തുടങ്ങിയിരുന്നു. സന്തോഷ് തോട്ടിങ്ങലിനെയും എന്നെയും കോ-ഓഥര്‍മാരാക്കി ആരംഭിച്ച 
ആ പരിശ്രമം സൈബര്‍ പെരുമ്പാതയില്‍ പാതിവഴിയില്‍ കിടക്കുകയാണു്. ഈ ലേഖനം മലയാളത്തിലാണെങ്കില്‍ 
Economic and Political Weeklyയ്ക്കു വേണ്ടി അനിവറുമായി ചേര്‍ന്നു് ഇതേവിഷയത്തില്‍ മറ്റൊരു ലേഖനവും 
തയ്യാറാക്കുന്നുണ്ടായിരുന്നു. അതും അപൂര്‍ണ്ണതയില്‍ വിട്ടാണു് ജിനേഷ് വിടവാങ്ങിയതു്.

% ചിത്രം ചേര്‍ക്കണം
% commits to pypdflib

silpaയിലേക്കും pypdflib ലേക്കും രോഗത്തിന്റെ മൂര്‍ദ്ധന്യത്തിലും ജിനേഷ് നടത്തിയ കമ്മിറ്റുകള്‍ ശ്രദ്ധേയങ്ങളാണു്. യൂണിക്കോഡ് 
മലയാളത്തിലുള്ള ലേഖനങ്ങള്‍ വെബ്ബില്‍നിന്നു് നേരിട്ടു പിഡിഎഫ് ആക്കി മാറ്റാനുതകുന്ന library ആണിതു്.

ജിനേഷ് ചെയ്തതൊന്നും വെറുതെയാകില്ല എന്നുറപ്പിക്കാന്‍ കഴിയുന്ന സുഹൃദ്‌വലയം സ്വയം ഇന്‍ട്രോവെര്‍ട്ടു് എന്നുകുരുതുന്ന 
ഈ ചെറുപ്പക്കാരനുണ്ടായിരുന്നു. ഒരു യഥാര്‍ത്ഥ ഹാക്കര്‍ക്കു് നല്‍കാവുന്ന ഏറ്റവും വലിയ ബഹുമാനം അവര്‍ നേതൃത്വം 
നല്‍കിയ പ്രോജക്ടുകളെ വിജയത്തിലേക്കു നയിക്കുകയാണു്. ആ തരത്തിലുള്ള ആലോചനകള്‍ ആഷിക്‍ സലാഹുദ്ദീന്റെയും 
വാസുദേവിന്റെയും അനിവറിന്റെയും മറ്റും മുന്‍കയ്യില്‍ തുടങ്ങിക്കഴിഞ്ഞു.

പൊടുന്നനെ ഞാന്‍ അവസാനിപ്പിക്കുകയാണു്. ഇതിനപ്പുറമെഴുതാന്‍ എനിക്കുകഴിയില്ല. ഇതെഴുതുമ്പോള്‍ വീണ്ടും വാവിട്ടുള്ള 
കരച്ചിലിലേക്കു് ഞാന്‍ വഴുതിവീഴുകയാണു്. നീ മരിച്ചുവെന്നു വിശ്വസിക്കാന്‍ എനിക്കിപ്പോഴും കഴിയുന്നില്ല. ജിനേഷ്, നീ ജീവിക്കുന്നു, 
ഞങ്ങളിലൂടെ.

\begin{flushright}(30  September 2011)\footnote{http://malayal.am/പലവക/മുഖം/12927/ജീവിക്കുവാനുള്ള-കാരണങ്ങള്‍}\\സെബിന്‍ ഏബ്രഹാം ജേക്കബ്\\ (മലയാളം ഇന്റര്‍നെറ്റ് വാര്‍ത്താ പോര്‍ട്ടലിന്റെ എഡിറ്റര്‍)\end{flushright}

\newpage
