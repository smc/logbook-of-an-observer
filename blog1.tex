\secstar{Hospital Log 1}
\vskip 2pt

On Monday, the 4\textsuperscript{th} of October, I got diagnosed with a case of Acute Lymphoblastic Luckhemia 
(I didn't know it till Thursday). After 12 days, I find myself in one of the best places in
 India for haematology related treatments and am sitting in a room worth Rupees 2k a day
 with AC/TV/attached, bath/nurses on call and to top it, complete room service.

The words and worries of others who love us and still not having the slightest idea of
who I am is always an interesting problem. To start with, I am writing down some of
my thoughts. Don't know whether to publish or not, but it is not my immediate concern,
I have a lot of unpublished works in my documents folder :)

\subsection*{The question of being moody} 

Everyone tells us not to feel depressed or sad or moody. Treatment is difficult, it 
will turn your physiology upside down for sometime. It does cost a lot and some 
estimates put it at as much as 20 lakhs. But actually in mind, I am relieved. Why? May 
be it is because of the mentality of the scientist inside me who doesn't like unknown variables. I loved
it when I was diagnosed with Lukhemia. I at least have a diagnosis. I should say I 
felt like Dr.Gregory House in House M.D. To be frank, I didn't even stop to think whether
it is life threatening or not, I was so happy that I had a diagnosable issue which was 
creating all that stupid or weird pains on my thighs, legs and my general loss of appetite.

Still, I don't have anything to get moody about here. From both the nurses' and the doctors' reactions, I guess
I am one of the happiest patients they have. I think it will be difficult for them to 
imagine someone who is confined to a room 24 hrs a day (I can walk through corridors of
8\textsuperscript{th} floor) being happy. But the fact is, it is not new to me. I have all the experience in 
the world in doing this. Being lethargic and making an organized lethargic behavior of
solitary life inside a room. So, I am kind of not moody at all. There is nothing new to it,
I hear all the news, I do call people, I am online, my father is around for any help (he might
be getting bored). If not, nurses will be there. But most people don't understand this. The
idea of socializing without really being in society is a difficult thing for many to think, 
grasp and understand. They believe in showing support. However, the problem is how to show it. My 
friends back in Hyderabad understand me. They know what I want, will get the things done for
me, so that system will run smoothly. 

\subsection*{"Don't you get bored?" or "You are bored eh?"} 

I think it is a question anyone who is undergoing treatment for diseases like this will encounter. 
Especially when you are having a good social life. For people who know me, I don't think it
will be a surprise when I say that I don't feel much bored, when confined to a room. I have done it before.
Just for the sake of being lazy. My social status is that of a guy who fits in everywhere. So 
everyone will miss me as well as not miss me at the same time. Still, interestingly, I do have an active
social life, with calls to parties in weekends, birthday celebrations or whatever. Why? Someone
with a ponytail walking around half dressed or 3/4 trousers and t-shirts and ready to dress up and 
go for night outs is an urban youth. Then, in parties, it all comes down to money, and for 
someone who spends all his stipend and doesn't save anything, its not difficult to get invites
:) Still I don't party much because I don't like dancing, and my idea of party is usually a good
food in a restaurant or a nice bar with lots of alcohol (I do drink). Still I do get invites
for just the opposite ends of discussions too. Still the question people might ask is, ``don't
your lab mates or friends miss you when you are away for, like 6 months or more?'' :) May be. I don't
know. Still there is not much to miss about me since my presence there is still felt. I do
talk to some of them, I am available on phone on the same number, and also I am online once in a while
through GPRS :) May be they will miss my physical presence of 5 foot 6 inches and pony tail. But
that is not much to miss. When you think about it, some of my best friends are still people who I
haven't met yet :) 

\subsection*{``Question of concern!''} 

"Why aren't you concerned?". Like aren't I concerned about losing 6 months, an entire year in MS because of the half 
semester split for diagnosis, change in life (may be entire life), etc. Still no idea whether my
ALL is L2 or L3 and all these are a matter of worry. The thing people don't understand is, I 
don't live life in normalcy. If I had, I might not be doing an MS in IIT Hyderabad. I do 
calculate, when the numbers do not fall in places that I want them to, I will just check where they are 
falling and try to adjust my parameters accordingly. It might be difficult to understand. But
that is what I do. There are no unexpected turn of events in life. Everything is normal because
they are not.

Systems are made to put people in a mechanized way of functioning. When you are used to a system,
it will be very difficult or depressing to get out of it and to be a part of another. But for people
who understands that it is the system which controls them, hopping from one system to another is very
easy. Especially when the change from a lonely hostel room in IIT, Hyderabad to a room in CMC 
Vellore with father is not that path breaking. 

I might not do the same things here. There I watch movies most the time, follow all the US serials
I like, do some research and studies in between, go out and have biriyani once in a while or order
in a pizza if I feel to get some money spent. Here, I have a different routine, different style, like 
when I go home. I hop from that to this easily because I don't do the same thing at both places.
I might read a research paper or two in my house, but I won't be the same 24/7 online guy at home.
That goes for the hospital too. I will just fix the numbers to keep the balance intact. The aim 
is survival and to survive, sometimes we have to reinvent what we understand. Now I have enough
time to reinvent the system I was a part of for the last 3 years. I can go in as a new person 
(hopefully). The system is not changing and I am not changing too. But a break from the system gives you time 
to understand where you were losing your ground.

Interestingly enough, that does give a positive approach to the whole thing. I have nothing to be 
concerned about, except may be financials, Which I am quite sure is a waste of time. I can have 
concerns on how bad my lukhemia is, but the again that is another waste of time once you decide
to go for treatment. Concern on how I would complete my MS will be my last since, if treatment
works, I will get enough time to go for finishing the studies. If my ALL is life threatening
and I am not responding well, there is no point in thinking about that.

My life in Hyderabad will change. May be dramatically because, when you are taking medicine 
everyday, you will follow a routine and once you start doing that, you will never be the free
spirit people used to see. But it is just a change from a lethargic MS student to a normal 
MS student, which again is positive. Yeah, I will definitely have difficulty with some details 
on life. It may be that my social life will change, but I don't think it will bother me as much 
as it bothers everyone else.

\subsection*{Question of "getting impatient about the details"} 

Everyone wants to know what kind of lukhemia it is that I have. What I know is that it is just Acute 
Lymphoblastic Lukhemia. Then they want to know whether it is life threatening or not. I do believe 
it is. Then next question is about the stage and I don't know which stage it is in. I don't really know 
how lukhemias are staged. I guess they are not staged as per time, but according to the types. 
Some of them make genetic mutation and there is ALL with genetic mutation, which is somewhat 
dangerous. I do have one with a genetic mutation. But I am not an expert on what it is and 
how it is. People think diagnosing lukhemia, its types, then understanding details is like 
diagnosing a normal blunt wound or common cold. It is not. There is a larger scientific process 
involved and there is no point in getting impatient. But interestingly, only me, who is not 
supposed to understand it, understands this detail.

I am a researcher. My job is to conduct experiments, get results, analyze them and crunch them 
to make them reveal what is there hiding inside. So, amazingly, when doctors say that they need to do 
more tests, I do understand. I always have to run new experiments to find out about the working system 
and for me, breaking a system didn't even matter. However for doctors who work in these kind of 
situations, its not like that. They can't screw up whereas I was allowed to. More importantly, having 
a result and analyzing it is not the same. There is a huge difference between the two. Results are 
usually just blunt numbers which are weighed against the vital or normal parameters. They 
don't reveal anything. No X-Ray, MRI, CT reveals anything unless you know how to read it. It 
will be the same as me giving you numbers of my SVM-feature study experiments. To draw conclusions, 
there is a need for prior understanding. Total control and know how of what is normal and more 
importantly, eyes to see the anomalies. 

When you have a genetic study done, I do believe it shows a million possibilities and the proper 
understanding of my life and ways are important for doctors (including how I react to different 
medicines). Like how I reacted well to the first dose of chemo. I might not do that for the next. They didn't
give another dose of the medicine because some detail regarding my body was not right. Then a report came in and 
they decided to drop the medicines all together and start a new one. It happens. I have done it on 
my systems. There is no strict code of conduct when you work on situations like this. It changes 
as you see results. The entire idea of protocol for 2 years will be the same. However, in the case of medicines, it might not be so. Only for 
generic cases does it remain the same. 

I don't really know, but I think genetic mutations don't happen as per lamarkian principles but on darwin's 
theories (there was a recent news on lamarkian effects of genes too, haven't read that in detail). 
So, if I had this problem, I guess it was there for the last 24 years (not sure, will ask my doctor 
to confirm). It is not a disease outside my body but a change in my system from the normal being. My 
body is built to have this mutation and it will be difficult to make it believe it is not. It 
feels like a tedious task and not an easy one, so no point in getting impatient and I don't believe in 
a complete cure in this case (if it is like this). I am not sure whether someone else will 
understand this the same way. If they do, great, and if not, I can't say anything. There are 
possibilities, chances, drugs, treatments, but none of that  means it is final and curable. The 
word cure might not even be appropriate in case of cancer since we are not sure whether we are curing 
or breaking :)

The way to obtaining results is always through continuous aggression of experiments and thoughts. Keeping 
up to date is always a big part of it. Where I was lagging for sometime, now I am keeping upto date. 
As my professor pointed out a week before I left for treatment, I needed to increase my reading. 
These six or so months are a good chance for that. 

The problem is, people don't like ambiguities and half answers when it concerns life and death. 
But after being a part of the world of probabilities, pattern recognition and machine learning, the thing 
I understand is, you can never avoid it. Starting of a treatment doesn't guarantee full diagnosis. 
It just means that the beginning of the treatment is the same. It is easy for me to understand this since, 
datasets don't change for experiments. For entirely different experiments, we can use the same datasets, 
may be with different annotations or labels. There will be ambiguities and half answers, 
like "it is ALL and no idea which type". People don't understand that diagnosing from lukhemia to 
the farthest details of its reach is a big process. There is no direct answer for patients' queries 
like "How bad am I?". It depends. Something can be predicted but not everything. The treatment is not 
straight forward and it is a little scary, but what to do, our whole life is like that. At least do it 
when we can afford it. 

\subsection*{Question of Life} 

Then usually comes the question of life. How it will change your future. Usually I will give a 
blunt answer saying that I don't care about it. My future is my concern and what I am more concerned now 
is of being present. Still my outlook on future is always loose, I do want to make money to live (now 
I have to since, the treatment if it goes on like this might put a visible liability of some lakhs 
on me). I always keep my options open. I was on the verge of starting to work for an NGO in "No UID" campaign 
since I want to mark my presence in the scene. May be now I can put a little more time into that. Can 
read up some stuff and get the details, use my expertise and contacts in pattern recognition academia in India
(believe it or not, I have friends in some good institutes :) ). May be I can write more regularly 
for malayal.am. I missed two races while I was not well. I could try to report the rest and publish. 
Get a name of being the first exclusive malayalee formula 1 reviewer. There is always options and 
all these are the ones I pursue right now. 

Things can change. New ideas will come up. The choice or the idea of changing life according to 
what comes up next usually doesn't bring much joy to anyone. Everyone wants to keep status quo. 
But the world doesn't do that. Systems are just adapting to emulate status quo. So, from what I see, 
there is not much chaos in breaking status quos. It is bound to happen. ``Life changing experience'' is 
a phrase everyone usually uses. I think it is just a myth. Life always changes, you don't live the same 
one for long. You can't. Small things might be the same (like your favourite breakfast or time to wake up) 
but you don't get a sudden outlook on life on a single day. It happens over a period of time. How you
realize it doesn't represent how it was changed. It is just similar to what is called, ``you don't see 
the whole picture''. You can't realize something in a single moment, you will apply all your apriori 
before doing that. Interestingly, that does play the greater part in understanding, than anything 
else. But people like to believe that they get realizations on the moment :) May be that is because it gives us a kind 
of relief that life is taking sudden turns. Life doesn't do that, it just appears like a sudden turn. 
Nothing is sudden, it is just that our views are skewed.

\newpage 
