\secstar{Hospital Log 1}
\vskip 2pt

On Monday 4th october I got diagnosed with a case of Acute Lymphoblastic Luckhemia 
(I didn't know it till thursday). after 12 days, I am in one of the best places in
 india for haematology related treatments and is sitting in a room worth rs 2k a day
 with AC/TV/attached bath/nurses on call and to top it, complete room service.

The words and worries of others who love us and still don't have the slightest idea
 who I am is always an interesting problem. To start with, I am writing down some of
 my thoughts. Don't know whether to publish or not, but its not my immediate concern,
 I have lot of unpublished works in my documents folder :)

\subsection*{The question of being moody} 

Everyone is saying don't feel depressed or sad or moody. Treatment is difficult, it 
will turn your physiology upside down for sometime. It does costs a lot and some 
estimates put it at as much as 20lakh. But, actually in mind i am relieved. Why? May 
be its the mentality of scientist inside me who doesn't like unknown variables. I loved
 it when I was diagnosed with Lukhemia. I at least have a diagnosis. I should say I 
felt like Dr.Gregory House in House M.D. To be frank, I didn't even stop to think whether
 it is life threatening or not, I was so happy that I had a diagnosable issue which was 
creating all that stupid or weird pains on my thighs,legs and my general loss of appettite.

Still I don't have anything to get moody here. From nurses and doctors reactions, I guess
I am one of the happiest patients they have. I think it will be difficult for them to 
imagine someone who is confined to a room 24 hrs a day (I can walk through corridors of
 8th floor) be happy. But the fact is, its not new to me. I have all the experience in 
the world in doing this. Being lethargic and making an organized lethargic behaviour of
solitary life inside a room. So, I am kind of not moody at all. There is nothing new in it,
I hear all the news I do call people, I am online, my father is around for any help (he might
be getting bored). If not nurses will be there. But most people don't understand this. The
idea of socialising without really being in society is a difficult thing for many to think, 
gasp and understand. They believe in showing support but the problem is how to show it. My 
friends back in Hyderabad understand me. They know what I want, will get the things done for
me, so that system will run smoothly. 

\subsection*{``Don't you get bored?'' or ``You are bored eh?''} 

I think it is a question anyone who is in treatment of diseases like this will encounter. 
Especially when you are having a good social life. For people who know me, I don't think it
will be a surprise that I don't feel much bored, when confined to a room. I have done it before.
Just for the sake of being lazy. My social status is that of a guy who fits in everywhere. So 
everyone will miss me and don't miss me at the same time. Still interestingly I do have an active
social life, with calls to parties in weekends/birthday celebrations/or whatever. Why? Someone
with a ponytail walking around in half or 3/4 trousers and t-shirts and ready to dress up and 
go for night outs is an urban youth. Then, in parties, it all comes down to money, and for 
someone who spends all his stipent and doesn't save anything, its not difficult to get invites
:) Still I don't party much because, I don't like dancing and my idea of party is usually a good
food in a restaurant or a nice bar with lots of alchohol (I do drink). Still I do get invites
for the just opposite ends of discussions too. Still the question people might ask is, don't
your labmates or friends miss you when you are away for like 6 months or more :) May be, I don't
know. Still there is not much to miss about me there since my presence there is still felt. I do
talk to some of them, I am available on phone on the same number, also I am online once in a while
through GPRS :) May be they will miss my physical presence of 5 foot 6 inches and pony tail. But
thats not much to miss. When you think about it, some of my best friends are still people who I
haven't met yet :) 

\subsection*{``Question of concern!''} 

"Why aren't you concerned?". Like losing 6 months, an entire year in MS because of the half 
semester split for diagnosis, change in life(may be entire life), etc. Still no idea whether my
ALL is L2 or L3 and all these are a matter of worry. The thing people don't understand is, I 
don't live life in normalcy. If I had, I might not be doing an MS in IIIT Hyderabad. I do 
calculate, when numbers doesn't fall in places I want them, I will just check where they are 
falling and try to adjust my parameters accordingly. It might be difficult to understand. But
thats what I do. There are no unexpected turns of events in life. Everything is normal because
they are not.

Systems are made to put people in a mechanized way of functioning. When you are used to a system,
it will be very difficult or depressing to get out of it and be part of another. But for people
who understands that its the system which controls them, hop from one system to another is very
easy. Especially when the change from a lonely hostel room in IIIT,Hyderabad to a room in CMC 
Vellore with father is not that path breaking. 

I might not do the same things here. There I watch movies most the time, follow all the US serials
I like, do some research and studies in between, go out and have biriyani once in a while or order
in a pizza if I feel to get spent money. Here, I have different routine, different style, like 
when I go home. I hop from that to this easily because I don't do the same thing at both places.
I might read a research paper or two in my house, but I wont be the same 24/7 online guy at home.
Thats the same in hospital too. I will just fix the numbers to keep the balance intact. The aim 
is survival and to survive sometimes we have to reinvent what we understand. Now I have enough
time to reinvent the system I was a part of for last 3 years, I can go in as a new person 
(hopefully). System is not changing I am not changing but a break from system gives you time 
to understand where were you losing ground.

Interestingly enough, that do give a positive approach the whole thing. I have nothing to get 
concerned about, except may be financials. Which I am quite sure is a waste of time. I can have 
concerns on how bad my lukhemia is, but again thats another waste of time once you decide
to go for treatment. Concern on how will I complete my MS will be my last since, if treatment
works I will get enough time to go for finishing the studies. If my ALL is life threatening
and I am not responding well, there is no point in thinking about that.

My life in Hyderabad will change. May be dramatically since, when you are taking medicine 
everyday you will follow a routine and once you start doing that, you will never be the free
spirit people used to see. But its just a change from a lethargic MS student to normal 
MS student. Which again is positive. Yeah I will defnitely have difficulty with some details 
on life. May be my social life will change, but I don't think it will bother me as much 
as it bothers everyone else.

\subsection*{Question of "getting impatient about the details"} 

Everyone wants to know what kind of lukhemia it is. What I know is its just Acute 
Lymphoblastic Lukhemia. Then they want to know whether it is life threatening. I do believe 
it is. Then next question is the stage and I don't know which stage it is. I don't really know 
how lukhemias are staged. I guess they are not staged as per time, but according to the types. 
Some of them make genetic mutation and there is ALL with genetic mutation, which is somewhat 
dangerous. I do have one with a genetic mutation. But I am not the expert on what it is and 
how it is. People think diagnosing lukhemia, its types, then understanding details is like 
diagnosing a normal blunt wound or common cold. It is not, there is a larger scientific process 
involved and there is no point in getting impatient. But interestingly, only me who is not 
supposed to understand it, understands this detail.

I am a researcher. My job is to conduct experiments get results, analyze them and crunch them 
to make them reveal whats there hiding inside. So, amazingly, when doctors say they need to do 
more tests, I do understand. I always have to run new experiments to find out the working system 
and for me, breaking a system didn't even matter but for doctors who work in these kind of 
situations its not like that. They can't screw up, I was allowed to. More importantly, having 
a result and analyzing it is not the same. There is a huge difference between two. Results are 
usually just blunt numbers which are weighed against the vital or normal parameters. They 
don't reveal anything. No X-Ray, MRI, CT reveals anything unless you know how to read it. It 
will be same as me giving you numbers of my SVM-feature study experiments. To draw conclusions, 
there is a need for prior understanding. Total control and know how of what is normal and 
importantly eyes to see the anomalies. 

When you have a genetic study done, I do believe it shows a million possibilities and proper 
understanding of my life and ways are important for doctors (including how I react to different 
medicines). Like to the first dose of chemo, I did good. I might not do that for next. They didn't
give another medicine because some detail of my body was not right. Then a report came and 
they decided to drop medicine all together and start a new one. It happens. I have done it on 
my systems, there is no strict code of conduct when you work on situations like this, it changes 
as you see results. The total idea of protocol for 2 years is same but medicines might not. Only 
generic cases it remains same. 

I don't know, but think genetic mutations don't happen as per lamarkian principles but on darwin's 
theories (there was a recent news on lamarkian effects of genes too, haven't read in detail). 
So, if I had this problem, I guess it was there for last 24 years (not sure will ask my doctor 
to confirm). It is not a disease outside my body, its a change in my system from normal one. My 
body is built to have this mutation and it will be difficult to make it believe it is not. It 
feels a tedious task and not an easy one, so no point in getting impatient and I don't believe in 
a complete cure in this case (if it is like this). I am not sure whether someone else will 
understand this the sameway. If they do, great, if not I can't say anything. There are 
possibilities, chances, drugs, treatments, but that doesn't mean its final and curable. The 
word cure might not even be right in case of cancer since we are not sure whether we are curing 
or breaking :)

The way to results is always through continuous aggression of experiments and thoughts. Keeping 
up to date is always a big part of it. Where I was lagging for sometime is keeping upto date. 
As my professor pointed out a week before I left for treatment, I need to increase my reading. 
These six or so months are a good chance for that. 

The problem is, people don't like ambiguities and half answers, when it concerns life and death. 
But after being part of world of probablities, pattern recognition and machine learning, the thing 
I understand is, you can never avoid it. Start of a treatment doesn't guarantee full diagnosis. 
Its just means that the beginning of treatment is same. Its easy for me to understand since, 
datasets don't change for experiments. For entirely different experiments, we can use same dataset 
may be we will use different annotations or labels. There will be ambiguities and half answers, 
like "it is ALL and no idea which type". People don't understand that diagnosing from lukhemia to 
the farthest details of its reach is a big process. There is no direct answer for patients queries 
like "How bad am I?". It is depends. Something can be predicted but not all. The treatment is not 
straight forward and its little scary, but what to do, whole life is like that. At least do it 
when we can afford it. 

\subsection*{Question of Life} 

Then usually comes the question of, life. How it will change your future. Usually I will give a 
blunt answer saying I don't care about it. My future is my concern and what I am more concerned now 
is of being present. Still my outlook on future is always loose, I do want to make money to live (now 
I have to since, the treatment if it goes on like this might put a visible liability of some lakhs 
on me). I always keep options, I was on the verge of start working for an NGO in "No UID" campaign 
since I want to mark my presence in the scence. May be now I can put little more time into that. Can 
read some stuff get details, use my expertise and contacts in pattern recognition academia in India
(believe it or not, I have friends in some good institutes :) ). May be I can write more regualarly 
for malayal.am. I missed two races while I was not well, can try to report the rest and publish. 
Get a name of being the first exclusive malayalee formula 1 reviewer. There is always options and 
all these are ones I pursue right now. 

Things can change, new ideas will come up. The choice or the idea or changing life according to 
what comes up next usually doesn't bring much cheers to anyone. Everyone wants to keep status quo. 
But world doesn't do that, systems are just adapting to emulate status quo. So, from what I see, 
there is not much chaos in breaking status quos. It is bound to happen. Life changing experience is 
a word everyone usually uses. I just think its a myth. Life always changes, you don't live the same 
one for long. You can't. Small things might be same (like your favourite breakfast or time to wake up) 
but you don't get a sudden outlook on life on a single day. It happens over a period of time. How you
realize it doesn't represent how it was changed. Its just similar to what is called, you don't see 
the whole picture. You can't realize something in a single moment, you will apply all your apriori 
before doing that. Interestingly that does play the greater part in understanding than anything 
else. But people like to believe, you got realizations on the moment :) May be, it gives us a kind 
of relief that life is taking sudden turn. Life doesn't do that, it just appears like a sudden turn. 
Nothing is sudden, it is just that our views are skewed.

\newpage 
