\newpage
\begin{english}
\secstar{Jinesh's mails to FEC}

\subsection*{On education   - Oct 15 2009}
%=======================================


hi all,

The discussions in the thread are very interesting. But it just seems highlighting of a set of facts everybody in the entire state knows. If we have to change the system, then either we should go above the system. A lot of things should change in order for the youngsters to opt for a career of his/her choice.

Everyone in the society has peer and parental pressure and its not an easy one to avoid. I was determined to go for an engg. education while all in my family wanted me to go for medicine. I just refused and its not easy for everyone. I dont have to earn penny for my parents, so its easy for me to make choice, but its not the same for all.

As Sudeep pointed out, society plays a greater role in what anyone to become and its not easy for all to fight against what society decides as best for them. From what i see, the good family values of Indian society plays a greater role in the career selection. For a student who decides on a careers, the things at stake is much higher to someone in US for example. 

I study and work in a research lab. All my peers are from the cream of society with absolutely no burden of a family or societal pressure on them. Or i should say, people who belong to so called high society circle. From what i see, they are the only people who are capable of making a choice in their career. Every other strata of the society the one who makes the different choice is crusader.

It is this social problem we have to address. For that a blind implementation of credit system in professional courses(i haven't even understood how it will work in an affiliated college system with centralized prescribed syllabus) will do no good. Career advising might work on a smaller set of aspirants, but the ones who fear the society still go for the so called "low risk" path. For the SC/ST having concessions in marks is still an acceptable solution, but there should be measures to tackle the lack of resources they have at disposal.

RVG was talking about implementing the "year out" rule firmly. But i see it will be difficult in current scenario. While i was doing b tech, my exams for the previous semester usually came after one or two months of next semester class. To get into class i have to pay all my fees and register for the semester. When the results for the semester comes, i usually will be doing the next next semester. Only twice(after first year and final year) exams were conducted on time. Still while joining IIIT, my final year results were not out. For any successful implementation of year out rule, its not student organizations you have to bash, but the ailing university administration(there was a case when university forgot to schedule an exam and students had to go on strike for university to schedule it). University system is designed to ensure even standards and minimum quality of education, but when there is exponential increase in number of students, i am quite skeptical about the change in standard higher education in Kerala.

If the problem in higher education scenario to be solved, we need to see and tackle problems at a variety of levels and So far no one is even talking about those. Just having a weird credit system implemented might make the professional students a little more lazy and relaxed and less serious about their subject.

Jinesh K J

%--------------

\subsection*{On Knowledge society - Oct 15 2009}
%====================
Sajan,

On Thu, Oct 15, 2009 at 4:50 PM, Sajan G \url{<sajangopalan@gmail.com>} wrote:

    OK broadly take it as jobs in the IT sector, whether it creates knowledge or not .....\\
    sajan

I think i may have to disagree here about the perception of knowledge industry. IT is really a Knowledge-based industry than Knowledge Industry(there are people who produces and markets knowledge in IT field also). The main difference of both is, in Knowledge based industry knowledge is a tool, while it is a product in knowledge industry. Knowledge industry is in its infancy in India and the questions and problems raised by knowledge industry is really different. I think the difference between knowledge industry and knowledge based industry is important though it looks very trivial.

The prime factor which thrive a knowledge industry or knowledge based industry is really not real estate(the day real estate dictates the development, then i dont think it has anything to do with knowledge). Just naming a zone hitech,smart, knowledge doesn't make it anything remotely related to that.


\subsection*{On Sachin - Nov 6 2009}
%============

hi all,

This is a common criticism everyone takes against Tendulkar. Yes, he didn't finish most of the games where he scored best. But from what I see, only Lara is a better finisher than Sachin just because, he was the only one who was in a similar situation when you have to build an innings from nowhere, with zero support(just dont compare what happened yesterday with the real master, his career was at peak 11 years before, in 98-99 season just look what was happening in Indian team then). Whatever we are getting from Sachin now is bonus. He can finish games and he had shown us that too, around the world. But the factor is, when it comes to Sachin, we like to find out where he is not perfect and we do remember games where India lost when Sachin took us just until the end out the game. Infact, by statistics, he might only player who figures in the top 10 wasted chasing centuries thrice. What is bad is Indian team, after 20 years of active cricket, India still asks him to shoulder responsibility. I don't think anyone else would have taken such a task and played consistently for  this long. Despite losing 6 wickets and having no more batsmen to come out, India was still in picture when Sachin got out. That chance which he gave was the only real chance in the entire innings except a half c\&b in Hussey's over. Its not Sachin who didn't finish the game, the rest of the team, who was unable to.

Jinesh K J

2009/11/6 Jayasankar \url{<jayasankar.peethambaran@gmail.com>}

    Sachin is definitely not the greatest, or even a great finisher of the game. Steven Waugh was, Dhoni can be.\\

    That, in itself doesn't make the case to not recognize the insane greatness he has exhibited throughout his career. Peter Roebuck once wrote that Indians are surprised to find out how high rest of the world rates Sachin as for them, to Sachin, human laws don't apply. He always has to perform superhuman feats under all conditions, damn it.



%=======================
\subsection*{Engineering Education - declining standard - 12 -08 2009}
%==============================================

hi all,

A change in syllabus might not change the quality of engineering education in a day. We have to change a lot of things associated with the education. Even the university system have some problems. I would say it averages out the student performances. Undergraduate engineering, like any other engg. program should orient student towards the different opportunities and paths he can follow after his UG program. It should have some basic elements as compulsory(i am not quite sure of the civil and mechanical components, may be people who are in robotics can help on how these courses might help). Electronics, electrical and mathematics components are very important and there should never be a question of replacing them than improving them. Students should be made understood how important these courses are.

About the Physics chemistry courses, I do think there is a requirement for serious overhaul of the syllabus and I don't think practicals on these two are not that important for comp. sci. student(actually in my UG syllabus we didn't have practicals for both).

On top of all these, we should have good oriented teachers. Like, we had so many electives in our final two years, but not enough teachers to teach any, so finally ended up by taking ones which our teachers elected to teach than what we elected to study :) Similarly, now if we have a good oriented teacher, he/she could not take much freedom in using his/her understanding of subject in teaching it but have to restrain to teach it according to the standard norms. Like, even if an author decided on revising his book, i dont think university question papers are bothered about it(one base of computer science subject is computer organization and design, the state of questions for the subjects can be read here \url{http://www.jainbasil.net/2009/10/calicut-university-style-of-computer-science-engineering/} ).

I don't think rationalizing syllabus might bring any change in the state of students(if it is done as per wish of corporates, we might help them make some more profits). If we really have to improve the standard of education system, i do think there is a need to address the problems at different levels from how the courses are organized,administered, taught and evaluated.

If our concern is declining standard of engg. education, its not just the syllabus who is culprit(it is one, but from what i see, even the current syllabus can do wonders if it is taught properly). It is numerous other factors like teaching, lack of understanding on subject and course relation, lack of freedom for student and teacher etc.(again freedom without responsibility will just negate the effects).

All what i wrote are just my thoughts and i have seen most of the people talked here pointed many of them. I just like to comprehend and argue that, the perfect solution is to improve all the parameters related to education to the best possible level. Since it is the ideal case, we need think of identifying the critical parameters and try to improve them. From what i see, syllabus might not be the critical factor(it can be wrong too, but anyway not for any reasons stated in the thread).


Regards\\
Jinesh K J
\end{english}

\subsection*{വൈദ്യുതസ്കൂട്ടര്‍  -മെയ് 2010}
%=====================

\begin{english}
hi,

Very good article on EVs. I think Honda was selling Hydrogen fuel
cars(FCX clarity) in LA suburbs.
By the way, recently some people have come up with the waste disposal
problems when the number of EVs go up(i think mainly its about
Lead/Lithium in batteries). I dont know whether these studies were
funded by anybody like shell or total(they are quite famous for this).
TV motor shows like TopGear,fifth gear(both British), dont support the
hybrids/EVs. But in one episode they were talking good about the Honda
Hydrogen powered car(FCX clarity). There are now some manufacturers
even producing electric/hybrid power super cars(Ariel Atom based cars,
Wrightspeed X1 is the most famous i think). When Prius came out and
became popular, there was a joke going around saying it was the local
gangsters who were buying all prius to silently sneak on the enemy :)
So i think we should expect weirdest arguments on the feasibility of
electric motors. But, there is a lot of fuss now a days about large
scale production and management of high capacity electric batteries
and its disposal.

Even F1 is going for low fuel conception, KERS(Kinetic Energy Recovery
System) etc., so the future is anyway will be of alternate fuels, lets
wait and see.

Regards\\
Jinesh K J
\end{english}

\subsection*{മലയാളം നിര്‍ബന്ധമാക്കുന്നതിനെപ്പറ്റി  - ജനുവരി 12 2011 }
%======================================================

ഒന്നുറക്കെച്ചിരിക്കാനാണെനിക്കു പല വാദങ്ങളും വയിച്ചിട്ടു തോന്നുന്നതു്.
ആദ്യ ആറുവര്‍ഷം മലയാളത്തില്‍ പരീക്ഷയെഴുതി ഏഴാം വര്‍ഷം മുതല്‍
ഇംഗ്ലീഷില്‍ പരീക്ഷയെഴുതിത്തുടങ്ങിയവനാണു ഞാന്‍. ആദ്യമൊക്കെക്കുറച്ചു
ബുദ്ധിമുട്ടുമുണ്ടായിരുന്നു(ഞാന്‍ പഠിക്കുന്ന കാലത്തു് ഇംഗ്ലീഷ് നാലാം
ക്ലാസ് മുതലാണു് പഠിപ്പിച്ചു തുടങ്ങിയിരുന്നതു്). എന്നാലും തീരെ
മോശമൊന്നുമായിരുന്നില്ല, ഒരുകൊല്ലം കൊണ്ടു് കുറെയൊക്കെ ശരിയാവുകയും
ചെയ്തു. ഇന്നിപ്പോ എനിക്കു മലയാളം പേനകൊണ്ടെഴുതാനാണു ബുദ്ധിമുട്ടു്.

പിന്നെ പലരും കുട്ടികളുടെ തിരഞ്ഞെടുക്കാനുള്ള അവകാശത്തെപ്പറ്റിയൊക്കെ
വാചാലരാവുന്നതുകണ്ടു. കുട്ടികള്‍ തിരഞ്ഞെടുക്കുമെന്നു മാത്രം പറയരുതു്,
ഭൂരിപക്ഷത്തിന്റെ കാര്യത്തിലും അച്ഛനമ്മമാരും അവരെ ഉപദേശം കൊണ്ടുമൂടുന്ന
ബന്ധുക്കളുമാണു് തീരുമാനമെടുക്കുന്നതു്. ഏതു സ്കൂളില്‍ പഠിക്കണമെന്നും,
ഏതു വിഷയം പഠിക്കണമെന്നും. അവരോടു് വിദ്യാഭ്യാസം ജോലിലക്ഷ്യം
വച്ചുള്ളതല്ല, സംസ്കാരവും സാമൂഹ്യബോധവുമുള്ള സമൂഹത്തെ
വളര്‍ത്താനുള്ളതാണെന്നു RVGയല്ല, ദേവേന്ദ്രന്‍ നേരിട്ടുവന്നു പറഞ്ഞാലും
അംഗീകരിക്കുമെന്നു തോന്നുന്നില്ല.

ബഹുഭാഷാ സമൂഹത്തില്‍ ഇടപെടേണ്ടിവരുമ്പോള്‍ സ്വാഭാവികമായും നമുക്ക്
പലപ്പോഴും ഇംഗ്ലീഷ് ഉപയോഗിക്കേണ്ടിവരും. പക്ഷേ അതിനു
ഓക്സ്‌ഫോര്‍ഡ്/കേംബ്രിഡ്ജ് ഇംഗ്ലീഷോ, അമേരിക്കന്‍ ഇംഗ്ലീഷോ ഒന്നും തന്നെ
വേണമെന്നില്ല. അതിനു വേണ്ട ഇംഗ്ലീഷ് സ്കൂളില്‍ പഠിപ്പിക്കുമെന്നും
തോന്നുന്നില്ല. ഇംഗ്ലീഷ് നിര്‍ബന്ധമായും ഒരു വിഷയമായുണ്ടാകുമ്പോള്‍ അതിലെ
നിയമങ്ങളും കാര്യങ്ങളും പരീക്ഷയ്ക്കുവേണ്ടിയാണെങ്കില്‍പ്പോലും
പഠിക്കുകയും ചെയ്യുന്ന അറിവുമതി ജീവിച്ചുപോകാന്‍. പക്ഷെ പഠിക്കുന്ന
കാലത്തതു മര്യാദയ്ക്കു പഠിയ്ക്കണം, അതിനു പരീക്ഷയെഴുതാനുള്ള മീഡിയം
ഇംഗ്ലീഷാക്കിയതുകൊണ്ടു പ്രത്യേകിച്ചുകാര്യമൊന്നുമില്ല.

വീട്ടില്‍ അച്ഛനും അമ്മയും സംസാരിയ്ക്കുന്ന ഭാഷയില്‍ അദ്ധ്യയനം
നടക്കണമെന്നു പറയുന്നതു് ഒരു ചെറിയ കാര്യം മാത്രം ലക്ഷ്യം
വച്ചുകൊണ്ടാണു്. അല്ലെങ്കില്‍ത്തന്നെ മനസ്സിലാവാത്ത സിദ്ധാന്തങ്ങളും
കോണ്‍സെപ്റ്റുകളും ഭാഷാന്തരണത്തില്‍ക്കൂടിത്തട്ടി നഷ്ടപ്പെടരുതെന്നു
കരുതി. അല്ലെങ്കില്‍പ്പിന്നെ, ക്ലാസ് ടീച്ചര്‍മുതല്‍ ട്യൂഷന്‍ ടീച്ചറും
അമ്മയും അടക്കം വിദ്യാഭ്യാസത്തില്‍ ഉള്‍പ്പെടുന്ന സര്‍വ്വരും അതു ഒരേ
അദ്ധ്യയനമാദ്ധ്യമത്തിലൂടെ പകര്‍ന്നുകൊടുക്കുകയും, കുട്ടിയ്ക്കതു
മനസ്സിലാവുകയും വേണം. കുട്ടിയ്ക്കു വേഗം മനസ്സിലാവണമെങ്കില്‍ അതവന്‍
ചിന്തിയ്ക്കുന്ന ഭാഷയിലാവണം. എന്റേതൊക്കെ ഇപ്പോളൊരു അവയില്‍ പരുവമാണു്.
ദേസി ഇംഗ്ലീഷില്‍ പറഞ്ഞാല്‍ മലയാളത്തില്‍ പറയുന്നതിലും കൂടുതല്‍ എളുപ്പം
മനസ്സിലാവുമെന്നു തോന്നുന്നു.

എന്നാല്‍ മലയാളം ഞാന്‍ പത്തുവരെ പഠിച്ചിരുന്നില്ലങ്കില്‍
നഷ്ടപ്പെടുമായിരുന്നതു് വളരെ വലുതാണു്. എന്റേതെന്നെനിക്കുറപ്പിച്ചു
പറയാന്‍ കഴിയുന്ന കേരളസമൂഹത്തില്‍ ഇടപെടാന്‍ എനിക്കതു
ബുദ്ധിമുട്ടുകളുണ്ടാക്കുമായിരുന്നു. ഇപ്പൊ ഇവിടെ പലരും പറയുന്ന
കാര്യങ്ങള്‍ മനസ്സിലാക്കാന്‍ പോലും ആവില്ലായിരുന്നു. വിദ്യാഭ്യാസം കൊണ്ടു
സമൂഹത്തിന്റെ പുരോഗമനം ലക്ഷ്യം വയ്ക്കുന്നുണ്ടെങ്കില്‍, മാതൃഭാഷ
പഠിക്കേണ്ടതു് അത്യാവശ്യം തന്നെയാണു്. താന്‍ ഭാവിയില്‍ സമൂഹത്തില്‍
ഇടപെടുമോ എന്നൊന്നും പ്രൈമറി-സെക്കണ്ടറി വിദ്യാഭ്യാസകാലത്തു കുട്ടികള്‍
ആലോചിക്കാത്തതു കൊണ്ടും, അവര്‍ക്കു കൊടുക്കുന്ന ചോയ്സ്
ഉപയോഗപ്പെടുത്തുന്ന മാതാപിതാക്കളും സമൂഹവും ഇത്തരമൊരു വീക്ഷണം പൊതുവേ
അംഗീകരിക്കാത്തതു കൊണ്ടും നിര്‍ബന്ധിതമാക്കുന്നതില്‍ എതിര്‍പ്പുണ്ടാവേണ്ട
കാര്യമെന്താണെന്നാണെനിക്കു മനസ്സിലാവാത്തതു്.

പിന്നെ ഭാഷാന്യൂനപക്ഷങ്ങള്‍ക്കു മലയാളം നിര്‍ബന്ധമാക്കുന്നുണ്ടെങ്കില്‍
കുറച്ചു കടന്ന കൈയ്യായിപ്പോയി :)

ജിനേഷ്
%---------

\subsection*{സോഷ്യല്‍ നെറ്റ്‌‌വര്‍ക്കുകളെപ്പറ്റി}
%===========================================
\begin{english}

facebook is not a democracy. They deserve the right to remove any
content in their database. In the Eben Moglen speech he highlights it.
A privately owned database becoming the main backbone of high profile
political revolution and thus the weekest link in communication.

Dont expect too much from social networks :) they will change their
policies in a blink of an eye. Though we help making a lot of their
money, we dont really own anything there. The same i think will hold
quite true for google groups too. This is one of the main reasons why
many activists groups move to their own space.

\end{english}

%-----
\subsection*{ചാവക്കാട്  ലോവേസ്റ്റ് ജീന്‍സ് ധരിച്ചവരെ പോലീസ് പിടികൂടിയതിനെപ്പറ്റിയുള്ള ചര്‍ച്ചയില്‍  2011 ആഗസ്റ്റ് 11-14 വരെ }
%------
\begin{english}

Oh my god. Good to know this. I should be careful with my dressing.
I have been wearing low waist jeans, 3/4ths,half trousers and dhothi
for many years. Though, happen to get some moral policing around MES
Engg. college for 3/4ths and half trousers its first time i am hearing
low waist jeans is also a problem. I guess my pretty good and
expensive collection of short shirts and t shirts might also create
trouble then. Anyway, i dont understand what my parents have to do
with this, since i buy these with my own money and they cant really
make me wear what they want(at max. what my mother does is suggest
alternatives).

I do have a professor and scores of students in IIIT who wear basket
ball jersey's and half trousers to class. I have seen many wearing
jeans too(low waist or otherwise). May be its only a problem in
kerala, and i do still hope wearing "mundu" (where the opportunities
of exposure is plenty and even a mild wind can do wonders) is still
not offensive :)

---

I dont know whether yours was an explicit reply to me or not. Since I
am someone who wears low waist and short shirts, i do like to reply :)

I know when i offend someone and when they tell me that what i wear
offends me. Its a matter of personal preferences. But, when a there is
a social ban on some dresses i dont really like it. whether its a
bikini or my short shirts and low waist jeans. I also understands when
someone says this dress wont suit you(i dont wear skinny jeans since i
got the same comment from different parts of the world).

I dont really understand it when shadow police is spending their time
on finding out whether my jeans is low waist or not and whether my
shirt is big enough to cover for the jeans. On top I dont understand
what my parents have to do with it.

---

I quite have a feeling now that I am the only one in FEC who wears low
waist jeans and short shirts :) .
That will surely happen if police is going to arrest anyone who wears
a low waist jeans. I dont know to what extent it goes, but if i travel
by car and incidentally had to step out to show my papers to police
and they decide to arrest me for wearing low waist jeans? Or is it for
only those who fool around in buses and bus stops? There is no law and
hence there is no chance for policing or misuse of law. I dont know
under what section they will charge me, may be offensive dressing to
disturb public(something like that is defined in IPC right?). If low
waist jeans is offensive so is skinny jeans. What about the pretty
good and easy to wear jeans plus short kurta combination? If police
start taking cases against them, i would say they need to do so
against mundu too. none of these dresses might not do as much exposure
as a mundu does. Its just ridiculous for the police to dictate terms
of what to wear and what not. If the girls who have problem complains
to me, I can understand. I do understand when they arrest someone who
behaves impolitely to such requests. But is the way is a blanket ban
on low waist jeans and short shirts?

I just don't believe so thats all.

---

On 6/12/11, Sashikumar kurup \url{<sashikumarkurup@gmail.com>} wrote:
> Jineesh, even a flying lungi or mundu is likely to attract the
> attention of this Shamsudeen.

I dont think he will dare to tell anyone who wears a mundu and lungi
not to wear those because in good wind they will fly off(if he starts
to do that, i dont know whom all he might need to tell).
I accept personal choice anyone to like or dislike any fashion,
whether its exhibitionism or not. But when you happen to live in a
metro city and used to go to parties likes to get entry to clubs
without much questions, there are some requirements(its another kind
of social pressure). The same thing is not usable in most other
places. But for just 10 or 15 days outside my home metro i cant keep
an entirely new wardrobe. So, i usually wear 3/4ths or mundu when I am
around in kerala. If I am traveling both are not quite acceptable to
crowd so i switch to t shirts and jeans(sometimes jeans happen to be
low waist but it can be adjusted with a little discomfort to be not
that troublesome in Kerala crowds). I also wear baggy trousers and
shirts. But majority of my wardrobe is such that its made for my life
in metro. Really if you wear a low waist jeans, it calls for wearing a
good under garments(from good brands with brand name in waist or
designer ones).

I dont think the fashion freaks in chavakad are bothered about all
these things, but for many its a real discomfort to keep an entirely
different wardrobe for a place where you wont even be there for 15
days a year(thankfully, my father's and my shirt sizes are same, so i
can use his wardrobe in case of urgency).

I dont know whether it was the exhibiting dressing which offended the
girls of chavakad or behavior of the exhibition freaks. I do think its
the latter. Because if you follow the right way of dressing i do know
as a matter of fact that the exhibiting kind of fashion can be a lot
of less sore to eye and comforting(I buy and wear jeans from good
brands who retail above 3k and if you tell me i cant wear it in a
crowd, really i dont like it :)).

---

i know how difficult it is to grow and cut my own hair or go to a
unisex beauty parlour in kerala for me. it is double tough for my
female cousins to do so.

though it has nothing to do with what preethi said about
masculanization of a public intellectual discussion forum, i do think
stereotypes have much more bigger root in our society than any of us
like. from those circumstance only many comments here can be seen. its
good to let them know i didn like that.
---

hi all,\\
last time when the issue came up i happened to be only one concerned
about police dictating me what to wear. as harish pointed out it is
part and parcel of metro life. from trendy designer dresses(i had
classmates who wear them to class), to public display of affection.
since a lot of malayalees make their livlihood in metros, their wards
seek to follow trendy lifestyle. if people really know party scenes in
kochi, may be fx will call for sreeram sena also. every kind of crowd
has good, bad and ugly. just because you happened to see the worst
ones doesnt mean everyone is so.
i had pretty long hair until 10 months ago and unisex dressing(short
shirts, ¾ths, sometimes hair bands or flowing hair). i know how many
people speed their bike up break in front and try to show off and get
a glimpse of my face. i can easily laugh it off, but it is not the
same for girls.

i can go on talking about these matters, i usually find no problem in
using trendy dresses as long as they are comfortable for me and am
intelligent enough to gauge the attitude of people around me. still in
my remote village i wear half sleeves and half trousers since its my
place, and people have problem with my dress will let me know for
sure.

i really dont like police putting a blanket ban on these, mainly
because who is going to decide what is indecent or not? its a
subjective matter, not something that can be resolved by law.

i hope fx also understands this. i quite know what happens in
chavakad(in fact its not just a problem in chavakkad, when i was in b
tech there was a joke going around on how difficult it was for some
north kerala people to find a shop with normal dress. that too in
heavily conservative muslim areas.

a lot of factors including money plays a larger role in these and most
of these people make their own pocket money. through legal or illegal
ways. the shamsuddeen way will just push them more into criminal world
just for protect. if you dont know ask around how much havala or
kuzhalpanam people pay their carriers. and who are their first choice.
also ask around who will be the guardians of these kids if they are
underage. most houses in chavakkad or similar gulf areas have minimal
male presence and the chances of these kids being the senior most guy
in house is always a possibility.

there are many dimensions to the problem. i pointed out one in other
thread. above are some others. no offence or problem is ever simple..

regards
\end{english}

