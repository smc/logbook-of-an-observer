\secstar{ബുദ്ധിജീവികളുടെ സ്വത്വപ്രതിസന്ധി}
\vskip 2pt

­ആ­ശ­യ­സം­ഘ­ട്ട­ന­ങ്ങ­ളും സം­വാ­ദ­ങ്ങ­ളും വ്യ­ത്യ­സ്ത­വി­ശ­ക­ല­ന­ങ്ങ­ളും അഭി­പ്രാ­യ­ങ്ങ­ളും കമ്യൂ­ണി­സ്റ്റ് മാര്‍­ക്സി­സ്റ്റ് പാര്‍­ട്ടി­യില്‍ 
ഇന്നും ഇന്ന­ലെ­യും ഉള്ള പ്ര­തി­ഭാ­സ­മ­ല്ല. മാര്‍­ക്സി­സ­മെ­ന്ന ചി­ന്താ­പ­ദ്ധ­തി തന്നെ വൈ­രു­ദ്ധ്യാ­ത്മ­ക­തയില്‍ (dialectics)­ 
അടി­യു­റ­ച്ച­താ­യ­തു കൊ­ണ്ട്, അവ­സാന ഇന്‍­ഫ­റന്‍­സി­ലെ­ത്ത­ണ­മെ­ങ്കില്‍ തീ­സി­സും അതി­നൊ­രു ആന്റി തീ­സി­സും 
അത്യാ­വ­ശ്യ­മാ­ണു­താ­നും. ഇതു രണ്ടി­ലും ഊന്നി­യു­ള്ള ഉള്‍­പാര്‍­ട്ടി വാ­ദ­ങ്ങ­ളി­ലൂ­ടെ­യും സം­വാ­ദ­ങ്ങ­ളി­ലൂ­ടെ­യും പു­രോ­ഗ­മി­ച്ച് 
അവ­സാ­നം പാര്‍­ട്ടി ഏക­ക­ണ്ഠ­മാ­യി ഇന്‍­ഫ­റന്‍­സി­ലെ­ത്തു­ക­യാ­ണു പതി­വ്. പല­പ്പോ­ഴും ഉള്‍­പാര്‍­ട്ടി സം­വാ­ദ­ങ്ങ­ളില്‍ സ്വ­ന്തം 
ഭാ­ഗം പാര്‍­ട്ടി ശരി­ക്കും കണ­ക്കി­ലെ­ടു­ത്തി­ല്ലെ­ന്നോ (ഒ­തു­ക്കി­ക്ക­ള­ഞ്ഞെ­ന്നോ) ഒക്കെ­യു­ള്ള കാ­ര­ണ­ങ്ങ­ളാല്‍ 
പല­കാ­ല­ഘ­ട്ട­ങ്ങ­ളില്‍ പല­രും പാര്‍­ട്ടി വി­ട്ടു­പോ­യി­ട്ടു­മു­ണ്ട്. ചില തീ­സി­സു­ക­ളെ റി­വി­ഷ­നി­സ്റ്റ് അഭി­പ്രാ­യ­ങ്ങ­ളാ­ണെ­ന്നു 
വക­യി­രു­ത്തി പാര്‍­ട്ടി തള്ളി­ക്ക­ള­യു­ക­യും, തീ­സി­സി­ന്റെ അവ­താ­ര­കര്‍ അതം­ഗീ­ക­രി­ക്കാ­ത്ത­തു കൊ­ണ്ട് 
അച്ച­ട­ക്ക­ലം­ഘ­ന­മാ­യി കണ­ക്കാ­ക്കി പാര്‍­ട്ടി­യില്‍ നി­ന്നും പു­റ­ത്താ­ക്കു­ക­യും ഉണ്ടാ­യി­ട്ടു­ണ്ട്.

അ­തു­പോ­ലെ ചില പാര്‍­ട്ടി­ന­യ­ങ്ങ­ളെ കാ­ലോ­ചി­ത­മാ­യി പരി­ഷ്ക­രി­ക്കു­ക­യും ­തെ­റ്റു­തി­രു­ത്തല്‍ എന്നു മാ­ദ്ധ്യ­മ­ങ്ങള്‍ വി­ളി­ക്കു­ന്ന 
പരി­ഷ്ക­ര­ണ­പ്ര­ക്രി­യ­യി­ലൂ­ടെ തി­രു­ത്തു­ക­യും ചെ­യ്തി­ട്ടു­ണ്ട്. പാര്‍­ട്ടി­യില്‍ ആശയ സം­ഘ­ട്ട­ന­മു­ണ്ടാ­യ­ഘ­ട്ട­ങ്ങ­ളില്‍ പല­പ്പോ­ഴും 
അച്ച­ട­ക്ക­ന­ട­പ­ടി­ക­ളു­മു­ണ്ടാ­യി­ട്ടു­ണ്ട് (പാര്‍­ട്ടി വി­രു­ദ്ധ­പ്ര­വര്‍­ത്ത­ന­ത്തി­ന്റെ പേ­രി­ലു­ള്ള അച്ച­ട­ക്ക­ന­ട­പ­ടി­യ­ല്ല, 
ആശ­യ­വ്യ­തി­യാ­ന­ത്തി­ന്റെ പേ­രില്‍). ഇത്ത­ര­ത്തില്‍ ബൌ­ദ്ധിക വ്യ­വ­ഹാ­ര­ങ്ങ­ളും, ജന­കീയ പി­ന്തു­ണ­യും, നയ­ങ്ങ­ളു­ടെ 
നി­ശി­ത­മായ ഉള്‍­പാര്‍­ട്ടി­വി­മര്‍­ശ­ന­വും നട­ത്തു­ന്ന പാര്‍­ട്ടി­യാ­യ­തി­നാ­ലാ­വ­ണം കമ്യൂ­ണി­സ്റ്റ് മാര്‍­ക്സി­സ്റ്റു പാര്‍­ട്ടി­ക്ക് 
സമൂ­ഹ­ത്തി­ന്റെ എല്ലാ­ത­ല­ങ്ങ­ളി­ലും നല്ല പി­ന്തു­ണ­യു­മു­ണ്ട്. എന്നാ­ലും മറ്റേ­തൊ­രു പാര്‍­ല­മെ­ന്റ­റി പാര്‍­ട്ടി­യേ­യും പോ­ലെ 
അധി­കാ­ര­ത്തി­ന്റെ ഇട­നാ­ഴി­ക­ളി­ലേ­ക്കു­ള്ള കണ്ണും നട്ടി­രി­ക്കു­ന്ന­വ­രും പാര്‍­ട്ടി­യി­ലു­ണ്ട്. അവര്‍ ഇത്ത­രം ഉള്‍­പാര്‍­ട്ടി 
പ്ര­ത്യ­യ­ശാ­സ്ത്ര­സം­വാ­ദ­ങ്ങ­ളെ പൊ­തു­ജ­ന­സ­മ­ക്ഷം കൊ­ണ്ടു­പോ­യി സ്വ­ന്തം കാ­ര്യം നട­ത്താ­നാ­ണ് ശ്ര­മി­ക്കാ­റ്.

­കേ­ര­ള­ത്തി­ന്റെ കാ­ര്യ­ത്തില്‍, പാര്‍­ട്ടി അധി­കാ­ര­ത്തി­ലി­രി­ക്കു­ന്ന അവ­സാ­ന­നാ­ളു­ക­ളി­ലോ, അധി­കാ­രം നഷ്ട­പ്പെ­ട്ട 
ഉട­നെ­യോ ആണ് നയ­പ­രി­ഷ്ക­ര­ണ­മെ­ന്ന ആശ­യ­വു­മാ­യി ബു­ദ്ധി­ജീ­വി­ക­ളി­റ­ങ്ങാ­റു­ള്ള­ത്. കേ­ര­ള­ത്തി­ലെ ഇല­ക്ഷ­നി­ലെ 
ജയ­പ­രാ­ജ­യ­ങ്ങള്‍ പല­പ്പോ­ഴും ജന­പി­ന്തു­ണ­യേ­ക്കാള്‍ പാര്‍­ട്ടി­ക­ളു­ടെ അട­വു­ന­യ­ങ്ങ­ളോ­ട് (പാര്‍­ട്ടി­യില്‍ കൂ­ട്ടായ ചര്‍­ച്ച 
നട­ത്താ­തെ, പാര്‍­ല­മെ­ന്റ­റി ആവ­ശ്യ­ങ്ങള്‍­ക്കാ­യി നേ­തൃ­ത്വം കാ­ലാ­കാ­ല­ങ്ങ­ളില്‍ എടു­ക്കു­ന്ന തീ­രു­മാ­ന­ങ്ങള്‍) ജന­ങ്ങ­ളു­ടെ 
പ്ര­തി­ക­ര­ണ­മാ­ണെ­ന്ന­ത് പര­സ്യ­മായ രഹ­സ്യ­മാ­ണെ­ങ്കി­ലും ജന­കീ­യ­പി­ന്തുണ നഷ്ട­പ്പെ­ട്ട­താ­ണ് ഇല­ക്ഷ­നില്‍ 
പരാ­ജ­യ­പ്പെ­ടു­ന്ന­തി­നു കാ­ര­ണ­മെ­ന്ന മുന്‍­വി­ധി­യോ­ടെ­യാ­ണ് നയ­പ­രി­ഷ്ക­ര­ണ­വാ­ദ­ങ്ങള്‍ ഉയ­രാ­റു­ള്ള­ത്.

­പ­ല­പ്പോ­ഴും പ്ര­ത്യ­യ­ശാ­സ്ത്ര­സം­വാ­ദ­ങ്ങ­ളെ പൊ­തു­ജ­ന­സ­മ­ക്ഷം അവ­ത­രി­പ്പി­ക്കു­ന്ന­ത് അപ­ക­ട­മാ­യി­ത്തീ­രാ­റു­ണ്ട്. 
പാര്‍­ട്ടി­വേ­ദി­ക­ളില്‍ ചര്‍­ച്ച­ചെ­യ്യു­മ്പോള്‍ പാര്‍­ട്ടി തത്വ­ശാ­സ്ത്ര­ങ്ങ­ളില്‍ കാ­ലാ­കാ­ല­ങ്ങ­ളില്‍ സ്റ്റ­ഡി­ക്ലാ­സു­ക­ളി­ലൂ­ടെ­യും 
വാ­യ­ന­യി­ലൂ­ടെ­യും നേ­ടിയ മി­നി­മം അറി­വ് പൊ­തു­വേ­ദി­ക­ളില്‍ കാ­ണാ­നാ­വി­ല്ല. അതി­നാല്‍­ത്ത­ന്നെ, ബൌ­ദ്ധി­ക­മാ­യി 
ഉന്ന­ത­നി­ല­വാ­രം പു­ലര്‍­ത്തു­ന്ന­വ­രും, തത്വ­ശാ­സ്ത്ര­പാ­ണ്ഡി­ത്യ­മു­ള്ള­വ­രും പൊ­തു­ചര്‍­ച്ച­ക­ളില്‍ കപ­ട­ബു­ദ്ധി­ജീ­വി­ക­ളാ­യി 
മു­ദ്ര­കു­ത്ത­പ്പെ­ടു­ക­യും, മു­റി­മൂ­ക്കന്‍ എന്നു വി­ളി­ക്കാ­നാ­വു­ന്ന, ഉളു­പ്പി­ല്ലാ­തെ ആരെ­യും ഉദ്ധ­രി­ച്ച് ജന­ങ്ങ­ളെ അമ്പ­ര­പ്പി­ക്കു­ന്ന 
വാ­ഗ്‌­വി­ലാ­സ­ക്കാര്‍ പാര്‍­ട്ടി­സൈ­ദ്ധാ­ന്തി­കന്‍­മാ­രാ­വു­ക­യും ചെ­യ്യാ­റു­ണ്ട്. ഇന്ന­ത്തെ­ക്കാ­ല­ത്തെ ഒരു മണി­ക്കൂര്‍ 
ടെ­ലി­വി­ഷന്‍ ഫോ­ക്ക­സ് ചര്‍­ച്ച­കള്‍ ഇത്ത­ര­ത്തില്‍ പൊ­തു­ജ­ന­സ­മ­ക്ഷം പാര്‍­ട്ടി­ക്ക് പല­പ്പോ­ഴും ക്ഷീ­ണ­മു­ണ്ടാ­ക്കി­യി­ട്ടു­ണ്ട്.

ഇ­ത്ത­ര­ത്തില്‍ പാര്‍­ട്ടി­ന­യ­രൂ­പി­ക­ര­ണ­ത്തി­നും അടി­സ്ഥാ­ന­പ്ര­മാ­ണ­ങ്ങ­ളെ ദൃ­ഡ­മാ­ക്കു­ന്ന­തി­നും ഏറെ സഹാ­യി­ച്ചി­ട്ടു­ള്ള 
ഉള്‍­പാര്‍­ട്ടി ചര്‍­ച്ച­ക­ളും സം­വാ­ദ­ങ്ങ­ളും പല­പ്പോ­ഴും പാര്‍­ട്ടി­യു­ടെ പാര്‍­ല­മെ­ന്റ­റി സാ­ധ്യ­ത­കള്‍­ക്ക് ഒരു ബാ­ധ്യ­ത­യാ­വു­ന്ന 
കാ­ഴ്ച ഈയ­ടു­ത്ത­കാ­ല­ത്താ­യി സാ­ധാ­ര­ണ­മാ­ണ്. ഇത് പ്ര­തി­പ­ക്ഷ­സ്വ­ര­ത്തി­ന് അര്‍­ഹ­മായ സ്ഥാ­നം കൊ­ടു­ക്കാന്‍ 
ശ്ര­മി­ക്കു­ന്ന പാര്‍­ട്ടി എന്ന­തി­നു­പ­ക­രം പ്ര­തി­പ­ക്ഷ­സ്വ­ര­ങ്ങ­ളെ അനു­വ­ദി­ക്കാ­ത്ത പാര്‍­ട്ടി എന്നൊ­രി­മേ­ജി­നും കാ­ര­ണ­മാ­യി. 
'കാ­റ്റും വെ­ളി­ച്ച­വും കട­ക്കാ­ന­നു­വ­ദി­ക്ക­രു­തെ­ന്ന' മട്ടി­ലു­ള്ള പ്ര­സ്താ­വ­ന­ക­ളും അതി­നാ­ദ്യം പാര്‍­ട്ടി­ക്കു­ള്ളില്‍­ത്ത­ന്നെ കി­ട്ടിയ 
സ്വീ­ക­ര­ണ­വും എല്ലാം കാ­ര്യ­ങ്ങള്‍ കൂ­ടു­തല്‍ മോ­ശ­മാ­ക്കി­യ­തേ­യു­ള്ളൂ. പല­പ്പോ­ഴും ഇത്ത­ര­മൊ­ര­വ­സ്ഥ സം­ജാ­ത­മാ­ക്കി­യ­ത്,
ഇ.എം­.എ­സ്സി­നെ­പ്പോ­ലെ, പ്ര­ത്യ­യ­ശാ­സ്ത്ര­വ്യാ­ഖ്യാ­ന­ങ്ങള്‍ നല്‍­കി ഉള്‍­പാര്‍­ട്ടി ആശ­യ­സം­വാ­ദ­ങ്ങള്‍­ക്ക് നേ­തൃ­ത്വം കൊ­ടു­ക്കാന്‍
പൊ­തു­സ­മ്മ­ത­നായ ഒരു നേ­താ­വോ പ്ര­ത്യ­യ­ശാ­സ്ത്ര­വി­ശാ­ര­ദ­നോ ഇല്ലാ­തെ പോ­യ­താ­ണ്. 50 വര്‍­ഷ­ത്തോ­ളം പാര്‍­ട്ടി­യു­ടെ 
പ്ര­ത്യ­യ­ശാ­സ്ത്ര അജ­ണ്ട നി­ശ്ച­യി­ക്കു­ന്ന­ത് ഒരു ദേ­ഹ­ത്തി­ന് ഏല്‍­പ്പി­ച്ചു കൊ­ടു­ത്ത­തി­ന്റെ പ്ര­ത്യാ­ഘാ­തം­.

ഇ­പ്പോള്‍ പാര്‍­ട്ടി­യില്‍ നട­ക്കു­ന്ന ആശ­യ­സം­വാ­ദ­ങ്ങ­ളും ചില ഉന്നം വച്ചു­ള്ള ഒതു­ക്ക­ലു­ക­ളും ഇ.എം­.എ­സ്. ഒഴി­ച്ചി­ട്ടു­പോയ 
കസേ­ര­യും അധി­കാ­ര­ത്തി­ന്റെ ഇട­നാ­ഴി­യില്‍ കണ്ണും നട്ടി­രി­ക്കു­ന്ന­വ­രും സൃ­ഷ്ടി­ക്കു­ന്ന ഓള­ങ്ങള്‍ മാ­ത്ര­മാ­ണ്. അത് ഈ 
സം­വാ­ദ­ത്തി­ന് കാ­ര­ണ­മാ­യി എഴു­ത­പ്പെ­ട്ട ലേ­ഖ­ന­ങ്ങ­ളും അവ­യു­ടെ വാ­യ­ന­ക­ളും ശ്ര­ദ്ധി­ച്ചാ­ല­റി­യാം­.

­പാര്‍­ട്ടി­യില്‍ ഇ.എം­.എ­സ്. ഒഴി­ച്ചി­ട്ടി­ട്ടു­പോയ പ്ര­ത്യ­യ­ശാ­സ്ത്ര­വ്യാ­ഖ്യാ­താ­വി­ന്റെ കസേ­ര­യ്ക്ക് ഒരു വ്യാ­ഴ­വ­ട്ട­ത്തി­നു ശേ­ഷ­വും 
വ്യ­ക്ത­മായ അവ­കാ­ശി­ക­ളൊ­ന്നും ഉണ്ടാ­യി­ട്ടി­ല്ല. പ്ര­ധാ­ന­കാ­ര­ണം, ഈയെ­മ്മ­സ്സി­നു ശേ­ഷം പാര്‍­ട്ടി­യ്ക്കു വേ­ണ്ടി 
പ്ര­ത്യ­യ­ശാ­സ്ത്ര­വ്യാ­ഖ്യാ­താ­ക്ക­ളു­ടെ ജോ­ലി ഏറ്റെ­ടു­ത്ത­വ­രെ­ല്ലാ­വ­രും ബു­ദ്ധി­ജീ­വി­യാ­യ­തി­നു ശേ­ഷം മാര്‍­ക്സി­സ്റ്റു­കാ­രായ 
പു­രോ­ഗ­മന കലാ­സാ­ഹി­ത്യ സം­ഘ­ക്കാ­രോ, സി­ഡി­യെ­സ്സു­കാ­രോ, പരി­ഷ­ത്തു­കാ­രോ ആയി­രു­ന്നു എന്ന­താ­ണ്. 
കാ­ലാ­കാ­ല­ങ്ങ­ളില്‍ പാര്‍­ട്ടി­യു­ടെ ദീര്‍­ഘ­കാ­ല­ന­യ­ങ്ങള്‍ രൂ­പ­പ്പെ­ടു­ത്തു­ന്ന­തി­ലും അവ­യ്ക്കു കൃ­ത്യ­മായ വ്യാ­ഖ്യാ­ന­ങ്ങള്‍ 
ചമ­യ്ക്കു­ന്ന­തി­ലും അവ­രാ­രും ഈയെ­മ്മ­സ്സി­നെ അപേ­ക്ഷി­ച്ച് മോ­ശ­മാ­യി­രു­ന്നി­ല്ല. എന്നാല്‍ പാര്‍­ല­മെ­ന്റ­റി 
രാ­ഷ്ട്രീ­യ­ത്തി­ല­ധി­ഷ്ഠി­ത­മായ ഹ്ര­സ്വ­കാല അട­വു­ന­യ­ങ്ങ­ളും, അവ­യും പാര്‍­ട്ടി­യു­ടെ പ്ര­ഖ്യാ­പി­ത­ന­യ­ങ്ങ­ളു­മാ­യു­ള്ള 
വൈ­രു­ദ്ധ്യ­ങ്ങ­ളും, ഈയെ­മ്മ­സ്സി­നെ­പ്പോ­ലെ ജനാ­ധി­പ­ത്യ­ക്ര­മ­ത്തി­ന­ക­ത്തു പ്ര­വര്‍­ത്തി­ക്കു­ന്ന കമ്യൂ­ണി­സ്റ്റ് പാര്‍­ട്ടി നട­ത്തു­ന്ന 
നീ­ക്കു­പോ­ക്കു­ക­ളാ­യി കാ­ണാ­ന­വര്‍­ക്കു കഴി­യാ­തെ പോ­യി. ചി­ലര്‍ അത്ത­രം അട­വു­ന­യ­ങ്ങ­ളെ­ത്ത­ന്നെ 
വ്യ­തി­യാ­ന­ങ്ങ­ളാ­യി­ക്ക­ണ്ട­പ്പോള്‍ (ഇ­ത്ത­ര­ക്കാര്‍­ക്ക് യഥാര്‍­ത്ഥ ഇട­തു­പ­ക്ഷ­മെ­ന്നെ­ല്ലാം പേ­രു­നല്‍­കി മാ­ദ്ധ്യ­മ­ങ്ങ­ളും 
ആഘോ­ഷി­ച്ചു) മറ്റു ചി­ലര്‍ ഇത്ത­രം നയ­ങ്ങ­ളെ­ക്കൂ­ടി ദീര്‍­ഘ­കാല പരി­പാ­ടി­യില്‍ ഉള്‍­പ്പെ­ടു­ത്തി നയ­വി­പു­ലീ­ക­ര­ണം നട­ത്താ­നും 
പു­തിയ വ്യാ­ഖ്യാ­ന­ങ്ങള്‍ ചമ­യ്ക്കാ­നു­മാ­ണ് ശ്ര­മി­ച്ച­ത്. ഒരു­പ­ക്ഷേ ബു­ദ്ധി­ജീ­വി­കള്‍ നേ­രി­ടു­ന്ന സ്വ­ത്വ­പ്ര­തി­സ­ന്ധി­യാ­യി­രി­ക്ക­ണം 
അവ­രെ­ക്കൊ­ണ്ടി­തെ­ല്ലാം ചെ­യ്യി­ച്ച­ത്.

ഇ­ത്ത­ര­ത്തി­ല­ല്ലാ­തെ­ത്ത­ന്നെ പാര്‍­ട്ടി­യു­ടെ താല്‍­ക്കാ­ലിക പാര്‍­ലി­മെ­ന്റ­റി നീ­ക്കു­പോ­ക്കു­ക­ളെ അങ്ങ­നെ­ത്ത­ന്നെ കാ­ണാ­നും,
അവ­യെ പ്ര­ത്യ­ശാ­സ്ത്ര­പ­ര­മാ­യി വ്യാ­ഖ്യാ­നി­ച്ചു ബു­ദ്ധി­മു­ട്ടേ­ണ്ട­തി­ല്ലെ­ന്നും വ്യ­ക്ത­മാ­യി മന­സ്സി­ലാ­ക്കിയ ചു­രു­ക്കം ചില പോ­സ്റ്റ്
ഇ.എം­.എ­സ്. ബു­ദ്ധി­ജീ­വി­ക­ളി­ലൊ­രാ­ളാ­ണ് ഡോ­.­തോ­മ­സ് ഐസ­ക്ക്. വി­ദേ­ശ­പ­ഠ­ന­ത്തി­ന്റേ­യും അന്താ­രാ­ഷ്ട്ര ഗ്രാ­ന്റു­ക­ളു­ടെ­യും
ബല­ത്തില്‍ ദീര്‍­ഘ­കാ­ലം സി­.­ഡി­.എ­സ്സില്‍ ഗവേ­ഷ­ക­നാ­യി­രു­ന്ന ധന­കാ­ര്യ­ബു­ദ്ധി­ജീ­വി. ഈ വി­ദേ­ശ­ഗ്രാ­ന്റു­ക­ളു­ടേ­യും മറ്റും 
പേ­രില്‍ 'യ­ഥാര്‍­ത്ഥ ഇട­തു­പ­ക്ഷ­ക്കാര്‍ ' ഏറെ­ക്കാ­ലം വേ­ട്ട­യ­ടി­യെ­ങ്കി­ലും അവ­സാ­നം ഐസ­ക് പാര്‍­ട്ടി­യ്ക്ക­ക­ത്തും സോ 
കാള്‍­ഡ് 'യ­ഥാര്‍­ത്ഥ ഇട­തു­പ­ക്ഷ­ക്കാര്‍' പാര്‍­ട്ടി­യ്ക്കു പു­റ­ത്തു­മാ­യി. ഐസ­ക്കി­നു കൂ­ടെ­യു­ണ്ടാ­യി­രു­ന്ന ബു­ദ്ധി­ജീ­വി 
സഖാ­ക്ക­ളെ­ല്ലാം തന്നെ ഏതാ­ണ്ടു പാര്‍­ട്ടി­യ്ക്കു പു­റ­ത്താ­യെ­ന്നു കൂ­ടി­യ­റി­യു­മ്പോ­ഴാ­ണ് ഈ 'വ­രേ­ണ്യ ബു­ദ്ധി­ജീ­വി' സഖാ­വ് 
കമ്യൂ­ണി­സ്റ്റ് മാര്‍­ക്സി­സ്റ്റു പാര്‍­ട്ടി­ക്ക­ക­ത്തെ രാ­ഷ്ട്രീ­യം എത്ര­പെ­ട്ട­ന്നു മന­സ്സി­ലാ­ക്കി­യെ­ന്നു നമ്മള്‍ തി­രി­ച്ച­റി­യു­ന്ന­ത്.

­പ­ക്ഷെ ഇക്കാ­ലം വരെ തോ­മ­സ് ഐസ­ക്ക് പാര്‍­ട്ടി­യ്ക്കു വേ­ണ്ടി തന്റെ ബു­ദ്ധി­ജീ­വി­ക്കു­പ്പാ­യം ഏറെ­യൊ­ന്നും ഉപ­യോ­ഗി­ച്ചി­ട്ടി­ല്ല 
(അ­ല്ലെ­ങ്കില്‍ അതി­ന്റെ ആവ­ശ്യ­മു­ണ്ടാ­യി­ട്ടി­ല്ല). ഒരു പ്രാ­വ­ശ്യം എം. പി. പര­മേ­ശ്വ­ര­നോ­ടൊ­പ്പം ചി­ല­തൊ­ക്കെ ചെ­യ്യാന്‍ നോ­ക്കി
കൈ പൊ­ള്ളിയ അനു­ഭ­വം ചെ­റു­താ­യു­ണ്ടു­താ­നും. ഈ സര്‍­ക്കാ­രി­ലെ തോ­മ­സ് ഐസ­ക്കി­ന്റെ പ്ര­വര്‍­ത്ത­ന­ങ്ങള്‍ (അ­തി­ന്
ഐസ­ക്ക് മുന്‍ ധന­കാ­ര്യ­മ­ന്ത്രി­മാ­രോ­ടു­കൂ­ടി നന്ദി പറ­യ­ണം) ഐസ­ക്കി­ന­വ­കാ­ശ­പ്പെ­ടാ­നി­ല്ലാ­തി­രു­ന്ന ജന­പി­ന്തു­ണ­യെ­ന്ന 
ഘട­ക­ത്തില്‍ വലി­യൊ­ര­ള­വ് മാ­റ്റ­മു­ണ്ടാ­ക്കി­യെ­ന്ന­തു സത്യ­മാ­ണ്. പാര്‍­ട്ടി കീ­ഴ്ഘ­ട­ക­ങ്ങ­ളി­ലെ സഖാ­ക്കള്‍­ക്ക് വരെ നേ­ട്ട­മാ­യി
എടു­ത്തു­പ­റ­യാന്‍ ഈ മന്ത്രി­സ­ഭ­യി­ലെ ചു­രു­ക്കം ചില രജ­ത­രേ­ഖ­ക­ളി­ലൊ­ന്നാ­ണ് ധന­കാ­ര്യ­വ­കു­പ്പ്.

ഈ സര്‍­ക്കാ­രി­ന്റെ കാ­ല­ശേ­ഷം പതി­വു­പോ­ലെ ജന­ങ്ങള്‍ യു­.­ഡി­.എ­ഫി­നെ തി­ര­ഞ്ഞെ­ടു­ത്ത­യ­ക്കാ­നാ­ണ് സാ­ധ്യ­ത. അപ്പോള്‍
പാര്‍­ലി­മെ­ന്റ­റി ജോ­ലി­ക്കു പാര്‍­ട്ടി നി­യോ­ഗി­ച്ച ഐസ­ക്ക­ട­ക്ക­മു­ള്ള മുന്‍­നിര നേ­താ­ക്കള്‍ മു­ഴു­വന്‍ സമയ പാര്‍­ട്ടി 
പ്ര­വര്‍­ത്ത­ന­ത്തി­ലേ­ക്ക് കട­ക്കാ­നു­ള്ള സാ­ധ്യത വി­ര­ള­മ­ല്ല (വീ­ണ്ടും ജയി­ച്ച് എം­.എല്‍.എ. ആവു­ക­യാ­ണെ­ങ്കില്‍ വീ­ണ്ടും 
പാര്‍­ലി­മെ­ന്റ­റി രം­ഗ­ത്തു­ത­ന്നെ കാ­ണും­). എന്നാല്‍ പാര്‍­ട്ടി­യില്‍ പു­തു­താ­യി കൈ­വ­ന്ന സ്വാ­ധീ­ന­വും (മ­ന്ത്ര­യെ­ന്ന നി­ല­യി­ലെ 
പ്ര­ക­ട­നം വഴി) സ്വ­ന്തം പാ­ണ്ഡി­ത്യ­വും ഐസ­ക്കി­നെ ഈയെ­മ്മ­സ്സ് ഒഴി­ച്ചി­ട്ട കസേ­ര­യി­ലെ ഒരു ഭാ­ഗ­ത്തി­ലേ­ക്ക് ആകര്‍­ഷി­ച്ചാല്‍
അത്ഭു­ത­മൊ­ന്നു­മി­ല്ല. പ്രാ­യോ­ഗിക സാ­മ്പ­ത്തിക ശാ­സ്ത്ര­ത്തി­ലെ സ്വാ­ധീ­നം പു­ത്തന്‍ കാ­ല­ഘ­ട്ട­ങ്ങ­ളി­ലെ 
മാര്‍­ക്സി­സ്റ്റു­വ്യാ­ഖ്യാ­ന­ങ്ങ­ളു­ടെ സാ­മ്പ­ത്തി­ക­ശാ­സ്ത്ര­വ­ഴി തനി­ക്കു വഴ­ങ്ങ­മെ­ന്നു തെ­ളി­യി­ക്കാന്‍ കു­റ­ച്ചു­കാ­ല­മാ­യി 
പണി­യി­ല്ലാ­തി­രി­ക്കു­ന്ന ഐസ­ക്കി­ലെ ബു­ദ്ധി­ജീ­വി­ക്കു തോ­ന്നി­യാല്‍ അതു പാ­റ്റ­യെ­പ്പി­ടി­ച്ചി­ടു­ന്ന­ത് ആസ്ഥാന വ്യാ­ഖ്യാ­താ­വി­ന്റെ 
കു­പ്പാ­യം ലക്ഷ്യ­മി­ട്ടി­രി­ക്കു­ന്ന അര്‍­ദ്ധ­ബു­ദ്ധി­ജീ­വി സഖാ­ക്ക­ളു­ടെ കഞ്ഞി­യി­ലാ­യി­രി­ക്കും­.

­പു­.­ക.­സ.­യി­ലെ അതി­ബു­ദ്ധി­ജീ­വി സഖാ­ക്ക­ളു­ടെ വൃ­ത്ത­മൊ­പ്പി­ക്കല്‍ കൊ­ണ്ട് അധി­ക­കാ­ലം പി­ടി­ച്ചു നില്‍­ക്കാ­നാ­വി­ല്ലെ­ന്ന് 
പാര്‍­ട്ടി നേ­തൃ­ത്വ­ത്തി­നു തന്നെ ബോ­ധ­മു­ണ്ടാ­ക­ണം. വ്യാ­ഖ്യാ­ന­ങ്ങള്‍ സാ­മൂ­ഹ്യ­ശാ­സ്ത്ര­പ­ര­മാ­യും സാ­മ്പ­ത്തി­ക­പ­ര­മാ­യും 
ചമ­യ്ക്കു­ന്ന­തില്‍ അവ­രൊ­രു­പ­ക്ഷെ ഈയെ­മ്മെ­സ്സി­നെ വരെ കട­ത്തി­വെ­ട്ടും. പക്ഷെ വ്യാ­ഖ്യാ­ന­ങ്ങ­ളെ നേ­താ­ക്ക­ളി­ലേ­ക്കും 
അണി­ക­ളി­ലേ­ക്കും പൊ­തു­ജ­ന­ങ്ങ­ളി­ലേ­ക്കും കമ്യൂ­ണി­ക്കേ­റ്റു ചെ­യ്യാന്‍ പു­.­ക.­സ. സഖാ­ക്ക­ളു­ടെ അതി­ക­ഠിന അക്കാ­ദ­മിക ഭാ­ഷ­യും
ഭാ­വ­ങ്ങ­ളും ഒരു തട­സ്സ­മാ­ണ്. തത്വ­ശാ­സ്ത്ര അക്കാ­ദ­മി­ക­രം­ഗ­ത്തെ പദ­പ്ര­യോ­ഗ­ങ്ങ­ളില്‍ മി­നി­മം അവ­ഗാ­ഹ­വും പൊ­തു­വായ 
തത്വ­ശാ­സ്ത്ര­രീ­തി­ക­ളില്‍ അറി­വു­മി­ല്ലാ­ത്ത­വര്‍­ക്ക് സം­സ്കൃ­ത­ത്തേ­ക്കാ­ളും കഠി­ന­മാ­യേ­ക്കാം പു­.­ക.­സ. ബു­ദ്ധി­ജീ­വി­ക­ളു­ടെ വാ­ചക
കസര്‍­ത്ത്. കൂ­ടാ­തെ അക്കാ­ദ­മി­ക് ഇന്റ­ഗ്രി­റ്റി അഥ­വാ ബൌ­ദ്ധിക സത്യ­സ­ന്ധത എന്നൊ­രു വാള്‍ അവ­രെ പല­പ്പോ­ഴും 
പാര്‍­ട്ടി­ക്കൊ­രു ബാ­ദ്ധ്യ­ത­യാ­ക്കു­ക­യും ചെ­യ്യും. ഇത് ശരി­ക്ക­റി­യാ­വു­ന്ന ചില സഖാ­ക്കള്‍ ആസ്ഥാന ബു­ദ്ധി­ജീ­വി വൃ­ന്ദ­ത്തില്‍ 
തന്റെ പേ­രു­കൂ­ടി ഉള്‍­പ്പെ­ടു­ത്താന്‍ കി­ണ­ഞ്ഞു പരി­ശ്ര­മി­ക്കു­ന്ന­തി­ന്റെ ഫല­മാ­യി വേ­ണ­മെ­ങ്കില്‍ പാര്‍­ട്ടി­ക്ക­ക­ത്തെ ഇപ്പോ­ഴ­ത്തെ
ആശ­യ­സ­മ­ര­ത്തെ കാ­ണാം. അതി­നി­ട­യില്‍ ചില തല­മു­തിര്‍­ന്ന നേ­താ­ക്കള്‍ നട­ത്തു­ന്ന ഇട­പെ­ട­ലു­കള്‍ മന്ത്രി­പ്പ­ണി­ക്കു ശേ­ഷം
പാര്‍­ട്ടി­യില്‍ പു­റ­മ്പോ­ക്കി­ലാ­വാ­തി­രി­ക്കാ­നു­ള്ള വെ­പ്രാ­ള­ത്തി­ന്റെ ബാ­ക്കി­പ­ത്ര­മാ­ണെ­ന്നും സം­ശ­യി­ക്ക­ണം­.

­പൊ­തു­വേ­ദി­യില്‍ രണ്ടു­കൂ­ട്ട­രും കൂ­ടി ഒരു­മി­ച്ച് ആക്ര­മി­ച്ച് കീ­ഴ്പെ­ടു­ത്താന്‍ മാ­ത്രം ശക്ത­രൊ­ന്നു­മ­ല്ല പു­.­ക.­സ. ബു­ദ്ധി­ജീ­വി­കള്‍. 
ശരീ­ര­പ്ര­കൃ­തി­പോ­ലെ­ത്ത­ന്നെ, ആഞ്ഞ ഒരു കാ­റ്റില്‍ പാ­റി­പ്പോ­കാ­നു­ള്ള­തേ­യു­ള്ളു അവ­രു­ടെ പൊ­തു­സ­മ്മ­തി. എന്നാല്‍ കു­റ­ച്ച്
കാ­ലം മു­മ്പ് ശക്ത­മായ എതി­രാ­ളി­ക­ളെ നേ­രി­ടാന്‍ 'കാ­റ്റും വെ­ളി­ച്ച­വും കട­ക്കാന്‍ പാ­ടി­ല്ലാ­ത്ത' പാര്‍­ട്ടി­യു­ടെ വക്താ­ക്കള്‍ 
നട­ത്തിയ തര­ത്തി­ലു­ള്ള ഇട­പെ­ട­ലു­ക­ളു­ടെ തല­ത്തി­ലേ­ക്ക് ഈ സം­വാ­ദം പോ­കു­ന്ന­ത് ഒരു­പ­ക്ഷേ പു­.­ക.­സ. 
സഖാ­ക്കള്‍­ക്കു­മ­പ്പു­റ­മാ­കാം ലക്ഷ്യ­മെ­ന്നൊ­രു സം­ശ­യ­ത്തി­നും ഇട നല്‍­കു­ന്നു­.

എ­ന്താ­യാ­ലും സന്നാ­ഹ­ങ്ങ­ളു­ടെ ബാ­ഹു­ല്യം വ്യ­ക്ത­മായ ഒരു സൈ­ദ്ധാ­ന്തി­ക­സ­മ­രം സഖാ­വ് വി­.എ­സ്സി­നൊ­പ്പ­മു­ള്ള 
ബു­ദ്ധി­ജീ­വി­കള്‍ ആസൂ­ത്ര­ണം ചെ­യ്ത് നട­പ്പാ­ക്കിയ 'വി­ശു­ദ്ധ വി­.എ­സ്. നശി­ച്ച പാര്‍­ട്ടി' ടൈ­പ്പ് ചേ­രി­തി­രി­വു­ക­ളി­ലേ­ക്ക് 
നയി­ക്കു­മോ എന്നു കാ­ത്തി­രു­ന്നു കാ­ണാം­.

­പോ­സ്റ്റ് സ്ക്രി­പ്റ്റ്: ലോ­ക­ത്തു­ള്ള കമ്യൂ­ണി­സ്റ്റും അല്ലാ­ത്ത­തു­മായ ബു­ദ്ധി­ജീ­വി­ക­ളു­ടെ­യും ചി­ന്ത­കന്‍­മാ­രു­ടേ­യും ഐഡ­ന്റി­റ്റി 
പൊ­ളി­റ്റി­ക്സി­ലു­ള്ള (സ്വ­ത്വ രാ­ഷ്ട്രീ­യം) രച­ന­ക­ളൊ­ന്നും പോ­രാ­ഞ്ഞി­ട്ടാ­യി­രി­ക്കും സഖാ­വു ബേ­ബി മേ­യ് 25­ന് 'ദ ഹി­ന്ദു­'­വി­ലെ 
(24­ന് 'ദ ന്യൂ­യോര്‍­ക്ക് ടൈം­സി­ലെ­') പോള്‍ ക്രൂ­ഗ്മാ­ന്റെ കോ­ള­ത്തി­ലെ ഐഡ­ന്റി­റ്റി പൊ­ളി­റ്റി­ക്‌­സെ­ന്ന പദ­ത്തി­ലേ­ക്ക് 
റെ­ഫര്‍ ചെ­യ്ത­ത്. പോ­ക്ക­റും പാര്‍­ട്ടി­യും എന്തി­നു രാ­ജീ­വും വരെ പാര്‍­ശ്വ­വ­ത്ക­രി­ക്ക­പ്പെ­ട്ട­വ­രു­ടെ സ്വ­ത്വ­രാ­ഷ്ട്രീ­യ­ത്തെ­പ്പ­റ്റി 
സം­സാ­രി­ക്കു­മ്പോ, അവി­ടെ മു­ഖ്യ­ധാ­ര­യി­ലു­ള്ള "അ­മേ­രി­ക്ക­നി­സം" എന്ന ഐഡ­ന്റി­റ്റി റി­പ്പ­ബ്ലി­ക്കന്‍ പാര്‍­ട്ടി­യും അതി­നോ­ടു 
ചേര്‍­ന്നു നില്‍­ക്കു­ന്ന കോര്‍­പ്പ­റേ­റ്റ് അമേ­രി­ക്ക­യും ചേര്‍­ന്ന് വരു­ന്ന സെ­ന­റ്റ്-കോണ്‍­ഗ്ര­സ്സ് ഇല­ക്ഷ­നില്‍ പ്ര­യോ­ഗി­ക്കാന്‍ 
പോ­കു­ന്ന­തി­നെ­പ്പ­റ്റി­യാ­ണ് ക്രൂ­ഗ്മാന്‍ വാ­ചാ­ല­നാ­വു­ന്ന­ത്.

'ഇ­ര­ക­ളു­ടെ മാ­നി­ഫെ­സ്റ്റൊ­'­യില്‍ നി­ന്നും മു­ഖ്യ­ധാ­രാ സ്വ­ത്വ­ങ്ങ­ളു­ടെ രാ­ഷ്ട്രീയ മാ­നി­ഫെ­സ്റ്റൊ­കള്‍­ക്കു­ള്ള വ്യ­ത്യാ­സം പോ­ലും 
മന­സ്സി­ലാ­ക്കാ­തെ­യാ­ണോ ബേ­ബി സഖാ­വു സ്വ­ത്വ­രാ­ഷ്ട്രീ­യ­ത്തി­ന്റെ അപ­ക­ട­ങ്ങ­ളെ­ക്കു­റി­ച്ചു വാ­ചാ­ല­നാ­യ­ത്? ബേ­ബി­ക്കു 
നാ­ണം വന്നി­ല്ലെ­ങ്കി­ലും ഇതി­നു മറു­പ­ടി­പ­റ­യാന്‍ ഒരു­പ­ക്ഷേ കെ­.ഇ­.എ­ന്നി­നും പി­.­കെ. പോ­ക്കര്‍­ക്കും നാ­ണം കാ­ണും. 
രാ­ജീ­വ് സഖാ­വ് പറ­ഞ്ഞ­ത്, സ്വ­ത്വ­രാ­ഷ്ട്രീ­യ­ത്തി­ന്റെ ചതി­ക്കു­ഴി­ക­ളെ­ക്കു­റി­ച്ച് പാര്‍­ട്ടി സ്റ്റ­ഡി­ക്ലാ­സ്സു­ക­ളില്‍ അടു­ത്ത 
കാ­ല­ത്താ­യി നല്ല­പോ­ലെ വി­ഷ­യ­മാ­യി­ട്ടു­ണ്ടെ­ന്നാ­ണ്. സ്റ്റ­ഡി­ക്ലാ­സു­ക­ളൊ­ന്നും സഖാ­വ് ബേ­ബി­യാ­യി­രി­ക്കി­ല്ല കൈ­കാ­ര്യം 
ചെ­യ്ത­തെ­ന്നും നമു­ക്ക് പ്ര­തീ­ക്ഷി­ക്കാം­.

(16 June 2010)\footnote{http://malayal.am/പലവക/പരമ്പര/സ്വത്വം/6150/ബുദ്ധിജീവികളുടെ-സ്വത്വപ്രതിസന്ധി}

\newpage
