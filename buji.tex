\secstar{ബുദ്ധിജീവികളുടെ സ്വത്വപ്രതിസന്ധി}
\vskip 2pt

ആശയസംഘട്ടനങ്ങളും സംവാദങ്ങളും വ്യത്യസ്തവിശകലനങ്ങളും അഭിപ്രായങ്ങളും കമ്യൂണിസ്റ്റ് മാര്‍ക്സിസ്റ്റ് പാര്‍ട്ടിയില്‍ 
ഇന്നുംഇന്നലെയുമുള്ള പ്രതിഭാസമല്ല. മാര്‍ക്സിസമെന്ന ചിന്താപദ്ധതിതന്നെ വൈരുദ്ധ്യാത്മകതയില്‍ (dialectics) 
അടിയുറച്ചതായതുകൊണ്ടു്, അവസാന ഇന്‍ഫറന്‍സിലെത്തണമെങ്കില്‍ തീസിസും അതിനൊരു ആന്റിതീസിസും 
അത്യാവശ്യമാണുതാനും. ഇതു രണ്ടിലും ഊന്നിയുള്ള ഉള്‍പാര്‍ട്ടിവാദങ്ങളിലൂടെയും സംവാദങ്ങളിലൂടെയും പുരോഗമിച്ചു് 
അവസാനം പാര്‍ട്ടി ഏകകണ്ഠമായി ഇന്‍ഫറന്‍സിലെത്തുകയാണു പതിവു്. പലപ്പോഴും ഉള്‍പാര്‍ട്ടിസംവാദങ്ങളില്‍ സ്വന്തംഭാഗം 
പാര്‍ട്ടി ശരിക്കും കണക്കിലെടുത്തില്ലെന്നോ (ഒതുക്കിക്കളഞ്ഞെന്നോ) ഒക്കെയുള്ള കാരണങ്ങളാല്‍ 
പലകാലഘട്ടങ്ങളില്‍ പലരും പാര്‍ട്ടി വിട്ടുപോയിട്ടുമുണ്ടു്. ചില തീസിസുകളെ റിവിഷനിസ്റ്റ് അഭിപ്രായങ്ങളാണെന്നു 
വകയിരുത്തി പാര്‍ട്ടി തള്ളിക്കളയുകയും, തീസിസിന്റെ അവതാരകര്‍ അതംഗീകരിക്കാത്തതുകൊണ്ടു് 
അച്ചടക്കലംഘനമായി കണക്കാക്കി പാര്‍ട്ടിയില്‍നിന്നും പുറത്താക്കപ്പെടുയും ഉണ്ടായിട്ടുണ്ടു്.

അതുപോലെ ചില പാര്‍ട്ടിനയങ്ങളെ കാലോചിതമായി പരിഷ്കരിക്കുകയും തെറ്റുതിരുത്തല്‍ എന്നു് മാദ്ധ്യമങ്ങള്‍ വിളിക്കുന്ന 
പരിഷ്കരണപ്രക്രിയയിലൂടെ തിരുത്തുകയും ചെയ്തിട്ടുണ്ടു്. പാര്‍ട്ടിയില്‍ ആശയസംഘട്ടനമുണ്ടായ ഘട്ടങ്ങളില്‍ പലപ്പോഴും 
അച്ചടക്കനടപടികളുമുണ്ടായിട്ടുണ്ടു്. (പാര്‍ട്ടി വിരുദ്ധപ്രവര്‍ത്തനത്തിന്റെ പേരിലുള്ള അച്ചടക്കനടപടിയല്ല, 
ആശയവ്യതിയാനത്തിന്റെ പേരില്‍.) ഇത്തരത്തില്‍ ബൗദ്ധികവ്യവഹാരങ്ങളും ജനകീയപിന്തുണയും നയങ്ങളുടെ 
നിശിതമായ ഉള്‍പ്പാര്‍ട്ടിവിമര്‍ശനവും നടത്തുന്ന പാര്‍ട്ടിയായതിനാലാവണം കമ്യൂണിസ്റ്റ് മാര്‍ക്സിസ്റ്റു പാര്‍ട്ടിക്കു് 
സമൂഹത്തിന്റെ എല്ലാതലങ്ങളിലും നല്ല പിന്തുണയുള്ളതു്. എന്നാലും മറ്റേതൊരു പാര്‍ലമെന്ററി പാര്‍ട്ടിയേയുംപോലെ 
അധികാരത്തിന്റെ ഇടനാഴികകളിലേക്കു് കണ്ണുംനട്ടിരിക്കുന്നവരും പാര്‍ട്ടിയിലുണ്ടു്. അവര്‍ ഇത്തരം ഉള്‍പ്പാര്‍ട്ടി 
പ്രത്യയശാസ്ത്രസംവാദങ്ങളെ പൊതുജനസമക്ഷം കൊണ്ടുപോയി സ്വന്തംകാര്യം നടത്താനാണു് ശ്രമിക്കാറു്.

കേരളത്തിന്റെ കാര്യത്തില്‍, പാര്‍ട്ടി അധികാരത്തിലിരിക്കുന്ന അവസാനനാളുകളിലോ, അധികാരം നഷ്ടപ്പെട്ട 
ഉടനെയോ ആണു് നയപരിഷ്കരണമെന്ന ആശയവുമായി ബുദ്ധിജീവികളിറങ്ങാറുള്ളതു്. കേരളത്തിലെ ഇലക്‌ഷനിലെ 
ജയപരാജയങ്ങള്‍ പലപ്പോഴും ജനപിന്തുണയേക്കാള്‍ പാര്‍ട്ടികളുടെ അടവുനയങ്ങളോടു് (പാര്‍ട്ടിയില്‍ കൂട്ടായ ചര്‍ച്ച 
നടത്താതെ, പാര്‍ലമെന്ററി ആവശ്യങ്ങള്‍ക്കായി നേതൃത്വം കാലാകാലങ്ങളില്‍ എടുക്കുന്ന തീരുമാനങ്ങള്‍) ജനങ്ങളുടെ 
പ്രതികരണമാണെന്നതു് പരസ്യമായ രഹസ്യമാണെങ്കിലും ജനകീയപിന്തുണ നഷ്ടപ്പെട്ടതാണു് ഇലക്‌ഷനില്‍ 
പരാജയപ്പെടുന്നതിനു കാരണമെന്ന മുന്‍വിധിയോടെയാണു് നയപരിഷ്കരണവാദങ്ങള്‍ ഉയരാറുള്ളതു്.

പലപ്പോഴും പ്രത്യയശാസ്ത്രസംവാദങ്ങളെ പൊതുജനസമക്ഷം അവതരിപ്പിക്കുന്നതു് അപകടമായിത്തീരാറുണ്ടു്. 
പാര്‍ട്ടിവേദികളില്‍ ചര്‍ച്ചചെയ്യുമ്പോള്‍ തത്വശാസ്ത്രങ്ങളില്‍ കാലാകാലങ്ങളില്‍ സ്റ്റഡിക്ലാസുകളിലൂടെയും 
വായനയിലൂടെയും പാര്‍ട്ടി നേടിയ മിനിമം അറിവു് പൊതുവേദികളില്‍ കാണാനാവില്ല. അതിനാല്‍ത്തന്നെ, ബൗദ്ധികമായി 
ഉന്നതനിലവാരം പുലര്‍ത്തുന്നവരും തത്വശാസ്ത്രപാണ്ഡിത്യമുള്ളവരും പൊതുചര്‍ച്ചകളില്‍ കപടബുദ്ധിജീവികളായി 
മുദ്രകുത്തപ്പെടുകയും, മുറിമൂക്കന്‍ എന്നു വിളിക്കാനാവുന്ന, ഉളുപ്പില്ലാതെ ആരെയും ഉദ്ധരിച്ചു് ജനങ്ങളെ അമ്പരപ്പിക്കുന്ന 
വാഗ്‌വിലാസക്കാര്‍ പാര്‍ട്ടിസൈദ്ധാന്തികന്‍മാരാവുകയും ചെയ്യാറുണ്ടു്. ഇന്നത്തെക്കാലത്തെ ഒരു മണിക്കൂര്‍ 
ടെലിവിഷന്‍ ഫോക്കസ് ചര്‍ച്ചകള്‍ ഇത്തരത്തില്‍ പൊതുജനസമക്ഷം പാര്‍ട്ടിക്കു് പലപ്പോഴും ക്ഷീണമുണ്ടാക്കിയിട്ടുണ്ടു്.

ഇത്തരത്തില്‍ പാര്‍ട്ടിനയരൂപീകരണത്തിനും അടിസ്ഥാനപ്രമാണങ്ങള്‍ ദൃഢമാക്കുന്നതിനും ഏറെ സഹായിച്ചിട്ടുള്ള 
ഉള്‍പാര്‍ട്ടി ചര്‍ച്ചകളും സംവാദങ്ങളും പലപ്പോഴും പാര്‍ട്ടിയുടെ പാര്‍ലമെന്ററി സാധ്യതകള്‍ക്കു് ഒരു ബാധ്യതയാവുന്ന 
കാഴ്ച ഈയടുത്തകാലത്തായി സാധാരണമാണു്. ഇതു് പ്രതിപക്ഷസ്വരത്തിനു് അര്‍ഹമായ സ്ഥാനം കൊടുക്കാന്‍ 
ശ്രമിക്കുന്ന പാര്‍ട്ടി എന്നതിനുപകരം പ്രതിപക്ഷസ്വരങ്ങളെ അനുവദിക്കാത്ത പാര്‍ട്ടി എന്നൊരിമേജിനും കാരണമായി. 
'കാറ്റും വെളിച്ചവും കടക്കാനനുവദിക്കരുതെന്ന' മട്ടിലുള്ള പ്രസ്താവനകളും അതിനാദ്യം പാര്‍ട്ടിക്കുള്ളില്‍ത്തന്നെ കിട്ടിയ 
സ്വീകരണവും എല്ലാം കാര്യങ്ങള്‍ കൂടുതല്‍ മോശമാക്കിയതേയുള്ളൂ. പലപ്പോഴും ഇത്തരമൊരവസ്ഥ സംജാതമാക്കിയതു്
ഇ.എം.എസ്സിനെപ്പോലെ പ്രത്യയശാസ്ത്രവ്യാഖ്യാനങ്ങള്‍ നല്‍കി ഉള്‍പാര്‍ട്ടി ആശയസംവാദങ്ങള്‍ക്ക് നേതൃത്വം കൊടുക്കാന്‍
പൊതുസമ്മതനായ ഒരു നേതാവോ പ്രത്യയശാസ്ത്രവിശാരദനോ ഇല്ലാതെ പോയതാണു്. 50 വര്‍ഷത്തോളം പാര്‍ട്ടിയുടെ 
പ്രത്യയശാസ്ത്ര അജണ്ട നിശ്ചയിക്കുന്നതു് ഒരു ദേഹത്തിനു് ഏല്‍പ്പിച്ചു കൊടുത്തതിന്റെ പ്രത്യാഘാതം.

ഇപ്പോള്‍ പാര്‍ട്ടിയില്‍ നടക്കുന്ന ആശയസംവാദങ്ങളും ചില ഉന്നംവച്ചുള്ള ഒതുക്കലുകളും, ഇ.എം.എസ്. ഒഴിച്ചിട്ടുപോയ 
കസേരയും അധികാരത്തിന്റെ ഇടനാഴിയില്‍ കണ്ണുംനട്ടിരിക്കുന്നവരും സൃഷ്ടിക്കുന്ന ഓളങ്ങള്‍ മാത്രമാണു്. ഈ 
സംവാദത്തിനു് കാരണമായി എഴുതപ്പെട്ട ലേഖനങ്ങളും അവയുടെ വായനകളും ശ്രദ്ധിച്ചാലിതറിയാം.

പാര്‍ട്ടിയില്‍ ഇ.എം.എസ്. ഒഴിച്ചിട്ടുപോയ പ്രത്യയശാസ്ത്രവ്യാഖ്യാതാവിന്റെ കസേരയ്ക്കു് ഒരു വ്യാഴവട്ടത്തിനുശേഷവും 
വ്യക്തമായ അവകാശികളൊന്നും ഉണ്ടായിട്ടില്ല. പ്രധാനകാരണം, ഈയെമ്മസ്സിനുശേഷം പാര്‍ട്ടിയ്ക്കുവേണ്ടി 
പ്രത്യയശാസ്ത്രവ്യാഖ്യാതാക്കളുടെ ജോലി ഏറ്റെടുത്തവരെല്ലാവരും ബുദ്ധിജീവിയായതിനുശേഷം മാര്‍ക്സിസ്റ്റുകാരായ 
പുരോഗമന കലാസാഹിത്യ സംഘക്കാരോ, സിഡിയെസ്സുകാരോ, പരിഷത്തുകാരോ ആയിരുന്നു എന്നതാണു്. 
കാലാകാലങ്ങളില്‍ പാര്‍ട്ടിയുടെ ദീര്‍ഘകാലനയങ്ങള്‍ രൂപപ്പെടുത്തുന്നതിലും അവയ്ക്കു് കൃത്യമായ വ്യാഖ്യാനങ്ങള്‍ 
ചമയ്ക്കുന്നതിലും അവരാരും ഈയെമ്മസ്സിനെ അപേക്ഷിച്ചു് മോശമായിരുന്നില്ല. എന്നാല്‍ പാര്‍ലമെന്ററി 
രാഷ്ട്രീയത്തിലധിഷ്ഠിതമായ ഹ്രസ്വകാല അടവുനയങ്ങളും, അവയും പാര്‍ട്ടിയുടെ പ്രഖ്യാപിതനയങ്ങളുമായുള്ള 
വൈരുദ്ധ്യങ്ങളും, ജനാധിപത്യക്രമത്തിനകത്തു പ്രവര്‍ത്തിക്കുന്ന കമ്യൂണിസ്റ്റ് പാര്‍ട്ടി നടത്തുന്ന 
നീക്കുപോക്കുകളായി ഈയെമ്മസ്സിനെപ്പോലെ കാണാന്‍ അവര്‍ക്കു കഴിയാതെപോയി. ചിലര്‍ അത്തരം അടവുനയങ്ങളെത്തന്നെ 
വ്യതിയാനങ്ങളായിക്കണ്ടപ്പോള്‍ (ഇത്തരക്കാര്‍ക്കു് യഥാര്‍ത്ഥ ഇടതുപക്ഷമെന്നെല്ലാം പേരുനല്‍കി മാദ്ധ്യമങ്ങളും 
ആഘോഷിച്ചു), മറ്റുചിലര്‍ ഇത്തരം നയങ്ങളെ ദീര്‍ഘകാലപരിപാടിയില്‍ ഉള്‍പ്പെടുത്തി നയവിപുലീകരണം നടത്താനും 
പുതിയ വ്യാഖ്യാനങ്ങള്‍ ചമയ്ക്കാനുമാണു് ശ്രമിച്ചതു്. ഒരുപക്ഷേ ബുദ്ധിജീവികള്‍ നേരിടുന്ന സ്വത്വപ്രതിസന്ധിയായിരിക്കണം 
അവരെക്കൊണ്ടിതെല്ലാം ചെയ്യിച്ചതു്.

ഇത്തരത്തിലല്ലാതെ പാര്‍ട്ടിയുടെ താല്‍ക്കാലിക പാര്‍ലിമെന്ററി നീക്കുപോക്കുകളെ അങ്ങനെത്തന്നെ കാണാനും,
അവയെ പ്രത്യശാസ്ത്രപരമായി വ്യാഖ്യാനിച്ചു ബുദ്ധിമുട്ടേണ്ടതില്ലെന്നും വ്യക്തമായി മനസ്സിലാക്കിയ ചുരുക്കം ചില പോസ്റ്റ്
ഇ.എം.എസ്. ബുദ്ധിജീവികളിലൊരാളാണു് ഡോ. തോമസ് ഐസക്ക്. വിദേശപഠനത്തിന്റേയും അന്താരാഷ്ട്ര ഗ്രാന്റുകളുടെയും
ബലത്തില്‍ ദീര്‍ഘകാലം സി.ഡി.എസ്സില്‍ ഗവേഷകനായിരുന്ന ധനകാര്യബുദ്ധിജീവി. ഈ വിദേശഗ്രാന്റുകളുടേയും മറ്റും 
പേരില്‍ 'യഥാര്‍ത്ഥ ഇടതുപക്ഷക്കാര്‍ ' ഏറെക്കാലം വേട്ടയാടിയെങ്കിലും അവസാനം ഐസക് പാര്‍ട്ടിയ്ക്കകത്തും, സോ 
കാള്‍ഡ് 'യഥാര്‍ത്ഥ ഇടതുപക്ഷക്കാര്‍' പാര്‍ട്ടിയ്ക്കു പുറത്തുമായി. ഐസക്കിന്റെ കൂടെയുണ്ടായിരുന്ന ബുദ്ധിജീവി 
സഖാക്കളെല്ലാംതന്നെ ഏതാണ്ടു് പാര്‍ട്ടിയ്ക്കു പുറത്തായെന്നു കൂടിയറിയുമ്പോഴാണു് ഈ 'വരേണ്യ ബുദ്ധിജീവി' സഖാവു് 
കമ്യൂണിസ്റ്റ് മാര്‍ക്സിസ്റ്റു പാര്‍ട്ടിക്കകത്തെ രാഷ്ട്രീയം എത്രപെട്ടന്നു മനസ്സിലാക്കിയെന്നു് നമ്മള്‍ തിരിച്ചറിയുന്നതു്.

പക്ഷെ ഇക്കാലംവരെ തോമസ് ഐസക്ക് പാര്‍ട്ടിയ്ക്കുവേണ്ടി തന്റെ ബുദ്ധിജീവിക്കുപ്പായം ഏറെയൊന്നും ഉപയോഗിച്ചിട്ടില്ല, 
അല്ലെങ്കില്‍ അതിന്റെ ആവശ്യമുണ്ടായിട്ടില്ല. ഒരു പ്രാവശ്യം എം.പി. പരമേശ്വരനോടൊപ്പം ചിലതൊക്കെ ചെയ്യാന്‍നോക്കി
കൈപൊള്ളിയ അനുഭവം ചെറുതായുണ്ടുതാനും. ഈ സര്‍ക്കാരിലെ തോമസ് ഐസക്കിന്റെ പ്രവര്‍ത്തനങ്ങള്‍ (അതിനു്
ഐസക് മുന്‍ധനകാര്യമന്ത്രിമാരോടുകൂടി നന്ദി പറയണം) ഐസക്കിനവകാശപ്പെടാനില്ലാതിരുന്ന ജനപിന്തുണയെന്ന 
ഘടകത്തില്‍ വലിയൊരളവു് മാറ്റമുണ്ടാക്കിയെന്നതു് സത്യമാണു്. പാര്‍ട്ടി കീഴ്ഘടകങ്ങളിലെ സഖാക്കള്‍ക്കുവരെ നേട്ടമായി
എടുത്തുപറയാന്‍ ഈ മന്ത്രിസഭയിലെ ചുരുക്കം ചില രജതരേഖകളിലൊന്നാണു് ധനകാര്യവകുപ്പു്.

ഈ സര്‍ക്കാരിന്റെ കാലശേഷം പതിവുപോലെ ജനങ്ങള്‍ യു.ഡി.എഫിനെ തിരഞ്ഞെടുത്തയക്കാനാണു് സാധ്യത. അപ്പോള്‍
പാര്‍ലിമെന്ററി ജോലിക്കു് പാര്‍ട്ടി നിയോഗിച്ച ഐസക്കടക്കമുള്ള മുന്‍നിരനേതാക്കള്‍ മുഴുവസമയ പാര്‍ട്ടി 
പ്രവര്‍ത്തനത്തിലേക്കു് കടക്കാനുള്ള സാധ്യത വിരളമല്ല. വീണ്ടും ജയിച്ചു് എം.എല്‍.എ. ആവുകയാണെങ്കില്‍ വീണ്ടും 
പാര്‍ലിമെന്ററി രംഗത്തുതന്നെ കാണും. എന്നാല്‍ പാര്‍ട്ടിയില്‍ പുതുതായി കൈവന്ന സ്വാധീനവും, മന്ത്രിയെന്ന നിലയിലെ 
പ്രകടനംവഴി സ്വന്തം പാണ്ഡിത്യവും ഐസക്കിനെ ഈയെമ്മസ്സ് ഒഴിച്ചിട്ട കസേരയിലെ ഒരു ഭാഗത്തിലേക്കു് ആകര്‍ഷിച്ചാല്‍
അത്ഭുതമൊന്നുമില്ല. പ്രായോഗിക സാമ്പത്തികശാസ്ത്രത്തിലെ പാഠങ്ങളില്‍നിന്നു് പുത്തന്‍കാലഘട്ടങ്ങളിലെ 
മാര്‍ക്സിസ്റ്റുവ്യാഖ്യാനങ്ങളുടെ സാമ്പത്തികശാസ്ത്രവഴി തനിക്കു വഴങ്ങുമെന്നു തെളിയിക്കാന്‍ ഐസക്കിലെ ബുദ്ധിജീവിക്കു തോന്നിയാല്‍ 
അതു പാറ്റയെപ്പിടിച്ചിടുന്നതു് ആസ്ഥാനവ്യാഖ്യാതാവിന്റെ കുപ്പായം ലക്ഷ്യമിട്ടിരിക്കുന്ന 
അര്‍ദ്ധബുദ്ധിജീവി സഖാക്കളുടെ കഞ്ഞിയിലായിരിക്കും.

പു.ക.സ.യിലെ അതിബുദ്ധിജീവി സഖാക്കളുടെ വൃത്തമൊപ്പിക്കല്‍കൊണ്ടു് അധികകാലം പിടിച്ചുനില്‍ക്കാനാവില്ലെന്നു് 
പാര്‍ട്ടിനേതൃത്വത്തിനുതന്നെ ബോധ്യമുണ്ടാകണം. സാമൂഹ്യശാസ്ത്രപരമായും സാമ്പത്തികപരമായും 
വ്യാഖ്യാനങ്ങള്‍ ചമയ്ക്കുന്നതില്‍ അവരൊരുപക്ഷെ ഈയെമ്മെസ്സിനെവരെ കടത്തിവെട്ടും. പക്ഷെ വ്യാഖ്യാനങ്ങളെ നേതാക്കളിലേക്കും 
അണികളിലേക്കും പൊതുജനങ്ങളിലേക്കും കമ്യൂണിക്കേറ്റു ചെയ്യാന്‍ പു.ക.സ. സഖാക്കളുടെ അതികഠിന അക്കാദമികഭാഷയും
ഭാവങ്ങളും തടസ്സമാണു്. തത്വശാസ്ത്ര അക്കാദമികരംഗത്തെ പദപ്രയോഗങ്ങളില്‍ മിനിമം അവഗാഹവും പൊതുവായ 
തത്വശാസ്ത്രരീതികളില്‍ അറിവുമില്ലാത്തവര്‍ക്കു് സംസ്കൃതത്തേക്കാളും കഠിനമായേക്കാം പു.ക.സ. ബുദ്ധിജീവികളുടെ വാചക
കസര്‍ത്തുകള്‍. കൂടാതെ അക്കാദമിക് ഇന്റഗ്രിറ്റി അഥവാ ബൗദ്ധികസത്യസന്ധത എന്നൊരു വാള്‍ അവരെ പലപ്പോഴും 
പാര്‍ട്ടിക്കൊരു ബാദ്ധ്യതയാക്കുകയും ചെയ്യും. ഇത് ശരിക്കറിയാവുന്ന ചില സഖാക്കള്‍ ആസ്ഥാനബുദ്ധിജീവിവൃന്ദത്തില്‍ 
തന്റെ പേരുകൂടി ഉള്‍പ്പെടുത്താന്‍ കിണഞ്ഞു പരിശ്രമിക്കുന്നതിന്റെ അനന്തരഫലമായി വേണമെങ്കില്‍ പാര്‍ട്ടിക്കകത്തെ ഇപ്പോഴത്തെ
ആശയസമരത്തെ കാണാം. അതിനിടയില്‍ ചില തലമുതിര്‍ന്ന നേതാക്കള്‍ നടത്തുന്ന ഇടപെടലുകള്‍ മന്ത്രിപ്പണിക്കുശേഷം
പാര്‍ട്ടിയില്‍ പുറമ്പോക്കിലാവാതിരിക്കാനുള്ള വെപ്രാളത്തിന്റെ ബാക്കിപത്രമാണെന്നും സംശയിക്കണം.

പൊതുവേദിയില്‍ രണ്ടുകൂട്ടരേയുംകൂടി ഒരുമിച്ചു് ആക്രമിച്ചു് കീഴ്പെടുത്താന്‍ മാത്രം ശക്തരൊന്നുമല്ല പു.ക.സ. ബുദ്ധിജീവികള്‍. 
ശരീരപ്രകൃതിപോലെത്തന്നെ, ആഞ്ഞ ഒരു കാറ്റില്‍ പാറിപ്പോകാനുള്ളതേയുള്ളു അവരുടെ പൊതുസമ്മതി. എന്നാല്‍ കുറച്ചുകാലംമുമ്പു് 
എതിരാളികളെ നേരിടാന്‍ 'കാറ്റും വെളിച്ചവും കടക്കാന്‍ പാടില്ലാത്ത' പാര്‍ട്ടിയുടെ വക്താക്കള്‍ 
നടത്തിയതരത്തിലുള്ള ശക്തമായ ഇടപെടലുകളുടെ തലത്തിലേക്കു് ഈ സംവാദം പോകുന്നതു് ഒരുപക്ഷേ പു.ക.സ. 
സഖാക്കള്‍ക്കുമപ്പുറമാകാം ലക്ഷ്യമെന്നൊരു സംശയത്തിനും ഇടനല്‍കുന്നു.

എന്തായാലും സന്നാഹങ്ങളുടെ ബാഹുല്യംകൊണ്ടു് ശ്രദ്ധേയമായിത്തീരുന്ന സൈദ്ധാന്തികസമരം, സഖാവു് വി.എസ്സിനൊപ്പമുള്ള 
ബുദ്ധിജീവികള്‍ ആസൂത്രണംചെയ്തു് നടപ്പാക്കിയ 'വിശുദ്ധ വി.എസ്., നശിച്ച പാര്‍ട്ടി' ടൈപ്പ് ചേരിതിരിവുകളിലേക്കു് 
നയിക്കുമോ എന്നു് കാത്തിരുന്നു കാണാം.

പോസ്റ്റ് സ്ക്രിപ്റ്റ്: ലോകത്തുള്ള കമ്യൂണിസ്റ്റും അല്ലാത്തതുമായ ബുദ്ധിജീവികളുടെയും ചിന്തകന്‍മാരുടേയും ഐഡന്റിറ്റി 
പൊളിറ്റിക്സിലുള്ള (സ്വത്വരാഷ്ട്രീയം) രചനകളൊന്നും പോരാഞ്ഞിട്ടായിരിക്കും സഖാവു് ബേബി മേയ് 25നു് 'ദ ഹിന്ദു'വിലെ 
(24ന് 'ദ ന്യൂയോര്‍ക്ക് ടൈംസിലെ') പോള്‍ ക്രൂഗ്‌മാന്റെ കോളത്തിലെ ഐഡന്റിറ്റി പൊളിറ്റിക്സെന്ന പദത്തിലേക്കു് 
റെഫര്‍ ചെയ്തതു്. പോക്കറും പാര്‍ട്ടിയും, എന്തിനു് രാജീവുംവരെ പാര്‍ശ്വവത്കരിക്കപ്പെട്ടവരുടെ സ്വത്വരാഷ്ട്രീയത്തെപ്പറ്റി 
സംസാരിക്കുമ്പോള്‍, റിപ്പബ്ലിക്കന്‍ പാര്‍ട്ടിയും അതിനോടൊട്ടിനില്‍ക്കുന്ന കോര്‍പ്പറേറ്റ് അമേരിക്കയും സെനറ്റ്-കോണ്‍ഗ്രസ്സ് ഇലക്‌ഷനില്‍ 
പ്രയോഗിക്കാന്‍പോകുന്ന മുഖ്യധാരയിലുള്ള "അമേരിക്കനിസം" എന്ന ഐഡന്റിറ്റിയെപ്പറ്റിയാണു് ക്രൂഗ്‌മാന്‍ വാചാലനാവുന്നതു്.

'ഇരകളുടെ മാനിഫെസ്റ്റൊ'യില്‍നിന്നും മുഖ്യധാരാസ്വത്വങ്ങളുടെ രാഷ്ട്രീയമാനിഫെസ്റ്റൊകള്‍ക്കുള്ള വ്യത്യാസംപോലും 
മനസ്സിലാക്കാതെയാണോ ബേബി സഖാവു് സ്വത്വരാഷ്ട്രീയത്തിന്റെ അപകടങ്ങളെക്കുറിച്ചു് വാചാലനായതു്? ബേബിക്കു 
നാണംവന്നില്ലെങ്കിലും ഇതിനു മറുപടിപറയാന്‍ ഒരുപക്ഷേ കെ.ഇ.എന്നിനും പോക്കര്‍ക്കും നാണം കാണും. 
രാജീവ് സഖാവു് പറഞ്ഞതു്, സ്വത്വരാഷ്ട്രീയത്തിന്റെ ചതിക്കുഴികള്‍ പാര്‍ട്ടി സ്റ്റഡിക്ലാസ്സുകളില്‍ അടുത്ത 
കാലത്തായി നല്ലപോലെ വിഷയമായിട്ടുണ്ടെന്നാണു്. സ്റ്റഡിക്ലാസുകളൊന്നും സഖാവു് ബേബിയായിരിക്കില്ല കൈകാര്യം 
ചെയ്തതെന്നും നമുക്ക് പ്രതീക്ഷിക്കാം.

(16 June 2010)\footnote{http://malayal.am/പലവക/പരമ്പര/സ്വത്വം/6150/ബുദ്ധിജീവികളുടെ-സ്വത്വപ്രതിസന്ധി}

\newpage
