\secstar{തന്ത്രങ്ങളുടെ ഇസ്താംബൂള്‍}
\vskip 2pt

­ഫോര്‍­മുല വണ്‍ പോ­രാ­ട്ട­ത്തി­ന്റെ ഏഴാം റൌ­ണ്ടാ­ണ് തുര്‍­ക്കി­യി­ലെ ഇസ്താം­ബുള്‍ പാര്‍­ക്കില്‍ മേ­യ് മു­പ്പ­തി­ന് 
അര­ങ്ങേ­റി­യ­ത്. ഈ റേ­സോ­ടെ ഫോര്‍­മുല വണ്ണി­ന്റെ ആദ്യ യൂ­റോ­പ്യന്‍ പാ­ദം അവ­സാ­നി­ച്ചു. ഇനി കാ­ന­ഡ­യി­ലെ 
ഒരു റേ­സി­നു ശേ­ഷം വലന്‍­സി­യ­യില്‍ ജൂണ്‍ അവ­സാ­ന­വാ­രം നട­ക്കു­ന്ന യൂ­റോ­പ്യന്‍ ഗ്രാന്‍­പ്രീ­യി­ലൂ­ടെ­യാ­ണ് 
യൂ­റോ­പ്പി­ലേ­ക്ക് ചാ­മ്പ്യന്‍­ഷി­പ്പ് പോ­രാ­ട്ട­ങ്ങള്‍ തി­രി­ച്ചു വരു­ന്ന­ത്.

­തുര്‍­ക്കി­യി­ലെ റേ­സ് പല­കാ­ര്യ­ങ്ങള്‍ കൊ­ണ്ടും പ്ര­ത്യേ­കത നി­റ­ഞ്ഞ­താ­ണ്. ആന്റി­ക്ലോ­ക്ക്‌­വൈ­സ് റേ­സും, സീ­സ­ണി­ലെ 
ഏറ്റ­വും വലിയ വള­വാ­യി കണ­ക്കൂ­കൂ­ട്ടു­ന്ന എട്ടാം വള­വും റേ­സി­ന്റെ പ്ര­ത്യേ­ക­ത­യാ­ണ്. മാ­ത്ര­മ­ല്ല, ടയ­റു­കള്‍­ക്ക് കൂ­ടു­തല്‍ 
ആയാ­സം നല്‍­കു­ന്ന റേ­സാ­യ­തി­നാല്‍ പി­റ്റ്സ്റ്റോ­പ് സമ­യ­ങ്ങ­ളും, ടയര്‍ പരി­പാ­ല­ന­വും വള­രെ പ്രാ­ധാ­ന്യ­മര്‍­ഹി­ക്കു­ന്ന 
റേ­സ് കൂ­ടി­യാ­ണി­ത്. തുര്‍­ക്കി­യി­ലെ റേ­സോ­ടു­കൂ­ടി ­ഫോര്‍­മുല വണ്‍ ചാ­മ്പ്യന്‍­ഷി­പ്പി­ലെ മൂ­ന്നി­ലൊ­ന്നു റേ­സു­ക­ളും പൂര്‍­ത്തി­യാ­യി­.

അ­പ­ക­ട­ങ്ങ­ളു­ടെ പര­മ്പ­ര­കൊ­ണ്ടാ­ണ് മോണ്ടേകാര്‍ലോ\footnote{അപകടങ്ങളുടെ മോണ്ടേകാര്‍ലോ} ശ്ര­ദ്ധ­പി­ടി­ച്ചു­പ­റ്റി­യ­തെ­ങ്കില്‍, തുര്‍­ക്കി­യി­ലെ കാ­ര്യം മറി­ച്ചാ­യി­രു­ന്നു. 
ആവേ­ശ­ക­ര­മായ മറി­ക­ട­ക്ക­ലു­ക­ളും, തന്ത്ര­ങ്ങ­ളും മറ്റു­മാ­ണ് ഇസ്താം­ബുള്‍ പാര്‍­ക്കി­ലെ പോ­രാ­ട്ട­ത്തെ അവി­സ്മ­ര­ണീ­യ­മാ­ക്കി­യ­ത്. 
സീ­സ­ണി­ലെ മി­ക­ച്ച പ്ര­ക­ട­ന­ക്കാ­രായ ­റെ­ഡ്ബുള്‍ ഒന്നും മൂ­ന്നും ഗ്രി­ഡ്ഡു­ക­ളി­ലും, മക്‌­ലാ­രന്‍ രണ്ടും നാ­ലും ഗ്രി­ഡ്ഡു­ക­ളി­ലും റേ­സ് 
ആരം­ഭി­ച്ച­പ്പോള്‍, മെ­ഴ്സി­ഡ­സും ഫെ­റാ­രി­യും നി­രാ­ശ­പ്പെ­ടു­ത്തി. അഞ്ചും ആറും സ്ഥാ­ന­ങ്ങ­ളില്‍ നി­ന്ന് ­മെ­ഴ്സി­ഡ­സ് പോ­രാ­ട്ടം 
തു­ട­ങ്ങി­യെ­ങ്കി­ലും, വേ­ഗത ഒരു പ്ര­ശ്ന­മാ­യി മാ­റു­ക­യാ­യി­രു­ന്നു. മു­ന്നില്‍ ഓടിയ റെ­ഡ്ബുള്‍-മക്‌­ലാ­രന്‍ കാ­റു­കള്‍ മാ­ത്ര­മ­ല്ല, 
യോ­ഗ്യ­താ റൌ­ണ്ടില്‍ പി­ന്നി­ലാ­യി­പ്പോയ ഫെ­റാ­രി-റെ­നോ കാ­റു­ക­ളും മെ­ഴ്സി­ഡ­സി­നേ­ക്കാള്‍ വേ­ഗ­മേ­റി­യ­താ­ണെ­ന്ന് 
വ്യ­ക്ത­മാ­യി­കാ­ണാ­മാ­യി­രു­ന്നു­.

ആ­ദ്യ­ലാ­പ്പില്‍ മക്‌­ലാ­ര­ന്റെ നി­ല­വി­ലെ ചാ­മ്പ്യന്‍ ജെന്‍­സണ്‍ ബട്ട­ണെ മറി­ക­ട­ന്ന് ഷു­മാ­ക്കര്‍ നില മെ­ച്ച­പ്പെ­ടു­ത്തി­യെ­ങ്കി­ലും 
വള­രെ വേ­ഗം തന്നെ, വേ­ഗ­മേ­റിയ ബട്ട­ന്റെ കാ­റി­നു മു­മ്പില്‍ അടി­യ­റ­വു പറ­ഞ്ഞു. എങ്കി­ലും റേ­സി­ന്റെ തു­ട­ക്കം മു­തല്‍ റോ­ബര്‍­ട്ട് 
കു­ബി­ത്സ­യെ പി­ന്നില്‍ തള­ച്ചി­ട്ട് ­നി­കൊ റോ­സ്ബര്‍­ഗ് മി­ടു­ക്കു കാ­ട്ടി. നീ­ള­മേ­റിയ സ്ട്രൈ­റ്റു­കള്‍ ഏറെ­യു­ള്ള മോ­ണ്ട്രി­യ­ലില്‍ 
മുന്‍­നിര കാ­റു­കള്‍­ക്ക് വെ­ല്ലു­വി­ളി­യു­യര്‍­ത്ത­ണ­മെ­ങ്കില്‍ മെ­ഴ്സി­ഡ­സ് ഇനി­യും മെ­ച്ച­പ്പെ­ട്ടേ­മ­തി­യാ­കൂ എന്ന് തുര്‍­ക്കി­യില്‍ 
വ്യ­ക്ത­മാ­യി­.

­ചാ­മ്പ്യന്‍­ഷി­പ്പ് പോ­യി­ന്റ് നി­ല­യില്‍ മു­ന്നി­ട്ടു നില്‍­ക്കു­ന്ന മാര്‍­ക്ക് വെ­ബ്ബ­റി­ന് ­ലൂ­യി­സ് ഹാ­മില്‍­ട്ടണ്‍ കടു­ത്ത 
വെ­ല്ലു­വി­ളി­യാ­ണു­യര്‍­ത്തി­യ­ത്.  പതി­നാ­റാം ലാ­പ്പില്‍ പി­റ്റ് സ്റ്റോ­പ്പില്‍ വച്ച് വെ­ബ്ബ­റെ മറി­ക­ട­ക്കാ­മെ­ന്നു കണ­ക്കു­കൂ­ട്ടിയ 
ഹാ­മില്‍­ട്ട­ണെ ഞെ­ട്ടി­ച്ചു കൊ­ണ്ട് വെ­ബ്ബര്‍ മു­ന്നില്‍ കട­ക്കു­ക­യും, ട്രാ­ക്കില്‍ വച്ച് ­സെ­ബാ­സ്റ്റ്യന്‍ വെ­റ്റല്‍ മു­ന്നി­ലെ­ത്തു­ക­യും 
ചെ­യ്ത­ത് തി­രി­ച്ച­ടി­യാ­യി. ആദ്യ റൌ­ണ്ട് പി­റ്റ് സ്റ്റോ­പ്പു­കള്‍ കഴി­ഞ്ഞ­പ്പോള്‍ റെ­ഡ്ബു­ളി­ന്റെ, വെ­ബ്ബര്‍ ഒന്നാ­മ­തും, 
വെ­റ്റല്‍ രണ്ടാ­മ­തു­മാ­യി­രു­ന്നു. തൊ­ട്ടു പി­റ­കില്‍ രണ്ടു മക്‌­ലാ­രന്‍ കാ­റു­ക­ളില്‍ ഹാ­മില്‍­ട്ട­ണും ബട്ട­ണും­.

അ­ഞ്ചാ­മ­തോ­ടി­യി­രു­ന്ന ഷു­മാ­ക്കര്‍ ഏതാ­ണ്ട് മു­പ്പ­തു സെ­ക്ക­ന്റോ­ളം പി­റ­കി­ലാ­യി­രു­ന്ന­ത്, ഷു­മാ­ക്ക­റില്‍ നി­ന്ന് ബട്ടന്‍ 
പൊ­സി­ഷന്‍ തി­രി­ച്ചു പി­ടി­ച്ച­ത് എത്ര­മാ­ത്രം ക്രി­ട്ടി­ക്ക­ലാ­യി­രു­ന്ന മൂ­വാ­യി­രു­ന്നു­വെ­ന്ന് വ്യ­ക്ത­മാ­ക്കി. യാ­നോ ട്രൂ­ലി­യു­ടെ 
ലോ­ട്ട­സ് വഴി­യില്‍ കി­ട­ന്ന­പ്പോള്‍ മു­പ്പ­ത്ത­ഞ്ചാം ലാ­പ്പില്‍ ­യെ­ല്ലോ ഫ്ലാ­ഗ് വന്നെ­ങ്കി­ലും ഒരു­ലാ­പ്പു മാ­ത്ര­മേ നീ­ണ്ടു നി­ന്നു­ള്ളൂ. 
പി­ന്നെ യെ­ല്ലോ ഫ്ലാ­ഗ് വന്ന­ത്, ഒന്നാം സ്ഥാ­ന­ത്തി­നു വേ­ണ്ടി രണ്ടു റെ­ഡ്ബുള്‍ കാ­റു­കള്‍ തമ്മില്‍ ഏറ്റു­മു­ട്ടി­യ­പ്പോ­ഴാ­ണ്. 
സെ­ബാ­സ്റ്റ്യന്‍ വെ­റ്റല്‍ മാര്‍­ക് വെ­ബ്ബ­റെ ഏതാ­ണ്ട് മറി­ക­ട­ന്നെ­ങ്കി­ലും ചെ­റിയ ഒരു പി­ഴ­വില്‍ റണ്ടു കാ­റു­ക­ളും ട്രാ­ക്കില്‍ 
നി­ന്നും പു­റ­ത്തു കട­ക്കു­ക­യും, വെ­റ്റ­ലി­ന്റെ കാര്‍ തരി­പ്പ­ണ­മാ­വു­ക­യും ചെ­യ്തു. വെ­ബ്ബ­റി­ന്റെ കാ­റി­നും ചെ­റിയ തോ­തില്‍ പരി­ക്കു 
പറ്റി, മുന്‍­വി­ങ് മാ­റ്റി വെ­ക്കേ­ണ്ടി വന്നു. അവ­സ­രം മു­ത­ലാ­ക്കി മക്‌­ലാ­രന്‍ കാ­റു­കള്‍ ഒന്നും രണ്ടും സ്ഥാ­ന­ങ്ങള്‍ കൈ­ക്ക­ലാ­ക്കി. 
ബാഴ്സിലോണയിലെപ്പോലെ\footnote{സ്പാനിഷ് ലെഗ്ഗോടെ യൂറോപ്യന്‍ പാദത്തിന് തുടക്കം} അഞ്ചാ­മ­തോ­ടി­യി­രു­ന്ന ­മൈ­ക്കല്‍ ഷു­മാ­ക്കര്‍ ബഹു­ദൂ­രം പി­റ­കി­ലാ­യി­രു­ന്ന­തി­നാല്‍ വെ­ബ്ബ­റി­നു 
പോ­ഡി­യ­വും ചാ­മ്പ്യന്‍­ഷി­പ്പ് ലീ­ഡും നി­ല­നിര്‍­ത്താ­നാ­യി. അവ­സാ­ന­ഘ­ട്ടം വരെ ആദ്യ­പ­ത്തു സ്ഥാ­ന­ങ്ങള്‍­ക്കു പു­റ­ത്താ­യി­രു­ന്ന 
സു­ട്ടില്‍ സൌ­ബ­റി­ന്റെ കൊ­ബി­യാ­ഷി­യെ മറി­ക­ട­ന്ന­തും, റെ­നോ­യു­ടെ വി­റ്റാ­ലി പെ­ട്രോ­വി­നു പറ്റിയ അബ­ദ്ധ­വും ഫോ­ഴ്സ് 
ഇന്ത്യ­ക്ക് രണ്ടു­പോ­യി­ന്റ് നേ­ടി­ക്കൊ­ടു­ത്തു­.

­മൊ­ണാ­കൊ ഗ്രാന്‍­പ്രീ കഴി­ഞ്ഞ­പ്പോള്‍ നാ­ലാ­മ­താ­യി­രു­ന്ന ­ജെന്‍­സണ്‍ ബട്ടണ്‍ 88 പോ­യി­ന്റു­മാ­യി രണ്ട­മ­താ­യാ­ണ് 
തുര്‍­ക്കി­യില്‍ നി­ന്നും പോ­കു­ന്ന­ത്. തുര്‍­ക്കി­യില്‍ മൂ­ന്നാ­മ­തെ­ത്തിയ വെ­ബ്ബര്‍ 93 പോ­യി­ന്റോ­ടെ ചാ­മ്പ്യന്‍­ഷി­പ്പ് ലീ­ഡ് 
നി­ല­നിര്‍­ത്തി­യ­പ്പോള്‍, സീ­സ­ണി­ലെ ആദ്യ ഒന്നാം സ്ഥാ­നം തുര്‍­ക്കില്‍ കര­സ്ഥ­മാ­ക്കിയ ഹാ­മില്‍­ട്ടണ്‍ 84 പോ­യി­ന്റോ­ടെ 
മൂ­ന്നാ­മ­താ­ണ്.

ഒ­രു­പ­ക്ഷേ ചാ­മ്പ്യന്‍­ഷി­പ്പ് ലീ­ഡ് ചെ­യ്യാ­നു­ള്ള ആഗ്ര­ഹ­മൂ­ല­മാ­യി­രി­ക്കാം, ജെന്‍­സണ്‍ ബട്ടണ്‍ നാല്‍­പ്പ­ത്തി­യെ­ട്ടാം ലാ­പ്പില്‍ 
ഹാ­മില്‍­ട്ട­ണെ മറി­ക­ട­ന്നു. എന്നാല്‍ ഉടന്‍­ത­ന്നെ ഒന്നാം­സ്ഥാ­നം തി­രി­ച്ചു­പി­ടി­ച്ച് ഹാ­മില്‍­ട്ടണ്‍ സ്വ­ന്തം നി­ല­പാ­ടു വ്യ­ക്ത­മാ­ക്കി.
ഫെര്‍­ണാ­ണ്ടൊ അലോണ്‍­സൊ­യും സെ­ബാ­സ്റ്റ്യന്‍ വെ­റ്റ­ലും 78ഉം 79ഉം പോ­യി­ന്റു­മാ­യി ഇപ്പോ­ഴും ചാ­മ്പ്യന്‍­മാര്‍­ക്ക് 
ശക്ത­മായ വെ­ല്ലു­വി­ളി­യു­യര്‍­ത്തു­ന്നു­ണ്ടെ­ങ്കി­ലും, റോ­ബര്‍­ട്ട് കു­ബി­ത്സ (67),­ഫെ­ലി­പെ മസ്സ (67), നി­കൊ റൊ­സ്ബര്‍­ഗ് (66)
എന്നി­വര്‍­ക്ക് ചാ­മ്പ്യന്‍­ഷി­പ്പ് പോ­രാ­ട്ട­ത്തില്‍ നി­ല­നില്‍­ക്ക­ണ­മെ­ങ്കില്‍ പോ­ഡി­യ­ങ്ങ­ളും ഒന്നാം­സ്ഥാ­ന­ങ്ങ­ളും കൂ­ടി­യേ­തീ­രു. 
നിര്‍­മ്മാ­താ­ക്ക­ളു­ടെ കാ­ര്യ­ത്തില്‍ മക്‌­ലാ­ര­നും ­(172) റെ­ഡ്ബു­ളും ­(171) തമ്മില്‍ വെ­റും ഒരു പോ­യി­ന്റ് വ്യ­ത്യാ­സ­മേ­യു­ള്ളൂ.
എന്നാല്‍ മൂ­ന്നാം സ്ഥാ­ന­ത്തു­ള്ള ­ഫെ­റാ­രി­ 25 പോ­യി­ന്റി­നു പി­റ­കി­ലാ­ണ്.

­മെ­ഴ്സി­ഡ­സ് ടീം പ്രിന്‍­സി­പ്പാള്‍ റോ­സ് ബ്രാ­വ്‌ണ്‍ പറ­ഞ്ഞ­ത്, അടു­ത്ത മൂ­ന്നു­നാ­ലു റേ­സു­കള്‍­ക്ക­കം ഈ സീ­സ­ണി­ലെ പു­തിയ
ഡെ­വ­ല­പ്മെ­ന്റു­കള്‍ നിര്‍­ത്തി­വെ­ച്ച് അടു­ത്ത സീ­സ­ണില്‍ ശ്ര­ദ്ധ­കേ­ന്ദ്രീ­ക­രി­ക്കു­മെ­ന്നാ­ണ്. ഈ സീ­സ­ണില്‍ ഇനി­യും മി­ക­ച്ച 
റി­സല്‍­ട്ടു­കള്‍ ലഭി­ക്ക­ണ­മെ­ങ്കില്‍ ആദ്യം കാ­റി­ന്റെ വേ­ഗം വര്‍­ദ്ധി­പ്പി­ക്കാ­നാ­വ­ണ­മെ­ന്നും അദ്ദേ­ഹം ഊന്നി­പ്പ­റ­ഞ്ഞി­രു­ന്നു­.

­മോ­ണ്ട്രി­യാ­ലി­ലും അത്ത­ര­ത്തില്‍ ചാ­മ്പ്യന്‍­ഷി­പ്പ് വേ­ഗം കണ്ടെ­ത്താ­നാ­യി­ല്ലെ­ങ്കില്‍ മെ­ഴ്സി­ഡ­സ് ഈ വര്‍­ഷ­ത്തെ മത്സ­രം 
ഒഴി­വാ­ക്കി അടു­ത്ത വര്‍­ഷ­ത്തെ കാ­റില്‍ ശ്ര­ദ്ധ­കേ­ന്ദ്രീ­ക­രി­ക്കാ­നാ­ണ് സാ­ധ്യ­ത. ഇത് റെ­നോ­ക്കും റോ­ബര്‍­ട്ട് കു­ബി­ത്സ­യ്ക്കും, 
ഫോ­ഴ്സ് ഇന്ത്യ­ക്കും നല്ല വാര്‍­ത്ത­യാ­ണ്. എന്നാല്‍ യോ­ഗ്യ­താ റൌ­ണ്ടില്‍ ഷു­മാ­ക്ക­റി­നു പി­ന്നില്‍ പെ­ട്ടു­പോ­കു­ന്ന 
ചാ­മ്പ്യന്‍­ഷി­പ്പ് മോ­ഹി­കള്‍­ക്കു മോ­ശം വാര്‍­ത്ത­യും. എത്ര വേ­ഗം കു­റ­ഞ്ഞ കാ­റി­ലാ­യാ­ലും, ട്രാ­ക്കില്‍ ഷു­മാ­ക്ക­റെ മറി­ക­ട­ക്കാന്‍
സാ­ധാ­രണ അട­വു­ക­ളൊ­ന്നും പോ­രെ­ന്ന­തു കൊ­ണ്ടാ­ണ­ത്. മൊ­ണാ­കൊ­യില്‍ കണ്ട­തു പോ­ലെ, ഒരു പോ­യി­ന്റി­നു വേ­ണ്ടി 
മു­ഴു­വന്‍ റേ­സും കള­യാന്‍ മടി­യി­ല്ലാ­ത്ത­വ­നാ­ണ് ഷു­മാ­ക്കര്‍. അട­വു­ക­ളു­ടെ രാ­ജാ­വും­.

­വാല്‍­ക്ക­ഷ­ണം­: പു­സി­ക്യാ­റ്റ്ഡോള്‍­സ് ഗ്രൂ­പ്പി­ലെ പാ­ട്ടു­കാ­രി­യും മുന്‍ ലോ­ക­ചാ­മ്പ്യന്‍ ലൂ­യി­സ് ഹാ­മില്‍­ട്ട­ണി­ന്റെ 
കാ­മു­കി­യു­മായ ­നി­കോള്‍ ഷെര്‍­സി­ങ്ങര്‍ 'ഡാന്‍­സി­ങ് വി­ത് സ്റ്റാര്‍­സ്' ഷോ­യില്‍ ജേ­ത്രി­യാ­യ­ത് മേ­യ് 25­നാ­ണ്, 
ഞാ­യ­റാ­ഴ്ച ലൂ­യി­സ് ഹാ­മില്‍­ട്ടണ്‍ തുര്‍­ക്കി­യി­ലും വി­ജ­യി­ച്ചു. ദമ്പ­തി­കള്‍­ക്ക് ഇപ്പോള്‍ ശു­ക്ര­നാ­ണെ­ന്നു തോ­ന്നു­ന്നു :)

(2 June 2010)\footnote{http://malayal.am/വിനോദം/കായികം/5784/തന്ത്രങ്ങളുടെ-ഇസ്താംബൂള്‍}

\newpage
