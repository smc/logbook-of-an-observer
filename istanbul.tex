\secstar{തന്ത്രങ്ങളുടെ ഇസ്താംബൂള്‍}
\vskip 2pt

ഫോര്‍മുല വണ്‍ പോരാട്ടത്തിന്റെ ഏഴാം റൌണ്ടാണു് തുര്‍ക്കിയിലെ ഇസ്താംബുള്‍ പാര്‍ക്കില്‍ മേയ് മുപ്പതിനു് 
അരങ്ങേറിയതു്. ഈ റേസോടെ ഫോര്‍മുല വണ്ണിന്റെ ആദ്യ യൂറോപ്യന്‍പാദം അവസാനിച്ചു. ഇനി കാനഡയിലെ 
ഒരു റേസിനുശേഷം വലന്‍സിയയില്‍ ജൂണ്‍ അവസാനവാരം നടക്കുന്ന യൂറോപ്യന്‍ ഗ്രാന്‍പ്രീയിലൂടെയാണു് 
യൂറോപ്പിലേക്കു് ചാമ്പ്യന്‍ഷിപ്പു് പോരാട്ടങ്ങള്‍ തിരിച്ചുവരുന്നതു്.

തുര്‍ക്കിയിലെ റേസ് പലകാര്യങ്ങള്‍കൊണ്ടും പ്രത്യേകത നിറഞ്ഞതാണു്. ആന്റിക്ലോക്ക്‌വൈസ് റേസും, സീസണിലെ 
ഏറ്റവും വലിയ വളവായി കണക്കാക്കുന്ന എട്ടാം വളവും റേസിന്റെ പ്രത്യേകതയാണു്. മാത്രമല്ല, ടയറുകള്‍ക്കു് കൂടുതല്‍ 
ആയാസം നല്‍കുന്ന റേസായതിനാല്‍ പിറ്റ്സ്റ്റോപ് സമയങ്ങളും ടയര്‍ പരിപാലനവും വളരെ പ്രാധാന്യമര്‍ഹിക്കുന്ന 
റേസ് കൂടിയാണിതു്. തുര്‍ക്കിയിലെ റേസോടുകൂടി ഫോര്‍മുല വണ്‍ ചാമ്പ്യന്‍ഷിപ്പിലെ മൂന്നിലൊന്നു റേസുകളും പൂര്‍ത്തിയായി.

അപകടങ്ങളുടെ പരമ്പരകൊണ്ടാണു് മോണ്ടേകാര്‍ലോ\footnote{അപകടങ്ങളുടെ മോണ്ടേകാര്‍ലോ} ശ്രദ്ധപിടിച്ചുപറ്റിയതെങ്കില്‍, 
തുര്‍ക്കിയിലെ കാര്യം മറിച്ചായിരുന്നു. ആവേശകരമായ മറികടക്കലുകളും തന്ത്രങ്ങളും 
മറ്റുമാണു് ഇസ്താംബുള്‍ പാര്‍ക്കിലെ പോരാട്ടത്തെ അവിസ്മരണീയമാക്കിയതു്. 
സീസണിലെ മികച്ച പ്രകടനക്കാരായ റെഡ്ബുള്‍ ഒന്നും മൂന്നും ഗ്രിഡ്ഡുകളിലും, മക്‌ലാരന്‍ രണ്ടും നാലും ഗ്രിഡ്ഡുകളിലും റേസ് 
ആരംഭിച്ചപ്പോള്‍, മെഴ്സിഡസും ഫെറാരിയും നിരാശപ്പെടുത്തി. അഞ്ചും ആറും സ്ഥാനങ്ങളില്‍നിന്നു് മെഴ്സിഡസ് പോരാട്ടം 
തുടങ്ങിയെങ്കിലും, വേഗത ഒരു പ്രശ്നമായി മാറുകയായിരുന്നു. മുന്നില്‍ ഓടിയ റെഡ്ബുള്‍-മക്‌ലാരന്‍ കാറുകള്‍ മാത്രമല്ല, 
യോഗ്യതാറൌണ്ടില്‍ പിന്നിലായിപ്പോയ ഫെറാരി-റെനോ കാറുകളും മെഴ്സിഡസിനേക്കാള്‍ വേഗമേറിയതാണെന്നു് 
വ്യക്തമായി കാണാമായിരുന്നു.

ആദ്യലാപ്പില്‍ മക്‌ലാരന്റെ നിലവിലെ ചാമ്പ്യന്‍ ജെന്‍സണ്‍ ബട്ടണെ മറികടന്നു് ഷുമാക്കര്‍ നില മെച്ചപ്പെടുത്തിയെങ്കിലും 
വളരെ വേഗംതന്നെ, വേഗമേറിയ ബട്ടന്റെ കാറിനു മുമ്പില്‍ അടിയറവു പറഞ്ഞു. എങ്കിലും റേസിന്റെ തുടക്കംമുതല്‍ റോബര്‍ട്ടു് 
കുബിത്സയെ പിന്നില്‍ തളച്ചിട്ടു് നികൊ റോസ്ബര്‍ഗ് മിടുക്കു കാട്ടി. നീളമേറിയ സ്ട്രൈറ്റുകള്‍ ഏറെയുള്ള മോണ്ട്രിയലില്‍ 
മുന്‍നിര കാറുകള്‍ക്കു് വെല്ലുവിളിയുയര്‍ത്തണമെങ്കില്‍ മെഴ്സിഡസ് ഇനിയും മെച്ചപ്പെട്ടേമതിയാകൂ എന്നു് തുര്‍ക്കിയില്‍ 
വ്യക്തമായി.

ചാമ്പ്യന്‍ഷിപ്പ് പോയിന്റ് നിലയില്‍ മുന്നിട്ടുനില്‍ക്കുന്ന മാര്‍ക്ക് വെബ്ബറിനു് ലൂയിസ് ഹാമില്‍ട്ടണ്‍ കടുത്ത 
വെല്ലുവിളിയാണുയര്‍ത്തിയതു്.  പതിനാറാം ലാപ്പില്‍ പിറ്റ് സ്റ്റോപ്പില്‍വച്ചു് വെബ്ബറെ മറികടക്കാമെന്നു കണക്കുകൂട്ടിയ 
ഹാമില്‍ട്ടണെ ഞെട്ടിച്ചുകൊണ്ടു് വെബ്ബര്‍ മുന്നില്‍ കടക്കുകയും, ട്രാക്കില്‍വച്ചു് സെബാസ്റ്റ്യന്‍ വെറ്റല്‍ മുന്നിലെത്തുകയും 
ചെയ്തതു് തിരിച്ചടിയായി. ആദ്യ റൌണ്ട് പിറ്റ് സ്റ്റോപ്പുകള്‍ കഴിഞ്ഞപ്പോള്‍ റെഡ്ബുള്ളിന്റെ വെബ്ബര്‍ ഒന്നാമതും, 
വെറ്റല്‍ രണ്ടാമതുമായിരുന്നു. തൊട്ടുപിറകില്‍ രണ്ടു മക്‌ലാരന്‍ കാറുകളില്‍ ഹാമില്‍ട്ടണും ബട്ടണും.

അഞ്ചാമതോടിയിരുന്ന ഷുമാക്കര്‍ ഏതാണ്ടു് മുപ്പതു സെക്കന്റോളം പിറകിലായിരുന്നതു്, അദ്ദേഹത്തില്‍നിന്നു് ബട്ടന്‍ 
പൊസിഷന്‍ തിരിച്ചുപിടിച്ചതു് എത്രമാത്രം ക്രിട്ടിക്കലായിരുന്ന മൂവായിരുന്നുവെന്നു് വ്യക്തമാക്കി. യാനോ ട്രൂലിയുടെ 
ലോട്ടസ് വഴിയില്‍ കിടന്നപ്പോള്‍ മുപ്പത്തഞ്ചാം ലാപ്പില്‍ യെല്ലോ ഫ്ലാഗ് വന്നെങ്കിലും ഒരു ലാപ്പുമാത്രമേ നീണ്ടുനിന്നുള്ളൂ. 
പിന്നീടു് യെല്ലോ ഫ്ലാഗ് വന്നതു് ഒന്നാംസ്ഥാനത്തിനുവേണ്ടി രണ്ടു റെഡ്ബുള്‍ കാറുകള്‍ തമ്മില്‍ ഏറ്റുമുട്ടിയപ്പോഴാണു്. 
മാര്‍ക് വെബ്ബറെ സെബാസ്റ്റ്യന്‍ വെറ്റല്‍ ഏതാണ്ടു് മറികടന്നെങ്കിലും ചെറിയ ഒരു പിഴവില്‍ രണ്ടു കാറുകളും ട്രാക്കില്‍നിന്നും 
പുറത്തുകടക്കുകയും, വെറ്റലിന്റെ കാര്‍ തരിപ്പണമാവുകയും ചെയ്തു. വെബ്ബറിന്റെ കാറിനും ചെറിയ തോതില്‍ പരിക്കു 
പറ്റി മുന്‍വിങ് മാറ്റിവെക്കേണ്ടിവന്നു. അവസരം മുതലാക്കി മക്‌ലാരന്‍ കാറുകള്‍ ഒന്നും രണ്ടും സ്ഥാനങ്ങള്‍ കൈക്കലാക്കി. 
ബാഴ്സിലോണയിലെപ്പോലെ\footnote{സ്പാനിഷ് ലെഗ്ഗോടെ യൂറോപ്യന്‍പാദത്തിനു് തുടക്കം} അഞ്ചാമതോടിയിരുന്ന മൈക്കല്‍ ഷുമാക്കര്‍ ബഹുദൂരം 
പിറകിലായിരുന്നതിനാല്‍ വെബ്ബറിനു പോഡിയവും ചാമ്പ്യന്‍ഷിപ്പു് ലീഡും 
നിലനിര്‍ത്താനായി. അവസാനഘട്ടംവരെ ആദ്യപത്തു സ്ഥാനങ്ങള്‍ക്കു പുറത്തായിരുന്ന 
സുട്ടില്‍ സൌബറിന്റെ കൊബിയാഷിയെ മറികടന്നതും, റെനോയുടെ വിറ്റാലി പെട്രോവിനു പറ്റിയ അബദ്ധവും ഫോഴ്സ് 
ഇന്ത്യക്കു് രണ്ടുപോയിന്റ് നേടിക്കൊടുത്തു.

മൊണാകൊ ഗ്രാന്‍പ്രീ കഴിഞ്ഞപ്പോള്‍ നാലാമതായിരുന്ന ജെന്‍സണ്‍ ബട്ടണ്‍ 88 പോയിന്റുമായി രണ്ടാമതായാണു് 
തുര്‍ക്കിയില്‍നിന്നും പോകുന്നതു്. തുര്‍ക്കിയില്‍ മൂന്നാമതെത്തിയ വെബ്ബര്‍ 93 പോയിന്റോടെ ചാമ്പ്യന്‍ഷിപ്പ് ലീഡ് 
നിലനിര്‍ത്തിയപ്പോള്‍, സീസണിലെ ആദ്യ ഒന്നാംസ്ഥാനം തുര്‍ക്കില്‍ കരസ്ഥമാക്കിയ ഹാമില്‍ട്ടണ്‍ 84 പോയിന്റോടെ 
മൂന്നാമതാണു്.

ഒരുപക്ഷേ ചാമ്പ്യന്‍ഷിപ്പ് ലീഡ് ചെയ്യാനുള്ള ആഗ്രഹമൂലമായിരിക്കാം, ജെന്‍സണ്‍ ബട്ടണ്‍ നാല്‍പ്പത്തിയെട്ടാം ലാപ്പില്‍ 
ഹാമില്‍ട്ടണെ മറികടന്നതു്. എന്നാല്‍ ഉടന്‍തന്നെ ഒന്നാംസ്ഥാനം തിരിച്ചുപിടിച്ചു് ഹാമില്‍ട്ടണ്‍ സ്വന്തംനിലപാടു വ്യക്തമാക്കി.
ഫെര്‍ണാണ്ടൊ അലോണ്‍സൊയും സെബാസ്റ്റ്യന്‍ വെറ്റലും 78ഉം 79ഉം പോയിന്റുമായി ഇപ്പോഴും ചാമ്പ്യന്‍മാര്‍ക്കു് 
ശക്തമായ വെല്ലുവിളിയുയര്‍ത്തുന്നുണ്ടെങ്കിലും, റോബര്‍ട്ട് കുബിത്സ (67), ഫെലിപെ മസ്സ (67), നികൊ റൊസ്ബര്‍ഗ് (66)
എന്നിവര്‍ക്കു് ചാമ്പ്യന്‍ഷിപ്പ് പോരാട്ടത്തില്‍ നിലനില്‍ക്കണമെങ്കില്‍ പോഡിയങ്ങളും ഒന്നാംസ്ഥാനങ്ങളും കൂടിയേതീരൂ. 
നിര്‍മ്മാതാക്കളുടെ കാര്യത്തില്‍ മക്‌ലാരനും (172) റെഡ്ബുള്ളും (171) തമ്മില്‍ വെറും ഒരു പോയിന്റ് വ്യത്യാസമേയുള്ളു.
എന്നാല്‍ മൂന്നാംസ്ഥാനത്തുള്ള ഫെറാരി 25 പോയിന്റിനു പിറകിലാണു്.

മെഴ്സിഡസ് ടീം പ്രിന്‍സിപ്പാള്‍ റോസ് ബ്രാവ്‌ണ്‍ പറഞ്ഞതു്, അടുത്ത മൂന്നുനാലു റേസുകള്‍ക്കകം ഈ സീസണിലെ പുതിയ
ഡെവലപ്മെന്റുകള്‍ നിര്‍ത്തിവെച്ചു് അടുത്ത സീസണില്‍ ശ്രദ്ധകേന്ദ്രീകരിക്കുമെന്നാണു്. ഈ സീസണില്‍ ഇനിയും മികച്ച 
റിസല്‍ട്ടുകള്‍ ലഭിക്കണമെങ്കില്‍ ആദ്യം കാറിന്റെ വേഗം വര്‍ദ്ധിപ്പിക്കാനാവണമെന്നും അദ്ദേഹം ഊന്നിപ്പറഞ്ഞിരുന്നു.

മോണ്ട്രിയാലിലും അത്തരത്തില്‍ ചാമ്പ്യന്‍ഷിപ്പ് വേഗം കണ്ടെത്താനായില്ലെങ്കില്‍ മെഴ്സിഡസ് ഈ വര്‍ഷത്തെ മത്സരം 
ഒഴിവാക്കി അടുത്തവര്‍ഷത്തെ കാറില്‍ ശ്രദ്ധകേന്ദ്രീകരിക്കാനാണു് സാധ്യത. ഇതു് റെനോക്കും റോബര്‍ട്ട് കുബിത്സയ്ക്കും 
ഫോഴ്സ് ഇന്ത്യക്കും നല്ല വാര്‍ത്തയാണു്. എന്നാല്‍ യോഗ്യതാ റൌണ്ടില്‍ ഷുമാക്കറിനു പിന്നില്‍ പെട്ടുപോകുന്ന 
ചാമ്പ്യന്‍ഷിപ്പ് മോഹികള്‍ക്കു മോശം വാര്‍ത്തയും. എത്ര വേഗംകുറഞ്ഞ കാറിലായാലും, ട്രാക്കില്‍ ഷുമാക്കറെ മറികടക്കാന്‍
സാധാരണ അടവുകളൊന്നും പോരെന്നതു കൊണ്ടാണതു്. മൊണാകൊയില്‍ കണ്ടതുപോലെ, ഒരു പോയിന്റിനു വേണ്ടി 
മുഴുവന്‍ റേസും കളയാന്‍ മടിയില്ലാത്തവനാണു് ഷുമാക്കര്‍. അടവുകളുടെ രാജാവും.

വാല്‍ക്കഷണം: പുസിക്യാറ്റ് ഡോള്‍സ് ഗ്രൂപ്പിലെ പാട്ടുകാരിയും മുന്‍ലോകചാമ്പ്യന്‍ ലൂയിസ് ഹാമില്‍ട്ടണിന്റെ 
കാമുകിയുമായ നികോള്‍ ഷെര്‍സിങര്‍ 'ഡാന്‍സിങ് വിത് സ്റ്റാര്‍സ് ' ഷോയില്‍ ജേത്രിയായതു് മേയ് 25നാണു്, 
ഞായറാഴ്ച ലൂയിസ് ഹാമില്‍ട്ടണ്‍ തുര്‍ക്കിയിലും വിജയിച്ചു. ദമ്പതികള്‍ക്ക് ഇപ്പോള്‍ ശുക്രനാണെന്നു തോന്നുന്നു :)

(2 June 2010)\footnote{http://malayal.am/വിനോദം/കായികം/5784/തന്ത്രങ്ങളുടെ-ഇസ്താംബൂള്‍}

\newpage
