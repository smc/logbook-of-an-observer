\section*{Politics in campus: the need of time}
\vskip 2pt
I have read a lot, seen a lot and heard a lot about campus politics. 
When thinking with the open mind of student who is bothered about 
the situation of the country today, i have a small comment on the campus politics.

I just completed my professional course from one of the most criticized campuses of 
today in the name of campus politics. So I guess I am eligible to propose a view on the issue. 
Banning politics in campus is just like banning internet for the thousands of websites with 
so called inappropriate content.


What I am telling is, the most important part of the country’s administration system is legislature. 
It is where the decisions are taken and new rules are made. Other pillars of democracy, 
like executive,judiciary and media; does the job of implementing and cross checking the decisions 
made by legislature. So by banning the politics in campuses what is todays society aiming at? Creating an 
uneducated legislature, so that it acts as a spoon in the arms of the enemies of the state?


The main problem authorities raises about the campus politics is the violence in the process. The campus
is mostly youngsters between 17 and 25, the hot and energetic period, in malayalam ചോര തിളയ്ക്കുന്ന പ്രായം. 
It is the duty of society and authorities to control law and order situation. If students are able to think about
parents when any problem comes, then they will never go to make any violence in campus. Also if somebody wants 
to make problems in campus, he will make that irrespective of politics. It is the duty of society to provide good
political basis, election mechanism and facilities. It is never the right solution to ban politics in campus just 
because the society and authorities fail to implement it in the right way. 


Kept in anyway, campus politics acts as a direct impression of what happens in the original society. 
Is it because of that authorities opposes campus politics? I don’t know. I can’t express all I wanted. 
Please post your comments. Then I guess I will be able to open up better. 
(July 15, 2007)
\newpage
