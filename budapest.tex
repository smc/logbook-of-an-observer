\secstar{ബുഡാപെസ്റ്റിലെ തണുപ്പന്‍ കാറോട്ടം}
\vskip 2pt

ആഗസ്റ്റ് ഒന്നിനു് ബുഡാപെസ്റ്റില്‍ നടന്ന ഹംഗേറിയന്‍ ഗ്രാന്‍പ്രീയോടെ ഫോര്‍മുലാ വണ്‍ 2010 സീസണിലെ 12 റേസുകള്‍ക്കു് തീരുമാനമായി. ശക്തമായ ചില പോരാട്ടങ്ങള്‍ ട്രാക്കിലുണ്ടായെങ്കിലും, ഹംഗറിയിലെ ചൂടുള്ള ട്രാക്കില്‍ വളരെ തണുത്ത പോരാട്ടമായിരുന്നു അരങ്ങേറിയതു്. മാര്‍ക് വെബ്ബറും റെഡ്ബുള്ളും മക്‌ലാരനില്‍നിന്നു് പോയിന്റ് നിലയില്‍ ലീഡ് തിരിച്ചുപിടിച്ചു് ശക്തി തെളിയിച്ചു. ജര്‍മനിയിലെ അത്ര ശക്തമായിരുന്നില്ലെങ്കിലും രണ്ടും നാലും സ്ഥാനങ്ങളിലെത്തി ഫെറാരിയും കരുത്തു കാണിച്ചു.

യോഗ്യതാറൗണ്ടില്‍ ഏഴാംതവണയും സെബാസ്റ്റ്യന്‍ വെറ്റല്‍ പോള്‍ നേടിയപ്പോള്‍, നാലുതവണ പോള്‍ നേടിയ 
വെബ്ബര്‍ രണ്ടാമതെത്തി. സീസണില്‍ ആറാം തവണയാണു്, ഗ്രിഡ്ഡിലെ മുന്‍നിര റെഡ്ബുള്‍ സ്വന്തമാക്കിയതു്. 
തൊട്ടുപിന്നില്‍ ഫെറാരികള്‍ അലോണ്‍സൊയുടെ നേതൃത്വത്തില്‍ അണിനിരന്നപ്പോള്‍ നിലവിലെ ചാമ്പ്യന്‍ ബട്ടണ്‍ 
മൂന്നാംപാദം കണ്ടില്ല. ഫോഴ്സ് ഇന്ത്യയുടെ കാറുകള്‍ തുടര്‍ച്ചയായി രണ്ടാംതവണയും മൂന്നാംപാദത്തിലെത്തുന്നതില്‍ 
പരാജയപ്പെട്ടു. വിറ്റാലി പെട്രോവ് സീസണില്‍ ആദ്യമായി സഹ റെനോഡ്രൈവര്‍ കുബിത്സയ്ക്കു മുന്നില്‍ 
യോഗ്യതനേടിയപ്പോള്‍ പെഡ്രോ ഡി ലാ റൊസയും നികൊ ഹള്‍ക്കെന്‍ബെര്‍ഗും മൂന്നാംപാദത്തിലെത്തി സൗബര്‍, 
വില്യംസ് ടീമുകള്‍ ശക്തമായ മദ്ധ്യനിര സാന്നിദ്ധ്യമാണെന്നു തെളിയിച്ചു. ഇന്ത്യന്‍ ഡ്രൈവര്‍ കരണ്‍ ചന്ദോക്കിനു് 
ഇത്തവണയും അവസരം കിട്ടിയില്ല.

ആദ്യലാപ്പില്‍ ശക്തമായ സ്റ്റാര്‍ട്ടിലൂടെ ഫെറാരികള്‍ റെഡ്ബുള്ളുകളുടെമേല്‍ ചെറിയ ആധിപത്യം നേടി. അലോണ്‍സൊ 
വെബ്ബറിനെ മറികടക്കുകയും വെറ്റലിനു് വളരെ അടുത്തെത്തുകയും ചെയ്തപ്പോള്‍ ആദ്യവളവിനുമുമ്പു് ഉള്‍വശത്തുകൂടെ 
വെബ്ബറെ മറികടക്കാനുള്ള മസ്സയുടെ ശ്രമം പാളിപ്പോയി. എന്നാല്‍ ഏഴാമതുനിന്നു് റൊസ്ബര്‍ഗിനേയും ഹാമില്‍ട്ടണേയും
മറികടന്നു് അഞ്ചാംസ്ഥാനത്തെത്തിയ പെട്രോവ് തന്റെ കഴിവു് പുറത്തെടുത്തു. മൈക്കല്‍ ഷുമാക്കറാവട്ടെ, മറ്റൊരു 
ശക്തമായ സ്റ്റാര്‍ട്ടിലൂടെ പുതിയ തന്റെ അവതാരം ഇപ്പോള്‍ നല്ല സ്റ്റാര്‍ട്ടറാണെന്നു കാണിച്ചുതന്നു. മികച്ച ഫ്ലൈയിങ് 
ലാപ്പുകളും പിറ്റ് സ്റ്റോപു്/ടയര്‍ ഓപ്ഷന്‍ തീരുമാനങ്ങളും എടുത്തിരുന്ന പഴയ സ്വരൂപം കൂടി തിരിച്ചെടുക്കാനായാലെ പക്ഷേ 
ഷുമാക്കര്‍ക്കു് രക്ഷയുള്ളൂ. ആദ്യലാപ്പില്‍ തന്റെ മുന്നില്‍ കടന്നെങ്കിലും രണ്ടാംലാപ്പില്‍ പെട്രോവിന്റെ പരിചയക്കുറവു് 
മുതലെടുത്തു് ഹാമില്‍ട്ടണ്‍ അഞ്ചാംസ്ഥാനത്തു് തിരിച്ചെത്തി. റേസിലെ ആദ്യ റിട്ടയര്‍മെന്റ് ടോറോ റോസോയുടെ 
ജെയ്മി അല്‍ഗ്യുസാരിയുടേതായിരുന്നു. എന്‍ജിന്‍ പ്രശ്നം കാരണമായിരുന്നു വിരമിക്കല്‍.

നല്ല ചൂടുള്ള ട്രാക്കു് സൂപ്പര്‍ സോഫ്റ്റ് ടയറുകള്‍ക്കു് കൂടുതല്‍ ആയുസ്സുനല്‍കിയതു് വിരസമായ ഒരു റേസിനു് 
പ്രധാനകാരണമായെന്നു വേണമെങ്കില്‍ പറയാം. ടയറുകളോ അപകടങ്ങളിലൂടെയുണ്ടായ അപ്രവചനീയതയോ ആണു് 
സീസണിലെ മികച്ചതെന്നു പറയാവുന്ന റേസുകള്‍ക്കു് വഴിയൊരുക്കിയതു്. ഇവിടെയും പതിനഞ്ചാം ലാപ്പില്‍ ബട്ടന്റെയും 
ലിയുസ്സിയുടെയും കാറുകള്‍ തമ്മിലുരസുകയും അതിനുശേഷം തുടരെത്തുടരെ കാറുകള്‍ പിറ്റ് ചെയ്യുകയും ചെയ്തപ്പോള്‍ 
ഹാമില്‍ട്ടണ്‍ മസ്സയെ മറികടക്കുകയും, പിറ്റില്‍ നടന്ന ബഹളത്തില്‍ അപകടത്തിലൂടെ സുട്ടിലും റൊസ്ബര്‍ഗും 
വിരമിക്കുകയും, കുബിത്സയുടെ പോയിന്റ് പ്രതീക്ഷകള്‍ അസ്തമിക്കുകയും ചെയ്തെങ്കിലും നല്ല മൈലേജ് നല്‍കിയ സോഫ്റ്റ് 
ടയറുകള്‍ മാര്‍ക് വെബ്ബര്‍ക്കു് ലീഡ് നേടിക്കൊടുക്കുകയായിരുന്നു. എല്ലാവരും സേഫ്റ്റികാര്‍ ഇറങ്ങുന്നതിനുമുമ്പു് പിറ്റ് 
ചെയ്തപ്പോള്‍ വെബ്ബറിനും ബാരിക്കെല്ലോക്കും അതിനു കഴിഞ്ഞില്ല. തുടര്‍ന്നു് പ്രതിരോധത്തിലായ വെബ്ബറെ സഹായിക്കാന്‍ 
വെറ്റല്‍ മറ്റുകാറുകളെ സേഫ്റ്റികാറിനുപിന്നില്‍ പത്തു് കാര്‍ ദൂരത്തിനുമപ്പുറം തളച്ചിട്ടു. ഇതിനു പിന്നീടു് വെറ്റലിനു് ഡ്രൈവ് 
ത്രൂ പെനാല്‍ട്ടി ലഭിച്ചു.

സേഫ്റ്റികാര്‍ പിന്മാറിയശേഷം വെബ്ബര്‍ സോഫ്റ്റ് ടയറുകളുടെ ആനുകൂല്യവും ഹംഗറിയില്‍ ഫെറാരിക്കുമേല്‍ കണ്ടെത്തിയ 
വേഗവും മുതലെടുത്തു് ലീഡ് വര്‍ദ്ധിപ്പിക്കുന്നതില്‍ ശ്രദ്ധചെലുത്തി. ഇതിനിടയില്‍ മക്‌ലാരന്റെ നിരാശയുടെ ആഴം 
വര്‍ദ്ധിപ്പിച്ചുകൊണ്ടു് ഹാമില്‍ട്ടണ്‍ ട്രാന്‍സ്‌മിഷന്‍ പ്രശ്നവുമായി ഇരുപത്തിനാലാം ലാപ്പില്‍ വിരമിച്ചു. പിറ്റ് സ്റ്റോപ്പിലെ 
അപകടത്തിനു് 10 സെക്കന്റ് സ്റ്റോപ് ഗോ ശിക്ഷയുംകൂടി ലഭിച്ച കുബിത്സ അവസാനം ഇരുപത്തിയാറാം ലാപ്പില്‍ 
റേസ് അവസാനിപ്പിച്ചു. വെബ്ബര്‍ ലീഡ് വര്‍ദ്ധിപ്പിച്ചു കൊണ്ടിരുന്നെങ്കിലും, ഡ്രൈവ് ത്രൂ വെറ്റലിനു്
നിശ്ചയമായിരുന്ന രണ്ടാംസ്ഥാനം നഷ്ടമാക്കി. അവസാനം നാല്‍പ്പത്തിനാലാം ലാപ്പില്‍ പിറ്റ് ചെയ്യുമ്പോള്‍ രണ്ടാംസ്ഥാനത്തുണ്ടായിരുന്ന 
ഫെറാരിയുടെ അലോണ്‍സൊയുടെമേല്‍ വെബ്ബറിനു് 23.7 സെക്കന്റ് ലീഡ് ഉണ്ടായിരുന്നു.

എന്നാല്‍ പ്രൈം ടയറുകളില്‍ റേസ് തുടങ്ങുകയും പെട്രോവില്‍നിന്നും ഹള്‍ക്കെന്‍ബെര്‍ഗില്‍നിന്നും ശക്തമായ സമ്മര്‍ദ്ദം 
നേരിടുകയും ചെയ്ത ബാരിക്കെല്ലോയ്ക്കു് കാര്യങ്ങള്‍ അത്ര എളുപ്പമായിരുന്നില്ല. അവസാനം ഒരു പോയിന്റിനുവേണ്ടി ജീവന്‍ 
പണയംവച്ചുള്ള പോരാട്ടമാണു് മൈക്കല്‍ ഷുമാക്കറില്‍നിന്നും നേരിടേണ്ടിവന്നതു്. അന്‍പത്തിയാറാം ലാപ്പില്‍ പിറ്റ് 
ചെയ്ത ബാരിക്കെല്ലോ എതാണ്ടു് പത്തുലാപ്പോളം നീണ്ട പോരാട്ടത്തിനൊടുവില്‍ തലനാരിഴയ്ക്കാണു് ഷുമാക്കറില്‍നിന്നും 
പത്താംസ്ഥാനം നേടിയതു്.

വിരസമായ റേസായിരുന്നുവെങ്കിലും വെബ്ബര്‍ ഒന്നാമതെത്തുകയും, വെറ്റല്‍ മൂന്നാമതെത്തുകയും ചെയ്തതു്, റെഡ്ബുള്ളിനു് 
(312) മക്‌ലാരനുമേല്‍ (304) എട്ടു പോയിന്റ് ലീഡ് നേടിക്കൊടുത്തു. ഡ്രൈവര്‍മാരുടെ പോരാട്ടം ശരിക്കും ഒരു 'ഫൈവ് 
വേ' പോരാട്ടമാവുകയും ചെയ്തു. വെബ്ബര്‍ (161) ചെറിയൊരു ലീഡുമായി ഹാമില്‍ട്ടണു (157) മുകളില്‍ ഒന്നാമതാണിപ്പോള്‍. 
മൂന്നാമതു് വെറ്റലും (151). നാലും അഞ്ചും സ്ഥാനങ്ങളില്‍ നിലവിലെ ചാമ്പ്യന്‍ ബട്ടണും (147), 
അലോണ്‍സൊയുമാണു് (141). ആദ്യ അഞ്ചുസ്ഥാനങ്ങളെ പിരിക്കുന്നത് വെറും 20 പോയിന്റുമാത്രം. ഒരാഴ്ചയുടെ 
ഇടവേളയില്‍നടന്ന ജര്‍മന്‍-ഹംഗേറിയന്‍ റേസുകളില്‍ റെഡ്ബുള്‍ കാറുകളുടെ വേഗവ്യത്യാസം ശ്രദ്ധിച്ചാല്‍ത്തന്നെ 
ഇതെത്ര ചെറിയ വിടവാണെന്നു മനസ്സിലാവും. ജര്‍മനിയില്‍ ഫെറാരികള്‍ റെഡ്ബുള്ളിനൊപ്പത്തിനൊപ്പമായിരുന്നു, 
എന്നാല്‍ ഹംഗറിയിലെത്തിയപ്പോള്‍ അത് 24 സെക്കന്റ് ലീഡ് വരെ കൊടുക്കുന്ന രീതിയിലെത്തി. ട്രാക്കിനനുസരിച്ചു് 
കാര്‍ സെറ്റ് ചെയ്യുന്നതില്‍ റെഡ്ബുള്‍ ഫെറാരിയേക്കാള്‍ മികവു കാണിച്ചതു മാത്രമാണു് ഈ മുന്നേറ്റത്തിനടിസ്ഥാനം.

ഫോര്‍മുല വണ്ണിലെ വേനലവധിയാണു് ഇനി വരുന്ന രണ്ടാഴ്ചകള്‍. അതിനുശേഷം ആഗസ്റ്റ് അവസാനം 
ബെല്‍ജിയത്തിലും പിന്നീടു് സെപ്തംബര്‍ രണ്ടാംവാരം ഇറ്റലിയിലും നടക്കുന്ന പോരാട്ടങ്ങളോടെ ഫോര്‍മുല വണ്‍ 2010 
സീസണിന്റെ യൂറോപ്യന്‍ പാദം അവസാനിക്കും. പിന്നെ ഫാര്‍ ഈസ്റ്റിലെ മൂന്നു റേസുകളും (സിംഗപ്പൂര്‍, ജപ്പാന്‍, കൊറിയ),
ഏക ലാറ്റിനമേരിക്കന്‍ റേസും (ബ്രസീല്‍), മിഡില്‍ ഈസ്റ്റിലെ രണ്ടാം റേസുമാണു് (അബുദാബി) ബാക്കിയുള്ളതു്. ഈ 
റേസുകള്‍ പലതും പുതിയവയും കൃത്യമായി മനസ്സിലാക്കാനാവാത്ത ട്രാക്കുകളില്‍ നടക്കുന്നവയുമായതിനാല്‍ വരുന്ന രണ്ടു് 
യൂറോപ്യന്‍ റേസുകള്‍ ശക്തമായ തയ്യാറെടുപ്പുകളോടെയായിരിക്കും ടീമുകളെല്ലാം നേരിടുന്നതു്. ഫോഴ്സ് ഇന്ത്യ അവരുടെ 
കരിയറിലെ ഏറ്റവും മികച്ച പ്രകടനങ്ങള്‍ പുറത്തെടുത്തതു് കഴിഞ്ഞവര്‍ഷം ബെല്‍ജിയത്തിലും ഇറ്റലിയിലുമാണു്. 
അതു് ഇന്ത്യന്‍ ആരാധകര്‍ക്കു് വലിയ പ്രതീക്ഷകളാണു് നല്‍കുന്നത്.

\hspace*{2em}(5 August, 2010)\footnote{http://malayal.am/വിനോദം/കായികം/7191/ബുഡാപെസ്റ്റിലെ-തണുപ്പന്‍-കാറോട്ടം}

\newpage
