\secstar{ബുഡാപെസ്റ്റിലെ തണുപ്പന്‍ കാറോട്ടം}
\vskip 2pt

ആ­ഗ­സ്റ്റ് ഒന്നി­ന് ബു­ഡാ­പെ­സ്റ്റില്‍ നട­ന്ന ഹം­ഗേ­റി­യന്‍ ഗ്രാന്‍­പ്രീ­യോ­ടെ ഫോര്‍­മു­ലാ വണ്‍ 2010 സീ­സ­ണി­ലെ 12 റേ­സു­കള്‍­ക്ക് തീ­രു­മാ­ന­മാ­യി. ശക്ത­മായ ചില പോ­രാ­ട്ട­ങ്ങള്‍ ട്രാ­ക്കി­ലു­ണ്ടാ­യെ­ങ്കി­ലും, ഹം­ഗ­റി­യി­ലെ ചൂ­ടു­ള്ള ട്രാ­ക്കില്‍ വള­രെ തണു­ത്ത പോ­രാ­ട്ട­മാ­യി­രു­ന്നു അര­ങ്ങേ­റി­യ­ത്. മാര്‍­ക് വെ­ബ്ബ­റും റെ­ഡ്ബു­ള്ളും മക്‌­ലാ­ര­നില്‍­നി­ന്ന് പോ­യി­ന്റ് നി­ല­യില്‍ ലീ­ഡ് തി­രി­ച്ചു­പി­ടി­ച്ച് ശക്തി തെ­ളി­യി­ച്ചു. ജര്‍­മ­നി­യി­ലെ അത്ര ശക്ത­മാ­യി­രു­ന്നി­ല്ലെ­ങ്കി­ലും രണ്ടും നാ­ലും സ്ഥാ­ന­ങ്ങ­ളി­ലെ­ത്തി ഫെ­റാ­രി­യും കരു­ത്തു കാ­ണി­ച്ചു­.

­യോ­ഗ്യ­താ റൌ­ണ്ടില്‍ ഏഴാം തവ­ണ­യും ­സെ­ബാ­സ്റ്റ്യന്‍ വെ­റ്റല്‍ പോള്‍ നേ­ടി­യ­പ്പോള്‍, നാ­ലു­ത­വണ പോള്‍ നേ­ടിയ 
വെ­ബ്ബര്‍ രണ്ടാ­മ­തെ­ത്തി. സീ­സ­ണില്‍ ആറാം തവ­ണ­യാ­ണ്, ഗ്രി­ഡ്ഡി­ലെ മുന്‍ നിര ­റെ­ഡ്ബുള്‍ സ്വ­ന്ത­മാ­ക്കി­യ­ത്. 
തൊ­ട്ടു­പി­ന്നില്‍ ഫെ­റാ­രി­കള്‍ അലോണ്‍­സൊ­യു­ടെ നേ­തൃ­ത്വ­ത്തില്‍ അണി­നി­ര­ന്ന­പ്പോള്‍ നി­ല­വി­ലെ ചാ­മ്പ്യന്‍ ബട്ടണ്‍ 
മൂ­ന്നാം പാ­ദം കണ്ടി­ല്ല. ഫോ­ഴ്സ് ഇന്ത്യ­യു­ടെ കാ­റു­കള്‍ തു­ടര്‍­ച്ച­യാ­യി രണ്ടാം തവ­ണ­യും മൂ­ന്നാം പാ­ദ­ത്തി­ലെ­ത്തു­ന്ന­തില്‍ 
പരാ­ജ­യ­പ്പെ­ട്ടു. ­വി­റ്റാ­ലി പെ­ട്രോ­വ് സീ­സ­ണില്‍ ആദ്യ­മാ­യി സഹ ­റെ­നോ­ ഡ്രൈ­വര്‍ കു­ബി­ത്സ­യ്ക്കു മു­ന്നില്‍ 
യോ­ഗ്യ­ത­നേ­ടി­യ­പ്പോള്‍ പെ­ഡ്രോ ഡി ലാ റൊ­സ­യും നി­കൊ ഹള്‍­ക്കെന്‍­ബെര്‍­ഗും മൂ­ന്നാം പാ­ദ­ത്തി­ലെ­ത്തി ­സൌ­ബര്‍, ­
വി­ല്യം­സ് ടീ­മു­കള്‍ ശക്ത­മായ മദ്ധ്യ­നിര സാ­ന്നി­ദ്ധ്യ­മാ­ണെ­ന്നു തെ­ളി­യി­ച്ചു. ഇന്ത്യന്‍ ഡ്രൈ­വര്‍ കരണ്‍ ചന്ദോ­ക്കി­ന് 
ഇത്ത­വ­ണ­യും അവ­സ­രം കി­ട്ടി­യി­ല്ല.

ആ­ദ്യ­ലാ­പ്പില്‍ ശക്ത­മായ സ്റ്റാര്‍­ട്ടി­ലൂ­ടെ ഫെ­റാ­രി­കള്‍ റെ­ഡ്ബു­ളു­ക­ളു­ടെ മേല്‍ ചെ­റിയ ആധി­പ­ത്യം നേ­ടി. അലോണ്‍­സൊ 
വെ­ബ്ബ­റി­നെ മറി­ക­ട­ക്കു­ക­യും വെ­റ്റ­ലി­ന് വള­രെ അടു­ത്തെ­ത്തു­ക­യും ചെ­യ്ത­പ്പോള്‍ ആദ്യ­വ­ള­വി­ന് മു­മ്പ് ഉള്‍­വ­ശ­ത്തു­കൂ­ടെ 
വെ­ബ്ബ­റെ മറി­ക­ട­ക്കാ­നു­ള്ള മസ്സ­യു­ടെ ശ്ര­മം പാ­ളി­പ്പോ­യി. എന്നാല്‍ ഏഴാ­മ­തു­നി­ന്ന് റൊ­സ്ബര്‍­ഗി­നേ­യും ഹാ­മില്‍­ട്ട­ണേ­യും
മറി­ക­ട­ന്ന് അഞ്ചാം സ്ഥാ­ന­ത്തെ­ത്തിയ പെ­ട്രോ­വ് തന്റെ കഴി­വ് പു­റ­ത്തെ­ടു­ത്തു. മൈ­ക്കല്‍ ഷു­മാ­ക്ക­റാ­വ­ട്ടെ, മറ്റൊ­രു 
ശക്ത­മായ സ്റ്റാര്‍­ട്ടി­ലൂ­ടെ പു­തിയ തന്റെ അവ­താ­രം ഇപ്പോള്‍ നല്ല സ്റ്റാര്‍­ട്ട­റാ­ണെ­ന്നു കാ­ണി­ച്ചു­ത­ന്നു. മി­ക­ച്ച ഫ്ലൈ­യി­ങ് 
ലാ­പ്പു­ക­ളും പി­റ്റ് സ്റ്റോ­പ്/­ട­യര്‍ ഓപ്ഷന്‍ തീ­രു­മാ­ന­ങ്ങ­ളും എടു­ത്തി­രു­ന്ന പഴയ സ്വ­രൂ­പം കൂ­ടി തി­രി­ച്ചെ­ടു­ക്കാ­നാ­യാ­ലെ പക്ഷേ 
ഷു­മാ­ക്കര്‍­ക്ക് രക്ഷ­യു­ള്ളൂ. ആദ്യ­ലാ­പ്പില്‍ തന്റെ മു­ന്നില്‍ കട­ന്നെ­ങ്കി­ലും രണ്ടാം ലാ­പ്പില്‍ പെ­ട്രോ­വി­ന്റെ പരി­ച­യ­ക്കു­റ­വ് 
മു­ത­ലെ­ടു­ത്ത് ഹാ­മില്‍­ട്ടണ്‍ അഞ്ചാം സ്ഥാ­ന­ത്ത് തി­രി­ച്ചെ­ത്തി. റേ­സി­ലെ ആദ്യ റി­ട്ട­യര്‍­മെ­ന്റ് ടോ­റോ റോ­സോ­യു­ടെ 
ജെ­യ്മി അല്‍­ഗ്യു­സാ­രി­യു­ടേ­താ­യി­രു­ന്നു. എന്‍­ജിന്‍ പ്ര­ശ്നം കാ­ര­ണ­മാ­യി­രു­ന്നു വി­ര­മി­ക്കല്‍.

­ന­ല്ല ചൂ­ടു­ള്ള ട്രാ­ക്ക് സൂ­പ്പര്‍ സോ­ഫ്റ്റ് ടയ­റു­കള്‍­ക്ക് കൂ­ടു­തല്‍ ആയു­സ്സു­നല്‍­കി­യ­ത് വി­ര­സ­മായ ഒരു റേ­സി­ന് 
പ്ര­ധാ­ന­കാ­ര­ണ­മാ­യെ­ന്നു വേ­ണ­മെ­ങ്കില്‍ പറ­യാം. ടയ­റു­ക­ളോ അപ­ക­ട­ങ്ങ­ളി­ലൂ­ടെ­യു­ണ്ടായ അപ്ര­വ­ച­നീ­യ­ത­യോ ആണ് 
സീ­സ­ണി­ലെ മി­ക­ച്ച­തെ­ന്നു പറ­യാ­വു­ന്ന റേ­സു­കള്‍­ക്ക് വഴി­യൊ­രു­ക്കി­യ­ത്. ഇവി­ടെ­യും പതി­ന­ഞ്ചാം ലാ­പ്പില്‍ ബട്ട­ന്റെ­യും 
ലി­യു­സ്സി­യു­ടെ­യും കാ­റു­കള്‍ തമ്മി­ലു­ര­സു­ക­യും അതി­നു ശേ­ഷം തു­ട­രെ­ത്തു­ട­രെ കാ­റു­കള്‍ പി­റ്റ് ചെ­യ്യു­ക­യും ചെ­യ്ത­പ്പോള്‍ 
ഹാ­മില്‍­ട്ടണ്‍ മസ്സ­യെ മറി­ക­ട­ക്കു­ക­യും, പി­റ്റില്‍ നട­ന്ന ബഹ­ള­ത്തില്‍ അപ­ക­ട­ത്തി­ലൂ­ടെ സു­ട്ടി­ലും റൊ­സ്ബര്‍­ഗും 
വി­ര­മി­ക്കു­ക­യും കു­ബി­ത്സ­യു­ടെ പോ­യി­ന്റ് പ്ര­തീ­ക്ഷ­കള്‍ അസ്ത­മി­ക്കു­ക­യും ചെ­യ്തെ­ങ്കി­ലും, നല്ല മൈ­ലേ­ജ് നല്‍­കിയ സോ­ഫ്റ്റ് 
ടയ­റു­കള്‍ മാര്‍­ക് വെ­ബ്ബര്‍­ക്ക് ലീ­ഡ് നേ­ടി­ക്കൊ­ടു­ക്കു­ക­യാ­യി­രു­ന്നു. എല്ലാ­വ­രും സേ­ഫ്റ്റി­കാര്‍ ഇറ­ങ്ങു­ന്ന­തി­നു മു­മ്പ് പി­റ്റ് 
ചെ­യ്ത­പ്പോള്‍ വെ­ബ്ബ­റി­നും ബാ­രി­ക്കെ­ല്ലോ­ക്കും അതി­നു കഴി­ഞ്ഞി­ല്ല. തു­ടര്‍­ന്ന് പ്ര­തി­രോ­ധ­ത്തി­ലായ വെ­ബ്ബ­റെ സഹാ­യി­ക്കാന്‍ 
വെ­റ്റല്‍ മറ്റു­കാ­റു­ക­ളെ സേ­ഫ്റ്റി­കാ­റി­നു­പി­ന്നില്‍ പത്ത് കാര്‍ ദൂ­ര­ത്തി­നു­മ­പ്പു­റം തള­ച്ചി­ട്ടു. ഇതി­നു പി­ന്നീ­ട് വെ­റ്റ­ലി­ന് ഡ്രൈ­വ് 
ത്രൂ പെ­നാല്‍­ട്ടി ലഭി­ച്ചു­.

­സേ­ഫ്റ്റി­കാര്‍ പി­ന്മാ­റിയ ശേ­ഷം വെ­ബ്ബര്‍ സോ­ഫ്റ്റ് ടയ­റു­ക­ളു­ടെ ആനു­കൂ­ല്യ­വും ഹം­ഗ­റി­യില്‍ ഫെ­റാ­രി­ക്കു­മേല്‍ കണ്ടെ­ത്തിയ 
വേ­ഗ­വും മു­ത­ലെ­ടു­ത്ത് ലീ­ഡ് വര്‍­ദ്ധി­പ്പി­ക്കു­ന്ന­തില്‍ ശ്ര­ദ്ധ­ചെ­ലു­ത്തി. ഇതി­നി­ട­യില്‍ മക്‌­ലാ­ര­ന്റെ നി­രാ­ശ­യു­ടെ ആഴം 
വര്‍­ദ്ധി­പ്പി­ച്ചു­കൊ­ണ്ട് ഹാ­മില്‍­ട്ടണ്‍ ട്രാന്‍­സ്‌­മി­ഷന്‍ പ്ര­ശ്ന­വു­മാ­യി ഇരു­പ­ത്തി­നാ­ലാം ലാ­പ്പില്‍ വി­ര­മി­ച്ചു. പി­റ്റ് സ്റ്റോ­പ്പി­ലെ 
അപ­ക­ട­ത്തി­ന് 10 സെ­ക്ക­ന്റ് സ്റ്റോ­പ് ഗോ ശി­ക്ഷ­യും കൂ­ടി ലഭി­ച്ച കു­ബി­ത്സ അവ­സാ­നം ഇരു­പ­ത്തി­യാ­റാം ലാ­പ്പില്‍ 
റേ­സ് അവ­സാ­നി­പ്പി­ച്ചു. വെ­ബ്ബര്‍ ലീ­ഡ് വര്‍­ദ്ധി­പ്പി­ച്ചു കൊ­ണ്ടി­രു­ന്നെ­ങ്കി­ലും, ഡ്രൈ­വ് ത്രൂ വെ­റ്റ­ലി­ന് ഒരു 
നി­ശ്ച­യ­മാ­യി­രു­ന്ന രണ്ടാം സ്ഥാ­നം നഷ്ട­മാ­ക്കി. അവ­സാ­നം നാല്‍­പ്പ­ത്തി­നാ­ലാം ലാ­പ്പില്‍ പി­റ്റ് ചെ­യ്യു­മ്പോള്‍ രണ്ടാം 
സ്ഥാ­ന­ത്തു­ണ്ടാ­യി­രു­ന്ന ഫെ­റാ­രി­യു­ടെ അലോണ്‍­സൊ­യു­ടെ മേല്‍ വെ­ബ്ബ­റി­ന് 23.7 സെ­ക്ക­ന്റ് ലീ­ഡ് ഉണ്ടാ­യി­രു­ന്നു­.

എ­ന്നാല്‍ പ്രൈം ടയ­റു­ക­ളില്‍ റേ­സ് തു­ട­ങ്ങു­ക­യും പെ­ട്രോ­വില്‍ നി­ന്നും ഹള്‍­ക്കെന്‍­ബെര്‍­ഗില്‍ നി­ന്നും ശക്ത­മായ സമ്മര്‍­ദ്ദം 
നേ­രി­ടു­ക­യും ചെ­യ്ത ബാ­രി­ക്കെ­ല്ലോ­യ്ക്ക് കാ­ര്യ­ങ്ങള്‍ അത്ര എളു­പ്പ­മാ­യി­രു­ന്നി­ല്ല. അവ­സാ­നം ഒരു പോ­യി­ന്റി­നു വേ­ണ്ടി ജീ­വന്‍ 
പണ­യം വെ­ച്ചു­ള്ള പോ­രാ­ട്ട­മാ­ണ് മൈ­ക്കല്‍ ഷു­മാ­ക്ക­റില്‍ നി­ന്നും നേ­രി­ടേ­ണ്ടി വന്ന­ത്. അന്‍­പ­ത്തി­യാ­റാം ലാ­പ്പില്‍ പി­റ്റ് 
ചെ­യ്ത ­ബാ­രി­ക്കെ­ല്ലോ­ എതാ­ണ്ട് പത്തു­ലാ­പ്പോ­ളം നീ­ണ്ട പോ­രാ­ട്ട­ത്തി­നൊ­ടു­വില്‍ തല­നാ­രി­ഴ­യ്ക്കാ­ണ് ഷു­മാ­ക്ക­റില്‍ നി­ന്നും 
പത്താം സ്ഥാ­നം നേ­ടി­യ­ത്.

­വി­ര­സ­മായ റേ­സാ­യി­രു­ന്നു­വെ­ങ്കി­ലും വെ­ബ്ബര്‍ ഒന്നാ­മ­തെ­ത്തു­ക­യും, വെ­റ്റല്‍ മൂ­ന്നാ­മ­തെ­ത്തു­ക­യും ചെ­യ്ത­ത്, റെ­ഡ്ബു­ളി­ന് 
(312) മക്‌­ലാ­ര­നു­മേല്‍ (304) എട്ടു പോ­യി­ന്റ് ലീ­ഡ് നേ­ടി­ക്കൊ­ടു­ത്തു. ഡ്രൈ­വര്‍­മാ­രു­ടെ പോ­രാ­ട്ടം ശരി­ക്കും ഒരു 'ഫൈ­വ് 
വേ' പോ­രാ­ട്ട­മാ­വു­ക­യും ചെ­യ്തു. വെ­ബ്ബര്‍ (161) ചെ­റി­യൊ­രു ലീ­ഡു­മാ­യി ഹാ­മില്‍­ട്ട­ണു ­(157) മു­ക­ളില്‍ ഒന്നാ­മ­താ­ണി­പ്പോള്‍. 
മൂ­ന്നാ­മ­ത് വെ­റ്റ­ലും ­(151). നാ­ലും അഞ്ചും സ്ഥാ­ന­ങ്ങ­ളില്‍ നി­ല­വി­ലെ ചാ­മ്പ്യന്‍ ബട്ട­ണും ­(147), 
അലോണ്‍­സൊ­യു­മാ­ണ് (141). ആദ്യ അഞ്ചു സ്ഥാ­ന­ങ്ങ­ളെ പി­രി­ക്കു­ന്ന­ത് വെ­റും 20 പോ­യി­ന്റു­മാ­ത്രം. ഒരാ­ഴ്ച­യു­ടെ 
ഇട­വേ­ള­യില്‍ നട­ന്ന ജര്‍­മന്‍-ഹം­ഗേ­റി­യന്‍ റേ­സു­ക­ളില്‍ റെ­ഡ്ബുള്‍ കാ­റു­ക­ളു­ടെ വേ­ഗ­വ്യ­ത്യാ­സം ശ്ര­ദ്ധി­ച്ചാല്‍­ത്ത­ന്നെ 
ഇതെ­ത്ര ചെ­റിയ വി­ട­വാ­ണെ­ന്നു മന­സ്സി­ലാ­വും. ജര്‍­മ­നി­യില്‍ ഫെ­റാ­രി­കള്‍ റെ­ഡ്ബു­ളി­നൊ­പ്പ­ത്തി­നൊ­പ്പ­മാ­യി­രു­ന്നു, 
എന്നാല്‍ ഹം­ഗ­റി­യി­ലെ­ത്തി­യ­പ്പോള്‍ അത് 24 സെ­ക്ക­ന്റ് ലീ­ഡ് വരെ കൊ­ടു­ക്കു­ന്ന രീ­തി­യി­ലെ­ത്തി. ട്രാ­ക്കി­ന­നു­സ­രി­ച്ച് 
കാര്‍ സെ­റ്റ് ചെ­യ്യു­ന്ന­തില്‍ റെ­ഡ്ബുള്‍ ഫെ­റാ­രി­യേ­ക്കാള്‍ മി­ക­വു കാ­ണി­ച്ച­തു മാ­ത്ര­മാ­ണ് ഈ മു­ന്നേ­റ്റ­ത്തി­ന­ടി­സ്ഥാ­നം­.

­ഫോര്‍­മുല വണ്ണി­ലെ വേ­ന­ല­വ­ധി­യാ­ണ് ഇനി വരു­ന്ന രണ്ടാ­ഴ്ച­കള്‍. അതി­നു ശേ­ഷം ആഗ­സ്റ്റ് അവ­സാ­നം 
ബെല്‍­ജി­യ­ത്തി­ലും പി­ന്നീ­ട് സെ­പ്തം­ബര്‍ രണ്ടാം വാ­രം ഇറ്റ­ലി­യി­ലും നട­ക്കു­ന്ന പോ­രാ­ട്ട­ങ്ങ­ളോ­ടെ ­ഫോര്‍­മുല വണ്‍ 2010 
സീ­സ­ണി­ന്റെ ­യൂ­റോ­പ്യന്‍ പാ­ദം­ അവ­സാ­നി­ക്കും. പി­ന്നെ ഫാര്‍ ഈസ്റ്റി­ലെ മൂ­ന്നു റേ­സു­ക­ളും (സിം­ഗ­പ്പൂര്‍, ­ജ­പ്പാന്‍, കൊ­റി­യ),
ഏക ലാ­റ്റി­ന­മേ­രി­ക്കന്‍ റേ­സും (­ബ്ര­സീല്‍), മി­ഡി­ലീ­സ്റ്റി­ലെ രണ്ടാം റേ­സു­മാ­ണ് (അ­ബു­ദാ­ബി­) ബാ­ക്കി­യു­ള്ള­ത്. ഈ 
റേ­സു­കള്‍ പല­തും പു­തി­യ­വ­യും കൃ­ത്യ­മാ­യി മന­സ്സി­ലാ­ക്കാ­നാ­വാ­ത്ത ട്രാ­ക്കു­ക­ളില്‍ നട­ക്കു­ന്ന­വ­യു­മാ­യി­തി­നാല്‍ വരു­ന്ന രണ്ട് 
യൂ­റോ­പ്യന്‍ റേ­സു­കള്‍ ശക്ത­മായ തയ്യാ­റെ­ടു­പ്പു­ക­ളോ­ടെ­യാ­യി­രി­ക്കും ടീ­മു­ക­ളെ­ല്ലാം നേ­രി­ടു­ന്ന­ത്. ­ഫോ­ഴ്സ് ഇന്ത്യ അവ­രു­ടെ 
കരി­യ­റി­ലെ ഏറ്റ­വും മി­ക­ച്ച പ്ര­ക­ട­ന­ങ്ങള്‍ പു­റ­ത്തെ­ടു­ത്ത­തും കഴി­ഞ്ഞ വര്‍­ഷം ബെല്‍­ജി­യ­ത്തി­ലും ഇറ്റ­ലി­യി­ലു­മാ­ണ്. 
അത് ഇന്ത്യന്‍ ആരാ­ധ­കര്‍­ക്കും വലിയ പ്ര­തീ­ക്ഷ­ക­ളാ­ണ് നല്‍­കു­ന്ന­ത്.

(5 August 2010)\footnote{http://malayal.am/വിനോദം/കായികം/7191/ബുഡാപെസ്റ്റിലെ-തണുപ്പന്‍-കാറോട്ടം}

\newpage
