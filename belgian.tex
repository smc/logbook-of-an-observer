\secstar{പോയിന്റ് നിലയിലെ വ്യത്യാസം കൂട്ടിയ ബെല്‍ജിയന്‍ ഗ്രാന്‍പ്രി}
\vskip 2pt

ഫോര്‍­മുല വണ്‍ കാ­ല­ണ്ട­റി­ലെ ഏറ്റ­വും ആവേ­ശ­ക­ര­മായ റേ­സു­ക­ളി­ലൊ­ന്നാ­ണ് സ്പാ­യി­ലെ ട്രാ­ക്കില്‍ നട­ക്കു­ന്ന 
ബെല്‍­ജി­യന്‍ ഗ്രാന്‍­പ്രീ. ആദ്യ­കാ­ലം മു­ത­ലേ ഉള്ള ട്രാ­ക്കാ­ണെ­ങ്കി­ലും, അപ്ര­വ­ച­നീ­യ­മായ കാ­ലാ­വ­സ്ഥ­യും ദുര്‍­ഘ­ട­മായ 
ട്രാ­ക്കും സ്പാ­യി­ലെ പോ­രാ­ട്ട­ങ്ങള്‍­ക്ക് ആവേ­ശം പക­രാ­റു­ണ്ട്. കഴി­ഞ്ഞ വര്‍­ഷം ­ഫോ­ഴ്സ് ഇന്ത്യ ആദ്യ പോ­ളും പോ­ഡി­യ­വും 
നേ­ടി­യ­ത് സ്പാ­യി­ലാ­ണ്. ഇക്കൊ­ല്ല­ത്തെ ­ഫോര്‍­മുല വണ്‍ സീ­സ­ണി­ന്റെ ആവ­സാ­ന­പാ­ദ­ത്തി­ന്റെ ആരം­ഭ­മാ­യി­രു­ന്നു 
സ്പാ­യി­ലെ പതി­മൂ­ന്നാം റൌ­ണ്ട് പോ­രാ­ട്ട­മെ­ന്നു വേ­ണ­മെ­ങ്കില്‍ പറ­യാം. സമ്മര്‍ ബ്രേ­ക്ക് കഴി­ഞ്ഞ് ആദ്യ റേ­സ്. സ്പാ­യ്ക്കു 
ശേ­ഷം മോണ്‍­സ­യില്‍ നട­ക്കാ­നി­രി­ക്കു­ന്ന ഇറ്റാ­ലി­യന്‍ ഗ്രാന്‍­പ്രീ­യും കഴി­യു­ന്ന­തോ­ടെ ഇക്കൊ­ല്ല­ത്തെ യു­റോ­പ്യന്‍ 
പാ­ദ­ത്തി­ന് അവ­സാ­ന­മാ­കും­.

­കാ­ല­ണ്ട­റി­ലെ ഏറ്റ­വും നീ­ളം കൂ­ടിയ ട്രാ­ക്കായ സ്പാ­യി­ലെ യോ­ഗ്യ­താ­റൌ­ണ്ടു­കള്‍ എന്ന­ത്തേ­യും പോ­ലെ 
ആവേ­ശ­ക­ര­മാ­യി­രു­ന്നു. ഗി­യര്‍ ബോ­ക്സ് പെ­നാല്‍­ട്ടി നി­കൊ റൊ­സ്ബര്‍­ഗി­ന് ഗ്രി­ഡ്ഡില്‍ പതി­നാ­ലാം സ്ഥാ­നം 
നല്‍­കി­യ­പ്പോള്‍, ഹം­ഗ­റി­യില്‍ ബാ­രി­ക്കെ­ല്ലോ­യോ­ടു കാ­ണി­ച്ച വക­തി­രി­വി­ല്ലാ­യ്മ­യ്ക്ക് സഹ ­മെ­ഴ്സി­ഡ­സ് ഡ്രൈ­വര്‍ മൈ­ക്കല്‍ 
ഷു­മാ­ക്കര്‍ 10 സ്ഥാ­നം പെ­നാല്‍­ട്ടി­യോ­ടു­കൂ­ടി ഗ്രി­ഡ്ഡില്‍ ഇരുപത്തൊന്നാം സ്ഥാ­ന­ത്താ­യി. കഴി­ഞ്ഞ വര്‍­ഷം ഫി­സി­ക്കെ­ല്ല­യി­ലൂ­ടെ 
പോള്‍ നേ­ടി­യി­രു­ന്നെ­ങ്കി­ലും ഇപ്രാ­വ­ശ്യം സു­ട്ടി­ലി­ന്റെ മൂ­ന്നാം പാ­ദ­പ്ര­വേ­ശ­വും ഗ്രി­ഡ്ഡില്‍ എട്ടാം സ്ഥാ­ന­ത്തു­ള്ള തു­ട­ക്ക­വും 
കൊ­ണ്ട് ഫോ­ഴ്സ് ഇന്ത്യ­ക്ക് തൃ­പ്തി­പ്പെ­ടേ­ണ്ടി­വ­ന്നു. മഴ­യ്ക്കു­മുന്‍­പ് നല്ല സമ­യം കണ്ടെ­ത്താ­നാ­യി ട്രാ­ക്കില്‍ പൊ­രി­ഞ്ഞ പോ­രാ­ട്ടം 
നട­ന്ന­ത് യോ­ഗ്യ­താ­റൌ­ണ്ട് ആവേ­ശ­ക­ര­മാ­ക്കി. അവ­സാ­നം പതി­വു­പോ­ലെ റെ­ഡ്ബു­ള്ളി­നു വേ­ണ്ടി ­മാര്‍­ക് വെ­ബ്ബര്‍ 
പോള്‍ നേ­ടി­യ­പ്പോള്‍ മഴ­യ­ത്തു നട­ത്തിയ മി­ക­ച്ചൊ­രു പ്ര­ക­ട­നം കൊ­ണ്ട് ഹാ­മില്‍­ട്ടണ്‍ മുന്‍ നി­ര­യില്‍ സ്ഥാ­നം നേ­ടി. 
അലോണ്‍­സൊ­യ്ക്ക് പത്താ­മ­തെ­ത്താ­നെ കഴി­ഞ്ഞു­ള്ളൂ. എന്നാല്‍ രണ്ടാം ഫെ­റാ­രി­യില്‍ ­ഫെ­ലി­പെ മസ്സ ആറാ­മ­തെ­ത്തി. 
മൂ­ന്നു­മു­തല്‍ അഞ്ചു­വ­രെ സ്ഥാ­ന­ങ്ങ­ളില്‍ യഥാ­ക്ര­മം കു­ബി­ത്സ, വെ­റ്റല്‍, ബട്ടണ്‍ എന്നി­വര്‍ സ്ഥാ­ന­മു­റ­പ്പി­ച്ച­പ്പോള്‍ ­
കോ­സ്‌­വര്‍­ത്ത് എന്‍­ജി­നു­മാ­യി ­വി­ല്യം­സ് ഏഴും ഒന്‍­പ­തും സ്ഥാ­ന­ങ്ങള്‍ നേ­ടി­.

­ട്രാ­ക്കി­ന്റെ പെ­രു­മ­യ്ക്കൊ­ത്ത പോ­രാ­ട്ട­മാ­യി­രു­ന്നു സ്പാ­യില്‍ അര­ങ്ങേ­റി­യ­ത്. ആദ്യ­ലാ­പ്പില്‍­ത്ത­ന്നെ ഹാ­മില്‍­ട്ടണ്‍ വെ­ബ്ബ­റെ 
മറി­ക­ട­ന്ന­തും വള­രെ അടു­ത്തു­ള്ള ആദ്യ വള­വി­ന്റെ സമ്മര്‍­ദ്ദം കാ­ര­ണം ഏഴാം സ്ഥാ­ന­ത്തേ­യ്ക്ക് വെ­ബ്ബര്‍ 
പി­ന്ത­ള്ള­പ്പെ­ട്ടു­പോ­യ­തും തു­ട­ക്കം ആവേ­ശ­ക­ര­മാ­ക്കി. എന്നാല്‍ ആദ്യ­ലാ­പ്പില്‍ ചെ­റു­താ­യി ചാ­റിയ മഴ അവ­സാന 
ഷി­കേ­നില്‍ കാ­റു­ക­ളു­ടെ നി­യ­ന്ത്ര­ണം തെ­റ്റി­ച്ചു. തന്റെ മു­ന്നൂ­റാ­മ­ത്തെ റേ­സില്‍ ആദ്യ­ലാ­പ്പില്‍ ബാ­രി­ക്കെ­ല്ലോ­യ്ക്ക് അത് 
പു­റ­ത്തേ­ക്കു­ള്ള വഴി­യു­മൊ­രു­ക്കി. നന­ഞ്ഞ ട്രാ­ക്കില്‍ നി­യ­ന്ത്ര­ണം കി­ട്ടാ­തെ വല­ഞ്ഞ ­ബാ­രി­ക്കെ­ല്ലോ­ അലോണ്‍­സൊ­യു­മാ­യി 
കൂ­ട്ടി­യി­ടി­ക്കു­ക­യാ­യി­രു­ന്നു. ഇത് ആദ്യ­ലാ­പ്പില്‍­ത്ത­ന്നെ സേ­ഫ്റ്റി­കാര്‍ ഇറ­ങ്ങു­ന്ന­തി­നു കാ­ര­ണ­മാ­യി. പ്രൈം 
ടയ­റു­ക­ളി­ലാ­യി­രു­ന്ന പെ­ട്രോ­വും ഷു­മാ­ക്ക­റും ഈ ബഹ­ള­ങ്ങള്‍ മു­ത­ലെ­ടു­ത്ത് ഈ സമ­യം കൊ­ണ്ട് മുന്‍ നി­ര­യില്‍ 
എത്തി­യി­രു­ന്നു­.

എ­ന്നാല്‍ അപ­ക­ട­ത്തെ­ത്തു­ടര്‍­ന്ന് പി­റ്റ് ചെ­യ്ത് ടയ­റു­കള്‍ മാ­റാ­നു­ള്ള അലോണ്‍­സൊ­യു­ടെ തീ­രു­മാ­നം തി­രി­ച്ച­ടി­ച്ചു. മഴ 
പി­ന്നെ ഒഴി­ഞ്ഞു നി­ന്ന­തോ­ടെ ഒരി­ക്കല്‍ കൂ­ടി പി­റ്റ് ചെ­യ്ത് സ്ലി­ക് ടയ­റു­ക­ളി­ലേ­ക്ക് മാ­റേ­ണ്ടി­വ­ന്നു അദ്ദേ­ഹ­ത്തി­ന്. ഈ 
സമ­യം മുന്‍­നി­ര­യില്‍ മക്‌­ലാ­രന്‍ കാ­റു­കള്‍ ഒന്നും രണ്ടും സ്ഥാ­ന­ങ്ങ­ളി­ലും, പി­ന്നീ­ട് ക്ര­മ­മാ­യി വെ­റ്റല്‍, കു­ബി­ത്സ, 
വെ­ബ്ബര്‍ എന്നി­വ­രു­മാ­യി­രു­ന്നു. ആറാം സ്ഥാ­ന­ത്ത് ഫെ­ലി­പെ മസ്സ­യു­ടെ ഫെ­റാ­രി­യും, അലോണ്‍­സൊ, അപ­ക­ട­ത്തി­നും 
പി­റ്റ് സ്റ്റോ­പ്പി­നും ശേ­ഷം പിന്‍­നി­ര­യില്‍ നി­ന്നും കടു­ത്ത­പോ­രാ­ട്ട­ങ്ങ­ളി­ലൂ­ടെ മദ്ധ്യ­നി­ര­യില്‍ തി­രി­ച്ചെ­ത്തി. പതി­നൊ­ന്നാം 
ലാ­പ്പില്‍ ഉഗ്ര­നൊ­രു മറി­ക­ട­ക്ക­ലി­ലൂ­ടെ പെ­ട്രോ­വും ഷു­മാ­ക്ക­റും ആദ്യ­പ­ത്തി­നു­ള്ളി­ലെ­ത്തി. ടീം മേ­റ്റ് റൊ­സ്ബര്‍­ഗി­നെ 
മറി­ക­ട­ക്കു­മ്പോള്‍ ടയ­റു­കള്‍ തമ്മില്‍ ചെ­റു­താ­യി ഉര­സി­യെ­ങ്കി­ലും വലിയ അപ­ക­ട­ങ്ങ­ളൊ­ന്നു­മി­ല്ലാ­തെ ഷു­മാ­ക്കര്‍ 
ഇരു­പ­തില്‍­നി­ന്നും ആദ്യ­പ­ത്തി­ലെ­ത്തി­.

­പി­ന്നീ­ട് കു­റ­ച്ചു­സ­മ­യം തണു­ത്തു പോ­വു­ക­യും, വലിയ പ്ര­ശ്ന­ങ്ങ­ളൊ­ന്നു­മി­ല്ലാ­തെ മക്‌­ലാ­രന്‍ ഒന്നും രണ്ടും സ്ഥാ­ന­ങ്ങള്‍ 
നേ­ടു­മെ­ന്നും തോ­ന്നി­ച്ച റേ­സ് ബട്ട­നും വെ­റ്റ­ലും തമ്മി­ലു­ള്ള ഒരു കൂ­ട്ടി­യി­ടി­യി­ലൂ­ടെ വീ­ണ്ടും ട്രാ­ക്കില്‍ പോ­രാ­ട്ട­ങ്ങ­ളു­യര്‍­ത്തി. 
പതി­നേ­ഴാം ലാ­പ്പില്‍ നട­ന്ന കൂ­ട്ടി­യി­ടി ശരി­ക്കും വെ­റ്റ­ലി­ന്റെ പി­ഴ­വാ­യി­രു­ന്നു. എങ്കി­ലും അത് ബട്ട­ന്റെ സ്പാ­യി­ലെ പോ­രാ­ട്ടം 
അവ­സാ­നി­പ്പി­ച്ചു. പി­റ്റ് ലേ­നി­ന­ടു­ത്താ­യ­ത് വെ­റ്റ­ലി­ന് പി­റ്റ് സ്റ്റോ­പ്പെ­ടു­ത്ത് റേ­സില്‍ തി­രി­ച്ചു വരാന്‍ അവ­സ­ര­മൊ­രു­ക്കി. 
എന്നാല്‍ ഒരു ഡ്രൈ­വ് ത്രൂ പെ­നാല്‍­റ്റി ലഭി­ച്ച­ത് ­റെ­ഡ്ബുള്‍ ഡ്രൈ­വ­റു­ടെ ബെല്‍­ജി­യന്‍ റേ­സി­ന്റെ കാ­ര്യ­ത്തില്‍ ഒരു 
തീ­രു­മാ­ന­മു­ണ്ടാ­ക്കി­യെ­ന്നു പറ­യാം. ഈ അപ­ക­ട­ത്തോ­ടെ ചാ­മ്പ്യന്‍­ഷി­പ്പ് പോ­രാ­ട്ട­ത്തി­ലേര്‍­പ്പെ­ട്ടി­രി­ക്കു­ന്ന അഞ്ചു­പേ­രില്‍ 
മൂ­ന്നു പേ­രും പോ­യി­ന്റി­നു വെ­ളി­യി­ലാ­യി­.

­വെ­റ്റല്‍ പി­ന്നീ­ട് പൊ­സി­ഷന്‍ തി­രി­ച്ചു പി­ടി­ക്കാന്‍ ശ്ര­മി­ച്ചെ­ങ്കി­ലും ഇരു­പ­ത്തി­യേ­ഴാം ലാ­പ്പില്‍ മറ്റൊ­രു അപ­ക­ട­വും തു­ടര്‍­ന്ന് 
പിന്‍­ട­യ­റു­ക­ളി­ലൊ­ന്നു പഞ്ച­റാ­യ­തും കാ­ര്യ­ങ്ങള്‍­ക്ക് ഏക­ദേ­ശം തീ­രു­മാ­ന­മാ­ക്കി. ഈ സമ­യ­മൊ­ക്കെ­യും ഹാ­മില്‍­ട്ടണ്‍ ലീ­ഡ് 
നി­ല­നിര്‍­ത്തി­യി­രു­ന്നു. കു­ബി­ത്സ­യും വെ­ബ്ബ­റും തമ്മില്‍ രണ്ടാം സ്ഥാ­ന­ത്തി­നു വേ­ണ്ടി കടു­ത്ത­മ­ത്സ­ര­വും ട്രാ­ക്കില്‍ അര­ങ്ങേ­റി. 
മു­പ്പ­താം ലാ­പ്പെ­ത്തി­യ­പ്പോ­ഴെ­യ്ക്കും മുന്‍­നിര ഡ്രൈ­വര്‍­മാ­രെ­ല്ലാം ടയ­റു­കള്‍ മാ­റ്റി­യി­രു­ന്നു. ആദ്യ­പ­ത്തില്‍ ഷു­മാ­ക്ക­റും 
റൊ­സ്ബര്‍­ഗും മാ­ത്ര­മേ പി­റ്റ് സ്റ്റോ­പ്പെ­ടു­ക്കാ­തി­രു­ന്നു­ള്ളു­.

­റേ­സി­ന­വ­സാ­നം മഴ­പെ­യ്യു­മ്പോള്‍ പി­റ്റ് ചെ­യ്ത് ഗ്രൂ­വു­ക­ളു­ള്ള ഇന്റര്‍­മീ­ഡി­യ­റ്റ് ടയ­റു­ക­ളി­ലേ­ക്ക് മാ­റാ­നാ­യി­രു­ന്നു 
മെ­ഴ്സി­ഡ­സി­ന്റെ തീ­രു­മാ­നം. അത് ദി­വ­സ­ത്തെ ഏറ്റ­വും നല്ല സ്ട്രാ­റ്റ­ജി­ക­ളി­ലൊ­ന്നാ­യി മാ­റു­ക­യും ചെ­യ്തു. മു­പ്പ­ത്തി­യ­ഞ്ചാം 
ലാ­പ്പില്‍ മഴ­പെ­യ്തു തു­ട­ങ്ങി­യ­പ്പോള്‍ എല്ലാ­വ­രും പി­റ്റ് ചെ­യ്തു ടയ­റു­കള്‍ മാ­റി. പി­റ്റ് ലേ­നില്‍ ചെ­റു­താ­യി പി­ഴ­ച്ച കു­ബി­ത്സ­യെ 
മു­ത­ലെ­ടു­ത്ത് വെ­ബ്ബര്‍ രണ്ടാം സ്ഥാ­നം പി­ടി­ച്ചെ­ടു­ക്കു­ക­യും ചെ­യ്തു. എന്നാല്‍ നി­യ­ന്ത്ര­ണം നഷ്ട­പ്പെ­ട്ട് ട്രാ­ക്കി­നു വി­ല­ങ്ങ­നെ 
കി­ട­ന്ന അലോണ്‍­സൊ­യു­ടെ ­ഫെ­റാ­രി­ വീ­ണ്ടും സേ­ഫ്റ്റി കാ­റി­നെ രം­ഗ­ത്തെ­ത്തി­ച്ചു. മഴ­യും സേ­ഫ്റ്റി­കാ­റും കാ­ര്യ­ങ്ങള്‍ 
ദുര്‍­ഘ­ട­മാ­ക്കി­യെ­ങ്കി­ലും, സേ­ഫ്റ്റി കാര്‍ പിന്‍­വ­ലി­ച്ച­യു­ട­നെ നട­ത്തിയ ഉഗ്ര­നൊ­രു നീ­ക്ക­ത്തി­ലൂ­ടെ കൊ­ബി­യാ­ഷി­യേ­യും 
ഷു­മാ­ക്ക­റെ­യും റൊ­സ്ബര്‍­ഗ് ഒരു­മി­ച്ചു മറി­ക­ട­ന്നു­.

­സം­ഭ­വ­ബ­ഹു­ല­മായ ­ബെല്‍­ജി­യന്‍ ഗ്രാന്‍­പ്രീ­ ഹാ­മില്‍­ട്ട­ണ് കി­രീ­ടം നേ­ടി­ക്കൊ­ടു­ത്ത­തി­നൊ­പ്പം ചാ­മ്പ്യന്‍­ഷി­പ്പ് 
പോ­രാ­ട്ട­ങ്ങ­ളി­ലെ കടു­പ്പം കു­റ­ച്ചു കു­റ­യ്ക്കു­ക­യും ചെ­യ്തു. സ്പാ­യ്ക്കു­മുന്‍­പ് ആദ്യ അഞ്ചു സ്ഥാ­ന­ക്കാ­രെ വെ­റും 20 പോ­യി­ന്റാ­യി­രു­ന്നു 
പി­രി­ച്ചി­രു­ന്നു­ത്. ഇപ്പോള്‍ അത് നാല്‍­പ്പ­തു പോ­യി­ന്റാ­യി ഉയര്‍­ന്നു. പോ­രാ­ട്ട­ത്തില്‍ ഹാ­മില്‍­ട്ട­ണും ­(182) വെ­ബ്ബ­റും ­(179) 
ഒപ്പ­ത്തി­നൊ­പ്പ­മാ­ണെ­ങ്കില്‍ വെ­റ്റ­ലും അലോണ്‍­സൊ­യും ബട്ട­ണു­മെ­ല്ലാം അവ­ര­വ­രു­ടെ പഴ­യ­സ്ഥാ­ന­ങ്ങ­ളില്‍­ത്ത­ന്നെ­യാ­ണ്.
സു­ട്ടില്‍ ഒരു പോ­യി­ന്റു വ്യ­ത്യാ­സ­ത്തില്‍ മൈ­ക്കല്‍ ഷു­മാ­ക്ക­റെ മറി­ക­ട­ന്ന് ഒന്‍­പ­താ­മെ­ത്തി­യ­താ­ണ് പി­ന്നെ­യൊ­രു 
മാ­റ്റ­മെ­ന്നു പറ­യാ­വു­ന്ന­ത്. കണ്‍­സ്ട്ര­ക്റ്റര്‍­മാ­രു­ടെ കാ­ര്യ­ത്തില്‍ റെ­ഡ്ബുള്‍ (330) ഒരു പോ­യി­ന്റി­ന് മക്‌­ലാ­ര­നു­മു­ന്നി­ലാ­ണ്. 
മൂ­ന്നാ­മ­ത് ഫെ­റാ­രി­യും ­(250).

ഇ­നി മോണ്‍­സ­യി­ലെ ഇറ്റാ­ലി­യന്‍ ഗ്രാന്‍­പ്രീ കഴി­ഞ്ഞാല്‍ നാ­ലു ഏഷ്യന്‍ റേ­സു­ക­ളും സാ­വോ­പോ­ളോ­യു­മാ­ണ് ബാ­ക്കി. 
ബെല്‍­ജി­യ­ത്തി­നു മു­മ്പു­ണ്ടാ­യി­രു­ന്ന അവ­സ്ഥ­യി­ലേ­ക്ക് കാ­ര്യ­ങ്ങള്‍ എത്താന്‍ വി­ഷ­മ­മാ­ണെ­ങ്കി­ലും അങ്ങ­നെ 
സം­ഭ­വി­ക്കു­ക­യും അവ­സാന രണ്ടു റേ­സു­കള്‍ നിര്‍­ണ്ണാ­യ­ക­മാ­വു­ക­യും ചെ­യ്യു­ന്ന­ത് ആരാ­ധ­കര്‍­ക്ക് ആവേ­ശം 
പകര്‍­ന്നേ­ക്കും­.

­പിന്‍­കു­റി­പ്പ്: ഇന്ത്യന്‍ ഗ്രാന്‍­പ്രീ കൂ­ടി കല­ണ്ട­റി­ലി­ടം പി­ടി­ച്ച­തോ­ടെ ഏതാ­ണ്ട് യൂ­റൊ­പ്പി­നൊ­പ്പം തന്നെ റേ­സു­കള്‍ ഇപ്പോള്‍ 
ഏഷ്യ­യി­ലും എത്തി­യി­രി­ക്കു­ന്നു, ഏഷ്യന്‍ ട്രാ­ക്കു­കള്‍ കു­റ­ച്ചു­കൂ­ടി നല്ല മത്സ­ര­ങ്ങള്‍ കൂ­ടി കാ­ഴ്ച­വെ­ച്ചാല്‍ ഭാ­വി­യില്‍ ഫോര്‍­മുല 
വണ്ണി­ന്റെ വള­യം ഏഷ്യന്‍ വമ്പന്‍­മാ­രി­ലെ­ത്തി­ക്കൂ­ടെ­ന്നി­ല്ല. യൂ­റോ­പ്പി­ലെ പല ട്രാ­ക്കു­ക­ളും ജന­ങ്ങ­ളു­ടെ അതൃ­പ്തി 
നേ­രി­ടു­ന്നു­വെ­ന്ന­തും സു­ര­ക്ഷ­യു­ടെ കാ­ര്യ­ത്തില്‍ പഴ­യ­ട്രാ­ക്കു­കള്‍­ക്ക് പല പ്ര­ശ്ന­ങ്ങ­ളു­മു­ണ്ടെ­ന്ന­തും കൂ­ടി കണ­ക്കി­ലെ­ടു­ത്താല്‍,
ആഞ്ഞു­പി­ടി­ച്ചാല്‍ ഇം­ഗ്ല­ണ്ടി­ലി­രി­ക്കു­ന്ന ഫോര്‍­മുല വണ്‍ ഫാ­ക്റ്റ­റി­ക­ളെ­ല്ലാം ക്വാ­ലാ­ലം­പൂ­രി­ലോ, ഡെല്‍­ഹി­യി­ലോ­,­ 
ഷാ­ങ്ഹാ­യി­ലോ എത്തി­ക്കാ­വു­ന്ന­തേ­യു­ള്ളൂ­.

(11 September 2010)\footnote{http://malayal.am/വിനോദം/കായികം/7894/പോയിന്റ്-നിലയിലെ-വ്യത്യാസം-കൂട്ടിയ-ബെല്‍ജിയന്‍-ഗ്രാന്‍പ്രി}

\newpage
