\secstar{പോയിന്റ് നിലയിലെ വ്യത്യാസം കൂട്ടിയ ബെല്‍ജിയന്‍ ഗ്രാന്‍പ്രി}
\vskip 2pt

ഫോര്‍മുല വണ്‍ കലണ്ടറിലെ ഏറ്റവും ആവേശകരമായ റേസുകളിലൊന്നാണു് സ്പായിലെ ട്രാക്കില്‍ നടക്കുന്ന 
ബെല്‍ജിയന്‍ ഗ്രാന്‍പ്രീ. ആദ്യകാലംമുതലേയുള്ള ട്രാക്കാണെങ്കിലും, അപ്രവചനീയമായ കാലാവസ്ഥയും ദുര്‍ഘടമായ 
ട്രാക്കും സ്പായിലെ പോരാട്ടങ്ങള്‍ക്കു് ആവേശം പകരാറുണ്ടു്. കഴിഞ്ഞവര്‍ഷം ഫോഴ്സ് ഇന്ത്യ ആദ്യ പോളും പോഡിയവും 
നേടിയതു് സ്പായിലാണു്. ഇക്കൊല്ലത്തെ ഫോര്‍മുല വണ്‍ സീസണിന്റെ ആവസാനപാദത്തിന്റെ ആരംഭമായിരുന്നു 
സ്പായിലെ പതിമൂന്നാം റൗണ്ട് പോരാട്ടമെന്നു് വേണമെങ്കില്‍ പറയാം. സമ്മര്‍ ബ്രേക്ക് കഴിഞ്ഞു് ആദ്യ റേസ്. 
സ്പായ്ക്കുശേഷം മോണ്‍സയില്‍ നടക്കാനിരിക്കുന്ന ഇറ്റാലിയന്‍ ഗ്രാന്‍പ്രിയും കഴിയുന്നതോടെ ഇക്കൊല്ലത്തെ യുറോപ്യന്‍ 
പാദത്തിനു് അവസാനമാകും.

കലണ്ടറിലെ ഏറ്റവും നീളംകൂടിയ ട്രാക്കായ സ്പായിലെ യോഗ്യതാറൗണ്ടുകള്‍ എന്നത്തേയുംപോലെ 
ആവേശകരമായിരുന്നു. ഗിയര്‍ ബോക്സ് പെനാല്‍ട്ടി നികൊ റൊസ്ബര്‍ഗിനു് ഗ്രിഡ്ഡില്‍ പതിനാലാം സ്ഥാനം 
നല്‍കിയപ്പോള്‍, ഹംഗറിയില്‍ ബാരിക്കെല്ലോയോടു കാണിച്ച വകതിരിവില്ലായ്മയ്ക്കു് സഹ മെഴ്സിഡസ് ഡ്രൈവര്‍ മൈക്കല്‍ 
ഷുമാക്കര്‍ 10 സ്ഥാനം പെനാല്‍ട്ടിയോടുകൂടി ഗ്രിഡ്ഡില്‍ ഇരുപത്തൊന്നാം സ്ഥാനത്തായി. കഴിഞ്ഞവര്‍ഷം ഫിസിക്കെല്ലയിലൂടെ 
പോള്‍ നേടിയിരുന്നെങ്കിലും ഇപ്രാവശ്യം സുട്ടിലിന്റെ മൂന്നാംപാദ പ്രവേശവും ഗ്രിഡ്ഡില്‍ എട്ടാം സ്ഥാനത്തുള്ള തുടക്കവും 
കൊണ്ടു് ഫോഴ്സ് ഇന്ത്യക്കു് തൃപ്തിപ്പെടേണ്ടിവന്നു. മഴയ്ക്കുമുന്‍പു് നല്ലസമയം കണ്ടെത്താനായി ട്രാക്കില്‍ പൊരിഞ്ഞ പോരാട്ടം 
നടന്നതു് യോഗ്യതാറൗണ്ട് ആവേശകരമാക്കി. അവസാനം പതിവുപോലെ റെഡ്ബുള്ളിനുവേണ്ടി മാര്‍ക് വെബ്ബര്‍ 
പോള്‍ നേടിയപ്പോള്‍ മഴയത്തു നടത്തിയ മികച്ചൊരു പ്രകടനംകൊണ്ടു് ഹാമില്‍ട്ടണ്‍ മുനിരയില്‍ സ്ഥാനം നേടി. 
അലോണ്‍സൊയ്ക്ക് പത്താമതെത്താനെ കഴിഞ്ഞുള്ളൂ. എന്നാല്‍ രണ്ടാം ഫെറാരിയില്‍ ഫെലിപെ മസ്സ ആറാമതെത്തി. 
മൂന്നുമുതല്‍ അഞ്ചുവരെ സ്ഥാനങ്ങളില്‍ യഥാക്രമം കുബിത്സ, വെറ്റല്‍, ബട്ടണ്‍ എന്നിവര്‍ സ്ഥാനമുറപ്പിച്ചപ്പോള്‍ 
കോസ്‌വര്‍ത്ത് എന്‍ജിനുമായി വില്യംസ് ഏഴും ഒന്‍പതും സ്ഥാനങ്ങള്‍ നേടി.

ട്രാക്കിന്റെ പെരുമയ്ക്കൊത്ത പോരാട്ടമായിരുന്നു സ്പായില്‍ അരങ്ങേറിയതു്. ആദ്യലാപ്പില്‍ത്തന്നെ ഹാമില്‍ട്ടണ്‍ വെബ്ബറെ 
മറികടന്നതും വളരെ അടുത്തുള്ള ആദ്യവളവിന്റെ സമ്മര്‍ദ്ദം കാരണം ഏഴാംസ്ഥാനത്തേയ്ക്ക് വെബ്ബര്‍ 
പിന്തള്ളപ്പെട്ടുപോയതും തുടക്കം ആവേശകരമാക്കി. എന്നാല്‍ ആദ്യലാപ്പില്‍ ചെറുതായി ചാറിയ മഴ കാറുകളുടെ നിയന്ത്രണം തെറ്റിച്ചു. തന്റെ മുന്നൂറാമത്തെ റേസില്‍ ആദ്യലാപ്പില്‍ ബാരിക്കെല്ലോയ്ക്കു് അതു് 
പുറത്തേക്കുള്ള വഴിയുമൊരുക്കി. നനഞ്ഞ ട്രാക്കില്‍ നിയന്ത്രണംകിട്ടാതെ വലഞ്ഞ ബാരിക്കെല്ലോ അലോണ്‍സൊയുമായി 
കൂട്ടിയിടിക്കുകയായിരുന്നു. ഇതു് ആദ്യലാപ്പില്‍ത്തന്നെ സേഫ്റ്റികാര്‍ ഇറങ്ങുന്നതിനു കാരണമായി. പ്രൈം 
ടയറുകളിലായിരുന്ന പെട്രോവും ഷുമാക്കറും ബഹളങ്ങള്‍ മുതലെടുത്തു് ഈ സമയംകൊണ്ടു് മുനിരയില്‍ 
എത്തിയിരുന്നു.

അപകടത്തെത്തുടര്‍ന്നു് പിറ്റ് ചെയ്തു് ടയറുകള്‍ മാറാനുള്ള അലോണ്‍സൊയുടെ തീരുമാനം തിരിച്ചടിച്ചു. മഴ 
പിന്നെ ഒഴിഞ്ഞുനിന്നതോടെ ഒരിക്കല്‍ കൂടി പിറ്റ് ചെയ്തു് സ്ലിക് ടയറുകളിലേക്ക് മാറേണ്ടിവന്നു അദ്ദേഹത്തിനു്. ഈസമയം 
മുന്‍നിരയില്‍ മക്‌ലാരന്‍ കാറുകള്‍ ഒന്നും രണ്ടും സ്ഥാനങ്ങളിലും, പിന്നീടു് ക്രമമായി വെറ്റല്‍, കുബിത്സ, 
വെബ്ബര്‍ എന്നിവരുമായിരുന്നു. ആറാംസ്ഥാനത്തു് ഫെലിപെ മസ്സയുടെ ഫെറാരിയും. അപകടത്തിനും 
പിറ്റ്സ്റ്റോപ്പിനും ശേഷം പിന്‍നിരയിനിന്നും കടുത്തപോരാട്ടങ്ങളിലൂടെ അലോണ്‍സൊ മദ്ധ്യനിരയില്‍ തിരിച്ചെത്തി. പതിനൊന്നാം 
ലാപ്പില്‍ ഉഗ്രനൊരു മറികടക്കലിലൂടെ പെട്രോവും ഷുമാക്കറും ആദ്യപത്തിനുള്ളിലെത്തി. ടീം മേറ്റ് റൊസ്ബര്‍ഗിനെ 
മറികടക്കുമ്പോള്‍ ടയറുകള്‍ തമ്മില്‍ ചെറുതായി ഉരസിയെങ്കിലും വലിയ അപകടങ്ങളൊന്നുമില്ലാതെയാണു് ഷുമാക്കര്‍ 
ഇരുപതില്‍നിന്നും ആദ്യപത്തിലെത്തിയതു്.

പിന്നീടു് കുറച്ചുസമയം തണുത്തുപോവുകയും, ഒന്നും രണ്ടും സ്ഥാനങ്ങള്‍ നേടുമെന്നുതോന്നിച്ച റേസ് ബട്ടനും വെറ്റലും 
തമ്മിലുള്ള ഒരു കൂട്ടിയിടിയിലൂടെ വലിയ പ്രശ്നങ്ങളൊന്നുമില്ലാതെ മക്‌ലാരന്‍ വീണ്ടും ട്രാക്കില്‍ പോരാട്ടങ്ങളുയര്‍ത്തുകയും ചെയ്തു. 
പതിനേഴാം ലാപ്പില്‍ നടന്ന കൂട്ടിയിടി ശരിക്കും വെറ്റലിന്റെ പിഴവായിരുന്നു. എങ്കിലും അതു് ബട്ടന്റെ സ്പായിലെ പോരാട്ടം 
അവസാനിപ്പിച്ചു. പിറ്റ് ലേനിനടുത്തായതു് വെറ്റലിനു് പിറ്റ് സ്റ്റോപ്പെടുത്തു് റേസില്‍ തിരിച്ചുവരാന്‍ അവസരമൊരുക്കി. 
എന്നാല്‍ ഒരു ഡ്രൈവ് ത്രൂ പെനാല്‍റ്റി ലഭിച്ചതു് റെഡ്ബുള്‍ ഡ്രൈവറുടെ ബെല്‍ജിയന്‍ റേസിന്റെ കാര്യത്തില്‍ ഒരു 
തീരുമാനമുണ്ടാക്കിയെന്നു പറയാം. ഈ അപകടത്തോടെ ചാമ്പ്യന്‍ഷിപ്പ് പോരാട്ടത്തിലേര്‍പ്പെട്ടിരിക്കുന്ന അഞ്ചുപേരില്‍ 
മൂന്നുപേരും പോയിന്റിനു വെളിയിലായി.

വെറ്റല്‍ പിന്നീടു് പൊസിഷന്‍ തിരിച്ചുപിടിക്കാന്‍ ശ്രമിച്ചെങ്കിലും ഇരുപത്തിയേഴാം ലാപ്പില്‍ മറ്റൊരു അപകടവും, തുടര്‍ന്നു് 
പിന്‍ടയറുകളിലൊന്നു പഞ്ചറായതും കാര്യങ്ങള്‍ക്കു് ഏകദേശം തീരുമാനമാക്കി. ഈ സമയമൊക്കെയും ഹാമില്‍ട്ടണ്‍ ലീഡ് 
നിലനിര്‍ത്തിയിരുന്നു. കുബിത്സയും വെബ്ബറും തമ്മില്‍ രണ്ടാംസ്ഥാനത്തിനുവേണ്ടി കടുത്ത മത്സരവും ട്രാക്കില്‍ അരങ്ങേറി. 
മുപ്പതാം ലാപ്പെത്തിയപ്പോഴെയ്ക്കും മുന്‍നിര ഡ്രൈവര്‍മാരെല്ലാം ടയറുകള്‍ മാറ്റിയിരുന്നു. ആദ്യപത്തില്‍ ഷുമാക്കറും 
റൊസ്ബര്‍ഗും മാത്രമേ പിറ്റ് സ്റ്റോപ്പെടുക്കാതിരുന്നുള്ളു.

റേസിനവസാനം മഴപെയ്യുമ്പോള്‍ പിറ്റ് ചെയ്തു് ഗ്രൂവുകളുള്ള ഇന്റര്‍മീഡിയറ്റ് ടയറുകളിലേക്കു് മാറാനായിരുന്നു 
മെഴ്സിഡസിന്റെ തീരുമാനം. അതു് ദിവസത്തെ ഏറ്റവും നല്ല സ്ട്രാറ്റജികളിലൊന്നായി മാറുകയും ചെയ്തു. മുപ്പത്തിയഞ്ചാം 
ലാപ്പില്‍ മഴപെയ്തു തുടങ്ങിയപ്പോള്‍ എല്ലാവരും പിറ്റ് ചെയ്തു് ടയറുകള്‍ മാറി. പിറ്റ് ലേനില്‍ ചെറുതായി പിഴച്ച കുബിത്സയെ 
മുതലെടുത്തു് വെബ്ബര്‍ രണ്ടാംസ്ഥാനം പിടിച്ചെടുക്കുകയും ചെയ്തു. എന്നാല്‍ നിയന്ത്രണം നഷ്ടപ്പെട്ടു് ട്രാക്കിനു വിലങ്ങനെ 
കിടന്ന അലോണ്‍സൊയുടെ ഫെറാരി വീണ്ടും സേഫ്റ്റി കാറിനെ രംഗത്തെത്തിച്ചു. മഴയും സേഫ്റ്റികാറും കാര്യങ്ങള്‍ 
ദുര്‍ഘടമാക്കിയെങ്കിലും, സേഫ്റ്റി കാര്‍ പിന്‍വലിച്ചയുടനെ നടത്തിയ ഉഗ്രനൊരു നീക്കത്തിലൂടെ കൊബിയാഷിയേയും 
ഷുമാക്കറെയും റൊസ്ബര്‍ഗ് ഒരുമിച്ചു മറികടന്നു.

സംഭവബഹുലമായ ബെല്‍ജിയന്‍ ഗ്രാന്‍പ്രീ ഹാമില്‍ട്ടണു് കിരീടം നേടിക്കൊടുത്തതിനൊപ്പം ചാമ്പ്യന്‍ഷിപ്പ് 
പോരാട്ടങ്ങളിലെ കടുപ്പം കുറച്ചു കുറയ്ക്കുകയും ചെയ്തു. സ്പായ്ക്കുമുന്‍പു് ആദ്യ അഞ്ചു സ്ഥാനക്കാര്‍ വെറും 20 പോയിന്റായിരുന്നു 
പിടിച്ചിരുന്നതു്. ഇപ്പോള്‍ അതു് നാല്‍പ്പതു പോയിന്റായി ഉയര്‍ന്നു. പോരാട്ടത്തില്‍ ഹാമില്‍ട്ടണും (182) വെബ്ബറും (179) 
ഒപ്പത്തിനൊപ്പമാണെങ്കില്‍, വെറ്റലും അലോണ്‍സൊയും ബട്ടണുമെല്ലാം അവരവരുടെ പഴയ സ്ഥാനങ്ങളില്‍ത്തന്നെയാണു്.
സുട്ടില്‍ ഒരു പോയിന്റു വ്യത്യാസത്തില്‍ മൈക്കല്‍ ഷുമാക്കറെ മറികടന്നു് ഒന്‍പതാമെത്തിയതാണു് പിന്നെയൊരു 
മാറ്റമെന്നു പറയാവുന്നതു്. കണ്‍സ്ട്രക്റ്റര്‍മാരുടെ കാര്യത്തില്‍ റെഡ്ബുള്‍ (330) ഒരു പോയിന്റിനു് മക്‌ലാരനുമുന്നിലാണു്. 
മൂന്നാമത് ഫെറാരിയും (250).

ഇനി മോണ്‍സയിലെ ഇറ്റാലിയന്‍ ഗ്രാന്‍പ്രീ കഴിഞ്ഞാല്‍ നാലു ഏഷ്യന്‍ റേസുകളും സാവോപോളോയുമാണു് ബാക്കി. 
ബെല്‍ജിയത്തിനു മുമ്പുണ്ടായിരുന്ന അവസ്ഥയിലേക്കു് കാര്യങ്ങളെത്താന്‍ വിഷമമാണെങ്കിലും, അങ്ങനെ 
സംഭവിക്കുകയും അവസാന രണ്ടു റേസുകള്‍ നിര്‍ണ്ണായകമാവുകയും ചെയ്യുന്നതു് ആരാധകര്‍ക്കു് ആവേശം 
പകര്‍ന്നേക്കും.

പിന്‍കുറിപ്പു്: ഇന്ത്യന്‍ ഗ്രാന്‍പ്രീ കൂടി കലണ്ടറിലിടം പിടിച്ചതോടെ ഏതാണ്ടു് യൂറൊപ്പിനൊപ്പംതന്നെ റേസുകള്‍ ഇപ്പോള്‍ 
ഏഷ്യയിലും എത്തിയിരിക്കുന്നു. ഏഷ്യന്‍ ട്രാക്കുകള്‍ കുറച്ചുകൂടി നല്ല മത്സരങ്ങള്‍ കാഴ്ചവെച്ചാല്‍ ഭാവിയില്‍ ഫോര്‍മുല 
വണ്ണിന്റെ വളയം ഏഷ്യന്‍ വമ്പന്‍മാരിലെത്തിക്കൂടെന്നില്ല. യൂറോപ്പിലെ പല ട്രാക്കുകളും ജനങ്ങളുടെ അതൃപ്തി 
നേരിടുന്നുവെന്നതും സുരക്ഷയുടെ കാര്യത്തില്‍ പഴയട്രാക്കുകള്‍ക്ക് പല പ്രശ്നങ്ങളുമുണ്ടെന്നതും കൂടി കണക്കിലെടുത്താല്‍,
ആഞ്ഞുപിടിച്ചാല്‍ ഇംഗ്ലണ്ടിലിരിക്കുന്ന ഫോര്‍മുല വണ്‍ ഫാക്റ്ററികളെല്ലാം ക്വാലാലംപൂരിലോ ഡെല്‍ഹിയിലോ 
ഷാങ്ഹായിലോ എത്തിക്കാവുന്നതേയുള്ളൂ.

\begin{flushright}(11 September, 2010)\footnote{http://malayal.am/വിനോദം/കായികം/7894/പോയിന്റ്-നിലയിലെ-വ്യത്യാസം-കൂട്ടിയ-ബെല്‍ജിയന്‍-ഗ്രാന്‍പ്രി}\end{flushright}

\newpage
