\secstar{ജീവിക്കുവാനുള്ള കാരണങ്ങള്‍}
{\vskip 2pt}

% It would be good to have this in a box
‌\begin{quotation}
``ലളിത എന്ന ഇന്‍പുട്ട് മെഥേഡിന്റെ ഉപജ്ഞാതാവ്, 2007ലെ ഗൂഗിള്‍ സമ്മര്‍ ഓഫ് കോഡ് പ്രോജക്ടില്‍ എസ്എംസിയെ 
പ്രതിനിധീകരിച്ച അഞ്ചുവിദ്യാര്‍ത്ഥികളില്‍ ഒരാള്‍, ശില്‍പ്പ പ്രോജക്ടിലും pypdflibലും ഒട്ടേറെ കമ്മിറ്റുകള്‍, മലയാളം ഒസിആര്‍ 
വികസിപ്പിക്കാനുള്ള രണ്ടുവ്യത്യസ്ത പരിശ്രമങ്ങളില്‍ പങ്കാളി, ഗ്നൂ താളുകള്‍ മലയാളത്തിലാക്കുന്ന യജ്ഞത്തിന്റെ കാര്‍മികന്‍, 
ഭാഷാസാങ്കേതികവിദ്യയിലെ മാനകീകരണ പ്രവര്‍ത്തനങ്ങളില്‍ മുന്‍നിരപ്പോരാളി''
\end{quotation}

{\vskip 12pt}

എ­ത്ര­ത­വണ ഈ കു­റി­പ്പെ­ഴു­താ­നി­രു­ന്നി­ട്ടു് കര­ഞ്ഞു­വീര്‍­ത്ത കണ്ണു­ക­ളു­മാ­യി എഴു­ന്നേ­റ്റു­പോ­യെ­ന്ന­റി­യി­ല്ല. ഏതു സ്വ­കാ­ര്യ­ദുഃ­ഖ­ത്തേ­യും 
സ്റ്റോ­റി­യാ­യി മാ­ത്രം കണ്ടു­പ­രി­ച­യ­മു­ള്ള ഒരു പ്രൊ­ഫ­ഷ­നില്‍ ഇങ്ങ­നെ സം­ഭ­വി­ക്കാന്‍ പാ­ടു­ള്ള­ത­ല്ല. പക്ഷെ ജി­നേ­ഷി­ന്റെ 
കാ­ര്യ­ത്തില്‍ നി­യ­മ­ങ്ങള്‍ തെ­റ്റു­ന്നു.

ഇ­ന്ന­ലെ ഉച്ച­യ്ക്കാ­ണു് ജി­നേ­ഷി­ന്റെ വി­യോ­ഗ­മ­റി­യു­ന്ന­ത്. വെ­ല്ലൂര്‍ സി­എം­സി­യില്‍ ക്യാന്‍­സ­റി­നോ­ടു­പൊ­രു­തി­ത്തോ­റ്റ ഞങ്ങ­ളു­ടെ 
പ്രി­യ­പ്പെ­ട്ട സഹോ­ദ­ര­നു് ഇന്നു മണ്ണാര്‍­ക്കാ­ടു് അന്ത്യ­വി­ശ്ര­മ­മാ­യി. ജി­നേ­ഷ് കാ­ഞ്ഞി­ര­ങ്ങാ­ട്ടില്‍ ജയ­രാ­മന്‍ എന്ന ജിന്‍സ്ബോണ്ടിനു്
മല­യാ­ള­ത്തി­ന്റെ കണ്ണീ­രില്‍­ക്കു­തിര്‍­ന്ന യാ­ത്രാ­മൊ­ഴി­.

അ­റി­യി­ല്ല, എന്തൊ­ക്കെ­യാ­ണെ­ഴു­തേ­ണ്ട­തെ­ന്നു്. പല­ത­ര­ത്തി­ലു­ള്ള ബന്ധ­മാ­യി­രു­ന്നു, എനി­ക്കു് ജി­നേ­ഷു­മാ­യി ഉണ്ടാ­യി­രു­ന്ന­തു്. 
സ്വ­ത­ന്ത്ര ­മ­ല­യാ­ളം­ കമ്പ്യൂ­ട്ടി­ങ്ങി­ലെ സജീ­വാം­ഗം എന്ന നി­ല­യി­ലു­ള്ള ബന്ധ­മാ­ണു് ആദ്യ­ത്തേ­തു്. അതു­വ­ഴി ജി­നേ­ഷി­ന്റെ 
ബ്ലോ­ഗി­ലേ­ക്കും എത്തി­പ്പെ­ട്ടു. The log book of an observer എന്നാ­യി­രു­ന്നു അവ­ന്റെ ബ്ലോ­ഗി­നു പേ­രി­ട്ടി­രു­ന്ന­തു്. നി­രീ­ക്ഷ­ക­ന്റെ 
ആ നാള്‍­വ­ഴി­പ്പു­സ്ത­കം ഇന്നു് ഇന്‍­വൈ­റ്റ­ഡ് റീ­ഡേ­ഴ്സി­നു വേ­ണ്ടി മാ­ത്രം തു­റ­ന്നി­ട്ടി­രി­ക്ക­യാ­ണു്. അതി­ലെ കു­റി­പ്പു­കള്‍ 
എവി­ടെ­പ്പോ­യോ എന്തോ­... ബ്ലോ­ഗ് വാ­യ­ന­യി­ലൂ­ടെ ശക്ത­മായ ബന്ധം പി­ന്നീ­ടു് ഗാ­ഢ­മായ സു­ഹൃ­ദ്ബ­ന്ധ­മാ­യി മാ­റി­.

മലയാളം എന്ന ഈ വെ­ബ്സൈ­റ്റ് തു­റ­ന്ന­തോ­ടെ തു­ട­ക്കം­മു­തല്‍ തന്നെ ഞങ്ങ­ളു­ടെ ടീ­മില്‍ ഒരം­ഗ­മാ­യി ജി­നേ­ഷ് മാ­റി. 
മലയാളരാജ്യത്തിനുവേണ്ടി ഫോര്‍­മുല വണ്‍ റേ­സു­കള്‍ റി­വ്യൂ ചെ­യ്തു. ഐപി­എല്‍ എന്ന കാ­യി­ക­മാ­മാ­ങ്ക­ത്തി­ന്റെ വാ­തില്‍­പ്പു­റം 
കളി­ക­ളെ­ക്കു­റി­ച്ചെ­ഴു­തി. ആസ്റ്റെ­റി­ക്സ് എന്ന കാര്‍­ട്ടൂണ്‍ പര­മ്പ­ര­യെ­ക്കു­റി­ച്ചും ദി ബി­ഗ് ബാ­ങ് തി­യ­റി എന്ന സി­റ്റ്കോ­മി­നെ­ക്കു­റി­ച്ചും 
സ്വ­ത്വ­രാ­ഷ്ട്രീ­യ­ത്തെ­ക്കു­റി­ച്ചും മാ­ദ്ധ്യ­മ­ങ്ങ­ളു­ടെ പരി­ണാ­മ­ത്തെ­ക്കു­റി­ച്ചും ഒക്കെ ഒരേ തീ­വ്ര­ത­യോ­ടെ ജി­നേ­ഷ് എഴു­തി. തല­മു­ടി­യെ­ക്കു­റി­ച്ച് 
ഒരു­പ­ന്യാ­സം എന്ന ഏറെ­ക്കു­റെ പേ­ഴ്സ­ണ­ലായ കു­റി­പ്പും മല­യാ­ള­ത്തി­ലെ­ഴു­തി. ഇവി­ടെ ഞങ്ങള്‍ പ്ര­സി­ദ്ധീ­ക­രി­ക്കു­ന്ന സ്റ്റോ­റി­ക­ളില്‍ 
ഫാ­ക്ച്വല്‍ എറേ­ഴ്സ് വരു­മ്പോ­ളെ­ല്ലാം അതു­തി­രു­ത്താന്‍ ഓടി­യെ­ത്തി. എഴു­തി­യ­തി­നേ­ക്കാള്‍ കൂ­ടു­തല്‍ ജി­മെ­യ്ല്‍ ചാ­റ്റു­ക­ളില്‍ 
പറ­ഞ്ഞു­തീര്‍­ത്തു­.

­ഫി­ലോ­സ­ഫി, പൊ­ളി­റ്റി­ക്സ്, സൊ­സൈ­റ്റി, സയന്‍­സ്, കമ്പ്യൂ­ട്ടേ­ഷ­നല്‍ ലി­ങ്വ­സ്റ്റി­ക്സ്, മെ­ട്രോ സെ­ക്ഷ്വാ­ലി­റ്റി എന്നി­ങ്ങ­നെ വി­വിധ 
വി­ഷ­യ­ങ്ങ­ളില്‍ പടര്‍­ന്നു­കി­ട­ന്നു, ഞങ്ങ­ളു­ടെ സം­ഭാ­ഷ­ണ­ങ്ങള്‍. രോ­ഗ­ക്കി­ട­ക്ക­യില്‍ അനാ­രോ­ഗ്യ­ത്തി­ന്റെ ഇട­വേ­ള­ക­ളില്‍ 
അല്‍­പ്പാല്‍­പ്പം സം­സാ­രി­ക്കാ­റാ­വു­മ്പോ­ഴെ­ല്ലാം ജി­ചാ­റ്റില്‍ വന്നു­നി­റ­ഞ്ഞൂ, ജി­നേ­ഷ്. തന്നെ കാര്‍­ന്നു­തി­ന്നു­ന്ന ക്യാന്‍­സ­റി­നെ­ക്കു­റി­ച്ച്, 
ക്യാന്‍­സ­റി­ന്റെ വെ­ളി­ച്ച­ത്തില്‍ ജീ­വി­ത­ത്തെ­ക്കു­റി­ച്ച്, തന്റെ­യ­ടു­ത്ത ബഡ്ഡില്‍ കി­ട­ക്കു­ന്ന രോ­ഗി­ണി­യായ പെണ്‍­കു­ട്ടി­യെ­ക്കു­റി­ച്ച് 
ഒക്കെ ഇം­ഗ്ലീ­ഷില്‍ ഏതാ­നും കു­റി­പ്പു­ക­ളെ­ഴു­തി, അവ എനി­ക്ക­യ­ച്ചു­ത­ന്നു. മല­യാ­ള­ത്തി­ലേ­ക്കു് തീ­വ്ര­ത­കു­റ­യാ­തെ പരി­ഭാഷ ചെ­യ്യാന്‍ 
അവ ബ്ലോ­ഗി­ലൂ­ടെ പരി­ചി­ത­നായ ഡോ. സൂ­ര­ജി­നെ ഏല്‍­പ്പി­ച്ചു. സൂ­ര­ജു­മാ­യി തന്റെ മെ­ഡി­ക്കല്‍ ഹി­സ്റ്റ­റി പങ്കു­വ­ച്ചു. മര­ണ­മ­ല്ലാ­തെ 
മറ്റൊ­രു രക്ഷാ­മാര്‍­ഗ്ഗ­മി­ല്ലാ­ത്ത അര്‍­ബു­ദ­വ­ക­ഭേ­ദ­മാ­യി­രു­ന്നു, ജി­നേ­ഷി­ന്റേ­തു്. അക്കാ­ര്യം ഡോ­ക്ടര്‍­മാ­രില്‍ നി­ന്നു കൃ­ത്യ­മാ­യി 
ജി­നേ­ഷ് മന­സ്സി­ലാ­ക്കി­യി­രു­ന്നു. സൂ­ര­ജും അതു­ത­ന്നെ എന്നോ­ടു­പ­റ­ഞ്ഞ­പ്പോള്‍ ഞാന്‍ നീ­റിയ നീ­റ്റല്‍ ...

എ­ന്നാല്‍ ഇതൊ­ന്നും വലിയ കാ­ര്യ­മ­ല്ല എന്ന മട്ടി­ലാ­യി­രു­ന്നു, ജി­നേ­ഷ്. തന്റെ ആരോ­ഗ്യ­ത്തെ­ക്കു­റി­ച്ചും ചി­കി­ത്സ­യെ­ക്കു­റി­ച്ചും 
ചോ­ദി­ക്കു­ന്ന സൈ­ബര്‍ സു­ഹൃ­ത്തു­ക്കള്‍­ക്കു ജി­നേ­ഷ് കാ­ട്ടി­ക്കൊ­ടു­ത്തി­രു­ന്ന­തു് xkcd­യി­ലെ രണ്ടു കാര്‍­ട്ടൂ­ളു­ക­ളാ­യി­രു­ന്നു 
(http://xkcd.com/931/  , http://xkcd.com/938/) എന്നു് അനി­വര്‍ അര­വി­ന്ദ് എ­സ്എം­സി­ മെ­യി­ലി­ങ് ലി­സ്റ്റില്‍ എഴു­തിയ 
ചെ­റു­കു­റി­പ്പില്‍\footnote{See article 'Anivar's email to Swathanthra Malayalam Computing'} അനു­സ്മ­രി­ക്കു­ന്നു­. 

നിര്‍­ത്താ­ത്ത പനി­യും നടു­വേ­ദ­ന­യും കാ­ലി­നു കഴ­പ്പും ചു­മ­യു­മൊ­ക്കെ­യാ­യി­ട്ടാ­ണു്, ജി­നേ­ഷി­നെ ആദ്യം പരി­ശോ­ധ­ന­യ്ക്കു 
കൊ­ണ്ടു­പോ­കു­ന്ന­തു്. അകാ­ര­ണ­മായ ക്ഷീ­ണ­മാ­യി­രു­ന്നു, മറ്റൊ­രു ലക്ഷ­ണം. ആശു­പ­ത്രി­യി­ലെ­ത്തു­മ്പോ­ഴേ­ക്കും ഈ 
കാ­ര­ണ­ങ്ങള്‍ വച്ചു് ജി­നേ­ഷ് അര്‍­ബു­ദം ഊഹി­ച്ചി­രു­ന്നു. കണ്‍­ഫേം ചെ­യ്ത­ശേ­ഷം അതേ­ക്കു­റി­ച്ചു് ഒരു ഗവേ­ഷ­ക­നു­മാ­ത്രം 
കഴി­യു­ന്ന നിര്‍­മ്മ­മ­ത­യോ­ടെ വാ­യി­ച്ചു­പ­ഠി­ച്ചു. പു­തിയ പു­തിയ മെ­ഡി­ക്കല്‍ ജേ­ണ­ലു­ക­ളില്‍ അതേ­ക്കു­റി­ച്ചു­ള്ള വി­വ­ര­ങ്ങള്‍ പര­തി. 
താന്‍ പെ­ട്ടി­രി­ക്കു­ന്ന അവ­സ്ഥ കൃ­ത്യ­മാ­യി മന­സ്സി­ലാ­ക്കി­.

­ജി­നേ­ഷി­നു് താന്‍ തി­രി­ച്ചു­വ­രി­ല്ലെ­ന്നു് അറി­യാ­മാ­യി­രു­ന്നു. അന്ത്യ­ശ്വാ­സം വരേ­യും അര്‍­ബു­ദ­ത്തെ അവ­ഗ­ണി­ച്ചും താന്‍ 
ചെ­യ്തു­കൊ­ണ്ടി­രു­ന്ന കാ­ര്യ­ങ്ങ­ളില്‍ വല്ലാ­ത്തൊ­രാ­സ­ക്തി­യോ­ടെ വ്യാ­പൃ­ത­നാ­വാന്‍ ആ തി­രി­ച്ച­റി­വു് ജി­നേ­ഷി­നു ബലം 
പക­രു­ക­യാ­ണു­ണ്ടാ­യ­തു്. മര­ണ­ഭീ­തി­യില്‍ തകര്‍­ന്നി­ല്ലാ­തെ­യാ­കു­ന്ന­താ­യി­രു­ന്നി­ല്ല, ഭാ­സു­ര­മായ ഭാ­വി­യെ­ക്കു­റി­ച്ചു മാ­ത്രം 
ചി­ന്തി­ക്കു­ന്ന­താ­യി­രു­ന്നു, ആ നി­ശ്ച­യ­ദാര്‍­ഢ്യം. താന്‍ ചെ­യ്യു­ന്ന കാ­ര്യ­ങ്ങള്‍ തനി­ക്കു് ഉപ­കാ­ര­പ്പെ­ട്ടി­ല്ലെ­ങ്കി­ലും മറ്റു­ള്ള­വര്‍­ക്കു് 
ഉപ­കാ­ര­പ്പെ­ടു­മെ­ന്നു് ജി­നേ­ഷി­നു് ഉറ­പ്പു­ണ്ടാ­യി­രു­ന്നു. അതില്‍ കണ്ടെ­ത്തിയ ആന­ന്ദ­മാ­ണു് വേ­ദ­ന­യെ അല്‍­പ്പാല്‍­പ്പ­മെ­ങ്കി­ലും 
മറി­ക­ട­ക്കാന്‍ സഹാ­യി­ച്ച­തു്.

ഇ­ട­യ്ക്കു് ക്യാന്‍­സര്‍ ശരീ­ര­ത്തില്‍­നി­ന്നു പൂര്‍­ണ്ണ­മാ­യി മാ­റി­യെ­ന്നു പറ­ഞ്ഞ ഘട്ട­ത്തില്‍ മാ­ത്ര­മാ­ണു് ജി­നേ­ഷ് 
സ്വ­സ്ഥ­ജീ­വി­ത­ത്തി­ലേ­ക്കു­ള്ള മട­ക്കം പ്ര­തീ­ക്ഷി­ച്ച­തു്. അപ്പോ­ഴാ­ക­ട്ടെ, ജീ­വി­താ­സ­ക്തി അതി­ന്റെ എല്ലാ­ത്തി­ള­ക്ക­ത്തോ­ടും 
വന്നു­ദി­ക്കു­ന്ന­തു­ക­ണ്ടു. രോ­ഗ­മ­ട­ങ്ങി പാ­ല­ക്കാ­ടു വീ­ട്ടി­ലെ­ത്തി­യാ­ലു­ടന്‍ കൂ­ട്ടു­കാ­രെ­ക്കൂ­ട്ടി ഒരു ഒത്തു­കൂ­ടല്‍ നട­ത്തു­ന്ന­തി­നെ­ക്കു­റി­ച്ചു­വ­രെ 
സെ­പ്തം­ബ­റില്‍ സം­സാ­രി­ച്ചു­.

­കീ­മോ­യ്ക്കു ശേ­ഷ­മു­ള്ള ബല­ഹീ­ന­ത­യു­ടെ നാ­ളു­ക­ളില്‍ കു­ഴല്‍­ഭ­ക്ഷ­ണം മാ­ത്രം ഉള്ളി­ലേ­ക്കെ­ടു­ക്ക­വേ കമ്പ്യൂ­ട്ടര്‍ ഉപ­യോ­ഗി­ക്കാ­നാ­വാ­തെ 
വന്ന­പ്പോ­ഴും ഞങ്ങ­ളോ­ടൊ­ക്കെ ഓരോ പു­സ്ത­ക­ത്തി­ന്റെ പേ­രു­പ­റ­ഞ്ഞു് അവ വാ­ങ്ങി അയ­ച്ചു­ത­രാന്‍ ആവ­ശ്യ­പ്പെ­ട്ടു. ചരി­ത്ര­വും 
സമൂ­ഹ­വു­മാ­യി ബന്ധ­പ്പെ­ട്ട ഒട്ടേ­റെ പു­സ്ത­ക­ങ്ങ­ളാ­ണു് ഇക്കാ­ല­യ­ള­വില്‍ ജി­നേ­ഷ് കു­ടി­ച്ചു­വ­റ്റി­ച്ച­തു്. അവ­യില്‍ നി­ന്നെ­ല്ലാം 
വ്യ­ത്യ­സ്ത­മാ­യി കസാന്‍­ദ്സാ­ക്കീ­സി­ന്റെ റി­പ്പോര്‍­ട്ട് ടു ഗ്രീ­ക്കോ എന്ന ആത്മ­ക­ഥാ­പു­സ്ത­ക­ത്തി­ലും മു­ങ്ങി­ത്താ­ഴ്ന്നു. ഓരോ വാ­യ­ന­യി­ലും 
പു­തു­മ­സ­മ്മാ­നി­ക്കു­ന്ന ഓരോ അട­രു­ക­ളി­ലും ജീ­വി­ത­കാ­മന കലര്‍­ന്നി­രി­ക്കു­ന്ന ഇപ്പു­സ്ത­കം വാ­യി­ക്കാ­നെ­ന്തേ, ഇത്ര­യും താ­മ­സി­ച്ചൂ 
എന്നു് എന്നോ­ടു് അത്ഭു­തം­കൂ­റി­.   അതേ തീ­ത്തി­ള­ക്ക­ത്തോ­ടെ ചി­ന്ത­യില്‍ വി­സ്ഫോ­ട­നം സൃ­ഷ്ടി­ച്ച തത്വ­ചി­ന്ത­ക­രേ­യും ഉള്‍­ക്കൊ­ണ്ടു­.

എഫ്ഇ­സി പോ­ലെ­യു­ള്ള സൈ­ബര്‍ ഇട­ങ്ങ­ളില്‍ വല്ല­പ്പോ­ഴും മാ­ത്രം വാ­യ്‌­തു­റ­ന്നു. പറ­യാന്‍ കാ­ര്യ­മി­ല്ലാ­ത്ത­താ­യി­രു­ന്നി­ല്ല, 
നി­റ്റ്പി­ക്കി­ങ് തീ­രെ താ­ത്പ­ര്യ­മി­ല്ലാ­തി­രു­ന്ന­താ­ണു് ഈ ഒതു­ങ്ങ­ലി­നു കാ­ര­ണം. തനി­ക്കു് പറ­യാ­നു­ള്ള­തു് സു­ഹൃ­ത്തു­ക്ക­ളോ­ടു 
പറ­ഞ്ഞു­തീര്‍­ക്കു­ക­യാ­യി­രു­ന്നു, ജി­നേ­ഷ്. എന്നി­ട്ടും ഫോര്‍­ത്ത് എസ്റ്റേ­റ്റ് ക്രി­ട്ടി­ക്‍ എന്ന ഡി­സ്ക­ഷന്‍ ഗ്രൂ­പ്പില്‍ ജി­നേ­ഷ് എഴു­തിയ 
എണ്ണി­പ്പെ­റു­ക്കിയ മെ­യി­ലു­കള്‍ സു­ചി­ന്തി­ത­വും ആലോ­ച­നാ­മൃ­ത­വു­മായ വാ­ദ­മു­ഖ­ങ്ങള്‍­ക്കു കു­ന്ത­മു­ന­ക­ളാ­യി. ലോ വെ­യ്സ്റ്റ് 
ജീന്‍­സി­നെ­ക്കു­റി­ച്ചും സി­നി­മാ­റ്റി­ക്‍ ഡാന്‍­സി­നെ­ക്കു­റി­ച്ചും ആഫ്റ്റര്‍ മാ­ച്ച് പാര്‍­ട്ടി­ക­ളെ­ക്കു­റി­ച്ചും യു­ഐ­ഡി­യെ­ക്കു­റി­ച്ചും ഒക്കെ 
ജി­നേ­ഷ് പങ്കു­വ­ച്ച അഭി­പ്രാ­യ­ങ്ങള്‍ കണ്‍­സര്‍­വേ­റ്റീ­വ് മനോ­ഘ­ട­ന­ക­ളു­ടെ സ്വാ­സ്ഥ്യം­കെ­ടു­ത്തി. ഷാ­മ്പൂ ചെ­യ്തു മനോ­ഹ­ര­മാ­ക്കി 
സൂ­ക്ഷി­ച്ചി­രു­ന്ന തന്റെ നീ­ളന്‍ മു­ടി­യെ­ക്കു­റി­ച്ചും കീ­മോ­യു­ടെ ഫല­മാ­യി അതു­കൊ­ഴി­ഞ്ഞ­തി­നെ­ക്കു­റി­ച്ചു­മൊ­ക്കെ­യെ­ഴു­തിയ 
ഹൃ­ദ­യ­ഭേ­ദ­മായ കു­റി­പ്പാ­ണു് എഫ്ഇ­സി­യില്‍ ജി­നേ­ഷ് അവ­സാ­ന­മാ­യി എഴു­തിയ മെ­യി­ലു­ക­ളില്‍ ഒന്നു്. യൂ­ണി­ക്കോ­ഡ് 
മെ­യി­ലി­ങ് ലി­സ്റ്റി­ലും എസ്എം­സി ലി­സ്റ്റി­ലു­മാ­യി ചി­ല്ല­ക്ഷ­ര­വി­വാ­ദം കൊ­ടു­മ്പി­രി­ക്കൊ­ണ്ട കാ­ല­ത്തു് ജി­നേ­ഷ് എഴു­തിയ വരി­ക­ളും 
ഓര്‍­ക്കാ­തെ­വ­യ്യ.

അ­തേ നി­മി­ഷം, ജീ­വി­ച്ച 24 വര്‍­ഷ­ങ്ങ­ളി­ലും അവ­നോ­ടു് അടു­ത്തു­പ­രി­ച­യ­പ്പെ­ടാന്‍ സാ­ധി­ച്ച ഓരോ­രു­ത്തര്‍­ക്കും അഭി­മാ­ന­ത്തോ­ടെ 
ഓര്‍­ക്കാ­നു­ള്ള­തു­മാ­ത്രം മി­ച്ചം­വ­ച്ചു. ചൂ­ടേ­റിയ സൈ­ബര്‍ തര്‍­ക്ക­ങ്ങ­ളില്‍ വല്ല­പ്പോ­ഴും തല­യി­ട്ടു് അഭി­പ്രാ­യം പറ­ഞ്ഞി­രു­ന്ന ഈ 
ചെ­റു­പ്പ­ക്കാ­ര­നോ­ടു കാ­ലു­ഷ്യം സൂ­ക്ഷി­ക്കാന്‍ എതി­ര­ഭി­പ്രാ­യ­ക്കാര്‍­ക്കു­പോ­ലും കഴി­ഞ്ഞി­രു­ന്നി­ല്ല. അത്ര­യ്ക്കു സൌ­മ്യ­ദീ­പ്ത­മാ­യി 
സം­ഭാ­ഷ­ണ­ങ്ങ­ളെ ഒരു­ക്കി­യെ­ടു­ക്കാന്‍ ജി­നേ­ഷി­നു കഴി­ഞ്ഞി­രു­ന്നു­.

­മ­ല­യാ­ളം കമ്പ്യൂ­ട്ടി­ങ്ങി­നു് ജി­നേ­ഷ് നല്‍­കിയ സം­ഭാ­വ­ന­ക­ളെ എണ്ണി­പ്പ­റ­യാ­തെ ഈ കു­റി­പ്പു് അവ­സാ­നി­പ്പി­ക്കാ­നാ­വി­ല്ല. 
ഇന്ത്യന്‍ ഭാ­ഷ­കള്‍­ക്കാ­യി സി­ഡാ­ക്‍ വി­ക­സി­പ്പി­ച്ച ഇന്‍­സ്ക്രി­പ്റ്റ് ഇന്‍­പു­ട്ട് മെ­ഥേ­ഡി­ന്റെ­യും ആം­ഗ­ലേ­യ­ത്തില്‍ മല­യാ­ള­മെ­ഴു­തു­ന്ന 
മൊ­ഴി­യും സ്വ­ന­ലേ­ഖ­യും അട­ക്ക­മു­ള്ള phoenomic രീ­തി­ക­ളു­ടെ­യും സങ്ക­ല­ന­മാ­യി ഗ്നൂ ലി­ന­ക്സ് ഓപ്പ­റേ­റ്റി­ങ് സി­സ്റ്റ­ങ്ങ­ളില്‍ ലഭ്യ­മായ 
ലളിത എന്ന ഇന്‍­പു­ട്ട് മെ­ഥേ­ഡ് ജി­നേ­ഷി­ന്റെ സൃ­ഷ്ടി­യാ­ണു്. 2007­ലെ ഗൂ­ഗിള്‍ സമ്മര്‍ ഓഫ് കോ­ഡ് പ്രോ­ജ­ക്ടി­ലേ­ക്കു് എസ്എം­സി 
തി­ര­ഞ്ഞെ­ടു­ത്ത അഞ്ചു­വി­ദ്യാര്‍­ത്ഥി­ക­ളില്‍ ഒരാ­ളാ­യാ­ണു് ജി­നേ­ഷ് സ്വ­ത­ന്ത്ര മല­യാ­ളം കമ്പ്യൂ­ട്ടി­ങ്ങില്‍ സജീ­വ­മാ­കു­ന്ന­തു്. ഗ്നൂ 
പ്രോ­ജ­ക്ടി­ന്റെ താ­ളു­കള്‍ മല­യാ­ള­ത്തി­ലേ­ക്കു പരി­ഭാ­ഷ­പ്പെ­ടു­ത്തു­ന്ന ടീ­മി­ന്റെ ചു­മ­തല വഹി­ച്ചി­രു­ന്ന ശ്യാം ആത്മ­ഹ­ത്യ 
ചെ­യ്ത­തി­നെ­ത്തു­ടര്‍­ന്നു് ആ പ്രോ­ജ­ക്ടി­ന്റെ കണ്‍­വീ­നര്‍ ആയും ജി­നേ­ഷ് പ്ര­വര്‍­ത്തി­ച്ചു. ശ്യാം കോ­ഡി­നേ­റ്റ് ചെ­യ്ത പരി­ഭാ­ഷ­ക­ളെ 
പു­സ്ത­ക­മാ­ക്കുക എന്ന ലക്ഷ്യ­ത്തോ­ടെ അവ എഡി­റ്റ് ചെ­യ്യു­ന്ന കര്‍­ത്ത­വ്യ­വും ജി­നേ­ഷ് നിര്‍­വ്വ­ഹി­ച്ചു. കോ എഡി­റ്റ­റാ­വാന്‍ 
എന്നോ­ടാ­വ­ശ്യ­പ്പെ­ട്ടി­രു­ന്നെ­ങ്കി­ലും സ്വ­ത­സി­ദ്ധ­മായ മടി­യും മറ്റു­തി­ര­ക്കു­ക­ളും മൂ­ലം എനി­ക്കു കഴി­ഞ്ഞ­തു­മി­ല്ല.

­മ­ല­യാ­ള­ത്തി­ലാ­രം­ഭി­ച്ച് സന്തോ­ഷ് തോ­ട്ടി­ങ്ങ­ലി­ന്റെ­യും വാ­സു­ദേ­വി­ന്റെ­യും മറ്റും നേ­തൃ­ത്വ­ത്തില്‍ ഇന്ത്യ­യി­ലും വി­ദേ­ശ­ത്തു­മു­ള്ള ഒട്ടേ­റെ 
ഭാ­ഷ­ക­ളി­ലേ­ക്കു് വളര്‍­ന്ന ശില്‍പ്പ പ്രോ­ജ­ക്ടില്‍ തു­ട­ക്കം മു­തല്‍ തന്നെ ജി­നേ­ഷി­ന്റെ പങ്കാ­ളി­ത്ത­മു­ണ്ടാ­യി­രു­ന്നു. ശില്‍­പ്പ­യെ­ക്കു­റി­ച്ച് 
ഒട്ടേ­റെ പ്ര­തീ­ക്ഷ­കള്‍ വച്ചു­പു­ലര്‍­ത്തു­ന്ന­യാ­ളെ­ന്ന നി­ല­യില്‍ ഈ പ്രോ­ജ­ക്ടില്‍ ജി­നേ­ഷ് അവ­ത­രി­പ്പി­ച്ച മാ­റ്റ­ങ്ങ­ളും 
നിര്‍­ദ്ദേ­ശ­ങ്ങ­ളു­മൊ­ക്കെ എത്ര­മാ­ത്രം വി­ല­പ്പെ­ട്ട­താ­ണെ­ന്നു തി­രി­ച്ച­റി­യു­ന്നു­.

ഹൈ­ദ­രാ­ബാ­ദ് ഐഐ­ഐ­ടി­യില്‍ ജി­നേ­ഷ് ഉള്‍­പ്പെ­ടു­ന്ന ഗവേ­ഷ­ക­സം­ഘം മല­യാ­ളം ഒ­സി­ആര്‍ വി­ക­സി­പ്പി­ക്കാ­നു­ള്ള 
പരി­ശ്ര­മ­ത്തി­ലാ­യി­രു­ന്നു. കേ­ന്ദ്ര­സര്‍­ക്കാര്‍ സ്പോണ്‍­സര്‍ ചെ­യ്യു­ന്ന ഈ പ്രോ­ജ­ക്ട് പ്രൊ­പ്രൈ­റ്റ­റി മോ­ഡില്‍ നീ­ങ്ങു­ന്ന­തി­നെ­ക്കു­റി­ച്ചു് 
ഉള്ളു­രു­കു­ന്ന സങ്ക­ടം ജി­നേ­ഷ് സൂ­ക്ഷി­ച്ചു. അതി­നെ മറി­ക­ട­ക്കാന്‍ ടെ­സ­റാ­ക്ട് ഒസി­ആ­റി­നെ മല­യാ­ളം പഠി­പ്പി­ക്കാ­നു­ള്ള സ്വ­ത­ന്ത്ര 
സോ­ഫ്റ്റ്‌­വെ­യര്‍ പ്രോ­ജ­ക്ടി­നു­വേ­ണ്ടി­യും ജി­നേ­ഷ് കോ­ഡെ­ഴു­തി. നിര്‍­ഭാ­ഗ്യ­വ­ശാല്‍ ഇവ­യൊ­ന്നും പൂര്‍­ണ്ണ­മാ­യും 
ഉപ­യോ­ഗ­യു­ക്ത­മാ­കു­ന്ന അവ­സ്ഥ­യി­ലേ­ക്കു് ഇനി­യും എത്തി­യി­ട്ടി­ല്ല.

­സാ­ങ്കേ­തി­ക­വി­ദ്യ­യും ഭാ­ഷ­യും തമ്മി­ലു­ള്ള സങ്ക­ല­ന­ത്തില്‍ ഉട­ലെ­ടു­ക്കു­ന്ന പ്ര­ശ്ന­ങ്ങള്‍ പരി­ഹ­രി­ക്കു­ന്ന­തില്‍ പ്ര­ത്യേക ശ്ര­ദ്ധ ജി­നേ­ഷ് 
അര്‍­പ്പി­ച്ചി­രു­ന്നു. മല­യാ­ള­ത്തില്‍ ഇന്റര്‍­നാ­ഷ­ണല്‍ ഡൊ­മെ­യ്ന്‍ നെ­യിം (IDN) സ്റ്റാന്‍­ഡേര്‍­ഡ് സെ­റ്റ് ചെ­യ്യു­ന്ന­തി­നു് സി­ഡാ­ക്‍ 
സമര്‍­പ്പി­ച്ച ഡ്രാ­ഫ്റ്റി­ലെ പോ­രാ­യ്മ­കള്‍ ചൂ­ണ്ടി­ക്കാ­ട്ടി എസ്എം­സി­ക്കു­വേ­ണ്ടി തയ്യാ­റാ­ക്കിയ ക്രി­ട്ടി­ക്‍ ഡോ­ക്യു­മെ­ന്റ് 
എഴു­തി­യ­വ­രി­ലൊ­രാള്‍ ജി­നേ­ഷ് ആണു്. ഈ ക്രി­ട്ടി­ക്കി­ന്റെ വെ­ളി­ച്ച­ത്തില്‍ സി­ഡാ­ക്‍ മുന്‍­ക­യ്യെ­ടു­ത്തു് എസ്എം­സി അട­ക്ക­മു­ള്ള 
വി­വിധ സം­ഘ­ങ്ങ­ളു­മാ­യി ഇതു­സം­ബ­ന്ധി­ച്ച് തി­രു­വ­ന­ന്ത­പു­ര­ത്തു് കഴി­ഞ്ഞ മാ­സം കണ്‍­സല്‍­ട്ടേ­ഷന്‍ സം­ഘ­ടി­പ്പി­ച്ചി­രു­ന്നു­.

ഇന്‍­സ്ക്രി­പ്റ്റ് കീ­ബോര്‍­ഡ് ലേ­ഔ­ട്ട് കൂ­ടു­തല്‍ മെ­ച്ച­പ്പെ­ടു­ത്താ­നു­ള്ള സി­ഡാ­ക്കി­ന്റെ നിര്‍­ദ്ദേ­ശ­ത്തില്‍ കട­ന്നു­കൂ­ടിയ ചില പാ­ക­പ്പി­ഴ­കള്‍ 
ചൂ­ണ്ടി­ക്കാ­ട്ടാ­നും ജി­നേ­ഷ് ആയി­രു­ന്നു മുന്‍­ക­യ്യെ­ടു­ത്ത­തു്. ഈ ഡ്രാ­ഫ്റ്റില്‍ പി­ന്നീ­ടു സി­ഡാ­ക്‍ തി­രു­ത്തല്‍ വരു­ത്തു­ക­യു­ണ്ടാ­യി. 
സെ­പ്തം­ബ­റില്‍ അവ­സാ­ന­മാ­യി സം­സാ­രി­ക്കു­മ്പോള്‍ ജി­നേ­ഷ് എന്നോ­ടു് ആവ­ശ്യ­പ്പെ­ട്ട­തു്, ഈ പു­തു­ക്കിയ ഡ്രാ­ഫ്റ്റി­ന്റെ 
പശ്ചാ­ത്ത­ല­ത്തില്‍ ഉട­നെ തന്നെ ക്രി­ട്ടി­ക്‍ പു­തു­ക്ക­ണ­മെ­ന്നും തനി­ക്കു് അതി­നി സാ­ധി­ക്കു­മെ­ന്നു കരു­തു­ന്നി­ല്ല, എന്നു­മാ­ണു്. ­മ­ര­ണം­ 
തൊ­ട്ടു­മു­മ്പില്‍ കണ്ടു­കൊ­ണ്ടാ­ണു് ഇതു­പ­റ­ഞ്ഞ­തെ­ന്നു് ഇതെ­ഴു­തു­മ്പോ­ഴും വി­ശ്വ­സി­ക്കാ­നാ­വു­ന്നി­ല്ല.

% താഴെ സൂചിപ്പിയ്ക്കുന്ന കുറിപ്പുകള്‍ കിട്ടാന്‍ വഴിയുണ്ടോ?
­യു­ഐ­ഡി­യു­ടെ ഭാ­ഗ­മാ­യി നട­പ്പാ­ക്കു­ന്ന ബയോ­മെ­ട്രി­ക്‍ ഐഡ­ന്റി­ഫി­ക്കേ­ഷ­നി­ലെ ചതി­ക്കു­ഴി­കള്‍ ചൂ­ണ്ടി­ക്കാ­ട്ടു­ന്ന ഒരു ലേ­ഖ­നം 
തയ്യാ­റാ­ക്കു­വാന്‍ ഇതി­നി­ടെ ജി­നേ­ഷ് തു­ട­ങ്ങി­യി­രു­ന്നു. സന്തോ­ഷ് തോ­ട്ടി­ങ്ങ­ലി­നെ­യും എന്നെ­യും കോ-ഓഥര്‍­മാ­രാ­ക്കി ആരം­ഭി­ച്ച 
ആ പരി­ശ്ര­മം സൈ­ബര്‍ പെ­രു­മ്പാ­ത­യില്‍ പാ­തി­വ­ഴി­യില്‍ കി­ട­ക്കു­ക­യാ­ണു്. ഈ ലേ­ഖ­നം മല­യാ­ള­ത്തി­ലാ­ണെ­ങ്കില്‍ 
economic and political weekly­യ്ക്കു വേ­ണ്ടി അനി­വ­റു­മാ­യി ചേര്‍­ന്നു് ഇതേ വി­ഷ­യ­ത്തില്‍ മറ്റൊ­രു ലേ­ഖ­ന­വും 
തയ്യാ­റാ­ക്കു­ന്നു­ണ്ടാ­യി­രു­ന്നു. അതും അപൂര്‍­ണ്ണ­ത­യില്‍ വി­ട്ടാ­ണു് ജി­നേ­ഷ് വി­ട­വാ­ങ്ങി­യ­തു്.

% ചിത്രം ചേര്‍ക്കണം
commits to pypdflib

silpa­യി­ലേ­ക്കും pypdflib ലേ­ക്കും രോ­ഗ­ത്തി­ന്റെ മൂര്‍­ദ്ധ­ന്യ­ത്തി­ലും ജി­നേ­ഷ് നട­ത്തിയ കമ്മിറ്റുകള്‍ ശ്ര­ദ്ധേ­യ­ങ്ങ­ളാ­ണു്. യൂ­ണി­ക്കോ­ഡ് 
മല­യാ­ള­ത്തി­ലു­ള്ള ലേ­ഖ­ന­ങ്ങള്‍ വെ­ബ്ബില്‍ നി­ന്നു നേ­രി­ട്ടു ­പി­ഡി­എ­ഫ് ആക്കി മാ­റ്റാന്‍ ഉത­കു­ന്ന library ആണി­തു്.

­ജി­നേ­ഷ് ചെ­യ്ത­തൊ­ന്നും വെ­റു­തെ­യാ­കി­ല്ല, എന്നു­റ­പ്പി­ക്കാന്‍ കഴി­യു­ന്ന സു­ഹൃ­ദ്‌­വ­ല­യം സ്വ­യം ഇന്‍­ട്രോ­വെര്‍­ട്ട് എന്നു­കു­രു­തു­ന്ന 
ഈ ചെ­റു­പ്പ­ക്കാ­ര­നു­ണ്ടാ­യി­രു­ന്നു. ഒരു യഥാര്‍­ത്ഥ ഹാ­ക്കര്‍­ക്കു് നല്‍­കാ­വു­ന്ന ഏറ്റ­വും വലിയ ബഹു­മാ­നം അവര്‍ നേ­തൃ­ത്വം 
നല്‍­കിയ പ്രോ­ജ­ക്ടു­ക­ളെ വി­ജ­യ­ത്തി­ലേ­ക്കു നയി­ക്കു­ക­യാ­ണു്. ആ തര­ത്തി­ലു­ള്ള ആലോ­ച­ന­കള്‍ ആഷി­ക്‍ സലാ­ഹു­ദ്ദീ­ന്റെ­യും 
വാ­സു­ദേ­വി­ന്റെ­യും അനി­വ­റി­ന്റെ­യും മറ്റും മുന്‍­ക­യ്യില്‍ തു­ട­ങ്ങി­ക്ക­ഴി­ഞ്ഞു­.

­പൊ­ടു­ന്ന­നെ ഞാന്‍ അവ­സാ­നി­പ്പി­ക്കു­ക­യാ­ണു്. ഇതി­ന­പ്പു­റ­മെ­ഴു­താന്‍ എനി­ക്കു­ക­ഴി­യി­ല്ല. ഇതെ­ഴു­തു­മ്പോള്‍ വീ­ണ്ടും വാ­വി­ട്ടു­ള്ള 
കര­ച്ചി­ലി­ലേ­ക്കു് ഞാന്‍ വഴു­തി­വീ­ഴു­ക­യാ­ണു്. നീ മരി­ച്ചു­വെ­ന്നു വി­ശ്വ­സി­ക്കാന്‍ എനി­ക്കി­പ്പോ­ഴും കഴി­യു­ന്നി­ല്ല. ജി­നേ­ഷ്, നീ ജീ­വി­ക്കു­ന്നു, 
ഞങ്ങ­ളി­ലൂ­ടെ­.

(30  September 2011)\footnote{http://malayal.am/പലവക/മുഖം/12927/ജീവിക്കുവാനുള്ള-കാരണങ്ങള്‍}\\
സെ­ബിന്‍ ഏബ്ര­ഹാം ജേ­ക്ക­ബ് (മലയാളം ഇന്റര്‍നെറ്റ് വാര്‍ത്താ പോര്‍ട്ടലിന്റെ എഡിറ്റര്‍)

\newpage

\secstar{Anivar's email to Swathanthra Malayalam Computing}
{\vskip 2pt}

Yes, I am writing this with tears. He was a close friend,
co-traveler, and colleague in SMC.

He was a student in IIIT hyderabad, working on CVIT lab, in the team
to develop a malayalam OCR. He was under the treatment for Leukemia in
CMC Vellore for past 2 years. He was very active contributor to
Swathanthra malayalam Computing's projects and various other FOSS
projects such as pypdflib, silpa project etc even from his hospital
bed. In addition he used to write articles in malayal.am on various
topics. In FEC also he was quite active and contributed to various
discussion threads. He was one among SMC's  Google summer of code
candidates in 2007. His contribution to standardization discussions
such as critique to International Domain names standard for malayalam,
Inscript 2 draft keyboard standard etc was also commendable.

He was always available on google chat when he is in the intervals
between treatments, we used to talk a lot about various contemporary
topics. Jinesh was working on a paper about pitfalls in Biometric
standards UID adopted, and he used to send a lot of contemporary
research papers on biometrics to me .

Our last chat (around sep 4th) was about planning get together, when
he reaches home after treatments.
Whenever we asks about his treatment details he used to show this XKCD comic
http://xkcd.com/931/

When he was in MES , he was the organiser of mny FOSS activities in
their campus and he was an initiator for Swatantra software user group
Malappuram.

\subsection*{­ജി­നേ­ഷി­ന്റെ മറ്റ് സൈ­ബര്‍ സാ­ന്നി­ദ്ധ്യ­ങ്ങള്‍}

Blog \url{http://www.jinsbond.in/} \\
Facebook \url{https://www.facebook.com/jinesh.jayaraman} \\
Twitter \url{http://twitter.com/#!/jinsbond007} \\
Picasa \url{http://picasaweb.google.com/jinesh.k} \\
\url{http://fci.wikia.com/wiki/User:Jinesh.k} \\
\url{https://github.com/santhoshtr/pypdflib/commits/master} \\
\url{http://code.google.com/p/tesseractindic/wiki/IndicLanguageTesseract} \\

\newpage
