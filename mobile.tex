\secstar{മൊബൈലും വിദ്യാര്‍ത്ഥിയും}
\vskip 2pt

കുറച്ചുകാലത്തെ ഇടവേളയ്ക്കുശേഷം കാലികപ്രസക്തമെന്നു് എനിക്കു് തോന്നിയ ഒരു വിഷയവുമായി.

ഈയടുത്തൊരു ദിവസം ടെലിവിഷനില്‍ കണ്ടൊരു പരിപാടിയാണു് എന്നെക്കൊണ്ടിതെഴുതിക്കുന്നതു്. 
വിദ്യാര്‍ത്ഥിയും മൊബൈലും ആയിരുന്നു ആ ചര്‍ച്ചയുടെ വിഷയം. ചര്‍ച്ചയില്‍ പങ്കെടുത്ത മാര്‍ ഇവാനിയോസ് കോളേജിലെ 
വിദ്യാര്‍ത്ഥികളടക്കമുള്ള പലരുടെയും അഭിപ്രായം പുത്തന്‍തലമുറഫോണുകള്‍ വിദ്യാര്‍ത്ഥികള്‍ക്കു് അനാവശ്യമാണെന്നായിരുന്നു.
 വൈദ്യുതി കനിയാഞ്ഞതു കാരണം പരിപാടി മുഴുവനും കാണാനായില്ല. അതുകൊണ്ടു് ചര്‍ച്ച സംഗ്രഹിച്ചതെങ്ങനെ
 എന്നറിയാനായില്ല. എന്തായാലും കഴിഞ്ഞ നാലുവര്‍ഷം ഒരു പ്രൊഫഷണല്‍ ബിരുദവിദ്യാര്‍ത്ഥിയായിരുന്ന 
ഈ ഞാനും അനുഭവത്തിന്റെ വെളിച്ചത്തില്‍ ചിലതു് പറയട്ടെ. ചര്‍ച്ചയില്‍ പങ്കെടുത്ത പലരും ധരിച്ചിരിക്കുന്നതുപോലെ 
വിലകൂടിയ ഫോണുകളും അത്യന്താധുനിക സാങ്കേതികവിദ്യയും സ്റ്റാറ്റസ് സിംബലല്ല, മറിച്ചു് ലോകം 
കൈവെള്ളയിലൊതുക്കാനുള്ള സങ്കേതങ്ങളാണു്. ചര്‍ച്ചയില്‍ പങ്കെടുത്ത അത്യന്തം പഠനതല്‍പ്പരരായ സുഹൃത്തുക്കള്‍ 
മാത്രമല്ല ക്യാമ്പസുകളിലുള്ളതു്. മറിച്ചു്, ഒരു ഭൂരിഭാഗം പഠനത്തോടൊപ്പം ജീവിതം ആസ്വദിക്കുകയും ചെയ്യുന്നവരാണു്. 
ചിരിക്കുകയും കളിക്കുകയും ചെയ്യുന്ന മനുഷ്യജീവികള്‍. അവര്‍ക്കു് ജീവിതത്തിലെ അനശ്വര മുഹൂര്‍ത്തങ്ങള്‍ പകര്‍ത്താന്‍ 
ക്യാമറയും, സംഗീതം ആസ്വദിക്കാന്‍ മ്യൂസിക് പ്ലെയറും, ഇന്റര്‍നെറ്റുമായി ബന്ധപ്പെടാന്‍ മോഡവും, സര്‍വ്വോപരി ഫോണും 
എല്ലാമായി പ്രവര്‍ത്തിക്കാന്‍ ഈ വിലയ്ക്കു്, ഈ വലുപ്പത്തില്‍ ഒരുപകരണം വേറെയുണ്ടോ? ഈ ലളിതമായ സമസ്യക്കു്
എനിക്കൊരുത്തരം തരിക. സ്റ്റാറ്റസിനുവേണ്ടി ഇത്തരം ഫോണുകള്‍ കൊണ്ടുനടക്കുന്നവര്‍ വിദ്യാര്‍ത്ഥികളുടെ ഇടയില്‍
 ന്യൂനപക്ഷമാണു്.

മൊബൈലിന്റെ ദുരുപയോഗം തടയാന്‍ ഹോസ്റ്റലുകള്‍ നടപടികള്‍ വല്ലതും സ്വീകരിക്കാറുണ്ടോ എന്ന ചോദ്യത്തിനു് 
അഭിമാനത്തോടെ ഒരു വിദ്യാര്‍ത്ഥിനി പറഞ്ഞ ഉത്തരം എന്നെ അക്ഷരാര്‍ത്ഥത്തില്‍ ഞെട്ടിച്ചുകളഞ്ഞു. ഹോസ്റ്റലില്‍ 
മൊബൈല്‍ കൈവശംവയ്ക്കാന്‍ വാര്‍ഡന്‍ അനുവദിക്കാറില്ലത്രേ. അത്ര നിര്‍ബന്ധമുള്ളവര്‍ക്കു് വാര്‍ഡന്റെ കൈയ്യിലേല്‍പ്പിക്കാം. 
ഫോണ്‍ വരുമ്പോള്‍ വാര്‍ഡന്റെ മുമ്പില്‍വച്ചു സംസാരിച്ചിട്ടു് തിരിച്ചുകൊടുക്കണം. ഇങ്ങനെ മൊബൈലുപയോഗിക്കുന്ന 
സുഹൃത്തിനും കൂട്ടുകാര്‍ക്കും പുത്തന്‍തലമുറയെന്നല്ല, മൊബൈല്‍ തന്നെ ആവശ്യമില്ല. അവര്‍ക്കൊക്കെ അത് കുരങ്ങന്റെ 
കൈയ്യിലുള്ള പൂമാലയാണു്. സാധാരണ മനുഷ്യരായി ജീവിക്കുകയും അവരെപ്പോലെ ചിന്തിക്കുകയും ചെയ്യുന്ന ഒരുപാടു 
വിദ്യാര്‍ത്ഥി സുഹൃത്തുക്കളുണ്ടു് ഈ കൊച്ചുകേരളത്തില്‍. അവര്‍ക്കു് മൊബൈല്‍ ഒരത്യാവശ്യമാണു്. സംസാരിക്കാനും 
പരസ്പരം (സ്നേഹ)സന്ദേശം കൈമാറാനും മാത്രമല്ല, ജീവിതാഘോഷങ്ങളുടെ നേര്‍കാഴ്ചകള്‍ സൂക്ഷിക്കാന്‍, തന്നെ 
ത്രസിപ്പിച്ച ഈരടികള്‍ വീണ്ടുംവീണ്ടും ആസ്വദിക്കാന്‍, ഒരിക്കലും തീരാത്ത മായകാഴ്ചകള്‍ക്കായി ഇന്റര്‍നെറ്റ് പരതാന്‍,
 അങ്ങനെ പലതിനും. നേരംകൊല്ലികളായ കളികളുടെ പേരില്‍ മത്സരം സംഘടിപ്പിക്കാനും, സോഫ്റ്റ്‌വെയറിന്റെ
 ഉള്ളുകളികളിലേക്കിറങ്ങിച്ചെന്നു് തിരുത്താനും പോലും അവര്‍ക്കു് മൊബൈല്‍ ഒഴിവാക്കാനാവാത്ത കൂട്ടാണു്. 
പഠിത്തം നന്നാക്കാനെന്ന പേരില്‍ വിദ്യാര്‍ത്ഥികളുടെനേരെ നടക്കുന്ന ഇത്തരം ചെയ്തികള്‍ക്കെതിരെ ആരും
 പ്രതികരിക്കാത്തതല്ല എന്നെ അത്ഭുതപ്പെടുത്തിയതു്, വിദ്യാര്‍ത്ഥികള്‍തന്നെ അതിനെ ന്യായീകരിക്കുന്നതാണു്. 
ഞാന്‍ കണ്ടിടത്തോളം, മൊബൈല്‍ കാരണം മാത്രം പഠിത്തം മോശമാവുന്ന ഒരു കുട്ടി പോലും ലോകത്തിലുണ്ടാവില്ല.
 പഠിക്കാന്‍ കഴിയാത്തവര്‍ക്കു്, മൊബൈലല്ലെങ്കില്‍ മറ്റൊരു കാരണം കാണും.

പിന്നെ ദുരുപയോഗം. വ്യക്തവും ശക്തവുമായ നിയമങ്ങളുണ്ടായിട്ടും മൊബൈല്‍ ദുരുപയോഗം ചെയ്യപ്പെടുമ്പോള്‍,
 നമ്മുടെ നിയമവ്യവസ്ഥിതിയുടെ പ്രശ്നമായാണു് ഞാനതു് കാണുന്നത്. ദുരുപയോഗങ്ങള്‍ റിപ്പോര്‍ട്ടു് ചെയ്യപ്പെടുകയും,
 നടപടികളും ശിക്ഷകളും ഉണ്ടാവുകയും ചെയ്താല്‍ ഒരു പരിധിവരെ ഇത്തരം പ്രശ്നങ്ങള്‍ കുറക്കാനാവും. 
സാങ്കേതികവിദ്യക്കനുസരിച്ചു് നമ്മുടെ നിയമവ്യവസ്ഥിതിയും ഭരണസംവിധാനങ്ങളും വളരാത്തതാണെന്നു തോന്നുന്നു
 ഇതിനുള്ള തടസ്സം. പിന്നെ, പോലീസിനോടിടപെടാന്‍ നമുക്കെല്ലാര്‍ക്കുമുള്ള മടിയും.

എറിയാനറിയാവുന്നവന്റെ കയ്യില്‍ വടി കൊടുക്കരുതെന്നപോലെ, വിദ്യാര്‍ത്ഥിക്കു് മൊബൈല്‍ കൊടുക്കരുതെന്നു വാശി 
പിടിക്കുന്നവരോടു് : വടി പിടിക്കാനെങ്കിലും പഠിപ്പിച്ചശേഷം, എറിയുന്നവരെ നന്നാക്കുക. മൊബൈല്‍ ഉപയോഗത്തില്‍ നിയന്ത്രണം 
വേണ്ടെന്നല്ല എന്റെ അഭിപ്രായം, നിയന്ത്രണവും നിരോധനവും ഫലത്തില്‍ ഒന്നാവരുതെന്നാണു്. അതു ഗുണത്തേക്കാളേറെ 
ദോഷമേ ചെയ്യൂ. ഔചിത്യമില്ലാത്ത മൊബൈല്‍ ഉപയോഗം മാത്രമേ നിയന്ത്രിക്കേണ്ടതുള്ളൂ എന്നാണെന്റെ അഭിപ്രായം.

\hspace*{2em}(August 28, 2007)
\newpage
