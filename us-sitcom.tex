\secstar{യുഎസ് സിറ്റ്കോമിലെ ഇന്ത്യക്കാരന്‍}
\vskip 2pt

വര്‍ഷങ്ങള്‍ നീളുന്ന സീരിയല്‍ ബഹളങ്ങള്‍ ഇന്ത്യന്‍ ടെലിവിഷന്റെ തനതു സംഭാവനയൊന്നുമല്ല. പലതരത്തിലുള്ള ടെലിവിഷന്‍ 
പരാക്രമങ്ങള്‍ക്കു് പേരുകേട്ടതാണു് അമേരിക്കന്‍ ടെലിവിഷന്‍ മാധ്യമങ്ങളും. പകല്‍സമയങ്ങളിലെ സോപ്പുകളും 
(സോപ്പു് ഓപ്പറ അഥവാ കണ്ണീര്‍ സീരിയല്‍), വെക്കേഷനല്ലാത്ത സമയത്തു് പ്രൈം ടൈമിലെത്തുന്ന ഒരു സീസണില്‍ 
ഇരുപത്തിനാല് എപ്പിസോഡുകള്‍ കാണിക്കുന്ന ആഴ്ച (വീക്ക്‌ലി) പരമ്പരകളുമാണു് അവിടുത്തെ പ്രധാന സീരിയല്‍ അവതാരങ്ങള്‍. 
കൂടാതെ സീസണല്‍ റിയാലിറ്റി ഷോ ബഹളങ്ങളും പ്രൈം ടൈമില്‍ ടെലിവിഷന്‍ നിറക്കാനെത്താറുണ്ടു്.

ഈ സീസണല്‍ പരിപാടിയൊഴികെ, ഏതാണ്ടെല്ലാ രീതിയിലും മട്ടിലുമുള്ള പ്രോഗ്രാമുകള്‍ ഇന്ത്യന്‍ ടി.വി. രംഗത്തും ഏതാണ്ടതേ
 രൂപഭാവത്തോടെ കാണാറുണ്ടു്. നമ്മുടെ ആഴ്ചപരമ്പരകളും ചില റിയാലിറ്റി ഷോകള്‍ പോലും 365 ദിവസവും നീണ്ടുനില്‍ക്കുന്നവയാണു്.
  എന്തായാലും ഇന്ത്യന്‍ അമേരിക്കന്‍ ടെലിവിഷന്‍ വിനോദരംഗത്തെ വിവിധ ട്രെന്റുകളെ വിലയിരുത്തലല്ല എന്റെ ലക്ഷ്യം.

സീസണലായി, സമ്മറിനുശേഷം തുടങ്ങി സമ്മറിനുമുന്‍പു് അവസാനിക്കുന്ന (ഇടയ്ക്കു് താങ്സ് ഗിവിങ്ങിനും ക്രിസ്മസിനും എല്ലാം 
ബ്രേക്കുമുണ്ടാകും) പരമ്പരകളില്‍ പല വിഭാഗങ്ങളുണ്ടു്. സിറ്റ്കോമുകള്‍ എന്നറിയപ്പെടുന്ന സിറ്റുവേഷനല്‍ കോമഡികള്‍, ഇന്ത്യയില്‍ 
നല്ല പ്രചാരമുള്ള ആക്ഷന്‍ ഡ്രാമകള്‍, മെട്രോ ഉപരിവര്‍ഗ്ഗത്തിന്റെ ഇഷ്ടവിഭാഗമായ ടീന്‍ ഡ്രാമകള്‍, ചരിത്രകഥകളുടെ ചെലവേറിയ
 പുനര്‍നിര്‍മ്മാണങ്ങളായ ഹിസ്റ്റോറിക്കല്‍ ഡ്രാമകള്‍, ആശുപത്രികളും അവിടുത്തെ അന്തരീക്ഷവും ചികിത്സയും മറ്റും പ്രധാന 
 വിഷയമായ മെഡിക്കല്‍ ഡ്രാമകള്‍, രാഷ്ട്രീയം പ്രധാന വിഷയമാകുന്ന പൊളിറ്റിക്കല്‍ ഡ്രാമകള്‍, കൂടാതെ യുദ്ധങ്ങളെ 
 അതിജീവിച്ചുണ്ടാവുന്ന സീരിയലുകളും വിരളമല്ല.

ഇങ്ങനെ പലവിഭാഗങ്ങളിലായി, പല സീസണുകള്‍ നീണ്ടുനില്‍ക്കുന്ന ഈ സീരിയലുകളില്‍ മുന്‍പന്തിയില്‍ നില്‍ക്കുന്ന പലതും 
ഇന്ത്യയില്‍ ലഭ്യമാണു്. സീ കഫെ, സ്റ്റാര്‍ വേള്‍ഡ്, ഫോക്സ്, എ.എക്സ്.എന്‍., ഹോം ബോക്സ് ഓഫീസ് തുടങ്ങി വിവിധ ചാനലുകളാണു് 
ഇവ സംപ്രേഷണം ചെയ്യുന്നതു്. ഇവയില്‍ സിറ്റ്കോം വിഭാഗത്തില്‍പ്പെട്ട ഒരു സീരിയലാണു് സി.ബി.എസ്. കാണിക്കുന്ന 
'ദ ബിഗ് ബാംഗ് തിയറി'. ഇന്ത്യയില്‍ സീ കഫെയാണു് ഇതു കാണിക്കുന്നതു്.

കാല്‍ടെക്കില്‍ ജോലിചെയ്യുന്ന 'അള്‍ട്ടിമേറ്റ് ഗീക്ക് ' എന്നു വിളിക്കാവുന്ന രണ്ടു ഫിസിക്സ് ശാസ്ത്രജ്ഞരുടെയും അവരുടെ
 സാമൂഹ്യജീവിതത്തെയുമാണു് 'ദ ബിഗ് ബാംഗ് തിയറി' വിഷയമാക്കുന്നതു്. ഈ സീരിയലിനെ പ്രത്യേകം ഓര്‍ക്കാന്‍ കാരണം 
 അതിലെ ഇന്ത്യന്‍ വംശജനായ ശാസ്ത്രജ്ഞനാണു്. കുനാല്‍ നയ്യാര്‍ അവതരിപ്പിക്കുന്ന രജേഷ് കൂത്രപ്പള്ളി എന്ന ഈ കഥാപാത്രം 
 ഇന്ത്യക്കാരെപ്പറ്റി പ്രചാരത്തിലുള്ള ഒരുപാടു ക്ലീഷേകളെയും അര്‍ദ്ധസത്യങ്ങളെയും വളരെ ഹാസ്യംകലര്‍ത്തി തനതായി
  അവതരിപ്പിച്ചിരിക്കുന്ന ഒന്നാണു്.

സീരിയലിലെ പ്രധാന കഥാപാത്രങ്ങളായ കാല്‍ടെക് ഫിസിക്സ് വിഭാഗത്തിലെ ലെനോര്‍ഡ് ഹോഫ്സ്റ്റഡറുടെയും ഷെല്‍ഡന്‍ 
കൂപ്പറുടെയും അടുത്ത സുഹൃത്തുക്കളിലൊരാളും, ആസ്ട്രോഫിസിസിസ്റ്റുമാണു് രജേഷ്. കാല്‍ടെക്കില്‍ എഞ്ചിനീയറായ
 ഹൊവാര്‍ഡ് വോളോവിറ്റ്സിന്റെ 'വിങ്മാനാ'യും പലപ്പോഴും നമുക്കു രജേഷിനെ കാണാം. ഇവര്‍ നാലുപേരും, പിന്നെ 
 ലെനൊര്‍ഡിന്റെയും ഷെല്‍ഡന്റെയും അയല്‍ക്കാരിയുമായ പെന്നിയുമാണു് പ്രധാന കഥാപാത്രങ്ങള്‍.

ഓരോ എപ്പിസോഡും ഓരോ കഥയാണു് പറയാറെങ്കിലും ലെനോര്‍ഡിന്റെയും പെന്നിയുടെയും 'പ്രേമ'ബന്ധത്തിനും, ഷെല്‍ഡന്റെ 
വിചിത്രമായ പെരുമാറ്റങ്ങള്‍ക്കുമൊപ്പം രജേഷിന്റെ സ്വഭാവപ്രത്യേകതകളും, ഹൊവാര്‍ഡിന്റെ സ്ത്രീകളോടുള്ള ഇടപഴകലുമാണു് 
ഹാസ്യരംഗങ്ങള്‍ സൃഷ്ടിക്കാറു്. ലെനോര്‍ഡും ഷെല്‍ഡനും 'അള്‍ട്ടിമേറ്റ് ഗീക്കു'കളുടെ ക്ലാസിക് ഉദാഹരണങ്ങളായാണു് 
പ്രത്യക്ഷപ്പെടുന്നതു്. ഹൊവാര്‍ഡാകട്ടെ 'മാമാസ് ബോയ്' എന്ന ക്ലീഷെയെയും, ഒപ്പം താനൊരു കാസനോവയാണെന്നു 
വീമ്പുപറയുന്ന പൊങ്ങച്ചക്കാരെയുമാണു് പ്രതിനിധികരിക്കുന്നതു്. രണ്ടും അമേരിക്കന്‍ ടെലിവിഷന്‍ രംഗത്തു് (സമൂഹത്തിലും)
 വളരെ എസ്റ്റാബ്ലിഷ്ഡായ ഹാസ്യകഥാപാത്രങ്ങളാണു്.

പെന്നിയാകട്ടെ 'ബ്ലോണ്ടു് ഷോബിസ് ആസ്പിരന്റ്' ആയി ലൊസാഞ്ചല്‍സിലെത്തി പല ചെറിയ ജോലികളും (ഇവിടെ വെയിട്രസ്സ്)
 ചെയ്തു ജീവിച്ചുപോകുന്ന മറ്റൊരു എസ്റ്റാബ്ലിഷ്ഡ് ഹാസ്യരൂപത്തെയാണു് പ്രതിനിധാനം ചെയ്യുന്നതു്.

അക്കാദമികമായി വളരെ ആക്റ്റീവായ 'ഗീക്കി' സ്ത്രീകളെന്ന മറ്റൊരു സാമ്പ്രദായിക ക്ലീഷെയെ പ്രതിനിധീകരിക്കുന്നവരും 
പലപ്പോഴായി സീരിയലില്‍ പ്രത്യക്ഷപ്പെടാറുണ്ടു്. ഷെല്‍ഡന്റെ കാര്യത്തില്‍ താനാണേറ്റവും ബുദ്ധിമാന്‍, അതുകൊണ്ടു് 
താനാണെപ്പോഴും ശരിയെന്നും, മറ്റെല്ലാവരും തെറ്റാണെന്നുമുള്ള (ഇതു പലപ്പോഴും ഷോയില്‍ ആവര്‍ത്തിച്ചു പ്രത്യക്ഷപ്പെടാറുണ്ടു്) 
ഭാവവും, അതിനെ മറ്റുള്ളവര്‍ കൈകാര്യംചെയ്യുന്ന രീതിയുമാണു് ഹാസ്യമുണ്ടാക്കുന്നതു്. പലപ്പോഴും വിചിത്രമായ ഷെല്‍ഡന്റെ 
ശീലങ്ങളും ചെറിയതോതില്‍ തമാശയുണ്ടാക്കാറുണ്ടു് (റൂം മേറ്റ്സ് അഗ്രിമെന്റ്, സീറ്റിങ്, അങ്ങനെ.)

ലെനോര്‍ഡിനു് താനൊരു ഗീക്കും, പ്രത്യേകിച്ചു് യാതൊരു സോഷ്യല്‍ ലൈഫുമില്ലാത്തയാളാണെന്നും പൂര്‍ണ്ണബോധ്യമുണ്ടു്. 
എന്നാല്‍ ഗീക്കീ സ്വഭാവങ്ങളായ കോമിക്, ഗെയിം അഡിക്ഷനും പെന്നിയുമായുള്ള ബന്ധമെന്ന സാധാരണ ജീവിതവും തമ്മിലുള്ള
 വടംവലിയാണു് ലെനോര്‍ഡിനെ ചുറ്റിപ്പറ്റിയുള്ള തമാശകള്‍ സൃഷ്ടിക്കുന്നതു്. ഷെല്‍ഡനെ സഹിക്കുന്ന ലെനോര്‍ഡും പലരംഗങ്ങളിലും
  ചിരിയുണര്‍ത്താറുണ്ടു്.

നാല്‍വര്‍സംഘത്തില്‍ ഡോക്റ്ററേറ്റ് ഇല്ലാത്തതു് ഹൊവാര്‍ഡിനു മാത്രമാണു്. ഇതിനെ ഹൊവാര്‍ഡ് മറികടക്കുന്നതു്, 
താനുണ്ടാക്കുന്ന സാധനങ്ങള്‍ ശരിക്കും ചൊവ്വയിലും മറ്റും പോയി പര്യവേഷണം നടത്താറുണ്ടെന്നു പറഞ്ഞാണു്. (തിയറിറ്റിക്കല്‍ 
ഫിസിസിസ്റ്റായ ഷെല്‍ഡനും എക്സ്പിരിമെന്റല്‍ ഫിസിസിസ്റ്റായ ലെനോര്‍ഡും രജേഷും എല്ലാം ഒന്നും ഉണ്ടാക്കുന്നവരല്ല എന്നതു്
 വേറെ കാര്യം.) ഹൊവാര്‍ഡിന്റെ അമ്മയുമായുള്ള ബന്ധവും സ്ത്രീകളോടുള്ള പെരുമാറ്റവുമാണു് തമാശയായി വരാറുള്ളതു്. 
 പെന്നിയുടെ കാര്യത്തില്‍, ഈ നാല്‍വര്‍സംഘത്തില്‍ പെന്നി ഉള്‍പ്പെടുന്നതുതന്നെ തമാശ സൃഷ്ടിക്കുന്നു. പലപ്പോഴും നാലുപേരിലും
  വിവേകപൂര്‍വ്വം പെരുമാറാന്‍ കഴിയുന്നതു് പെന്നിക്കാണു്. കൂട്ടത്തില്‍ പെന്നിയുടെ റിലേഷനുകളും പലപ്പോഴും വിഷയമാവാറുണ്ടു്.

ഇങ്ങനെ സാമ്പ്രദായികമായ ഹാസ്യരൂപങ്ങളെ വ്യക്തമായി കൂട്ടിയിണക്കി നിര്‍മ്മിച്ച 'ദ ബിഗ് ബാംഗ് തിയറി'യില്‍ രജേഷ് ഒരു 
പുതിയ വിഭാഗത്തെയാണു് പ്രതിനിധാനം ചെയ്യുന്നതു്. അമേരിക്കന്‍ പ്രേക്ഷകര്‍ അത്രയ്ക്കങ്ങോട്ടു് കണ്ടിട്ടില്ലാത്ത, എന്നാല്‍ അമേരിക്കന്‍ 
അക്കാദമിക രംഗത്തു് ധാരാളമുള്ള ഇന്ത്യന്‍ ശാസ്ത്രജ്ഞരെ. ചൈനയില്‍നിന്നും, കമ്യൂണിസ്റ്റും അല്ലാത്തതുമായ മറ്റു ഏഷ്യന്‍ 
രാജ്യങ്ങളില്‍നിന്നും കുടിയേറിപാര്‍ത്തവര്‍, അമേരിക്കന്‍ ജീവിതത്തിന്റെയും അതുവഴി ടെലിവിഷന്റെയും ഭാഗമായിട്ടു് വളരെക്കാലമായി. 
എന്നാല്‍ ഇന്ത്യന്‍ വംശജര്‍ വളരെ അപൂര്‍വ്വമായിരുന്നു ഈയടുത്തകാലം വരെ.

കാല്‍പെന്‍ മോഡിയും, കുനാല്‍ നയ്യാറും, നവീന്‍ ആന്‍ഡ്രൂസും, ഇന്ദിരാ വര്‍മ്മയും, നവി റാവത്തുമൊക്കെ അമേരിക്കന്‍ 
സിനിമയുടെയും ടെലിവിഷന്റെയും ഭാഗമായിത്തുടങ്ങിയിട്ടു് അധികകാലമായിട്ടില്ല. ഇവര്‍തന്നെ പലപ്പോഴും ഇന്ത്യന്‍ കഥാപാത്രങ്ങളില്‍ 
ഒതുങ്ങി നില്‍ക്കാറുമില്ല. എങ്കിലും ദീപക് ചോപ്രയും, ഓഷോ രജനീഷും, ഹരേ കൃഷ്ണാ പ്രസ്ഥാനവും നല്‍കിയ ഐഡന്റിറ്റിയും 
തങ്ങളിലേക്കുള്‍വലിയുന്ന സ്വഭാവവും നല്‍കിയ ക്ലീഷേകളിലൂടെയാണു് രജേഷ് വികസിക്കുന്നതു്. ഇന്ത്യക്കാരെല്ലാവരും സ്പരിച്വല്‍ 
ഭ്രാന്തന്‍മാരല്ലെന്നു കാണിക്കാനാകണം, രജേഷിന്റെ വിഷയത്തിലുള്ള അഭിപ്രായങ്ങള്‍ ഷെല്‍ഡന്‍ തിരുത്തുന്നതു് സാധാരണയാണു്.
 അതിനു മറുപടിയായി ഞാന്‍ ന്യൂ ഡല്‍ഹി എന്ന മെട്രോയില്‍ നിന്നാണുവരുന്നതു്, അല്ലാതെ യോഗാ സ്കൂളില്‍ നിന്നല്ല എന്നു് 
 രജേഷ് ഒരിടത്തു് മറുപടിയും നല്‍കുന്നുണ്ടു്.

എന്നാല്‍ സ്വാഭാവികമായി ഇന്ത്യക്കാര്‍ വലിയ നാണക്കാരാണു് എന്നതും, രണ്ടെണ്ണം വിട്ടാലെ നാവിനു ബലംവയ്ക്കു എന്നതും 
വളരെനന്നായി ഷോയില്‍ ഉപയോഗിച്ചിട്ടുണ്ടു്. കൂട്ടത്തില്‍ പെണ്ണുങ്ങളോടു് ഇടപഴകാനും പെണ്ണുങ്ങളുള്ള സദസ്സില്‍ 
സംസാരിക്കാനുമുള്ള ബുദ്ധിമുട്ടു് 'സെലക്റ്റീവു് മ്യൂട്ടിസം' എന്നൊരു മെഡിക്കല്‍ കണ്ടീഷനായിത്തന്നെ കാണിച്ചു് പൊളിറ്റിക്കലി
 കറക്റ്റാവാനും ഷോ ശ്രദ്ധിക്കുന്നുണ്ടു്. (സൈക്കോളജിസ്റ്റായ ലെനോര്‍ഡിന്റെ അമ്മയാണു് ഇതു തിരിച്ചറിയുന്നതു്.) ഈ പ്രശ്നം കാരണം
  രജേഷിനു് പലപ്പോഴും സ്വന്തം അഭിപ്രായം പറയാനാകാത്തതും, ഹൊവാര്‍ഡിനോടു് ചെവിയില്‍ പറയുന്നതിനോടു് രണ്ടിരട്ടി 
  ശബ്ദത്തില്‍ ഹൊവാര്‍ഡ് മറുപടി പറയുന്നതും സാധാരണവും സ്ഥിരംതമാശ സൃഷ്ടിക്കുന്നതുമായ രംഗമാണു്. ഷെല്‍ഡന്റെ 
  വിചിത്രമായ സ്വഭാവങ്ങളോടും പെരുമാറ്റരീതികളോടും ഏറ്റവും കൂടുതല്‍ അനുഭാവം കാണിക്കുന്നതും രജേഷാണു്.

എങ്കിലും ബ്രിട്ടീഷ് സീരിയലുകളിലെ സ്ഥിരസാന്നിധ്യമായ 'കറി ലൌവിങ്' ഇന്ത്യന'ല്ല രാജ്. ഇന്ത്യന്‍ രുചിയോടു് ചെറുതല്ലതാത്ത
 വിമുഖതകാണിക്കുന്ന രജേഷിന്റെ ആക്സെന്റോടു കൂടിയതെങ്കിലും 'ഡ്യൂഡ്' തുടങ്ങിയ സംബോധനകളും ഇന്ത്യയിലെ മെട്രോ
  സംസ്കാരത്തില്‍ നിന്നാണു് വരവെന്നു സൂചിപ്പിക്കുന്നു.
ഡല്‍ഹിയില്‍ ഡോക്ടറായ അച്ഛനും, റിസര്‍ച്ചു് വഴിമുട്ടുമ്പോള്‍ വിസ പ്രശ്നം പേടിച്ചു് ഷെല്‍ഡന്റെ കീഴില്‍ പണിയെടുക്കാന്‍ 
സന്നദ്ധനാവുന്നതും എല്ലാം പുതുതലമുറ ഇന്ത്യന്‍ കുടിയേറ്റക്കാരന്റെ പ്രശ്നങ്ങളാണു് വിഷയമാക്കുന്നതു്. ഇതിലെയൊക്കെ തമാശകള്‍ 
പലതും സാമ്പ്രദായിക അമേരിക്കന്‍ ഹാസ്യരൂപങ്ങളില്‍നിന്നു് വളരെ അകലെയുള്ളതും.

ഇന്ത്യയിലെ പട്ടിണിയില്‍ വളര്‍ന്നവനാണു് താനെന്ന രാജിന്റെ അവകാശവാദത്തെ, കൂട്ടുകാര്‍ തുറന്നുകാട്ടുന്നത്, 'ബെന്റ്ലി' 
ഉപയോഗിക്കുന്ന ഗൈനക്കോളജി സ്പെഷലിസ്റ്റായ ഡോക്ടറാണു് രാജിന്റെ അച്ഛനെന്നോര്‍മ്മിപ്പിച്ചാണു്. കാള്‍ സെന്ററില്‍ ജോലി 
ചെയ്യുന്ന കസിനും, വിവാഹത്തിനു നിര്‍ബന്ധിക്കുന്ന മാതാപിതാക്കളും തുടങ്ങി സാധാരണ ഇന്ത്യന്‍ ക്ലീഷേകളും പലപ്പോഴായി
 രംഗത്തു വരുന്നുണ്ടു്.

പണത്തിനുമപ്പുറം, വിദ്യാഭ്യാസത്തിനും അക്കാദമിക ആവശ്യങ്ങള്‍ക്കുമായിത്തന്നെ അമേരിക്കയിലെത്തുന്ന ആളാണു് രാജ്. ഒപ്പം 
സാമ്പ്രദായിക ഇന്ത്യന്‍ രീതികളോടുള്ള അവജ്ഞയുമുണ്ടു്. മറ്റു കഥാപാത്രങ്ങളില്‍നിന്നു വ്യത്യസ്തമായി, ഒരു പുതിയ ഹാസ്യ 
സ്റ്റീരിയോടൈപ്പിനെ സൃഷ്ടിക്കുകയാണു് രാജിലൂടെ ചക് ലോറിയും ബില്‍ പ്രാഡിയും ചെയ്തതു്. പല ഇന്ത്യന്‍ സ്റ്റീരിയോ ടൈപ്പുകളുടെയും 
എതിര്‍ധ്രുവത്തില്‍ നില്‍ക്കുന്ന രാജ്, പുതുതലമുറവിജ്ഞാനകുടിയേറ്റക്കാരെയാണു് പ്രതിനിധാനം ചെയ്യുന്നതു്.

പ്രശസ്തരായ ചക് ലോറിയും ബില്‍ പ്രാഡിയും നിര്‍മ്മിക്കുന്ന സീരിയലില്‍, ലെനോര്‍ഡിനെ അവതരിപ്പിക്കുന്നതു് ജോണി 
ഗാലെക്കിയാണു്. ഷെല്‍ഡനായി വേഷമിടുന്നതു് ജിം പാര്‍സണ്‍സും, പെന്നിയായെത്തുന്നതു് കേലി ക്വാകൊയുമാണു്. സൈമണ്‍ 
ഹെല്‍ബര്‍ഗ് ഹൊവാര്‍ഡായെത്തുന്നു. തിങ്കളാഴ്ചകളില്‍ രാത്രി ഒന്‍പതര ഈസ്റ്റേണ്‍ സമയത്തായിരുന്നു ഇതുവരെ 
'ദ ബിഗ് ബാംഗ് തിയറി' കാണിച്ചിരുന്നതു്. ഇനിയുള്ള സീസണുകളില്‍ അതു് വ്യാഴാഴ്ചകളില്‍ രാത്രി എട്ടുമണിക്കാവുമെന്നാണു് സൂചന. 
വളരെ പോപ്പുലറായ ഈ സീരിയല്‍ ഒരു നാലാംവര്‍ഷത്തിനുകൂടി പുതുക്കിയിട്ടുണ്ടെന്നാണു് വിക്കിപീഡിയ പറയുന്നതു്.

പിന്‍കുറിപ്പു്:

ലേഖനത്തിലുടനീളം 'ഗീക്ക്' എന്നാണുപയോഗിച്ചിരിക്കുന്നതെങ്കിലും കുറച്ചുകൂടി ചേരുന്ന പദം പലപ്പോഴും നെര്‍ഡ് എന്നതാണു്. 
 ആവശ്യത്തില്‍ കൂടുതല്‍ ഇംഗ്ലീഷ് അല്ലാതെതന്നെ ഉപയോഗിച്ചു എന്നു തോന്നിയതുകൊണ്ടാണു് അതൊഴിവാക്കിയതു്. 
 സാമ്പ്രദായികരൂപങ്ങളെപ്പറ്റിയും രജേഷിന്റെ അവതരണത്തെപ്പറ്റിയുമുള്ള എന്റെ നിരീക്ഷണങ്ങള്‍ തികച്ചും വ്യക്തിപരമാണു്, 
 സാധാരണ ഷോ/സിനിമ ക്രിട്ടിക്കുകളുടെ നിര്‍വ്വചനമായിരക്കണമെന്നില്ല എന്റേതു്.
 
 (May 7, 2011)\footnote{http://malayal.am/വിനോദം/ടി-വി/10385/യുഎസ്-സിറ്റ്കോമിലെ-ഇന്ത്യക്കാരന്‍}

\newpage
