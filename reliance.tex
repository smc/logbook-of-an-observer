\secstar{റിലയന്‍സിന് ഇനി പമ്പുകള്‍ തുറക്കാം!}
\vskip 2pt

‌\begin{quotation}
``പെ­ട്രോ­ളി­യം ഉത്പ­ന്ന­വി­പ­ണി ഡീ­റെ­ഗു­ലേ­റ്റ് ചെ­യ്യാന്‍ സര്‍­ക്കാര്‍ ഭാ­ഗ­ത്തു­നി­ന്ന് ശ്ര­മം തു­ട­ങ്ങു­ന്ന­ത് 2002­ലാ­ണ്. 
ബി­ജെ­പി നേ­തൃ­ത്വം നല്‍­കിയ എന്‍­ഡി­എ­ ഗവണ്‍­മെ­ന്റ് പക്ഷെ പൂര്‍­ണ്ണ­മാ­യും ­വി­ല­നി­യ­ന്ത്ര­ണം­ വി­പ­ണി­ക്കു 
വി­ട്ടു­കൊ­ടു­ത്തി­രു­ന്നി­ല്ല. പി­ന്നീ­ട് അധി­കാ­ര­ത്തി­ലെ­ത്തിയ ­യു­പി­എ­ ഗവണ്‍­മെ­ന്റാ­ക­ട്ടെ, ഇട­തു­സ­മ്മര്‍­ദ്ദ­ത്തി­നു വഴ­ങ്ങി­യും 
മന്ത്രി മണി­ശ­ങ്കര്‍ അയ്യ­റു­ടെ നി­ല­പാ­ടു­ക­ളെ തു­ടര്‍­ന്നും വി­ല­നി­യ­ന്ത്ര­ണം വീ­ണ്ടും ഏര്‍­പ്പെ­ടു­ത്തു­ക­യാ­ണു ചെ­യ്ത­ത്. രണ്ടാം 
യു­പിഎ ഗവണ്‍­മെ­ന്റ് പൊ­ളി­ച്ചെ­ഴു­തി­യ­ത് ഈ ഇട­പെ­ട­ലി­നെ­യാ­ണ്. ഇത്ത­വ­ണ­ത്തെ ഡീ­റെ­ഗു­ലേ­ഷന്‍ സ്ഥാ­യി­യാ­യി
നില്‍­ക്കു­മെ­ന്ന് വി­ദ­ഗ്ദ്ധര്‍ പറ­യു­ന്നു. രണ്ടാ­ഴ്ച­യി­ലൊ­രി­ക്കല്‍ മാ­റി­മ­റി­യു­ന്ന വി­ല­യു­മാ­യി നി­ശ്ചി­ത­ശ­മ്പ­ള­ക്കാര്‍­ക്ക് 
മത്സ­രി­ക്കേ­ണ്ടി­വ­രു­ന്ന അവ­സ്ഥ!''
\end{quotation}

{\vskip 12pt}


­ക­ഴി­ഞ്ഞ ആറു­മാ­സ­ത്തി­നു­ള്ളില്‍ ഇന്ധ­ന­വി­ല­വര്‍­ദ്ധന നട­ന്ന­ത് പല രീ­തി­യി­ലാ­ണ്. ബജ­റ്റില്‍ ഡ്യൂ­ട്ടി കൂ­ട്ടി­യ­പ്പോള്‍ 
ഏതാ­ണ്ട് മൂ­ന്നു രൂ­പ­യോ­ട­ടു­ത്താ­ണ് പെ­ട്രോ­ളി­നും ഡീ­സ­ലി­നും വി­ല­കൂ­ടി­യ­ത്. പി­ന്നീ­ട് ഏപ്രില്‍ മാ­സ­ത്തില്‍ യൂ­റോ IV 
മാ­ന­ദ­ണ്ഡ­ങ്ങള്‍ പ്ര­ധാ­ന­ന­ഗ­ര­ങ്ങ­ളില്‍ നട­പ്പാ­ക്കു­ന്ന­തി­ന്റെ ഭാ­ഗ­മാ­യി എണ്ണ റി­ഫൈ­ന­റി­കള്‍ ചെ­ല­വാ­ക്കിയ പണം 
തി­രി­ച്ചു­പി­ടി­ക്കാ­നാ­യി വീ­ണ്ടും ചെ­റു­താ­യി വി­ല­കൂ­ട്ടി. ഇനി­യും വി­ല­കൂ­ട്ടാ­നു­ള്ള സാ­ധ്യ­ത­ക­ളെ­ക്കു­റി­ച്ച് കഴി­ഞ്ഞ ഒരു മാ­സ­മാ­യി
കേ­ട്ടു­കൊ­ണ്ടി­രി­ക്കു­ക­യാ­ണ്. ഇപ്പോള്‍ അതും പ്രാ­വര്‍­ത്തി­ക­മാ­ക്കി­യി­രി­ക്കു­ന്നു. ഇതി­നു പു­റ­മെ­യാ­ണ് ചില സം­സ്ഥാ­ന­ങ്ങ­ളി­ലെ
നി­കു­തി­കൂ­ട്ടി­യ­തി­ന­നു­സ­രി­ച്ചും മറ്റും വി­ല­യില്‍ വന്ന വര്‍­ദ്ധ­ന­വ്.

­സാ­ധാ­രണ ഗതി­യില്‍ ഒരു പ്രാ­വ­ശ്യം വി­ല­കൂ­ട്ടി, സാ­ധ­ന­വി­ല­യൊ­ക്കെ­കൂ­ടി, ബസ്സു­കാ­രു­ടെ സമ­ര­മൊ­ക്കെ തീര്‍­ന്ന 
ശേ­ഷ­മേ അടു­ത്ത വി­ല­കൂ­ട്ട­ലി­ന്റെ മണം അടി­ക്കാ­റു­ള്ളൂ. എന്നാല്‍ അടി­ക്ക­ടി ഇങ്ങ­നെ വില കൂ­ട്ടാന്‍ സര്‍­ക്കാ­രി­നെ 
പ്രേ­രി­പ്പി­ക്കു­ന്ന കാ­ര്യ­മെ­ന്താ­ണ്?

അ­ടു­ത്തു നട­ന്ന വി­ല­വര്‍­ദ്ധ­ന­വ്, അന്താ­രാ­ഷ്ട്ര വി­പ­ണി­യില്‍ ക്രൂ­ഡി­ന്റെ വില വര്‍­ദ്ധി­ച്ച­തു­മൂ­ല­മു­ള്ള­ത­ല്ല. കഴി­ഞ്ഞ ഒരു 
വര്‍­ഷ­മാ­യി, അതു കു­ഴ­പ്പ­മി­ല്ലാ­ത്ത (70-80 ഡോ­ളര്‍/­ബാ­രല്‍) നി­ല­വാ­ര­ത്തി­ലാ­ണ്. കു­റെ­യൊ­ക്കെ സ്ഥാ­യി­യാ­ണെ­ന്നു
പറ­യാം. എന്നാല്‍ ഇന്ത്യ­യി­ലെ എണ്ണ വില്‍­പ്പന കമ്പ­നി­കള്‍ സര്‍­ക്കാര്‍ സബ്സി­ഡി­യോ­ടെ­യാ­ണ് ഇപ്പോ­ഴും 
എണ്ണ­യു­ത്പ­ന്ന­ങ്ങള്‍ വില്‍­ക്കു­ന്ന­ത്. അതില്‍ പെ­ട്രോ­ളി­ന്റെ സബ്സി­ഡി പൂര്‍­ണ്ണ­മാ­യും എടു­ത്തു­ക­ള­യു­ക­യാ­ണ് 
ഇന്ന­ല­ത്തെ വി­ല­കൂ­ട്ട­ലി­ലൂ­ടെ ചെ­യ്ത­ത്. ഡീ­സ­ലി­ന്റെ വി­ല­യില്‍ ഇപ്പോ­ഴും ലി­റ്റ­റി­ന് 80 പൈസ സര്‍­ക്കാര്‍ 
സബ്സി­ഡി നല്‍­കു­ന്നു­ണ്ട്.

­ക്രൂ­ഡ് വി­ല­യില്‍ വലി­യൊ­രു ചാ­ഞ്ചാ­ട്ടം വി­പ­ണി പ്ര­വ­ചി­ക്കു­ന്നു­മി­ല്ല. അതി­നാല്‍ വി­പ­ണി തു­റ­ന്നു കൊ­ടു­ക്കു­ന്ന­ത് 
കഴി­ഞ്ഞ­കു­റേ വര്‍­ഷ­ങ്ങ­ളാ­യി നഷ്ട­ത്തില്‍ പ്ര­വര്‍­ത്തി­ക്കു­ന്ന സര്‍­ക്കാര്‍ കമ്പ­നി­ക­ളെ രക്ഷി­ക്കാ­നാ­ണെ­ന്നാ­ണ് സര്‍­ക്കാര്‍ 
പറ­യു­ന്ന­ത്. മാ­ത്ര­മ­ല്ല സ്വ­കാ­ര്യ­ക­മ്പ­നി­ക­ളു­ടെ ഉല്‍­പ്പ­ന്ന­ങ്ങ­ളും ലഭ്യ­മാ­വു­ന്ന­ത് വി­പ­ണി­യില്‍ മത്സ­രം ഉണ്ടാ­കാന്‍ 
സഹാ­യ­ക­മാ­വു­മെ­ന്നും പറ­യു­ന്നു. ഒരു പക്ഷേ, സബ്സി­ഡി­കള്‍ ഘട്ടം ഘട്ട­മാ­യി എടു­ത്തു­ക­ള­യാ­നു­ള്ള പ്ലാ­നി­ങ് 
കമ്മീ­ഷന്‍ ശു­പാര്‍­ശ­ന­ട­പ്പി­ലാ­ക്കു­ന്ന­തി­ന്റെ ഭാ­ഗ­വു­മാ­കാം ഇത് (ദാ­രി­ദ്ര്യ രേ­ഖ­യ്ക്കു­മു­ക­ളി­ലു­ള്ള­വ­രു­ടെ പൊ­തു­വി­ത­ര­ണം 
സമ്പ്ര­ദാ­യം വഴി­യു­ള്ള ധാ­ന്യ­വില ഉയര്‍­ത്താ­നാ­ണ് പ്ലാ­നി­ങ് പാ­ന­ലി­ന്റെ മറ്റൊ­രു ശു­പാര്‍­ശ).

ഇ­തി­നു­മു­മ്പ് മോ­ട്ടോര്‍ ഇന്ധ­ന­ങ്ങ­ളു­ടെ ഡി­റെ­ഗു­ലേ­ഷ­ന് ഒരു ശ്ര­മം നട­ത്തി­യ­ത് 2002­ലാ­ണ്. അന്ന­ത്തെ 
എന്‍.­ഡി­.എ. സര്‍­ക്കാര്‍ വി­ല­നി­യ­ന്ത്ര­ണം എടു­ത്തു­ക­ള­ഞ്ഞെ­ങ്കി­ലും വില നി­ശ്ച­യി­ക്കാ­നു­ള്ള അവ­കാ­ശം സര്‍­ക്കാര്‍ 
കമ്പ­നി­കള്‍­ക്കാ­ണ് നല്‍­കി­യ­ത്. സര്‍­ക്കാര്‍ കമ്പ­നി­കള്‍ നി­യ­ന്ത്രി­ക്കു­ന്ന ­പെ­ട്രോ­ളി­യം­ മാര്‍­ക്ക­റ്റില്‍ ഫല­ത്തില്‍ വില 
നി­യ­ന്ത്രി­ച്ചി­രു­ന്ന­ത് സര്‍­ക്കാര്‍ തന്നെ­യാ­യി­രു­ന്നു. തി­ര­ഞ്ഞെ­ടു­പ്പ് പ്ര­മാ­ണി­ച്ച് ഒരു വര്‍­ഷ­ത്തോ­ളം (2003-04 
കാ­ല­ഘ­ട്ട­ത്തില്‍) എണ്ണ­ക്ക­മ്പ­നി­കള്‍ അന്താ­രാ­ഷ്ട്ര വി­ല­ക­ളെ മാ­നി­ക്കാ­തെ വി­ല­യു­യര്‍­ത്താ­തി­രു­ന്ന­പ്പോള്‍ 
അതു­വ്യ­ക്ത­മാ­വു­ക­യും ചെ­യ്തു. അതാ­യ­ത്,­വി­പ­ണി സു­താ­ര്യ­മാ­ക്കി­യെ­ന്നു മേ­നി­ന­ടി­ച്ചെ­ങ്കി­ലും പ്ര­ത്യ­ക്ഷ­നി­യ­ന്ത്ര­ണ­ത്തില്‍ 
നി­ന്നും പരോ­ക്ഷ­നി­യ­ന്ത്ര­ണ­ത്തി­ലേ­ക്ക് കാ­ര്യ­ങ്ങള്‍ മാ­റു­ക­മാ­ത്ര­മാ­ണു­ണ്ടാ­യ­ത്. അതു­കൊ­ണ്ടു­ത­ന്നെ, വി­പ­ണി­യില്‍ 
പ്ര­വേ­ശി­ച്ച പല സ്വ­കാ­ര്യ കമ്പ­നി­ക­ളും സര്‍­ക്കാര്‍ കമ്പ­നി­ക­ളു­ടെ വി­ല­യോ­ടു­മ­ത്സ­രി­ക്കാ­നാ­വാ­തെ പമ്പു­ക­ളും 
സം­വി­ധാ­ന­ങ്ങ­ളും അട­ച്ച് സ്ഥ­ലം വി­ടു­ക­യും ചെ­യ്തു­.

­തി­ര­ഞ്ഞെ­ടു­പ്പി­നു ശേ­ഷം അധി­കാ­ര­ത്തില്‍ വന്ന യു­.­പി­.എ. വി­ല­നി­യ­ന്ത്ര­ണം പു­നഃ­സ്ഥാ­പി­ക്കു­ക­യും, 
പൊ­തു­മേ­ഖ­ലാ­ക­മ്പ­നി­കള്‍­ക്ക് സര്‍­ക്കാര്‍ ഇന്ധന സബ്സി­ഡി നല്‍­കു­ന്ന സം­വി­ധാ­നം വീ­ണ്ടും കൊ­ണ്ടു­വ­രി­ക­യും 
ചെ­യ്തു. അതു­കൊ­ണ്ട് ക്രൂ­ഡ് വില ക്ര­മാ­തീ­ത­മാ­യി ഉയര്‍­ന്ന 2007-08 കാ­ല­ഘ­ട്ട­ത്തി­ലും തി­ര­ഞ്ഞെ­ടു­പ്പു­ക­ളെ 
അതി­ജീ­വി­ക്കാന്‍ യു­.­പി­.എ­.­ക്കു കഴി­ഞ്ഞു. അന്ന­ത്തെ മന്ത്രി മണി­ശ­ങ്കര്‍ അയ്യ­രു­ടെ നി­ല­പാ­ടു­ക­ളും, ഇട­തു­പ­ക്ഷ­ത്തി­ന്റെ 
ഇട­പെ­ട­ലു­ക­ളും, ഒരു­പ­ക്ഷെ ഈ തീ­രു­മാ­ന­ത്തി­നു കാ­ര­ണ­മാ­യി­ട്ടു­ണ്ടാ­വ­ണം­.

എ­ന്താ­യാ­ലും, ഇപ്പോ­ഴ­ത്തെ വി­ല­യു­യര്‍­ത്തല്‍ കൊ­ണ്ട് സബ്സി­ഡി പുര്‍­ണ്ണ­മാ­യും ഒഴി­വാ­കു­ന്നി­ല്ല. പൊ­തു­മേ­ഖ­ലാ 
സ്ഥാ­പ­ന­ങ്ങ­ളു­ടെ നഷ്ടം കു­റ­ച്ചു കു­റ­യു­മെ­ന്നു മാ­ത്രം. എങ്കില്‍­പ്പി­ന്നെ, ഈ വി­ല­യൊ­ക്കെ പതു­ക്കെ പതു­ക്കെ (ഒ­രു രൂപ 
വച്ച് അഞ്ചു പ്രാ­വ­ശ്യം കൂ­ട്ടി­യാ­ലും അഞ്ചു­രൂ­പ­യാ­വി­ല്ലെ­?) കൂ­ട്ടി­യാല്‍ മതി­യാ­യി­രു­ന്നി­ല്ലെ എന്ന ചോ­ദ്യ­ത്തി­നു­ള്ള മറു­പ­ടി, 
ഇനി വരാ­നി­രി­ക്കു­ന്ന സം­സ്ഥാന തി­ര­ഞ്ഞെ­ടു­പ്പു­ക­ളും, വെ­ള്ള­പ്പൊ­ക്ക­ദു­രി­ത­ങ്ങ­ളു­മൊ­ക്കെ­യാ­ണ്. പറ­ഞ്ഞ­പോ­ലെ 
വി­പ­ണി­ക്ക­നു­സ­രി­ച്ച് പൊ­തു­മേ­ഖ­ലാ എണ്ണ­ക­മ്പ­നി­കള്‍ വില വര്‍­ദ്ധി­പ്പി­ക്കു­മോ (താ­ഴ്ത്തു­മോ) എന്നു നമു­ക്ക് കാ­ത്തി­രു­ന്നു 
കാ­ണേ­ണ്ടി­വ­രും (സ്വ­ന്തം ലാ­ഭം കൂ­ട്ടാ­ന­ല്ലാ­തെ, കു­റ­യ്ക്കാന്‍ ആര്‍­ക്കും താല്‍­പ്പ­ര്യ­മു­ണ്ടാ­കാന്‍ ഒരു വഴി­യു­മി­ല്ല).

ആ­പല്‍­ക്ക­ര­മായ രീ­തി­യില്‍ അന്താ­രാ­ഷ്ട്ര­വില ഉയര്‍­ന്നാ­ല­ല്ലാ­തെ, ഇനി സര്‍­ക്കാര്‍ ഇട­പെ­ട്ടു ­പെ­ട്രോള്‍ വില കു­റ­ക്കാന്‍
സാ­ധ്യ­ത­യി­ല്ല. കാ­ര­ണം, എന്‍.­ഡി­.എ. സര്‍­ക്കാ­രി­ന്റെ പരാ­ജ­യ­ത്തില്‍ നി­ന്നും പാ­ഠ­മുള്‍­ക്കൊ­ണ്ട് കു­റ­ച്ചു­കൂ­ടി 
സമര്‍­ത്ഥ­മാ­യാ­ണ് പെ­ട്രോള്‍ വി­പ­ണി­തു­റ­ക്കാ­നു­ള്ള യു­.­പി­.എ. ശ്ര­മ­ങ്ങള്‍ നീ­ങ്ങു­ന്ന­ത്. ഇത്ത­വണ ശരി­ക്കും ഒരു 
ഡി­റെ­ഗു­ലേ­ഷന്‍ നട­ക്കാ­നു­ള്ള സാ­ധ്യ­ത­ക­ളാ­ണ് മു­ന്നില്‍ കാ­ണു­ന്ന­ത്. ഒരു പക്ഷേ ധാ­രാ­ളം പെ­ട്രോ­ളും, ഡീ­സ­ലും 
കൈ­യ്യി­ലു­ണ്ടാ­വു­ക­യും, ചി­ല്ല­റ­വില്‍­പ്പ­ന­ശാ­ല­കള്‍ വഴി അവ വി­റ്റ­ഴി­ക്കാന്‍ കഴി­യാ­തി­രി­ക്കു­ക­യും ചെ­യ്യു­ന്ന സ്വ­കാ­ര്യ 
എണ്ണ കമ്പ­നി­ക­ളു­ടെ സമ്മര്‍­ദ്ദ­വും കാ­ര­ണ­മാ­യി­രി­ക്കാം­.

­വി­പ­ണി സാ­ധ്യ­ത­ക­ളും, വി­ല­കൂ­ട്ട­ലി­നു തി­ര­ഞ്ഞെ­ടു­ത്ത സമ­യ­വും, കാ­ണി­ക്കു­ന്ന­ത്, ഇപ്പോ­ഴ­ത്തെ വി­ല­യില്‍ നി­ന്ന് 
(അ­ന്താ­രാ­ഷ്ട്ര­വി­ല­യ്ക്കും, എണ്ണ­ക്ക­മ്പ­നി­ക­ളു­ടെ മനോ­ഗ­ത­ത്തി­നും അനു­സൃ­ത­മാ­യി) ചി­ല്ലറ തി­രി­ച്ചു പോ­ക്ക­ല്ലാ­തെ 2010­നു 
മുന്‍­പ­ത്തെ അവ­സ്ഥ­യി­ലേ­ക്ക് പോ­കാന്‍ യാ­തൊ­രു സാ­ധ്യ­ത­യു­മി­ല്ലെ­ന്നാ­ണ്. അല്ലെ­ങ്കില്‍ സര്‍­ക്കാ­രി­നെ 
പി­ടി­ച്ചു­കു­ലു­ക്കാന്‍ മാ­ത്രം സമ്മര്‍­ദ്ധം പാര്‍­ല­മെ­ന്റില്‍ ഉണ്ടാ­വ­ണം. മമ­താ ബാ­നര്‍­ജി എതിര്‍­പ്പു പ്ര­ക­ടി­പ്പി­ച്ചു­ക­ഴി­ഞ്ഞെ­ങ്കി­ലും,
ഒരി­ക്ക­ലും പ്ര­വ­ചി­ക്കാന്‍ കഴി­യാ­ത്തൊ­രു സ്ത്രീ­യാ­ണെ­ങ്കി­ലും, ബം­ഗാ­ളി­ലെ അവ­രു­ടെ സാ­ധ്യ­ത­ക­ളെ തു­ര­ങ്കം വയ്ക്കാന്‍ മാ­ത്രം
ഇന്ധ­ന­വില വര്‍­ദ്ധ­ന­യ്ക്കു കഴി­യി­ല്ലെ­ന്ന­തി­നാല്‍ പാര്‍­ല­മെ­ന്റില്‍ അവര്‍ ഇട­യു­മെ­ന്നു കരു­താ­നാ­വി­ല്ല. സ്പെ­ക്ട്രം ലേ­ലം
തത്കാ­ല­ത്തി­നു മൌ­നി­ക­ളാ­ക്കി­യി­രി­ക്കു­ന്ന ഡി­.എം­.­കെ. കടും­കൈ വല്ല­തും ചെ­യ്യു­മോ എന്ന­ത് കാ­ത്തി­രു­ന്നു 
കാ­ണേ­ണ്ട­താ­ണ്.

(27 June 2010)\footnote{http://malayal.am/വാര്‍ത്ത/വിശകലനം/6410/റിലയന്‍സിന്-ഇനി-പമ്പുകള്‍-തുറക്കാം}

\newpage
