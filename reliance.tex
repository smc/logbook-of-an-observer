\secstar{റിലയന്‍സിനു് ഇനി പമ്പുകള്‍ തുറക്കാം!}
\vskip 2pt

‌\begin{framed}
"പെട്രോളിയം ഉത്പന്നവിപണി ഡീറെഗുലേറ്റ് ചെയ്യാന്‍ സര്‍ക്കാര്‍ ഭാഗത്തുനിന്നു് ശ്രമം തുടങ്ങുന്നത് 2002ലാണു്. 
ബിജെപി നേതൃത്വം നല്‍കിയ എന്‍ഡിഎ ഗവണ്‍മെന്റ് പക്ഷെ, പൂര്‍ണ്ണമായും വിലനിയന്ത്രണം വിപണിക്കു 
വിട്ടുകൊടുത്തിരുന്നില്ല. പിന്നീടു് അധികാരത്തിലെത്തിയ യുപിഎ ഗവണ്‍മെന്റാകട്ടെ, ഇടതുസമ്മര്‍ദ്ദത്തിനു വഴങ്ങിയും 
മന്ത്രി മണിശങ്കര്‍ അയ്യറുടെ നിലപാടുകളെ തുടര്‍ന്നും വിലനിയന്ത്രണം വീണ്ടും ഏര്‍പ്പെടുത്തുകയാണു ചെയ്തതു്. രണ്ടാം 
യുപിഎ ഗവണ്‍മെന്റ് പൊളിച്ചെഴുതിയതു് ഈ ഇടപെടലിനെയാണു്. ഇത്തവണത്തെ ഡീറെഗുലേഷന്‍ സ്ഥായിയായി
നില്‍ക്കുമെന്നു് വിദഗ്ദ്ധര്‍ പറയുന്നു. രണ്ടാഴ്ചയിലൊരിക്കല്‍ മാറിമറിയുന്ന വിലയുമായി നിശ്ചിതശമ്പളക്കാര്‍ക്കു് 
മത്സരിക്കേണ്ടിവരുന്ന അവസ്ഥ!"
\end{framed}

{\vskip 12pt}


കഴിഞ്ഞ ആറുമാസത്തിനുള്ളില്‍ ഇന്ധനവിലവര്‍ദ്ധന നടന്നതു് പല രീതിയിലാണു്. ബജറ്റില്‍ ഡ്യൂട്ടി കൂട്ടിയപ്പോള്‍ 
ഏതാണ്ടു് മൂന്നു രൂപയോടടുത്താണു് പെട്രോളിനും ഡീസലിനും വിലകൂടിയത്. പിന്നീടു് ഏപ്രില്‍ മാസത്തില്‍ യൂറോ IV 
മാനദണ്ഡങ്ങള്‍ പ്രധാനനഗരങ്ങളില്‍ നടപ്പാക്കുന്നതിന്റെ ഭാഗമായി എണ്ണ റിഫൈനറികള്‍ ചെലവാക്കിയ പണം 
തിരിച്ചുപിടിക്കാനായി വീണ്ടും ചെറുതായി വിലകൂട്ടി. ഇനിയും വിലകൂട്ടാനുള്ള സാധ്യതകളെക്കുറിച്ചു് കഴിഞ്ഞ ഒരു മാസമായി
കേട്ടുകൊണ്ടിരിക്കുകയാണു്. ഇപ്പോള്‍ അതും പ്രാവര്‍ത്തികമാക്കിയിരിക്കുന്നു. ഇതിനുപുറമെയാണു് ചില സംസ്ഥാനങ്ങളിലെ
നികുതികൂട്ടിയതിനനുസരിച്ചും മറ്റും വിലയില്‍ വന്ന വര്‍ദ്ധനവു്.

സാധാരണഗതിയില്‍ ഒരു പ്രാവശ്യം വിലകൂട്ടി, സാധനവിലയൊക്കെകൂടി, ബസ്സുകാരുടെ സമരമൊക്കെ തീര്‍ന്ന 
ശേഷമേ അടുത്ത വിലകൂട്ടലിന്റെ മണം അടിക്കാറുള്ളൂ. എന്നാല്‍ അടിക്കടി ഇങ്ങനെ വില കൂട്ടാന്‍ സര്‍ക്കാരിനെ 
പ്രേരിപ്പിക്കുന്ന കാര്യമെന്താണു്?

അടുത്തു നടന്ന വിലവര്‍ദ്ധനവു്, അന്താരാഷ്ട്ര വിപണിയില്‍ ക്രൂഡിന്റെ വില വര്‍ദ്ധിച്ചതുമൂലമുള്ളതല്ല. കഴിഞ്ഞ ഒരു 
വര്‍ഷമായി, അതു കുഴപ്പമില്ലാത്ത (70-80 ഡോളര്‍/ബാരല്‍) നിലവാരത്തിലാണു്. കുറെയൊക്കെ സ്ഥായിയാണെന്നു
പറയാം. എന്നാല്‍ ഇന്ത്യയിലെ എണ്ണവില്‍പ്പന കമ്പനികള്‍ സര്‍ക്കാര്‍ സബ്സിഡിയോടെയാണു് ഇപ്പോഴും 
എണ്ണയുത്പന്നങ്ങള്‍ വില്‍ക്കുന്നതു്. അതില്‍ പെട്രോളിന്റെ സബ്സിഡി പൂര്‍ണ്ണമായും എടുത്തുകളയുകയാണു് 
ഇന്നലത്തെ വിലകൂട്ടലിലൂടെ ചെയ്തതു്. ഡീസലിന്റെ വിലയില്‍ ഇപ്പോഴും ലിറ്ററിനു് 80 പൈസ സര്‍ക്കാര്‍ 
സബ്സിഡി നല്‍കുന്നുണ്ടു്.

ക്രൂഡ് വിലയില്‍ വലിയൊരു ചാഞ്ചാട്ടം വിപണി പ്രവചിക്കുന്നുമില്ല. അതിനാല്‍ വിപണി തുറന്നുകൊടുക്കുന്നതു് 
കഴിഞ്ഞ കുറേവര്‍ഷങ്ങളായി നഷ്ടത്തില്‍ പ്രവര്‍ത്തിക്കുന്ന സര്‍ക്കാര്‍കമ്പനികളെ രക്ഷിക്കാനാണെന്നാണു് സര്‍ക്കാര്‍ 
പറയുന്നതു്. മാത്രമല്ല സ്വകാര്യകമ്പനികളുടെ ഉല്‍പ്പന്നങ്ങള്‍ ലഭ്യമാവുന്നതു് വിപണിയില്‍ മത്സരം ഉണ്ടാകാന്‍ 
സഹായകമാവുമെന്നും പറയുന്നു. ഒരു പക്ഷേ, സബ്സിഡികള്‍ ഘട്ടംഘട്ടമായി എടുത്തുകളയാനുള്ള പ്ലാനിങ് 
കമ്മീഷന്‍ ശുപാര്‍ശ നടപ്പിലാക്കുന്നതിന്റെ ഭാഗവുമാകാം ഇതു്. (ദാരിദ്ര്യരേഖയ്ക്കുമുകളിലുള്ളവരുടെ പൊതുവിതരണസമ്പ്രദായംവഴിയുള്ള ധാന്യവില ഉയര്‍ത്താനാണു് പ്ലാനിങ് പാനലിന്റെ മറ്റൊരു ശുപാര്‍ശ.)

ഇതിനുമുമ്പു് മോട്ടോര്‍ ഇന്ധനങ്ങളുടെ ഡിറെഗുലേഷനു് ഒരു ശ്രമം നടത്തിയത് 2002ലാണു്. അന്നത്തെ 
എന്‍.ഡി.എ. സര്‍ക്കാര്‍ വിലനിയന്ത്രണം എടുത്തുകളഞ്ഞെങ്കിലും വില നിശ്ചയിക്കാനുള്ള അവകാശം സര്‍ക്കാര്‍ 
കമ്പനികള്‍ക്കാണു് നല്‍കിയതു്. സര്‍ക്കാര്‍കമ്പനികള്‍ നിയന്ത്രിക്കുന്ന പെട്രോളിയം മാര്‍ക്കറ്റില്‍ ഫലത്തില്‍ വില 
നിയന്ത്രിച്ചിരുന്നതു് സര്‍ക്കാര്‍ തന്നെയായിരുന്നു. തിരഞ്ഞെടുപ്പു് പ്രമാണിച്ചു് ഒരു വര്‍ഷത്തോളം (2003-04 
കാലഘട്ടത്തില്‍) എണ്ണക്കമ്പനികള്‍ അന്താരാഷ്ട്രവിലകളെ മാനിക്കാതെ വിലയുയര്‍ത്താതിരുന്നപ്പോള്‍ 
അതു് വ്യക്തമാവുകയും ചെയ്തു. അതായത്, വിപണി സുതാര്യമാക്കിയെന്നു മേനിനടിച്ചെങ്കിലും പ്രത്യക്ഷനിയന്ത്രണത്തിനിന്നും പരോക്ഷനിയന്ത്രണത്തിലേക്കു് കാര്യങ്ങള്‍ മാറുകമാത്രമാണുണ്ടായതു്. അതുകൊണ്ടുതന്നെ, വിപണിയില്‍ 
പ്രവേശിച്ച പല സ്വകാര്യകമ്പനികളും സര്‍ക്കാര്‍കമ്പനികളുടെ വിലയോടു് മത്സരിക്കാനാവാതെ പമ്പുകളും 
സംവിധാനങ്ങളും അടച്ചു് സ്ഥലംവിടുകയും ചെയ്തു.

തിരഞ്ഞെടുപ്പിനുശേഷം അധികാരത്തില്‍ വന്ന യു.പി.എ. വിലനിയന്ത്രണം പുനഃസ്ഥാപിക്കുകയും, 
പൊതുമേഖലാകമ്പനികള്‍ക്ക് സര്‍ക്കാര്‍ ഇന്ധന സബ്സിഡി നല്‍കുന്ന സംവിധാനം വീണ്ടും കൊണ്ടുവരികയും 
ചെയ്തു. അതുകൊണ്ടു് ക്രൂഡ് വില ക്രമാതീതമായി ഉയര്‍ന്ന 2007-08 കാലഘട്ടത്തിലും തിരഞ്ഞെടുപ്പുകളെ 
അതിജീവിക്കാന്‍ യു.പി.എ.ക്കു കഴിഞ്ഞു. അന്നത്തെ മന്ത്രി മണിശങ്കര്‍ അയ്യരുടെ നിലപാടുകളും, ഇടതുപക്ഷത്തിന്റെ 
ഇടപെടലുകളും, ഒരുപക്ഷെ ഈ തീരുമാനത്തിനു കാരണമായിട്ടുണ്ടാവണം.

എന്തായാലും, ഇപ്പോഴത്തെ വിലയുയര്‍ത്തല്‍കൊണ്ടു് സബ്സിഡി പുര്‍ണ്ണമായും ഒഴിവാകുന്നില്ല. പൊതുമേഖലാ 
സ്ഥാപനങ്ങളുടെ നഷ്ടം കുറച്ചു കുറയുമെന്നു മാത്രം. എങ്കില്‍പ്പിന്നെ, ഈ വിലയൊക്കെ പതുക്കെപതുക്കെ (ഒരു രൂപ 
വച്ചു് അഞ്ചു പ്രാവശ്യം കൂട്ടിയാലും അഞ്ചുരൂപയാവില്ലെ?) കൂട്ടിയാല്‍ മതിയായിരുന്നില്ലെ എന്ന ചോദ്യത്തിനുള്ള മറുപടി, 
ഇനി വരാനിരിക്കുന്ന സംസ്ഥാന തിരഞ്ഞെടുപ്പുകളും, വെള്ളപ്പൊക്കദുരിതങ്ങളുമൊക്കെയാണു്. പറഞ്ഞപോലെ 
വിപണിക്കനുസരിച്ചു് പൊതുമേഖലാ എണ്ണകമ്പനികള്‍ വില വര്‍ദ്ധിപ്പിക്കുമോ (താഴ്ത്തുമോ) എന്നു് നമുക്കു് കാത്തിരുന്നു 
കാണേണ്ടിവരും. (സ്വന്തം ലാഭം കൂട്ടാനല്ലാതെ, കുറയ്ക്കാന്‍ ആര്‍ക്കും താല്‍പ്പര്യമുണ്ടാകാന്‍ ഒരു വഴിയുമില്ല.)

ആപല്‍ക്കരമായ രീതിയില്‍ അന്താരാഷ്ട്രവില ഉയര്‍ന്നാലല്ലാതെ, ഇനി സര്‍ക്കാര്‍ ഇടപെട്ടു പെട്രോള്‍ വില കുറക്കാന്‍
സാധ്യതയില്ല. കാരണം, എന്‍.ഡി.എ. സര്‍ക്കാരിന്റെ പരാജയത്തില്‍നിന്നും പാഠമുള്‍ക്കൊണ്ടു് കുറച്ചുകൂടി 
സമര്‍ത്ഥമായാണു് പെട്രോള്‍ വിപണിതുറക്കാനുള്ള യു.പി.എ. ശ്രമങ്ങള്‍ നീങ്ങുന്നതു്. ഇത്തവണ ശരിക്കും ഒരു 
ഡിറെഗുലേഷന്‍ നടക്കാനുള്ള സാധ്യതകളാണു് മുന്നില്‍ കാണുന്നതു്. ഒരു പക്ഷേ ധാരാളം പെട്രോളും ഡീസലും 
കൈയ്യിലുണ്ടാവുകയും, ചില്ലറവില്‍പ്പനശാലകള്‍ വഴി അവ വിറ്റഴിക്കാന്‍ കഴിയാതിരിക്കുകയും ചെയ്യുന്ന സ്വകാര്യ 
എണ്ണകമ്പനികളുടെ സമ്മര്‍ദ്ദവും കാരണമായിരിക്കാം.

വിപണി സാധ്യതകളും വിലകൂട്ടലിനു തിരഞ്ഞെടുത്ത സമയവും കാണിക്കുന്നതു്, ഇപ്പോഴത്തെ വിലയില്‍നിന്നു് 
(അന്താരാഷ്ട്രവിലയ്ക്കും, എണ്ണക്കമ്പനികളുടെ മനോഗതത്തിനും അനുസൃതമായി) ചില്ലറ തിരിച്ചുപോക്കല്ലാതെ 2010നു 
മുന്‍പത്തെ അവസ്ഥയിലേക്കു് പോകാന്‍ യാതൊരു സാധ്യതയുമില്ലെന്നാണു്. അല്ലെങ്കില്‍ സര്‍ക്കാരിനെ 
പിടിച്ചുകുലുക്കാന്‍ മാത്രം സമ്മര്‍ദ്ധം പാര്‍ലമെന്റില്‍ ഉണ്ടാവണം. മമതാ ബാനര്‍ജി എതിര്‍പ്പു പ്രകടിപ്പിച്ചുകഴിഞ്ഞെങ്കിലും,
ഒരിക്കലും പ്രവചിക്കാന്‍ കഴിയാത്തൊരു സ്ത്രീയാണെങ്കിലും, ബംഗാളിലെ അവരുടെ സാധ്യതകളെ തുരങ്കംവയ്ക്കാന്‍ മാത്രം
ഇന്ധന വിലവര്‍ദ്ധനയ്ക്കു കഴിയില്ലെന്നതിനാല്‍ പാര്‍ലമെന്റില്‍ അവര്‍ ഇടയുമെന്നു കരുതാനാവില്ല. സ്പെക്ട്രം ലേലം
തത്കാലത്തിനു മൗനികളാക്കിയിരിക്കുന്ന ഡി.എം.കെ. കടുംകൈ വല്ലതും ചെയ്യുമോ എന്നതു് കാത്തിരുന്നു 
കാണേണ്ടതാണു്.

\hspace*{2em}(27 June, 2010)\footnote{http://malayal.am/വാര്‍ത്ത/വിശകലനം/6410/റിലയന്‍സിനു്-ഇനി-പമ്പുകള്‍-തുറക്കാം}

\newpage
