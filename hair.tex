\secstar{തലമുടിയെക്കുറിച്ച് ഒരുപന്യാസം}
\vskip 2pt

ത­ല­മു­ടി ഒരു പ്ര­തി­ഭാ­സ­മാ­ണു്. സക്ക­റി­യ­യു­ടെ ആഫ്രി­ക്കന്‍ യാ­ത്ര­കള്‍ വാ­യി­ക്കു­ന്ന­തി­നും മു­മ്പ് മു­ടി നീ­ട്ടി­വ­ളര്‍­ത്താന്‍ 
തു­ട­ങ്ങി­യ­താ­ണു് ഞാന്‍. എന്നാല്‍ അതു പി­ന്നീ­ടു് ഇക്ക­ഴി­ഞ്ഞ ഒക്റ്റോ­ബ­റില്‍ ആരോ­ഗ്യ­പ­ര­മായ കാ­ര­ണ­ങ്ങ­ളാല്‍ 
ഒഴി­വാ­ക്ക­പ്പെ­ടും വരെ ഒരു പി­ടി അനു­ഭ­വ­ങ്ങ­ളാ­ണു് സമ്മാ­നി­ച്ച­തു്. ആഫ്രി­ക്കന്‍ യാ­ത്ര­ക­ളെ പരാ­മര്‍­ശി­ക്കാന്‍ കാ­ര­ണം, 
അതി­ലൊ­രി­ട­ത്തു് ബസ്സില്‍ യാ­ത്ര­ചെ­യ്യു­ന്ന ­സ­ക്ക­റി­യ മു­ടി പറ്റെ വെ­ട്ടി­ക്ക­ള­ഞ്ഞ ആഫ്രി­ക്കന്‍ സ്ത്രീ­യു­ടെ 
സൌ­ന്ദ­ര്യ­ത്തെ­പ്പ­റ്റി­പ്പ­റ­യു­ന്നു­ണ്ടു്. അതോ­ടൊ­പ്പം മു­ണ്ഡ­നം ചെ­യ്ത­ത­ല­യു­മാ­യി നട­ക്കു­ന്ന ചില നാ­ടന്‍ 
പരി­ഷ്കാ­രി­ക­ളെ­പ്പോ­ലെ അതു മനം­പി­ര­ട്ട­ലു­ണ്ടാ­ക്കാ­ത്ത­തി­നെ­പ്പ­റ്റി­യും (കൃ­ത്യ­മായ പ്ര­യോ­ഗം ഓര്‍­മ്മ­യി­ല്ല, എന്താ­യാ­ലും 
സാ­രം ആഫ്രി­ക്കന്‍ സ്ത്രീ­യു­ടെ സൌ­ന്ദ­ര്യ­സ­ങ്കല്‍­പ്പ­ത്തില്‍ മാ­ത്ര­മേ മു­ണ്ഡ­നം ചെ­യ്ത തല ചേ­രൂ എന്നാ­യി­രു­ന്നു­).

­മ­നം­പി­ര­ട്ടല്‍ അവി­ടെ നില്‍­ക്ക­ട്ടെ, എന്നെ­ക്കൂ­ടു­തല്‍ പി­ടി­ച്ചു­ല­ച്ച­തു്, നീ­ണ്ടു­വ­ളര്‍­ന്ന തല­മു­ടി­യെ­ന്ന­തു് സ്ത്രീ­യു­ടെ മാ­ത്രം ചി­ഹ്ന­വും 
ഭാ­ണ്ഡ­വും ആണെ­ന്ന തി­രി­ച്ച­റി­വാ­യി­രു­ന്നു. അല്ലെ­ന്നു വാ­ദി­ക്കു­ന്ന­വര്‍­ക്കു് എടു­ത്തു­ത­രാന്‍ ഉദാ­ഹ­ര­ണ­ങ്ങ­ളൊ­ന്നു­മി­ല്ലെ­ങ്കി­ലും, 
എന്നോ­ളം മു­ടി­യി­ല്ലെ­ന്നു സങ്ക­ട­പ്പെ­ട്ടി­രു­ന്ന സഹോ­ദ­രി­മാ­രും, പല പ്രാ­വ­ശ്യം എന്റെ മു­ടി­യും സ്കൂ­ളില്‍ നിര്‍­ബ­ന്ധ­മാ­യി ബോ­ബ് 
ചെ­യ്തു തോ­ളൊ­പ്പം നിര്‍­ത്തി­യി­രി­ക്കു­ന്ന അവ­രു­ടെ മു­ടി­യും താ­ര­ത­മ്യം ചെ­യ്തു നെ­ടു­വീര്‍­പ്പി­ട്ടി­ട്ടു­ള്ള സ്കൂള്‍ കി­ടാ­ങ്ങ­ളും മു­തല്‍, 
ഇപ്പോ­ഴെ­ന്റെ തൊ­ട്ട­പ്പു­റ­ത്തു താ­മ­സി­ക്കു­ന്ന തല­മൊ­ട്ട­യ­ടി­ക്കാന്‍ നിര്‍­ബ­ന്ധി­ത­യായ പെണ്‍­കു­ട്ടി­യും വരെ സാ­ക്ഷ്യം പറ­യും­.

എ­ന്തൊ­ക്കെ­യാ­യാ­ലും നീ­ട്ടി­വ­ളര്‍­ത്തിയ മു­ടി ഒരു പു­തിയ സാ­മൂ­ഹ്യാ­നു­ഭ­വ­മാ­ണെ­നി­ക്കു സമ്മാ­നി­ച്ച­തു്. എക്സെന്‍­ട്രി­ക്കു­കള്‍ 
അപൂര്‍­വ്വ­മ­ല്ലാ­ത്ത ശാ­സ്ത്ര­ത്തി­ന്റെ ലോ­ക­ത്താ­യ­തു­കൊ­ണ്ടു്, ആരും നി­ങ്ങ­ളു­ടെ വേ­ഷ­വി­ധാ­ന­ങ്ങ­ളെ മുന്‍­വി­ധി­യോ­ടെ 
കാ­ണി­ല്ലെ­ന്ന­തു് സമ്മാ­നി­ച്ച സ്വാ­ത­ന്ത്ര്യം ശരി­ക്കും ഉപ­യോ­ഗി­ച്ചെ­ന്നു പറ­യാം. അതി­നു സ്തു­തി­പ­റ­യേ­ണ്ട­തു ഐന്‍­സ്റ്റീ­ന്റെ 
ഒരു കാ­രി­ക്കേ­ച്ച­റി­നാ­ണെ­ന്നു തോ­ന്നു­ന്നു. അത്ര­യ്ക്കും വരി­ല്ലെ­ങ്കി­ലും മാ­സ­ങ്ങ­ളോ­ളം ഷേ­വ് ചെ­യ്യാ­ത്ത മു­ഖ­വും നീ­ട്ടി വളര്‍­ത്തിയ 
മു­ടി­യും ഹാ­ഫ് ട്രൌ­സ­റും ടീ ഷര്‍­ട്ടു­മ­ട­ങ്ങു­ന്ന എന്റെ പതി­വു­രൂ­പം എനി­ക്കും ചെ­റു­ത­ല്ല­താ­ത്ത വി­സി­ബി­ലി­റ്റി തന്നി­രു­ന്നു. 
മാ­ത്ര­മോ, ബഹു­സ്വ­ര­ത­യു­ടെ ഒരു സമൂ­ഹ­ത്തില്‍ ഒന്നി­ന്റേ­യും പ്ര­തി­നി­ധി­യാ­വാ­തെ എന്റെ മാ­ത്രം പ്ര­തി­നി­ധി­യാ­വാ­നും 
അതെ­ന്നെ സഹാ­യി­ച്ചി­ട്ടു­ണ്ടു്.

­മു­ണ്ഡ­നം ചെ­യ്ത തല വൈ­ധ­വ്യ­ത്തി­ന്റെ പ്ര­തീ­ക­മാ­യി­രു­ന്നു മുന്‍­പ്. വി­ധ­വ­യാ­യി­ട്ടും മു­ടി നീ­ട്ടി­വ­ളര്‍­ത്തു­ന്ന­തു്, അഭി­സാ­രി­ക­യു­ടെ 
ലക്ഷ­ണ­മാ­യാ­ണു് കണ്ടി­രു­ന്ന­തു്. ദീപ മേ­ത്ത­യു­ടെ "വാ­ട്ട­റില്‍" ലിസ റേ­യു­ടെ കഥാ­പാ­ത്ര­ത്തെ ഓര്‍­ക്കു­ക, പി­ന്നീ­ടു് 
സ്വ­ജീ­വി­ത­ത്തില്‍ കാന്‍­സര്‍ ഗ്ര­സ്ത­യാ­യി തല മു­ണ്ഡ­നം ചെ­യ്യേ­ണ്ടി­വ­ന്ന­പ്പോള്‍ എന്താ­യി­രു­ന്നി­രി­ക്കു­മാ­വോ ആ മന­സ്സില്‍ 
കട­ന്നു­പോ­യ­തു്, ഈയ­ടു­ത്ത കാ­ല­ത്തു് കീ­മോ­ത്തെ­റാ­പ്പി കഴി­ഞ്ഞു് രോ­ഗ­മു­ക്ത­യാ­യി­വ­ന്ന മം­താ മോ­ഹന്‍­ദാ­സു് മു­ടി 
നഷ്ട­പ്പെ­ട്ട­തി­നേ­യും മറ്റും വള­രെ വി­കാ­ര­ര­ഹി­ത­മാ­യി ഒരി­ന്റര്‍­വ്യൂ­വില്‍ സമീ­പി­ക്കു­ന്ന­തു കണ്ടു. നല്ല­തു്. നീ­ണ്ടു വളര്‍­ന്ന ഇട­തൂര്‍­ന്ന 
മു­ടി­യോ­ടു­ള്ള അഭി­നി­വേ­ശ­മി­ല്ലാ­ത്ത ചില പെണ്‍­കു­ട്ടി­ക­ളെ­ങ്കി­ലു­മു­ണ്ടീ­ലോ­ക­ത്തു്.

അ­ഴി­ച്ചി­ട്ടാല്‍ മു­ട്ടൊ­പ്പ­മെ­ത്തു­ന്ന കു­ടു­മ­യു­മാ­യി നട­ന്നി­രു­ന്ന മാ­ധ­വന്‍­മാ­രില്‍ നി­ന്നും (ഇ­ന്ദു­ലേ­ഖ) 
പു­രു­ഷ­സൌ­ന്ദ­ര്യ­സ­ങ്കല്‍­പ്പ­ത്തില്‍ പറ്റെ­യൊ­തു­ക്കിയ മു­ടി സ്ഥാ­നം നേ­ടി­യ­തു്, കഴി­ഞ്ഞ നൂ­റ്റാ­ണ്ടി­ന്റെ തു­ട­ക്ക­ത്തി­ലെ 
നവോ­ത്ഥാന പ്ര­സ്ഥാ­ന­ങ്ങ­ളെ­പ്പിന്‍­പ­റ്റി­യാ­ക­ണം. ഞാ­നാ­ദ്യം മു­ടി­വ­ളര്‍­ത്തി വീ­ട്ടി­ലെ­ത്തി­യ­പ്പോള്‍ ഒരു പരാ­മര്‍­ശം തി­രി­ച്ചു 
വരു­ന്ന കു­ടു­മ­യെ­പ്പ­റ്റി­ത്ത­ന്നെ­യാ­യി­രു­ന്നു. അന്നെ­ന്തു­കൊ­ണ്ടാ­ണാ­വോ പു­രു­ഷന്‍­മാര്‍ മാ­ത്രം കു­ടു­മ­മു­റി­ച്ച­തു്. എന്താ­യാ­ലും 
നവോ­ത്ഥാ­ന­കാ­ല­മാ­യി­രി­ക്ക­ണം നീ­ണ്ട­മു­ടി­യു­ടെ എല്ലാ­ഭാ­ര­വും സ്ത്രീ­യി­ലേ­ക്കു­മാ­ത്ര­മാ­യി ചു­രു­ക്കി­യ­തു്.  ദേ­വി­ക­യു­ടെ 
പു­രു­ഷ­കേ­ന്ദ്രീ­കൃത നവോ­ത്ഥാ­ന­ശ്ര­മ­ങ്ങ­ളില്‍ സൃ­ഷ്ടി­ക്ക­പ്പെ­ട്ട സ്ത്രീ പു­രു­ഷ­നാ­ഗ്ര­ഹി­ച്ച സ്ത്രീ­യാ­ണെ­ന്ന നി­രീ­ക്ഷ­ണ­ത്തില്‍ 
കു­റ­ച്ചെ­ങ്കി­ലും ശരി­യി­ല്ലേ എന്നൊ­രു തോ­ന്നല്‍ (കു­ല­സ്ത്രീ­യും ചന്ത­പ്പെ­ണ്ണും ഉണ്ടാ­യ­തെ­ങ്ങ­നെ­).

­സ്വാ­ഭാ­വി­ക­മാ­യി­ത്ത­ന്നെ തഴ­ച്ചു വള­രു­ന്ന തല­മു­ടി­യു­ള്ള ആണ്‍­കു­ട്ടി­ക­ളെ നീ­ട്ടി­വ­ളര്‍­ത്തു­ന്ന­തില്‍ നി­ന്നും വി­ല­ക്കു­ക­യും, മു­ടി 
കൊ­ഴി­ച്ചി­ലും മറ്റു കേ­ശ­സം­ബ­ന്ധി­യായ അസു­ഖ­വും മൂ­ലം വി­ഷ­മി­ക്കു­ക­യും ചെ­യ്യു­ന്ന പെണ്‍­കു­ട്ടി­ക­ളെ നീ­ട്ടി വളര്‍­ത്താ­ത്ത­തി­നു 
ഭര്‍­സി­ക്കു­ക­യും ചെ­യ്യു­ന്ന സമൂ­ഹ­മാ­ണു നമ്മു­ടേ­തു്.

­മൂ­ന്നു വര്‍­ഷം നീ­ണ്ടു­വ­ളര്‍­ന്ന തല­മു­ടി­യു­മാ­യി നട­ന്ന അനു­ഭ­വ­ത്തില്‍ നി­ന്നു പറ­യ­ട്ടെ, ­ത­ല­മു­ടി­ വളര്‍­ത്തു­ക­യെ­ന്ന­തും 
പരി­പാ­ലി­ക്കു­ക­യെ­ന്ന­തും വള­രെ ചെ­ല­വേ­റിയ ഒരു പണി­യാ­ണു്. തേ­ക്കു­ന്ന എണ്ണ­യും കഴു­കു­ന്ന വെ­ള്ള­വും ചെ­ളി­ക­ള­യാ­നും 
മര്യാ­ദ­യ്ക്കു നില്‍­ക്കാ­നും വേ­ണ്ടി ഉപ­യോ­ഗി­ക്കേ­ണ്ടി വരു­ന്ന ഷാം­പൂ­വും കണ്ടീ­ഷ­ന­റും എല്ലാം പോ­ക്ക­റ്റില്‍ വലിയ 
ദ്വാ­ര­ങ്ങ­ളാ­ണു­ണ്ടാ­ക്കു­ക. അതി­നു പു­റ­മേ­യാ­ണു്, വെ­ട്ടി­യൊ­തു­ക്കി കൊ­ണ്ടു­ന­ട­ക്കേ­ണ്ടു­ന്ന­തി­ന്റെ ചെ­ല­വു്.

­ഹൈ­ദ­രാ­ബാ­ദ് നഗ­ര­ത്തി­ലെ പ്ര­ശ­സ്ത­വും അല്ലാ­ത്ത­തു­മായ മി­ക്ക യു­ണി­സെ­ക്സ് സലൂ­ണു­ക­ളി­ലും ഇക്ക­ഴി­ഞ്ഞ മൂ­ന്നു 
വര്‍­ഷ­ത്തി­നു­ള്ളില്‍ ഞാന്‍ പോ­യി­ട്ടു­ണ്ടു്. എല്ലാ­യി­ട­ത്തു­നി­ന്നും വി­വിധ തര­ത്തി­ലെ മു­ടി­വെ­ട്ടു­ക­ളും നട­ത്തി­യി­ട്ടു­ണ്ടു് (200 
‌മു­തല്‍ 1500 രൂപ വരെ ചെ­ല­വു­ള്ള­വ). എന്നാല്‍ ബില്‍ തരു­മ്പോള്‍ എന്നും എന്റെ മേല്‍ അവര്‍ ഹെ­യര്‍ കട്ട് ഫോര്‍ 
വി­മന്‍ നട­ത്തി­യ­തി­ന്റെ വി­ല­യാ­ണു മേ­ടി­ക്കാ­റ്. എപ്പോള്‍ ചോ­ദി­ച്ചാ­ലും പെ­ണ്ണു­ങ്ങ­ളേ­ക്കാ­ളും മു­ടി­യു­ണ്ടാ­യി­രു­ന്നു സാര്‍ 
എന്നാ­യി­രി­ക്കും റി­സ­പ്ഷ­നി­ലെ കു­ട്ടി പറ­യു­ന്ന­തു്.

ഈ പാര്‍­ല­റു­കള്‍ പല­തും സമൂ­ഹ­ത്തി­ലെ ഉന്നത ശ്രേ­ണി­യി­ലെ സോ­ഷ്യ­ലൈ­റ്റു­ക­ളു­ടെ നി­ത്യ­സ­ന്ദര്‍­ശ­ന­കേ­ന്ദ്ര­ങ്ങ­ളാ­ണു്. 
ലിം­ഗ, മത, വംശ വ്യ­ത്യാ­സ­മി­ല്ലാ­തെ പണ­ത്തി­ന്റേ­യും കു­ടും­ബ­മ­ഹി­മ­യു­ടെ­യും മാ­ത്രം കാ­ര്യം നോ­ക്കി ആളു­ക­ളോ­ടു 
ഇട­പ­ഴ­കു­ന്ന­വ­രാ­ണു ഞങ്ങള്‍ എന്നു­റ­ക്കെ പ്ര­ഖ്യാ­പി­ക്കു­ന്ന­വ­രു­ടെ സമൂ­ഹ­ത്തില്‍­പ്പോ­ലും നീ­ണ്ടു­വ­ളര്‍­ന്ന തല­മു­ടി പെണ്‍­കു­ട്ടി­ക്കു 
മാ­ത്രം അവ­കാ­ശ­പ്പെ­ട്ട­താ­ണെ­ന്നു സാ­രം­.

­സ­മൂ­ഹ­ത്തോ­ടു പു­റം­തി­രി­ഞ്ഞു നില്‍­ക്കു­ന്ന­തി­നാ­ലാ­വ­ണം, എന്നെ പറ­ഞ്ഞു നന്നാ­ക്കാന്‍ ആദ്യ­കാ­ല­ത്തു് എന്റെ 
അമ്മ­യ­ല്ലാ­തെ വേ­റെ­യാ­രും ശ്ര­മി­ച്ചി­ട്ടി­ല്ല. അമ്മ­ത­ന്നെ, രണ്ടു പ്രാ­വ­ശ്യം പറ­ഞ്ഞി­ട്ടും കേള്‍­ക്കാ­താ­യ­പ്പോള്‍ എന്നാല്‍ 
നി­ന­ക്കു മര്യാ­ദ­യ്ക്കു വാ­ലും തല­യു­മൊ­ക്കെ ഒതു­ക്കി നട­ന്നു­കൂ­ടെ എന്ന ലൈ­നി­ലെ­ത്തു­ക­യും ചെ­യ്തു. ചില ചി­ല്ലറ 
സം­ശ­യാ­ലു­ക്ക­ളെ അര്‍­ഹി­ക്കു­ന്ന ഉത്ത­ര­ങ്ങ­ളി­ലൂ­ടെ നി­ശ­ബ്ദ­രാ­ക്കു­ക­യും കൂ­ടി­ചെ­യ്ത­തോ­ടെ ഞാന്‍ മു­ടി വളര്‍­ത്തു­ന്ന­തില്‍ 
പര­സ്യ­മാ­യി എതിര്‍­പ്പു­ള്ള­വ­രു­ടെ എണ്ണം കു­റ­ഞ്ഞു­.

­മു­ടി വെ­ട്ടാന്‍ തീ­രു­മാ­നി­ക്കു­ന്ന ഒരു പെണ്‍­കു­ട്ടി­യേ­യും സമൂ­ഹം ഈ രീ­തി­യി­ലാ­ണോ സ്വീ­ക­രി­ക്കുക എന്ന­റി­യി­ല്ല. മു­ടി പറ്റെ 
ബോ­ബ് ചെ­യ്തു നട­ക്കു­ന്ന ഒരു ഡോ­ക്റ്റ­റു­ണ്ടെ­നി­ക്കി­വി­ടെ. അവ­രു മല­യാ­ളി­യാ­ണെ­ന്നും പറ­യു­ന്നു (കേ­ട്ട­റി­വു മാ­ത്ര­മേ­യു­ള്ളൂ, 
ഇന്നു­വ­രെ ഒര­ക്ഷ­രം മല­യാ­ളം പറ­ഞ്ഞു കേ­ട്ടി­ട്ടി­ല്ല). അവ­രോ­ടു ചോ­ദി­ച്ചു നോ­ക്ക­ണം ഇനി­കാ­ണു­മ്പോള്‍. സ്വാ­ഭാ­വി­ക­മാ­യി 
തല­യി­ലു­ണ്ടാ­വു­ക­യും ആണ്‍ പെണ്‍ ഭേ­ദ­മി­ല്ലാ­തെ വള­രു­ക­യും ചെ­യ്യു­ന്ന മു­ടി­യെ­ന്ന സാ­ധ­ന­ത്തെ­പ്പി­ടി­ച്ചു 
സ്ത്രീ­ലിം­ഗ­സ്വ­ത്വ­ത്തി­ന്റെ പ്ര­ത്യ­ക്ഷ അട­യാ­ള­മാ­ക്കി­യ­താ­രാ­ണാ­വോ. തല­മു­ടി പരി­ച­ര­ണ­ത്തി­ന്റെ ബു­ദ്ധി­മു­ട്ടു മന­സ്സി­ലാ­ക്കിയ സ്ത്രീ 
വി­ദ്വേ­ഷി­യായ ആരെ­ങ്കി­ലു­മാ­യി­രി­ക്ക­ണം­.

ഇ­ത്ത­ര­ത്തില്‍ ലിം­ഗ­പ­ര­മാ­യി സ്ത്രീ­സ്വ­ത്വ­മു­ള്ള നീ­ണ്ടു­വ­ളര്‍­ന്ന മു­ടി പല അനു­ഭ­വ­ങ്ങ­ളും സമ്മാ­നി­ച്ചി­ട്ടു­ണ്ടെ­നി­ക്കു്. 
ഒരു­ല്ലാ­സ­യാ­ത്ര­യ്ക്കി­ട­യ്ക്കു് പാ­റി­പ്പ­റ­ന്നു­പോയ മു­ടി­യൊ­തു­ക്കാന്‍­പോ­യ­പ്പോള്‍ അവി­ടെ കൂ­ടെ­യു­ണ്ടാ­യി­രു­ന്ന 
പെണ്‍­കു­ട്ടി­ക­ളു­മു­ണ്ടാ­യി­രു­ന്നു. അവ­രോ­ടു ഞാന്‍ കാ­ര്യ­മാ­യി­ത്ത­ന്നെ, ഏതു ഷാം­പൂ­വാ­ണു­പ­യോ­ഗി­ക്കാ­റു്, എങ്ങ­നെ­യെ­ണ്ണ­തേ­ക്കും 
തു­ട­ങ്ങിയ കാ­ര്യ­ങ്ങ­ളൊ­ക്കെ അന്വേ­ഷി­ച്ചു. അവ­രു കൃ­ത്യ­മായ മറു­പ­ടി­യും ഉപ­ദേ­ശ­ങ്ങ­ളും തരി­ക­യും ചെ­യ്തു. അതി­നു ശേ­ഷം അവര്‍ പി­ന്നീ­ടൊ­രി­ക്കല്‍ പറ­ഞ്ഞ­തു്, ജീ­വി­ത­ത്തി­ലൊ­രി­ക്ക­ലും ഒരാണ്‍­കു­ട്ടി­യോ­ടു നട­ത്തേ­ണ്ടി­വ­രും എന്നു­ക­രു­തിയ സം­ഭാ­ഷ­ണ­മ­ല്ല 
അതെ­ന്നാ­ണു്.

എ­ന്താ­യാ­ലും ഒരേ ലാ­ബില്‍ അപ്പു­റ­ത്തു­മി­പ്പു­റ­ത്തു­മി­രു­ന്നു ജോ­ലി­യെ­ടു­ക്കു­ന്ന­വ­രാ­യ­തി­നാല്‍ ഈ വി­ഷ­യ­ത്തില്‍ പി­ന്നെ­യും 
പല പ്രാ­വ­ശ്യം സം­ഭാ­ഷ­ണ­ങ്ങ­ളു­ണ്ടാ­യി­ട്ടു­ണ്ടു്. സലോ­ണു­ക­ളെ­പ്പ­റ്റി­യും, പു­തിയ ഹെ­യര്‍­സ്റ്റൈ­ലു­ക­ളെ­പ്പ­റ്റി­യു­മ­ട­ക്കം. 
സ്ഥി­ര­മാ­യി ഞാന്‍ കേ­ട്ടി­രു­ന്ന ഒരു ചോ­ദ്യം എന്താ മു­ടി­വ­ളര്‍­ത്താന്‍ കാ­ര­ണ­മെ­ന്നാ­യി­രു­ന്നു. ആണ്‍ പെണ്‍ ഭേ­ദ­മി­ല്ലാ­തെ 
പല­രും ചോ­ദി­ച്ചി­ട്ടു­ണ്ടു്. മറു­പ­ടി നി­ഷേ­ധാ­ത്മ­ക­മാ­യി­ട്ടാ­ണെ­ങ്കി­ലും സത്യം തന്നെ­യാ­ണു് ഞാന്‍ പറ­യാ­റു­ള്ള­തും. ഞാന്‍ മു­ടി 
വളര്‍­ത്താ­ന­ല്ല, വെ­ട്ടാ­തി­രി­ക്കാ­നാ­ണു തീ­രു­മാ­നി­ച്ച­തെ­ന്നു്.

­സ്വ­ന്തം തല­യില്‍ വള­രു­ന്ന മു­ടി വെ­ട്ടാ­നും വളര്‍­ത്താ­നും സ്വാ­ത­ന്ത്ര്യ­മ­നു­വ­ദി­ക്കു­ന്ന, മു­ണ്ഡ­നം ചെ­യ്ത­ത­ല­യി­ലും 
സൌ­ന്ദ­ര്യ­ബോ­ധം കാ­ണാ­നും കഴി­വു­ള്ള ഒരു ലോ­ക­സ­മൂ­ഹം വളര്‍­ന്നു വരു­മെ­ന്നു പ്ര­ത്യാ­ശി­ച്ചു­കൊ­ണ്ടു്.

­പിന്‍­കു­റി­പ്പ്:

­ത­ല­മു­ടി­യെ­ക്കു­റി­ച്ചു­ള്ള ഈ വി­ചാ­ര­ങ്ങള്‍­ക്കു കട­പ്പാ­ടു് തൊ­ട്ട­ടു­ത്ത മു­റി­യില്‍ മു­ടി­യി­ല്ലാ­തെ കി­ട­ന്നി­രു­ന്ന പെണ്‍­കു­ട്ടി അവ­ളു­ടെ ഇട­തൂര്‍­ന്ന മു­ടി­യോ­ടു­കൂ­ടിയ പൂര്‍­വ്വാ­ശ്ര­മ­ചി­ത്ര­ങ്ങള്‍ കാ­ണി­ച്ചു­ത­ന്ന­പ്പോള്‍ ആ കണ്ണു­ക­ളില്‍ മി­ന്നി­മ­റ­ഞ്ഞ വി­കാ­ര­ങ്ങള്‍­ക്കു്.

(6 January 2011)\footnote{http://malayal.am/പലവക/9817/തലമുടിയെക്കുറിച്ച്-ഒരുപന്യാസം}

\newpage
