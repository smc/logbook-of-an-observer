\secstar{തലമുടിയെക്കുറിച്ച് ഒരുപന്യാസം}
\vskip 2pt

തലമുടി ഒരു പ്രതിഭാസമാണു്. സക്കറിയയുടെ ആഫ്രിക്കന്‍ യാത്രകള്‍ വായിക്കുന്നതിനും മുമ്പ് മുടി നീട്ടിവളര്‍ത്താന്‍ 
തുടങ്ങിയതാണു് ഞാന്‍. എന്നാല്‍ അതു പിന്നീടു് ഇക്കഴിഞ്ഞ ഒക്റ്റോബറില്‍ ആരോഗ്യപരമായ കാരണങ്ങളാല്‍ 
ഒഴിവാക്കപ്പെടും വരെ ഒരു പിടി അനുഭവങ്ങളാണു് സമ്മാനിച്ചതു്. ആഫ്രിക്കന്‍ യാത്രകളെ പരാമര്‍ശിക്കാന്‍ കാരണം, 
അതിലൊരിടത്തു് ബസ്സില്‍ യാത്രചെയ്യുന്ന സക്കറിയ മുടി പറ്റെ വെട്ടിക്കളഞ്ഞ ആഫ്രിക്കന്‍ സ്ത്രീയുടെ 
സൌന്ദര്യത്തെപ്പറ്റിപ്പറയുന്നുണ്ടു്. അതോടൊപ്പം മുണ്ഡനം ചെയ്തതലയുമായി നടക്കുന്ന ചില നാടന്‍ 
പരിഷ്കാരികളെപ്പോലെ അതു മനംപിരട്ടലുണ്ടാക്കാത്തതിനെപ്പറ്റിയും (കൃത്യമായ പ്രയോഗം ഓര്‍മ്മയില്ല, എന്തായാലും 
സാരം ആഫ്രിക്കന്‍ സ്ത്രീയുടെ സൌന്ദര്യസങ്കല്‍പ്പത്തില്‍ മാത്രമേ മുണ്ഡനം ചെയ്ത തല ചേരൂ എന്നായിരുന്നു).

മനംപിരട്ടല്‍ അവിടെ നില്‍ക്കട്ടെ, എന്നെക്കൂടുതല്‍ പിടിച്ചുലച്ചതു്, നീണ്ടുവളര്‍ന്ന തലമുടിയെന്നതു് സ്ത്രീയുടെ മാത്രം ചിഹ്നവും 
ഭാണ്ഡവും ആണെന്ന തിരിച്ചറിവായിരുന്നു. അല്ലെന്നു വാദിക്കുന്നവര്‍ക്കു് എടുത്തുതരാന്‍ ഉദാഹരണങ്ങളൊന്നുമില്ലെങ്കിലും, 
എന്നോളം മുടിയില്ലെന്നു സങ്കടപ്പെട്ടിരുന്ന സഹോദരിമാരും, പല പ്രാവശ്യം എന്റെ മുടിയും സ്കൂളില്‍ നിര്‍ബന്ധമായി ബോബ് 
ചെയ്തു തോളൊപ്പം നിര്‍ത്തിയിരിക്കുന്ന അവരുടെ മുടിയും താരതമ്യം ചെയ്തു നെടുവീര്‍പ്പിട്ടിട്ടുള്ള സ്കൂള്‍ കിടാങ്ങളും മുതല്‍, 
ഇപ്പോഴെന്റെ തൊട്ടപ്പുറത്തു താമസിക്കുന്ന തലമൊട്ടയടിക്കാന്‍ നിര്‍ബന്ധിതയായ പെണ്‍കുട്ടിയും വരെ സാക്ഷ്യം പറയും.

എന്തൊക്കെയായാലും നീട്ടിവളര്‍ത്തിയ മുടി ഒരു പുതിയ സാമൂഹ്യാനുഭവമാണെനിക്കു സമ്മാനിച്ചതു്. എക്സെന്‍ട്രിക്കുകള്‍ 
അപൂര്‍വ്വമല്ലാത്ത ശാസ്ത്രത്തിന്റെ ലോകത്തായതുകൊണ്ടു്, ആരും നിങ്ങളുടെ വേഷവിധാനങ്ങളെ മുന്‍വിധിയോടെ 
കാണില്ലെന്നതു് സമ്മാനിച്ച സ്വാതന്ത്ര്യം ശരിക്കും ഉപയോഗിച്ചെന്നു പറയാം. അതിനു സ്തുതിപറയേണ്ടതു ഐന്‍സ്റ്റീന്റെ 
ഒരു കാരിക്കേച്ചറിനാണെന്നു തോന്നുന്നു. അത്രയ്ക്കും വരില്ലെങ്കിലും മാസങ്ങളോളം ഷേവ് ചെയ്യാത്ത മുഖവും നീട്ടി വളര്‍ത്തിയ 
മുടിയും ഹാഫ് ട്രൌസറും ടീ ഷര്‍ട്ടുമടങ്ങുന്ന എന്റെ പതിവുരൂപം എനിക്കും ചെറുതല്ലതാത്ത വിസിബിലിറ്റി തന്നിരുന്നു. 
മാത്രമോ, ബഹുസ്വരതയുടെ ഒരു സമൂഹത്തില്‍ ഒന്നിന്റേയും പ്രതിനിധിയാവാതെ എന്റെ മാത്രം പ്രതിനിധിയാവാനും 
അതെന്നെ സഹായിച്ചിട്ടുണ്ടു്.

മുണ്ഡനം ചെയ്ത തല വൈധവ്യത്തിന്റെ പ്രതീകമായിരുന്നു മുന്‍പ്. വിധവയായിട്ടും മുടി നീട്ടിവളര്‍ത്തുന്നതു്, അഭിസാരികയുടെ 
ലക്ഷണമായാണു് കണ്ടിരുന്നതു്. ദീപ മേത്തയുടെ "വാട്ടറില്‍" ലിസ റേയുടെ കഥാപാത്രത്തെ ഓര്‍ക്കുക, പിന്നീടു് 
സ്വജീവിതത്തില്‍ കാന്‍സര്‍ ഗ്രസ്തയായി തല മുണ്ഡനം ചെയ്യേണ്ടിവന്നപ്പോള്‍ എന്തായിരുന്നിരിക്കുമാവോ ആ മനസ്സില്‍ 
കടന്നുപോയതു്, ഈയടുത്ത കാലത്തു് കീമോത്തെറാപ്പി കഴിഞ്ഞു് രോഗമുക്തയായിവന്ന മംതാ മോഹന്‍ദാസു് മുടി 
നഷ്ടപ്പെട്ടതിനേയും മറ്റും വളരെ വികാരരഹിതമായി ഒരിന്റര്‍വ്യൂവില്‍ സമീപിക്കുന്നതു കണ്ടു. നല്ലതു്. നീണ്ടു വളര്‍ന്ന ഇടതൂര്‍ന്ന 
മുടിയോടുള്ള അഭിനിവേശമില്ലാത്ത ചില പെണ്‍കുട്ടികളെങ്കിലുമുണ്ടീലോകത്തു്.

അഴിച്ചിട്ടാല്‍ മുട്ടൊപ്പമെത്തുന്ന കുടുമയുമായി നടന്നിരുന്ന മാധവന്‍മാരില്‍ നിന്നും (ഇന്ദുലേഖ) 
പുരുഷസൌന്ദര്യസങ്കല്‍പ്പത്തില്‍ പറ്റെയൊതുക്കിയ മുടി സ്ഥാനം നേടിയതു്, കഴിഞ്ഞ നൂറ്റാണ്ടിന്റെ തുടക്കത്തിലെ 
നവോത്ഥാന പ്രസ്ഥാനങ്ങളെപ്പിന്‍പറ്റിയാകണം. ഞാനാദ്യം മുടിവളര്‍ത്തി വീട്ടിലെത്തിയപ്പോള്‍ ഒരു പരാമര്‍ശം തിരിച്ചു 
വരുന്ന കുടുമയെപ്പറ്റിത്തന്നെയായിരുന്നു. അന്നെന്തുകൊണ്ടാണാവോ പുരുഷന്‍മാര്‍ മാത്രം കുടുമമുറിച്ചതു്. എന്തായാലും 
നവോത്ഥാനകാലമായിരിക്കണം നീണ്ടമുടിയുടെ എല്ലാഭാരവും സ്ത്രീയിലേക്കുമാത്രമായി ചുരുക്കിയതു്.  ദേവികയുടെ 
പുരുഷകേന്ദ്രീകൃത നവോത്ഥാനശ്രമങ്ങളില്‍ സൃഷ്ടിക്കപ്പെട്ട സ്ത്രീ പുരുഷനാഗ്രഹിച്ച സ്ത്രീയാണെന്ന നിരീക്ഷണത്തില്‍ 
കുറച്ചെങ്കിലും ശരിയില്ലേ എന്നൊരു തോന്നല്‍ (കുലസ്ത്രീയും ചന്തപ്പെണ്ണും ഉണ്ടായതെങ്ങനെ).

സ്വാഭാവികമായിത്തന്നെ തഴച്ചു വളരുന്ന തലമുടിയുള്ള ആണ്‍കുട്ടികളെ നീട്ടിവളര്‍ത്തുന്നതില്‍ നിന്നും വിലക്കുകയും, മുടി 
കൊഴിച്ചിലും മറ്റു കേശസംബന്ധിയായ അസുഖവും മൂലം വിഷമിക്കുകയും ചെയ്യുന്ന പെണ്‍കുട്ടികളെ നീട്ടി വളര്‍ത്താത്തതിനു 
ഭര്‍സിക്കുകയും ചെയ്യുന്ന സമൂഹമാണു നമ്മുടേതു്.

മൂന്നു വര്‍ഷം നീണ്ടുവളര്‍ന്ന തലമുടിയുമായി നടന്ന അനുഭവത്തില്‍ നിന്നു പറയട്ടെ, തലമുടി വളര്‍ത്തുകയെന്നതും 
പരിപാലിക്കുകയെന്നതും വളരെ ചെലവേറിയ ഒരു പണിയാണു്. തേക്കുന്ന എണ്ണയും കഴുകുന്ന വെള്ളവും ചെളികളയാനും 
മര്യാദയ്ക്കു നില്‍ക്കാനും വേണ്ടി ഉപയോഗിക്കേണ്ടി വരുന്ന ഷാംപൂവും കണ്ടീഷനറും എല്ലാം പോക്കറ്റില്‍ വലിയ 
ദ്വാരങ്ങളാണുണ്ടാക്കുക. അതിനു പുറമേയാണു്, വെട്ടിയൊതുക്കി കൊണ്ടുനടക്കേണ്ടുന്നതിന്റെ ചെലവു്.

ഹൈദരാബാദ് നഗരത്തിലെ പ്രശസ്തവും അല്ലാത്തതുമായ മിക്ക യുണിസെക്സ് സലൂണുകളിലും ഇക്കഴിഞ്ഞ മൂന്നു 
വര്‍ഷത്തിനുള്ളില്‍ ഞാന്‍ പോയിട്ടുണ്ടു്. എല്ലായിടത്തുനിന്നും വിവിധ തരത്തിലെ മുടിവെട്ടുകളും നടത്തിയിട്ടുണ്ടു് (200 
‌മുതല്‍ 1500 രൂപ വരെ ചെലവുള്ളവ). എന്നാല്‍ ബില്‍ തരുമ്പോള്‍ എന്നും എന്റെ മേല്‍ അവര്‍ ഹെയര്‍ കട്ട് ഫോര്‍ 
വിമന്‍ നടത്തിയതിന്റെ വിലയാണു മേടിക്കാറ്. എപ്പോള്‍ ചോദിച്ചാലും പെണ്ണുങ്ങളേക്കാളും മുടിയുണ്ടായിരുന്നു സാര്‍ 
എന്നായിരിക്കും റിസപ്ഷനിലെ കുട്ടി പറയുന്നതു്.

ഈ പാര്‍ലറുകള്‍ പലതും സമൂഹത്തിലെ ഉന്നത ശ്രേണിയിലെ സോഷ്യലൈറ്റുകളുടെ നിത്യസന്ദര്‍ശനകേന്ദ്രങ്ങളാണു്. 
ലിംഗ, മത, വംശ വ്യത്യാസമില്ലാതെ പണത്തിന്റേയും കുടുംബമഹിമയുടെയും മാത്രം കാര്യം നോക്കി ആളുകളോടു 
ഇടപഴകുന്നവരാണു ഞങ്ങള്‍ എന്നുറക്കെ പ്രഖ്യാപിക്കുന്നവരുടെ സമൂഹത്തില്‍പ്പോലും നീണ്ടുവളര്‍ന്ന തലമുടി പെണ്‍കുട്ടിക്കു 
മാത്രം അവകാശപ്പെട്ടതാണെന്നു സാരം.

സമൂഹത്തോടു പുറംതിരിഞ്ഞു നില്‍ക്കുന്നതിനാലാവണം, എന്നെ പറഞ്ഞു നന്നാക്കാന്‍ ആദ്യകാലത്തു് എന്റെ 
അമ്മയല്ലാതെ വേറെയാരും ശ്രമിച്ചിട്ടില്ല. അമ്മതന്നെ, രണ്ടു പ്രാവശ്യം പറഞ്ഞിട്ടും കേള്‍ക്കാതായപ്പോള്‍ എന്നാല്‍ 
നിനക്കു മര്യാദയ്ക്കു വാലും തലയുമൊക്കെ ഒതുക്കി നടന്നുകൂടെ എന്ന ലൈനിലെത്തുകയും ചെയ്തു. ചില ചില്ലറ 
സംശയാലുക്കളെ അര്‍ഹിക്കുന്ന ഉത്തരങ്ങളിലൂടെ നിശബ്ദരാക്കുകയും കൂടിചെയ്തതോടെ ഞാന്‍ മുടി വളര്‍ത്തുന്നതില്‍ 
പരസ്യമായി എതിര്‍പ്പുള്ളവരുടെ എണ്ണം കുറഞ്ഞു.

മുടി വെട്ടാന്‍ തീരുമാനിക്കുന്ന ഒരു പെണ്‍കുട്ടിയേയും സമൂഹം ഈ രീതിയിലാണോ സ്വീകരിക്കുക എന്നറിയില്ല. മുടി പറ്റെ 
ബോബ് ചെയ്തു നടക്കുന്ന ഒരു ഡോക്റ്ററുണ്ടെനിക്കിവിടെ. അവരു മലയാളിയാണെന്നും പറയുന്നു (കേട്ടറിവു മാത്രമേയുള്ളൂ, 
ഇന്നുവരെ ഒരക്ഷരം മലയാളം പറഞ്ഞു കേട്ടിട്ടില്ല). അവരോടു ചോദിച്ചു നോക്കണം ഇനികാണുമ്പോള്‍. സ്വാഭാവികമായി 
തലയിലുണ്ടാവുകയും ആണ്‍ പെണ്‍ ഭേദമില്ലാതെ വളരുകയും ചെയ്യുന്ന മുടിയെന്ന സാധനത്തെപ്പിടിച്ചു 
സ്ത്രീലിംഗസ്വത്വത്തിന്റെ പ്രത്യക്ഷ അടയാളമാക്കിയതാരാണാവോ. തലമുടി പരിചരണത്തിന്റെ ബുദ്ധിമുട്ടു മനസ്സിലാക്കിയ സ്ത്രീ 
വിദ്വേഷിയായ ആരെങ്കിലുമായിരിക്കണം.

ഇത്തരത്തില്‍ ലിംഗപരമായി സ്ത്രീസ്വത്വമുള്ള നീണ്ടുവളര്‍ന്ന മുടി പല അനുഭവങ്ങളും സമ്മാനിച്ചിട്ടുണ്ടെനിക്കു്. 
ഒരുല്ലാസയാത്രയ്ക്കിടയ്ക്കു് പാറിപ്പറന്നുപോയ മുടിയൊതുക്കാന്‍പോയപ്പോള്‍ അവിടെ കൂടെയുണ്ടായിരുന്ന 
പെണ്‍കുട്ടികളുമുണ്ടായിരുന്നു. അവരോടു ഞാന്‍ കാര്യമായിത്തന്നെ, ഏതു ഷാംപൂവാണുപയോഗിക്കാറു്, എങ്ങനെയെണ്ണതേക്കും 
തുടങ്ങിയ കാര്യങ്ങളൊക്കെ അന്വേഷിച്ചു. അവരു കൃത്യമായ മറുപടിയും ഉപദേശങ്ങളും തരികയും ചെയ്തു. അതിനു ശേഷം അവര്‍ പിന്നീടൊരിക്കല്‍ പറഞ്ഞതു്, ജീവിതത്തിലൊരിക്കലും ഒരാണ്‍കുട്ടിയോടു നടത്തേണ്ടിവരും എന്നുകരുതിയ സംഭാഷണമല്ല 
അതെന്നാണു്.

എന്തായാലും ഒരേ ലാബില്‍ അപ്പുറത്തുമിപ്പുറത്തുമിരുന്നു ജോലിയെടുക്കുന്നവരായതിനാല്‍ ഈ വിഷയത്തില്‍ പിന്നെയും 
പല പ്രാവശ്യം സംഭാഷണങ്ങളുണ്ടായിട്ടുണ്ടു്. സലോണുകളെപ്പറ്റിയും, പുതിയ ഹെയര്‍സ്റ്റൈലുകളെപ്പറ്റിയുമടക്കം. 
സ്ഥിരമായി ഞാന്‍ കേട്ടിരുന്ന ഒരു ചോദ്യം എന്താ മുടിവളര്‍ത്താന്‍ കാരണമെന്നായിരുന്നു. ആണ്‍ പെണ്‍ ഭേദമില്ലാതെ 
പലരും ചോദിച്ചിട്ടുണ്ടു്. മറുപടി നിഷേധാത്മകമായിട്ടാണെങ്കിലും സത്യം തന്നെയാണു് ഞാന്‍ പറയാറുള്ളതും. ഞാന്‍ മുടി 
വളര്‍ത്താനല്ല, വെട്ടാതിരിക്കാനാണു തീരുമാനിച്ചതെന്നു്.

സ്വന്തം തലയില്‍ വളരുന്ന മുടി വെട്ടാനും വളര്‍ത്താനും സ്വാതന്ത്ര്യമനുവദിക്കുന്ന, മുണ്ഡനം ചെയ്തതലയിലും 
സൌന്ദര്യബോധം കാണാനും കഴിവുള്ള ഒരു ലോകസമൂഹം വളര്‍ന്നു വരുമെന്നു പ്രത്യാശിച്ചുകൊണ്ടു്.

പിന്‍കുറിപ്പ്:

തലമുടിയെക്കുറിച്ചുള്ള ഈ വിചാരങ്ങള്‍ക്കു കടപ്പാടു് തൊട്ടടുത്ത മുറിയില്‍ മുടിയില്ലാതെ കിടന്നിരുന്ന പെണ്‍കുട്ടി അവളുടെ ഇടതൂര്‍ന്ന മുടിയോടുകൂടിയ പൂര്‍വ്വാശ്രമചിത്രങ്ങള്‍ കാണിച്ചുതന്നപ്പോള്‍ ആ കണ്ണുകളില്‍ മിന്നിമറഞ്ഞ വികാരങ്ങള്‍ക്കു്.

(6 January 2011)\footnote{http://malayal.am/പലവക/9817/തലമുടിയെക്കുറിച്ച്-ഒരുപന്യാസം}

\newpage
