\newpage
\secstar{ജിനേഷ്, സ്വതന്ത്രസോഫ്റ്റ്‌‌വെയർ}
ജിനേഷിനെപ്പറ്റി അവന്റെ ഡെവലപ്പര്‍ എന്ന രീതിയിലുള്ള സംഭാവനകളെപ്പറ്റി ഈ കുറിപ്പെഴുതാനിരിക്കുമ്പോള്‍ ഒരു കാര്യം വളരെ വ്യക്തമാകുന്നു. ജിനേഷിനെപ്പറ്റി മാത്രമായി ഒരു ഓര്‍മ്മക്കുറിപ്പെഴുതാനാവില്ല. ജിനേഷിനെപ്പറ്റി പറയുന്നവ മിക്കതും അക്കാലത്തെ ഒരു സ്വതന്ത്രസോഫ്റ്റ്‌വെയര്‍ കൂട്ടത്തിന്റെയും അവരുടെ ഇടപെടലുകളുടെയും കൂടി ചരിത്രമാവുന്നുണ്ടു്.എങ്കിലും വിസ്താരഭയത്താല്‍ പരമാവധി ജിനേഷിലോട്ടും അവനുമായുള്ള വ്യക്തിപരമായ ചര്‍ച്ചകളിലോട്ടും അവന്റെ ചില കുറിപ്പുകളിലോട്ടും ഒതുക്കിയെടുക്കാന്‍ ശ്രമിക്കട്ടെ .അതോടൊപ്പം ഈ കുറിപ്പ് സ്വതന്ത്രമായ മലയാളം ഒസിആര്‍ എന്ന സ്വപ്നത്തെക്കുറിച്ചും അതിലേക്കായി ജിനേഷ് നടന്ന രണ്ടുവഴികളേയും കുറിച്ചാണു്. അനുബന്ധമായ സാങ്കേതിക സംബന്ധിയായ ബ്ലോഗ് പോസ്റ്റുകളെ ആ കാലഘട്ടവുമായി ബന്ധിപ്പിക്കുക എന്നതുമാത്രമാണു് ഈ കുറിപ്പ് ചെയ്യുന്നതും 
 
പണ്ട്   കോളേജില്‍ ഗ്നു/ഹര്‍ഡ് എന്ന സ്വതന്ത്ര ഓപ്പറേറ്റിങ്ങ് സിസ്റ്റത്തെപ്പറ്റി  ഒരു സെമിനാര്‍ എടുക്കാനായി ചില സംശയങ്ങളുന്നയിച്ചുള്ള  ഒരു മെയിലായാണു് 2006 ഓഗസ്റ്റ് അവസാനത്തില്‍ ഞാന്‍  ജിനേഷിനെ പരിചയപ്പെടുന്നതു് . ഒപ്പം ഒരു വില്ലേജ് ഓട്ടോമേഷന്‍ സോഫ്റ്റ്‌വെയര്‍ കോളേജിലെ സെമസ്റ്റര്‍ പ്രൊജക്റ്റായി നിര്‍മ്മിക്കാനുള്ള താല്പര്യവും, അങ്ങനെയൊന്നു് സ്വതന്ത്രസോഫ്റ്റ്‌വെയറില്‍ പ്രാവര്‍ത്തികമാക്കാന്‍ ഉള്ള സഹായങ്ങളും ആ മെയിലില്‍  അന്വേഷിച്ചിരുന്നു.  ഇങ്ങനെയുള്ള ചര്‍ച്ചകള്‍  ഹര്‍ഡ് കേണല്‍ ഡെവലപ്മെന്റിനെപ്പറ്റീയുള്ള ചര്‍ച്ചയായും കോളേജ് ടെക്നിക്കല്‍ മാഗസിനിലേക്ക് GPLv3 യെപ്പറ്റി എന്റെ ലേഖനം വാങ്ങലായും  ഫോണിലൂടെയുള്ള ചര്‍ച്ചകളും ഒക്കെയായി തുടര്‍ന്നുവന്നു. ഇതേ കാലത്താണു്, സ്വതന്ത്രമലയാളംകമ്പ്യൂട്ടിങ്ങിനെ പുനരുജ്ജീവിപ്പിക്കാനുള്ള ശ്രമങ്ങള്‍ പ്രവീണിന്റെയും എന്റെയും ഹിരണിന്റെയും മുന്‍കൈയില്‍ നടക്കുന്നതും ബൈജുവും പി.സുരേഷും ഹുസൈന്‍ മാഷും ഒക്കെ അടങ്ങുന്ന ഒരു ടീം രൂപപ്പെട്ടുവരുന്നതും. അക്കാലത്തു് ഞാന്‍ ഡയറക്ടര്‍ ബോര്‍ഡ് അംഗമായിരുന്ന ഗയ്യ എന്ന സംഘടയുടെ തൃശ്ശൂരിലെ ഓഫീസ് സ്വതന്ത്രമലയാളംകമ്പ്യൂട്ടിങ്ങിന്റെ മീറ്റിങ്ങുകള്‍ക്കായി ഉപയോഗിക്കാറുണ്ടായിരുന്നു. ഡെബിയന്‍ എച്ചിന്റെ മലയാള പരിഭാഷ പരിശോധിക്കാനും പാന്‍ഗോയിലെ മലയാള ചിത്രീകരണപ്രശ്നങ്ങള്‍ പരിഹരിക്കാന്‍ സുരേഷ്.പി നിര്‍മ്മിച്ച പാച്ച് റിവ്യൂ ചെയ്യാനുമായി സംഘടിപ്പിച്ചതായിരുന്നു സ്വതന്ത്രമലയാളംകമ്പ്യൂട്ടിങ്ങ് വീണ്ടും തുടങ്ങിയ ശേഷം നടക്കുന്ന രണ്ടാമത്തെ മീറ്റിങ്ങ്. ഈ യോഗത്തിലാണു് ജിനേഷിനെ ആദ്യം നേരില്‍ കാണുന്നതു്. ജിനേഷ് അടക്കം 13 പേര്‍ പങ്കെടുത്ത ഈ യോഗം വിലയിരുത്തി അയച്ച പാന്‍ഗോ പാച്ചാണു് കുറ്റമറ്റ മലയാളം റെന്‍ഡറിങ്ങ് ഡെബിയന്‍ എച്ചിലും  അതുപയോഗിക്കുന്ന ഐടി അറ്റ് സ്കൂളിലും വൈകാതെ പാന്‍ഗോയിലും സാധ്യമാക്കിയതു്. 

അതുകഴിഞ്ഞു് സ്വതന്ത്രമലയാളംകമ്പ്യൂട്ടിങ്ങിന്റെ യോഗങ്ങളിലും ഡെവലപ്പര്‍ സ്പ്രിന്റുകളിലും സ്ഥിരം സാന്നിധ്യമായിരുന്നു ജിനേഷ്. എംഇഎസ് കോളേജിലെ പുതിയ ഒരുപാടു വിദ്യാര്‍ത്ഥികളെ സ്വതന്ത്രമലയാളംകമ്പ്യൂട്ടിങ്ങിലേക്കും സ്വതന്ത്രസോഫ്റ്റ്‌വെയറിലേക്കും ആകര്‍ഷിക്കാനും അവനു കഴിഞ്ഞിരുന്നു. 2007ല്‍ പ്രവീണിനോടൊപ്പം സ്വതന്ത്രസോഫ്റ്റ്‌വെയര്‍ യൂസര്‍ഗ്രൂപ്പ് മലപ്പുറം എന്നൊരു ഗ്രൂപ്പിനും അവന്‍ തുടക്കമിട്ടു . 2007 ലെ ഗൂഗിള്‍ സമ്മര്‍ഓഫ് കോഡില്‍ സ്വതന്ത്രമലയാളംകമ്പ്യൂട്ടിങ്ങിനു കീഴില്‍ തെരഞ്ഞെടുക്കപ്പെട്ട ഒരു പ്രൊജക്റ്റ് ജിനേഷിന്റേതായിരുന്നു. ബോല്‍നാഗിരി രീതിയില്‍ ലളിത എന്ന XKB കീബോര്‍ഡ് , മലയാളം ടൈപ്പ്‌റൈറ്റര്‍, മിന്‍സ്ക്രിപ്റ്റ്‌ കീബോര്‍ഡുകള്‍ എന്നീ ഇന്‍പുട്ട് സംവിധാനങ്ങള്‍ ഈ പ്രൊജക്റ്റിന്റെ ഭാഗമായി  സുരേഷ്.പി യുടെ ഉപദേശങ്ങളനുസരിച്ചു് ജിനേഷ് നിര്‍മ്മിച്ചു . 

ഇതിനിടയില്‍ ബി.ടെക്കിനുശേഷം IIIT ഹൈദരാബാദിലെ CVITയില്‍ ജവഹര്‍ സാറിനു കീഴില്‍ മലയാളം ഒസിആര്‍ നിര്‍മ്മാണത്തിനും ഒപ്പം ഉപരിപഠനത്തിനും ആയി ജിനേഷ് ചേര്‍ന്നുകഴിഞ്ഞിരുന്നു. അക്കാലത്തു് സ്വതന്ത്രസോഫ്റ്റ്‌വെയര്‍  വിതരണങ്ങളില്‍ മലയാളപിന്തുണ പൂര്‍ണ്ണമായി ലഭ്യമായിരുന്നില്ല എന്നതുകൊണ്ടുതന്നെ സുപ്രധാന സ്വതന്ത്രസോഫ്റ്റ്‌വെയര്‍ വിതരണങ്ങള്‍ക്ക് സ്വതന്ത്രമലയാളംകമ്പ്യൂട്ടിങ്ങ് താല്‍ക്കാലിക സംഭരണികള്‍ പരിപാലിച്ചിരുന്നു . ഡെബിയന്റേത് പ്രവീണും ഫെഡോറയുടേതു് ജിനേഷും സെന്റ് ഓഎസ് , ആര്‍ച്ച്‌ലിനക്സ് എന്നിവയുടേതു് ആഷിക്കുമായിരുന്നു പരിപാലിച്ചിരുന്നതു്. അതോടൊപ്പം മലയാളം ഓസിആര്‍ റിസര്‍ച്ച് നടക്കുന്ന CVIT യിലെ ഫെഡോറ 7 അധിഷ്ഠിത സിസ്റ്റങ്ങളില്‍ സ്വതന്ത്രമലയാളംകമ്പ്യൂട്ടിങ്ങ് പുറത്തിറക്കുന്ന ഓരോ ടൂളുകളും ലഭ്യമാണെന്നു് അവന്‍ ഉറപ്പുവരുത്തുകയും ചെയ്തു . 

എന്റെ ഇടക്കുള്ള ഹൈദരാബാദ് യാത്രകളിലെ താമസം പതിവുള്ള സെന്റ്രല്‍ യൂണിവേഴ്സിറ്റി ഹോസ്റ്റലില്‍ നിന്ന് IIITയിലെ ജിനേഷ് ഒപ്പിച്ചു തരുന്ന ഗസ്റ്റ് റൂമുകളിലേക്ക് മാറുന്നതും ഇതേ കാലത്തുതന്നെ . എന്റെ മീറ്റിങ്ങുകള്‍ കഴിഞ്ഞെത്തിയാല്‍ പുലര്‍ച്ചെ രണ്ട് മൂന്നു മണി വരെ അവരുടെ ലാബിലും , ഭക്ഷണത്തിനു് കാമ്പസ്സിനു പുറത്തെ കൈരളി എന്ന മലയാളി മെസ്സിലും  രാത്രി ചായകിട്ടുന്ന കാമ്പസ്സിനുള്ളിലെ ഡാബയ്ക്കരികിലും ഒക്കെയായി ഞങ്ങളുടെ ചര്‍ച്ചകള്‍ നടക്കുമായിരുന്നു . മലയാളം ഒസിആറില്‍ അവനോടൊപ്പം വര്‍ക്ക് ചെയ്തിരുന്ന നീബയും  അവന്റെ എംഇഎസ്സിലെ  ക്ലാസ്മേറ്റ് IIITയില്‍ തന്നെയുള്ള സുഹൈലും ചിലപ്പോഴെല്ലാം കാണാറുമുണ്ടു്. വിഷയം മലയാളം കമ്പ്യൂട്ടിങ്ങും , അക്കാലത്തെ OOXML ഡോക്യുമെന്റ് സ്റ്റാന്‍ഡേര്‍ഡൈസേഷന്‍ ഡിബേറ്റും , IIIT വിശേഷങ്ങളും മുതല്‍ അവരെഴുതുന്ന സോഫ്റ്റ്‌വെയറിന്റെ വിശദാംശങ്ങള്‍ വരെ എന്തും ആവാറുണ്ടു്. ജിനേഷ് പൂര്‍ണ്ണമായും മലയാളം ഒസിആര്‍ പ്രൊജക്റ്റില്‍ മുങ്ങിയ കാലമായിരുന്നു അതു് . 

അവരുണ്ടാക്കുന്ന മലയാളം ഒസിആര്‍ സ്വതന്ത്രമാവില്ലേ എന്നുള്ള എന്റെ ഓരോ തവണയുമുള്ള ചോദ്യത്തിനു് അതുറപ്പാണെന്നും ജവഹര്‍ സാറിന്റെ പൂര്‍ണ്ണപിന്തുണയുണ്ടെന്നും കോഡില്‍ വരെ ജിപിഎല്‍ ഇട്ടിട്ടാണ് അവന്‍ കോഡ് ചെയ്യുന്നതെന്നും പറയാറുണ്ടായിരുന്നു. ഇന്ത്യയില്‍ സ്വതന്ത്രസോഫ്റ്റ്‌വെയര്‍ പ്രവര്‍ത്തകര്‍ ഒരു പാടു കാമ്പൈനുകളില്‍ ഏര്‍പ്പെട്ട സമയമായിരുന്നു 2007 ഉം 2008ഉം. സ്വതന്ത്ര സ്റ്റാന്‍ഡേര്‍ഡുകള്‍ക്കുവേണ്ടിയുള്ള നാഷണല്‍ കാമ്പൈനും പരിഷ്കരിച്ച പേറ്റന്റു് മാനുവല്‍ ഉപയോഗിച്ച് ഇന്ത്യന്‍ പാര്‍ലമെന്റ് തള്ളിയ സോഫ്റ്റ്‌വെയര്‍ പേറ്റന്റുകള്‍ പിന്‍വാതില്‍ വഴി കൊണ്ടുവരാനുള്ള ശ്രമങ്ങള്‍ക്കെതിരായ കാമ്പൈനും  ഒക്കെ ഈ സമയത്താണു് നടക്കുന്നതു് . കേന്ദ്രഗവണ്‍മെന്റ് പണം ചിലവഴിക്കുന്ന അക്കാദമിക് ഗവേഷണത്തിന്റെ മുഴുവന്‍ കോപ്പിറൈറ്റ് , ഗവേഷകര്‍ക്കു് തീരുമാനിക്കാനിവില്ലെന്നും അതു് സര്‍ക്കാരിനാവണമെന്ന ഒരു വിവാദ നിര്‍ദ്ദേശവും ഇക്കാലത്തു  വന്നിരുന്നു. ഈ കാലയളവില്‍ ഞാനടക്കം ഉള്ള സ്വതന്ത്ര സോഫ്റ്റ്‌വെയര്‍ പ്രവര്‍ത്തകരുടെ മുന്‍കൈയില്‍ സ്വതന്ത്രമലയാളംകമ്പ്യൂട്ടിങ്ങും ഫ്രീ സോഫ്റ്റ്‌വെയര്‍ യൂസര്‍ഗ്രൂപ്പ് ബാംഗ്ലൂരും അടക്കം 20 സംഘടനകള്‍ ചേന്നു് സോഫ്റ്റ്‌വെയര്‍ പേറ്റന്റുകള്‍ക്കെതിരെ  ഒരു നാഷണല്‍ പബ്ലിക് മീറ്റിങ്ങ്  ബാംഗ്ലൂരില്‍ വച്ച് 2008 ആഗസ്റ്റില്‍ നടത്തിയിരുന്നു. അതിനു തൊട്ടുപിന്നലെ ഒരു ദിവസം ജിനേഷ് വിളിച്ചു് ഓപ്പണ്‍സോഴ്സില്‍ നിന്ന് ഈ പ്രൊജക്റ്റ് മാറ്റാനുള്ള ശ്രമങ്ങള്‍ നടക്കുന്നുണ്ടെന്നും ഡിഫന്‍സീവ് പേറ്റന്റുകള്‍ എന്ന പേരില്‍ ഒസിആര്‍ പ്രൊജക്റ്റില്‍ സോഫ്റ്റ്‌വെയര്‍ പേറ്റന്റുകള്‍ എടുക്കാനുള്ള ശ്രമങ്ങള്‍  കേന്ദ്ര ഐടി ഡിപ്പാര്‍ട്ട്മെന്റിന്റെ ഇന്റലക്ച്വല്‍ പ്രോപ്പര്‍ട്ടി സെല്ലിന്റെ നിര്‍ദ്ദേശാനുസരണം നടക്കുന്നുണ്ടെന്നും അങ്ങനെയുള്ള കുത്തകവല്‍ക്കരണം തടയാനായി എന്തൊക്കെ ചെയ്യാം എന്നും ജവഹര്‍ സാറിനാലോചിച്ച് അദ്ദേഹത്തോടൊപ്പം ഇത്തരം കാര്യങ്ങള്‍ ഉയര്‍ത്താം എന്നും പറഞ്ഞിരുന്നു. 

അതിനുപിന്നലെ കേന്ദ്രഗവണ്‍മെന്റിന്റെ പിന്തുണയോടെ മലയാളം ഓസിആര്‍ അടക്കം ഉള്ള ഇന്ത്യന്‍ഭാഷ ഒസിആറുകള്‍ നിര്‍മ്മിക്കുന്നതു് ഒത്തൊരുമിപ്പിക്കുന്ന  ഒസിആര്‍ കണ്‍സോര്‍ഷ്യത്തിന്റെ  ഇന്റേണല്‍ മെയിലിങ്ങ് ലിസ്റ്റില്‍ വന്ന കണ്‍സോര്‍ഷ്യം  പ്രൊജക്റ്റുകളുടെ 'ബൌദ്ധികസ്വത്തവകാശം' സംരക്ഷിക്കുന്നതിനെപ്പറ്റിയും കോപ്പിറൈറ്റുകളും പേറ്റന്റുകളും സംരക്ഷിച്ച് എങ്ങനെ വിതരണം നടത്തണമെന്നതിനെപ്പറ്റി മെമ്പര്‍മാരുടെ അഭിപ്രായമാരായുന്ന കേന്ദ്ര ഐടി ഡിപ്പാര്‍ട്ട്മെന്റിന്റെ  മെയില്‍ എനിക്കു ഫോര്‍വേഡ് ചെയ്ത് ജിനേഷ് ഇങ്ങനെയെഴുതി.  
\begin{english}
hi,

This is a letter which came to the OCR consortium mailing list from
Department of IT. It doesn't talk about licensing but by distributing, i
think they meant licensing. Close advisers of ministry may not appreciate an
Open Source based move. I will update you as things go on. I think
discussion on licensing is going to happen very soon and external pressure
will be helpful for consortia to put forward its idea of FREEing the
outcome. It was decided in the beginning about open source but not
specifically about any license. So, i think we need to pressure on that too.
I am not in a position to pressure but i think you have enough sources to
ask for the details and pressure on the regard.

If i get a chance i will discuss with sir also about the meeting you
suggested. Currently we(consortia) are in hurry to meet targets and since
the project time span is exceeding the timeline, we(consortia) are on
defence kind of situation.

A lot of money got pumped into the consortia efforts and i believe it is
important to keep these things in Freedom domain(on OCR side, mainly
development of techniques to handle problems specific to Indian Languages
happened a lot and at least a good structure and whole lot of
implementations is available including a complete annotated set of training
and testing data).

Please go through it and i will get to you the responses from the consortia
also(if i get an access to those).

cheers

Jinesh K J
\end{english}

ഇതെത്തുടര്‍ന്നു് ഞങ്ങള്‍ ഈ വിഷയത്തില്‍ ഒരുപാടു് സംസാരിക്കുകയുണ്ടായി . അല്‍ഗോരിതങ്ങളും ബിസിനസ് മെത്തേഡുകളും , മാത്തമാറ്റിക്കല്‍ ഇക്വേഷനുകളും കമ്പ്യൂട്ടര്‍ പ്രോഗ്രാമുകളും പേഠ്ടന്റബിളല്ലെന്നു ഇന്ത്യന്‍ പേറ്റന്റ് ആക്റ്റ് പറയുമ്പോള്‍ തന്നെ തുടര്‍യോഗങ്ങളില്‍ ഏതോ ഇമേജ് റക്കഗ്നിഷന്‍ അല്‍ഗോരിതത്തിനുമേല്‍ ഐസിഐസിഐ ബാങ്ക് എടുത്ത ചട്ടവിരുദ്ധമായ പേറ്റന്റ് കാട്ടി, ഡിഫന്‍സ് പേറ്റെന്റെടുത്തില്ലെങ്കില്‍ പേറ്റന്റ് ട്രോളുകളുടെ പിടിയില്‍ പെടുമെന്നു പ്രൊഫസര്‍മാരെ ഉപദേശിക്കുന്ന കേന്ദ്രഗവണ്‍മെന്റിന്റെ ഐടി വകുപ്പിന്റെ സമീപനവും  പേറ്റെന്റെടുക്കാന്‍ പ്രൊജക്റ്റില്‍ തുക വക കൊള്ളിച്ചില്ലെങ്കില്‍ കൂടി പേറ്റന്റിനു് പണം നല്‍കാന്‍ കേന്ദ്രഗവണ്‍മെന്റ് തയ്യാറാണെന്ന വസ്തുതയും കൂടി ഈ പ്രൊജക്റ്റ് പേറ്റന്റുകളുടെ പിടിയിലേക്കു പോകുമെന്നു് അപ്പോഴേക്കും ഉറപ്പായി കഴിഞ്ഞിരുന്നു. അതിനിടയില്‍ ചാറ്റില്‍ ഒരു ദിവസം ജിനേഷ് ഇങ്ങനെയെഴുതി 

\begin{english}
we have to go through process since ministry is saying IPR generated will be part of project evaluation!\\
profs are saying like, if all legal matters will be taken care by ministry, applying for patent is pretty easy than getting a paper accepted in International Journals\\
since ministry said, they dont take any money from our account, they are like we will do what we are asked for!
\end{english}

2009 ജനുവരിയോടെ പേറ്റന്റുകള്‍ക്കപ്പുറം ഈ പ്രൊജക്റ്റ് സ്വതന്ത്ര സോഫ്റ്റ്‌വെയറിലിറങ്ങില്ലെന്നും റിസര്‍ച്ച് ആവശ്യങ്ങള്‍ക്കു മാത്രം ലഭ്യമാവുകയേ ഉള്ളൂ എന്നും ഏകദേശം ഉറപ്പായിക്കഴിഞ്ഞിരുന്നു . 2008 ഡിസംബറില്‍  ഐക്യരാഷ്ട്ര സഭയുടെ ഇന്റര്‍നെറ്റ് ഗവര്‍ണന്‍സ് ഫോറം ഹൈദരാബാദില്‍ വച്ചു നടക്കുമ്പോള്‍ ഞാന്‍ IIIT ഹൈദരാബാദിലെ ഹോസ്റ്റല്‍ ഗസ്റ്റ് റൂമില്‍ താമസിച്ചായിരുന്നു  ഓരോ ദിവസവും പോയിവന്നിരുന്നതു്. ജിനേഷ് അന്നു് മുടിയൊക്കെ നീട്ടി വളര്‍ത്തി അടിപൊളിയായി നടക്കുന്ന സമയം. ആ സമയത്തു് ജിനേഷ് ലൈസന്‍സിങ്ങ് പേറ്റന്റ് പ്രശ്നത്തില്‍ ജവഹര്‍സാറും ഒരു സ്റ്റുഡന്റായ അവനും ഇതു സ്വതന്ത്രമാക്കാന്‍ നടത്തിയ ശ്രമങ്ങള്‍ എങ്ങനെ വിജയിക്കാതെ പോയി എന്നു വിശദമാക്കിയിരുന്നു. സ്വതന്ത്രമലയാളം കമ്പ്യൂട്ടിങ്ങിന്റെ ശില്പ പ്രൊജക്റ്റിന്റെയൊക്കെപോലെ ഒരു ഓണ്‍ലൈന്‍ സര്‍വ്വീസായിട്ടെങ്കിലും മലയാളം ഒസിആര്‍ ജനങ്ങള്‍ക്കു ലഭ്യമാക്കാന്‍ ജവഹര്‍ സാര്‍ ശ്രമിക്കുന്നതിനെപ്പറ്റിയും പറഞ്ഞിരുന്നു. 2008 ല്‍ തന്നെയാണു് ഡെബായന്‍ ബാനര്‍ജി എന്ന വിദ്യാര്‍ത്ഥി ടെസറാക്റ്റില്‍ ഇന്ത്യന്‍ ഭാഷകള്‍ക്കുള്ള പിന്തുണ കൊണ്ടുവരാനുള്ള ശ്രമങ്ങള്‍ തുടങ്ങിവെച്ചതു് . മലയാളം അടക്കമുള്ള complex script അധിഷ്ഠിതമായ ഇന്ത്യന്‍ ഭാഷകള്‍ക്കുള്ള പിന്തുണക്കായി ടെസറാക്റ്റ്ഇന്‍ഡിക് എന്ന പ്രൊജക്റ്റ് ഡെബായന്‍ ഡല്‍ഹിയിലെ 'സരൈ' എന്ന സംഘടനയുടെ ഫെല്ലോഷിപ്പോടെ നിര്‍മ്മിക്കുന്ന സമയമായിരുന്നു അതു് . IIITയിലെ മലയാളം ഒസിആര്‍ പ്രൊജക്റ്റിന്റെ പ്രവര്‍ത്തിപരിചയം ഉപയോഗിച്ചുകൊണ്ട് ലഭ്യമായ സമയത്തു് ടെസറാക്റ്റിലെ മലയാള പിന്തുണ മെച്ചപ്പെടുത്തുകയും അങ്ങനെ സ്വതന്ത്രമായ മലയാളം ഒസിആര്‍ എന്ന സ്വപ്നം നടത്താന്‍ ശ്രമിച്ചുകൂടെ എന്നു ഞാന്‍ ജിനേഷിനോടു് അന്വേഷിക്കുകയും  ചെയ്യാമെന്നു് അവന്‍ ഏല്‍ക്കുകയും ചെയ്തു 

2009 മാര്‍ച്ച് ആദ്യവാരത്തില്‍ ജിനേഷ് അതിലെ പുരോഗതിയെപ്പറ്റി സ്വതന്ത്രമലയാളം കമ്പ്യൂട്ടിങ്ങ് ലിസ്റ്റില്‍ എഴുതിയ കുറിപ്പ് ജിനേഷിന്റെ ബ്ലോഗ് പോസ്റ്റുകളില്‍  My experiments with Tesseract എന്ന പേരില്‍ കാണാം .
അതിനു പിന്നാലെ ജിനേഷ് ചെയ്യേണ്ട പ്രവര്‍ത്തനങ്ങള്‍ സ്വതന്ത്രമലയാളം കമ്പ്യൂട്ടിങ്ങ് വിക്കിയില്‍ \footnote{\url{http://wiki.smc.org.in/OCR}} ടെസറാക്റ്റില്‍ ഇനി ചെയ്യേണ്ട പ്രവര്‍ത്തനങ്ങളെപ്പറ്റി ഇങ്ങനെ എഴുതി 

---
ഇന്ന് ലഭ്യമായ സ്വതന്ത്ര ഓസിആര്‍ സംവിധാനങ്ങളില്‍, എറ്റവും മികച്ചതാണ് ടെസ്സറാക്റ്റ്. ഇംഗ്ലീഷീനും മറ്റു ലാറ്റിന്‍ ഭാഷകളിലും സുഗമമായി പ്രവര്‍ത്തിക്കുന്ന ടെസ്സറാക്റ്റ് യുണികോഡ് വളരെ നല്ല രീതിയില്‍ പിന്തുണയ്ക്കുകയും ചെയ്യുന്നുണ്ട്. 
ടെസ്സറാക്റ്റിനു മുകളിലുള്ള പരീക്ഷണങ്ങളില്‍ നമ്മള്‍ പ്രധാനമായി ഉന്നം വയ്ക്കുന്നതിവയാണ്,

\begin{itemize}
\item സിംബല്‍ ക്ലാസിഫിക്കേഷന്‍ സംവിധാനം മലയാളത്തിന് 99\% കൃത്യത നല്‍കുമെന്നുറപ്പാക്കുക.
\item പ്രീ-പോസ്റ്റ് പ്രോസസ്സിങ് സംവിധാനത്തില്‍ വേണ്ട മാറ്റങ്ങള്‍ വരുത്തുക. 
\end{itemize}

ഇപ്പോള്‍ ഉണ്ടാക്കിയിട്ടുള്ള ഒരു രൂപരേഖ ഏതാണ്ടിങ്ങനെയാണ്,

\begin{itemize}
\item  ടെസ്സറാക്റ്റിനെ സാധാരണകാണുന്ന മലയാളം സിംബലുകള്‍ക്കായി പരിശീലിപ്പിക്കുക.
\item  ഈ ട്രെയിന്‍ ചെയ്തെടുത്ത ടെസ്സറാക്റ്റ് സാമാന്യം വലിയ ഒരു കോര്‍പ്പസില്‍ ടെസ്റ്റ് ചെയ്യകയും, റിസല്‍ട്ടുകള്‍ വിശദമായി വിശകലനം ചെയ്യകയും ചെയ്യുക.
\item  ടെസ്സറാക്റ്റിന്റെ കോഡും വര്‍ക്ക് ഫ്ലോയും വിശദമായി മനസ്സിലാക്കുക.
\item  എറര്‍ സോഴ്സുകള്‍ മനസ്സിലാക്കാന്‍ വിവിധതരം പരീക്ഷണങ്ങള്‍ തയ്യാറാക്കുകയും നടത്തുകയും ചെയ്യുക.
\item  ആവശ്യമെങ്കില്‍ പുതിയ വര്‍ക്ക്ഫ്ലോയും മെത്തേഡുകളും ഉണ്ടാക്കുക. 
\end{itemize}
---

ഇതോടൊപ്പം മലയാളം ടെക്സ്റ്റ് സ്കാന്‍ ചെയ്യുമ്പോള്‍ കിട്ടുന്ന മലയാളത്തിലെ പ്രീബേസ് പോസ്റ്റ്ബേസ് അക്ഷരങ്ങളുടെയും സ്വരചിഹ്നങ്ങളുടെയും പുനക്രമീകരണത്തിനായി ഒരു ചെറിയ പൈത്തണ്‍ പ്രോഗ്രാമും ജിനേഷ് എഴുതി \footnote{\url{https://code.google.com/p/tesseractindic/downloads/detail?name=vowel_reordering.tar.gz&can=2&q=}}. ഇക്കാലത്തു് CVIT യിലെ ഡോക്യുമെന്റ് അനാലിസിസ് പ്രൊജക്റ്റില്‍ വന്ന ഒഴിവില്‍ സ്വതന്ത്രമലയാളം കമ്പ്യൂട്ടിങ്ങിലെ ആഷിക്കിനെ ജിനേഷ് വിളിക്കുകയും അവനവിടെ ചേരുകയും ചെയ്തു. ഒപ്പം \url{mukth.in} എന്ന ഹൈദരാബാദില്‍ നടക്കാറുള്ള സ്വതന്ത്രസോഫ്റ്റ്വെയര്‍ കോണ്‍ഫറന്‍സ് IIITയില്‍ നടത്തിയാലോ എന്നുള്ള ആലോചനകളും പരിശ്രമങ്ങളും ഈ സമയത്തു് ജിനേഷ് സുഹൃത്തുക്കളോടൊപ്പം നടത്തുന്നുണ്ടായിരുന്നു. 

സ്വതന്ത്രമലയാളംകമ്പ്യൂട്ടിങ്ങില്‍ ജീനേഷ് ഒരു ഇടവേളക്കു ശേഷം വളരെ സജീവമായ സമയമായിരുന്നു അതു്. അക്കാദമിക് കോണ്‍ഫറന്‍സുകളില്‍ സ്വതന്ത്രമലയാളംകമ്പ്യൂട്ടിങ്ങ് പങ്കെടുക്കേണ്ടതിന്റെ അനിവാര്യതയെപ്പറ്റിയും പേപ്പറുകള്‍ ജേര്‍ണലുകളില്‍ പ്രസിദ്ധപ്പെടുത്തേണ്ടതിനെപ്പറ്റിയും സാങ്കേതികപ്രവര്‍ത്തനങ്ങള്‍ക്കപ്പുറം അങ്ങനെ ഒരു അക്കാദമിക് മുഖം കൂടി സ്വതന്ത്രമലയാളം കമ്പ്യൂട്ടിങ്ങിനുണ്ടാക്കേണ്ടതിനെപ്പറ്റിയും അവന്‍ ഓര്‍മ്മിപ്പിക്കുമായിരുന്നു. നിരവധി തിരക്കുകളുള്ള വ്യക്തികള്‍ ജോലിക്കു ശേഷമുള്ള സമയം മലയാളം കമ്പ്യൂട്ടിങ്ങിനുപയോഗിക്കുന്നതു് സോഫ്റ്റ്‌വെയര്‍ കോണ്ട്രിബ്യൂഷനുകള്‍ക്കുപയോഗിക്കണോ അതോ പേപ്പറെഴുതാന്‍ ഉപയോഗിക്കണോ എന്ന ചര്‍ച്ച എവിടെയും എത്താറും ഉണ്ടായിരുന്നില്ല. വ്യക്തമായ സംഘടനാരൂപമുണ്ടായതിനുശേഷം മതി ആ ആലോചന എന്ന അഭിപ്രായമായിരുന്നു അതില്‍ എന്റേതു് . അന്നുമുതലേ സംഘടനാ രജിസ്ട്രേഷന്‍ എന്നതു് ജിനേഷ് ഇടക്കിടെ ഓര്‍മ്മപ്പെടുത്താനും തുടങ്ങി. Natural Language Processing (NLP) എന്ന രംഗത്ത് സ്വതന്ത്രമലയാളംകമ്പ്യൂട്ടിങ്ങിന്റെ ശില്പ പ്രൊജക്റ്റിന്റെ നേട്ടങ്ങള്‍ ഒരു പേപ്പറാക്കണമെന്നു് അവന്‍ സ്ഥിരം പറയാറുണ്ടായിരുന്നു .
ജിനേഷ് കൂടി പങ്കാളിയായ  നാലു് അക്കാദമിക് പേപ്പറുകള്‍ അതിനകം പ്രസിദ്ധീകരിച്ചിരുന്നു . അവയുടെ വിവരങ്ങള്‍ താഴെ 

\begin{english}
\begin{enumerate}
\item
Managing multilingual OCR project using XML\\
Gaurav Harit, K. J. Jinesh, Ritu Garg, C. V. Jawahar, Santanu Chaudhury\\
July 2009\\
MOCR '09: Proceedings of the International Workshop on Multilingual OCR\\
Publisher: ACM

\item 
Towards recognition of degraded words by probabilistic parsing\\
Karthika Mohan, K. J. Jinesh, C. V. Jawahar\\
December 2010\\
ICVGIP '10: Proceedings of the Seventh Indian Conference on Computer Vision, Graphics and Image Processing\\
Publisher: ACM

\item
Book Chapter in "Guide to OCR for Indic Scripts"\\
Advances in Pattern Recognition 2010, pp 3-25\\
Publisher: Springer\\
Building Data Sets for Indian Language OCR Research\\
C.V. Jawahar, Anand Kumar, A. Phaneendra, K.J. Jinesh

\item
Book Chapter in "Information Systems for Indian Languages"\\
Communications in Computer and Information Science Volume 139, 2011, pp 86-91\\
Publisher: Springer\\
On Multifont Character Classification in Telugu\\
Venkat Rasagna, K. J. Jinesh, C. V. Jawahar\\
\end{enumerate}
\end{english}

ജിനേഷുമായി മുക്ത് എന്ന സ്വതന്ത്രസോഫ്റ്റ്‌വെയര്‍ കോണ്‍ഫറന്‍സിനെപ്പറ്റിയും ലൈസന്‍സിങ്ങിനെപ്പറ്റിയും നടത്തിയ ഒരു സംഭാഷണമാണു് താഴെ 

\begin{english}
7:26 PM jinesh: Deciding dates for mukth.in will take little more time :) \\
  only with sakthi from outside am not ready to run it here\\
 me: Decide dates and time frame . അല്ലാതതു നടക്കില്ല \\
  Riyaz is willing to help\\
7:27 PM jinesh: it will then simply become a iiit event, which doesnt do any good for user group in hyderabad\\
  we can look into\\
 me: If it is so call them first and tell them to organise support outside first\\
7:28 PM jinesh: yeah\\
  i will do it tomorrow and sakthi actually wants to come over here once\\
 me: You only need to take care about internal issues, logistics and iiitlug involment\\
 jinesh: understand the ground.   i know\\
  but the faculty is a complex issue to handle\\
  they said like, iiit students should also benefit from the stuff\\
  and it seems some one organized something similar before and it ended up as a mess\\
  so they are very skeptic about having something\\
7:31 PM me: You can utilize this as an opportunity for Higlighting the FOSS contributions from IIIT and make more awareness on FOSS  and its Licensing. So issues you faced on OCR will not come again with better awareness. \\
7:33 PM jinesh: thats ok\\
7:34 PM i talked with a faculty, she is like faculty is confused on many fronts of really implementing foss such that industry simply wont take advantage of us\\
7:35 PM she said like a panel discussion with people from industry, community and academia is something they look forward to\\
7:38 PM on the licensing, industry attitude, etc.\\
 me: That is good\\
7:39 PM jinesh: by the way, she was telling like, people from TCS once asked her that all research done by govt. institutes are to be published freely(which is good in one sense) so for them to use it\\
7:40 PM me: :-p\\
  License it\\
  public domain is a trap\\
 jinesh: then it seems she said like we are not govt.\\
7:41 PM jinesh: even she asked like what will i do if some one packages it so nicely that his violations are not easily trackable\\
  you see the thing is institute doesnt like to go behind the law and stuff a lot\\
 me: that is the problem with public domain and reserchers access to code. it is better to license the work always under gpl compatib\\
  thats good\\
7:42 PM what ever we publish is now published with GPL/Apache only\\
7:43 PM jinesh: but who will keep track of things\\
  you see, the project we work on, codes are handled by a lot of people, mainly CDAC\\
 me: Who will keep track of resercher only access license, which is now popularly used in DIT funded projects\\
7:44 PM jinesh: and in case of OCR, CDAC is having a notorious reputation of stalling indic ocr research for 10 years\\
 me: Licensing offers a better solution. \\
  Thats why licensing on each file getting importent\\
 jinesh: yeah\\
7:45 PM me: Each researcher usually owns copyright of their paper. So why not code\\
 jinesh: even for code, we put copyright,\\
  you know what happened, CDAC funded a research and bought everything related to it\\
7:46 PM jinesh: they never made it public\\
 me: That is needed. I will say you must put copyright and license on each file you code. \\
 jinesh: by the way, anyway  I am using templated in Emacs/Vim to add GPL to beginning of each file?\\
 jinesh: since, now we dont have any proof that the code we gave is ours and if someone pulls us into court we cant do anything\\
\end{english}

എന്തായാലും ആ കോണ്‍ഫറന്‍സ് പിന്നെ പലകാരണങ്ങളാല്‍ നടന്നില്ല. ഇതിനിടയില്‍ ഇന്ത്യന്‍ ഇന്‍സ്റ്റിറ്റ്യൂട്ട് ഓഫ് സയന്‍സില്‍ വച്ചു നടക്കുന്ന ഒസിആര്‍ കണ്‍സോര്‍ഷ്യം  യോഗങ്ങള്‍ക്കായി ജിനേഷ് ബാംഗ്ലൂരില്‍ വരുമ്പോള്‍ എന്റെ ഒപ്പം താമസിക്കാറുണ്ടായിരുന്നു. അക്കാലത്താണു് സെന്റര്‍ ഫോര്‍ ഇന്റര്‍നെറ്റ് ആന്റ് സൊസൈറ്റി ചെന്നൈയില്‍വെച്ചു നടത്തുന്ന National Conference on ICTs for the differently- abled/under privileged communities in Education, Employment and Entrepreneurship 2009 - (NCIDEEE 2009)\footnote{\url{http://cis-india.org/events/ncideee-2009}} എന്ന കോണ്‍ഫറന്‍സിനെപ്പറ്റിയും അതിലേക്കു് ഒസിആര്‍ കണ്‍സോര്‍ഷ്യത്തിനു് ക്ഷണം ലഭിച്ചിട്ടുണ്ടെന്നും ജിനേഷ് പറയുന്നതു് . അതു് സ്വതന്ത്രസോഫ്റ്റ്‌വെയറുകളില്‍ ചിതറിക്കിടക്കുന്ന നിരവധി പ്രൊജക്റ്റുകളെ സംയോജിപ്പിക്കാന്‍ ഉള്ള അവസരമായി ഞങ്ങള്‍ കരുതുകയും ടെസറാക്റ്റ് ഒസിആര്‍ നിര്‍മ്മാണത്തില്‍ ഇടപെടുന്ന ഡെബായന്‍ ബാനര്‍ജിയേയും ധ്വനി സ്വരസംവേദിനി നിര്‍മ്മിക്കുന്ന സ്വതന്ത്രമലയാളംകമ്പ്യൂട്ടിങ്ങിലെ  സന്തോഷ് തോട്ടിങ്കലിനേയും , ഓര്‍ക്ക എന്ന സ്പീച്ച് എഞ്ചിന്‍ നിര്‍മ്മാണത്തില്‍ സഹകരിക്കുന്ന അന്ധപ്രോഗ്രാമര്‍ കൃഷ്ണകാന്ത് മനെയേയും ക്ഷണിക്കാന്‍ ആവശ്യപ്പെട്ടു് ഞാന്‍ ആ സംഘടനയുടെ എക്സിക്യുട്ടീവ് ഡയറക്ടര്‍ സുനില്‍ എബ്രഹാമിനു് എഴുതുകയും ചെയ്തു. സ്വതന്ത്രസോഫ്റ്റ്‌വെയര്‍ ശ്രമങ്ങളേയും സര്‍ക്കാര്‍ ഫണ്ടണ്ട് അക്കാദമിക് പ്രൊജക്റ്റുകളേയും ഒരേ മേശയ്ക്കു ചുറ്റും കൊണ്ടുവരാനുള്ള ഒരവസരമായും അതിന്റെ സാധ്യതകളേയും വളരെ പ്രതീക്ഷയോടെയാണു് ജിനേഷ് കണ്ടിരുന്നതു് . ഒരേ സമയം IIITയിലെ പ്രൊജക്റ്റിനെ അവിടെ പ്രതിനിധീകരിക്കാന്‍ പോകുമ്പോള്‍ തന്നെ ജിനേഷിന്റെ മനസ്സ് പൂര്‍ണ്ണമായും സ്വതന്ത്രസോഫ്റ്റ്‌വെയര്‍ രംഗത്തായിരുന്നു. ടെസറാക്റ്റ് പ്രൊജക്റ്റ്എന്തൊക്കെ  പോയന്റുകള്‍ പറയണമെന്നു് ഡേബായനും ഞാനും ജിനേഷുമായി നിരവധി മെയിലുകള്‍ നടന്നിരുന്നു. എന്നാല്‍ കോണ്‍ഫറന്‍സ് പൂര്‍ണ്ണമായും സര്‍ക്കാര്‍ പ്രൊജക്റ്റുകള്‍ക്കായി വിട്ടുകൊടുത്ത രീതിയിലായിരുന്നു. ജിനേഷിന്റെ കോണ്‍ഫറന്‍സിനുശേഷമുള്ള നിരാശയും ദേഷ്യവും ആദ്യം എനിക്കെഴുതുകയും പിന്നെ  ഒരു ബ്ലോഗ് പോസ്റ്റാക്കി മാറ്റുകയും ചെയ്തു \footnote{\url{http://logbookofanobserver.wordpress.com/2009/12/13/conference-on-icts-for-differently-abledunderprivileged-in-education-employment-and-entrepreneurship/}} സ്വതന്ത്രസോഫ്റ്റ്‌വെയര്‍ അധിഷ്ഠിതമായ ആക്സസിബിലിറ്റി പ്രൊജക്റ്റുകള്‍ക്കായി ദേശീയതലത്തില്‍ ഒരു കണ്‍സോര്‍ഷ്യമോ കോണ്‍ഫറന്‍സോ ഉണ്ടാകണമെന്നും എങ്കില്‍ മാത്രമേ ലൈസന്‍സിങ്ങ് പ്രശ്നങ്ങള്‍ പരിഹരിക്കപ്പെടുകയുള്ളൂ എന്നും അവന്‍ ഉറച്ചുവിശ്വസിക്കാനും തുടങ്ങി .ആ ബ്ലോഗ്പോസ്റ്റിലും അത്തരത്തിലുള്ള ചര്‍ച്ചകള്‍ നടന്നുതുടങ്ങിയിരുന്നു . ഇതിനിടയില്‍ ഇന്‍സ്ക്രിപ്റ്റ് കീബോര്‍ഡ് സ്റ്റാന്‍ഡേര്‍ഡൈസേഷന്‍ ശ്രമങ്ങളിലൊക്കെ ജിനേഷ് ഇടപെടുന്നുണ്ട്. അതു സന്തോഷിന്റെ കുറിപ്പില്‍ പറഞ്ഞതിനാല്‍ ഞാന്‍ കൂടുതല്‍ വിശദീകരിക്കുന്നില്ല.

ജിനേഷ് പല സ്വതന്ത്രസോഫ്റ്റ്വെയര്‍ ഡെവലപ്പര്‍മാരെയും പോലെ നാട്ടില്‍ വരുന്ന ദിവസങ്ങള്‍ക്കുമുമ്പായി പോകാന്‍ പറ്റുന്ന കാമ്പസ്സുകളില്‍ എന്തെങ്കിലും ടെക്നോളജി സെഷനുകള്‍ക്കുള്ള സാധ്യത ആരായാനും തുടങ്ങീരുന്നു . 2009 ഡിസംബറില്‍ മലപ്പുറം സ്വതന്ത്രസോഫ്റ്റ്‌വെയര്‍ ലിസ്റ്റില്‍ അയച്ച ഈ മെയില്‍ കാണുക 

\begin{english}
hi all,

I will be in Kerala may be for two weeks in January. Most probably from 15-31\textsuperscript{st}.\\
As usual i like to come and see all may be share what i learned.\\
What you want me to go blabbering on i am leaving to you.\\

\begin{enumerate}
\item I can continue sessions on character recognition(since i haven't
received even a single response from the 30+ people in attendance last
time i am not sure about this).
\item I can go into much more larger spectrum of Language computing and
Natural Language Processing(again introduction and mostly info on
projects what it does, explaining some algorithms, i don't think you
might like the theories and Machine Learning techniques behind it and
if someone is there who is that interested, i don't think he needs a
push from me).
\item May be some info on setting up a computing grid(i am not sure, from
this week onwards i will start setting a grid up and whether i can
talk or not depends on what happens through it)
\item May be some introduction and explanations to version control
systems(mainly on svn, i am not much familiar with git, i am setting
up one git repo soon so can include details from that also then).
\item Or some common boring advocacy topics.
\end{enumerate}

Also let me know your schedule and details. I am definitely
unavailable on 20,21 and 22 since my purpose of visit is a marriage in
family.

Plan your things and let me know so that i can book my tickets also
accordingly(please be quick and talk to people in college soon
otherwise, the flight fare will reach the sky).

Regards
Jinesh K J
\end{english}

എന്റെ വിവാഹത്തിനുശേഷം 2010ല്‍ കുറച്ചുകാലം ഞങ്ങള്‍ തമ്മിലുള്ള സംഭാഷണങ്ങള്‍ കുറവായിരുന്നു . 2010 ജൂലൈയിലാണു് പിന്നെ ചാറ്റിലൂടെ സംസാരിക്കുന്നതു് . ജോലി വേണ്ടപോലെ നീങ്ങുന്നില്ലെന്നും ഒരു ഡള്‍ സ്റ്റേറ്റിലാണെന്നും മടി തോന്നുന്നുവെന്നുമൊക്കെ പറഞ്ഞു . എവിടേക്കെങ്കിലും കുറച്ചുനാള്‍ എന്തെങ്കിലും ഇന്റേന്‍ഷിപ്പോക്കെ പോലെ എന്തെങ്കിലുമായി മാറിനില്‍ക്കണമെന്നും കുറച്ചുനാള്‍ ICT4D യിലോ മറ്റൊ പ്രവര്‍ത്തിക്കണമെന്നും കമ്പനി ജോലി വേണ്ടെന്നും പറഞ്ഞു. ശമ്പളം ആവശ്യമില്ലെന്നും ജീവിതച്ചെലവ് നടന്നുപോയാല്‍മതിയെന്നും യുഐഡിക്കെതിരായ നാഷണല്‍ കാമ്പൈന്‍ പോലെ എന്തിലെങ്കിലും അതിനു സാധ്യത ഉണ്ടാവുമോ എന്നും ആരാഞ്ഞു. കേറിപ്പോരെ എല്ലാം ശരിയാക്കമെന്നു ഞാനും. കുറച്ചുദിവസം കഴിയട്ടെ എന്നിട്ടുപറയാമെന്നു ജിനേഷും . 

ഫോസ്കോം ലിസ്റ്റില്‍  മലയാളമടക്കമുള്ള ഇന്ത്യന്‍ ഭാഷാ ഒസിആര്‍ പ്രൊജക്റ്റുകള്‍ കുത്തകവല്‍ക്കരിക്കപ്പെട്ട കഥ തുറന്നുപറഞ്ഞുകൊണ്ടാണു് പിന്നെ ജിനേഷ് വരുന്നതു്  ആ മെയില്‍ താഴെ \footnote{\url{http://article.gmane.org/gmane.user-groups.foss.india.fosscomm/1991}}

\begin{english}
On Wed, Aug 11, 2010 at 3:31 PM, Venkatesh Hariharan
\url{<venky@knowledgecommons.in>} wrote:
> To sum up, we have been largely successful in "opposing all software patents
> as a whole" at the policy level. However, at the level of the Indian Patent
> Office, there is still some work to do.

I do have some concerns in the above point.

I have seen a quite a strange thing in a project(i should say set of
projects) funded by ministry of communication information technology.
As a student of one of the institutions getting a part of the funds, i have some
first hand information of related discussions.

Ministry of Communication Information Technology(MCIT) was advised by
their legal consultant to do defensive patenting. Then professors and
institutions who were developing
the softwares were convinced by a set of meetings and brainstorming
sessions. MCIT decided to take defensive patents for all the work and
set up a separate fund for covering the expenses(majorly because when
the project was conceived in 2006 and later there were no heads
accounted for patenting expenses). How they got convinced? Patents
from Microsoft, IBM and Google in different subjects where
cited as samples on how industry like to restrict access to knowledge
by a group from CDAC.

As far as i know, all the hungama started from fear of many companies
waiting for fruits of govt. funded research to pick ideas. That actually
happened when a group in IBM invited a professor to share his latest
experiences
and thoughts on some subjects related to language technology. There
actually he talked
about something in the back of his mind and the company took a big
initiative to patent the idea as their own in such a way that, any
thought on the line is illegal :)(this is information from a reliable
source for me, but i dont have any evidence for it). This actually
shocked the professor and he inquired  MCIT about their thoughts on
new ideas or approaches explored as a part of crores of ministry funds
and their legal protection(which is quite unnecessary).

Any information on the projects, funding, plans all should be
available under RTI i believe(Anyway, all these projects are MCIT
projects and not any kind of sensitive information). The
problem is a little bigger in another sense where, a part of govt.
itself is doing an illegal thing(patenting software) which will give a
bigger standing ground for anyone who is for software patenting.

Regards\\
Jinesh K J
\end{english}
---
പിന്നീട് 2010 ഒക്റ്റോബറില്‍ ജിനേഷ് എന്നെ വെല്ലൂര് അഡ്മിറ്റ് ചെയ്ത വിവരം പറയാനാണു് വിളിച്ചതു് . അതുകഴിഞ്ഞുള്ള പരിചയ ചരിത്രം സെബിന്റെ കുറിപ്പിലും മെയിലിങ്ങ് ലിസ്റ്റിലയച്ച എന്റെ കുറിപ്പിലും ഉണ്ടു്. ആ കാലത്തേക്കു കടക്കാന്‍ മനസ്സ് സമ്മതിക്കുന്നില്ല. പക്ഷേ ഒരുകാര്യം പറയാതെ വയ്യ . ജിനേഷിന്റെ ഏറ്റവും അധികം സംഭാവനകള്‍ കണ്ട കാലമായിരുന്നു അതു് . അതുവരെ ചെയ്തതിന്റെ എത്രയോ മടങ്ങ് സംഭാവനകളാണു് ഹോസ്പിറ്റല്‍ ദിനങ്ങളില്‍ ജിനേഷില്‍ നിന്നുണ്ടായിരുന്നതു്. ഇതിനുമപ്പുറം ഫോര്‍ത്ത് എസ്റ്റേറ്റ്ക്രിട്ടിക് എന്ന മെയില്‍ ഗ്രൂപിലെ ചര്‍ച്ചകളിലും ഹോസ്പിറ്റലിനുമുമ്പും ഹോസ്പിറ്റലിലുള്ളപ്പോഴും ജിനേഷ് വളരെ സജീവമായി പങ്കെടുത്തിരുന്നു. ആ കുറിപ്പുകളും ഈ പുസ്തകത്തിലുണ്ടു്. ഈ കുറിപ്പ് ഒരു അവസാനമില്ലാതെ ഒരു തുടര്‍ച്ചയും കാത്തു് ഇവിടെ ഇങ്ങനെ നിര്‍ത്തുകയാണു്. 

