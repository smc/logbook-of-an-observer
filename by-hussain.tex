\secstar{തെളിഞ്ഞ കാലം}
{\vskip 2pt}

\hspace*{6em}\parbox{6cm}{
വിരിഞ്ഞ പൂവുമായ് വസന്തം വന്ന നാള്‍\\
വെറും ചെമ്മണ്‍ പാതയരികില്‍ നിന്നു ഞാന്‍\\
പറഞ്ഞിതള്‍ വാടിക്കൊഴിയും നേരത്തും\\
കനിഞ്ഞു നില്‍ക്കണേ, തെളിഞ്ഞകാലമേ!\\
\hspace*{10em} - വിജയലക്ഷ്മി
}

{\vskip 12pt}

ജിനേഷിനെ അപൂര്‍വ്വമായേ കണ്ടിട്ടുള്ളു. ഹ്രസ്വമായ മെയിലുകളിലും ഫോണ്‍ സംഭാഷണങ്ങളിലും ഓര്‍മ്മകള്‍ ഒതുങ്ങുന്നു. മലയാളം കീബോര്‍ഡിനെക്കുറിച്ചു് സിഡാക്കിന്റെ തെറ്റുകള്‍ ചൂണ്ടിക്കാണിക്കാനുള്ള അടിയന്തിര ഫോണ്‍വിളി മരണം വലയംചെയ്ത നിമിഷങ്ങളിലായിരുന്നു എന്നറിഞ്ഞതു് പിന്നീടാണു്. സമയങ്ങള്‍ തീര്‍ന്നുകൊണ്ടിരിക്കുന്നു എന്ന തിരിച്ചറിവായിരിക്കണം ആ വിളിയില്‍ വല്ലാത്തൊരു പിടച്ചില്‍ ഒളിപ്പിച്ചുവച്ചതു്.

സാങ്കേതികതയ്ക്കകത്തു് വ്യാപരിക്കുമ്പോഴും സമൂഹമായിരുന്നു ജിനേഷിന്റെ ചിന്തകളുടെ പരിസരം. വ്യക്തിജീവിതവും വീക്ഷണങ്ങളും മാറിവരുന്നസങ്കേതങ്ങളില്‍ പുനഃക്രമീകരിക്കേണ്ടതിനെക്കുറിച്ചു് എഴുതുമ്പോഴും സമൂഹം, ലോകം, ചരിത്രം എന്നിവയെ മാറ്റിനിര്‍ത്തിക്കൊണ്ടു് ജിനേഷ് ഒന്നുംതന്നെ ചിന്തിച്ചിട്ടില്ല. സ്വന്തം അഭിപ്രായങ്ങള്‍ കുറിച്ചിടുമ്പോഴും സ്നേഹിതരുടെ കമന്റുകളില്‍നിന്നും കിട്ടുന്ന വെളിച്ചങ്ങള്‍ക്കായി കാത്തിരുന്നു. അവയില്‍ സ്വന്തം ആശയങ്ങള്‍ നവീകരിക്കാനുള്ള ജ്വലനങ്ങള്‍ കണ്ടെത്തുമെന്നു് എപ്പോഴും ആശിച്ചു. സ്വതന്ത്ര സോഫ്റ്റ്‌വെയറിനെക്കുറിച്ചു് സംസാരിക്കുമ്പോഴാണു്, "മനസ്സിലാക്കിയതു് രണ്ടു പേരോടുകൂടി പറയുക" എന്നു് അദ്ദേഹം ഉദ്ബോധിപ്പിച്ചതു്. അറിവിന്റെ അനന്തശ്രേണിയില്‍ സാഹോദര്യത്തെ സ്ഥാനപ്പെടുത്തുകയായിരുന്നു ജിനേഷ്. സ്വാശ്രയകോളേജിലെ പഠനത്തെക്കുറിച്ചുള്ള ചര്‍ച്ചയില്‍ "സഹജീവികളെ മാനിക്കാനോ മനസ്സിലാക്കാനോ കഴിയാത്ത പൗരന്മാരായി" വളര്‍ത്തുന്ന വിദ്യാഭ്യാസ സമ്പ്രദായത്തെ കുറിച്ചോര്‍ത്തു് അദ്ദേഹം വല്ലാതെ വ്യാകുലപ്പെട്ടു.

കാറോട്ട മത്സരത്തിലെ മുന്നേറ്റങ്ങളും, നിമിഷനേരംകൊണ്ടു് മാറിമറിയുന്ന ഗതിവിഗതികളും അപ്രവചനീയതയും ആവേശപൂര്‍വ്വം പിന്തുടര്‍ന്ന ജിനേഷിനു് സമകാലീന രാഷ്ട്രീയത്തിലെ കുതിപ്പിനേയും കിതപ്പിനേയും വേറിട്ടൊരു വിതാനത്തില്‍ നിരീക്ഷിക്കാന്‍ പ്രാപ്തിനല്കി. മറ്റുപലര്‍ക്കും അന്യമായ ചില രാഷ്ട്രീയ ഉള്‍ക്കാഴ്ചകള്‍ ജിനേഷിനു് കൈവന്നതു് സ്പോര്‍ട്സില്‍ പ്രത്യക്ഷവല്‍ക്കരിക്കപ്പെടുന്ന ഗെയ്‌മുകളെ കൗതുകപൂര്‍വ്വം നോക്കിക്കണ്ടതുകൊണ്ടാണു്. ഫോര്‍മുല വണ്ണില്‍നിന്നും ഉരുത്തിരിച്ചെടുത്ത മറ്റൊരു ഫോര്‍മുല സാമൂഹ്യചിന്തകളില്‍ വിന്യസിപ്പിച്ചപ്പോഴുണ്ടായ ദര്‍ശനദീപ്തികള്‍ വളര്‍ന്നുവികസിക്കാന്‍ പക്ഷെ കാലം കൂട്ടുനിന്നില്ല.

'എന്തുകൊണ്ടു് പത്രങ്ങള്‍ സോഷ്യല്‍ മീഡിയകളെ ഭയക്കുന്നു' എന്ന ലേഖനം പുതുലോകത്തിന്റെ പ്രവേഗവും സാദ്ധ്യതകളും അസാധാരണമായ വ്യക്തതയോടെ അനാച്ഛാദനം ചെയ്യുന്നു. പരമ്പരാഗത മാദ്ധ്യമങ്ങളുടെ ശേഷി എത്രയ്ക്കുണ്ടെങ്കിലും, അതിനെ ദുര്‍ബ്ബലപ്പെടുത്തുകയും പേടിപ്പിക്കുകയും ചെയ്യുന്ന വെബ്ബ്സങ്കേതങ്ങളുടെ പ്രഹരശേഷിയെക്കുറിച്ചു് മറ്റാരേയുംക്കാള്‍ ജിനേഷ് ബോധവാനായിരുന്നു. ഒറ്റപ്പെട്ടവന്റെയും പാര്‍ശ്വവല്‍ക്കരിക്കപ്പെട്ടവന്റേയും സൗഹൃദകൂട്ടായ്മയെ ദേശരാഷ്ട്രങ്ങളുടെ അതിര്‍വരമ്പുകളുല്ലംഘിക്കുന്ന ആഘോഷമായി ജിനേഷ് അതില്‍ വരച്ചിടുന്നു. അപരന്റെ ശബ്ദം സംഗീതമായി മാറുന്ന കാലത്തെക്കുറിച്ചുതന്നെയാണു് ജിനേഷ് കിനാവു കണ്ടതു്.

ശാസ്ത്രസാങ്കേതികതയുടെ വിദ്യാര്‍ത്ഥിയായിരുന്ന ജിനേഷിനു് ശാസ്ത്രത്തിന്റെ അസാംഗത്യങ്ങളെക്കുറിച്ചു് നന്നായറിയാമായിരുന്നു. വിഗ്രഹാരാധനയുടെ ശാസ്ത്രീയത തെളിയിക്കാന്‍ ശ്രമിക്കുന്നതിനുപകരം, അവ യാഥാര്‍ത്ഥ്യമാക്കിയ വ്യക്തി-സമൂഹ്യസാഹചര്യങ്ങളുടെ വസ്തുനിഷ്ഠതയിലാണ് അന്വേഷണം തുടരേണ്ടതെന്ന നിലപാടാണുള്ളതു്. കൊച്ചുത്രേസ്യയുടെ ആശങ്കകളുടെ കാരണക്കാരില്‍ താനുമുണ്ടെന്നു തുറന്നുസമ്മതിക്കുന്ന ജിനേഷ്, ഇതൊരു സമൂഹരോഗമാണെങ്കില്‍ അതു് കണ്ടെത്താനും സ്വയംതിരുത്താനുമുള്ള യത്നത്തില്‍ പങ്കുചേരാന്‍ ബൂലോകത്തിലെ എല്ലാ കൂട്ടുകാരേയും ക്ഷണിച്ചു. സഹിഷ്ണുതയെക്കുറിച്ചുള്ള വളരെ ചെറിയൊരു പോസ്റ്റില്‍ ഭൂരിപക്ഷ /ന്യൂനപക്ഷ വര്‍ഗ്ഗീയതയെ സഫലമായി പ്രതിരോധിക്കാനുള്ള ആയുധം പരസ്പരബഹുമാനമാണു്, അല്ലാതെ സഹനമല്ല എന്നും ജിനേഷ് വെളിവാക്കുന്നു.  "നീ അവനെ സഹിച്ചുവേണം ജീവിക്കാന്‍" എന്ന അപകടത്തെ "നീ അവനേയും ബഹുമാനിക്കുക" എന്ന നിലപാടുകൊണ്ടാണു് നേരിടേണ്ടതെന്നു് ജിനേഷ് വിശ്വസിച്ചു.

നിരീക്ഷണങ്ങള്‍ ഗൗരവമുള്ളതാകുമ്പോഴും എഴുത്തില്‍ ജിനേഷ് എന്നും നര്‍മ്മം വിതറി. "അഞ്ചരമീറ്റര്‍ തുണി അഴിഞ്ഞുവീഴാതെ ധരിച്ചു് സ്വതന്ത്രമായി നടക്കുക എന്നതു് ഒരു കഴിവു തന്നെയാണു് !" എന്നു് സാരിയെക്കുറിച്ചു് അദ്ദേഹം വിനീതമായി അഭിപ്രായപ്പെട്ടു.

ലോഗ്ബുക്കിന്റെ സമാഹരണവും പ്രസിദ്ധീകരണവും കൂട്ടുകാര്‍ ജിനേഷിനു നല്‍കുന്ന സമുചിതമായ സ്മാരകമാണു്. പോസ്റ്റുകളിലൂടേയും കമന്റുകളിലൂടേയും വികസിക്കുന്ന ലേഖനങ്ങള്‍, സുഹൃത്തുക്കളിലൂടെ സ്വയംതിരുത്താനുള്ള വെബ്ബ് രണ്ടിന്റെ സാദ്ധ്യതകളെയാണു് അനാവരണം ചെയ്യുന്നതു്. കൂട്ടായ്മയിലൂടെ രൂപപ്പെടുന്ന സത്യാന്വേഷണത്തിന്റെ മറ്റൊരു മാതൃക മലയാളത്തില്‍ അച്ചടിക്കപ്പെടുകയാണു്. യൂണികോഡില്‍ ടൈപ്‌സെറ്റ് ചെയ്തു് അച്ചടിക്കുന്ന ആദ്യത്തെ പുസ്തകം ജിനേഷിന്റെതായതു് ഭാഷാസാങ്കേതികതയിലെ അവിസ്മരണീയമായ മുഹൂര്‍ത്തമായി മാറുന്നു. വരുംകാല മലയാളപുസ്തകപ്രസാധനത്തില്‍ 'ടെക്കി'ന്റേയും സ്വതന്ത്ര സോഫ്റ്റ്‌വെയറിന്റേയും ഇടപെടലിന്റെ വിളംബരം കൂടിയാണിതു്. 

ഭൂമിയുടെ ഹൃത്തടത്തിലേക്കു് മടങ്ങിപ്പോയ നീര്‍ച്ചാല്‍ വേരുകളെ എന്നും നനച്ചുകൊണ്ടിരിക്കും. ശരവേഗത്തില്‍ പാഞ്ഞുപോകുന്ന ഫെറാരിയേയും റെഡ്ബുള്ളിനേയുമൊക്കെ ഇപ്പോഴും ഗ്യാലറിയിലിരുന്നു് ജിനേഷ് കാണുന്നുണ്ടാകണം. അതിന്റെ ആവേശവും ആവേഗവുമൊക്കെ കൂട്ടുകാര്‍ക്കായി ഇപ്പോഴും കുറിച്ചിടുന്നുണ്ടാകണം. എല്ലാ വേഗങ്ങള്‍ക്കുംമുമ്പേ സഞ്ചരിച്ച കൊച്ചുകൂട്ടുകാരനായിരുന്നു അവന്‍. മരണം ചുറ്റും പൊതിഞ്ഞ നാളുകളിലും കൂട്ടുകാര്‍ക്കായി അവന്‍ അറിവിന്റെ വളപ്പൊട്ടുകള്‍ കരുതിവച്ചു. അവയില്‍ കോമിക്സുണ്ടായിരുന്നു. കാറോട്ടങ്ങളും ക്രിക്കറ്റുമുണ്ടായിരുന്നു. മലയാളമുണ്ടായിരുന്നു. താന്‍ ജീവിച്ച സ്ഥലകാലങ്ങളുടെ സ്പന്ദനങ്ങള്‍ ഒന്നൊഴിയാതെ സ്പര്‍ശിച്ചറിയണമെന്ന മോഹമുണ്ടായിരുന്നു. വേഗത്തിലവസാനിച്ചുപോയ ഒരു കാലത്തിലിരുന്നു് ജീവിച്ചുപോകുന്നതിനെ മറ്റുള്ളവര്‍ക്കായി എങ്ങനെ അടയാളപ്പെടുത്താമെന്നാണു് ഈ വരികള്‍ നമ്മെ പഠിപ്പിക്കുന്നതു്. അകലെനിന്നു് അദൃശ്യമായി ഒഴുകിവന്നു് പെട്ടെന്നു് നിലച്ചുപോയ ഒരു ഒറ്റക്കുയിലിന്റെ പാട്ടുപോലെയാണതു്. എല്ലാ ക്ലേശങ്ങളും വേപഥുകളും നിമിഷനേരത്തേക്കു് മാഞ്ഞില്ലാതാകുന്നതുപോലെ. നമ്മുടെ സംത്രാസങ്ങളിലേക്കു് പിന്നീടു് നാം മടങ്ങിയെത്തുമ്പോള്‍ മനസ്സിലാക്കുന്നു, ജിനേഷുമൊത്തുള്ള ഈ ഹ്രസ്വമായ കൂടിച്ചേരലില്‍ ഒരു ചെറിയ തൂവല്‍ നമുക്കു് ലഭിച്ചെന്നു്. എപ്പോഴും സ്പര്‍ശിക്കാനായി ലാപ്‌ടോപ്പിനോടൊപ്പം നാമതു് മടിയില്‍ വച്ചിട്ടുണ്ടു്.

\begin{flushright}ഹുസൈന്‍ കെ.എച്ച്. \end{flushright}
\newpage
