\secstar{അഭിനവ് ബിന്ദ്രയും ഒളിമ്പിക് സ്വര്‍ണ്ണവും ചില ചിന്തകളും}
\vskip 2pt
\enlargethispage*{2\baselineskip}

താന്‍ എന്താണു ചെയ്യുന്നതെന്നും എന്തിനു വേണ്ടിയാണു് ബെയ്ജിങ്ങിലെത്തിയതെന്നും അഭിനവിനു നന്നായറിയാമായിരുന്നെന്നു തോന്നുന്നു. "It is the thrill of my life!" എന്നു് നിര്‍വ്വികാരനായി പറയുന്ന അഭിനവ് ബിന്ദ്രയില്‍ താന്‍ തീരുമാനിച്ചുറച്ചതിലപ്പുറമൊന്നും നേടിയില്ലെന്ന അഹങ്കാരത്തെക്കാളും, എങ്ങനെ ആഘോഷിക്കണമെന്നു തീരുമാനിക്കാനാവാത്ത ഒരു പ്രതിസന്ധിയാണു് കണ്ടതു്. ഡെറാഡൂണിലെ സ്കൂളിലും ഷൂട്ടിങ്ങു് റേഞ്ചിലും ബിസനസ്സിലും, ചിട്ടയും അച്ചടക്കവും ശീലിച്ച അഭിനവിനു് ആഘോഷങ്ങളിലും അച്ചടക്കം ഒഴിവാക്കാനാവാത്തതാണെന്നു കരുതാം.

മൂന്നൂറുകോടിയിലേറെ വിറ്റുവരവുള്ള സ്ഥാപനങ്ങള്‍ കുടുംബത്തിലുള്ള അഭിനവ്, ഇന്ത്യന്‍ വ്യവസായലോകത്തിന്റെ കായികപ്രതിനിധിയാണെന്നു തോന്നുന്നു. സ്വന്തമായി ഷൂട്ടിങ് റേഞ്ച് നിര്‍മ്മിച്ചും, സ്വന്തം പണം മുടക്കി ദക്ഷിണാഫ്രിക്കയില്‍ പോയി പരിശീലിച്ചും സ്വര്‍ണ്ണം നേടിയ അഭിനവ് തിളങ്ങുന്ന ഇന്ത്യയുടെ പ്രതിനിധിയാണു്.

ചിലപ്പോള്‍ ഒളിമ്പിക് വ്യക്തിഗത സ്വര്‍ണ്ണം നേടുന്ന ആദ്യത്തെ സി.ഇ.ഓ. യും ഇദ്ദേഹമായിരിക്കും. പ്രസിദ്ധ ജര്‍മ്മന്‍ ആയുധനിര്‍മ്മാതാക്കളായ വാള്‍ട്ടറിന്റെ (ജെയിംസ് ബോണ്ട് സിനിമകളിലൂടെയാണു് എനിക്കു് ഇവരെ പരിചയം!) ഇന്ത്യയിലെ ഒരേയൊരേജന്റായ അദ്ദേഹത്തിന്റെ കമ്പനി, 2010ഓടെ പ്രതീക്ഷിക്കുന്നത് 100 കോടിയുടെ വിറ്റുവരവാണത്രേ. കൂടാതെ കമ്പ്യൂട്ടര്‍ ഗെയിം ഉപകരണങ്ങളും മറ്റും വില്‍ക്കുന്ന കമ്പനിക്കു പറ്റിയ ഏറ്റവും വലിയ ബ്രാന്‍ഡ് അംബാസിഡറായിരിക്കും അഭിനവ്. എം.ബി.എ ബിരുദധാരിയായ ഈ ചെറുപ്പക്കാരനു് ബിസിനസ്സിന്റെ സാധ്യതകളെ പതിന്മടങ്ങാക്കാന്‍ സ്വന്തം ബ്രാന്‍ഡ് എങ്ങനെ ഉപയോഗിക്കാം എന്ന കാര്യത്തില്‍ ഒരു പ്രാക്റ്റിക്കല്‍ കാത്തിരിക്കുകയാണെന്നു തോന്നുന്നു. കായികരംഗത്തെ നേട്ടത്തിന്റെ പിന്‍ബലത്തില്‍ ബിസിനസ്സിലിറങ്ങിയവര്‍ പലരുമുണ്ടു്. ഒരു പക്ഷെ സ്വന്തം ബിസിനസ്സ് അഭിവൃദ്ധിപ്പെടുത്താന്‍ പാകത്തില്‍ ഒരു ഒളിമ്പിക് മെഡല്‍ ലഭിച്ച ആദ്യ കായികതാരവും ഇദ്ദേഹമായിരിക്കും.

ഇന്ത്യന്‍ കായികലോകത്തോടൊപ്പം ഇന്ത്യന്‍ ബിസിനസ്സ് ലോകത്തിനും ഒരുപാടു പാഠങ്ങള്‍ നല്‍കാന്‍ അഭിനവിനാകട്ടെ എന്നാശംസിച്ചുകൊണ്ടു് നിര്‍ത്തുന്നു.

\subsection*{പ്രതികരണങ്ങള്‍}
\begin{enumerate}
\item{വേണു | venu : }
എന്തായാലും വിജയത്തിന്റെ ഉത്തുംഗശൃംഗങ്ങളിലും കാത്തുസൂക്ഷിച്ച ഭാരതീയപാരമ്പര്യം. പക്വത എന്തെന്നും നിസ്സംഗത എന്തെന്നും അഭിനയിച്ചു കാണിച്ചതല്ല, ജീവിച്ചു കാണിക്കുകയായിരുന്നു. അഭിനവ് ബിന്ദ്രയെക്കുറിച്ചു് കൂടുതല്‍ അറിയിക്കാന്‍ കഴിഞ്ഞ ഈ പോസ്റ്റിനു നന്ദി.:)

\end{enumerate}

\newpage
