\secstar{ആശയത്തോളം  വളരുന്ന വ്യക്തി}

ഒരു ആശയത്തിന്റെ പൂര്‍ത്തീകരണത്തിനായി പ്രവര്‍ത്തിക്കുന്ന സമൂഹങ്ങളില്‍ നിന്ന് ചില വ്യക്തിത്വങ്ങള്‍ ആ ആശയത്തോളം  തന്നെ
വളരുന്ന സംഭവങ്ങള്‍ ചരിത്രത്തില്‍ പലയിടങ്ങളിലും  രേഖപ്പെടുത്തിയിരിക്കുന്നത് നമുക്ക് കാണാന്‍ സാധിക്കും. പ്രസ്തുത ആശയത്തിന്റെയും  
അതിന്റെ നിലനില്‍പിനായി പ്രവര്‍ത്തിക്കുന്ന സമൂഹത്തിന്റെയും  ചരിത്രത്തില്‍ ഇത്തരം  വ്യക്തികള്‍ എന്നെന്നേക്കുമായി വരച്ചു ചേര്‍ക്കപ്പെടും. 
അത്തരം  വ്യക്തികളുടെ ജീവിതം  അവസാനിച്ചാലും  ആ വ്യക്തിത്വങ്ങള്‍ അവര്‍ ഭാഗമായിരുന്ന  സമൂഹത്തില്‍ അവശേഷിക്കുന്നു. 


മലയാളത്തില്‍ പ്രസിദ്ധീകരിക്കപ്പെ മറ്റേതൊരു പുസ്തകത്തില്‍ നിന്നും  വ്യത്യസ്തമാണ് 'ഒരു നിരീക്ഷകന്റെ കുറിപ്പുകള്‍ '. 
സാങ്കേതികപരമായും  ആശയപരമായും  ഇതിന്റെ സാക്ഷാത്കാരത്തിനെ പിന്തുണച്ചവര്‍ അനവധിയാണ്. പൂര്‍ണ്ണമായും  സ്വതന്ത്രമായ 
സോഫ്റ്റ്‌‌വെയര്‍ ഉപകരണങ്ങള്‍ ഉപയോഗിച്ച് സീടെക്ക് (XeTeX) എന്ന ടെക്ക് (TeX) ടൈപ്പ് സെറ്റിങ്ങ് എഞ്ചിനില്‍ പൂര്‍ണ്ണമായി യുണീക്കോഡില്‍ ടൈപ്പ് സെറ്റ് ചെയ്ത
ഈ പുസ്തകം, ലോകത്തിന്റെ വിവിധ ഭാഗങ്ങളിലായി ചിതറിക്കിടക്കുന്ന ഒരു കൂട്ടം  ആളുകളുടെ ശ്രമദാനത്തിന്റെ ഫലമാണ്.   

ഈ പുസ്തകത്തിലെ നിരീക്ഷണങ്ങള്‍ പൊതുസമൂഹത്തിനു മുന്നിലെത്തിക്കുന്നതിനോടൊപ്പം , ജിനേഷ് തുടക്കമിട്ട  വലിയ പല പദ്ധതികളും  
തുടര്‍ന്ന് മുന്നോട്ടു കൊണ്ടു പോകുക എന്ന ഒരു വലിയ ദൗത്യം  നമ്മുടെ മുന്‍പിലുണ്ട്. പുതിയ വിപ്ലവവഴികളിലൂടെ നമുക്ക് സഞ്ചരിക്കേണ്ടതായുണ്ട്. 
ഒട്ടേറെ ദൂരം  നാം  പിന്നിട്ടു കഴിഞ്ഞിരിക്കുന്നു , ഏറെ ദൂരം  ഇനിയും  താണ്ടാനുണ്ട്.  പിന്നിട്ട വഴിയില്‍ നമ്മെ കൈ പിടിച്ചു നടത്തിയ ആശയത്തോടൊപ്പം  
ജിനേഷെന്ന വ്യക്തിയും  വളര്‍ന്ന് നമ്മുടെ വഴിവിളക്കായിരിക്കുന്നു.

വരും  താളുകളില്‍ നിന്ന് അനുവാചകനു ലഭ്യമാകുന്ന ചിന്താശകലങ്ങളുണര്‍ത്തുന്ന വിസ്ഫോടനങ്ങള്‍ മലയാള സാംസ്കാരികമേഖലയില്‍ പുതിയ ഓളങ്ങള്‍ സൃഷ്ടിക്കട്ടെ. 
ഭാഷാകമ്പ്യൂട്ടിങ്ങ് രംഗത്ത് ജിനേഷ് കൈപിടിച്ചു നടത്തിയ സങ്കേതങ്ങള്‍ പുതിയ ഉയരങ്ങള്‍ കണ്ടെത്തട്ടെ.  ആ നല്ല നാളെയെക്കുറിച്ചുള്ള നമ്മുടെ പ്രതീക്ഷകളില്‍ ,
അതിനായുള്ള ചെറു ചുവടുവെയ്പുകളില്‍ , ആ ആശയസാക്ഷാത്കാരത്തില്‍ ജിനേഷ് ജീവിക്കുന്നു. 

\hspace*{2em}ഋഷികേശ് കെ ബി,  സ്വതന്ത്രമലയാളം കമ്പ്യൂട്ടിങ്ങ്. 
\newpage



