\secstar{പിണറായി, വിദ്യാഭ്യാസം, സമൂഹം, മുന്‍വിധികള്‍, സ്വാതന്ത്ര്യം}
\vskip 2pt


സെബിന്‍ ഇവിടെ\footnote{\url{http://absolutevoid.blogspot.com/2009/11/blog-post.html}} എഴുതിയ നീണ്ടലേഖനത്തിനു മറുപടിയായി എഴുതിത്തുടങ്ങിയതാണു്. ഞാന്‍ എഴുതിത്തീര്‍ന്നപ്പോഴെയ്ക്കും അവിടെ ചര്‍ച്ച സിപിഎമ്മിന്റെ നയങ്ങളെക്കുറിച്ചും തട്ടിപ്പുകളെപ്പറ്റിയുമായതുകൊണ്ടു് ഇനി ഇതവിടെ കൊണ്ടിട്ടാല്‍ ഞാന്‍ വിഷയം മാറ്റാന്‍ നോക്കുന്ന കമ്യൂണിസ്റ്റുകാരനായാലോ എന്നു കരുതി ഇവിടെയിടുന്നു.

പിണറായിയുടെ തത്വങ്ങള്‍ക്കും ആദര്‍ശങ്ങള്‍ക്കും വിരുദ്ധമായി വിവേകും വീണയും സ്വകാര്യസ്വാശ്രയകോളേജില്‍ പഠിക്കുന്നതിനെ പഴിപറയുന്ന കാര്യത്തില്‍ എന്റെ ചില ചിന്തകളാണു് ഇവിടെ കുറിക്കുന്നത്. വിവേക് കിരണ്‍ സ്വകാര്യ സ്വാശ്രയകോളേജായ SCMSല്‍ MBAയ്ക്കു ചേര്‍ന്നതാണു് എല്ലാവരും ചോദ്യംചെയ്യുന്നതു്. വിവേക് അവിടെ ചേര്‍ന്നതു് കോഴകൊടുത്താണെങ്കില്‍ തീര്‍ച്ചയായും എതിര്‍ക്കപ്പെടേണ്ടതാണു്. (അതിനി വിവേകല്ല, ദേവേന്ദ്രനായാലും എതിര്‍ക്കേണ്ടതാണു്.) എന്നാല്‍ പ്രശ്നം, സ്വകാര്യ സ്വാശ്രയകോളേജുകളുടെ നയങ്ങളെ എതിര്‍ക്കുന്ന ആദര്‍ശങ്ങളില്‍ വിശ്വസിക്കുകയും, പലപ്പോഴും ഈ സ്ഥാപനങ്ങള്‍ക്കെതിരെ നിശിതവിമര്‍ശനം നടത്തുകയും, അവ അടച്ചിട്ടും പഠിപ്പുമുടക്കിയും സമരം നടത്തുകയും ചെയ്യുന്ന വിദ്യാര്‍ത്ഥി-യുവജന പ്രസ്ഥാനങ്ങളുടെ മാതൃസംഘടനയുടെ സംസ്ഥാന സെക്രട്ടറിയായിരിക്കുകയും ചെയ്യുന്ന വ്യക്തിയുടെ മകനായതുകൊണ്ടു്, സ്വകാര്യ സ്വാശ്രയകോളേജില്‍ ചേര്‍ന്നുപഠിക്കാന്‍ പാടില്ലെന്നതാണു്. ഈയൊരു വാദത്തില്‍ വന്‍പിശകുണ്ടെന്നാണു് എന്റെ അഭിപ്രായം.

പിണറായിയുടെ ഭാഗത്തുനിന്നു നോക്കിയാല്‍, കാര്യമെല്ലാം ശരിയാണു്; വിവേക് ഒരിക്കലും സ്വകാര്യ സ്വാശ്രയകോളേജില്‍ ചേര്‍ന്നു പഠിക്കരുതു്; (അന്നത്തെ സാഹചര്യങ്ങളില്‍. ഇപ്പോ മൊത്തം കോളേജുകളോടു് എതിര്‍പ്പൊന്നുമില്ലെന്നു തോന്നുന്നു.) പാര്‍ട്ടിക്കും തനിക്കും മാനക്കേടുണ്ടാക്കിവയ്ക്കുന്ന ഒന്നാന്തരം സംഭവം. സ്വന്തം ആദര്‍ശങ്ങള്‍ മകനെപ്പോലും പറഞ്ഞു പഠിപ്പിക്കാനാവാത്ത ദുര്‍ബലനാവാകുന്ന സാഹചര്യം. പക്ഷേ, വിവേകിന്റെ സ്ഥാനത്തുനിന്നു നോക്കിയാല്‍, സ്വന്തം വിദ്യാഭ്യാസകാര്യങ്ങളില്‍പോലും തീരുമാനമെടുക്കാന്‍ അച്ഛന്റെ ആദര്‍ശങ്ങള്‍ തടസ്സമാവുന്ന സ്ഥിതിയാണു്. താന്‍ എന്തു ചെയ്യണമെന്നും ആരാവണമെന്നും മൂന്നാമതൊരാള്‍ നിശ്ചയിക്കുന്ന അവസ്ഥ. അവിടെ തന്റെ സ്വകാര്യസ്വാതന്ത്ര്യം വിവേക് ഉപയോഗിച്ചിരിക്കാനുള്ള സാധ്യതയാണു് സെബിന്‍ അവിടെ സൂചിപ്പിച്ചതെന്നാണു് എനിക്കു മനസ്സിലായതു്. (സത്യം പറഞ്ഞാല്‍ എനിക്കു വിവേകിനേയോ പിണറായിയേയോ യാതൊരു പരിചയവുമില്ല. കൃത്യമായി എന്തു സംഭവിച്ചു എന്നു് വിവേകിനോടുതന്നെ ചോദിക്കേണ്ടി വരും.)

"സ്വന്തം മകനെ/മകളെ സ്വകാര്യ സ്വാശ്രയകോളേജില്‍ ചേര്‍ക്കുകവഴി പിണറായി തന്റെ അനുയായികളെ വഞ്ചിക്കുകയായിരുന്നു" എന്ന വിലയിരുത്തലിലെ പ്രധാനപ്രശ്നം, ഒരുപാടു മുന്‍വിധികളാണു്. ഒരാളുടെ വിദ്യാഭ്യാസകാര്യങ്ങളിലെ തീരുമാനങ്ങളില്‍ പ്രധാന സ്വാധീനം അച്ഛന്റെയായിരിക്കും എന്നതുമുതല്‍, വിവേകിനു് സ്വന്തമായി പിണറായിയെ എതിര്‍ത്തു് SCMSല്‍ പഠിക്കാന്‍ സാധ്യമല്ല എന്നതുവരെയെത്തുന്നു അതു്. ഇതു വിവേക് കിരണോ വീണയോ മാത്രം അനുഭവിക്കുന്ന പ്രശ്നമല്ല. കേരളത്തിലെ ഓരോ വിദ്യാര്‍ത്ഥികളും അനുഭവിക്കുന്ന സാമൂഹ്യസമ്മര്‍ദ്ദമാണു്. മെഡിസിനോ എഞ്ചിനീയറിങ്ങിനോ അഡ്മിഷന്‍ നേടാന്‍ താത്പര്യമില്ലാത്തവരെ താറടിക്കുന്നതില്‍ തുടങ്ങി, ഉന്നതവിദ്യാഭ്യാസത്തിനു ശ്രമിക്കുന്ന പിന്നോക്കക്കാരനെ സംശയദൃഷ്ടിയോടെ വീക്ഷിക്കുന്നതില്‍ എത്തിനില്‍ക്കുന്നു ഇതു്. പിണറായിയുടെയും വിവേകിന്റെയും പേരിനോടു് ചേര്‍ത്തുവയ്ക്കുന്നതുകൊണ്ടു് ഈ വിഷയത്തിനു് അതര്‍ഹിക്കുന്ന പ്രാധാന്യം ലഭിക്കാതെ പോകുന്നുണ്ടെന്നാണു് എന്റെ വിശ്വാസം. സെബിന്റെ ബ്ലോഗില്‍ ഈ വിഷയത്തില്‍ അഭിപ്രായങ്ങള്‍ രേഖപ്പെടുത്തിയവരെല്ലാവരും, പിണറായിയുടെയും പാര്‍ട്ടിയുടെയും ഭാഗത്തുനിന്നു മാത്രമെ പ്രശ്നത്തെ സമീപിച്ചുള്ളു എന്നാണെനിക്കു തോന്നുന്നത്. വിഷയം കമ്യൂണിസ്റ്റുകാര്‍ക്കെതിരെയുള്ള വിമര്‍ശനമായതുകൊണ്ടായിരിക്കാം.

പ്രായമായ ഒരു കുട്ടി സ്വന്തം വിദ്യാഭ്യാസകാര്യത്തില്‍ എടുക്കുന്ന തീരുമാനത്തിനെ, അതു് രക്ഷിതാവിന്റെയോ സമൂഹത്തിന്റേയോ പ്രതീക്ഷകള്‍ക്കൊത്തല്ല എന്ന കാരണംകൊണ്ടു് എതിര്‍ക്കരുതു്. അങ്ങനെ എതിര്‍ക്കുകയും, "സ്വന്തം മക്കളെപ്പോലും മര്യാദയ്ക്കു നിര്‍ത്താനറിയാത്തവര്‍" എന്നു് രക്ഷിതാക്കളെ പഴിപറയുകയും ചെയ്യുന്നവര്‍ ഒരാളുടെ വ്യക്തിസ്വാതന്ത്ര്യത്തിലും സ്വകാര്യതയിലുമാണു് തലയിടുന്നതെന്നു ചിന്തിക്കുകപോലും ചെയ്യാതെയാണു് വിധിയെഴുതുന്നതു്. സ്വന്തമായൊരഭിപ്രായവും ലക്ഷ്യവും ഉണ്ടായിപ്പോയതിനാല്‍ പഴികേള്‍ക്കേണ്ടിവരുന്നവരാണു് ആ കുട്ടികള്‍. സ്വന്തം കരിയറും ജീവിതവും എങ്ങനെ വേണമെന്നു് തീരുമാനിക്കാന്‍ ഓരോ വിദ്യാര്‍ത്ഥിയേയും പ്രാപ്തമാക്കുകയാണു വേണ്ടതെന്നു പറയുന്ന സമൂഹംതന്നെയാണു് ഇത്തരത്തിലും വിലയിരുത്തന്നതു്. ഇതു് പലപ്പോഴും സ്വന്തം ആഗ്രഹങ്ങളുടെ മുകളില്‍ മാതാപിതാക്കളുടെയും സമൂഹത്തിന്റെയും പ്രതീക്ഷകളെ പ്രതിഷ്ഠിക്കാനാണു് കുട്ടികളെ പ്രേരിപ്പിക്കുന്നതു്. തങ്ങളെ തുറിച്ചുനോക്കുകയും മുറുമുറുക്കയും ചെയ്യുന്നവരെ അവഗണിക്കാന്‍ കഴിയാത്ത ഓരോ വിദ്യാര്‍ത്ഥിയും ഈ സമ്മര്‍ദ്ദം അനുഭവിച്ചിട്ടുണ്ടാവണം. സ്വന്തം കുട്ടിയെ അവനിഷ്ടമുള്ള വഴി തിരഞ്ഞെടുക്കാനനുവദിച്ച മാതാപിതാക്കളും ഏതാണ്ടിതേതോതില്‍ സമ്മര്‍ദ്ദം അനുഭവിക്കുന്നുണ്ടാവണം.

ഇത്രയൊക്കെ ഒരാളുടെ സ്വകാര്യതയില്‍ ഇടപെടുന്ന സമൂഹം തന്നെ, ഇരട്ടത്താപ്പുകാണിക്കുന്നതും സ്വാഭാവികം. ഇന്ത്യന്‍ വിദ്യാര്‍ത്ഥികള്‍ "ഔട്ട് ഓഫ് ദ ബോക്സ്" ചിന്തിക്കാത്തവരാണെന്നും, "ക്നോളെജ് ജെനറേഷന്‍ പ്രോസസ്സി"ല്‍ ഇടപെടാന്‍ താത്പര്യമില്ലാത്തവരാണെന്നും വിധിയെഴുതും. വിദ്യാഭ്യാസത്തില്‍ പാശ്ചാത്യരീതികളും പ്രക്രിയകളും പിന്തുടര്‍ന്നാല്‍ ഈ പ്രശ്നം പരിഹരിക്കപ്പെടുമെന്നു കരുതുന്ന ബുദ്ധിജീവിസമൂഹം, സമാനസാഹചര്യങ്ങളില്‍ വിദ്യാര്‍ത്ഥികള്‍ നേരിടുന്ന വെല്ലുവിളികള്‍ രണ്ടിടത്തും വേറെയാണെന്നു സൗകര്യപൂര്‍വ്വം മറക്കുന്നു. മറക്കാത്തവര്‍ പരീക്ഷകള്‍ കുറച്ചും സ്വതന്ത്രമായി ചിന്തിക്കാനുള്ള സാഹചര്യം ഒരുക്കിയും സമ്മര്‍ദ്ദം കുറയ്ക്കാം എന്ന ചിന്തയാണു് വച്ചുപുലര്‍ത്തുന്നത്. ഇത്തരം പരിഷ്കാരങ്ങളെ നടപ്പാക്കുമ്പോള്‍ ഉണ്ടാവുന്ന ഫലങ്ങള്‍ പലപ്പോഴും വിപരീതമാണെന്നു മാത്രം.

പാശ്ചാത്യവിദ്യാര്‍ത്ഥികള്‍ക്കു് സ്കൂളില്‍ ലഭിക്കുന്ന "റൈറ്റിങ് ഓണ്‍ ഫ്രീ സ്ലേറ്റ് എക്സ്പീരിയന്‍സ്" മധ്യവര്‍ഗ്ഗ കുടുംബങ്ങളിലെ വിദ്യാര്‍ത്ഥികള്‍ക്കു് സ്വപ്നംപോലും കാണാന്‍ കഴിയാത്തതാണു്. പാശ്ചാത്യരീതിയില്‍ കുട്ടികളെ പഠിപ്പിച്ചുകൊണ്ടുവരുന്ന സമൂഹം, സ്വതന്ത്രമായി ചിന്തിക്കാനുള്ള കഴിവുതന്നെ ഓരോ വിദ്യാര്‍ത്ഥിയില്‍നിന്നും പറിച്ചുമാറ്റുന്നു. എന്നിട്ടും നന്നാവാത്തവരുടെ മുകളിലാണു് മേല്‍പ്പറഞ്ഞരീതിയിലുള്ള കുതിരകയറ്റം. ഇത്രയ്ക്കും ഇടുങ്ങിയ ചട്ടക്കൂടുകളില്‍ വളര്‍ത്തിക്കൊണ്ടുവരുന്ന വിദ്യാര്‍ത്ഥികള്‍ യാഥാസ്ഥിതികരീതികളുടെ പുറത്തേയ്ക്കു നോക്കാന്‍പോലും അശക്തരാവുന്ന അവസ്ഥയാണുണ്ടാവുന്നതു്. ഇത്തരത്തിലുള്ള ഒരു വിദ്യാഭ്യാസ സാമൂഹ്യസാഹചര്യങ്ങളിലൂടെ കടന്നുവരുന്ന വിദ്യാര്‍ത്ഥി, യാതൊരു സാമൂഹ്യബോധമോ ബാധ്യതയോ വികാരങ്ങളോ ഇല്ലാത്തയാളാവുന്നു. തന്റെ ലോകം തന്നിലേക്കു ചുരുക്കുകയും, സഹജീവികളെ മാനിക്കാനോ മനസ്സിലാക്കാനോ കഴിയാതിരിക്കുകയും ചെയ്യുന്ന പൗരനായി മാറുന്നു. സാര്‍വത്രിക വിദ്യാഭ്യാസത്തിലൂടെ അത്ഭുതങ്ങള്‍ സൃഷ്ടിച്ച കേരളത്തിലെ സമൂഹം ഇന്നു നേരിടുന്ന തകര്‍ച്ചയ്ക്കു് ഒരു പ്രധാന പങ്കു്, മേല്‍പ്പറഞ്ഞ സാമൂഹ്യസാഹചര്യങ്ങള്‍ക്കാണു്.

ഇനി, ഇതും സെബിന്റെ ലേഖനവും തമ്മിലെന്താണു് ബന്ധമെന്നു ചോദിച്ചാല്‍, ഒന്നുമില്ല. സമൂഹത്തിന്റെ മുന്‍വിധികള്‍ എങ്ങനെ വിദ്യാര്‍ത്ഥികളുടെ സ്വാതന്ത്ര്യത്തിലും സ്വകാര്യതയിലും കടന്നുകയറുന്നു എന്നാണു് ഞാന്‍ പറയാന്‍ ശ്രമിച്ചതു്.

\newpage
